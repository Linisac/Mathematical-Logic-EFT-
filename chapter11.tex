%Chapter XI----------------------------------------------------------------------------------------
{\LARGE \bfseries XI \\ \\ Free Models and Logic Programming}
\\
\\
\\
%Section XI.1--------------------------------------------------------------------------------------
{\large \S1. Herbrand's Theorem}
\begin{enumerate}[1.]
\item \textbf{Note to the Discussion at the Bottom of Page 190.} Actually $T_k^\Phi$ is the universe of \emph{exactly one} substructure of $\mathfrak{T}^\Phi$, namely $\mathfrak{T}_k^\Phi$, the substructure $[T_k^\Phi]^{\mathfrak{T}^\Phi}$ generated by $T_k^\Phi$ in $\mathfrak{T}^\Phi$ (cf. the discussion after III.5.4).
%
\item \textbf{Note to Lemma 1.2.} (a) and (b) follow from III.5.5.
%
\item \textbf{Note to the Proof of Lemma 1.3.} In general, $\Phi$ \emph{may fail} to be satisfied by $\mathfrak{I}_k$, where $\mathfrak{I}$ is a model of $\Phi_0$ constructed by the methods mentioned in V.2 and V.3; recall that we enlarged $S$ with (possibly infinitely many) constants into a ``bigger'' symbol set $S^\prime$. Thus $\mathfrak{I}$ contains, for instance, $\bar{c}$ as an element in the universe, in which $c \in S^\prime \setminus S$ is a new constant symbol. In the process of constructing $\mathfrak{I}$, $\varphi (x \mid c)$ may fail to be added into $\Phi_0$, where $\forall x \varphi \in \Phi$; and as a result, possibly,
\begin{center}
not $\mathfrak{I} \models \varphi (x \mid c)$.
\end{center}
That is why in this proof we applied arguments using the Compactness Theorem to restrict ourselves to the case in which $S$ is finite and directly dealt with $\Phi_0$ in which $\free{\Phi_0}$ is finite (so that we do not have to add constants into $S$).\\
\ \\
To see why it suffices to consider finite symbol set, notice that if $\Phi_0$ is satisfiable, then in particular $\Phi_0 \cap L^{S^\prime}$ is also satisfiable, where $S^\prime \subset S$ is an arbitrary finite symbol set. Through the argument of the proof, we have $\Phi \cap L^{S^\prime}$ is satisfiable. Since every finite subset of $\Phi$ is a subset of $\Phi \cap L^{S^\prime}$ for some suitable (finite) $S^\prime$ and hence is satisfiable, $\Phi$ itself is satisfiable as well, by the Compactness Theorem.\\
\ \\
Finally, we concluded
\[
\mathfrak{I}^\Theta_k \models \forall x_1 \ldots \forall x_m \varphi
\]
from
\begin{center}
for all $t_1, \ldots, t_m \in T^S_k$: $\mathfrak{I}^\Theta_k \sbst{\overline{t_1} \ldots \overline{t_m}}{x_1 \ldots x_m} \models \varphi$,
\end{center}
using the fact below:
\begin{quote}
\emph{Let $B$ be a nonempty set and $P(b)$ a statement about the elements $b \in B$ which depends only on the argument $b$. Also, let $A$ be a nonempty set and $f$ a mapping from $A$ to $B$. If
\begin{center}
for all $a \in A$, $P(f(a))$ holds,
\end{center}
then
\begin{center}
for all $b \in f(A) \subset B$, $P(b)$ holds.
\end{center}}
\end{quote}
%
\item \textbf{Note to 1.5 (b).} The logical equivalence betweeen (i) and (ii) follows from the argument below:
\begin{center}
\begin{tabular}{ll}
\   & $\exists x_1 \ldots \exists x_n \varphi$ \cr
iff & $\forall x_1 \, x_1 \equiv x_1 \vdash \exists x_1 \ldots \exists x_n \varphi$ \cr
iff & there are $j \geq 1$ and terms $t_{11}, \ldots t_{1n}, \ldots, t_{j1}, \ldots, t_{jn} \in T_k^S$ with\cr
\   & $\forall x_1 \, x_1 \equiv x_1 \vdash \varphi( \stackrel{n}{x} \mid \stackrel{n}{t_1}) \lor \ldots \lor \varphi(\stackrel{n}{x} \mid \stackrel{n}{t_j})$ \ \ \ (by 1.4) \cr
iff & there are $j \geq 1$ and terms $t_{11}, \ldots t_{1n}, \ldots, t_{j1}, \ldots, t_{jn} \in T_k^S$ with\cr
\   & $\vdash \varphi( \stackrel{n}{x} \mid \stackrel{n}{t_1}) \lor \ldots \lor \varphi(\stackrel{n}{x} \mid \stackrel{n}{t_j})$.
\end{tabular}
\end{center}
%
\item \textbf{Solution to Exercise 1.6.} (a) $\varphi$ is logically equivalent to $(Ry \lor \forall x \neg Rx)$. Thus, $\exists y \varphi$ is logically equivalent to $(\exists x Rx \lor \forall x \neg Rx)$. Therefore $\vdash \exists y \varphi$.\\
\ \\
(b) Suppose that for some $j \geq 1$ and some $t_1, \ldots, t_j \in T^S$,
\[
\vdash \varphi (y \mid t_1) \lor \ldots \lor \varphi (y \mid t_j).
\]
Then we have
\[
\models \varphi (y \mid t_1) \lor \ldots \lor \varphi (y \mid t_j)
\]
by the Adequacy Theorem.\\
\ \\
Choose an $S$-interpretation $\mathfrak{I} = (\mathfrak{A}, \beta)$ with
\begin{enumerate}[(i)]
\item $\{ a, b \}$ as the universe;
%%
\item $c^\mathfrak{A} = a$, and $\beta (v_n) = a$ for all $n \in \nat$;
%%
\item $R^\mathfrak{A} = \{ b \}$.
\end{enumerate}
Then $\mathfrak{I} \models \exists x Rx$, and for all $t \in T^S$, not $\mathfrak{I} \models Rt$. Hence
\begin{center}
not $\mathfrak{I} \models \varphi (y \mid t_1) \lor \ldots \lor \varphi (y \mid t_j)$,
\end{center}
a contradiction.
%
\item \textbf{Solution to Exercise 1.7.} To show that in general 1.5 (b)(ii) cannot be strengthened by claiming $j = 1$, let $S = \{ c, d \}$ and $\varphi = (\neg c \equiv d \rightarrow \neg v_1 \equiv v_0)$. As $\exists v_1 \varphi$ is logically equivalent to $(\neg c \equiv d \rightarrow \exists v_1 \neg v_1 \equiv v_0)$, we have $\vdash \exists v_1 \varphi.$\\
\ \\
However, there is no $t \in T^S$ such that $\vdash \varphi (v_1 \mid t)$, for we can always choose an $S$-interpretation $\mathfrak{I}$ such that not $\mathfrak{I} \models \varphi (v_1 \mid t)$:
\begin{enumerate}[(1)]
\item If $t = c$, let $\mathfrak{I}$ contain in the universe two distinct elements $a$, $b$; $\mathfrak{I}(c) = a$, $\mathfrak{I}(d) = b$, and $\mathfrak{I}(v_n) = a$ for all $n \in \nat$.
%%
\item If $t = d$, let $\mathfrak{I}(v_n) = b$ for $n \in \nat$ in $\mathfrak{I}$ above.
%%
\item If $t = v_n$ for some $n \in \nat$, let $\mathfrak{I}$ be the same as in (1).
\end{enumerate}
\ \\
It follows that we also cannot strenthen 1.4 by claiming $j = 1$ in (b) or (c), otherwise we could do this to 1.5 (b)(ii) as in the argument given by \textbf{Note to 1.5 (b)}.
\end{enumerate}
%End of Section XI.1-------------------------------------------------------------------------------
\
\\
\\
%Section XI.2-------------------------------------------------------------------------------------
{\large \S2. Free Models and Universal Horn Formulas}
\begin{enumerate}[1.]
\item \textbf{Note to Theorem 2.1.} For (ii), let $a_i = \overline{t_i}$ with suitable $t_i \in T^S$ for $1 \leq i \leq n$. Then
\[
\begin{array}{ll}
\ & \pi (f^{\mathfrak{T}^\Phi}(a_1, \ldots, a_n)) \cr
= & \pi (f^{\mathfrak{T}^\Phi}(\overline{t_1}, \ldots, \overline{t_n})) \cr
= & \pi (\overline{ft_1 \ldots t_n}) \cr
= & \mathfrak{I}(ft_1 \ldots t_n) \cr
= & f^\mathfrak{A} (\mathfrak{I}(t_1), \ldots, \mathfrak{I}(t_n)) \cr
= & f^\mathfrak{A} (\pi (\overline{t_1}), \ldots, \pi (\overline{t_n})) \cr
= & f^\mathfrak{A} (\pi (a_1), \ldots, \pi (a_n)).
\end{array}
\]
For (iii), we have
\[
\pi (c^{\mathfrak{T}^\Phi}) = \pi (\bar{c}) = \mathfrak{I}(c) = c^\mathfrak{A}.
\]
%
\item \textbf{Note to Lemma 2.3.} (a) This can be shown by induction on universal Horn formulas, i.e. formulas generated by the calculus given in 2.2:
\begin{enumerate}[(i)]
\item Formulas obtained by applying 2.2 (1) are formulas of the form (H1) if containing no negated atoms, and of the form (H2) otherwise. They are by themselves conjunctions (with single conjuncts).
%%
\item Fromulas obtained by applying 2.2 (2) are formulas of the form (H3). They are by themselves conjunctions (with single conjuncts).
%%
\item Let $\varphi$ and $\psi$ be universal Horn formulas such that the induction hypothesis holds, i.e. they are logically equivalent to $\varphi^\prime$ and $\psi^\prime$, respectively, each of which is a conjunction of formulas of the form (H1) - (H3).\\
\ \\
We have the formula $(\varphi \land \psi)$, which is obtained by applying 2.2 (3) to $\varphi$ and $\psi$, is logically equivalent to $(\varphi^\prime \land \psi^\prime)$, a conjuction of formulas of the form (H1) - (H3).
%%
\item Let $\varphi$ be a universal Horn formula such that the induction hypothesis holds, i.e. it is logically equivalent to $(\varphi_0^\prime \land \ldots \land \varphi_n^\prime)$, in which $\varphi_0^\prime, \ldots, \varphi_n^\prime$ are formulas of the form (H1) - (H3).\\
\ \\
We have $\forall x \varphi$, which is obtained by applying 2.2 (4) to $\varphi$, is logically equivalent to $\forall x (\varphi_0^\prime \land \ldots \land \varphi_n^\prime)$, which in turn is logically equivalent to $(\forall x \varphi_0^\prime \land \ldots \land \forall x \varphi_n^\prime)$, a conjunction of formulas of the form (H1) - (H3).
\end{enumerate}
In each case, if the formula to be dealt with is in $L_k^S$ then the resulting logically equivalent formula is also in $L_k^S$.\\
\ \\
(b) follows from (a) and the fact that
\begin{quote}
\emph{the formula $(\forall x \varphi \land \forall y \psi)$ is logically equivalent to $\forall z \forall u \left( \varphi\sbst{z}{x} \land \psi\sbst{u}{y} \right)$, where $z$ and $u$ do not occur in $\forall y \psi$ and $\forall x \varphi$, respectively.}
\end{quote}

(c) We show
\begin{quote}
\emph{for all universal Horn formulas $\varphi$, if $x_1, \ldots, x_n$ are pairwise distinct then, for $t_1, \ldots, t_n \in T^S$, $\varphi (\stackrel{n}{x} \mid \stackrel{n}{t})$ is also a universal Horn formula}
\end{quote}
by induction on formulas generated by the calculus given in 2.2:
\begin{enumerate}[(i)]
\item For formulas obtained by applying 2.2 (1) or (2), the statement is trivially true.
%%
\item Let $\varphi$ and $\psi$ be universal Horn formulas such that the induction hypothesis holds: If $x_1, \ldots, x_n$ are pairwise distinct, then for $t_1, \ldots, t_n \in T^S$, $\varphi (\stackrel{n}{x} \mid \stackrel{n}{t})$ and $\psi (\stackrel{n}{x} \mid \stackrel{n}{t})$ are universal Horn formulas. It immediately follows that $(\varphi \land \psi) (\stackrel{n}{x} \mid \stackrel{n}{t}) = (\varphi (\stackrel{n}{x} \mid \stackrel{n}{t}) \land \psi (\stackrel{n}{x} \mid \stackrel{n}{t}))$ is also a universal Horn formula, in which $(\varphi \land \psi)$ is obtained by applying 2.2 (3) to $\varphi$ and $\psi$. The statement is also true in this case.
%%
\item Let $\varphi$ be a universal Horn formula such that the induction hypothesis holds: If $x_1, \ldots, x_n$ are pairwise distinct, then for $t_1, \ldots, t_n \in T^S$, $\varphi (\stackrel{n}{x} \mid \stackrel{n}{t})$ is a universal Horn formula.\\
\ \\
So, if $x_1, \ldots, x_n$ are pairwise distinct, then for $t_1, \ldots, t_n \in T^S$,
\[
(\forall x \varphi ) (\stackrel{n}{x} \mid \stackrel{n}{t}) = \begin{cases}
\forall x (\varphi (\stackrel{n}{x} \mid \stackrel{n}{t})) & \mbox{if \(x\) does not occur in \(\seq[1]{x}{n}\)}; \cr
\forall x (\varphi (\stackrel{n}{x} \mid \stackrel{n}{t^\prime})) & \mbox{otherwise}
\end{cases}
\]
is also a universal Horn formula, where
\[
t_i^\prime = \begin{cases}
t_i & \mbox{if \(x_i \neq x\)};\cr
x & \mbox{otherwise}
\end{cases}
\]
for $1 \leq i \leq n$ and $\forall x \varphi$ is obtained by applying 2.2 (4) to $\varphi$. The statement is also true in this case.
\end{enumerate}
%
\item \textbf{Note to Corollary 2.5.} If moreover $\Phi \subset L_k^S$, then $\mathfrak{I}_k^\Phi$ is a model of $\Phi$ (cf. 1.2 (c)), but not necessarily a free model.
%
\item \textbf{Note to Exercise 2.8.} Since neither $\Phi \vdash Pc$ nor $\Phi \vdash Pd$, by 1.1 (b) we have neither $\mathfrak{I}^\Phi \models Pc$ nor $\mathfrak{I}^\Phi \models Pd$, and therefore
\begin{center}
not $\mathfrak{I}^\Phi \models (Pc \lor Pd)$,
\end{center}
that is, $\mathfrak{I}^\Phi$ is not a model of $\Phi$.\\
\ \\
As $(Pc \lor Pd)$ is satisfiable, it follows from 2.5 that it is not logically equivalent to a universal Horn formula, and of course not logically equivalent to a universal Horn sentence.\\
\ \\
We further show that $(Pc \lor Pd)$ is not even logically equivalent to a Horn sentence. To see this, take two $S$-structures $\mathfrak{A}_1$ and $\mathfrak{A}_2$, with $A_1 = \{ c_1, d_1 \}$ and $A_2 = \{ c_2, d_2 \}$, respectively, such that
\begin{center}
$c^{\mathfrak{A}_1} = c_1$, $d^{\mathfrak{A}_1} = d_1$, $P^{\mathfrak{A}_1} = \{ c_1 \}$
\end{center}
and
\begin{center}
$c^{\mathfrak{A}_2} = c_2$, $d^{\mathfrak{A}_2} = d_2$, $P^{\mathfrak{A}_2} = \{ d_2 \}$.
\end{center}
We have $\mathfrak{A}_1 \models (Pc \lor Pd)$ and $\mathfrak{A}_2 \models (Pc \lor Pd)$, but
\begin{center}
not $\mathfrak{A}_1 \times \mathfrak{A}_2 \models (Pc \lor Pd)$.
\end{center}
Thus, using III.4.16, we claim that $(Pc \lor Pd)$ is not logically equivalent to a Horn sentence.\\
\ \\
If in the above we take $P^{\mathfrak{A}_1} = \emptyset$ instead, and take two assignments $\beta_1$ and $\beta_2$ in $\mathfrak{A}_1$ and in $\mathfrak{A}_2$, respectively, such that:
\[
\beta_1 (x) = c_1, \beta_1 (y) = d_1,
\]
and
\[
\beta_2 (x) = \beta_2 (y) = d_2,
\]
then it follows that $(\mathfrak{A}_1, \beta_1) \models (\neg Px \lor Py \lor x \equiv y)$ and $(\mathfrak{A}_2, \beta_2) \models (\neg Px \lor Py \lor x \equiv y)$, but
\begin{center}
not $(\mathfrak{A}_1, \beta_1) \times (\mathfrak{A}_2, \beta_2) \models (\neg Px \lor Py \lor x \equiv y)$.
\end{center}
Hence, using III.4.16, we claim that $(\neg Px \lor Py \lor x \equiv y)$ is not logically equivalent to a Horn formula.
%
\item \textbf{Solution to Exercise 2.9.} Let $a_0, a_1, a_2, \ldots$ be an enumeration of the elements of $G$ in which all $a_i$'s are pairwise distinct; if $G$ is finite, say, contains $(n + 1)$ elements, then let $a_i \neq a_j$ for $0 \leq i < j \leq n$, and let $a_k = a_n$ for $k > n$.\\
\ \\
We take the assignment $\beta$ in $\mathfrak{G}$ with $\beta (v_i) \colonequals a_i$. It is clear that $\{ (\mathfrak{G}, \beta)(t) \mid t \in T^{S_\grp} \} = G$.\\
\ \\
Since $(\mathfrak{T}^{\Phi_\grp}, \beta^{\Phi_\grp})$ and $(\mathfrak{G}, \beta)$ are both models of $\Phi_\grp$ (by the Coincidence Lemma), by 2.1 the map $\pi: T^{\Phi_\grp} \to G$ in which
\begin{center}
$\pi (\bar{t}) \colonequals (\mathfrak{G}, \beta)(t)$ \ \ for $t \in T^{S_\grp}$
\end{center}
is a homomorphism from $\mathfrak{T}^{\Phi_\grp}$ to $\mathfrak{G}$, and even one from $\mathfrak{T}^{\Phi_\grp}$ \emph{onto} $\mathfrak{G}$ by the earlier discussion.\\
\ \\
Next we show every group $\mathfrak{G}$ generated by at most $k$ elements is a homomorphic image of $\mathfrak{T}_k^{\Phi_\grp}$. Let $A = \{ a_0, \ldots, a_{k - 1} \}$ be a generating set of $\mathfrak{G}$ ($A = \emptyset$ in case $k = 0$), namely every element in $G$ is a finite combination of members $a \in A$ and their inverses $a^{-1}$ in the binary operation $\circ$.\\
\ \\
Observe that for $t \in T_k^{S_\grp}$ and for $n \geq k$,
\begin{center}
neither \ $\Phi_\grp \vdash v_n \equiv t$ \ \ nor \ $\Phi_\grp \vdash \neg v_n \equiv t$.
\end{center}
For if $\Phi_\grp \vdash v_p \equiv t$ for some $t \in T_k^{S_\grp}$ and some $p \geq k$, we may pick the model $\mathfrak{I}_0 = (\mathfrak{Z}_0, \beta_0)$ of $\Phi_\grp$, in which $\mathfrak{Z}_0$ is the additive group over $\zah$ and
\[
\beta_0 (v_n) \colonequals \begin{cases}
1 & \mbox{if \(n = p\)} \cr
0 & \mbox{otherwise},
\end{cases}
\]
then we have $\mathfrak{I}_0 (v_p) = 1$, $\mathfrak{I}_0 (t) = 0$ and, as a consequence, $1 = 0$, which is impossible. The case that not $\Phi_\grp \vdash \neg v_n \equiv t$ is similar: we take $\beta_0 (v_n) = 0$ for $n \in \nat$ in the above model $\mathfrak{I}_0$.\\
\ \\
So, let us set
\[
\Phi \colonequals \Phi_\grp \cup \{ v_n \equiv e \mid n \geq k \},
\]
then $\Phi$ is satisfiable and, by 2.5, $\mathfrak{I}^\Phi$ is a free model of $\Phi$. Pick the assignment
\[
\beta (v_n) \colonequals \begin{cases}
a_n & \mbox{if \(n < k\)} \cr
e^\mathfrak{G} & \mbox{otherwise}
\end{cases}
\]
in $\mathfrak{G}$ ($\beta (v_n) = e^\mathfrak{G}$ for $n \in \nat$ in case $k = 0$), then $(\mathfrak{G}, \beta) \models \Phi$ and $\{ (\mathfrak{G}, \beta)(t) \mid t \in T^{S_\grp} \} = G$. As a result, the map $\pi : T^\Phi \to G$
\begin{center}
$\pi (\bar{t}^\Phi ) = (\mathfrak{G}, \beta )(t)$ \ for $t \in T^{S_\grp}$
\end{center}
is a homomorphism from $\mathfrak{T}^\Phi$ \emph{onto} $\mathfrak{G}$, where $\bar{t}^\Phi$ denotes the equivalence class $\{ t^\prime \in T^{S_\grp} \mid \Phi \vdash t \equiv t^\prime \}$ for $t \in T^{S_\grp}$.\\
\ \\
It remains to be shown that there is an isomorphism $\sigma$ from $\mathfrak{T}_k^{\Phi_\grp}$ to $\mathfrak{T}^\Phi$, for the composition $\pi \cdot \sigma$ of $\pi$ and $\sigma$ is a homomorphism from $\mathfrak{T}_k^{\Phi_\grp}$ to $\mathfrak{G}$.\\
\ \\
Let us denote by $\bar{t}^{\Phi_\grp}$ the equivalence class $\{ t^\prime \in T^{S_\grp} \mid \Phi_\grp \vdash t \equiv t^\prime \}$ for $t \in T^{S_\grp}$, and take the map $\sigma : T_k^{\Phi_\grp} \to T^\Phi$
\begin{center}
$\sigma (\bar{t}^{\Phi_\grp}) = \bar{t}^\Phi$ \ for $t \in T_k^{S_\grp}$.
\end{center}
We show $\sigma$ is an isomorphism from $\mathfrak{T}_k^{\Phi_\grp}$ to $\mathfrak{T}^\Phi$ or, more precisely:
\begin{enumerate}[(1)]
\item $\sigma$ is surjective;
%%
\item $\sigma$ is injective;
%%
\item $\sigma (e^{\mathfrak{T}_k^{\Phi_\grp}}) = e^{\mathfrak{T}^\Phi}$;
%%
\item For $t \in T_k^{S_\grp}$, $\sigma ((\bar{t}^{\Phi_\grp})^{-1}) = (\sigma (\bar{t}^{\Phi_\grp}))^{-1}$; (The first occurrence of $^{-1}$ stands for the inverse function in $\mathfrak{T}_k^{\Phi_\grp}$, while the second for that in $\mathfrak{T}^\Phi$.)
%%
\item For $t_1, t_2 \in T_k^{S_\grp}$, $\sigma (\overline{t_1}^{\Phi_\grp} \circ \overline{t_2}^{\Phi_\grp}) = \sigma (\overline{t_1}^{\Phi_\grp}) \circ \sigma (\overline{t_2}^{\Phi_\grp})$. (The first occurrence of $\circ$ stands for the product function in $\mathfrak{T}_k^{\Phi_\grp}$, while the second for that in $\mathfrak{T}^\Phi$.)
\end{enumerate}
Every element in $\mathfrak{T}^\Phi$ must be $\bar{t}^\Phi$ for some $t \in T_k^{S_\grp}$, and $\bar{t}^\Phi = \sigma (\bar{t}^{\Phi_\grp})$, so (1) follows.\\
\ \\
For (2), consider arbitrary $t_1, t_2 \in T_k^{S_\grp}$. If $\Phi \vdash t_1 \equiv t_2$ then $\Phi_\grp \vdash t_1 \equiv t_2$: Assume $\Phi \vdash t_1 \equiv t_2$. Since $\mathfrak{T}_k^{\Phi_\grp} \models \Phi_\grp$, the interpretation $(\mathfrak{T}_k^{\Phi_\grp}, \beta^\prime)$ is a model of $\Phi$, where
\[
\beta^\prime (v_n) \colonequals \begin{cases}
\overline{v_n}^{\Phi_\grp} & \mbox{if \(n < k\)} \cr
\overline{e}^{\Phi_\grp} & \mbox{otherwise}.
\end{cases}
\]
Hence $(\mathfrak{T}_k^{\Phi_\grp}, \beta^\prime ) \models t_1 \equiv t_2$ and, by the Coincidence Lemma, $\mathfrak{I}_k^{\Phi_\grp} = (\mathfrak{T}_k^{\Phi_\grp}, \beta_k^{\Phi_\grp}) \models t_1 \equiv t_2$. Further, by 1.2 (b) we have $\mathfrak{I}^{\Phi_\grp} \models t_1 \equiv t_2$. Finally, $\Phi_\grp \vdash t_1 \equiv t_2$ by 1.1 (b).\\
\ \\
Since $\mathfrak{I}^\Phi \models \Phi$, if $\sigma (\overline{t_1}^{\Phi_\grp}) = \sigma (\overline{t_2}^{\Phi_\grp})$, namely $\overline{t_1}^\Phi = \overline{t_2}^\Phi$, then $\Phi \vdash t_1 \equiv t_2$ by 1.1(b), and $\Phi_\grp \vdash t_1 \equiv t_2$ by the above discussion, and so $\overline{t_1}^{\Phi_\grp} = \overline{t_2}^{\Phi_\grp}$ again by 1.1(b). Thus (2) follows.\\
\ \\
(3) follows immediately from the definition of $\sigma$.\\
\ \\
As for (4), let $t \in T_k^{S_\grp}$. Then
\[
\begin{array}{ll}
\ & \sigma ((\bar{t}^{\Phi_\grp})^{-1}) \cr
= & \sigma (\overline{t^{-1}}^{\Phi_\grp}) \cr
= & \overline{t^{-1}}^\Phi \cr
= & (\bar{t}^\Phi )^{-1} \cr
= & (\sigma (\bar{t}^{\Phi_\grp}))^{-1}.
\end{array}
\]
To show (5), consider arbitrary $t_1, t_2 \in T_k^{S_\grp}$. Then
\[
\begin{array}{ll}
\ & \sigma (\overline{t_1}^{\Phi_\grp} \circ \overline{t_2}^{\Phi_\grp}) \cr
= & \sigma (\overline{t_1 \circ t_2}^{\Phi_\grp}) \cr
= & \overline{t_1 \circ t_2}^\Phi \cr
= & \overline{t_1}^\Phi \circ \overline{t_2}^\Phi \cr
= & \sigma (\overline{t_1}^{\Phi_\grp}) \circ \sigma (\overline{t_2}^{\Phi_\grp}).
\end{array}
\]
%
\item \textbf{Solution to Exercise 2.10.} (a) $\Phi_\grp \cup \Phi$ is satisfied by the trivial group, namely the group with a singleton universe.\\
\ \\
(b) It follows from (a) that $\Phi_\grp \cup \Phi$ is a consistent set of universal Horn sentences. By 2.5, $\mathfrak{I}^{\Phi_\grp \cup \Phi}$ and hence (by the Coincidence Lemma) $\mathfrak{T}^{\Phi_\grp \cup \Phi}$ are models of $\Phi_\grp \cup \Phi$.\\
\ \\
(c) Denote by $U$ the set $\{ \bar{t} \mid \mbox{\begin{math}t \in T^S\end{math} and \begin{math}\Phi_\grp \cup \Phi \vdash t \equiv e\end{math}}\}$. Then clearly $U \subset T^{\Phi_\grp}$. For $U$ to be the universe of a subgroup $\mathfrak{U}$ of $\mathfrak{T}^{\Phi_\grp}$, it suffices to verify:
\begin{enumerate}[(i)]
\item For all $t_1, t_2 \in T^{S_\grp}$, if $\overline{t_1}, \overline{t_2} \in U$ then $\overline{t_1} \circ \overline{t_2} \in U$. (Here $\circ$ stands for the product operation over the group $\mathfrak{T}^{\Phi_\grp}$.)
%%
\item For all $t \in T^{S_\grp}$, if $\bar{t} \in U$ then $\bar{t}^{-1} \in U$. (Here $^{-1}$ stands for the inverse operation over the group $\mathfrak{T}^{\Phi_\grp}$.)
\end{enumerate}
To show (i) holds, let $t_1, t_2 \in T^{S_\grp}$ such that $\Phi_\grp \cup \Phi \vdash t_1 \equiv e$ and $\Phi_\grp \cup \Phi \vdash t_2 \equiv e$, respectively, i.e. $\overline{t_1} \in U$ and $\overline{t_2} \in U$. Then $\Phi_\grp \cup \Phi \vdash t_1 \circ t_2 \equiv e$. So $\overline{t_1} \circ \overline{t_2} = \overline{t_1 \circ t_2} \in U$.\\
\ \\
To show (ii) is true, let $t \in T^{S_\grp}$ such that $\Phi_\grp \cup \Phi \vdash t \equiv e$, i.e. $\bar{t} \in U$. As $\Phi_\grp \vdash t^{-1} \equiv e \circ t^{-1}$ (cf. \textbf{Note to Theorem 1.1}) and $\Phi_\grp \cup \Phi \vdash e \equiv t$, we have $\Phi_\grp \cup \Phi \vdash t^{-1} \equiv e$ and so $\bar{t}^{-1} = \overline{t^{-1}} \in U$.\\
\ \\
To see $\mathfrak{U}$ is a normal subgroup, we pick arbitrary $t, u \in T^{S_\grp}$, where $\bar{u} \in U$, and show $\bar{t} \circ \bar{u} \circ \bar{t}^{-1} \in U$: Since $\bar{u} \in U$, $\Phi_\grp \cup \Phi \vdash u \equiv e$. Hence $\Phi_\grp \cup \Phi \vdash (t \circ u) \circ t^{-1} \equiv e$, i.e. $\bar{t} \circ \bar{u} \circ \bar{t}^{-1} = \overline{(t \circ u) \circ t^{-1}} \in U$.\\
\ \\
The rest is for showing $\mathfrak{T}^{\Phi_\grp \cup \Phi}$ is isomorphic to $\mathfrak{T}^{\Phi_\grp} \slash \mathfrak{U}$. In the following, we write $\bar{t}^{\Phi_\grp}$ or $\bar{t}^{\Phi_\grp \cup \Phi}$ for $t \in T^{S_\grp}$ to explicitly indicate whether the equivalence class of $t$ is taken with respect to $\Phi_\grp$ or to $\Phi_\grp \cup \Phi$.\\
\ \\
$\mathfrak{T}^{\Phi_\grp} \slash \mathfrak{U}$ is an $S_\grp$-structure, with:
\begin{itemize}
\item the universe $T^{\Phi_\grp} \slash U \colonequals \{ U\bar{t}^{\Phi_\grp} \mid t \in T^{S_\grp} \}$
%%
\item the identity $e^{\mathfrak{T}^{\Phi_\grp} \slash \mathfrak{U}} \colonequals U\bar{e}^{\Phi_\grp} = U$
%%
\item the inverse operation $^{-1}$: $\left(U\bar{t}^{\Phi_\grp}\right)^{-1} \colonequals U\overline{t^{-1}}^{\Phi_\grp}$ for $t \in T^{S_\grp}$
%%
\item the product operation $\circ$: $\left(U\overline{t_1}^{\Phi_\grp}\right) \circ \left(U\overline{t_2}^{\Phi_\grp}\right) \colonequals U\overline{t_1 \circ t_2}^{\Phi_\grp}$ for $t_1, t_2 \in T^{S_\grp}$.
\end{itemize}
\ \\
First, for $t \in T^{S_\grp}$,
\[
\bar{t}^{\Phi_\grp \cup \Phi} = \bigcup \left( U \bar{t}^{\Phi_\grp} \right).
\]
We show this as follows: Let $t^\prime \in T^{S_\grp}$. If $t^\prime \in \bar{t}^{\Phi_\grp \cup \Phi}$, then $\Phi_\grp \cup \Phi \vdash t^\prime \equiv t$, or $\Phi_\grp \cup \Phi \vdash t^\prime \circ t^{-1} \equiv e$, therefore $\overline{t^\prime \circ t^{-1}}^{\Phi_\grp} \in U$. Since $\Phi_\grp \vdash t^\prime \equiv (t^\prime \circ t^{-1}) \circ t$, we have $t^\prime \in \overline{t^\prime \circ t^{-1}}^{\Phi_\grp} \circ \bar{t}^{\Phi_\grp}$. Because $\overline{t^\prime \circ t^{-1}}^{\Phi_\grp} \circ \bar{t}^{\Phi_\grp} \in U\bar{t}^{\Phi_\grp}$, $t^\prime \in \bigcup \left( U\bar{t}^{\Phi_\grp} \right)$. So $\bar{t}^{\Phi_\grp \cup \Phi} \subset \bigcup \left( U \bar{t}^{\Phi_\grp} \right)$. Conversely, if $t^\prime \in \bigcup \left( U\bar{t}^{\Phi_\grp} \right)$, i.e. $t^\prime \in \overline{u \circ t}^{\Phi_\grp}$ with some $u \in T^{S_\grp}$ such that $\Phi_\grp \cup \Phi \vdash u \equiv e$, then $\Phi_\grp \vdash t^\prime \equiv u \circ t$ and hence $\Phi_\grp \cup \Phi \vdash t^\prime \equiv t$, namely $t^\prime \in \bar{t}^{\Phi_\grp \cup \Phi}$. So $\bigcup \left( U \bar{t}^{\Phi_\grp} \right) \subset \bar{t}^{\Phi_\grp \cup \Phi}$.\\
\ \\
Second, for $t_1, t_2 \in T^{S_\grp}$,
\begin{center}
$\overline{t_1}^{\Phi_\grp \cup \Phi} = \overline{t_2}^{\Phi_\grp \cup \Phi}$ \ \ \ iff \ \ \ $U\overline{t_1}^{\Phi_\grp} = U\overline{t_2}^{\Phi_\grp}$.
\end{center}
This can be argued: Let $\overline{t_1}^{\Phi_\grp \cup \Phi} = \overline{t_2}^{\Phi_\grp \cup \Phi}$, then $\Phi_\grp \cup \Phi \vdash t_1 \equiv t_2$ and $\Phi_\grp \cup \Phi \vdash t_1 \circ t_2^{-1} \equiv e$. Since every element in $U\overline{t_1}^{\Phi_\grp}$ takes the form $\bar{u}^{\Phi_\grp} \circ \overline{t_1}^{\Phi_\grp}$ for some $u \in T^{S_\grp}$ such that $\Phi_\grp \cup \Phi \vdash u \equiv e$, and
\[
\begin{array}{ll}
\ & \bar{u}^{\Phi_\grp} \circ \overline{t_1}^{\Phi_\grp} \cr
= & \overline{u \circ t_1}^{\Phi_\grp} \cr
= & \overline{u \circ (t_1 \circ e)}^{\Phi_\grp} \cr
= & \overline{u \circ (t_1 \circ (t_2^{-1} \circ t_2))}^{\Phi_\grp} \cr
= & \overline{(u \circ (t_1 \circ t_2^{-1})) \circ t_2}^{\Phi_\grp} \cr
= & \overline{u \circ (t_1 \circ t_2^{-1})}^{\Phi_\grp} \circ \overline{t_2}^{\Phi_\grp} \cr
\in & U\overline{t_2}^{\Phi_\grp}, \mbox{\ \ ($\Phi_\grp \cup \Phi \vdash u \circ (t_1 \circ t_2^{-1}) \equiv e$)}
\end{array}
\]
we have $U\overline{t_1}^{\Phi_\grp} \subset U\overline{t_2}^{\Phi_\grp}$; symmetrically we get $U\overline{t_2}^{\Phi_\grp} \subset U\overline{t_1}^{\Phi_\grp}$. So $U\overline{t_1}^{\Phi_\grp} = U\overline{t_2}^{\Phi_\grp}$. Conversely, if $U\overline{t_1}^{\Phi_\grp} = U\overline{t_2}^{\Phi_\grp}$, then
\[
\overline{t_1}^{\Phi_\grp \cup \Phi} = \bigcup \left( U\overline{t_1}^{\Phi_\grp} \right) = \bigcup \left( U\overline{t_2}^{\Phi_\grp} \right) = \overline{t_2}^{\Phi_\grp \cup \Phi}.
\]
Third, let $\beta$ be an assignment in $\mathfrak{T}^{\Phi_\grp} \slash \mathfrak{U}$ such that $\beta (v_n) \colonequals U \overline{v_n}$ for $n \in \nat$, and denote $\mathfrak{I} = (\mathfrak{T}^{\Phi_\grp} \slash \mathfrak{U}, \beta)$. Then $\mathfrak{I} (t) = U\bar{t}^{\Phi_\grp}$ for all $t \in T^{S_\grp}$, which can easily be proved by induction on $t$. If $\forall x_1 \ldots \forall x_r t \equiv t^\prime$ is an equation derivable from $\Phi_\grp \cup \Phi$, then we have, successively,
\begin{center}
\begin{tabular}{l}
for all $t_1, \ldots, t_r \in T^{S_\grp}$, $\Phi_\grp \cup \Phi \vdash [t \equiv t^\prime] \sbst{t_1 \ldots t_r}{x_1 \ldots x_r}$; \cr
for all $t_1, \ldots, t_r \in T^{S_\grp}$, $\Phi_\grp \cup \Phi \vdash t\sbst{t_1 \ldots t_r}{x_1 \ldots x_r} \equiv t^\prime\sbst{t_1 \ldots t_r}{x_1 \ldots x_r}$; \cr
for all $t_1, \ldots, t_r \in T^{S_\grp}$, $\overline{t\sbst{t_1 \ldots t_r}{x_1 \ldots x_r}}^{\Phi_\grp \cup \Phi} = \overline{t^\prime\sbst{t_1 \ldots t_r}{x_1 \ldots x_r}}^{\Phi_\grp \cup \Phi}$; \cr
for all $t_1, \ldots, t_r \in T^{S_\grp}$, $U\overline{t\sbst{t_1 \ldots t_r}{x_1 \ldots x_r}}^{\Phi_\grp} = U\overline{t^\prime\sbst{t_1 \ldots t_r}{x_1 \ldots x_r}}^{\Phi_\grp}$; \cr
for all $t_1, \ldots, t_r \in T^{S_\grp}$, $\mathfrak{I}\left( t\sbst{t_1 \ldots t_r}{x_1 \ldots x_r} \right) = \mathfrak{I}\left( t^\prime\sbst{t_1 \ldots t_r}{x_1 \ldots x_r} \right)$; \cr
for all $t_1, \ldots, t_r \in T^{S_\grp}$, $\mathfrak{I}\sbst{\mathfrak{I}(t_1) \ldots \mathfrak{I}(t_r)}{x_1 \ldots x_r}(t) = \mathfrak{I}\sbst{\mathfrak{I}(t_1) \ldots \mathfrak{I}(t_r)}{x_1 \ldots x_r}(t^\prime)$; \cr
for all $t_1, \ldots, t_r \in T^{S_\grp}$, $\mathfrak{I}\sbst{\mathfrak{I}(t_1) \ldots \mathfrak{I}(t_r)}{x_1 \ldots x_r} \models t \equiv t^\prime$; \cr
for all $t_1, \ldots, t_r \in T^{S_\grp}$, $\mathfrak{I}\sbst{U\overline{t_1}^{\Phi_\grp} \ldots U\overline{t_r}^{\Phi_\grp}}{x_1 \ldots x_r} \models t \equiv t^\prime$,
\end{tabular}
\end{center}
which entails $\mathfrak{I} \models \forall x_1 \ldots \forall x_r t \equiv t^\prime$. Using the Coincidence Lemma, we get $\mathfrak{T}^{\Phi_\grp} \slash \mathfrak{U} \models \forall x_1 \ldots \forall x_r t \equiv t^\prime$. In particular, $\mathfrak{T}^{\Phi_\grp \cup \Phi} \models \Phi_\grp \cup \Phi$.\\
\ \\
Finally, since $\mathfrak{I}^{\Phi_\grp \cup \Phi}$ is a free model of $\Phi_\grp \cup \Phi$ (by 2.5), the map $\pi$ from $T^{\Phi_\grp \cup \Phi}$ to $T^{\Phi_\grp} \slash U$ with
\begin{center}
$\pi \left( \bar{t}^{\Phi_\grp \cup \Phi} \right) \colonequals U\bar{t}^{\Phi_\grp}$ \ \ for $t \in T^{S_\grp}$
\end{center}
is a homomorphism. Actually, $\pi$ is an isomorphism, for the injectiveness has been proven earlier and the surjectiveness is trivial. So, $\mathfrak{T}^{\Phi_\grp \cup \Phi} \cong \mathfrak{T}^{\Phi_\grp} \slash \mathfrak{U}$.\\
\ \\
\textit{Remark}. $\mathfrak{T}^{\Phi_\grp} \slash \mathfrak{U}$ is the so-called \emph{quotient group} of $\mathfrak{T}^{\Phi_\grp}$ by $\mathfrak{U}$.
\end{enumerate}
%End of Section XI.2------------------------------------------------------------------------------
\
\\
\\
%Section XI.3--------------------------------------------------------------------------------------
{\large \S3. Herbrand Structures}
\begin{enumerate}[1.]
\item \textbf{A Typo in 3.4 (ii).} $T^S$ should be changed to $T_0^S$.
%
\item \textbf{Note to the Proof of 3.7.} The substructure $\mathfrak{B}$ of $\mathfrak{A}^\prime$ is the substructure $\left[ T_0^S \right]^{\mathfrak{A}^\prime}$ generated by $T_0^S$ in $\mathfrak{A}^\prime$ (cf. the discussion after III.5.4). Note that $\mathfrak{B}$ is not necessarily identical to $\mathfrak{T}_0^\Phi$, since for any $n$-ary $R \in S$ and $t_1, \ldots, t_n \in T_0^S$,
\begin{center}
$\mathfrak{B} \models Rt_1 \ldots t_n$ \ \ \ iff \ \ \ $\mathfrak{I} \models Rt_1 \ldots t_n$,
\end{center}
while
\begin{center}
$\mathfrak{T}_0^\Phi \models Rt_1 \ldots t_n$ \ \ \ iff \ \ \ $\Phi \vdash Rt_1 \ldots t_n$.
\end{center}
\ \\
That $\mathfrak{B}$ is a model of $\Phi$ follows from $\mathfrak{A}^\prime \models \Phi$ and III.5.8.
%
\end{enumerate}
%End of Section XI.3-------------------------------------------------------------------------------
\
\\
\\
%Section XI.4--------------------------------------------------------------------------------------
{\large \S4. Propositional Logic}
\begin{enumerate}[1.]
\item \textbf{A Proof of 4.2 Coincidence Lemma of Propositional Logic.} We use induction on propositional formulas.\\
\ \\
$\alpha = p_i$: $\pvar{\alpha} = \{ p_i \}$. If $b$ and $b^\prime$ are two assignments with $b(p_i) = b^\prime(p_i)$, then $\alpha[b] = b(p_i) = b^\prime(p_i) = \alpha[b^\prime]$.\\
\ \\
$\alpha = \neg \beta$: If $b$ and $b^\prime$ are two assignments with $b(p) = b^\prime(p)$ for $p \in \pvar{\alpha} = \pvar{\beta}$, then by induction hypothesis $\beta[b] = \beta[b^\prime]$. So $\alpha[b] = \negfunc (\beta[b]) = \negfunc (\beta[b^\prime]) = \alpha[b^\prime]$.\\
\ \\
$\alpha = (\beta \lor \gamma)$: If $b$ and $b^\prime$ are two assignments with $b(p) = b^\prime(p)$ for $p \in \pvar{\alpha} = \pvar{\beta} \cup \pvar{\gamma}$, then by induction hypothesis $\beta[b] = \beta[b^\prime]$ and $\gamma[b] = \gamma[b^\prime]$. So $\alpha[b] = \dsjfunc (\beta[b], \gamma[b]) = \dsjfunc (\beta[b^\prime], \gamma[b^\prime]) = \alpha[b^\prime]$.\nolinebreak\hfill$\talloblong$
%
\item \textbf{Note to the Paragraph Following 4.2.} Let $\alpha \in \pf_{n + 1}$, then the map from $\{ T, F \}^{n + 1}$ to $\{ T, F \}$ with
\begin{center}
$(b_0, \ldots, b_n) \mapsto \alpha[b_0, \ldots, b_n]$ \ \ for all $b_0, \ldots, b_n \in \{ T, F \}$
\end{center} 
is a function.
%
\item \textbf{Note to 4.3.} We use induction on equality-free and quantifier-free formulas $\varphi$ to show $\rho ( \pi ( \varphi ) )$:\\
\ \\
$\varphi$ is atomic: $\pi ( \varphi ) = \pi_0 ( \varphi )$. So $\rho ( \pi ( \varphi ) ) = \rho ( \pi_0 ( \varphi ) ) = \pi_0^{-1} ( \pi_ ( \varphi ) ) = \varphi$.\\
\ \\
$\varphi = \neg\psi$: $\rho ( \pi ( \neg\psi ) ) = \rho ( \neg \pi ( \psi ) ) = \neg \rho ( \pi ( \psi ) ) = \neg\psi$, by induction hypothesis.\\
\ \\
$\varphi = (\psi \lor \chi)$: $\rho ( \pi ( \psi \lor \chi ) ) = \rho ( \pi ( \psi ) \lor \pi ( \chi ) ) = \rho ( \pi ( \psi ) ) \lor \rho ( \pi ( \chi ) ) = \psi \lor \chi$, by induction hypothesis.\\
\ \\
Then we use induction on propositional formulas $\alpha$ to show $\pi ( \rho ( \alpha ) ) = \alpha$:\\
\ \\
$\alpha = p_i$: $\rho ( p_i ) = \pi_0^{-1} ( p_i )$, an equality-free atomic formula. So $\pi ( \rho ( p_i ) ) = \pi ( \pi_0^{-1} ( p_i ) ) = \pi_0 ( \pi_0^{-1} ( p_i ) ) = p_i$.\\
\ \\
$\alpha = \neg\beta$: $\pi ( \rho ( \neg\beta ) ) = \pi ( \neg\rho ( \beta ) ) = \neg\pi ( \rho ( \beta ) ) = \neg\beta$, by induction hypothesis.\\
\ \\
$\alpha = (\beta \lor \gamma)$: $\pi ( \rho ( \beta \lor \gamma ) ) = \pi ( \rho ( \beta ) \lor \rho ( \gamma ) ) = \pi ( \rho ( \beta ) ) \lor \pi ( \rho ( \gamma ) ) = \beta \lor \gamma$.
%
\item \textbf{Note to 4.4.} From the text we understand that for every set $\Phi$ of equality-free and quantifier-free $S$-formulas, $\pi (\Phi)$ denotes the set
\[
\{ \pi ( \varphi ) \mid \varphi \in \Phi \}
\]
of propositional formulas.\\
\ \\
Analogous to Lemma III.4.4, we have 
\begin{quote}
\emph{for all $\Delta \subset \pf$ and all $\alpha \in \pf$, $\Delta \models \alpha$ \ \ iff \ \ not $\sat \Delta \cup \{ \neg\alpha \}$.}
\end{quote}
The proof is similar to the one of III.4.4 given in text.\\
\ \\
Now assuming $b$ is a model of $\pi (\Phi)$, we prove that for all \emph{equality-free and quantifier-free $S$-formulas $\varphi$,}
\begin{center}
$\INT \models \varphi$ \ \ \ iff \ \ \ $\pi ( \varphi ) [b] = T$,
\end{center}
using induction on $\varphi$, where $\INT$ is the $S$-interpretation constructed in the proof in text. Then in particular, $\INT \models \Phi$ as $b$ is a model of $\pi (\Phi)$.\\
\ \\
$\varphi$ is atomic: It directly follows from $(*)$ in text.\\
\ \\
$\varphi = \neg\psi$:
\begin{center}
\begin{tabular}{lll}
\   & $\INT \models \neg\psi$ & \ \cr
iff & not \ $\INT \models \psi$ & \ \cr
iff & not \ $\pi ( \psi )[b] = T$ & \ (by induction hypothesis) \cr
iff & $\pi ( \psi )[b] = F$ & \ \cr
iff & $\negfunc (\pi ( \psi )[b]) = T$ & \ (by the definition of $\negfunc$) \cr
iff & $\neg\pi ( \psi )[b] = T$ & \ (cf. page 201) \cr
iff & $\pi (\neg\psi)[b] = T$ & \ (by the definition of $\pi$)
\end{tabular}
\end{center}
$\varphi = (\psi \lor \chi)$:
\begin{center}
\begin{tabular}{lll}
\   & $\INT \models (\psi \lor \chi)$ & \ \cr
iff & $\INT \models \psi$ \ \ or \ \ $\INT \models \chi$ & \ \cr
iff & $\pi ( \psi )[b] = T$ \ \ or \ \ $\pi ( \chi )[b] = T$ & \ (by induction hypothesis) \cr
iff & $\dsjfunc ( \pi ( \psi )[b], \pi ( \chi )[b] ) = T$ & \ (by the definition of $\dsjfunc$) \cr
iff & $( \pi ( \psi ) \lor \pi ( \chi ))[b] = T$ & \ (cf. page 201) \cr
iff & $\pi ( \psi\lor\chi )[b] = T$ & \ (by the definition of $\pi$).
\end{tabular}
\end{center}
%
\item \textbf{Note to the Paragraph Immediately below the Proof of 4.4.} An account for the counterexample mentioned in text is that, for any $S$-interpretation, the meaning of $\equiv$ is fixed; in other words, $\equiv$ is \emph{always} interpreted as the equality relation (cf. III.3.2).
%
\item \textbf{Note to the Compactness Theorem for Propositional Logic 4.5.} In the proof of it given in text, we may insert between the last two statements an addtional bi-implicational statement:
\begin{center}
\begin{tabular}{ll}
iff & for every finite subset $\Delta_0$ of $\Delta$, $\sat \pi^{-1}\left( \Delta_0 \right)$\cr
\   & (by the bijectiveness of $\pi^{-1}$)
\end{tabular}
\end{center}
As noted in \textbf{Note to 4.4}, the consequence relation and the satisfaction relation of propositional logic can be interrelated: for all $\Delta \subset \pf$ and all $\alpha \in \pf$, $\Delta \models \alpha$ iff not $\sat \Delta \cup \{ \neg\alpha \}$.\\
\ \\
Then the Compactness Theorem for the consequence relation of propositional logic:
\begin{quote}
\emph{``For all $\Delta \subset \pf$ and all $\alpha \in \pf$, $\Delta \models \alpha$ \ \ iff \ \ there is a finite $\Delta_0 \subset \Delta$ such that $\Delta_0 \models \alpha$.''}
\end{quote}
can be proved as follows: for all $\Delta \subset \pf$ and all $\alpha \in \pf$,
\begin{center}
\begin{tabular}{ll}
\   & $\Delta \models \alpha$ \cr
iff & not $\sat \Delta \cup \{ \neg\alpha \}$ \cr
\   & (by the earlier discussion) \cr
iff & there is a finite $\Delta_0 \subset \Delta$ such that not $\sat \Delta_0 \cup \{ \neg\alpha \}$ \cr
\   & (by 4.5) \cr
iff & there is a finite $\Delta_0 \subset \Delta$ such that $\Delta_0 \models \alpha$ \cr
\   & (by the earlier discussion).
\end{tabular}
\end{center}
%
\item \textbf{Note to the Paragraph Discussing DNF and CNF Following 4.5.} As having been noted in \textbf{Note to Lemma 4.2} in the annotations to Chapter VIII, we speak of disjuctions and conjunctions in a more general sense when it comes to DNF and CNF: A disjunction (or conjunction) may have only one disjunct (or conjunct, respectively).
%
\item \textbf{Note to the Proof of Theorem 4.6.} Here we show that 
\begin{quote}
for all $b_0, \ldots, b_n \in \{ T, F \}$, $h(b_0, \ldots, b_n) = \alpha_\mathrm{C}[b_0, \ldots, b_n]$
\end{quote}
to complete the proof.\\
\ \\
It is useful to consider the equivalent statement (2)$^\prime$ of (2):
\begin{center}
(2)$^\prime$ \hfill $\beta^{b_0, \ldots, b_n}[b_0^\prime, \ldots, b_n^\prime] = F$ \ \ iff \ \ $b_0 = b_0^\prime$ and \ldots and $b_n = b_n^\prime$. \hfill \phantom{(2)$^\prime$}
\end{center}
If $h(b_0, \ldots, b_n) = F$ then $\beta^{b_0, \ldots, b_n}$ is a member of the conjunction $\alpha_\mathrm{C}$. By (2)$^\prime$ we have $\beta^{b_0, \ldots, b_n}[b_0, \ldots, b_n] = F$, so $\alpha_\mathrm{C}[b_0, \ldots, b_n] = F$. Conversely, if $\alpha_\mathrm{C}[b_0, \ldots, b_n] = F$ then by definition there is a member $\alpha_0$ of the conjunction $\alpha_\mathrm{C}$ such that $\alpha_0[b_0, \ldots, b_n] = F$, i.e. there are truth-values $b_0^\prime, \ldots, b_n^\prime \in \{ T, F \}$ with $h(b_0^\prime, \ldots, b_n^\prime) = F$ and $\beta^{b_0^\prime, \ldots, b_n^\prime}[b_0, \ldots, b_n] = F$ ($\alpha_0 = \beta^{b_0^\prime, \ldots, b_n^\prime}$). From (2)$^\prime$ it follows that $b_0^\prime = b_0, \ldots, b_n^\prime = b_n$, therefore $h(b_0, \ldots, b_n) = F$.
%
\item \textbf{Solution to Exercise 4.9.} By 4.6, every truth-function can be defined with $\negfunc$ and $\dsjfunc$; more precisely, it can be defined by a propositional formula, which only involves the two connectives $\neg$ and $\lor$.\\
\ \\
For this exercise, therefore, it suffices to show $\negfunc$ and $\dsjfunc$ can be defined in terms of the truth-function(s) given in cases (a) and (b), respectively:
\begin{enumerate}[(a)]
\item For $b_0, b_1 \in \{ T, F \}$, $\dsjfunc (b_0, b_1) = \negfunc (\cnjfunc (\negfunc (b_0), \negfunc (b_1)))$.\\
\ \\
On the other hand, we could so argue: From the viewpoint of logical equivalence, $(\alpha \lor \beta)$ is equivalent to $\neg (\neg \alpha \land \neg \beta)$ for all $\alpha, \beta \in \pf$; by 4.4(c) together with Exercise III.4.12(b), every propositional formula is logically equivalent to one involving only $\neg$ and $\land$. Hence, we may regard $\lor$ as abbreviations of $\neg$ and $\land$ instead (in contrast with the discussion at the bottom of page 35).
%%
\item For $b_0, b_1 \in \{ T, F \}$, $\negfunc (b_0) = \mathop{\stackrel{.}{|}} (b_0, b_0)$ and $\dsjfunc (b_0, b_1) = \mathop{\stackrel{.}{|}}(\mathop{\stackrel{.}{|}}(b_0, b_0), \mathop{\stackrel{.}{|}}(b_1, b_1))$.\\
\ \\
As in (a), we could argue otherwise: For all $\alpha, \beta \in \pf$, $\neg\alpha$ is logically equivalent to $(\alpha \mathop{|} \alpha)$, and $(\alpha \lor \beta)$ to $((\alpha \mathop{|} \alpha) \mathop{|} (\beta \mathop{|} \beta))$; hence every propositional formula is logically equivalent to one involving only $\mathop{|}$. We may take $\neg$ and $\lor$ as abbreviations of $\mathop{|}$.
\end{enumerate}
%
\item \textbf{Solution to Exercise 4.10.} First we prove the Theorem on the Disjunctive Normal Form for Propositional Logic. To start, we set $S \colonequals \{ P \}$ with unary $P$ and define $\pi_0$ on $A^S = \{ Pv_n \mid n \in \nat \}$ by $\pi_0 ( Pv_n ) \colonequals p_n$ for $n \in \nat$. As in text, we extend $\pi_0$ to a bijection $\pi$ from the set of equality-free and quantifier-free $S$-formulas to the set $\pf$ of propositional formulas.\\
\ \\
Let $\alpha$ be a propositional formula. If $\alpha$ is not satisfiable, then $\alpha [b] = F$ for every assignment $b$; hence $(\alpha \leftrightarrow (p_0 \land \neg p_0))[b] = T$ for every assignment $b$, i.e. $\alpha$ is logically equivalent to $(p_0 \land \neg p_0)$, a propositional formula in disjunctive normal form.\\
\ \\
So let $\alpha$ be satisfiable. By 4.4(a), the equality-free and quantifier-free $S$-formula $\pi^{-1} ( \alpha )$ is also satisfiable. Note that all atomic subformulas in $\pi^{-1} ( \alpha )$ are also equality-free. From the proof of Theorem VIII.4.3 (and from Lemma VIII.4.2), $\pi^{-1} ( \alpha )$ is logically equivalent to an $S$-formula $\varphi$ in disjunctive normal form in which all of the atomic subformulas are equality-free; we have $\pi ( \varphi )$ is in disjunctive normal form. By 4.4(c), $\alpha$ is logically equivalent to $\pi ( \varphi )$.\\
\ \\
The Theorem on the Conjunctive Normal Form for Propositional Logic immdiately follows: For any propositional formula $\alpha$, apply the Theorem on the Conjunctive Normal Form for First-Order Logic (cf. Exercise VIII.4.7) to the equality-free and quantifier-free $S$-formula $\pi^{-1}( \alpha )$ to obtain an equivalent $S$-formula $\varphi$ in conjunctive normal form that is also equality-free and quantifier-free. By 4.4(c), $\alpha$ is logically equivalent to $\pi ( \varphi )$.
%
\item \textbf{Solution to Exercise 4.11.} In this exercise we provide a proof other than that of Theorem 4.5 given in text. More precisely, we shall prove the \emph{if}-part, as the other is trivial.\\
\ \\
Before we start, notice that if $\Delta$ is empty, then every subset of it is also empty; the conclusion is vacuously true, namely there are arbitrarily long good sequences and there is an assignment (actually, \emph{any} assignment) that is a model of $\Delta$ and of all finite subsets of it.\\
\ \\
So let us assume that $\Delta$ is nonempty, in addition to the premise that every finite subset of it is satisfiable.\\
\ \\
Suppose, for the sake of a contradiction, that for some $n \in \nat$ all sequences of truth-values $(b_0, \ldots, b_n)$ of length $(n + 1)$ are not good, i.e. for every sequence $(b_0, \ldots, b_n)$ of length $(n + 1)$ there is a finite subset $\Delta_0^\prime \subset \Delta$ that does not have any model $b$ with $b(p_i) = b_i$ for $i \leq n$. We take the union $\Delta_0$ of such $\Delta_0^\prime$ for each $(b_0, \ldots, b_n)$; thus, for all sequences $(b_0, \ldots, b_n)$, $\Delta_0$ does not have any model $b$ with $b(p_i) = b_i$ for $i \leq n$. So $\Delta_0$ is not satisfiable: If $b$ were a model of $\Delta_0$, there would be a sequence $(b_0, \ldots, b_n)$ such that $b_i = b(p_i)$ for $i \leq n$.\\
\ \\
On the other hand, however, $\Delta_0$ is finite, as there are finitely many sequences of truth-values of length $(n + 1)$ ($2^{n + 1}$ in total). By the premise $\Delta_0$ is satisfiable, a contradiction.\\
\ \\
Therefore, there are arbitrarily long good sequences. And since for every good sequence all of its prefixes (for example, $(T, F, T)$ is a prefix of $(T, F, T, T)$) are also good, there is an infinite chain of good sequences
\[
(b_0), (b_0, b_1), (b_0, b_1, b_2), \ldots
\]
If we take an assignment $b$ in which $b(p_i) = b_i$ for $i \in \nat$, then in particular $b$ is a model of $\{ \alpha \}$ (by the Coincidence Lemma) for every $\alpha \in \Delta$, hence a model of $\Delta$.
%
\item \textbf{Solution to Exercise 4.12.} We define some notions analogous to those of first-order logic:
\begin{enumerate}[(i)]
\item A sequent (of propositional formulas) is a finite nonempty sequence of propositional formulas.
%%
\item A derivation (over propositional formulas) is a finite nonempty sequence of sequents in which the first sequent is obtained by applying the rule $\assm$ and inductively all other sequents are obtained by applying the rule $\assm$ or other rules in $\mathfrak{S}_\mathrm{a}$ to previous sequents.
%%
\item Let $\Gamma$ be a (possibly empty) sequence of propositional formulas and $\alpha$ a propositional formula. The sequent $\Gamma \alpha$ is derivable in $\mathfrak{S}_\mathrm{a}$ (written: $\vdash_\mathrm{a} \Gamma \alpha$) iff there is a derivation of which the last sequent is $\Gamma \alpha$. If this is the case, we say $\alpha$ is derivable from $\Gamma$ (written: $\Gamma \vdash_\mathrm{a} \alpha$).
%%
\item For $\Delta \subset \pf$ and $\alpha \in \pf$, $\alpha$ is derivable from $\Delta$ in $\mathfrak{S}_\mathrm{a}$ (written: $\Delta \vdash_\mathrm{a} \alpha$) iff there are formulas $\alpha_1, \ldots, \alpha_n$ in $\Delta$ such that $\vdash_\mathrm{a} \alpha_1 \ldots \alpha_n \ \alpha$.
%%
\item For $\Delta \subset \pf$, $\Delta$ is consistent (written: $\con_\mathrm{a} \Delta$) iff there is no $\alpha \in \pf$ such that $\Delta \vdash_a \alpha$ and $\Delta \vdash_a \neg\alpha$. $\Delta$ is inconsistent (written: $\inc_\mathrm{a} \Delta$) iff it is not consistent.
%%
\item For $\Delta \subset \pf$, $\Delta$ is negation complete iff $\Delta \vdash_\mathrm{a} \alpha$ or $\Delta \vdash_\mathrm{a} \neg\alpha$ for every $\alpha \in \pf$.
\end{enumerate}
\ \\
Then, we pick a symbol set $S \colonequals \{ P \}$ with a unary relation symbol $P$, and define the (bijective) map $\pi$ from the set of equality-free and quantifier-free $S$-formulas to the set of propositional formulas by
\[
\begin{array}{lll}
\pi ( Pv_n ) & \colonequals & p_n, \ n \in \nat \cr
\pi ( \neg\varphi ) & \colonequals & \neg \pi ( \varphi ) \cr
\pi ( ( \varphi\lor\psi ) ) & \colonequals & \pi ( \varphi ) \lor \pi ( \psi ).
\end{array}
\]
In a natural way, a sequent of equality-free and quantifier-free $S$-formulas corresponds to a sequent of propositional formulas, and vice versa; a derivation in $\mathfrak{S}_\mathrm{a}$ that consists of only sequents of equality-free and quantifier-free $S$-formulas thus corresponds to a derivation in $\mathfrak{S}_\mathrm{a}$ over propositional formulas, and vice versa.\\
\ \\
The correctness part of the Adequacy Theorem is easy: For $\Delta \subset \pf$ and $\alpha \in \pf$, we have
\begin{center}
\begin{tabular}{lll}
\    & $\Delta \vdash_\mathrm{a} \alpha$ \cr
iff  & $\pi^{-1} ( \Delta ) \vdash_\mathrm{a} \pi^{-1} ( \alpha )$ & (by the above discussion)\cr
then & $\pi^{-1} ( \Delta ) \vdash \pi^{-1} ( \alpha )$ & (a derivation in $\mathfrak{S}_\mathrm{a}$ is also one in $\mathfrak{S}$) \cr
then & $\pi^{-1} ( \Delta ) \models \pi^{-1} ( \alpha )$ & (by the Correctness Theorem for \cr
\    & \ & \phantom{(} first-order logic) \cr
iff  & $\Delta \models \alpha$ & (by 4.4(b)). \cr
\end{tabular}
\end{center}
\ \\
To prove the completeness part, let us first note, analogous to first-order logic, the following:
\begin{enumerate}[(1)]
\item For all $\Delta \subset \pf$, $\con_\mathrm{a} \Delta$ if and only if $\con_\mathrm{a} \Delta_0$ for all subsets $\Delta_0$ of $\Delta$.
%%
\item For all $\Delta \subset \pf$ and all $\alpha \in \pf$ the following holds:
\begin{enumerate}[(a)]
\item $\Delta \vdash_\mathrm{a} \alpha$ \ \ \ iff \ \ \ $\inc_\mathrm{a} \Delta \cup \{\neg\alpha\}$.
%%%
\item $\Delta \vdash_\mathrm{a} \neg\alpha$ \ \ \ iff \ \ \ $\inc_\mathrm{a} \Delta \cup \{ \alpha \}$.
\end{enumerate}
(cf. Lemma IV.7.6)
%%
\item For $n \in \nat$, let $\Delta_n$ be a consistent set of propositional formulas such that
\[
\Delta_0 \subset \Delta_1 \subset \Delta_2 \subset \ldots
\]
Denote $\Delta \colonequals \bigcup_{n \in \nat} \Delta_n$. Then $\con_\mathrm{a} \Delta$. (cf. Lemma IV.7.7.)
%%
\item Let $\Delta \subset \pf$ be consistent and negation complete, and let $\alpha, \beta \in \pf$. Then the following holds:
\begin{enumerate}[(a)]
\item $\Delta \vdash_\mathrm{a} \neg\alpha$ iff not $\Delta \vdash_\mathrm{a} \alpha$.
%%%
\item $\Delta \vdash_\mathrm{a} (\alpha \lor \beta)$ iff $\Delta \vdash_\mathrm{a} \alpha$ or $\Delta \vdash_\mathrm{a} \beta$.
\end{enumerate}
(cf. V.1.9.)
\end{enumerate}
\ \\
Next, we show that for $\Delta \subset \pf$, if $\Delta$ is consistent and negation complete, then it has a model (cf. Henkin's Theorem V.1.10): Let $b$ be an assignment with $b(p_n) = T$ iff $\Delta \vdash_\mathrm{a} p_n$ for $n \in \nat$. We prove that for $\alpha \in \pf$,
\begin{center}
$\alpha [b] = T$ \ \ \ iff \ \ \ $\Delta \vdash_\mathrm{a} \alpha$,
\end{center}
by induction on $\alpha$:
\begin{enumerate}[(a)]
\item $\alpha = p_n$: By definition of $b$.
%%
\item $\alpha = \neg\beta$: $\neg\beta [b] = T$
\begin{quote}
\begin{tabular}{lll}
iff & $\beta [b] = F$ & \ \cr
iff & not $\Delta \vdash_\mathrm{a} \beta$ & (by induction hypothesis) \cr
iff & $\Delta \vdash_\mathrm{a} \neg\beta$ & (by (a) of (4)).
\end{tabular}
\end{quote}
%%
\item $\alpha = (\beta\lor\gamma)$: $(\beta\lor\gamma) [b] = T$
\begin{quote}
\begin{tabular}{lll}
iff & $\beta [b] = T$ or $\gamma [b] = T$ & \ \cr
iff & $\Delta \vdash_\mathrm{a} \beta$ or $\Delta \vdash_\mathrm{a} \gamma$ & (by induction hypothesis) \cr
iff & $\Delta \vdash_\mathrm{a} (\beta \lor \gamma)$ & (by (b) of (4)).
\end{tabular}
\end{quote}
\end{enumerate}
In particular, $\alpha [b] = T$ for $\alpha \in \Delta$, namely $b$ is a model of $\Delta$.\\
\ \\
Then, we show that for $\Delta \subset \pf$, if $\Delta$ is consistent then there is a set $\Delta^\prime \supset \Delta$ of propositional formulas that is consistent and negation complete (cf. V.2.2): Since $\pf$ is countable, we let $\alpha_0, \alpha_1, \alpha_2, \ldots$ be an enumeration of $\pf$. We define sets $\Delta_n^\prime$ inductively as follows:
\[
\Delta_0^\prime \colonequals \Delta,
\]
and for $n \in \nat$,
\[
\Delta_{n + 1}^\prime \colonequals \begin{cases}
\Delta_n^\prime \cup \{ \alpha_n \} & \mbox{if \(\con_\mathrm{a} \Delta_n^\prime \cup \{ \alpha_n \}\)} \cr
\Delta_n^\prime & \mbox{otherwise}.
\end{cases}
\]
An easy induction on $n$ shows that $\con_\mathrm{a} \Delta_n^\prime$ for $n \in \nat$. By (3) we have $\Delta^\prime \colonequals \bigcup_{n \in \nat} \Delta_n^\prime$ is consistent. On the other hand, if $\alpha = \alpha_n \in \pf$ for some $n \in \nat$ such that not $\Delta^\prime \vdash_\mathrm{a} \neg\alpha$, then $\con_\mathrm{a} \Delta^\prime \cup \{ \alpha \}$ by (b) of (2). By (1) we have $\con_\mathrm{a} \Delta_n^\prime \cup \{ \alpha \}$. So $\Delta_{n + 1}^\prime = \Delta_n^\prime \cup \{ \alpha \}$ and hence $\alpha \in \Delta_{n + 1}^\prime$, therefore $\Delta^\prime \vdash_\mathrm{a} \alpha$. It turns out that $\Delta^\prime$ is negation complete.
\\
\ \\
Since a model of a consistent and negation complete set $\Delta^\prime \supset \Delta$ is also one of $\Delta$, we have: For $\Delta \subset \pf$, if $\Delta$ is consistent then it is satisfiable.\\
\ \\
Finally, we are done if we can show that, for $\Delta \subset \pf$, the following $(*)$ implies $(**)$:\\
\ \\
\begin{tabular}{ll}
$(*)$ & If $\Delta$ is consistent then it is satisfiable. \cr
$(**)$ & For $\alpha \in \pf$, if $\Delta \models \alpha$ then $\Delta \vdash_\mathrm{a} \alpha$.
\end{tabular}\\
\ \\
Assume that $(*)$ holds. Let $\alpha \in \pf$ such that $\Delta \models \alpha$, then not $\sat \Delta \cup \{ \neg\alpha \}$ (cf. \textbf{Note to 4.4}). By $(*)$ we have that $\inc_\mathrm{a} \Delta \cup \{ \neg\alpha \}$, and furthermore that $\Delta \vdash_\mathrm{a} \alpha$ by (a) of (2).\\
\ \\
\textit{Remark.} The above $(**)$ also implies $(*)$: Assume that $(**)$ holds. If not $\sat \Delta$, then there is $\alpha \in \pf$ such that $\Delta \models \alpha$ and $\Delta \models \neg\alpha$. By $(**)$ we have both $\Delta \vdash_\mathrm{a} \alpha$ and $\Delta \vdash_\mathrm{a} \neg\alpha$, i.e. $\inc_\mathrm{a} \Delta$.
\end{enumerate}
%End of Section XI.4-------------------------------------------------------------------------------
\
\\
\\
%Section XI.5--------------------------------------------------------------------------------------
{\large \S5. Propositional Resolution}
\begin{enumerate}[1.]
\item \textbf{Note to Lemma 5.2(b).} It can be stated otherwise:
\begin{enumerate}[(1)]
\item For every propositional variable $q$: If $b^\Delta (q) = T$, then for every assignment $b$ which is a model of $\Delta$ we have $b (q) = T$; or
%%
\item For every propositional variable $q$: If $b^\Delta (q) = T$, then $\Delta \models q$.
\end{enumerate}
Since $b^\Delta (q) = T$ follows from $\Delta \models q$ as well for every propositional variable $q$, by (2) we obtain:\\
\ \\
$(+)$ \ \ \ For every propositional variable $q$: $b^\Delta (q) = T$ \ \ iff \ \ $\Delta \models q$. \ \ \ \phantom{$(+)$}\\
\ \\
\textit{Remark.} Alternatively, for all $n \in \nat$, $\Delta \models p_n$ iff $p_n$ is derivable in the calculus with the rules (T1) and (T2) given in text (i.e. $b^\Delta (p_n) = T$). This way, we also get the above result.
%
\item \textbf{Another Way to Prove (c) Follows from (b) in Theorem 5.3.} Let us assume that (c) does not hold, i.e. $b^{\Delta^+}$ is not a model of $\Delta$. Then there must be some $\alpha = (\neg q_0 \lor \ldots \lor \neg q_n) \in \Delta^-$ such that $\alpha [b^{\Delta^+}] = F$ or, $b^{\Delta^+}(q_i) = T$ for all $0 \leq i \leq n$, since $b^{\Delta^+}$ is a model of $\Delta^+$ (by 5.2(a)). By $(+)$ in \textbf{Note to Lemma 5.2(b)} we have $\Delta^+ \models q_i$ for $0 \leq i \leq n$, hence $\Delta^+ \models (q_0 \land \ldots \land q_n)$ and therefore not $\sat \Delta^+ \cup \{ \neg q_0 \lor \ldots \lor \neg q_n \}$.
%
\item \textbf{Note to the Remarks to Definition 5.4.} From the Compactness Theorem, it immediately follows that, for any set $\mathfrak{K}$ of clauses, $\mathfrak{K}$ is satisfiable iff each of its finite subset is satisfiable.\\
\ \\
Also, a more appropriate notation than $\bigwedge_{K \in \mathfrak{K}} \bigvee_{\lambda \in K} \lambda$ is $\{ \bigvee_{\lambda \in K} \lambda \mid K \in \mathfrak{K} \}$.
%
\item \textbf{Note to Definition 5.7.} Let $\mathfrak{K}$ be a set of clauses, and $\mathfrak{K}^\prime \subset \res{\mathfrak{K}} \backslash \mathfrak{K}$ be finite. If $K \in \res{\mathfrak{K}}$, then $K \in \res{\mathfrak{K} \cup \mathfrak{K}^\prime}$. So, in the case that $\mathfrak{K}^\prime = \{ K_1, \ldots, K_n \}$, repeated applications of 5.6 yield:
\begin{center}
\begin{tabular}{lll}
\   & $\mathfrak{K}$ is satisfiable & \ \cr
iff & $\mathfrak{K} \cup \{ K_1 \}$ is satisfiable & (apply to $\mathfrak{K}$) \cr
\multicolumn{2}{c}{$\vdots$} & \ \cr
iff & $\mathfrak{K} \cup \{ K_1, \ldots, K_n \}$ is satisfiable & (apply to $\mathfrak{K} \cup \{ K_1, \ldots, K_{n - 1} \}$). 
\end{tabular}
\end{center}
From the discussion in \textbf{Note to the Remarks to Definition 5.4}, we get (cf. the proof of 5.8)
\begin{center}
$\mathfrak{K}$ is satisfiable \ \ \ iff \ \ \ $\res{\mathfrak{K}}$ is satisfiable.
\end{center}
%
\item \textbf{Note to the Resolution Theorem 5.8.} There is a typo in the proof of it in text: In the third last line on page 213, ``$K_T \subset \mathfrak{R}_k$'' should be replaced by ``$K_T \in \mathfrak{R}_k$''.\\
\ \\
On the other hand, we state the following as a direct consequence of this theorem: Let $\mathfrak{K}$ be a set of clauses. By 5.7, we have
\begin{enumerate}[(1)]
\item $\res{\resi{\infty}{\mathfrak{K}}} = \resi{\infty}{\mathfrak{K}}$.
%%
\item $\resi{\infty}{\resi{\infty}{\mathfrak{K}}} = \resi{\infty}{\mathfrak{K}}$.
\end{enumerate}
Thus, by 5.8, $\resi{\infty}{\mathfrak{K}}$ is satisfiable iff $\emptyset \not\in \resi{\infty}{\mathfrak{K}}$.
%
\item \textbf{Note to the Paragraph under FIGURE XI.1.} Let $\mathfrak{K}$ be a set of clauses in which only literals from $P \colonequals \{ p_0, \ldots, p_{n - 1} \} \cup \{ \neg p_0, \ldots, \neg p_{n - 1} \}$ occur. Then for all $i \in \nat$, every resolvent formed from clauses of $\resi{i}{\mathfrak{K}}$ is a subset of $P$. Thus, It is clear that there are at most $2^{2n}$ resolvents. So $\resi{2^{2n}}{\mathfrak{K}}$ contains all resolvents obtained in finitely many steps together with all clauses from $\mathfrak{K}$, hence $\resi{2^{2n}}{\mathfrak{K}} = \resi{\infty}{\mathfrak{K}}$.
%
\item \textbf{Solution to Exercise 5.12.} Let us note that
\[
\begin{array}{lll}
\resi{1}{\mathfrak{K}} & = & \{ \{ p_0, p_1 \}, \{ p_0, p_2 \}, \{ p_0, p_1, p_2 \}, \{ p_0, \neg p_1, p_2 \}, \{ p_0, p_1, \neg p_2 \}, \cr
\ & \ & \phantom{ \{ } \{ p_0, p_1, \neg p_1, p_2 \}, \{ p_0, p_1, p_2, \neg p_2 \} \} \cup \{ \{ \neg p_i \} \mid i \geq 1 \}
\end{array}
\]
and
\[
\begin{array}{lll}
\resi{2}{\mathfrak{K}} & = & \{ \{ p_0 \}, \{ p_0, p_1 \}, \{ p_0, p_2 \}, \{ p_0, \neg p_1 \}, \{ p_0, \neg p_2 \}, \cr
\ & = & \phantom{ \{ } \{ p_0, p_1, p_2 \}, \{ p_0, \neg p_1, p_2 \}, \{ p_0, p_1, \neg p_2 \}, \{ p_0, \neg p_1, \neg p_2 \}, \cr
\ & \ & \phantom{ \} } \{ p_0, p_1, \neg p_1 \}, \{ p_0, p_2, \neg p_2 \}, \cr
\ & \ & \phantom{ \{ } \{ p_0, p_1, \neg p_1, p_2 \}, \{ p_0, p_1, p_2, \neg p_2 \}, \cr
\ & \ & \phantom{ \} } \{ p_0, p_1, \neg p_1, \neg p_2 \}, \{ p_0, \neg p_1, p_2, \neg p_2 \}, \cr
\ & \ & \phantom{ \{ } \{ p_0, p_1, \neg p_1, p_2, \neg p_2 \} \} \cup \{ \{ p_i \} \mid i \geq 1 \}.
\end{array}
\]
\begin{enumerate}[(a)]
\item Since any resolvent formed in finitely many steps from clauses of $\mathfrak{K}$ is a subset of $\{ p_0, p_1, \neg p_1, p_2, \neg p_2 \}$ and it must contain $p_0$ as an element, we have $\resi{\infty}{\mathfrak{K}} \subset \resi{2}{\mathfrak{K}}$. But obviously $\resi{2}{\mathfrak{K}} \subset \resi{\infty}{\mathfrak{K}}$, therefore $\resi{\infty}{\mathfrak{K}} = \resi{2}{\mathfrak{K}}$.
%%
\item We directly have
\[
\begin{array}{lll}
\resi{2}{\mathfrak{K}} \backslash \resi{1}{\mathfrak{K}} & = & \{ \{ p_0 \}, \{ p_0, \neg p_1 \}, \{ p_0, \neg p_2 \}, \cr
\ & \ & \phantom{ \{ } \{ p_0, \neg p_1, \neg p_2 \}, \{ p_0, p_1, \neg p_1 \}, \{ p_0, p_2, \neg p_2 \}, \cr
\ & \ & \phantom{ \} } \{ p_0, p_1, \neg p_1, \neg p_2 \}, \{ p_0, \neg p_1, p_2, \neg p_2 \}, \cr
\ & \ & \phantom{ \{ } \{ p_0, p_1, \neg p_1, p_2, \neg p_2 \} \}
\end{array}
\]
and
\[
\begin{array}{lll}
\resi{1}{\mathfrak{K}} \backslash \mathfrak{K} & = & \{ \{ p_0, p_1 \}, \{ p_0, p_2 \}, \cr
\ & \ & \phantom{ \{ } \{ p_0, p_1, \neg p_2 \}, \{ p_0, \neg p_1, p_2 \}, \cr
\ & \ & \phantom{ \} } \{ p_0, p_1, \neg p_1, p_2 \}, \{ p_0, p_1, p_2, \neg p_2 \} \},
\end{array}
\]
both of which are finite.
%%
\item Since $\emptyset \not\in \resi{\infty}{\mathfrak{K}}$ (cf. (a)), by the Resolution Theorem $\mathfrak{K}$ is satisfiable.
\end{enumerate}
\end{enumerate}
%End of Section XI.5-------------------------------------------------------------------------------
\
\\
\\
%Section XI.6-------------------------------------------------------------------------------------
{\large \S6. First-Order Resolution (without Unification)}
\begin{enumerate}[1.]
\item \textbf{Verifying ``for any term $t \in T^S_0$: $\Phi_2 \vdash Pc_ac_bt$ iff $t$ represents a path from $a$ to $b$ in $(G, R^G)$'', as Suggested in Text.} Let us take the $S$-structure $\mathfrak{G}_2 \colonequals (T^S_0, R^{G_2}, P^{G_2}, F, (c_a)_{a \in G})$, where
\[
\begin{array}{lll}
F(t_1, t_2) & \colonequals & f t_1 t_2, \cr
R^{G_2} & \colonequals & \{ (c_a, c_b) \mid \mbox{there is an edge from $a$ to $b$ in $(G, R^G)$} \}, \cr
P^{G_2} & \colonequals & \{ (c_a, c_b, t) \mid \mbox{$t$ represents a path from $a$ to $b$ in $(G, R^G)$} \}.
\end{array}
\]
Then obviously the direction from left to right holds since $\mathfrak{G}_2$ is a model of $\Phi_2$.\\
\ \\
As for the direction from right to left, let us first note that:\\
($\ast$) \ \ \begin{minipage}{10cm}
If $R^{\prime G} \subset T^S_0 \times T^S_0$ and $P^{\prime G} \subset T^S_0 \times T^S_0 \times T^S_0$, and if $(T^S_0, R^{\prime G}, P^{\prime G}, F, (c_a)_{a \in G}) \models \Phi_2$, then $R^{G_2} \subset R^{\prime G}$ and $P^{G_2} \subset P^{\prime G}$.\end{minipage}\\
In fact, the definition of $\Phi_0$ immediately tells us that $R^{G_2} \subset R^{\prime G}$. By definition of $P^{G_2}$, we have to show for $n \in \nat$ and $a_0, \ldots, a_n \in G$ with $R^G a_i a_{i + 1}$ for $i < n$, there is a term $t \in T^S_0$ such that $P^{\prime G} c_{a_0} c_{a_n} t$. This follows from the axioms in $\Phi_2$ by induction on $n$.\\
\ \\
Thus, ($\ast$) together with 3.8 entails that $\mathfrak{G}_2 = \mathfrak{T}_0^{\Phi_2}$.\\
\ \\
So, for any term $t \in T^S_0$, if $t$ represents a path from $a$ to $b$ in $(G, R^G)$ then by definition $\mathfrak{G}_2 \models P c_a c_b t$; by 3.9 we have $\Phi_2 \vdash P c_a c_b t$ (note that $P c_a c_b t$ is a \emph{sentence}).
%
\item \textbf{Note to the Discussion before Example 6.5.} For discussions later on (cf. the proofs of 7.14, of 7.17, and of 7.18 in text), we extend Definition 6.2 to accommodate sets of first-order clauses: A set $\mathfrak{K}$ of first-order clauses is called \emph{propositionally satisfiable} \ \ :iff \ \ $\{ \pi(K) \mid K \in \mathfrak{K} \}$ is satisfiable.
%
\item \textbf{Note to Example 6.6.} By 6.4, the satisfiability of the sentence
\[
\forall x \forall y ((Rxy \lor Qx) \land \neg Rxgx \land \neg Qy)
\]
is equivalent to the propositional satisfiability of
\[
\{ (Rt_1 t_2 \lor Qt_1) \land \neg Rt_1 gt_1 \land \neg Qt_2 \mid t_1, t_2 \in T^S_0 \},
\]
which in turn is equivalent to that of
\[
\{ Rt_1 t_2 \lor Qt_1 \mid t_1, t_2 \in T^S_0 \} \cup \{ \neg Rtgt \mid t \in T^S_0 \} \cup \{ \neg Qt \mid t \in T^S_0 \},
\]
which is equivalent to the satisfiability of the set of clauses
\[
\{ \{ Rt_1 t_2, Qt_1 \} \mid t_1, t_2 \in T^S_0 \} \cup \{ \{ \neg Rtgt \} \mid t \in T^S_0 \} \cup \{ \{ \neg Qt \} \mid t \in T^S_0 \}.
\]
To obtain the resolution tree in FIGURE XI.6, we should take the clauses
\begin{quote}
\begin{tabular}{ll}
$\{ Rt_1 t_2, Qt_1 \}$ & with $t_1 = ggc$ and $t_2 = gggc$, \cr
$\{ \neg Rtgt \}$ & with $t = ggc$, and \cr
$\{ \neg Qt \}$ & with $t = ggc$
\end{tabular}
\end{quote}
as the leaves.
%
\item \textbf{Solution to Exercise 6.11.} The statement
\[
c^{A/_{\scriptstyle E}} \colonequals \overline{c^A}
\]
is missing from the definition of $\mathfrak{A}/_{\displaystyle E}$.
\begin{enumerate}[(a)]
\item Let us first note that
\begin{enumerate}[(i)]
\item For any $a, b \in A$: $E^A a b$ \ iff \ $\overline{a} = \overline{b}$.
%%%
\item For any $t \in T^S$: $(\mathfrak{A}/_{\displaystyle E}, \beta/_{\displaystyle E})(t) = \overline{((\mathfrak{A}, E^A), \beta)(t)}$.
\end{enumerate}
Since $E^A$ is an equivalence relation, (i) immediately follows. We can prove (ii) by induction on $t$:\\
If $t = x$, then
\[
\begin{array}{ll}
\ & (\mathfrak{A}/_{\displaystyle E}, \beta/_{\displaystyle E})(x) \cr
= & \beta/_{\displaystyle E}(x) \cr
= & \overline{\beta(x)} \cr
= & \overline{((\mathfrak{A}, E^A), \beta)(x)}.
\end{array}
\]
If $t = c$, then
\[
\begin{array}{ll}
\ & (\mathfrak{A}/_{\displaystyle E}, \beta/_{\displaystyle E})(c) \cr
= & c^{A/_{\scriptstyle E}} \cr
= & \overline{c^A} \cr
= & \overline{((\mathfrak{A}, E^A), \beta)(c)}.
\end{array}
\]
If $t = ft_1 \ldots t_n$, then
\[
\begin{array}{ll}
\ & (\mathfrak{A}/_{\displaystyle E}, \beta/_{\displaystyle E})(ft_1 \ldots t_n) \cr
= & f^{A/_{\scriptstyle E}} ((\mathfrak{A}/_{\displaystyle E}, \beta/_{\displaystyle E})(t_1), \ldots, (\mathfrak{A}/_{\displaystyle E}, \beta/_{\displaystyle E})(t_n)) \cr
= & f^{A/_{\scriptstyle E}} (\overline{((\mathfrak{A}, E^A), \beta)(t_1)}, \ldots, \overline{((\mathfrak{A}, E^A), \beta)(t_n)}) \cr
\ & \multicolumn{1}{r}{\mbox{(by induction hypothesis)}} \cr
= & \overline{f^A (((\mathfrak{A}, E^A), \beta)(t_1), \ldots, ((\mathfrak{A}, E^A), \beta)(t_n))} \cr
= & \overline{((\mathfrak{A}, E^A), \beta)(ft_1 \ldots t_n)}.
\end{array}
\]
\ \\
We are done after we prove
\begin{quote}
for every $\varphi \in L^S$ and \emph{for any assignment $\beta$ in $(\mathfrak{A}, E^A)$}: $((\mathfrak{A}, E^A), \beta) \models \varphi^\ast$ \ iff \ $(\mathfrak{A}/_{\displaystyle E}, \beta/_{\displaystyle E}) \models \varphi$
\end{quote}
by induction on $\varphi$:\\
If $\varphi = t_1 \equiv t_2$, then $\varphi^\ast = Et_1 t_2$ and\\
\begin{tabular}{ll}
\   & $((\mathfrak{A}, E^A), \beta) \models Et_1 t_2$ \cr
iff & $E^A$ holds for $((\mathfrak{A}, E^A), \beta)(t_1), ((\mathfrak{A}, E^A), \beta)(t_2)$ \cr
iff & $\overline{((\mathfrak{A}, E^A), \beta)(t_1)} = \overline{((\mathfrak{A}, E^A), \beta)(t_2)}$ \ \ (by (i)) \cr
iff & $(\mathfrak{A}/_{\displaystyle E}, \beta/_{\displaystyle E})(t_1) = (\mathfrak{A}/_{\displaystyle E}, \beta/_{\displaystyle E})(t_2)$ \ \ (by (ii)) \cr
iff & $(\mathfrak{A}/_{\displaystyle E}, \beta/_{\displaystyle E}) \models t_1 \equiv t_2$.
\end{tabular}\\
\ \\
If $\varphi = Rt_1 \ldots t_n$, then $\varphi^\ast = Rt_1 \ldots t_n$ and\\
\begin{tabular}{ll}
\   & $((\mathfrak{A}, E^A), \beta) \models Rt_1 \ldots t_n$ \cr
iff & $R^A$ holds for $((\mathfrak{A}, E^A), \beta)(t_1), \ldots, ((\mathfrak{A}, E^A), \beta)(t_n)$ \cr
iff & $R^{A/_{\scriptstyle E}}$ holds for $\overline{((\mathfrak{A}, E^A), \beta)(t_1)}, \ldots, \overline{((\mathfrak{A}, E^A), \beta)(t_n)}$ \cr
iff & $R^{A/_{\scriptstyle E}}$ holds for $(\mathfrak{A}/_{\displaystyle E}, \beta/_{\displaystyle E})(t_1), \ldots, (\mathfrak{A}/_{\displaystyle E}, \beta/_{\displaystyle E})(t_n)$ \ (by (ii))\cr
iff & $(\mathfrak{A}/_{\displaystyle E}, \beta/_{\displaystyle E}) \models Rt_1 \ldots t_n$.
\end{tabular}\\
\ \\
If $\varphi = \neg\psi$, then $\varphi^\ast = \neg\psi^\ast$ and\\
\begin{tabular}{ll}
\   & $((\mathfrak{A}, E^A), \beta) \models \neg\psi^\ast$ \cr
iff & not $((\mathfrak{A}, E^A), \beta) \models \psi^\ast$ \cr
iff & not $(\mathfrak{A}/_{\displaystyle E}, \beta/_{\displaystyle E}) \models \psi$ \ (by induction hypothesis) \cr
iff & $(\mathfrak{A}/_{\displaystyle E}, \beta/_{\displaystyle E}) \models \neg\psi$.
\end{tabular}\\
\ \\
If $\varphi = \psi \lor \chi$, then $\varphi^\ast = \psi^\ast \lor \chi^\ast$ and\\
\begin{tabular}{ll}
\   & $((\mathfrak{A}, E^A), \beta) \models \psi^\ast \lor \chi^\ast$ \cr
iff & $((\mathfrak{A}, E^A), \beta) \models \psi^\ast$ or $((\mathfrak{A}, E^A), \beta) \models \chi^\ast$ \cr
iff & $(\mathfrak{A}/_{\displaystyle E}, \beta/_{\displaystyle E}) \models \psi$ or $(\mathfrak{A}/_{\displaystyle E}, \beta/_{\displaystyle E}) \models \chi$ \ (by induction hypothesis) \cr
iff & $(\mathfrak{A}/_{\displaystyle E}, \beta/_{\displaystyle E}) \models \psi \lor \chi$.
\end{tabular}\\
\ \\
If $\varphi = \exists x \psi$, then $\varphi^\ast = \exists x \psi^\ast$ and\\
\begin{tabular}{ll}
\   & $((\mathfrak{A}, E^A), \beta) \models \exists x \psi^\ast$ \cr
iff & there is an $a \in A$ such that $((\mathfrak{A}, E^A), \beta\sbst{a}{x}) \models \psi^\ast$ \cr
iff & there is an $a \in A$ such that $(\mathfrak{A}/_{\displaystyle E}, (\beta\sbst{a}{x})/_{\displaystyle E}) \models \psi$ \cr
\   & (by induction hypothesis; note that the statement we are proving \cr
\   & \phantom{(}applies to any assignment in $(\mathfrak{A}, E^A)$) \cr
iff & there is an $a \in A$ such that $(\mathfrak{A}/_{\displaystyle E}, (\beta/_{\displaystyle E})\sbst{\overline{a}}{x}) \models \psi$ \cr
\   & (since $(\beta\sbst{a}{x})/_{\displaystyle E} = (\beta/_{\displaystyle E})\sbst{\overline{a}}{x}$) \cr
iff & there is an $e \in A/_{\displaystyle E}$ such that $(\mathfrak{A}/_{\displaystyle E}, (\beta/_{\displaystyle E})\sbst{e}{x}) \models \psi$ \cr
iff & $(\mathfrak{A}/_{\displaystyle E}, \beta/_{\displaystyle E}) \models \exists x \psi$.
\end{tabular}
%%
\item Throughout $\Psi^\ast$ denotes the set $\{ \psi^\ast \mid \psi \in \Psi \}$ for $\Psi \subset L^S$.\\
\ \\
Before we start, note that there is a typo in the statement of this part of exercise: $\Phi \cup \Psi_E$ should be replaced by $\Phi^\ast \cup \Psi_E$. Actually, the assertion
\begin{quote}
for $\Phi \cup \{ \psi \} \subset L^S$: $\Phi \vdash \psi$ \ iff \ $\Phi \cup \Psi_E \vdash \psi^\ast$
\end{quote}
is \emph{not} true: Let $S = \emptyset$, $\psi = \neg v_0 \equiv v_1$ and $\Phi = \{ \psi \}$. It is clear that $\Phi \vdash \psi$. However, it does not hold that $\Phi \cup \Psi_E \vdash \psi^\ast$: Choose an $\{ E \}$-interpretation $(\mathfrak{A}, \beta)$ with $A = \{ a_0, a_1 \}$, $E^A = A \times A$, $\beta(v_0) = a_0$ and $\beta(v_1) = a_1$. We have $(\mathfrak{A}, \beta) \models \Phi \cup \Psi_E$ but not $(\mathfrak{A}, \beta) \models \psi^\ast$.\\
\ \\
Now let us complete this part of exercise. In the following, let $\Psi_\equiv \subset L^S$ such that $\Psi_\equiv^\ast = \Psi_E$. Note that all of the sentences in $\Psi_\equiv$ are valid. By III.4.4, it suffices to prove\\
($+$) \ \ \begin{minipage}{10cm}
for $\Phi \cup \{ \varphi \} \subset L^S$: $\Phi \cup \{ \neg\varphi \}$ is satisfiable \ iff \ $\Phi^\ast \cup \Psi_E \cup \{ \neg\varphi^\ast \}$ is satisfiable.
\end{minipage}\\
\ \\
The direction from right to left in ($+$) is shown below:\\
\begin{tabular}{ll}
\   & $\Phi^\ast \cup \Psi_E \cup \{ \neg\psi^\ast \}$ is satisfiable \cr
iff & there is an $S \cup \{ E \}$-interpretation $((\mathfrak{A}, E^A), \beta)$ such that \cr
\   & $((\mathfrak{A}, E^A), \beta) \models \Phi^\ast \cup \Psi_E \cup \{ \neg\psi^\ast \}$ \cr
iff & there is an $S \cup \{ E \}$-interpretation $((\mathfrak{A}, E^A), \beta)$ such that \cr
\   & $(\mathfrak{A}/_{\displaystyle E}, \beta/_{\displaystyle E}) \models \Phi \cup \{ \neg\psi \}$. \cr
\   & (by (a) and the fact that sentences in $\Psi_\equiv$ are valid)\cr
\end{tabular}\\
The last statement entails that there is a model of $\Phi \cup \{ \neg\psi \}$, namely $\Phi \cup \{ \neg\psi \}$ is satisfiable.\\
\ \\
For a (nonempty) universe $A$, denote by $=^A$ the equality relation over it. Then we prove the direction from left to right in ($+$):\\
\begin{tabular}{ll}
\   & $\Phi \cup \{ \neg\psi \}$ is satisfiable \cr
iff & $\Phi \cup \Psi_\equiv \cup \{ \neg\psi\}$ is satisfiable \cr
\   & (since sentences in $\Psi_\equiv$ are valid) \cr
iff & there is an $S$-interpretation $(\mathfrak{A}, \beta)$ that is a model of \cr
\   & $\Phi \cup \Psi_\equiv \cup \{ \neg\psi \}$ \cr
iff & there is an $S \cup \{ E \}$-interpretation $((\mathfrak{A}, =^A), \beta)$ that is a model of \cr
\   & $\Phi^\ast \cup \Psi_E \cup \{ \neg\psi^\ast \}$ \cr
\   & ($E^A$ is $=^A$; this equivalence of the two statements follows \cr
\   & \phantom{(}in a way analogous to the Substitution Lemma III.8.3, \cr
\   & \phantom{(}in the \emph{second-order} sense).
\end{tabular}
The last statement tells us that $\Phi^\ast \cup \Psi_E \cup \{ \neg\psi^\ast \}$ is satisfiable.
\end{enumerate}
\ \\
\textit{Remarks.} We shall continue to use the notations $\varphi^\ast$, $\Phi^\ast$, $\Psi_\equiv$ and $=^A$ (cf. the proof of part (b)) below.
\begin{enumerate}[(1)]
\item From the final part of the proof of part (b) we immediately obtain
\begin{center}
For $\varphi \in L^S$: if $\varphi$ is satisfiable, then $\varphi^\ast$ is satisfiable.
\end{center}
(For an $S$-interpretation $\mathfrak{I} = (\mathfrak{A}, \beta)$ that is a model of $\varphi$, let $E^A$ be $=^A$. Then the $S \cup \{ E \}$-interpretation $\mathfrak{I}^\prime = ((\mathfrak{A}, E^A), \beta)$ is a model of $\varphi^\ast$.)\\
\ \\
The converse, however, does \emph{not} hold in general: $\neg E x x$ is satisfiable while $\neg x \equiv x$ is not.
%%
\item Let a symbol set $S$ be given, and let $\mathfrak{S}^\circ$ be the sequent calculus associated to $S$ that consists of all rules in $\mathfrak{S}_S$ other than $\eq$ and $\sub$. Furthermore, we denote by $\mathfrak{S}^\ast$ the sequent calculus associated to $S \cup \{ E \}$ obtained by adding to $\mathfrak{S}^\circ$ the rules
\[
\begin{array}{ll}
\displaystyle \frac{\ }{\ \ E t t \ \ }, & t \in T^S
\end{array}
\]
and
\[
\begin{array}{ll}
\displaystyle \frac{\ \Gamma \ \phantom{E t t^\prime} \ \varphi\sbst{t}{x} \ }{\ \Gamma \ E t t^\prime \ \varphi\sbst{t^\prime}{x} \ }, & t, t^\prime \in T^S
\end{array}
\]
which correspond to $\eq$ and $\sub$, respectively. And then the derivability relations $\vdash^\circ$ and $\vdash^\ast$ associated to $\mathfrak{S}^\circ$ and to $\mathfrak{S}^\ast$, respectively, are defined accordingly.\\
\ \\
If we regard $E t_1 t_2$ as rewriting $t_1 \equiv t_2$, then it is trivial that
\begin{center}
for $\Phi \cup \{ \varphi \} \subset L^S$: $\Phi \vdash \varphi$ \ iff \ $\Phi^\ast \vdash^\ast \varphi^\ast$.
\end{center}
%%
\item (INCOMPLETE) On the other hand, the set $\Psi_\equiv$ is indeed the set of axioms of equality \emph{using the equality symbol $\equiv$}, i.e. it is the counterpart of $\Psi_E$. The use of sentences in $\Psi_\equiv$ as (additional) antecedents of sequents can eliminate applications of the rules $\eq$ and $\sub$ in derivations while the set of all derivable sequents remains unchanged. More precisely,
\begin{center}
for $\Phi \cup \{ \varphi \} \subset L^S$: $\Phi \vdash \varphi$ \ iff \ $\Phi \cup \Psi_\equiv \vdash^\circ \varphi$.
\end{center}
To assert this, we need to show, according to finiteness of derivations, that for sequents $\Gamma \, \varphi \subset L^S$:
\begin{quote}
$\Gamma \vdash \varphi$ \ iff \ there is a sequent $\Gamma^\prime \subset \Psi_\equiv$ such that $\Gamma\Gamma^\prime \vdash^\circ \varphi$.
\end{quote}
It is done below:\\
\ \\
The direction from right to left is easy: Let $\Gamma\Gamma^\prime \vdash^\circ \varphi$. Because $\mathfrak{S}^\circ$ consists of less rules than $\mathfrak{S}_S$, it is clear that $\Gamma\Gamma^\prime \vdash \varphi$. Furthermore, all sentences in $\Psi_\equiv$ and hence all those in $\Gamma^\prime$ are valid (in other words, they are derivable from $\emptyset$ in $\mathfrak{S}_S$), so $\Gamma \vdash \varphi$.\\
\ \\
As for the direction from left to right, we need to show that in all derivations over $\mathfrak{S}_S$ the applications of $\eq$ and $\sub$ (if any) can be replaced by those of other rules in $\mathfrak{S}_S$ together with the use of sentences in $\Psi_\equiv$ as (additional) antecedents. To be precise, we need to show:
\begin{enumerate}[1$^\circ$]
\item For $t \in T^S$, there is a $\Gamma^\prime \subset \Psi_\equiv$ such that $\Gamma^\prime \, t \equiv t$ is derivable in $\mathfrak{S}^\circ$.
%%%
\item For $\Gamma \, \varphi \subset L^S$ and $t, t^\prime \in T^S$, if we are given the sequent $\Gamma \, \varphi\sbst{t}{x}$, then we can provide a derivation of $\Gamma\Gamma^\prime \, t \equiv t^\prime \, \varphi\sbst{t^\prime}{x}$ in $\mathfrak{S}^\circ$, where $\Gamma^\prime \subset \Psi_\equiv$.
\end{enumerate}
\ \\
First, a derivation for case 1$^\circ$ is
\begin{center}
\begin{tabular}{llll}
1. & $\forall x \, x \equiv x$ & $\forall x \, x \equiv x$ & $\assm$ \cr
2. & $\forall x \, x \equiv x$ & $t \equiv t$ & IV.5.5(a1) applied to 1.
\end{tabular}
\end{center}
Now in the following we denote $\psi_1 \colonequals \forall x \forall y (x \equiv y \rightarrow y \equiv x)$ and $\psi_2 \colonequals \forall x \forall y \forall z (x \equiv y \land y \equiv z \rightarrow x \equiv z)$.\\
\ \\
Then, we deal with case 2$^\circ$, which requires much work: If we can show that\\
($\ast$) \hfill \begin{minipage}{10cm}for $t, t^\prime, t_0 \in T^S$, there is a sequent $\Gamma_0 \subset \Psi_\equiv$ such that\\$\Gamma_0 \ t \equiv t^\prime \ t_0\sbst{t}{x} \equiv t_0\sbst{t^\prime}{x}$ is derivable in $\mathfrak{S}^\circ$,\end{minipage}
then we can use it to prove case 2$^\circ$ by induction on $\varphi$, as follows:\\
$\varphi = t_1 \equiv t_2$: Given the sequent $\Gamma \, t_1\sbst{t}{x} \equiv t_2\sbst{t}{x}$ as a premise, we denote
\[
\begin{array}{lll}
\chi_1 & \colonequals & \left(t_1\sbst{t}{x} \equiv t_2\sbst{t}{x} \land t_2\sbst{t}{x} \equiv t_2\sbst{t^\prime}{x} \rightarrow t_1\sbst{t}{x} \equiv t_2\sbst{t^\prime}{x}\right), \cr
\chi_2 & \colonequals & \left(t_1\sbst{t^\prime}{x} \equiv t_1\sbst{t}{x} \land t_1\sbst{t}{x} \equiv t_2\sbst{t^\prime}{x} \rightarrow t_1\sbst{t^\prime}{x} \equiv t_2\sbst{t^\prime}{x}\right),
\end{array}
\]
and provide the derivation\\
\begin{tabular}{llll}
1. & $\Gamma$ & $t_1\sbst{t}{x} \equiv t_2\sbst{t}{x}$ & premise \cr
2. & $\Gamma \ t \equiv t^\prime$ & $t_1\sbst{t}{x} \equiv t_1\sbst{t^\prime}{x}$ & apply ($\ast$) \cr
3. & $\Gamma \ t \equiv t^\prime$ & $t_2\sbst{t}{x} \equiv t_2\sbst{t^\prime}{x}$ & apply ($\ast$) \cr
4. & $\psi_1$ & $\psi_1$ & $\assm$ \cr
5. & $\psi_1$ & $\left(t_1\sbst{t}{x} \equiv t_1\sbst{t^\prime}{x} \rightarrow t_1\sbst{t^\prime}{x} \equiv t_1\sbst{t}{x}\right)$ & IV.5.5(a1) applied to 4. \cr
6. & $\Gamma \ \psi_1 \ \psi_2 \ t \equiv t^\prime$ & $\left(t_1\sbst{t}{x} \equiv t_1\sbst{t^\prime}{x} \rightarrow t_1\sbst{t^\prime}{x} \equiv t_1\sbst{t}{x}\right)$ & $\ant$ applied to 5. \cr
7. & $\psi_2$ & $\psi_2$ & $\assm$ \cr
8. & $\psi_2$ & $\chi_1$ & IV.5.5(a1) applied to 7. \cr
9. & $\psi_2$ & $\chi_2$ & IV.5.5(a1) applied to 7. \cr
10. & $\Gamma \ \psi_1 \ \psi_2 \ t \equiv t^\prime$ & $t_1\sbst{t}{x} \equiv t_2\sbst{t}{x}$ & $\ant$ applied to 1. \cr
11. & $\Gamma \ \psi_1 \ \psi_2 \ t \equiv t^\prime$ & $t_2\sbst{t}{x} \equiv t_2\sbst{t^\prime}{x}$ & $\ant$ applied to 3. \cr
12. & $\Gamma \ \psi_1 \ \psi_2 \ t \equiv t^\prime$ & $\chi_1$ & $\ant$ applied to 8. \cr
13. & $\Gamma \ \psi_1 \ \psi_2 \ t \equiv t^\prime$ & $\left(t_1\sbst{t}{x} \equiv t_2\sbst{t}{x} \land t_2\sbst{t}{x} \equiv t_2\sbst{t^\prime}{x}\right)$ & IV.3.6(b) applied to 10. and 11. \cr
14. & $\Gamma \ \psi_1 \ \psi_2 \ t \equiv t^\prime$ & $t_1\sbst{t}{x} \equiv t_2\sbst{t^\prime}{x}$ & IV.3.5 applied to 12. and 13. \cr
15. & $\Gamma \ \psi_1 \ \psi_2 \ t \equiv t^\prime$ & $t_1\sbst{t}{x} \equiv t_1\sbst{t^\prime}{x}$ & $\ant$ applied to 2. \cr
16. & $\Gamma \ \psi_1 \ \psi_2 \ t \equiv t^\prime$ & $t_1\sbst{t^\prime}{x} \equiv t_1\sbst{t}{x}$ & IV.3.5 applied to 6. and 15. \cr
17. & $\Gamma \ \psi_1 \ \psi_2 \ t \equiv t^\prime$ & $\chi_2$ & $\ant$ applied to 9. \cr
18. & $\Gamma \ \psi_1 \ \psi_2 \ t \equiv t^\prime$ & $\left(t_1\sbst{t^\prime}{x} \equiv t_1\sbst{t}{x} \land t_1\sbst{t}{x} \equiv t_2\sbst{t^\prime}{x}\right)$ & IV.3.6(b) applied to 16. and 14. \cr
19. & $\Gamma \ \psi_1 \ \psi_2 \ t \equiv t^\prime$ & $t_1\sbst{t^\prime}{x} \equiv t_2\sbst{t^\prime}{x}$ & IV.3.5 applied to 17. and 18.
\end{tabular}\\
$\varphi = Rt_1 \ldots t_n$: Given the sequent $\Gamma \, Rt_1\sbst{t}{x} \ldots t_n\sbst{t}{x}$ as a premise, we denote
\[
\begin{array}{lll}
\psi_3 & \colonequals & \forall x_1 \ldots \forall x_n \forall y_1 \ldots \forall y_n \left(\bigwedge\limits^n_{i = 1} x_i \equiv y_i \land Rx_1 \ldots x_n \rightarrow Ry_1 \ldots y_n\right), \cr
\chi_3 & \colonequals & \bigwedge\limits^n_{i = 1} t_i\sbst{t}{x} \equiv t_i\sbst{t^\prime}{x}, \cr
\chi_4 & \colonequals & Rt_1\sbst{t}{x} \ldots t_n\sbst{t}{x}, \cr
\chi_5 & \colonequals & Rt_1\sbst{t^\prime}{x} \ldots t_n\sbst{t^\prime}{x},
\end{array}
\]
and provide the derivation\\
\begin{tabular}{llll}
1. & $\Gamma$ & $Rt_1\sbst{t}{x} \ldots t_n\sbst{t}{x}$ & premise \cr
2. & $\Gamma \ t \equiv t^\prime$ & $t_1\sbst{t}{x} \equiv t_1\sbst{t^\prime}{x}$ & apply ($\ast$) \cr
\multicolumn{4}{c}{$\vdots$} \cr
$(n + 1)$. & $\Gamma \ t \equiv t^\prime$ & $t_n\sbst{t}{x} \equiv t_n\sbst{t^\prime}{x}$ & apply ($\ast$) \cr
$(n + 2)$. & $\Gamma \ t \equiv t^\prime$ & $\bigwedge\limits^n_{i = 1} t_i\sbst{t}{x} \equiv t_i\sbst{t^\prime}{x}$ & successively apply IV.3.6(b) to 2. through $(n + 1)$. \cr
$(n + 3)$. & $\Gamma \ t \equiv t^\prime$ & $Rt_1\sbst{t}{x} \ldots t_n\sbst{t}{x}$ & $\ant$ applied to 1. \cr
$(n + 4)$. & $\Gamma \ t \equiv t^\prime$ & $(\chi_3 \land \chi_4)$ & IV.3.6(b) applied to $(n + 2)$. and $(n + 3)$. \cr
$(n + 5)$. & $\psi_3$ & $\psi_3$ & $\assm$ \cr
$(n + 6)$. & $\psi_3$ & $(\chi_3 \land \chi_4 \rightarrow \chi_5)$ & IV.5.5(a1) applied to $(n + 5)$. \cr
$(n + 7)$. & $\Gamma \ \psi_3 \ t \equiv t^\prime$ & $(\chi_3 \land \chi_4 \rightarrow \chi_5)$ & $\ant$ applied to $(n + 6)$. \cr
$(n + 8)$. & $\Gamma \ \psi_3 \ t \equiv t^\prime$ & $(\chi_3 \land \chi_4)$ & $\ant$ applied to $(n + 4)$. \cr
$(n + 9)$. & $\Gamma \ \psi_3 \ t \equiv t^\prime$ & $Rt_1\sbst{t^\prime}{x} \ldots t_n\sbst{t^\prime}{x}$ & IV.3.5 applied to $(n + 7)$. and $(n + 8)$.
\end{tabular}
$\varphi = \neg\psi$: By induction hypothesis, if $\Gamma^+ \, \psi \subset L^S$ and if $\Gamma^+ \, \psi\sbst{t^\prime}{x}$ is given, then $\Gamma^+ \, t^\prime \equiv t \, \psi\sbst{t}{x}$ is derivable in $\mathfrak{S}^\circ$. Given $\Gamma \, \neg\psi\sbst{t}{x}$, we provide the derivation\\
\begin{tabular}{llll}
1. & $\Gamma$ & $\neg\psi\sbst{t}{x}$ & premise \cr
2. & $\psi\sbst{t^\prime}{x}$ & $\psi\sbst{t^\prime}{x}$ & $\assm$ \cr
3. & $\psi\sbst{t^\prime}{x} \ t^\prime \equiv t$ & $\psi\sbst{t}{x}$ & by the above discussion \cr
4. & $\Gamma \ \psi_1 \ t \equiv t^\prime \ t^\prime \equiv t \ \psi\sbst{t^\prime}{x}$ & $\psi\sbst{t}{x}$ & $\ant$ applied to 3. \cr
5. & $\Gamma \ \psi_1 \ t \equiv t^\prime \ t^\prime \equiv t \ \neg\psi\sbst{t}{x}$ & $\neg\psi\sbst{t^\prime}{x}$ & IV.3.3(a) applied to 4. \cr
6. & $\Gamma \ \psi_1 \ t \equiv t^\prime \ t^\prime \equiv t$ & $\neg\psi\sbst{t}{x}$ & $\ant$ applied to 1. \cr
7. & $\Gamma \ \psi_1 \ t \equiv t^\prime \ t^\prime \equiv t$ & $\neg\psi\sbst{t^\prime}{x}$ & IV.3.2 applied to 6. and 5. \cr
8. & $\psi_1$ & $\psi_1$ & $\assm$ \cr
9. & $\psi_1$ & $(t \equiv t^\prime \rightarrow t^\prime \equiv t)$ & IV.5.5(a1) applied to 8. \cr
10. & $\Gamma \ \psi_1 \ t \equiv t^\prime$ & $(t \equiv t^\prime \rightarrow t^\prime \equiv t)$ & $\ant$ applied to 9. \cr
11. & $\Gamma \ \psi_1 \ t \equiv t^\prime$ & $t \equiv t^\prime$ & $\assm$ \cr
12. & $\Gamma \ \psi_1 \ t \equiv t^\prime$ & $t^\prime \equiv t$ & IV.3.5 applied 10. and 11. \cr
13. & $\Gamma \ \psi_1 \ t \equiv t^\prime$ & $\neg\psi\sbst{t^\prime}{x}$ & IV.3.2 applied to 12. and 7.
\end{tabular}\\
$\varphi = (\psi \lor \chi)$: By induction hypothesis, if $\Gamma^+ \, (\psi \lor \chi) \subset L^S$ and if $\Gamma^+ \, \psi\sbst{t}{x}$ (or $\Gamma^+ \, \chi\sbst{t}{x}$) is given, then $\Gamma^+ \, t \equiv t^\prime \, \psi\sbst{t^\prime}{x}$ (or $\Gamma^+ \, t \equiv t^\prime \, \chi\sbst{t^\prime}{x}$, respectively) is derivable in $\mathfrak{S}^\circ$. Given $\Gamma \, \left(\psi\sbst{t}{x} \lor \chi\sbst{t}{x}\right)$, we provide the derivation\\
\begin{tabular}{llll}
1. & $\Gamma$ & $\left(\psi\sbst{t}{x} \lor \chi\sbst{t}{x}\right)$ & premise \cr
2. & $\psi\sbst{t}{x}$ & $\psi\sbst{t}{x}$ & $\assm$ \cr
3. & $\chi\sbst{t}{x}$ & $\chi\sbst{t}{x}$ & $\assm$ \cr
4. & $\psi\sbst{t}{x} \ t \equiv t^\prime$ & $\psi\sbst{t^\prime}{x}$ & by the above discussion \cr
5. & $\chi\sbst{t}{x} \ t \equiv t^\prime$ & $\chi\sbst{t^\prime}{x}$ & by the above discussion \cr
6. & $\psi\sbst{t}{x} \ t \equiv t^\prime$ & $\left(\psi\sbst{t^\prime}{x} \lor \chi\sbst{t^\prime}{x}\right)$ & $\ors$ applied to 4. \cr
7. & $\chi\sbst{t}{x} \ t \equiv t^\prime$ & $\left(\psi\sbst{t^\prime}{x} \lor \chi\sbst{t^\prime}{x}\right)$ & $\ors$ applied to 5. \cr
8. & $\Gamma \ t \equiv t^\prime \ \psi\sbst{t}{x}$ & $\left(\psi\sbst{t^\prime}{x} \lor \chi\sbst{t^\prime}{x}\right)$ & $\ant$ applied to 6. \cr
9. & $\Gamma \ t \equiv t^\prime \ \chi\sbst{t}{x}$ & $\left(\psi\sbst{t^\prime}{x} \lor \chi\sbst{t^\prime}{x}\right)$ & $\ant$ applied to 7. \cr
10. & $\Gamma \ t \equiv t^\prime$ & $\left(\psi\sbst{t}{x} \lor \chi\sbst{t}{x}\right)$ & $\ant$ applied to 1. \cr
11. & $\Gamma \ t \equiv t^\prime \ \left(\psi\sbst{t}{x} \lor \chi\sbst{t}{x}\right)$ & $\left(\psi\sbst{t^\prime}{x} \lor \chi\sbst{t^\prime}{x}\right)$ & $\ora$ applied to 8. and 9. \cr
12. & $\Gamma \ t \equiv t^\prime$ & $\left(\psi\sbst{t^\prime}{x} \lor \chi\sbst{t^\prime}{x}\right)$ & IV.3.2 applied to 10. and 11.
\end{tabular}\\
$\varphi = \exists u \psi$ and $x \in \free{\varphi}$: Based on logical equivalence, we may assume w.~l.~o.~g.\ that $\varphi\sbst{t}{x} = \exists v \psi\sbst{v \, t}{ux}$ and $\varphi\sbst{t^\prime}{x} = \exists v \psi\sbst{vt^\prime}{ux}$, where $v = u$ if $t = t^\prime = x$ or $u \not\in \var{t} \cup \var{t^\prime}$, otherwise $v$ is the first variable in the list $v_0, v_1, v_2, \ldots$ not occurring in $\psi, t, t^\prime$ (cf. III.8.2(e)). Either way $v$ does not occur in $t, t^\prime$.\\
\ \\
By induction hypothesis, if $\Gamma^+ \, \psi \subset L^S$ and if $\Gamma^+ \, \psi\sbst{v \, t}{ux}$ is given, then $\Gamma^+ \, t \equiv t^\prime \, \psi\sbst{vt^\prime}{ux}$ is derivable in $\mathfrak{S}^\circ$. Given $\Gamma \, \exists v \psi\sbst{v \, t}{ux}$, we provide the derivation\\
\begin{tabular}{llll}
1. & $\Gamma$ & $\exists v \psi\sbst{v \, t}{ux}$ & premise \cr
2. & $\psi\sbst{v \, t}{ux}$ & $\psi\sbst{v \, t}{ux}$ & $\assm$ \cr
3. & $\psi\sbst{v \, t}{ux} \ t \equiv t^\prime$ & $\psi\sbst{vt^\prime}{ux}$ & by the above discussion \cr
4. & $\psi\sbst{v \, t}{ux} \ t \equiv t^\prime$ & $\exists v \psi\sbst{vt^\prime}{ux}$ & IV.5.1(a) applied to 3. \cr
5. & $\exists v \psi\sbst{v \, t}{ux} \ t \equiv t^\prime$ & $\exists v \psi\sbst{vt^\prime}{ux}$ & IV.5.1(b) applied to 4. \cr
6. & $\Gamma \ t \equiv t^\prime$ & $\exists v \psi\sbst{v \, t}{ux}$ & $\ant$ applied to 1. \cr
7. & $\Gamma \ t \equiv t^\prime \ \exists v \psi\sbst{v \, t}{ux}$ & $\exists v \psi\sbst{vt^\prime}{ux}$ & $\ant$ applied to 5. \cr
8. & $\Gamma \ t \equiv t^\prime$ & $\exists v \psi\sbst{vt^\prime}{ux}$ & IV.3.2 applied to 6. and 7.
\end{tabular}\\
$\varphi = \exists u \psi$ and $x \not\in \free{\varphi}$: We have $(\exists u \psi)\sbst{t}{x} = (\exists u \psi)\sbst{t^\prime}{x} = \exists u \psi$. Given $\Gamma \, \exists u \psi$, we provide the derivation\\
\begin{tabular}{llll}
1. & $\Gamma$ & $\exists u \psi$ & premise \cr
2. & $\Gamma \ t \equiv t^\prime$ & $\exists u \psi$ & $\ant$ applied to 1.
\end{tabular}\\
\ \\
\ \\
Now ($\ast$) remains to be shown. We do this by induction on $t_0$ below:\\
$t_0 = c$: Then $c\sbst{t}{x} = c\sbst{t^\prime}{x} = c$. We provide the derivation\\
\begin{tabular}{lllll}
1. & $\forall x \, x \equiv x$ & \ & $\forall x \, x \equiv x$ & $\assm$ \cr
2. & $\forall x \, x \equiv x$ & \ & $c \equiv c$ & IV.5.5(a1) applied to 1. \cr
3. & $\forall x \, x \equiv x$ & $t \equiv t^\prime$ & $c \equiv c$ & $\ant$ applied to 2.
\end{tabular}\\
\ \\
$t_0 = y \neq x$: Similar.\\
\ \\
$t_0 = x$: Then $x\sbst{t}{x} = t$ and $x\sbst{t^\prime}{x} = t^\prime$. We provide the derivation\\
\begin{tabular}{llll}
1. & $t \equiv t^\prime$ & $t \equiv t^\prime$ & $\assm$
\end{tabular}\\
\ \\
$t_0 = ft_1 \ldots t_n$: By induction hypothesis, there are $\Gamma_1, \ldots, \Gamma_n \subset \Psi_\equiv$ such that $\Gamma_1 \, t \equiv t^\prime \, t_1\sbst{t}{x} \equiv t_1\sbst{t^\prime}{x}, \ldots, \Gamma_n \, t \equiv t^\prime \, t_n\sbst{t}{x} \equiv t_n\sbst{t^\prime}{x}$ are derivable in $\mathfrak{S}^\circ$. Take $\Gamma \colonequals \Gamma_1 \cup \ldots \cup \Gamma_n$ then, using $\ant$, $\Gamma \, t \equiv t^\prime \, t_1\sbst{t}{x} \equiv t_1\sbst{t^\prime}{x}, \ldots, \Gamma \, t \equiv t^\prime \, t_n\sbst{t}{x} \equiv t_n\sbst{t^\prime}{x}$ are also derivable in $\mathfrak{S}^\circ$. We denote
\[
\begin{array}{lll}
\psi_4 & \colonequals & \forall x_1 \ldots \forall x_n \forall y_1 \ldots \forall y_n \left(\bigwedge\limits^n_{i = 1} x_i \equiv y_i \rightarrow fx_1 \ldots x_n \equiv fy_1 \ldots y_n \right), \cr
\chi_6 & \colonequals & \bigwedge\limits^n_{i = 1} t_i\sbst{t}{x} \equiv t_i\sbst{t^\prime}{x}, \cr
\chi_7 & \colonequals & ft_1\sbst{t}{x} \ldots t_n\sbst{t}{x} \equiv ft_1\sbst{t^\prime}{x} \ldots t_n\sbst{t^\prime}{x}
\end{array}
\]
and provide the derivation\\
\begin{tabular}{llll}
1. & $\Gamma \ \phantom{\psi_4} \ t \equiv t^\prime$ & $t_1\sbst{t}{x} \equiv t_1\sbst{t^\prime}{x}$ & by the above discussion \cr
\multicolumn{4}{c}{$\vdots$} \cr
$n$. & $\Gamma \ \phantom{\psi_4} \ t \equiv t^\prime$ & $t_n\sbst{t}{x} \equiv t_n\sbst{t^\prime}{x}$ & by the above discussion \cr
$(n + 1)$. & $\Gamma \ \phantom{\psi_4} \ t \equiv t^\prime$ & $\bigwedge\limits^n_{i = 1} t_i\sbst{t}{x} \equiv t_i\sbst{t^\prime}{x}$ & successively apply IV.3.6(b) to 1. through $n$. \cr
$(n + 2)$. & $\psi_4$ & $\psi_4$ & $\assm$ \cr
$(n + 3)$. & $\psi_4$ & $(\chi_6 \rightarrow \chi_7)$ & IV.5.5(a1) applied to $(n + 2)$. \cr
$(n + 4)$. & $\Gamma \ \psi_4 \ t \equiv t^\prime$ & $(\chi_6 \rightarrow \chi_7)$ & $\ant$ applied to $(n + 3)$. \cr
$(n + 5)$. & $\Gamma \ \psi_4 \ t \equiv t^\prime$ & $\bigwedge\limits^n_{i = 1} t_i\sbst{t}{x} \equiv t_i\sbst{t^\prime}{x}$ & $\ant$ applied to $(n + 1)$. \cr
$(n + 6)$. & $\Gamma \ \psi_4 \ t \equiv t^\prime$ & $ft_1\sbst{t}{x} \ldots t_n\sbst{t}{x} \equiv ft_1\sbst{t^\prime}{x} \ldots t_n\sbst{t^\prime}{x}$ & IV.3.5 applied to $(n + 4)$. and $(n + 5)$.
\end{tabular}\\
\ \\
\ \\
\textit{Concluding Note.} The proof system used in this text, namely the sequent calculus, consists solely of inference rules (i.e. the ten rules of $\mathfrak{S}_S$) and no \emph{logical axioms}.\footnote{Logical axioms are valid sentences in $L^S$ that play a role similar to that played by sequent rules of $\mathfrak{S}_S$.}\\
\ \\
Here we see that $\Psi_\equiv$ compensates the absence of the rules $\eq$ and $\sub$ from $\mathfrak{S}^\circ$. In this way, $\Psi_\equiv$ serves as the set of logical axioms for the proof system that consists of $\mathfrak{S}^\circ$ and $\Psi_\equiv$.\\
\ \\
In many other textbooks (such as \cite{Christos_Papadimitriou}), the proof systems are at the extreme: There are other logical axioms (for disjunctions and for quantifiers, to mention a few) besides those for equality, and the only inference rule is \emph{modus ponens} (cf. IV.3.5).
%%
\item Similar to the discussion in (2), if we regard $E t_1 t_2$ as rewriting $t_1 \equiv t_2$, then we obtain
\begin{center}
for $\Phi \cup \{ \varphi \} \subset L^S$: $\Phi \cup \Psi_\equiv \vdash^\circ \varphi$ \ iff \ $\Phi^\ast \cup \Psi_E \vdash^\circ \varphi^\ast$.
\end{center}
To summarize, the following are equivalent for $\Phi \cup \{ \varphi \} \subset L^S$:
\begin{enumerate}[(a)]
\item $\Phi \vdash \varphi$
%%%
\item $\Phi^\ast \vdash^\ast \varphi^\ast$
%%%
\item $\Phi \cup \Psi_\equiv \vdash^\circ \varphi$
%%%
\item $\Phi^\ast \cup \Psi_E \vdash^\circ \varphi^\ast$
%%%
\item $\Phi^\ast \cup \Psi_E \vdash \varphi^\ast$.
\end{enumerate}
This is an enhanced version of the statement given in part (b) of Exercise 6.11.
%%
\item Let $A$ be a given universe. It is clear that for every $S \cup \{ E \}$-interpretation $\mathfrak{I} = (\mathfrak{A}, \beta)$, $\mathfrak{I} \models \Psi_E$ iff $E^A$ is an equivalence relation over $A$; $\Psi_E$ is actually the set of \emph{axioms of equivalence relations} using the symbol $E$. Also note that the equality relation $=^A$ is itself an equivalence relation.\\
\ \\
So far, the equality symbol $\equiv$ has always been interpreted as $=^A$. However, many of the results we have obtained also hold \emph{when $\equiv$ is interpreted as any equivalence relation other than $=^A$}.\\
\ \\
In Henkin's Theorem, for instance, if $\Phi \subset L^S$ is consistent and if it is negation complete and contains witnesses, then the $S$-interpretation $(\mathfrak{A}, \beta)$ is a model of $\Phi$, where $A = T^S$, $c^A \colonequals c$,
\begin{center}
\begin{tabular}{lll}
$f^A (t_1, \ldots, t_n)$ & $\colonequals$ & $ft_1 \ldots t_n$, \cr
$t_1 \equiv^A t_2$       & :iff & $\Phi \vdash t_1 \equiv t_2$, \cr
$R^A (t_1, \ldots, t_n)$ & :iff & $\Phi \vdash Rt_1 \ldots t_n$,
\end{tabular}
\end{center}
and $\beta(v_n) \colonequals v_n$ for $n \in \nat$. Clearly $\equiv^A$ is an equivalence relation.\\
\ \\
As another example, we show the Adequacy Theorem holds in this situation: For $\Phi \cup \{ \varphi \} \subset L^S$,
\begin{center}
\begin{tabular}{ll}
\    & $\Phi \models \varphi$ when $\equiv$ is interpreted as any equivalence relation \cr
iff  & $\Phi^\ast \cup \Psi_E \models \varphi^\ast$ \cr
iff  & $\Phi^\ast \cup \Psi_E \vdash \varphi^\ast$ (by the usual Adequacy Theorem) \cr
iff  & $\Phi \vdash \varphi$ (by part (b) of Exercise 6.11).
\end{tabular}
\end{center}
And since $\Phi \vdash \varphi$ iff $\Phi \cup \Psi_\equiv \vdash^\circ \varphi$, $\Psi_\equiv$ is indeed the set of \emph{axioms of equivalence relations} using the symbol $\equiv$ in this situation.
\end{enumerate}
\end{enumerate}
%End of Section XI.6--------------------------------------------------------------------------
\
\\
\\
%Section XI.7---------------------------------------------------------------------
{\large \S7. Logic Programming}
\begin{enumerate}[1.]
\item \textbf{An Argument on 7.2.} The identity substitutor $\iota$ intuitively says that ``all variables remain unchanged''. So, part (a) is trivial.\\
\ \\
In part (b), both statements for terms and for quantifier-free formulas can be shown by induction. The case of terms is shown below:\\
$t = x$: By definition, $x (\sigma\tau) = (x \sigma)\tau$.\\
\ \\
$t = c$: $c (\sigma\tau) = c = c \tau = (c \sigma)\tau$.\\
\ \\
$t = ft_1 \ldots t_n$:\\
\begin{tabular}{lll}
\ & $(ft_1 \ldots t_n)(\sigma\tau)$ & \ \cr
= & $f(t_1(\sigma\tau)) \ldots (t_n(\sigma\tau))$ & \ \cr
= & $f((t_1\sigma)\tau) \ldots ((t_n\sigma)\tau)$ & (by induction hypothesis) \cr
= & $(f(t_1\sigma) \ldots (t_n\sigma))\tau$ & \ \cr
= & $((ft_1 \ldots t_n)\sigma)\tau$. & \ 
\end{tabular}\\
\ \\
And then the case of quantifier-free formulas:\\
$\varphi = t_1 \equiv t_2$:\\
\begin{tabular}{lll}
\ & $(t_1 \equiv t_2)(\sigma\tau)$ & \ \cr
= & $t_1(\sigma\tau) \equiv t_2(\sigma\tau)$ & \ \cr
= & $(t_1\sigma)\tau \equiv (t_2\sigma)\tau$ & (the result just shown) \cr
= & $(t_1\sigma \equiv t_2\sigma)\tau$ & \ \cr
= & $((t_1 \equiv t_2)\sigma)\tau$. & \ 
\end{tabular}\\
\ \\
$\varphi = Rt_1 \ldots t_n$:\\
\begin{tabular}{lll}
\ & $(Rt_1 \ldots t_n)(\sigma\tau)$ & \ \cr
= & $Rt_1(\sigma\tau) \ldots t_n(\sigma\tau)$ & \ \cr
= & $R(t_1\sigma)\tau \ldots (t_n\sigma)\tau$ & (the result just shown) \cr
= & $(Rt_1\sigma \ldots t_n\sigma)\tau$ & \ \cr
= & $((Rt_1 \ldots t_n)\sigma)\tau$. & \ 
\end{tabular}\\
\ \\
$\varphi = \neg\psi$:\\
\begin{tabular}{lll}
\ & $(\neg\psi)(\sigma\tau)$ & \ \cr
= & $\neg(\psi(\sigma\tau))$ & \ \cr
= & $\neg((\psi\sigma)\tau)$ & (by induction hypothesis) \cr
= & $((\neg\psi)\sigma)\tau$. & \ 
\end{tabular}\\
\ \\
$\varphi = \psi \lor \chi$:\\
\begin{tabular}{lll}
\ & $(\psi \lor \chi)(\sigma\tau)$ & \ \cr
= & $\psi(\sigma\tau) \lor \chi(\sigma\tau)$ & \ \cr
= & $(\psi\sigma)\tau \lor (\chi\sigma)\tau$ & (by induction hypothesis) \cr
= & $(\psi\sigma \lor \chi\sigma)\tau$ & \ \cr
= & $((\psi \lor \chi)\sigma)\tau$. & \ 
\end{tabular}\\
\ \\
Note that $\varphi (\sigma\tau) = (\varphi \sigma) \tau$ does not hold for formulas $\varphi$ \emph{with} quantifiers in general: Set $\varphi \colonequals \exists v_1 \, v_0 \equiv v_1$, $\sigma \colonequals \sbst{v_1}{v_0}$ and $\tau \colonequals \sbst{v_0}{v_1}$. Then $\varphi (\sigma\tau) = \exists v_1 \, v_0 \equiv v_1 \neq \exists v_2 \, v_0 \equiv v_2 = (\varphi \sigma) \tau$.\\
\ \\
Finally, part (c) can be argued: For $x \in V$,\\
\begin{tabular}{lll}
\ & $((\rho\sigma) \tau)(x)$ & \ \cr
= & $x ((\rho\sigma) \tau)$ & \ \cr
= & $(x (\rho\sigma)) \tau$ & (by definition) \cr
= & $((x \rho) \sigma) \tau$ & (by definition) \cr
= & $(x \rho) (\sigma\tau)$ & (by (b)) \cr
= & $x (\rho (\sigma\tau))$ & (by definition) \cr
= & $(\rho (\sigma\tau))(x)$. & \ 
\end{tabular}
%
\item \textbf{Note to Lemma on the Unifier 7.6.} In the proof, to see (1) is true, let $\psi_1, \psi_2$ be the two literals in $K\sigma_i$ considered at (UA4) in the iteration corresponding to the value $i$. Also, let $\S_1$ and $\S_2$ be the two letters thereof.\\
\ \\
Since $\sigma_{i + 1} = \sigma_i\sbst{t}{x}$, we may assume, according to (UA6), that $\S_1 = x$ and $t$ is the term that starts with $\S_2$ in $\psi_2$.\\
\ \\
Because $K\sigma_i\tau_i = (K\sigma_i)\tau_i$ contains a single element, we have $\psi_1 \tau_i = \psi_2 \tau_i$. In particular, $x\tau_i = t\tau_i$.\\
\ \\
On the other hand, as a corollary to this lemma, we have:
\begin{quote}
\emph{If a clause $K$ has a unifier, then it has a general unifier.}
\end{quote}
Assume $K$ has a unifier, then it is unifiable. Running the algorithm given in this lemma yields \emph{the} general unifier of $K$.
%
\item \textbf{Note to Definition 7.8.} Note that it is not necessary that
\begin{center}
$M_1 \cup L_1 = \emptyset$ \ \ \ and \ \ \ $M_2 \cup L_2 = \emptyset$.
\end{center}
So a U-resolvent $K$ of $K_1$ and $K_2$ may contain both literals $\psi$ and $\neg\psi$ for some atomic $\psi$ (cf. 5.5 and the footnote thereof).\\
\ \\
Also, a resolvent is automatically a U-resolvent.
%
\item \textbf{Note to the Discussion before Remark 7.9.} There is a typo in the fourth line from the bottom of page 231: ``variable free clauses'' should be replaced by ``variable-free clauses''.
%
\item \textbf{Note to Compatibility Lemma 7.10.} A ground instance of a clause $K$ is the clause counterpart of a ground instance of the disjunction of all literals in $K$. In other words, a ground instance of $K$ is the ground clause $K\sigma$ with some substitutor $\sigma$.\\
\ \\
In the part of proof for (b), by setting $\sigma_1 \colonequals \xi\eta\sigma$ and $\sigma_2 \colonequals \eta\sigma$, $K_1\sigma_1$ and $K_2\sigma_2$ may not be ground clauses. In fact, $\free{K_1\sigma_1 \cup K_2\sigma_2} = \free{L_1\sigma_1 \cup L_2\sigma_2} = \free{\varphi_0\sigma}$. Hence
\begin{center}
$K_1\sigma_1$ and $K_2\sigma_2$ are ground clauses \ \ \ iff \ \ \ $\varphi_0\sigma$ is a sentence.
\end{center}
%
\item \textbf{Note to Lemma 7.12.} Let us confirm that (b) follows from (a):\\
\begin{tabular}{lll}
\ & $\resi{\infty}{\gi{\mathfrak{K}}}$ & \ \cr
= & $\bigcup_{i \in \nat} \resi{i}{\gi{\mathfrak{K}}}$ & \ \cr
= & $\bigcup_{i \in \nat} \gi{\uresi{i}{\mathfrak{K}}}$ & (by (a)) \cr
= & $\bigcup_{i \in \nat} \bigcup_{K \in \uresi{i}{\mathfrak{K}}} \gi{K}$ & \ \cr
= & $\bigcup_{K \in \uresi{\infty}{\mathfrak{K}}} \gi{K}$ & \ \cr
= & $\gi{\uresi{\infty}{\mathfrak{K}}}$.
\end{tabular}
%
\item \textbf{Note to the Statement above Lemma 7.13.} Let $M$ be a set of clauses and $M^G$ be a ground instance of $M$. Then
\begin{center}
$M = \emptyset$ \ \ \ iff \ \ \ $M^G = \emptyset$.
\end{center}
So $\emptyset \in \gi{\uresi{\infty}{\mathfrak{K}}}$ iff $\emptyset \in \uresi{\infty}{\mathfrak{K}}$.
%
\item \textbf{Note to Lemma 7.13.} To appropriately use the resolution method suitable for propositional logic in first-order logic, our attempt to this may be to ``convert'' a set $\mathfrak{K}$ of clauses of first-order literals (i.e. literals that are first-order formulas) into the set $\resi{\infty}{\gi{\mathfrak{K}}}$ of clauses of propositional literals\footnote{As is mentioned in the discussion before 6.5, here we identify a ground clause with its image under $\pi$.} (i.e. literals that are propositional formulas) before applying this method.\\
\ \\
This lemma tells us that for this purpose it suffices to consider $\uresi{\infty}{\mathfrak{K}}$.
%
\item \textbf{Note to the Discussion after Main Lemma on the U-Resolution 7.13.} To see ($\ast$) holds, let us first note that for universal sentences $\varphi$ of the form
\[
\forall x_1 \ldots \forall x_m ((\varphi_{00} \lor \ldots \lor \varphi_{0l_0}) \land \ldots \land (\varphi_{s0} \lor \ldots \lor \varphi_{sl_s})),
\]
it can easily be verified that
\[
\gi{\mathfrak{K}(\varphi)} = \mathfrak{K}(\gi{\varphi}).
\]
Then ($\ast$) holds for sets $\Phi$ of sentences such as $\varphi$:\\
\begin{tabular}{lll}
\ & $\gi{\mathfrak{K}(\Phi)}$ & \ \cr
= & $\bigcup_{K \in \mathfrak{K}(\Phi)} \gi{K}$ & \ \cr
= & $\bigcup_{\varphi \in \Phi} \bigcup_{K \in \mathfrak{K}(\varphi)} \gi{K}$ & \ \cr
= & $\bigcup_{\varphi \in \Phi} \gi{\mathfrak{K}(\varphi)}$ & \ \cr
= & $\bigcup_{\varphi \in \Phi} \mathfrak{K}(\gi{\varphi})$ & (by the above discussion) \cr
= & $\bigcup_{\varphi \in \Phi} \bigcup_{\psi \in \gi{\varphi}} \mathfrak{K}(\psi)$ & ($\gi{\varphi}$ is also a set of sentences of the form \cr
\ & \ & \phantom{(}just mentioned) \cr
= & $\bigcup_{\chi \in \gi{\Phi}} \mathfrak{K}(\chi)$ & ($\gi{\Phi}$ is also a set of sentences of the form \cr
\ & \ & \phantom{(}just mentioned) \cr
= & $\mathfrak{K} (\gi{\Phi})$. & \ 
\end{tabular}
%
\item \textbf{Note to the Proof of Main Lemma on the UH-Resolution 7.16.} We provide a complete proof of this lemma.\\
\ \\
First, let $M$ be a clause and $M^G$ be a ground instance of $M$. Observe that
\begin{center}
$M$ is positive (negative) \ \ \ iff \ \ \ $M^G$ is positive (resp. negative).
\end{center}
By the Compatibility Lemma 7.10, we have: For any positive clause $K$ and any negative clause $N$:
\begin{enumerate}[(a)]
\item Every negative clause that is a resolvent of a ground instance of $K$ and a ground instance of $N$ is a ground instance of a negative clause that is a U-resolvent of $K$ and $N$.
%%
\item Every ground instance of a negative clause that is a U-resolvent of $K$ and $N$ is a negative clause that is a resolvent of a ground instance of $K$ and a ground instance of $N$.
\end{enumerate}
Next, we are ready to show\\
\ \\
($+$) \ \begin{minipage}{10cm}
for every set $\mathfrak{K}$ of clauses and for every $i \in \nat$:\\$\hresi{i}{\gi{\mathfrak{K}}} = \gi{\uhresi{i}{\mathfrak{K}}}$
\end{minipage}\\
\ \\
by induction on $i$: For $i = 0$, we have
\[
\hresi{0}{\gi{\mathfrak{K}}} = \gi{\mathfrak{K}} = \gi{\uhresi{0}{\mathfrak{K}}}.
\]
In the inductive step,\\
\begin{tabular}{lll}
\ & $\hresi{i + 1}{\gi{\mathfrak{K}}}$ & \ \cr
= & $\hres{\hresi{i}{\gi{\mathfrak{K}}}}$ & \ \cr
= & $\hres{\gi{\uhresi{i}{\mathfrak{K}}}}$ & (by induction hypothesis) \cr
= & $\gi{\uhres{\uhresi{i}{\mathfrak{K}}}}$ & (by the above discussion) \cr
= & $\gi{\uhresi{i + 1}{\mathfrak{K}}}$. & \ 
\end{tabular}\\
\ \\
\ \\
Then, as in 7.12(b), we obtain: for every set $\mathfrak{K}$ of clauses,\\
\begin{tabular}{lll}
\ & $\hresi{\infty}{\gi{\mathfrak{K}}}$ & \ \cr
= & $\bigcup_{i \in \nat} \hresi{i}{\gi{\mathfrak{K}}}$ & \ \cr
= & $\bigcup_{i \in \nat} \gi{\uhresi{i}{\mathfrak{K}}}$ & (by ($+$)) \cr
= & $\bigcup_{i \in \nat} \bigcup_{K \in \uhresi{i}{\mathfrak{K}}} \gi{K}$ & \ \cr
= & $\bigcup_{K \in \uhresi{\infty}{\mathfrak{K}}} \gi{K}$ & \ \cr
= & $\gi{\uhresi{\infty}{\mathfrak{K}}}$. & \ 
\end{tabular}\\
\ \\
In particular, for any set $\mathfrak{P}$ of positive clauses and any negative clause $N$,\\
\ \\
(1) \hfill $\hresi{\infty}{\gi{\mathfrak{P} \cup \{ N \}}} = \gi{\uhresi{\infty}{\mathfrak{P} \cup \{ N \}}}$. \hfill \phantom{(1)}\\
\ \\
Finally, as noted before 7.13, we have for every set $\mathfrak{K}$ of clauses: $\emptyset \in \gi{\uhresi{\infty}{\mathfrak{K}}}$ iff $\emptyset \in \uhresi{\infty}{\mathfrak{K}}$. And in particular, for any set $\mathfrak{P}$ of positive clauses and any negative clause $N$,\\
\ \\
(2) \hfill $\emptyset \in \gi{\uhresi{\infty}{\mathfrak{P} \cup \{ N \}}}$ \ \ iff \ \ $\emptyset \in \uhresi{\infty}{\mathfrak{P} \cup \{ N \}}$. \hfill \phantom{(2)}\\
\ \\
The lemma then immediately follows from (1) and (2).
%
\item \textbf{Note to Theorem on the UH-Resolution 7.17.} The statement
\begin{center}
\emph{$\emptyset$ is not UH-derivable from $\mathfrak{K}(\Phi)$ and $\mathfrak{K}(\varphi)$}
\end{center}
should be replaced by
\begin{center}
\emph{$\emptyset$ is not UH-derivable from $\mathfrak{K}(\Phi)$ and the negative clause $N \in \mathfrak{K}(\varphi)$,}
\end{center}
which is consistent with Definition 7.15.\\
\ \\
On the other hand, in the proof the equivalence of (4) and (5) actually follows from Definition 7.15 and the following statement: For sets $\mathfrak{P}$ of (first-order) clauses and for negative (first-order) clauses $N$ and $N^\prime$:
\begin{center}
\begin{minipage}{10cm}
$N^\prime \in \uhresi{i}{\mathfrak{P} \cup \{ N \}}$ \ iff \ there is a UH-derivation of $N^\prime$ of $\mathfrak{P}$ and $N$ of length $\leq i$
\end{minipage}
\end{center}
holds for $i \in \nat$. This is similar to that suggested in the proof of 5.11 and can be shown by induction on $i$.
%
\item \textbf{Note to the Proof of Theorem on Logic Programming 7.18.} In part (b), the two cases distinguished for showing\\
\ \\
($\ast$) \hfill $\Phi \vdash \psi_i \eta_1 \ldots \eta_k$ \hfill \phantom{($\ast$)}\\
\ \\
are actually the two:
\begin{itemize}
\item For $i \leq r$, $\neg\psi_i \eta_1 \in N_2$.
%%
\item There is \emph{exactly one} $i \leq r$ such that $\neg\psi_i \eta_1 \not\in N_2$.
\end{itemize}
In fact, if there is some $i \leq r$ such that $\neg\psi_i \eta_1 \not\in N_2$, then it is the only $i \leq r$ for which this happens, since there is at most one literal ``deleted'' from $N_1$ in the resolution step leading to $N_2$.\\
\ \\
In part (c), there is a typo in the fifth line from the bottom of page 240: ``Figure XI.11'' should be replaced by ``Figure XI.10''.
\end{enumerate}
%End of Section XI.7--------------------------------------------------------------
%End of Chapter XI----------------------------------------------------------------