%Author: Wei-Lin (Linisac) Wu
\documentclass[11pt, leqno]{report}


%Packages
\usepackage{amssymb}
\usepackage{enumerate}
\usepackage{graphicx}
\usepackage{stmaryrd}
\usepackage{colonequals}
\usepackage{amsthm}
\usepackage{array}
\usepackage{amsmath}
\usepackage{amscd}
\usepackage{paralist}
\usepackage{bm}
\usepackage{titlesec}
\usepackage[margin=3cm]{geometry}

%ifthen package
\usepackage{ifthen}


%New Commands for Abbreviations
%%Catalog <
%%%general
%%%definition
%%%enumeration, sequence, tuple, vector
%%%set operations
%%%function type, function operations
%%%special set(s)
%%%structure, symbol interpretation, structure operations, structure relations
%%%partial isomorphism
%%%elementary equivalence
%%%Ehrenfeucht-Fraisse game
%%%isomorphism types
%%%graph, graph functions, graph relations
%%%logic
%%%counting specifiers
%%%connectives, quantifiers, formula operations
%%%substitution
%%%logic relations
%%%logic operations
%%%finite notions
%%%nullary relations and predicates
%%%operations on formulas and structures
%%%global relations
%%%special classes of structures
%% >

%%general <
\newcommand{\header}[1]{{\rm\textbf{#1.}}}
\newcommand{\refitem}[1]{{\rm#1}}
%% >

%%definition <
\newcommand{\defas}{\colonequals}
%% >

%%enumeration, sequence, tuple, vector <
\newcommand{\etc}{\ldots} %et cetera
\newcommand{\length}{\mathrm{length}}
\newcommand{\emptyseq}{\emptyset} %empty sequence
\newcommand{\seq}[2]{(#1)_{#2}} %indexed sequence
\newcommand{\vect}[1]{\overline{#1}} %vector, \vect{a} = \overbar{a}
%% >

%%some common operators <
\newcommand{\mul}{\mathop{\cdot}} %multiplication
\newcommand{\abs}[1]{\left|#1\right|}
%% >

%%set operations <
\newcommand{\sete}[1]{{\{#1\}}} %set by enumeration. sete{a, b, c} = {a, b, c}
\newcommand{\setm}[2]{{\{#1 \mid #2\}}} %set by math description. setd{#1}{#2} = {#1 | #2}
\newcommand{\sett}[2]{{\{#1 \mid \mbox{#2}\}}} %set by text description. sett{#1}{#2} = {#1 | #2}, where #2 is a text
\newcommand{\cart}{\mathop{\times}} %cartesian product of two sets
\newcommand{\cartpwr}[2]{#1^{#2}} %(#2)nd cartesian power of set #1
\newcommand{\intsc}{\mathop{\cap}} %intersection
\newcommand{\union}{\mathop{\cup}} %union
\newcommand{\card}[1]{\|#1\|} %cardinality. card{A} = || A ||
\newcommand{\cmpl}[1]{#1^\mathit{c}} %complement of the set #1
%% >

%%function operations <
\newcommand{\dm}{\mathrm{do}} %domain (of a function)
\newcommand{\rg}{\mathrm{rg}} %range (of a function)
\newcommand{\emptymap}{\emptyset} %empty map
\newcommand{\inv}[1]{#1^{-1}} %inverse of function #1
%% >

%%special set(s) <
\newcommand{\nat}{\mathbb{N}} %set of natural numbers
\newcommand{\rand}{\mathrm{rand}} %random
\newcommand{\spec}{\mathrm{Spec}} %spectrum
%% >

%%structure, symbol interpretation, structure operations, structure relations <
\newcommand{\strct}[1]{\mathfrak{#1}} %structure
\newcommand{\intpr}[2]{#1^{#2}} %symbol #1 interpreted under #2
\newcommand{\sbstrct}[2]{#1^{#2}} %substructure of #2 induced by the subset #1
\newcommand{\isom}[1][]{\ifthenelse{\equal{#1}{}}{\cong}{\cong^{#1}}} %isomorphic in #1 number of variables
\newcommand{\product}{\cart} %product of two structures
\newcommand{\dsjuni}{\mathop{\dot{\cup}}} %disjoint union of two structures
\newcommand{\ordsum}{\mathop{\lhd}} %ordered sum of two ordered structures
\newcommand{\ultprd}[2]{#1^{#2}} %ultra product. \ultprd{A}{I} = A^I
\newcommand{\ball}[1][]{\ifthenelse{\equal{#1}{}}{{\mathit{S}}}{{\mathit{S}^{#1}}}}
\newcommand{\ballstrct}[1][]{\ifthenelse{\equal{#1}{}}{{\mathcal{S}}}{{\mathcal{S}^{#1}}}}
%% >

%%partial isomorphism <
\newcommand{\partisoms}{\mathrm{Part}} %set of partial isomorphisms between structures #1 and #2
%% >

%%elementary equivalence <
\newcommand{\equv}[1][]{\ifthenelse{\equal{#1}{}}{\equiv}{\equiv^{#1}}} %equivalence in (#1)-logic
%% >

%%Ehrenfeucht-Fraisse games <
\newcommand{\game}[2][]{\ifthenelse{\equal{#1}{}}{{\mathrm{G}_{#2}}}{{\mathrm{G}^{#1}_{#2}}}} %the game G^#1_#2
\newcommand{\winpos}[2][]{\ifthenelse{\equal{#1}{}}{\mathit{W}_{#2}}{\mathit{W}^{#1}_{#2}}}
\newcommand{\msogame}[1]{\operatorname{\MSO-\mathrm{G}}_{#1}} %the game in monadic second-order logic
%% >

%%isomorphism types <
\newcommand{\hint}[2]{\varphi^{#1}_{#2}} %(#1)-isomorphic type (Hintikka formula) over #2 in first-order logic
\newcommand{\ityp}[3][]{\ifthenelse{\equal{#1}{}}{{\psi^{#2}_{#3}}}{{^{#1}\psi^{#2}_{#3}}}} %(#2)-isomorphism type over #2 in other logics in #1 number of variables
%% >

%%graph, graph functions, graph relations <
\newcommand{\graph}{\mathrm{GRAPH}} %the set of finite graphs
\newcommand{\conn}{\mathrm{CONN}} %the set of connected graphs
\newcommand{\dist}[1][]{\ifthenelse{\equal{#1}{}}{\mathit{d}}{\mathit{d}_{#1}}} %distance between two points in a graph
\newcommand{\tree}{\mathrm{TREE}} %the set of finite trees
\newcommand{\gaifman}{\mathcal{G}} %the Gaifman graph of #1 (structure)
%% >

%%logics <
\newcommand{\fo}{\mathrm{FO}} %first-order logic
\newcommand{\foe}{\fo^\ast}
%\newcommand{\MSO}{\mathrm{MSO}} %monadic second-order logic
\newcommand{\logic}[2][]{\ifthenelse{\equal{#1}{}}{\mathrm{L}_{#2}}{\mathrm{L}^{#1}_{#2}}} %parameterized logic (like infinitary logics)
\newcommand{\logicvar}{\mathcal{L}} %(variable for ) logic
%% >

%%counting specifiers <
\newcommand{\atleast}[1]{\geq #1} %at least #1
\newcommand{\atmost}[1]{\leq #1} %at most #1
\newcommand{\exactly}[1]{= #1} %exactly #1
\newcommand{\cardatleast}[1]{\varphi_{\atleast{#1}}} %cardinality at least #1
\newcommand{\cardatmost}[1]{\varphi_{\atmost{#1}}} %cardinality at most #1
\newcommand{\cardexactly}[1]{\varphi_{\exactly{#1}}} %cardinality exactly #1
%% >

%%connectives, quantifiers, formula operations <
\newcommand{\lthen}{\rightarrow} %logicial then
\newcommand{\liff}{\leftrightarrow} %logical iff (between two formulas)
\newcommand{\blor}{\bigvee} %big disjunction
\newcommand{\bland}{\bigwedge} %big conjunction
\newcommand{\existatleast}[1]{\exists^{\atleast{#1}}} %exist at least #1
\newcommand{\existatmost}[1]{\exists^{\atmost{#1}}} %exist at most #1
\newcommand{\existexactly}[1]{\exists^{\exactly{#1}}} %exist exactly #1
\newcommand{\free}{\mathrm{free}} %set of free variables of #1
\newcommand{\qr}{\mathrm{qr}} %quantifier rank (of a formula)
\newcommand{\modclass}{\mathrm{Mod}} %model class (of a formula)
%% >

%%substitution <
\newcommand{\sbst}[2]{{\scriptstyle\frac{\displaystyle #1}{\displaystyle #2}}} %substitution of #2 with #1
%% >

%%logic relations <
\newcommand{\satis}{\models} %satisfaction relation
\newcommand{\consq}{\models} %consequence relation
%% >

%%logic operations <
\newcommand{\rel}[1]{#1^\mathit{r}} %#1 relationalized
\newcommand{\invrel}[1]{#1^{-\mathit{r}}} %#1 inverse-relationalized
%% >

%%finite notions <
\newcommand{\fin}{\mathrm{fin}} %finite
\newcommand{\consqfin}{\consq_\fin} %consequence relation restricted to finite structures
%% >

%%nullary relations and predicates <
\newcommand{\true}{\mathrm{TRUE}} %TRUE symbol
\newcommand{\false}{\mathrm{FALSE}} %FALSE symbol
\newcommand{\tr}{\mathrm{T}} %T symbol
\newcommand{\fls}{\mathrm{F}} %F symbol
%% >

%%operations on formulas and structures <
\newcommand{\tuplesby}[2]{{#1^{#2}(\_)}} %tuples defined by formula #1 and structure #2
\newcommand{\rltv}[2]{#1^{#2}} %formula #1 relativized to the set #2
%% >

%%global relations <
\newcommand{\transcls}{\mathrm{TC}} %transitive closure relation
%% >

%%special classes of structures <
\newcommand{\ordclass}{\mathcal{O}} %class of finite ordered structures (in a vocabulary)
\newcommand{\even}{\mathrm{EVEN}} %class of finite structures of even cardinality (in a vocabulary)
%% >


\newenvironment{quoteno}[1]{(#1)\hfill}{\hfill\phantom{(+)}}


%if resetting of equation counter is required, use the instruction below:
%\setcounter{equation}{0}


\theoremstyle{remark}
\newtheorem*{remark}{Remark}

\theoremstyle{note}
\newtheorem*{note}{Note}

\begin{document}
\noindent\textsc{\large First-Order Logic and Equality\bigskip\\}
I argue that the notion of equality cannot be axiomatized in first-order logic without the equality symbol below.
\medskip\\
Let $\tau$ be any vocabulary that contains a relation symbol (it may contain constant symbols $c$, function symbols $f$ or other relation symbols $R$). Let $\fo[\tau]$ and $\foe[\tau]$ denote the first-order language of $\tau$ with and without the equality symbol, respectively; the satisfaction/consequence relations $\satis$ for $\foe$ are defined in the standard way. Note that in $\foe[\tau]$, atomic formulas are of the form $Rt_1 \etc t_n$.
\medskip\\
Consider two $\tau$-structures $\strct{A}, \strct{B}$ such that
\begin{itemize}
%
\item $\strct{A}$ has domain $A = \sete{0, 1}$ and $\strct{B}$ has domain $B = \sete{0}$.
%
\item For constant symbols $c \in \tau$ (if any), $\intpr{c}{A} = 0$ and $\intpr{c}{B} = 0$.
%
\item For $n$-ary function symbols $f \in \tau$ (if any), $\intpr{f}{A} = \cartpwr{A}{n} \cart \sete{0}$ and $\intpr{f}{B} = \cartpwr{B}{n} \cart \sete{0}$.
%
\item For $n$-ary relation symbols $R \in \tau$, $\intpr{R}{A} = \cartpwr{A}{n}$ and $\intpr{R}{B} = \cartpwr{B}{n}$.
%
\end{itemize}
An assignment $\alpha$ in $\strct{A}$ (or $\beta$ in $\strct{B}$) is a mapping of variables $v_0, v_1, v_2, \etc$ to elements in $A$ (or $B$, respectively). Let $\alpha\sbst{a}{x}$ denote the assignment in $\strct{A}$ that maps the variable $x$ to $a \in A$ and is identical to $\alpha$ elsewhere (similarly for $\beta\sbst{b}{x}$).
\medskip\\
Note that if $t$ is a $\tau$-term, then $\intpr{t}{(\strct{A}, \alpha)} = 0$ or $\intpr{t}{(\strct{A}, \alpha)} = 1$ and $\intpr{t}{(\strct{B}, \beta)} = 0$; but both $(\strct{A}, \alpha)$ and $(\strct{B}, \beta)$ are models of the atomic formulas $Rt_1 \etc t_n$.
\bigskip\\
\textbf{Claim.} \emph{For every formula $\varphi$ in $\foe[\tau]$, every assignment $\alpha$ in $\strct{A}$ and every assignment $\beta$ in $\strct{B}$, $(\strct{A}, \alpha)$ satisfies $\varphi$ if and only if $(\strct{B}, \beta)$ satisfies $\varphi$.}
\begin{proof}
This is proved simultaneously for all assignments $\alpha$ in $\strct{A}$ and all assignments $\beta$ in $\strct{B}$ by induction on $\varphi$.
\medskip\\
For $\varphi$ atomic, $\varphi$ takes the form $Rt_1 \etc t_n$. This case follows by the previous observation.
\medskip\\
For $\varphi = \neg\psi$, this case follows by\\
\begin{tabular}[b]{lll}
\   & $(\strct{A}, \alpha) \satis \neg\psi$ & \ \cr
iff & not $(\strct{A}, \alpha) \satis \psi$ & \ \cr
iff & not $(\strct{B}, \beta) \satis \psi$ & (by induction hypothesis) \cr
iff & $(\strct{B}, \beta) \satis \neg\psi$. & \ \cr
\end{tabular}
\medskip\\
For $\varphi = \psi \lor \chi$, this case follows by\\
\begin{tabular}[b]{lll}
\   & $(\strct{A}, \alpha) \satis \psi \lor \chi$ & \ \cr
iff & $(\strct{A}, \alpha) \satis \psi$ or $(\strct{A}, \alpha) \satis \chi$ & \ \cr
iff & $(\strct{B}, \beta) \satis \psi$ or $(\strct{B}, \beta) \satis \chi$ & (by induction hypothesis) \cr
iff & $(\strct{B}, \beta) \satis \psi \lor \chi$. & \ \cr
\end{tabular}
\medskip\\
For $\varphi = \exists x \psi$, this case follows by\\
\begin{tabular}[b]{lll}
\   & $(\strct{A}, \alpha) \satis \exists x \psi$ & \ \cr
iff & $(\strct{A}, \alpha\sbst{0}{x}) \satis \psi$ or $(\strct{A}, \alpha\sbst{1}{x}) \satis \psi$ & (since $A = \sete{0, 1}$) \cr
iff & $(\strct{B}, \beta\sbst{0}{x}) \satis \psi$ or $(\strct{B}, \beta\sbst{0}{x}) \satis \psi$ & (by induction hypothesis) \cr
iff & $(\strct{B}, \beta\sbst{0}{x}) \satis \psi$ & \ \cr
iff & $(\strct{B}, \beta) \satis \exists x \psi$ & (since $B = \sete{0}$). \cr
\end{tabular}
\end{proof}
\noindent From this claim it follows that \emph{for every set of formulas $\Phi \subseteq \foe[\tau]$, every assignment $\alpha$ in $\strct{A}$ and every assignment $\beta$ in $\strct{B}$, $(\strct{A}, \alpha)$ is a model of $\Phi$ if and only if $(\strct{B}, \beta)$ is a model of $\Phi$} (because they satisfy the same formulas in $\foe[\tau]$).
\medskip\\
However, $(\strct{A}, (\alpha\sbst{0}{x})\sbst{1}{y})$ does not satisfy the $\fo[\tau]$-formula $x = y$ while $(\strct{B}, \beta)$ satisfies it, for every $\alpha$ in $\strct{A}$ and every $\beta$ in $\strct{B}$; moreover, $\strct{A}$ is a model of $\exists x \exists y \neg x = y$ while $\strct{B}$ is not.
\medskip\\
Therefore, the notion of equality cannot be axiomatized in first-order logic without the equality symbol.
\end{document}