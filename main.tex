%Author: Wei-Lin (Linisac) Wu
\documentclass[11pt, leqno]{report}


%Packages
\usepackage{amssymb}
\usepackage{enumerate}
\usepackage{graphicx}
\usepackage{stmaryrd}
\usepackage{colonequals}
\usepackage{amsthm}
\usepackage{array}
\usepackage{amsmath}
\usepackage{amscd}
\usepackage{paralist}
\usepackage{bm}
\usepackage{titlesec}
\usepackage[margin=2cm]{geometry}

%New Commands for Abbreviations
%%General
\newcommand{\mathmode}[1]{\begin{math}#1\end{math}}
\newcommand{\defas}{\colonequals}
\newcommand{\setenum}[1]{\{#1\}} %the set of which the elements are enumerated by #1
\newcommand{\setsum}{\cup} %`A \setsum B' stands for the union of sets A and B
\newcommand{\union}{\cup}
\newcommand{\setprod}{\cap} %`A \setprod B' stands for the intersection of sets A and B
\newcommand{\intsec}{\cap} %intersection
\newcommand{\bsetsum}{\bigcup} %big operator of union
\newcommand{\bunion}{\bigcup}
\newcommand{\bsetprod}{\bigcap} %big operator of intersection
\newcommand{\bintsec}{\bigcap}
\newcommand{\nat}{\mathbb{N}} %the set of natural numbers
\newcommand{\zah}{\mathbb{Z}} %the set of integers
\newcommand{\rat}{\mathbb{Q}} %the set of rational numbers
\newcommand{\real}{\mathbb{R}} %the set of real numbers
\newcommand{\cplx}{\mathbb{C}} %the set of complex numbers
\newcommand{\powerset}[1]{{\mathcal{P}\left(#1\right)}} %the power set of #1
\newcommand{\absval}[1]{{\left| #1 \right|}} %the absolute value of #1
\newcommand{\card}[1]{\absval{#1}} %the cardinality of the set #1
\newcommand{\limply}{\mathbin{\rightarrow}} %logical connective imply
\newcommand{\liff}{\mathbin{\leftrightarrow}} %logical connective if and only if
\newcommand{\blor}{\bigvee} %big operator or
\newcommand{\bland}{\bigwedge} %big operator and
\newcommand{\Iff}{\mbox{iff}} %!!
\newcommand{\scripttext}[1]{{\mbox{\scriptsize#1}}} %text of script size in math mode
\newcommand{\mbf}[1]{{\mbox{\boldmath\begin{math}#1\end{math}}}} %bold face in math mode
\newcommand{\mbfs}[1]{{\mbox{\scriptsize\boldmath\begin{math}\mathrm{#1}\end{math}}}} %bold face of script size in math mode
\newcommand{\mbff}[1]{{\mbox{\footnotesize\boldmath\begin{math}\mathrm{#1}\end{math}}}} %bold face of footnote size in math mode
\newcommand{\mbft}[1]{{\mbox{\tiny\boldmath\begin{math}\mathrm{#1}\end{math}}}} %bold face of tiny size in math mode
\newcommand{\formal}[1]{\bm{#1}} %formal version of the symbol #1
\newcommand{\reftitle}[1]{{\rm #1}} %reference title
\newcommand{\parenadj}[1]{\left(#1\right)} %adjustable parentheses
\newcommand{\brackadj}[1]{\left[#1\right]} %adjustable (square) brackets
\newcommand{\braceadj}[1]{\left\lbrace #1\right\rbrace} %adjustable (curly) braces
\newcommand{\angleadj}[1]{\left\langle #1 \right\rangle} %adjustable angular brackets
\newcommand{\opintv}[2]{\parenadj{#1, #2}} %the open interval (#1, #2)
\newcommand{\clintv}[2]{\brackadj{#1, #2}} %the closed interval [#1, #2]
\newcommand{\olintv}[2]{\left(#1, #2\right]} %the left-open right-closed interval (#1, #2]
\newcommand{\cpintv}[2]{\left[#1, #2\right)} %the left-closed right-open interval [#1, #2)
\newcommand{\pair}[2]{(#1, #2)} %the pair of #1 and #2
\newcommand{\tuple}[1]{(#1)} %tuple of #1; often used with \seq
\newcommand{\pairadj}[2]{\parenadj{#1, #2}} %the pair of #1 and #2 with adjustable parentheses
\newcommand{\tupleadj}[1]{\parenadj{#1}} %tuple of #1 with adjustable parentheses; often used with \seq
\newcommand{\seq}[3][0]{{#2}_{#1}, \ldots, {#2}_{#3}} %sequence; \seq[1]{a}{n} means the sequence a_1, \ldots, a_n
%\newcommand{\seqv}[2]{#1, \ldots, #2} %already abolished
\newcommand{\seqp}[2]{{#1}, \ldots, {#2}} %sequence with additional parameters; \seqv{a_1}{a_n} means the sequence a_1, \ldots, a_n
\newcommand{\seqi}[2]{(#1)_{#2}} %sequence #1 indexed by the set #2
\newcommand{\enum}[3][0]{{{#2}_{#1} \ldots {#2}_{#3}}} %enumeration; \enum[1]{t}{n} means the enumeration t_1 \ldots t_n
%\newcommand{\enumv}[2]{#1 \ldots #2} %already abolished
\newcommand{\enump}[2]{{{#1} \ldots {#2}}} %enumeration with parameter(s); \enump{a_1}{a_n} means the enumeration a_ \ldots a_n
\newcommand{\enumpop}[3]{#1 #2 \ldots #2 #3} %enumeration with operators; \enumpop{a}{op}{b} means repeated applications of the operator op to the sequence a, \ldots, b. For example, \enumpop{a}{+}{b} is a + ... + b.
%\newcommand{\enumprel}[3]{#1 #2 \cdots #2 #3} %??
\newcommand{\setm}[2]{\{ #1 \mid #2 \}} %set with description in math mode; \setm{a}{b} means the set \{ a \mid b \}
\newcommand{\sett}[2]{{\{ #1 \mid \mbox{#2} \}}} %set with description in text mode; \sett{a}{b} means the set \{ a \mid b \} with b a text
\newcommand{\restrict}[2]{{#1|_{#2}}} %the restriction of the function #1 to the domain #2
%%Chapter 2
\newcommand{\alphabet}{\mathcal{A}} %the general alphabet
\newcommand{\kleene}[1]{{#1}^\ast} %Kleene star operator
\newcommand{\nullstring}{\Box} %null, or empty, string
%\newcommand{\symbolset}{\mathit{S}} %a symbol set
\newcommand{\equal}{\equiv} %equal sign in object languages
\newcommand{\absymb}[1]{\alphabet_{#1}} %the alphabet for the first-order language determined by the symbol set #1 (see Section II.2)
\newcommand{\gr}{\mathrm{gr}} %an abbreviation for group theory
\newcommand{\eqr}{\mathrm{eq}} %an abbreviation for equivalence relation theory
\newcommand{\univsymb}{S_\infty} %the universal symbol set (see Section II.2)
\newcommand{\termbase}{\mathit{T}} %the base symbol for sets of terms
\newcommand{\term}[2][]{{\termbase_{#1}^{#2}}} %the set of terms
\newcommand{\languagebase}{\mathit{L}} %the base symbol for a (especially first-order) language
\newcommand{\fstordlang}[2][]{{\languagebase_{#1}^{#2}}} %first-order language
\newcommand{\fstordfmla}[2][0]{{\languagebase_{#1}^{#2}}} %!!
\newcommand{\calrule}[2]{{\displaystyle\frac{\;#1\;}{\;#2\;}}} %calculus rule in schema form
%%Chapter 3
\newcommand{\mul}{\cdot} %the multiplication operator used in arithemtics
\newcommand{\ar}{\mathrm{ar}} %the abbreviation for arithmetics
\newcommand{\arsymb}{S_\ar} %the symbol set for arithmetics
\newcommand{\symbarord}{S^<_\ar} %the symbol set for arithmetics and <
\newcommand{\grp}{\mathrm{grp}} %an abbreviation for group theory
\newcommand{\natstr}{\mathfrak{N}} %the structure (\nat, +, \cdot, 0, 1)
\newcommand{\zahstr}{\mathfrak{Z}} %the structure (\zah, +, \cdot, 0, 1)
\newcommand{\realstr}{\mathfrak{R}} %the structure (\real, +, \cdot, 0, 1)
\newcommand{\natstrord}{{\natstr^<}} %the ordered arithmetic structure (\nat, +, \cdot, 0, 1, <)
\newcommand{\negfunc}{\mathord{\overset{.}{\neg}}} %the negation function in the metalanguage
\newcommand{\dsjfunc}{\mathbin{\overset{.}{\lor}}} %the disjunction function in the metalanguage
\newcommand{\cnjfunc}{\mathbin{\overset{.}{\land}}} %the conjunction function in the metalanguage
\newcommand{\impfunc}{\mathbin{\overset{.}{\rightarrow}}} %the implication function in the metalanguage
\newcommand{\bimfunc}{\mathbin{\overset{.}{\leftrightarrow}}} %the bi-implication function in the metalanguage
\newcommand{\sat}{\mathrel{\mathrm{Sat}}} %the relation satisfiable
\newcommand{\satis}{\models}
\newcommand{\consq}{\models}
\newcommand{\modeled}{\ \mathrm{\reflectbox{$\models$}}} %!!
\newcommand{\bimodels}{\mathrel{\mathord{\reflectbox{$\models$}}\mathord{\models}}} %to be \logequiv
\newcommand{\logequiv}{\mathrel{\mathord{\reflectbox{$\models$}}\mathord{\models}}} %the relation logically equivalent
%\newcommand{\STR}[1]{\mathfrak{#1}} %already abolished
\newcommand{\struct}[1]{\mathfrak{#1}} %the structure of #1
\newcommand{\intpted}[2]{#1^{#2}} %function, constant or relation symbol #1 interpreted as such in the domain of structure #2
\newcommand{\reduct}[2]{{#1|_{#2}}} %the #2-reduct of (the structure) #1
\newcommand{\assgn}{\beta} %a (first-order) assignment
\newcommand{\INT}{\mathfrak{I}} %to be \intp
\newcommand{\intp}{\mathfrak{I}} %an interpretation
\newcommand{\INTP}[2]{(#1, #2)} %to be \intparg
\newcommand{\intpp}[2]{(#1, #2)} %to be \intparg
\newcommand{\intparg}[2]{\pair{#1}{#2}} %the interpretation of (#1, #2), where the arguements #1 and #2 are a structure and an assignment in #1, respectively
\newcommand{\substr}[2]{{[#1]^{#2}}} %substructure generated by #1 in #2
\newcommand{\iso}{\cong} %isomorphic; DEPENDENCY: \finiso, \partiso
\newcommand{\ord}{\mathrm{ord}} %abbreviation for ordering
\newcommand{\axiomsord}{\Phi_{\mathrm{ord}}} %system of axioms for ordering
\newcommand{\axiomspord}{\Phi_{\mathrm{pord}}} %system of axioms for partially defined ordering
\newcommand{\axiomsfd}{\Phi_{\mathrm{fd}}} %system of axioms for field
\newcommand{\axiomsofd}{\Phi_{\mathrm{ofd}}} %system of axioms for ordered field
\newcommand{\field}[1]{\mathop{\mathrm{field}} \intpted{<}{#1}} %the field of < in the structure #1
\newcommand{\pord}{\mathrm{pord}} %abbreviation for partially defined ordering
\newcommand{\suc}{\sigma} %the successor function over \nat
\newcommand{\natsuc}{{\natstr_\suc}} %the structure (\nat, successor, 0)
\newcommand{\sndordpeanoarith}{\Pi} %second-order Peano arithmetics
\newcommand{\df}[2]{\displaystyle\frac{#1}{#2}} %fraction in displaystyle
\newcommand{\varbase}{\mathrm{var}} %the base symbol for the function var; DEPENDENCY: \var
\newcommand{\var}[1]{{\varbase(#1)}} %the set of variables occurring in the term #1
\newcommand{\freebase}{\mathrm{free}} %the base symbol for the function free; DEPENDENCY: \free
\newcommand{\free}[1]{{\freebase(#1)}} %the set of free variables occurring in the formula #1
\newcommand{\SF}[1]{\mathop{\mathrm{SF}}(#1)} %to be \sbfmla
\newcommand{\sbfmlabase}{\mathrm{SF}} %the base symbol for the function sbfmla; DEPENDENCY: \sbfmla
\newcommand{\sbfmla}[1]{{\sbfmlabase(#1)}} %the set of subformulas of the formula #1
\newcommand{\sbst}[2]{{\scriptstyle\frac{\displaystyle #1}{\displaystyle #2}}} %the substitution operation
\newcommand{\scriptsbst}[2]{{\scriptscriptstyle\frac{\scriptstyle #1}{\scriptstyle #2}}} %the substitution used in script style
\newcommand{\exactly}[1]{\exists^{\mathrel{=} #1}} %there are exactly #1 element(s) such that
\newcommand{\exuni}{\exists^{=1}} %there is a unique element such that
\newcommand{\atmost}[1]{\exists^{\mathrel{\leq} #1}} %there are at most #1 element(s) such that
\newcommand{\atleast}[1]{\exists^{\mathrel{\geq} #1}} %there are at least #1 element(s) such that
%%Chapter 4
\newcommand{\derives}{\vdash} %the relation formally derives
\newcommand{\derive}{\ | \hspace{-.4em} -} %an alternative symbol for \vdash
\newcommand{\assm}{{(\mathrm{Assm})}} %the sequent rule (Assm)
\newcommand{\ant}{{(\mathrm{Ant})}} %the sequent rule (Ant)
\newcommand{\pc}{{(\mathrm{PC})}} %the sequent rule (PC)
\newcommand{\ctr}{{(\mathrm{Ctr})}} %the sequent rule (Ctr)
\newcommand{\ora}{{(\lor\mathrm{A})}} %the sequent rule (\lor A)
\newcommand{\ors}{{(\lor\mathrm{S})}} %the sequent rule (\lor S)
\newcommand{\ea}{{(\exists\mathrm{A})}} %the sequent rule (\exists A)
\newcommand{\es}{{(\exists\mathrm{S})}} %the sequent rule (\exists S)
\newcommand{\eq}{{(\equiv)}} %the sequent rule (\equal)
\newcommand{\sub}{{(\mathrm{Sub})}} %the sequent rule (Sub)
\newcommand{\seqcal}{{\mathfrak{S}}} %the sequent calculus \mathfrak{S}
\newcommand{\con}{\mathrel{\mathrm{Con}}} %the relation consistent
\newcommand{\inc}{\mathrel{\mathrm{Inc}}} %the relation inconsistent
%%Chapter 6
\newcommand{\modelclassbase}{\mathrm{Mod}} %the base symbol for \modelclass and \modelclassarg; DEPENDENCY: \modelclass, \modelclassarg
\newcommand{\modelclass}[2]{{\mathop{\modelclassbase^{#1}} #2}} %the set of #1-structures that are models of the #1-formula #2
\newcommand{\thr}[1]{\mathop{\mathrm{Th}}(#1)} %to be \theoarg
\newcommand{\Th}{\mathop{\mathrm{Th}}} %!!
\newcommand{\theorybase}{\mathrm{Th}} %the base symbol for a theory
\newcommand{\theoarg}[1]{{\theorybase(#1)}} %the theory of #1, which takes argument(s), such as Th(\mathfrak{A})
\newcommand{\divides}{\mathbin{|}}
%%Chapter 7
\newcommand{\zfc}{\mathrm{ZFC}} %the abbreviation for Zermelo-Fraenkel set theory with the axiom of choice
%%Chapter 8
\newcommand{\relational}[1]{#1^\mathit{r}} %the relational symbol set corresponding to #1
\newcommand{\invrelational}[1]{#1^{-\mathit{r}}} %the inverse operation of \relational to #1
\newcommand{\relativize}[2]{#1^{#2}} %the relativization of #1 to #2
%%Chapter 9
\newcommand{\logicalsystembase}{\mathcal{L}} %the base symbol for logical systems
\newcommand{\firstorder}{{\mathrm{I}}} %the abbreviation for first-order
\newcommand{\secondorder}{{\mathrm{II}}} %the abbreviation for second-order
\newcommand{\sndordassgn}{\gamma} %a second-order assignment
\newcommand{\freeII}{\free_\mathrm{II}} %to be replaced by \sndordfree
\newcommand{\sndordfree}[1]{{\freebase_\secondorder(#1)}} %the set of free relation variables occurring in the formula #1
\newcommand{\FOL}{\mathcal{L}_\mathrm{I}} %to be \fstordlog
\newcommand{\fstordlog}{{\logicalsystembase_\firstorder}} %the logical system first-order logic
\newcommand{\SOL}{\mathcal{L}_\mathrm{II}} %to be \sndordlog
\newcommand{\sndordlog}{{\logicalsystembase_\secondorder}} %the logical system second-order logic
\newcommand{\LII}{L_\mathrm{II}} %to be \sndordlang
\newcommand{\sndordlang}[1]{{\languagebase_\secondorder^{#1}}} %the second-order language with symbol set #1
\newcommand{\weak}{{\mathit{w}}}
\newcommand{\weaksndordlog}{{\logicalsystembase^\weak_\secondorder}} %weak second-order logic
\newcommand{\weaksndordlang}[1]{{\languagebase^{\weak, #1}_\secondorder}} %the weak second-order language with symbol set #1
\newcommand{\INFL}{\mathcal{L}_{\omega_1\omega}} %to be \infinlog
\newcommand{\infinlog}{{\logicalsystembase_{\omega_1\omega}}} %the logical system infinitary logic \omega_1\omega
\newcommand{\LINF}{L_{\omega_1\omega}} %to be replaced by \infinlang
\newcommand{\infin}{{\omega_1\omega}} %the modifier infinitary \omega_1 \omega
\newcommand{\infinlang}[1]{{\languagebase_\infin^{#1}}} %the infinitary language \omega_1\omega with symbol set #1
\newcommand{\qexist}{\mathit{Q}} %the Q quantifier in Q system
\newcommand{\QL}{\mathcal{L}_Q} %to be replaced by \qlog
\newcommand{\qlog}{{\logicalsystembase_\qexist}} %the logical system Q logic
\newcommand{\LQ}{L_Q} %to be replaced by \qlang
\newcommand{\qlang}[1]{{\languagebase_\qexist^{#1}}} %the Q language with symbol set #1
\newcommand{\varqlog}{{\logicalsystembase^\circ_\qexist}} %the variant of Q logic with Q quantifier interpreted as ``there are infinitarily many''
\newcommand{\domain}[1]{\mbox{the domain of } #1} %!!
\newcommand{\dist}{\mathrm{dist}} %!!
\newcommand{\nme}{\mathrm{NME}} %!!
\newcommand{\indexed}{\mathrm{index}} %!!
\newcommand{\fld}{\mathrm{field}} %!!
\newcommand{\abs}{\mathrm{abs}} %!!
%%Chapter 10
\newcommand{\procp}[1]{\mathfrak{#1}} %procedure that takes an argument
\newcommand{\proc}{\procp{P}} %a common procedure symbol
\newcommand{\R}{\mathrm{R}} %to be \REG
\newcommand{\REG}[1]{\mathrm{R}_{#1}} %register
\newcommand{\LET}{\mathrm{LET}}
\newcommand{\IF}{\mathrm{IF}}
\newcommand{\THEN}{\mathrm{THEN}}
\newcommand{\ELSE}{\mathrm{ELSE}}
\newcommand{\OR}{\mathrm{OR}}
\newcommand{\PRINT}{\mathrm{PRINT}}
\newcommand{\HALT}{\mathrm{HALT}}
\newcommand{\GOTO}{\mathrm{GOTO}}
\newcommand{\p}{\mathrm{P}} %to be \prog
\newcommand{\prog}{\mathrm{P}} %a (register) program
\newcommand{\halt}{\mathrm{halt}} %the abbreviation for halt
\newcommand{\length}{\mathit{l}}
\newcommand{\PA}[2]{\LET \ \R_{#1} = \R_{#1} + #2}
\newcommand{\PS}[2]{\LET \ \R_{#1} = \R_{#1} - #2}
\newcommand{\PI}[4]{\IF \ \R_{#1} = \Box \ \THEN \ #2 \ \ELSE \ #3 \ldots \ \OR \ #4}
\newcommand{\PII}[5]{\IF \ \R_{#1} = \Box \ \THEN \ #2 \ \ELSE \ #3 \ \OR \ldots \ #4 \ldots \ \OR \ #5}
\newcommand{\consqn}[1]{#1^{\models}} %consequence closure
\newcommand{\regdec}{R-decidable} %register-decidable
\newcommand{\regund}{R-undecidable} %register-undecidable
\newcommand{\regenum}{R-enumerable} %register-enumerable
\newcommand{\regcomp}{R-computable} %register-computable
\newcommand{\regaxm}{R-axiomatizable} %register-axiomatizable
\newcommand{\finsat}{fin-satisfiable} %satisfiable by a finite structure
\newcommand{\finval}{fin-valid} %satisfied by every finite structure
\newcommand{\theosub}[1]{\mathop{\mathrm{Th}}(#1)} %!!
\newcommand{\pa}{{\mathrm{PA}}} %the abbreviation of Peano
\newcommand{\peanotheory}{\theorybase_\pa} %the (first-order) Peano theory
\newcommand{\zfctheory}{\theorybase_\zfc} %the ZFC theory
\newcommand{\are}{{\ar^\prime}} %extended arithmetics
\newcommand{\goedel}[1]{n^{#1}} %the Goedel number of #1

\newcommand{\Der}[1]{\mathrm{Der}_{#1}}
\newcommand{\atm}{\mathrm{atm}}
\newcommand{\ngt}{\mathrm{ngt}}
\newcommand{\dsj}{\mathrm{dsj}}
\newcommand{\ext}{\mathrm{ext}}
\newcommand{\sbt}{\mathrm{sbt}}
\newcommand{\sbf}{\mathrm{sbf}}
\newcommand{\drn}{\mathrm{drn}}
\newcommand{\consis}[1]{\mathrm{Consis}_{#1}}
\newcommand{\der}[1]{\mathrm{der}(\underline{n^{#1}})}
\newcommand{\fvar}[1]{\mathop{\mathrm{fvar}(#1)}}
\newcommand{\rpl}{\mathop{\mathrm{rpl}}}
\newcommand{\sft}{\mathop{\mathrm{sft}}}
%%Chapter 11
\newcommand{\vect}[2]{{\overset{#2}{#1}}} %vector, \vect{a}{n} stands for the sequence a_1, \ldots, a_n
\newcommand{\pvarbase}{\mathrm{pvar}} %the base symbol for the function pvar; DEPENDENCY: \pvar
\newcommand{\pvar}[1]{{\pvarbase(#1)}} %the set of propositional variables occurring in the (propositional) formula #1
\newcommand{\pf}{\mathit{PF}} %to be \propfmla
\newcommand{\propfmla}{\mathit{PF}} %the set of propositional formulas
\newcommand{\clauses}{\mathfrak{K}} %a set of clauses
\newcommand{\pclauses}{\mathfrak{P}} %a set of positive clauses
\newcommand{\scls}[1]{{\mathfrak{K}(#1)}} %to be \setofclauses
\newcommand{\setofclauses}[1]{{\mathfrak{K}(#1)}} %the set of clauses associated to #1
\newcommand{\resolutionbase}{\mathrm{Res}} %the base symbol for the function Res; DEPENDENCY: \res, \resi
\newcommand{\res}[1]{{\resolutionbase(#1)}} %the resolution of #1
\newcommand{\resi}[2]{{\resolutionbase_{#1}(#2)}} %the resolution of #2 in at most #1 steps
\newcommand{\hornresolutionbase}{\mathrm{HRes}} %the base symbol for the function HRes; DEPENDENCY: \hres, \hresi
\newcommand{\hres}[1]{{\hornresolutionbase(#1)}} %the Horn-resolution of #1
\newcommand{\hresi}[2]{{\hornresolutionbase_{#1}(#2)}} %the Horn-resolution of #2 in at most #1 steps
\newcommand{\grndinstbase}{\mathrm{GI}} %the base symbol for the function GI; DEPENDENCY: \gi
\newcommand{\gi}[1]{{\grndinstbase(#1)}} %the set of ground instances of #1
\newcommand{\unifiedresolutionbase}{\mathrm{URes}} %the base symbol for the function URes; DEPENDENCY: \ures, \uresi
\newcommand{\ures}[1]{{\unifiedresolutionbase(#1)}} %the set of unified resolution of #1
\newcommand{\uresi}[2]{{\unifiedresolutionbase_{#1}(#2)}} %the set of unified resolution of #2 in at most #1 steps
\newcommand{\unifiedhornresolutionbase}{\mathrm{UHRes}} %the base symbol for the function UHRes; DEPENDENCY: \uhres, \uhresi
\newcommand{\uhres}[1]{{\unifiedhornresolutionbase(#1)}} %the set of unified Horn-resolution of #1
\newcommand{\uhresi}[2]{{\unifiedhornresolutionbase_{#1}(#2)}} %the set of unified Horn-resolution of #2 in at most #1 steps
%%Chapter 12
\newcommand{\funbyvect}[2]{#1 \mapsto #2}
\newcommand{\partismbase}{\mathrm{Part}} %the base symbol for the function Part; DEPENDENCY: \partism
\newcommand{\partism}[2]{{\partismbase(#1, #2)}} %partial isomorphism from #1 to #2
\newcommand{\domainbase}{\mathrm{dom}} %the base symbol for the function dom; DEPENDENCY: \dom
\newcommand{\dom}[1]{{\domainbase(#1)}} %domain of the map #1
\newcommand{\rangebase}{\mathrm{rg}} %the base symbol for the function rg; DEPENDENCY: \rg
\newcommand{\rg}[1]{{\rangebase(#1)}} %range of the map #1
\newcommand{\isop}[1]{\iso_{#1}} %!!
\newcommand{\finiso}{\iso_\mathit{f}} %finitely isomorphic
\newcommand{\partiso}{\iso_\mathit{p}} %partially isomorphic
%HERE
\newcommand{\emb}{\rightarrow} %the relation embeddable; DEPENDENCY: \finemb, \partemb.
\newcommand{\finemb}{\mathrel{\emb_\mathit{f}}} %finitely embeddable
\newcommand{\partemb}{\mathrel{\emb_\mathit{p}}} %partially embeddable
\newcommand{\qrbase}{\mathrm{qr}} %the symbol for the function qr; DEPENDENCY: \qr
\newcommand{\qr}[1]{{\qrbase(#1)}} %quantifier rank
\newcommand{\mrkbase}{\mathrm{mrk}} %the symbol for the function mrk; DEPENDENCY: \mrk
\newcommand{\mrk}[1]{{\mrkbase(#1)}} %modified quantifier rank
\newcommand{\ehrenfeuchtgamebase}{\mathrm{G}} %the base symbol for the function G; DEPENDENCY: \egame, \egamep
\newcommand{\egame}[2]{{\ehrenfeuchtgamebase(#1, #2)}} %ordinary Ehrenfeucht game
\newcommand{\egamep}[3][]{{\ehrenfeuchtgamebase_{#1}(#2, #3)}} %Ehrenfeucht game with parameter(s)
%%Chapter 13
\newcommand{\logsys}{\logicalsystembase} %logical system
\newcommand{\modelclassarg}[3][]{{\modelclassbase^{#1}_{#2}(#3)}} %the class of #1-structures that are models of #3 with respect to the logical system #2
\newcommand{\boole}[1]{\mathrm{Boole}(#1)} %the logical system #1 satisfies the condition Boole
\newcommand{\rel}[1]{\mathrm{Rel}(#1)} %the logical system #1 satisfies the condition Rel
\newcommand{\repl}[1]{\mathrm{Repl}(#1)} %the logical system #1 satisfies the condition Repl
\newcommand{\losko}[1]{\mathrm{L\ddot{o}Sko}(#1)} %the logical system #1 satisfies the condition LoeSko
\newcommand{\comp}[1]{\mathrm{Comp}(#1)} %the logical system #1 satisfies the condition Comp
\newcommand{\weakereq}{\leq} %A \weakereq B: the logical system B is at least as strong as A (A is weaker than or equally strong as B)
\newcommand{\eqstrong}{\sim} %A is equally strong as B
\newcommand{\eff}{\mathrm{eff}} %the modifier eff
\newcommand{\effwkereq}{\mathrel{\weakereq_\eff}} %A \effwkereq B: B is at least as effectively strong as A
\newcommand{\effeqstrng}{\mathrel{\eqstrong_\eff}} %A \effeqstrng B: A is effectively equally strong as B

%New commands for Appendix A
\newcommand{\numl}[1]{\underline{#1}} %the numeral corresponding to #1 in underline
\newcommand{\numb}[1]{\mbf{#1}} %the numeral corresponding to #1 in boldface

%Redefining Commands
\renewcommand{\qedsymbol}{$\talloblong$}
\renewcommand{\thechapter}{\Roman{chapter}}
\renewcommand{\thesection}{\arabic{section}}
\renewcommand{\theequation}{\arabic{equation}}

%New Lengths
\newlength{\DefaultQuoteLength}
\settowidth{\DefaultQuoteLength}{note that the terms here are only slightly different from those in}
\newlength{\DefaultTabularizedArgumentLength}
\settowidth{\DefaultTabularizedArgumentLength}{note that the terms here are only slightly different from aaaa}
\newlength{\DefaultDefinitionItemLength}
\settowidth{\DefaultDefinitionItemLength}{note that the terms here are only slightly from}

%New Environments
%%General
\newenvironment{definition}[1]{\textbf{#1.}\ }{}
\newenvironment{theorem}[1]{\textbf{#1.}\ \begin{em}}{\end{em}}
%\newenvironment{smallcenter}{\smallskip\\\phantom{.}\hfill}{\hfill\phantom{.}\smallskip\\}
\newenvironment{medcenter}{ \medskip\\ \phantom{(} \hfill }{ \hfill \phantom{)} \medskip\\}
\newenvironment{quoteno}[1]{#1\hfill}{\hfill\phantom{(+)}}
\newenvironment{bquoteno}[2]{#2\hfill\begin{minipage}[c]{#1}}{\end{minipage}}
%%Chapter 4
\newenvironment{seqrule}{\begin{array}}{\end{array}}
\newenvironment{derivation}{\begin{tabular}{llll}}{\end{tabular}}
%%Chapter 10
\newenvironment{program}{\begin{array}{rl}}{\end{array}}

\newtheorem*{claim}{Claim}

\renewcommand{\qedsymbol}{$\talloblong$}

\theoremstyle{remark}
\newtheorem*{remark}{Remark}

%if resetting of equation counter is required, use the instruction below:
%\setcounter{equation}{0}


\begin{document}
\begin{titlepage}
\noindent\textsc{\Huge Annotations to the Book\medskip\\``Mathematical Logic''\bigskip\\by H.-D.\ Ebbinghaus, J.\ Flum\smallskip\\and W.\ Thomas\bigskip\\\large with Solutions Manual}
\end{titlepage}
\titleformat{\chapter}[display]{\huge\bfseries}{\thechapter}{.5em}{}
\titleformat{\section}[hang]{\Large\rm}{\S\thesection.}{.5em}{}
%%Chapter I---------------------------------------------------------------------------
\chapter{Introduction}
%Section I.1-------------------------------------------------------------------------
\section{An Example from Group Theory}
\begin{enumerate}[1.]
\item \textbf{Note on Theorem~1.1.} From the proof we infer that a right inverse of an element is also a left inverse of that element. Analogously, one can show that a left inverse (of an element) is also a right inverse. Using this fact, one easily obtains the following property symmetrical to (G2):\medskip\\
For every $x$: \quad $e \circ x = x$.\medskip\\
It implies that any element has a unique (left or right) inverse; thus we may speak of \emph{the} inverse of an element.
\end{enumerate}
%End of Section I.1------------------------------------------------------------------
%%Section I.2-------------------------------------------------------------------------
%\section{}
%\begin{enumerate}[1.]
%\item
%%
%\end{enumerate}
%%End of Section I.2--------------------------------------------------------------------------------
%End of Chapter I--------------------------------------------------------------------
%%Chapter II--------------------------------------------------------------------------
\chapter{Syntax of First-Order Languages}
%Section II.1------------------------------------------------------------------------
\section{Alphabets}
\begin{enumerate}[1.]
%
\item \textbf{Note on the Discussion about Countability below the Proof of Lemma~1.1 on Page~12.} First, we verify that every subset $M_0$ of an at most countable set $M$ is at most countable: Clearly $M_0$ is at most countable if it is empty, by definition. So, let us suppose that $M_0$ is nonempty. By \reftitle{Lemma~1.1}, there is an injective map $\beta: M \to \nat$. The map $\restrict{\beta}{M_0}$ (the restriction of $\beta$ to $M_0$) is injective as well, ensuring that $M_0$ is at most countable by \reftitle{Lemma~1.1}.\par
Next, we verify that $M_1 \setsum M_2$ is at most countable provided that both $M_1$ and $M_2$ are at most countable: If $M_1$ is a subset of $M_2$ ($M_1 \setminus M_2 = \emptyset$) or $M_2$ is a subset of $M_1$ ($M_2 \setminus M_1 = \emptyset$), then $M_1 \setsum M_2 = M_2$ or $M_1 \setsum M_2 = M_1$, respectively; in each case, $M_1 \setsum M_2$ is at most countable by hypothesis.\par
So let us assume that both $M_1 \setminus M_2$ and $M_2 \setminus M_1$ are nonempty. Without loss of generality, we assume additionally that $M_1$ and $M_2$ are disjoint; if $M_1 \setprod M_2$ is nonempty, then take the two disjoint sets $M_1' \colonequals M_1$ and $M_2' \colonequals M_2 \setminus M_1$ ($M_2' \subset M_2$ is at most countable by the above result) and discuss similarly.\par
By \reftitle{Lemma~1.1}, there are injective maps $\beta_1: M_1 \to \nat$ and $\beta_2: M_2 \to \nat$. It is easy to check that the map $\beta: (M_1 \setsum M_2) \to \nat$, where
\[
\beta(a) \colonequals
\begin{cases}
2\beta_1(a) + 1 & \mbox{if \(a \in M_1\)}, \cr
2\beta_2(a)     & \mbox{if \(a \in M_2\)},
\end{cases}
\]
is injective, ensuring that $M_1 \setsum M_2$ is at most countable by \reftitle{Lemma~1.1}.
%
\item \textbf{Solution to Exercise~1.3.} Given $a, b \in \real$ with $a < b$, let $I \colonequals \clintv{a}{b}$. We define a sequence $\seqi{I_n}{n \in \nat}$ of closed intervals inductively as follows (we write $l_n$ and $u_n$ for the left and the right endpoints of $I_n$, namely $I_n = \clintv{l_n}{u_n}$):
\[
I_0 \colonequals I
\]
and for $n \geq 0$,
\[
I_{n + 1} \colonequals
\begin{cases}
\clintv{l_n}{\displaystyle\frac{l_n + \alpha(n)}{2}} & \mbox{if \(\alpha(n) = u_n\)}, \cr
\clintv{\displaystyle\frac{\alpha(n) + u_n}{2}}{u_n} & \mbox{if \(l_n \leq \alpha(n) < u_n\)}, \cr
I_n & \mbox{otherwise}.
\end{cases}
\]
A simple induction on $n$ yields $l_n \leq l_{n + 1} < u_{n + 1} \leq u_n$; hence, $I_n \supset I_{n + 1}$ (so $I_0 \supset I_1 \supset \ldots$) and $I_n \neq \emptyset$. By the choice of $\seqi{I_n}{n \in \nat}$, we have $\alpha(n) \not\in I_{n + 1}$.\par
We argue that
\[
\bigcap\limits_{n \in \mathbb{N}} I_n \not = \emptyset.
\]
By the completeness\footnote{Also known as the \emph{least-upper-bound property}: Every nonempty set which is bounded above has a supremum, i.e.\ a least upper bound. In fact, this supremum, if exists, is unique. A direct consequence of this property is, symmetrically: Every nonempty set which is bounded below has a (unique) infimum.} of $\realstr^<$ (the ordered field of $\real$, cf.\ \reftitle{Section~III.1}), we have:
\begin{itemize}
%%
\item The supremum $l$ of the set $\setm{l_n}{n \in \nat}$ exists because this set is bounded above by any $u_n$. Hence, $l_n \leq l \leq u_n$ for $n \in \nat$.
%%
\item The infimum $u$ of the set $\setm{u_n}{n \in \nat}$ exists because this set is bounded below by $l$. Hence, $l \leq u \leq u_n$ for $n \in \nat$.
%%
\end{itemize}
Therefore, $l_n \leq l \leq u \leq u_n$ for $n \in \nat$. Let
\[
c \defas \frac{l + u}{2}.
\]
It follows that
\[
l_n \leq l \leq c \leq u \leq u_n
\]
and so $c \in I_n$ for $n \in \nat$. In other words, our claim that $\bsetprod_{n \in \nat} I_n \neq \emptyset$ is true because $c \in \bsetprod_{n \in \nat} I_n$.
\medskip\\
More specifically, $c \in I$ and $c \not\in \setm{\alpha(n)}{n \in \nat}$. We conclude that $I$, and hence $\real$ also, are uncountable.
%
\item \textbf{Solution to Exercise~1.4.} (INCOMPLETE, RESOLVE THE REFERENCE)
\begin{asparaenum}[(a)]
\item \label{chiisec1.4a}
Let $M_0, M_1, \ldots$ be at most countable sets. Then there are injective maps $\beta_0: M_0 \to \nat, \beta_1: M_1 \to \nat, \ldots$ (cf.\ \reftitle{Lemma~1.1(c)}).\par
Let $p_n$ be the $n$th prime: $p_0 = 2, p_1 = 3, p_2 = 5$, and so on. We choose the map $\beta: \bsetsum_{n \in \nat} M_n \to \nat$ in which
\[
\beta(a) \colonequals p_k \mul p_{k + \beta_k(a) + 1},
\]
where $k = \min \setm{n \in \nat}{a \in M_n}$. It is easy to see that $\beta$ is injective.
%%
\item Let $\alphabet_n$ denote the set of strings over $\alphabet$ of length $n$ in which all characters are indexed $\leq n$, namely $\alphabet_n \colonequals \setm{\enum[i_1]{a}{i_n}}{\seq[1]{i}{n} \leq n}$ (note that $\alphabet_0 = \setenum{\nullstring}$). Clearly, $\mathcal{A}_n$ is finite and hence at most countable, and
\[
\kleene{\alphabet} = \bsetsum_{n \in \nat} \alphabet_n.
\]
By \ref{chiisec1.4a}, $\kleene{\alphabet}$ is at most countable. Also, $a_0, a_0a_0, a_0a_0a_0, \ldots$ are all in $\kleene{\alphabet}$, so $\kleene{\alphabet}$ is infinite. Therefore, $\kleene{\alphabet}$ is countable.
\end{asparaenum}
%
\item \textbf{Solution to Exercise~1.5.} The claim is trivial for the case $M = \emptyset$, so we shall assume that $M$ is nonempty below.\par
As suggested in hint, consider an arbitrary map $\alpha: M \to \powerset{M}$. We shall show that the set $C \colonequals \setm{a \in M}{a \not\in \alpha(a)}$ (a subset of $M$) is not in the range of $\alpha$, and conclude from this that there is no surjective (and hence no bijective) map from $M$ onto the power set $\powerset{M}$ of $M$.\par
For the sake of contradiction, we assume $\alpha(a) = C$ for some $a \in M$. We have either $a \in C$ or $a \not\in C$. If $a \in C$, then by the definition of $C$ we have that $a \not\in C$, as $\alpha(a) = C$. If $a \not\in C$, namely $a \not\in \alpha(a)$, then by the definition of $C$ we have that $a \in C$. Either way we get a contradicting statement. So there is no $a \in M$ with $\alpha(a) = C$, and the claim is proved.\par
\textit{Remark.} This is Cantor's famous \emph{diagonalization argument}.
%
\end{enumerate}
%End of Section II.1-----------------------------------------------------------------
%
%Section II.2------------------------------------------------------------------------
\section{The Alphabet of a First-Order Language}
\begin{enumerate}[1.]
%
\item \textbf{Note on Alphabets of First-Order Languages.} (INCOMPLETE) The set of variables $\{ v_0, v_1, \ldots \}$ is \emph{countable}.\par
In the literature, $v_0, v_1, \ldots$ are sometimes referred to as \emph{formal variables}, while $x, y, f, g, R, Q, \ldots$ as \emph{metavariables} (which serve as placeholders): Certainly $v_0 \neq v_1$, but it is possible that $x = y$; $x = y$ means that $x$ and $y$ \emph{denote} the same formal variable, say $v_{32}$. Likewise, $f = g$ and $R = Q$ mean, respectively, that $f$ and $g$ denote the same function symbol and that $R$ and $Q$ denote the same relation symbol. (HERE)
%
\end{enumerate}
%End of Section II.2-----------------------------------------------------------------
\ 
\\
\\
%Section II.3------------------------------------------------------------------------
\section{Terms and Formulas in First-Order Languages}
%{\large \S3. Terms and Formulas in First-Order Languages}
\begin{enumerate}[1.]
%
\item \textbf{Note on the Concept of Calculus.} For any set of objects defined by means of a system of \emph{rules} (or more formally, a \emph{calculus}, which will be introduced later in \reftitle{Section 4}) such as the set $\term{S}$ of $S$-terms (cf. \reftitle{Definition 3.1}), its' objects are constructed by \emph{finite} applications of these rules.\newline
\\
On the other hand, the calculus defining a set $A$ always follows the pattern:
\begin{enumerate}[(i)]
\item \textit{Base rules.} Enumerate some elements in $A$ (without premises).
%%
\item \textit{Inductive rules.} If an element $a$ in $A$ satisfies the premise, then $f(a)$ is also an element in $A$ ($f$ denotes some string operation here).
\end{enumerate}
In fact, when we define a set by means of a calculus, we mean that \emph{all} elements in it are constructed by applying the rules of that calculus: Every element in that set has a derivation. Put in another way, the set defined by the calculus is the \emph{smallest} set satisfying the calculus, where by the smallest set we mean the intersection of all sets satisfying the calculus.\footnote{Actually, some textbooks do use the term \emph{the smallest set}, cf. \cite{Dirk_van_Dalen}.}\newline
\\
Here we show that these two notions are equivalent, by proving that the set $A$ in which every element in it has a derivation is the smallest set satisfying the calculus. We first prove that $A$ satisfies the calculus, by induction on the \emph{length} (or the \emph{number of steps}) $n$ of a derivation of an element:
\begin{enumerate}[(1)]
\item $n = 1$. This corresponds to elements constructed by base rules.
%%
\item $n > 1$. Let $a \in A$ be an element with a derivation of length $n - 1$, then $f(a)$ is obtained by applying some inductive rule to $a$ and thus has a derivation of length $n$. We have that $f(a) \in A$.
\end{enumerate}
Secondly, we prove that every element $a \in A$ is also an element in every set $B$ satisfying the calculus, again by induction on the length $n$ of a derivation of $a$:
\begin{enumerate}[(1)]
\item $n = 1$. Every element $a$ constructed in this case clearly belongs to $B$.
%%
\item $n > 1$. Let $a \in A$ be an element with a derivation of length $n - 1$. By induction, $a \in B$. Furthermore, $f(a)$ is an element in $A$ with a derivation of length $n$. Since $B$ satisfies the calculus, we have $f(a) \in B$.
\end{enumerate}
%
\item \textbf{Note to the Paragraph after Lemma 3.3.} In fact, most mathematics textbooks use the symbols $\Rightarrow$ and $\Leftrightarrow$ to denote implication (if \ldots then) and bi-implication (if and only if), respectively, besides using $\forall$ and $\exists$. Note that these are usage in metalanguage. (In contrast, $\rightarrow$, $\leftrightarrow$, etc. are usage in object language.) Throughout this text we state them verbally, for example, using \emph{if \ldots then} instead of $\Rightarrow$.
\end{enumerate}
%End of Section II.3-------------------------------------------------------------------------------
\ 
\\
\\
%Section II.4------------------------------------------------------------------------------------------------
{\large \S4. Induction in the Calculus of Terms and in the Calculus of Formulas}
\begin{enumerate}[1.]
\item \textbf{Note to the Calculus of Formulas Mentioned in Page 19.} The following are the rules for the calculus of formulas:
\[
\begin{array}{ll}
\mbox{(F1) }\displaystyle \frac{\,}{t_1 \equiv t_2}; & \mbox{(F2) }\displaystyle \frac{\,}{Rt_1 \ldots t_n} \mbox{ if $R \in S$ is $n$-ary}; \\
\, & \, \\
\mbox{(F3) }\displaystyle \frac{\varphi}{\neg \varphi}; & \mbox{(F4) }\displaystyle \frac{\displaystyle {\varphi, \; \psi}}{(\varphi \ast \psi)} \mbox{ for $\ast = \land, \lor, \rightarrow, \leftrightarrow$}; \\
\, & \, \\
\mbox{(F5) }\displaystyle \frac{\varphi}{\forall x \varphi}, \;\; \displaystyle \frac{\varphi}{\exists x \varphi}. & \ 
\end{array}
\]
%
\item \textbf{Proof of 4.1 (a) for the Case of $S$-Formula.} Let $P$ be the same property on $\mathcal{A}_S^*$ as in 4.1 (a). \\
\\
(F1)$^\prime$, (F2)$^\prime$: Formulas of the form $t_1 \equiv t_2$ or $Rt_1 \ldots t_2$ are nonempty. \\
\\
(F3)$^\prime$, (F4)$^\prime$, (F5)$^\prime$: Every formula formed according to these three rules must contain $\neg, \land, \lor, \rightarrow, \leftrightarrow, \forall x,$ or $\exists x$, for some variable $x$, and hence is nonempty.\nolinebreak\hfill$\talloblong$
%
\item \textbf{Proof of 4.1 (b) (1).} Let $P$ be the property on $\mathcal{A}_S^*$ which holds for a string $\zeta$ iff $\zeta$ is distinct from $\circ$. We show by induction on terms that every $S_{\mathrm{gr}}$-term possesses this property. \\
\\
$t=x$, $t=e$: $t$ is distinct from $\circ$. \\
\\
$t=\circ t_1 t_2$: From (a) we know $t_1$ and $t_2$ are not empty. Therefore $t$ is distinct from $\circ$.\nolinebreak\hfill$\talloblong$
%
\item \textbf{Note to 4.1 (c).} \textit{No $S$-term contains a left or right parenthesis.} \\
\textit{Proof.} (T1)$^\prime$, (T2)$^\prime$: Terms of the form $x$ or $c$ (with $c \in S$) contain no left or right parenthesis. \\
\\
(T3)$^\prime$: $t = f t_1 \ldots t_n$ for some $n$-ary function symbol $f$ and $t_1$, \ldots, $t_n$ are terms. By induction hypothesis each of $t_1$, \ldots, $t_n$ contains no left or right parenthesis. At the same time, $f$ is distinct from left parenthesis ( and right parenthesis ). It turns out that $t$ contains no left or right parenthesis.\nolinebreak\hfill$\talloblong$
%
\item \textbf{Note to 4.2 (a).} \textit{For all variables $x$, $x$ is the only term which begins with the variable $x$}.\\
\\
\textit{Proof.} Let $P$ be the property on $\mathcal{A}_S^*$ which holds for a string $\zeta$ iff
\[
\mbox{if } \zeta \mbox{ begins with } x \mbox{, then } \zeta = x.
\]
(T1)$^\prime$: For the term (more precisely, variable) $x$, clearly the property holds; other terms are distinct from $x$ and do not begin with $x$, and still the property holds.\\
\\
(T2)$^\prime$: The argument is similar.\\
\\
(T3)$^\prime$: In this case every term begins with a function symbol, and hence the property holds.\nolinebreak\hfill$\talloblong$
%
\item \textbf{Proof of Lemma 4.3.}
\begin{enumerate}[(a)]
\item Immediately follows from the proof of 4.2 (a).
%%
\item We show $\varphi_1 = \varphi_1^\prime$ by considering the structure of $\varphi_1$: \\
\ 
\\ $\varphi_1 = t_1 \equiv t_2$: $\varphi_1^\prime$ must begin with a term (since otherwise contrary to the premise), and this term is $t_1$ by 4.2 (a). Following $t_1$ are the symbol $\equiv$ and the term $t_2$ by the similar argument. Therefore $\varphi_1 = \varphi_1^\prime$. \\
\ 
\\ $\varphi_1 = Rt_1 \ldots t_n$: $\varphi_1^\prime$ must begin with $R$ by the premise. Following $R$ is the string $t_1^\prime \ldots t_n^\prime$ (since $R$ is $n$-ary), and we can conclude that $t_i = t_i^\prime$ for $1 \leq i \leq n$ by 4.2 (a). Therefore $\varphi_1 = \varphi_1^\prime$. \\
\ 
\\ $\varphi_1 = \neg \varphi$: $\varphi_1^\prime$ must begin with $\neg$ by the premise. Following $\neg$ is the formula $\varphi^\prime$, and $\varphi = \varphi^\prime$ by 4.2 (b). Therefore $\varphi_1 = \varphi_1^\prime$. \\
\ 
\\ $\varphi_1 = (\varphi \ast \psi)$ or $Q\varphi$, where $\ast = \land, \lor, \rightarrow, \leftrightarrow$, and $Q = \forall x, \exists x$: Similar to the above case. \\
\ 
\\From the argument above, we obtain
\[
\varphi_2 \ldots \varphi_n = \varphi_2^\prime \ldots \varphi_n^\prime.
\]
After repeatedly applying this argument the proof is complete.\nolinebreak\hfill$\talloblong$
\end{enumerate}
%
\item \textbf{Proof of Theorem 4.4 (a).} The first statement is trivial by the calculus of terms. For the second statement, suppose there is a term $f^\prime t_1^\prime \ldots t_m^\prime$ (this is the only form of the term for the following equality to hold) such that
\[
f t_1 \ldots t_n = f^\prime t_1^\prime \ldots t_m^\prime.
\]
Then $f = f^\prime$ since both are function symbols and otherwise contrary to the above equality. From this we conclude that
\[
f t_1 \ldots t_n = f t_1^\prime \ldots t_m^\prime.
\]
Cancelling the symbol $f$ from both sides of the above equality sign, we obtain
\[
t_1 \ldots t_n = t_1^\prime \ldots t_m^\prime ,
\]
and by 4.3 (a) the conclusion follows.\nolinebreak\hfill$\talloblong$
%
%II.4.6------------------------------------------------------------------------------------------------------
\item \textbf{Solution to Exercise 4.6.}
\begin{enumerate}[(a)]
\item Let $P$ be the property on $T^S$ which holds for a term $t$ iff
\[
\mbox{for all variables } x, xt \mbox{ is derivable iff } x \in \var (t).
\]
\ 
\\
(T1)$^\prime$: For all variables $x$, the property holds for the term (more precisely, variable) $x$ since $xx$ is derivable and $x \in \var(x):=\{x\}$; for other terms $y$ distinct from $x$, $xy$ is not derivable and $x \not \in \var(y):=\{y\}$.\\
\\
(T2)$^\prime$: For all variables $x$, the property holds for the term $c$ (a constant in $S$) since neither $xc$ is derivable nor $x \in \var(c):=\emptyset$.\\
\\
(T3)$^\prime$: Without loss of generality, consider the term $ft_1 \ldots t_2$, where $f$ is a function symbol in $S$ and $t_1$, \ldots, $t_n$ are terms from $T^S$. Assume that $P$ holds for $t_i$, $1 \leq i \leq n$. Then for all variables $x$, \\
\[
\begin{array}{ll}
\, & xft_1 \ldots t_2 \mbox{ is derivable} \\
\Iff & xt_i \mbox{ is derivable for some } i \mbox{    (since this is the only way $xft_1 \ldots t_2$} \\
\, & \mbox{can be derived in $\mathfrak{C}_v$)} \\
\Iff & x \in \var(t_i) \mbox{ for some } i \mbox{    (by induction hypothesis)} \\
\Iff & x \in \var(ft_1 \ldots t_n).
\end{array}
\]
%%
\item Let the calculus $\mathfrak{C}_{sf}$ consist of the following rules:
\[
\begin{array}{l}
\displaystyle \frac{ \, }{\varphi \;\;\; \varphi}; \;\;\;\; \frac{\varphi \;\;\; \psi}{\varphi \;\;\; \neg \psi}; \\
\, \\
\displaystyle \frac{\varphi \;\;\; \psi_i}{\varphi \;\;\; (\psi_1 \ast \psi_2)} \mbox{ for } \ast = \land, \lor, \rightarrow, \leftrightarrow \mbox{ and } i \in \{1, 2\}; \\
\, \\
\displaystyle \frac{\varphi \;\;\; \psi}{\varphi \;\;\; \forall x \psi}; \;\;\;\; \displaystyle \frac{\varphi \;\;\; \psi}{\varphi \;\;\; \exists x \psi};
\end{array}
\]
For all $S$-formulas $\varphi$ and $\psi$, $\varphi\psi$ is derivable in $\mathfrak{C}_{sf}$ iff $\varphi \in \mathrm{SF}(\psi)$.\nolinebreak\hfill$\talloblong$
\end{enumerate}
%End of II.4.6-----------------------------------------------------------------------------------------------
%
%II.4.7------------------------------------------------------------------------------------------------------
\item \textbf{Solution to Exercise 4.7.} Consider the formula $\varphi \land \psi \lor \eta$: It can be interpreted as both
\begin{enumerate}[(1)]
\item $\varphi$ in conjunction ($\land$) with $\psi \lor \eta$, or
%%
\item $\varphi \land \psi$ in disjunction ($\lor$) with $\eta$,
\end{enumerate}
and hence is not well defined.\nolinebreak\hfill$\talloblong$
%End of II.4.7-----------------------------------------------------------------------------------------------
%
%II.4.8------------------------------------------------------------------------------------------------------
\item \textbf{Solution to Exercise 4.8.} We only show the analog of 4.3 (b). The proof is similar to that of 4.3 (b), except that the case $\varphi_1 = \ast \varphi \psi$ where $\ast = \land, \lor, \rightarrow, \leftrightarrow$ is confirmed by this: \\
\ 
\\$\varphi_1^\prime$ must begin with $\ast$ by the premise. Following $\ast$ are the formulas $\varphi^\prime$ and $\psi^\prime$, and $\varphi = \varphi^\prime$ and $\psi = \psi^\prime$ by 4.2 (b). Therefore $\varphi_1 = \varphi_1^\prime$.\nolinebreak\hfill$\talloblong$
%End of II.4.8-----------------------------------------------------------------------------------------------
%
%II.4.9------------------------------------------------------------------------------------------------------
\item \textbf{Solution to Exercise 4.9.} Notice that the first symbol of the string $\zeta$ which is a postfix of $t_1 \ldots t_n$ that begins from the $(i+1)$st symbol must lie somewhere in a term $t_j$ for some $1 \leq j \leq n$, i.e., $t_1 \ldots t_n = \xi \zeta$ and $\zeta = \varsigma t_{j+1} \ldots t_n$, where $\varsigma$ is a postfix of $t_j$. \\
\ 
\\It remains to show that for any term $t$ and $0 \leq k < \mbox{ length of } t$, there are uniquely determined $\xi^\prime$ and $\eta^\prime$ $\in \mathcal{A}_S^*$ and $t^\prime \in T^S$ such that $t = \xi^\prime t^\prime \eta^\prime$, where the length of $\xi^\prime = k$. \\
\ 
\\We show this by induction on terms: \\
\ 
\\$t = x$, $t = c$: It must be the case that $k = 0$, $\xi^\prime = \eta^\prime = \boxempty$, and $t^\prime = t$.\\
\ 
\\$t = ft_1 \ldots t_n$:
\begin{enumerate}[(i)]
\item $k = 0$ ($\xi^\prime = \boxempty$): Let $t^\prime = t$, and $\eta^\prime = \boxempty$.
%%
\item $k > 0$: Let $t = \xi^\prime \zeta^\prime$. Then the first symbol of $\zeta^\prime$ must lie somewhere in $t_j$ for some $1 \leq j \leq n$. By induction bypothesis, there are strings $\xi^{\prime \prime}$ and $\eta^{\prime \prime}$ $\in \mathcal{A}_S^*$ and $t^{\prime \prime} \in T^S$ such that $t_j = \xi^{\prime \prime} t^{\prime \prime} \eta^{\prime \prime}$. Specifically, $\xi^{\prime \prime}$ is a postfix of $\xi^\prime$ and $\eta^{\prime \prime}$ is a prefix of $\eta^\prime$ and, at the same time, $t^{\prime \prime} = t^\prime$.\nolinebreak\hfill$\talloblong$
\end{enumerate}
\end{enumerate}
%End of II.4.9-----------------------------------------------------------------------------------------------
%End of II.4-------------------------------------------------------------------------------------------------
\ 
\\
\\
%Section II.5------------------------------------------------------------------------------------------------
{\large \S5. Free Variables and Sentences}
\begin{enumerate}[1.]
%II.5.2------------------------------------------------------------------------------------------------------
\item \textbf{Solution to Exercise 5.2.} By observation we know that $\mathfrak{C}_{nf}$ permits to derive precisely those strings of the form $x \varphi$ for which $\varphi \in L^S$. Therefore, it remains to show that they are precisely those strings for which $x$ does not occur free in $\varphi$. \\
\ 
\\Let $P$ be the property on $L^S$ which holds for a formula $\varphi$ iff
\[
\mbox{for all variables $x$, } x \varphi \mbox{ is derivable in } \mathfrak{C}_{nf} \mbox{ iff } x \varphi \mbox{ does not occur free in } \varphi.
\]
We show by induction on formulas that every formula possesses this property:\\
\ 
\\$\varphi = t_1 \equiv t_2$: If $x \not \in \free(t_1) \cup \free(t_2)$ then $x \varphi$ is derivable; otherwise it isn't.\\
\ 
\\$\varphi = Rt_1 \ldots t_n$: Similar.\\
\ 
\\$\varphi = \forall x \psi, \varphi = \exists x \psi$: $x \not \in \free(\varphi) = \free(\psi) \setminus \{ x \}$ and $x \varphi$ is derivable.\\
\ 
\\$\varphi = \neg \psi$: $x \varphi$ is derivable iff $x \psi$ is derivable (since this is the only way $x \varphi$ can be derived)
\\iff $x \not \in \free(\psi)$ (by induction hypothesis)
\\iff $x \not \in \free(\varphi)$ (since $\free(\varphi) = \free(\neg \psi) = \free(\psi)$). \\
\ 
\\$\varphi = (\psi \ast \chi)$ for $\ast = \land, \lor, \rightarrow, \leftrightarrow$: $x \varphi$ is derivable iff $x \psi$ and $x \chi$ are both derivable (since this is the only way $x \varphi$ can be derived)
\\iff $x \not \in \free(\psi)$ and $x \not \in \free(\chi)$ (by induction hypothesis)
\\iff $x \not \in \free(\psi) \cup \free(\chi)$
\\iff $x \not \in \free((\psi \ast \chi)) = \free(\varphi)$. \\
\ 
\\$\varphi = \forall y \psi, \varphi = \exists y \psi$: $x \varphi$ is derivable iff $x \psi$ is derivable (since this is the only way $x \varphi$ can be derived)
\\iff $x \not \in \free(\psi)$ (by induction hypothesis)
\\iff $x \not \in \free(\psi) \setminus \{ y \} = \free(\varphi)$.\nolinebreak\hfill$\talloblong$
\end{enumerate}
%End of II.5.2-----------------------------------------------------------------------------------------------
%End of Section II.5-----------------------------------------------------------------------------------------
%End of Chapter II-------------------------------------------------------------------------------------------
%%Chapter III-------------------------------------------------------------------------------------------------
{\LARGE \bfseries III \\ \\ Semantics of First-Order Languages}
\\
\\
\\
%Section III.1-----------------------------------------------------------------------------------------------
{\large \S1. Structures and Interpretations}
\begin{enumerate}[1.]
%III.1.4-----------------------------------------------------------------------------------------------------
\item \textbf{Solution to Exercise 1.4.}
\begin{enumerate}[(a)]
\item ``There is a natural number such that the sum taken on it once equals 2.''
%%
\item ``There is a natural number such that the multiplication on it once equals 2.''
%%
\item ``There is a natural number equal to 0.''
%%
\item ``For any natural number there is a natural number equal to it.''
%%
\item ``For any pair of natural numbers there is a natural number which is greater than one and smaller than the other.''\nolinebreak\hfill$\talloblong$
\end{enumerate}
%End of III.1.4----------------------------------------------------------------------------------------------
%
%III.1.5-----------------------------------------------------------------------------------------------------
\item \textbf{Solution to Exercise 1.5.} Let $\mbox{card}(A)$, $\mbox{num}_r(S)$, $\mbox{num}_f(S)$, and $\mbox{num}_c(S)$ denote the cardinality of $A$, the number of relation symbols in $S$, the number of function symbols in $S$, and the number of constant symbols in $S$, respectively. \\
\ 
\\And let $\mbox{dim}(R_i)$ and $\mbox{dim}(f_i)$ denote the dimension of relation symbol $R_i$, and the dimension of function symbol $f_i$, respectively.\\
\ 
\\Then the number of structures that can be defined on domain $A$ and symbol set $S$ is
\[
\left( \prod\limits_{i = 1}^{ \mbox{\scriptsize num}_r(S)} 2^{ \mbox{\scriptsize card}(A)^{ \mbox{\tiny dim}(R_i)}} \right) \cdot \left( \prod\limits_{i = 1}^{ \mbox{\scriptsize num}_f(S)} \mbox{card}(A)^{ \mbox{\scriptsize card}(A)^{ \mbox{\tiny dim}(f_i) } } \right) \cdot \mbox{card}(A)^{ \mbox{ \scriptsize num}_c(S) },
\]
which is finite.\nolinebreak\hfill$\talloblong$
%End of III.1.5----------------------------------------------------------------------------------------------
%
%III.1.6-----------------------------------------------------------------------------------------------------
\item \textbf{Solution to Exercise 1.6.}
We confine ourselves to the \textit{proof of} (c). It seems reasonable to assign $(0,0)$ and $(1,1)$ as the additive and multiplicative identities, respectively, since
\[
\forall (x_1, x_2) \in A \times B, (x_1, x_2) + (0, 0) = (x_1 + 0, x_2 + 0) = (x_1, x_2),
\]
and
\[
\forall (x_1, x_2) \in (A \times B) \setminus \{ (0,0) \}, (x_1, x_2) \cdot (1, 1) = (x_1 \cdot 1, x_2 \cdot 1) = (x_1, x_2).
\]
Therefore $(1,0) \not = (0,0)$ should have a multiplicative inverse element but actually it does not, since
\[
\forall (y_1, y_2) \in (A \times B) \setminus \{ (0,0) \}, (1,0) \cdot (y_1, y_2) = (y_1, 0) \neq (1,1).
\]
\begin{flushright}$\talloblong$\end{flushright}
%End of III.1.6----------------------------------------------------------------------------------------------
\end{enumerate}
%End of Section III.1----------------------------------------------------------------------------------------
\ 
\\
\\
%Section III.2-----------------------------------------------------------------------------------------------
{\large \S2. Standardization of Connectives}
\begin{enumerate}[1.]
%III.2.1-----------------------------------------------------------------------------------------------------
\item \textbf{Solution to Exercise 2.1.} We restrict ourselves to the subproblem (a).
\begin{tabular}{cc|c|c|c}
$x$ & $y$ & $\stackrel{.}{\neg}(x)$ & $\stackrel{.}{\rightarrow}(x, y)$ & $\stackrel{.}{\lor}(\stackrel{.}{\neg}(x), y)$ \\ \hline
$T$ & $T$ & $F$ & $T$ & $T$ \\
$T$ & $F$ & $F$ & $F$ & $F$ \\
$F$ & $T$ & $T$ & $T$ & $T$ \\
$F$ & $F$ & $T$ & $T$ & $T$
\end{tabular}
\ 
\\\begin{flushright}$\talloblong$\end{flushright}
\end{enumerate}
%End of III.2.1----------------------------------------------------------------------------------------------
%End of Section III.2----------------------------------------------------------------------------------------
\ 
\\
\\
%Section III.3-----------------------------------------------------------------------------------------------
{\large \S3. The Satisfaction Relation}
\begin{enumerate}[1.]
\item \textbf{Note to Page 33.}
\[
\begin{array}{ll}
\, & \mathfrak{I} \frac{r}{v_0} \models v_0 \circ e \equiv v_0 \\
\Iff & \mathfrak{I} \frac{r}{v_0} (v_0 \circ e) = \mathfrak{I} \frac{r}{v_0} (v_0) \\
\Iff & +(\mathfrak{I} \frac{r}{v_0} (v_0), \mathfrak{I} \frac{r}{v_0} (e) ) = \mathfrak{I} \frac{r}{v_0} (v_0) \\
\Iff & r + 0 = r.
\end{array}
\]
%
%III.3.3----------------------------------------------------------------------------------------------
\item \textbf{Solution to Exercise 3.3.}
\\
\\
\begin{tabular}{lll} \hline
\multicolumn{1}{c}{\sc formula} & \multicolumn{2}{c}{\sc interpretation} \\ \hline
\  & \  & \  \\
\  & \textbf{satisfying} & \textbf{not satisfying} \\
\  & \  & \  \\
$\forall v_1 \, fv_0 v_1 \equiv v_0$ & $\mathfrak{I} = (\mathbb{N}, \cdot, <)$, & $\mathfrak{I} = (\mathbb{N}, +, <)$, \\
\  & $\mathfrak{I}(v_i) = i$. & $\mathfrak{I}(v_i) = i$. \\
$\exists v_0 \forall v_1  \, fv_0 v_1 \equiv v_1$ & $\mathfrak{I} = (\mathbb{N}, +, <)$, & $\mathfrak{I} = (\mathbb{Z}^+, +, <)$, \\
\  & $\mathfrak{I}(v_i) = i$. & $\mathfrak{I}(v_i) = i$. \\
$\exists v_0 (Pv_0 \land \forall v_1 \, Pfv_0 v_1)$ & $\mathfrak{I} = (\mathbb{N}, \cdot, \mbox{ is even})$, & $\mathfrak{I} = (\mathbb{Z}^+, +, \mbox{ is even})$, \\
\  & $\mathfrak{I}(v_i) = i$. & $\mathfrak{I}(v_i) = i+1$. \\ \hline
\end{tabular}
\\\begin{flushright}$\talloblong$\end{flushright}
%End of III.3.3----------------------------------------------------------------------------------------------
%
%III.3.4-----------------------------------------------------------------------------------------------------
\item \textbf{Solution to Exercise 3.4.} Consider the following rules for the calculus of \textit{positive} formulas:
\[
\begin{array}{ll}
\displaystyle \frac{\ }{t_1 \equiv t_2}; & \displaystyle \frac{\ }{Rt_1 \ldots t_n}; \\
\  & \  \\
\displaystyle \frac{ \displaystyle {\varphi, \; \psi} }{(\varphi \ast \psi)} \mbox{ for } \ast = \{ \land, \lor \}; & \  \\
\  & \  \\
\displaystyle \frac{\varphi}{\forall x \varphi}; & \displaystyle \frac{\varphi}{\exists x \varphi};
\end{array}
\]
Let $A = \{ \alpha \}$. And let $\mathfrak{I} = (\mathfrak{A}, \beta)$, where $\mathfrak{A}$ is a structure with domain $A$ and 
\[
\mbox{for all terms } t, \mathfrak{I}(t) = \alpha ; 
\]
\[
\mbox{for all $n$-ary relation symbols } R, R^\mathfrak{A} = \{ (\overbrace{\alpha, \ldots, \alpha}^{\mbox{\scriptsize $n$-terms}}) \}.
\]
We show by induction on \textit{positive} formulas that every positive $S$-formula is satisfied by $\mathfrak{I}$.\\
\\$t_1 \equiv t_2$: $\mathfrak{I} \models t_1 \equiv t_2$ since $\mathfrak{I}(t_1) = \alpha = \mathfrak{I}(t_2)$.\\
\\$Rt_1 \ldots t_n$: $\mathfrak{I} \models Rt_1 \ldots t_n$ since $R^\mathfrak{A} \mathfrak{I}(t_1) \ldots \mathfrak{I}(t_n)$, i.e., $R^\mathfrak{A} \overbrace{\alpha \ldots \alpha}^{\mbox{\scriptsize $n$-terms}}$.\\
\\$(\varphi \ast \psi)$, for $\ast = \land, \lor$: Suppose $\mathfrak{I} \models \varphi$ and $\mathfrak{I} \models \psi$ by induction hypothesis, which implies that \textit{at least} $\mathfrak{I} \models \varphi$ \textit{or} $\mathfrak{I} \models \psi$. Then immediately $\mathfrak{I} \models (\varphi \ast \psi)$.\\
\\$\forall x \varphi$: Suppose $\mathfrak{I} \models \varphi$ by induction hypothesis. Then since for all $a \in A$, $a = \alpha$, and for all variables $x$, $\mathfrak{I}(x) = \alpha$, it turns out that for all $a \in A$, $\mathfrak{I} \frac{a}{x} = \mathfrak{I} \frac{\alpha}{x} = \mathfrak{I}$, and $\mathfrak{I} \frac{a}{x} \models \varphi$. Therefore $\mathfrak{I} \models \forall x \varphi$.\\
\\$\exists x \varphi$: Suppose $\mathfrak{I} \models \varphi$ by induction hypothesis. Then since for all variables $x$, $\mathfrak{I}(x) = \alpha$, it turns out that $\mathfrak{I} \frac{\alpha}{x} = \mathfrak{I}$, and $\mathfrak{I} \frac{\alpha}{x} \models \varphi$. Hence $\mathfrak{I} \models \exists x \varphi$.\nolinebreak\hfill$\talloblong$
%End of III.3.4------------------------------------------------------------------------------------
\end{enumerate}
%End of Section III.3----------------------------------------------------------------------------------------
\ 
\\
\\
%Section III.4-----------------------------------------------------------------------------------------------
{\large \S4. The Consequence Relation}
\begin{enumerate}[1.]
\item \textbf{Note to Definition of the Consequence Relation 4.1.} The reader may have noticed something interesting: In Defintion of the Satisfaction Relation 3.2, we say that an interpretation $\mathfrak{I} = (\mathfrak{A}, \beta)$ satisfies $\exists x \varphi$ iff:
\begin{center}
there is an $a \in A$ such that $\mathfrak{I}\frac{a}{x}$ satisfies $\varphi$.
\end{center}
Nevertheless, we do not know whether $\mathfrak{I}$ itself is a model of $\varphi$. The condition for $\mathfrak{I}$ to satisfy $\exists x \varphi$ is that the interpretation $\mathfrak{I}\frac{a}{x}$ (possibly different from $\mathfrak{I}$) be a model of $\varphi$!\newline
\\
Let us say $\Phi = \{x \equiv c, \ \neg c \equiv d \}$, $\varphi = \neg x \equiv c$, and $\mathfrak{I} = (\mathfrak{A}, \beta)$, where $A$ consists of two elements $c^A$ and $d^A$ with $c^A \neq d^A$ and $\beta(x) = c^A$.\newline
\\
We see that $\mathfrak{I} \models \Phi$ and $\mathfrak{I} \models \exists x \varphi$. In fact, $\Phi \models \exists x \varphi$. Let us analyze this a little more: $\mathfrak{I}$ directly satisfies $\Phi$. However, the reason for $\mathfrak{I}$ to satisfy $\exists x \varphi$ is that a different interpretation $\mathfrak{I}\frac{d^A}{x}$ be a model of $\varphi$. Does $\mathfrak{I}\frac{d^A}{x}$ satisfy $\Phi$? No.
%
\item \textbf{Note to the Discussions about Negations on Page 34.} Note that, in contrast, $\mathfrak{I} \models \neg \varphi$ iff not $\mathfrak{I} \models \varphi$.
%
\item \textbf{Note to the Remark below Definition 4.2.} We show that for every formula $\varphi$, $\emptyset \models \varphi$ if and only if every interpretation $\mathfrak{I}$ is a model of $\varphi$: Suppose $\emptyset \models \varphi$ holds. Since for every interpretation $\mathfrak{I}$, $\mathfrak{I} \models \emptyset$, by assumption we have that every interpretation $\mathfrak{I}$ is a model of $\varphi$.\\
\ \\
Conversely, suppose that every interpretation $\mathfrak{I}$ is a model of $\varphi$. Then obviously for every interpretation $\mathfrak{I}$, \emph{if $\mathfrak{I} \models \emptyset$ then $\mathfrak{I} \models \varphi$,} hence we have $\emptyset \models \varphi$.
%
\item \textbf{Note to the Proof of Lemma 4.4.} \textit{There is no interpretation which is a model of $\Phi$ but not a model of $\varphi$ iff there is no interpretation which is a model of $\Phi$ and of $\neg \varphi$ (since $\mathfrak{I} \models \neg \varphi$ iff not $\mathfrak{I} \models \varphi$) iff there is no interpretation which is a model of $\Phi \cup \{ \neg \varphi \}$ (cf. 4.1)}.
%
\item \textbf{Proposition}: \textit{$\forall x (\varphi \leftrightarrow \psi)$ logically implies $\exists x \varphi \leftrightarrow \exists x \psi$.}\\
\textit{Proof.}
\[
\begin{array}{ll}
\,  & \forall x (\varphi \rightarrow \psi) \rightarrow (\exists x \varphi \rightarrow \exists x \psi) \\
= & \forall x (\varphi \rightarrow \psi) \rightarrow (\neg \forall x \neg \varphi \rightarrow \neg \forall x \neg \psi) \\
= & \forall x (\varphi \rightarrow \psi) \rightarrow (\forall x \neg \psi \rightarrow \forall x \neg \varphi) \\
= & (\forall x (\varphi \rightarrow \psi) \land \forall x \neg \psi) \rightarrow \forall x \neg \varphi \mbox{  (since $\varphi \rightarrow (\psi \rightarrow \chi) = (\varphi \land \psi) \rightarrow \chi$)} \\
= & \forall x ( (\varphi \rightarrow \psi) \land \neg \psi) \rightarrow \forall x \neg \varphi \\
= & \forall x ( \neg \varphi \land \neg \psi ) \rightarrow \forall x \neg \varphi \\
= & ( \forall x \neg \varphi \land \forall x \neg \psi ) \rightarrow \forall x \neg \varphi \\
= & \neg ( \forall x \neg \varphi \land \forall x \neg \psi ) \lor \forall x \neg \varphi \\
= & (\neg \forall x \neg \varphi \lor \neg \forall x \neg \psi ) \lor \forall x \neg \varphi \\
= & (\neg \forall x \neg \varphi \lor \forall x \neg \varphi ) \lor \neg \forall x \neg \psi \\
= & \mbox{\bf true} \lor \neg \forall x \neg \psi \\
= & \mbox{\bf true}.
\end{array}
\]
Similarly, $\forall x (\psi \rightarrow \varphi) \rightarrow (\exists x \psi \rightarrow \exists x \varphi)$.\\
\\Therefore,
\[
\begin{array}{ll}
\,  & \forall x (\varphi \leftrightarrow \psi) \\
=  & \forall x ( (\varphi \rightarrow \psi) \land (\psi \rightarrow \varphi) ) \\
=  & \forall x (\varphi \rightarrow \psi) \land \forall x (\psi \rightarrow \varphi)
\end{array}
\]
logically implies
\[
\begin{array}{ll}
\,  & (\exists x \varphi \rightarrow \exists x \psi) \land (\exists x \psi \rightarrow \exists x \varphi) \\
= & \exists x \varphi \leftrightarrow \exists x \psi.
\end{array}
\]
(Note that $( (\varphi \rightarrow \psi) \land (\chi \rightarrow \eta) ) \rightarrow ( (\varphi \land \chi) \rightarrow (\psi \land \eta) ) = \mbox{\bf true}$.)\nolinebreak\hfill$\talloblong$
%
\item \textbf{Note to Lemma 4.6 (b).} In the case that $\varphi = \exists x \psi$, we have for all pairs of interpretations $\mathfrak{I}_1$ and $\mathfrak{I}_2$ that agree on the $S$-symbols and on the variables occurring free in $\psi$, $\mathfrak{I}_1 \models \psi$ iff $\mathfrak{I}_2 \models \psi$, by induction hypothesis. This implies that for all $a \in A_1 = A_2$, $\mathfrak{I}_1 \frac{a}{x} \models \psi$ iff $\mathfrak{I}_2 \frac{a}{x} \models \psi$ under the same assumption. And by the last note, this in turn implies that there is an $a \in A_1 (= A_2)$ such that $\mathfrak{I}_1 \frac{a}{x} \models \psi$ iff there is an $a \in A_2 (= A_1)$ such that $\mathfrak{I}_2 \frac{a}{x} \models \psi$, under the same assumption. Actually the assumption holds (see the explanation on page 37 in textbook), and the result follows.
%
%III.4.9-----------------------------------------------------------------------------------------------------
\item \textbf{Solution to Exercise 4.9.}
\begin{enumerate}[(a)]
\item 
\[
\begin{array}{ll}
\,  & (\varphi \lor \psi) \models \chi \\
\Iff & \mbox{for every interpretation $\mathfrak{I}$ such that $\mathfrak{I} \models (\varphi \lor \psi)$, $\mathfrak{I} \models \chi$} \\
\Iff & \mbox{for every interpretation $\mathfrak{I}$ such that $\mathfrak{I} \models \varphi$ or $\mathfrak{I} \models \psi$, $\mathfrak{I} \models \chi$} \\
\Iff & \mbox{for every interpretation $\mathfrak{I}$ such that $\mathfrak{I} \models \varphi$, $\mathfrak{I} \models \chi$, and} \\
\,  & \mbox{for every interpretation $\mathfrak{I}$ such that $\mathfrak{I} \models \psi$, $\mathfrak{I} \models \chi$} \\
\,  & \mbox{(since $(\alpha \lor \beta) \rightarrow \gamma = (\alpha \rightarrow \gamma) \land (\beta \rightarrow \gamma)$)} \\
\Iff & \varphi \models \chi \mbox{ and } \psi \models \chi.
\end{array}
\]
%%
\item 
\[
\begin{array}{ll}
\,  & \models (\varphi \rightarrow \psi) \\
\Iff & \mbox{for every interpretation $\mathfrak{I}$ if $\mathfrak{I} \models \varphi$ then $\mathfrak{I} \models \psi$} \\
\Iff & \mbox{for every interpretation $\mathfrak{I}$ such that $\mathfrak{I} \models \varphi$, $\mathfrak{I} \models \psi$ } \\
\Iff & \varphi \models \psi
\end{array}
\]
\end{enumerate} \begin{flushright}$\talloblong$\end{flushright}
%End of III.4.9----------------------------------------------------------------------------------------------
%
%III.4.10----------------------------------------------------------------------------------------------------
\item \textbf{Solution to Exercise 4.10.}
\begin{enumerate}[(a)]
\item By 4.9 (b), it is equivalent to show that: $\models (\exists x \forall y \varphi \rightarrow \forall y \exists x \varphi)$.\\
\ 
\\For every interpretation $\mathfrak{I}$, \\
\[
\begin{array}{ll}
\,  & \mathfrak{I} \models \exists x \forall y \varphi \\
\Iff & \mbox{there is an $a \in \domain{\mathfrak{I}}$ such that $\mathfrak{I} \frac{a}{x} \models \forall y \varphi$} \\
\Iff & \mbox{there is an $a \in \domain{\mathfrak{I}}$ such that} \\
\,  & \mbox{for all $b \in \domain{\mathfrak{I}}$, $(\mathfrak{I} \frac{a}{x}) \frac{b}{y} \models \varphi$ (since $\domain{\mathfrak{I}}$} \\
\,  & \mbox{coincides with that of $\mathfrak{I} \frac{a}{x}$)} \\
\mbox{\it then} & \mbox{there is an $a \in \domain{\mathfrak{I}}$ such that } \\
\,  & \mbox{for all $b \in \domain{\mathfrak{I}}$, \it there is an \rm $a \in \domain{\mathfrak{I}}$} \\
\,  & \mbox{\it such that } ((\mathfrak{I} \frac{a}{x}) \frac{b}{y}) \frac{a}{x} \models \varphi \mbox{\ \ (since $\mathfrak{I} \models \varphi$ entails} \\
\,  & \mbox{\it there is an \rm $a \in \domain{\mathfrak{I}}$ such that $\mathfrak{I} \frac{a}{x} \models \varphi$)} \\
\Iff & \mbox{there is an $a \in \domain{\mathfrak{I}}$ such that } \\
\,  & \mbox{for all $b \in \domain{\mathfrak{I}}$, there is an $a \in \domain{\mathfrak{I}}$} \\
\,  & \mbox{such that } (\mathfrak{I} \frac{b}{y}) \frac{a}{x} \models \varphi \mbox{\ \ (since $((\mathfrak{I} \frac{a}{x}) \frac{b}{y}) \frac{a}{x} = (\mathfrak{I} \frac{b}{y}) \frac{a}{x}$)} \\
\Iff & \mbox{there is an $a \in \domain{\mathfrak{I}}$ such that} \\
\,  & \mbox{for all $b \in \domain{\mathfrak{I}}$, $\mathfrak{I} \frac{b}{y} \models \exists x \varphi$} \\
\Iff & \mbox{there is an $a \in \domain{\mathfrak{I}}$ such that $\mathfrak{I} \models \forall y \exists x \varphi$} \\
\Iff & \mathfrak{I} \models \forall y \exists x \varphi \mbox{\ \ (since the satisfaction $\mathfrak{I} \models \forall y \exists x \varphi$ has nothing to} \\
\,  & \mbox{do with the existence of $a$)}. 
\end{array}
\]
The proof is complete.
%%
\item Let $\varphi \in L^{S_{\mbox{\scriptsize gr}}}$, $\varphi = y \circ x \equiv e$, and let $\mathfrak{I} = (\mathfrak{A}, \beta)$ be an interpretation for $\varphi$, where $\mathfrak{A} = (\mathbb{R}, +, 0)$ and $\beta$ is arbitrarily defined.\\
\ 
\\If $\mathfrak{I} \models \exists x \forall y ( y \circ x \equiv e )$, i.e., there is an $a \in \domain{\mathfrak{I}}$ such that for all $b \in \domain{\mathfrak{I}}$, $( \mathfrak{I} \frac{a}{x} ) \frac{b}{y} \models y \circ x \equiv e$, or
\[
b + a = 0.
\]
Then $b = 1-a$ must also satisfy this formula, i.e.,
\[
(1-a)+a=1=0,
\]
a contradiction.\nolinebreak\hfill$\talloblong$
\end{enumerate}
\ \newline
\textit{Remark.} We provide below a matrix visualization of interpretations satisfying formulas preceded by both quantifiers $\forall$ and $\exists$. In our interpretations, the domain consists of 5 elements: 0, 1, 2, 3, 4. The variable $x$ takes values from rows, whereas $y$ from columns. A $\bullet$ (or $\circ$) positioned at row $x_0$ and column $y_0$ means that $\varphi$ holds (or does not hold, respectively) when $x$ takes value $x_0$ and $y$ takes value $y_0$.
\begin{enumerate}
\item $\forall x \forall y \varphi$.
\[
\begin{array}{cc|ccccc}
\multicolumn{4}{c}{\ } & y & \ & \ \cr
\multicolumn{2}{c}{\ } & 0 & 1 & 2 & 3 & 4 \cr\cline{3-7}
\ & 0 & \bullet & \bullet & \bullet & \bullet & \bullet \cr
\ & 1 & \bullet & \bullet & \bullet & \bullet & \bullet \cr
x & 2 & \bullet & \bullet & \bullet & \bullet & \bullet \cr
\ & 3 & \bullet & \bullet & \bullet & \bullet & \bullet \cr
\ & 4 & \bullet & \bullet & \bullet & \bullet & \bullet
\end{array}
\]
``there are $\bullet$ in all positions.''
%%
\item $\forall x \exists y \varphi$.
\[
\begin{array}{cc|ccccc}
\multicolumn{4}{c}{\ } & y & \ & \ \cr
\multicolumn{2}{c}{\ } & 0 & 1 & 2 & 3 & 4 \cr\cline{3-7}
\ & 0 & \circ & \bullet & \circ & \circ & \circ \cr
\ & 1 & \circ & \circ & \bullet & \circ & \circ \cr
x & 2 & \bullet & \circ & \circ & \circ & \circ \cr
\ & 3 & \circ & \bullet & \circ & \circ & \bullet \cr
\ & 4 & \bullet & \circ & \circ & \circ & \circ
\end{array}
\]
``there is at least one position containing a $\bullet$ in each row.''
%%
\item $\exists x \exists y \varphi$.
\[
\begin{array}{cc|ccccc}
\multicolumn{4}{c}{\ } & y & \ & \ \cr
\multicolumn{2}{c}{\ } & 0 & 1 & 2 & 3 & 4 \cr\cline{3-7}
\ & 0 & \circ & \circ & \circ & \circ & \circ \cr
\ & 1 & \circ & \circ & \bullet & \circ & \circ \cr
x & 2 & \circ & \circ & \circ & \circ & \circ \cr
\ & 3 & \circ & \circ & \circ & \bullet & \circ \cr
\ & 4 & \circ & \circ & \circ & \circ & \circ
\end{array}
\]
``there is at least one $\bullet$, positioned at some row and some column.''
%%
\item $\exists x \forall y \varphi$.
\[
\begin{array}{cc|ccccc}
\multicolumn{4}{c}{\ } & y & \ & \ \cr
\multicolumn{2}{c}{\ } & 0 & 1 & 2 & 3 & 4 \cr\cline{3-7}
\ & 0 & \bullet & \bullet & \bullet & \bullet & \bullet \cr
\ & 1 & \circ & \circ & \bullet & \circ & \bullet \cr
x & 2 & \bullet & \circ & \circ & \circ & \bullet \cr
\ & 3 & \bullet & \bullet & \bullet & \bullet & \bullet \cr
\ & 4 & \circ & \circ & \circ & \bullet & \circ
\end{array}
\]
``there is at least one row in which there are $\bullet$ in all positions.''
\end{enumerate}
This virtualization should make the sense more concrete.
%End of III.4.10-----------------------------------------------------------------------------------
%
\item \textbf{Proposition}: \textit{For every formula $\varphi$, $\varphi \bimodels \neg \neg \varphi$.}\\
\textit{Proof.} First note that, for every interpretation $\mathfrak{I}$ (appropriate to $\varphi$),
\[
\mbox{either $\mathfrak{I} \models \varphi$ or not $\mathfrak{I} \models \varphi$},
\]
or equivalently,
\[
\mbox{either $\mathfrak{I} \models \varphi$ or $\mathfrak{I} \models \neg \varphi$ \ (by definition).}
\]
Therefore, for every interpretation $\mathfrak{I}$,
\[
\begin{array}{ll}
\,  & \mathfrak{I} \models \varphi \\
\mbox{iff} & \mbox{not $\mathfrak{I} \models \neg \varphi$ \ (by the observation above)} \\
\mbox{iff} & \mathfrak{I} \models \neg \neg \varphi \mbox{ \ (take $\neg \varphi$ as some $\psi$)}.
\end{array}
\]
\begin{flushright}$\talloblong$\end{flushright}
%
\item \textbf{Proposition:} \textit{If $x \not \in \free(\varphi)$, then $\varphi \bimodels \forall x \varphi$.}
\\
\textit{Proof.} First note that $\forall x \varphi \models \varphi$ clearly holds. (A property that holds for all elements in the domain must also hold for some particular element.) Hence we only need to show that $\varphi \models \forall x \varphi$.\\
\\
For every interpretation $\mathfrak{I} = (\mathfrak{A}, \beta)$, $\mathfrak{I} \models \varphi$ implies that $\mathfrak{I} \frac{a}{x}$ for any $a \in \mathfrak{A}$ since $x \not \in \free(\varphi)$, i.e. $\mathfrak{I} \models \forall x \varphi$. The proof is complete.\nolinebreak\hfill$\talloblong$
%
%III.4.11----------------------------------------------------------------------------------------------------
\item \textbf{Solution to Exercise 4.11.}
\begin{enumerate}[(a)]
\item \label{item_a} For every interpretation $\mathfrak{I}$,
\[
\begin{array}{ll}
\,  & \mathfrak{I} \models \forall x ( \varphi \land \psi ) \\
\Iff & \mbox{for all $a \in \domain{\mathfrak{I}}$, $\mathfrak{I} \frac{a}{x} \models \varphi \mbox{ and } \mathfrak{I} \frac{a}{x} \models \psi$} \\
\Iff & \mbox{for all $a \in \domain{\mathfrak{I}}$, $\mathfrak{I} \frac{a}{x} \models \varphi$ and,} \\
\Iff & \mbox{for all $a \in \domain{\mathfrak{I}}$, $\mathfrak{I} \frac{a}{x} \models \psi$} \\
\Iff & \mathfrak{I} \models \forall x \varphi \mbox{ and } \mathfrak{I} \models \forall x \psi \\
\Iff & \mathfrak{I} \models \forall x \varphi \land \forall x \psi.
\end{array}
\]
%%
\item Similar to (a).
%%
\item
\[
\begin{array}{ll}
\         & \forall x (\varphi \lor \psi) \\
\bimodels & ((\forall x (\varphi \lor \psi) \land \varphi) \lor (\forall x (\varphi \lor \psi) \land \neg \varphi)) \\
\bimodels & ((\forall x (\varphi \lor \psi) \land \forall x \varphi) \lor (\forall x (\varphi \lor \psi) \land \forall x \neg \varphi)) \\
\         & \mbox{\ \ \ (since $x \not \in \free(\varphi)$ and by the above conjecture)} \\
\bimodels & (\forall x ((\varphi \lor \psi) \land \varphi) \lor \forall x ((\varphi \lor \psi) \land \neg \varphi)) \mbox{\ \ \ (by (a))} \\
\bimodels & (\forall x \varphi \lor \forall x (\neg \varphi \land \psi)) \\
\bimodels & (\forall x \varphi \lor (\forall x \neg \varphi \land \forall x \psi)) \mbox{\ \ \ (by (a))} \\
\bimodels & (\varphi \lor (\neg \varphi \land \forall x \psi)) \mbox{\ \ \ (by the above conjecture)}\\
\bimodels & ((\varphi \lor \neg \varphi) \land (\varphi \lor \forall x \psi)) \\
\bimodels & ((\varphi \land (\varphi \lor \forall x \psi)) \lor (\neg \varphi \land (\varphi \lor \forall x \psi))) \\
\bimodels & (\varphi \lor \forall x \psi)
\end{array}
\]
%%
\item Similar to (c).
%%
\item Consider the case for $S_{\mbox{\scriptsize ar}}$: Let $\varphi = x \equiv 0$ and $\varphi = \neg \psi$. Next let $\mathfrak{I} = ( \mathfrak{A}, \beta )$ where $\mathfrak{A} = ( \mathbb{N}, +, \cdot, 0, 1 )$ and $\beta$ is arbitrarily defined. It is clear that
\[
\forall x ( \varphi \lor \psi ) \not \models ( \forall x \varphi \lor \forall x \psi ),
\]
and
\[
( \exists x \varphi \land \exists x \psi ) \not \models \exists x ( \varphi \land \psi ).
\]
\end{enumerate} \begin{flushright}$\talloblong$\end{flushright}
%End of III.4.11---------------------------------------------------------------------------------------------
%
\item \textbf{Note to Exercise 4.11 (e).} Actually, $( \forall x \varphi \lor \forall x \psi ) \models \forall x ( \varphi \lor \psi )$ and $\exists x ( \varphi \land \psi ) \models ( \exists x \varphi \land \exists x \psi )$, since $\models (( \forall x \varphi \lor \forall x \psi ) \rightarrow \forall x ( \varphi \lor \psi ))$ and $\models ( \exists x ( \varphi \land \psi ) \rightarrow ( \exists x \varphi \land \exists x \psi ) )$.
%
%III.4.12----------------------------------------------------------------------------------------------------
\item \textbf{Solution to Exercise 4.12.}
\begin{enumerate}[(a)]
\item 
\[
\begin{array}{lll}
\varphi^\prime & := & \psi \\
\chi^\prime & := & \chi \mbox{ if $\chi$ is atomic and $\chi \not = \varphi$ } \\
\chi^\prime & := & \neg \eta^\prime \mbox{ if $\chi = \neg \eta$ and $\chi \not = \varphi$} \\
\chi^\prime & := & ( \chi_1^\prime \lor \chi_2^\prime ) \mbox{ if $\chi = ( \chi_1 \lor \chi_2 )$ and $\chi \not = \varphi$} \\
\chi^\prime & := & \exists x \eta^\prime \mbox{ if $\chi = \exists x \eta$ and $\chi \not = \varphi$ }.
\end{array}
\]
%%
\item We show by induction on formulas:\\
\ 
\\If $\chi = \varphi$, then $\chi^\prime = \psi$, and $\chi \bimodels \chi^\prime$ by hypothesis; Otherwise, \\
\ 
\\$\chi$ is atomic: $\chi^\prime = \chi$. Clearly $\chi \bimodels \chi^\prime$.\\
\ 
\\$\chi = \neg \eta$: $\chi^\prime = \neg \eta^\prime$. Thus, for every interpretation $\mathfrak{I}$,
\[
\begin{array}{ll}
\, & \mathfrak{I} \models \chi \\
\Iff & \mbox{not $\mathfrak{I} \models \eta$} \\
\Iff & \mbox{not $\mathfrak{I} \models \eta^\prime$ (by induction hypothesis)} \\
\Iff & \mathfrak{I} \models \chi^\prime .
\end{array}
\]
\ 
\\$\chi = ( \chi_1 \lor \chi_2 )$: $\chi^\prime = ( \chi_1^\prime \lor \chi_2^\prime )$. Thus, for every interpretation $\mathfrak{I}$,
\[
\begin{array}{ll}
\,  & \mathfrak{I} \models \chi \\
\Iff & \mbox{$\mathfrak{I} \models \chi_1$ or $\mathfrak{I} \models \chi_2$} \\
\Iff & \mbox{$\mathfrak{I} \models \chi_1^\prime$ or $\mathfrak{I} \models \chi_2^\prime$} \\
\,  & \mbox{(by induction hypothesis)} \\
\Iff & \mathfrak{I} \models \chi^\prime.
\end{array}
\]
\ 
\\$\chi = \exists x \eta$: $\chi^\prime = \exists x \eta^\prime$. Thus, for every interpretation $\mathfrak{I}$,
\[
\begin{array}{ll}
\,  & \mathfrak{I} \models \exists x \eta \\
\Iff & \mbox{there is an $a \in \domain{\mathfrak{I}}$ such that $\mathfrak{I} \frac{a}{x} \models \eta$} \\
\Iff & \mbox{there is an $a \in \domain{\mathfrak{I}}$ such that $\mathfrak{I} \frac{a}{x} \models \eta^\prime$} \\
\,  & \mbox{(by induction hypothesis)} \\
\Iff & \mathfrak{I} \models \exists x \eta^\prime.
\end{array}
\]
\end{enumerate} \begin{flushright}$\talloblong$\end{flushright}
%End of III.4.12---------------------------------------------------------------------------------------------
%
%III.4.13------------------------------------------------------------------------------------------
\item \textbf{Solution to Exercise 4.13.} The proof is the same as that of 4.8, except with a slight modification: In both directions, statements begin with ``For every interpretation $\mathfrak{I}$ ($\mathfrak{I}^\prime$),''.\nolinebreak\hfill$\talloblong$
%End of III.4.13-----------------------------------------------------------------------------------
%
%III.4.14------------------------------------------------------------------------------------------
\item \textbf{Solution to Exercise 4.14.} For every absence of formulas, we give an example as a \textit{witness} to $\Phi \setminus \{ \varphi \} \not \models \varphi$.\\
\ 
\\For group theory:
\begin{enumerate}[(i)]
\item Absence of $\forall v_0 \forall v_1 \forall v_2 ( v_0 \circ v_1 ) \circ v_2 \equiv v_0 \circ ( v_1 \circ v_2 )$: The structure $( \mathbb{N}, \dist, 0 )$, where $\dist(m, n) = |m - n|$.
%%
\item Absence of $\forall v_0 \, v_0 \circ e \equiv v_0$: The structure $(A, \cap, \emptyset)$, \\where $A = \{ \emptyset, \{ 0 \}, \{ 1 \}, \{ 0, 1 \} \}$.
%%
\item Absence of $\forall v_0 \exists v_1 \, v_0 \circ v_1 \equiv e$: The structure $(\mathbb{N}, +, 0)$.
\end{enumerate}
\ 
\\For the theory of equivalence relations:
\begin{enumerate}[(i)]
\item Absence of $\forall v_0 \, Rv_0 v_0$: The structure $(\mathbb{N}, Z^+)$, where $Z^+(m,n)$ means that $m \cdot n \in \mathbb{Z}^+$.
%%%
\item Absence of $\forall v_0 \forall v_1 (Rv_0v_1 \rightarrow Rv_1v_0)$: The structure $(\mathbb{N}, \leq)$.
%%%
\item Absence of $\forall v_0 \forall v_1 \forall v_2 ((Rv_0v_1 \land Rv_1v_2) \rightarrow Rv_0v_2)$: \\The structure $(A, \nme)$, where $A = \{ \{ 0 \}, \{ 1 \}, \{ 0, 1 \} \}$ and $\nme(v_0, v_1)$ means that $v_0$ and $v_1$ are \textit{not mutually exclusive}, i.e., $v_0 \cap v_1 \not = \emptyset$.\nolinebreak\hfill$\talloblong$
\end{enumerate}
\textit{Remark.} While we say that a set $\Phi$ of sentences is independent if there is no $\varphi \in \Phi$ such that $\Phi \setminus \{ \varphi \} \models \varphi$, we say that a formula is \emph{independent from} $\Phi$ if neither $\Phi \models \varphi$ nor $\Phi \models \neg\varphi$.\cite{Dirk_van_Dalen} Hence every valid formula is not independent from any $\Phi$; an unsatisfiable set $\Psi$ of sentences may be independent: For example, $\Psi := \{ \forall x \forall y \ x \equiv y, \ \neg\forall x \forall y \ x \equiv y \}$ is unsatisfiable and independent, while $\forall x \forall y \ x \equiv y$ (and $\neg\forall x \forall y \ x \equiv y$) is independent from $\Psi \setminus \{ \forall x \forall y \ x \equiv y \}$ (and $\Psi \setminus \{ \neg\forall x \forall y \ x \equiv y \}$, respectively).
%End of III.4.14---------------------------------------------------------------------------------------------
%
%III.4.15----------------------------------------------------------------------------------------------------
\item \textbf{Solution to Exercise 4.15.} We show by induction on terms: \\
\ 
\\$t = v_j$, $j \in \mathbb{N}$: $\var(t) = \{ t_j \} \subset \{ v_0, \ldots , v_j \}$. Thus,
\[
\begin{array}{ll}
\,  & t^{\mathfrak{A}} [ g_0, \ldots , g_j ] \\
= & v_j^{\mathfrak{A}} [ g_0, \ldots, g_j ] \\
= & g_j \\
= & \langle g_j (i) \, |\, i \in I \rangle \\
= & \langle v_j^{\mathfrak{A}_i} [ g_0(i), \ldots , g_j(i) \, |\, i \in I \rangle \\
= & \langle t^{\mathfrak{A}_i} [ g_0(i), \ldots , g_j(i) \, |\, i \in I \rangle .
\end{array}
\]
\ 
\\$t = c$: $\var(t) = \emptyset$. Therefore,
\[
\begin{array}{ll}
\,  & t^{\mathfrak{A}} \\
= & c^{\mathfrak{A}} \\
= & \langle c^{\mathfrak{A}_i} \, | \, i \in I \rangle \\
= & \langle t^{\mathfrak{A}_i} \, |\, i \in I \rangle .
\end{array}
\]
\ 
\\$t = ft_1 \ldots t_n$ ($f \in S$ is $n$-ary and $t_1, \ldots , t_n \in T^S$): Let $\var(t) \subset \{ v_0, \ldots , v_m \}$, $m \in \mathbb{N}$. Then
\[
\begin{array}{ll}
\,  & t^{\mathfrak{A}} [g_0, \ldots , g_m] \\
= & (ft_1 \ldots t_n)^{\mathfrak{A}} [g_0, \ldots , g_m] \\
= & f^{\mathfrak{A}} ( t_1^{\mathfrak{A}} [g_0, \ldots , g_m], \ldots , t_n^{\mathfrak{A}} [g_0, \ldots , g_m] ) \\
= & f^{\mathfrak{A}} ( \langle t_1^{\mathfrak{A}_i} [g_0(i), \ldots , g_m(i)] \, | \, i \in I \rangle , \ldots , \langle t_n^{\mathfrak{A}_i} [g_0(i), \ldots , g_m(i)] \, | \, i \in I \rangle ) \\
\,  & \mbox{(by induction hypothesis)} \\
= & \langle f^{\mathfrak{A}_i} ( t_1^{\mathfrak{A}_i} [g_0(i), \ldots , g_m(i)], \ldots , t_n^{\mathfrak{A}_i} [g_0(i), \ldots , g_m(i)] ) \, | \, i \in I \rangle \\
\,  & \mbox{(regard $\langle t_j^{\mathfrak{A}_i} [g_0(i), \ldots , g_m(i)] \, | \, i \in I \rangle$, where $1 \leq j \leq n$,} \\
\,  & \mbox{as some $g \in \prod_{i \in I} A_i$)} \\
= & \langle (ft_1 \ldots t_n)^{\mathfrak{A}_i} [g_0(i), \ldots , g_m(i)] \, | \, i \in I \rangle \\
= & \langle t^{\mathfrak{A}_i} [g_0(i), \ldots , g_m(i)] \, | \, i \in I \rangle .
\end{array}
\] \begin{flushright}$\talloblong$\end{flushright}
%End of III.4.15---------------------------------------------------------------------------------------------
%
%III.4.16----------------------------------------------------------------------------------------------------
\item \textbf{Solution to Exercise 4.16.} In general, if $\varphi$ is a Horn formula and if every $\mathfrak{I}_i$ is a model of $\varphi$ then $\prod_{i \in I} \mathfrak{A}_i \models \varphi [g_0, \ldots , g_n]$, where $\free(\varphi) \subset \{ v_0, \ldots , v_n \}$, $g_0, \ldots , g_n \in \prod_{i \in I} A_i$ and for $i \in I$ and $0 \leq j \leq n$, $\mathfrak{I}_i = (\mathfrak{A}_i, \beta_i)$ and $g_j(i) = \beta_i(v_j)$.\\
\ 
\\We show this by induction on formulas:\\
\ 
\\$\varphi = (\neg \varphi_1 \lor \ldots \lor \neg \varphi_n \lor \psi)$: Consider the following two disjoint subcases:
\begin{enumerate}[(i)]
\item There is a $j \in I$ such that $\mathfrak{I}_j \models \neg \varphi_m$, i.e., not $\mathfrak{I}_j \models \varphi_m$, for some $1 \leq m \leq n$: In this case, not $\prod_{i \in I} \mathfrak{A}_i \models \varphi_m [g_0, \ldots , g_n]$, since not all $\mathfrak{I}_i \models \varphi_m$, for at least $\mathfrak{I}_j \models \neg \varphi_m$. Therefore $\prod_{i \in I} \mathfrak{A}_i \models \neg \varphi_m [g_0, \ldots , g_n]$ and hence $\prod_{i \in I} \mathfrak{A}_i \models (\neg \varphi_1 \lor \ldots \lor \neg \varphi_n \lor \psi) [g_0, \ldots , g_n]$.\\
%%
\item For all $i \in I$ and for all $1 \leq m \leq n$, $\mathfrak{I}_i \models \varphi_m$, i.e., not $\mathfrak{I}_i \models \neg \varphi_m$: In this case, for every $\mathfrak{I}_i$, $\mathfrak{I}_i \models \psi$ since $\mathfrak{I}_i \models \varphi$ by premise. Clearly, $\prod_{i \in I} \mathfrak{A}_i \models \psi [g_0, \ldots , g_n]$ and hence $\prod_{i \in I} \mathfrak{A}_i \models (\neg \varphi_1 \lor \ldots \lor \neg \varphi_n \lor \psi) [g_0, \ldots , g_n]$.\\
\end{enumerate}
\ 
\\$\varphi = (\neg \varphi_0 \lor \ldots \lor \neg \varphi_n)$: Similar to subcase (i) in the above case.\\
\ 
\\$\varphi = (\psi \land \chi)$: For all $i \in I$, $\mathfrak{I}_i \models (\psi \land \chi)$
\[
\begin{array}{ll}
\mbox{iff} & \mbox{for all $i \in I$, $\mathfrak{I}_i \models \psi$ and for all $i \in I$, $\mathfrak{I}_i \models \chi$} \\
\mbox{then} & \mbox{$\prod_{i \in I} \mathfrak{A}_i \models \psi [g_0, \ldots , g_n]$ and $\prod_{i \in I} \mathfrak{A}_i \models \chi [g_0, \ldots , g_n]$} \\
\,  & \mbox{(by induction hypothesis)} \\
\mbox{iff} & \prod_{i \in I} \mathfrak{A}_i \models (\psi \land \chi) [g_0, \ldots , g_n].
\end{array}
\]
\ 
\\$\varphi = \forall x \psi$: For all $i \in I$, $\mathfrak{I}_i \models \forall x \psi$
\[
\begin{array}{ll}
\mbox{iff} & \mbox{for all $i \in I$ and for all $a \in A_i$, $\mathfrak{I}_i \frac{a}{x} \models \psi$} \\
\mbox{iff} & \mbox{for all $g \in \prod_{i \in I} A_i$ and for all $i \in I$, $\mathfrak{I}_i \frac{g(i)}{x} \models \psi$} \\
\mbox{then} & \mbox{for all $g \in \prod_{i \in I} A_i$, $\prod_{i \in I} \mathfrak{A}_i \models \varphi [g_0, \ldots , g_n, g]$}, \\
\,  & \mbox{where \textit{$x$ is mapped to $g$} (By induction hypothesis; $x$ may or} \\
\,  & \mbox{may not occur free in $\psi$. And it may or may not be among} \\
\,  & \mbox{$v_0, \ldots , v_n$. We write ``$[g_0, \ldots , g_n, g]$'' just to emphasize that} \\
\,  & \mbox{if $x$ occurs free in $\psi$ then it is mapped to $g$)} \\
\mbox{iff} & \prod_{i \in I} \mathfrak{A}_i \models \forall x \psi [g_0, \ldots , g_n].
\end{array}
\]
\ 
\\$\varphi = \exists x \psi$: For all $i \in I$, $\mathfrak{I}_i \models \exists x \psi$
\[
\begin{array}{ll}
\mbox{iff} & \mbox{for all $i \in I$, there is an $a \in A_i$ such that $\mathfrak{I}_i \frac{a}{x} \models \psi$} \\
\mbox{iff} & \mbox{there is a $g \in \prod_{i \in I} A_i$ such that for all $i \in I$, $\mathfrak{I}_i \frac{g(i)}{x} \models \psi$} \\
\mbox{then} & \mbox{there is a $g \in \prod_{i \in I} A_i$ such that $\prod_{i \in I} \mathfrak{A}_i \models \psi [g_0, \ldots , g_n, g]$,} \\
\,  & \mbox{where \textit{$x$ is mapped to $g$} (by induction hypothesis)}\\
\mbox{iff} & \prod_{i \in I} \mathfrak{A}_i \models \exists x \psi [g_0, \ldots , g_n].
\end{array}
\]
\ 
\\The result naturally follows from this.\nolinebreak\hfill$\talloblong$
%End of III.4.16---------------------------------------------------------------------------------------------
\end{enumerate}
%End of Section III.4----------------------------------------------------------------------------------------
\ 
\\
\\
%Section III.5-----------------------------------------------------------------------------------------------
{\large \S5. Two Lemmas on the Satisfaction Relation}
\begin{enumerate}[1.]
\item \textbf{Note to Lemma 5.2.} For every $S$-term $t$: $\pi(\mathfrak{I}(t)) = \mathfrak{I}^\pi(t)$.\\
\textit{Proof.} We show this by induction on terms:\\
\ 
\\$t = x$: $\pi(\mathfrak{I}(t)) = \pi(\beta(x)) = (\pi \circ \beta)(x) = \beta^\pi(x) = \mathfrak{I}^\pi(t)$.\\
\ 
\\$t = c$: $\pi(\mathfrak{I}(t)) = \pi(c^{\mathfrak{A}}) = c^{\mathfrak{B}} = \mathfrak{I}^\pi(t)$.\\
\ 
\\$t = ft_1 \ldots t_n$, where $f \in S$ is $n$-ary and $t_1 , \ldots , t_n \in T^S$:
\[
\begin{array}{ll}
\,  & \pi(\mathfrak{I}(t)) \\
= & \pi(f^{\mathfrak{A}} ( \mathfrak{I}(t_1) , \ldots , \mathfrak{I}(t_n) ) ) \\
= & f^{\mathfrak{B}}(\pi (\mathfrak{I}(t_1)) , \ldots , \pi (\mathfrak{I}(t_n)) ) \\
= & f^{\mathfrak{B}}( \mathfrak{I}^\pi(t_1) , \ldots , \mathfrak{I}^\pi(t_n) ) \\
\,  & \mbox{(by induction hypothesis)} \\
= & \mathfrak{I}^\pi ( ft_1 \ldots t_n ) \\
= & \mathfrak{I}^\pi (t) .
\end{array}
\] \begin{flushright}$\talloblong$\end{flushright}
%
\item \textbf{Note to Quantifier-Free Formulas in Page 42.} The following are the rules for the calculus of \textit{quantifier-free} formulas:
\[
\begin{array}{ll}
\displaystyle \frac{\,}{t_1 \equiv t_2}; & \displaystyle \frac{\,}{Rt_1 \ldots t_n} \mbox{ if $R \in S$ is $n$-ary}; \\
\,  & \,  \\
\displaystyle \frac{\varphi}{\neg \varphi}; & \displaystyle \frac{\varphi , \; \psi}{( \varphi \lor \psi )};
\end{array}
\]
%
\item \textbf{Note to Lemma 5.5.} The identity mapping preserves the properties listed in Definition 5.4, and hence the proof can easily be accomplished by slightly modifying the one of 5.2 Isomorphism Lemma. However, we prove this lemma below in the style as we did for others.\\
\ 
\\\textit{Proof.} First we show that for every $S$-terms $t$, $( \mathfrak{A}, \beta )(t) = ( \mathfrak{B}, \beta )(t)$:
\ 
\\
\\$t = x$: $( \mathfrak{A}, \beta )(t) = \beta (t) = ( \mathfrak{B}, \beta )(t)$.\\
\ 
\\$t = c$: $( \mathfrak{A}, \beta )(t) = c^{\mathfrak{A}} = c^{\mathfrak{B}} = ( \mathfrak{B}, \beta )(t)$ (since $\mathfrak{A} \subset \mathfrak{B}$).\\
\ 
\\$t = ft_1 \ldots t_n$, where $f \in S$ is $n$-ary and $t_1 \ldots t_n \in T^S$:
\[
\begin{array}{ll}
\,  & ( \mathfrak{A}, \beta )(t) \\
= & f^{\mathfrak{A}} ( ( \mathfrak{A}, \beta )(t_1), \ldots , ( \mathfrak{A}, \beta )(t_n) ) \\
= & f^{\mathfrak{B}} ( ( \mathfrak{A}, \beta )(t_1), \ldots , ( \mathfrak{A}, \beta )(t_n) ) \mbox{ (since $\mathfrak{A} \subset \mathfrak{B}$)} \\
= & f^{\mathfrak{B}} ( ( \mathfrak{B}, \beta )(t_1), \ldots , ( \mathfrak{B}, \beta )(t_n) ) \\
\,  & \mbox{(by induction hypothesis)}\\
= & ( \mathfrak{B}, \beta )(t).
\end{array}
\]
\ 
\\Next we show that for every quantier-free $S$-formula $\varphi$, $( \mathfrak{A}, \beta ) \models \varphi$ iff $( \mathfrak{B}, \beta ) \models \varphi$:\\
\ 
\\$\varphi = t_1 \equiv t_2$: $( \mathfrak{A}, \beta ) \models t_1 \equiv t_2$
\[
\begin{array}{ll}
\mbox{iff} & ( \mathfrak{A}, \beta )(t_1) = ( \mathfrak{A}, \beta )(t_2) \\
\mbox{iff} & ( \mathfrak{B}, \beta )(t_1) = ( \mathfrak{B}, \beta )(t_2) \mbox{ (since $( \mathfrak{A}, \beta )(t) = ( \mathfrak{B}, \beta )(t)$)} \\
\mbox{iff} & ( \mathfrak{B}, \beta ) \models t_1 \equiv t_2 .
\end{array}
\]
\ 
\\$\varphi = Rt_1 \ldots t_n$: $( \mathfrak{A}, \beta ) \models Rt_1 \ldots t_n$
\[
\begin{array}{ll}
\mbox{iff} & R^{\mathfrak{A}} ( \mathfrak{A}, \beta )(t_1) \ldots ( \mathfrak{A}, \beta )(t_n) \\
\mbox{iff} & R^{\mathfrak{B}} ( \mathfrak{A}, \beta )(t_1) \ldots ( \mathfrak{A}, \beta )(t_n) \mbox{ (since $\mathfrak{A} \subset \mathfrak{B}$)} \\
\mbox{iff} & R^{\mathfrak{B}} ( \mathfrak{B}, \beta )(t_1) \ldots ( \mathfrak{B}, \beta )(t_n) \mbox{ (since $( \mathfrak{A}, \beta )(t) = ( \mathfrak{B}, \beta )(t)$)} \\
\mbox{iff} & ( \mathfrak{B}, \beta ) \models Rt_1 \ldots t_n .
\end{array}
\]
\ 
\\$\varphi = \neg \psi$: $( \mathfrak{A}, \beta ) \models \neg \psi$: $( \mathfrak{A}, \beta ) \models \neg \psi$
\[
\begin{array}{ll}
\mbox{iff} & \mbox{not $( \mathfrak{A}, \beta ) \models \psi$}\\
\mbox{iff} & \mbox{not $( \mathfrak{B}, \beta ) \models \psi$ (by induction hypothesis)} \\
\mbox{iff} & ( \mathfrak{B}, \beta ) \models \neg \psi.
\end{array}
\]
\ 
\\$\varphi = ( \psi \lor \chi )$: $( \mathfrak{A}, \beta ) \models ( \psi \lor \chi )$
\[
\begin{array}{ll}
\mbox{iff} & ( \mathfrak{A}, \beta ) \models \psi \mbox{ or } ( \mathfrak{A}, \beta ) \models \chi \\
\mbox{iff} & ( \mathfrak{B}, \beta ) \models \psi \mbox{ or } ( \mathfrak{B}, \beta ) \models \chi \mbox{ (by induction hypothesis)}\\
\mbox{iff} & ( \mathfrak{B}, \beta ) \models ( \psi \lor \chi ).
\end{array}
\] \begin{flushright}$\talloblong$\end{flushright}
%
\item \textbf{Note to Definition 5.6.} By \reftitle{Theorem VIII.4.3 on the Disjunctive Normal Form} (or by \reftitle{Exercise VIII.4.7}, Theorem on the Conjunctive Normal Form), every universal formula is logically equivalent to a formula derivable from the following calculus, and vice versa:\medskip\\
\begin{tabular}{ll}
$\calrule{}{\varphi}$ if $\varphi$ is atomic or negated atomic; & $\calrule{\varphi, \psi}{(\varphi \ast \psi)}$ for $\ast = \land, \lor$; \cr
$\calrule{\varphi}{\forall x \varphi}$.
\end{tabular}
%
%III.5.9-----------------------------------------------------------------------------------------------------
\item \textbf{Solution to Exercise 5.9.} We prove this by giving a construction of such a sentence, $\varphi_{\mathfrak{A}}$.\\
\ 
\\First, observe that every structure isomorphic to $\mathfrak{A}$ must have its domain of the same size as $A$, and have its relations and functions the same as $\mathfrak{A}$, except possibly having its elements a permutation of those of $A$ or even completely different kind of elements.\\
\ 
\\Next, let the domain $A$ consist of $n+1$ elements, $a_0 , \ldots , a_n$, where $n \in \mathbb{N}$. Then the target sentence $\varphi_{\mathfrak{A}}$ is of the form:
\[
\exists v_0 \ldots \exists v_n \forall v_{n+1} \varphi .
\]
Intuitively, the variable $v_k$ will stand for $a_k$. We are now going to describe the size of $A$, along with the interpretation of each relation, function and constant, by giving the construction steps below.\\
\ 
\\Initially,
\[
\varphi = \left ( \left ( \bigwedge_{0 \leq i < j \leq n} \neg ( v_i \equiv v_j ) \right ) \land \left ( \bigvee_{0 \leq i \leq n} ( v_{n+1} \equiv v_i ) \right ) \right ) .
\]
Here we use an abbreviation for the conjunction of $n$ formulas $\varphi_0 , \ldots , \varphi_{n-1}$, which can be defined recursively as follows:
\[
\bigwedge_{0 \leq i < n} \varphi_i = \begin{cases}
\varphi_0, & \mbox{if \(n = 1\)}; \cr
\displaystyle\left(\bigwedge_{0 \leq i < n - 1} \varphi_i \land \varphi_{n - 1} \right), & \mbox{otherwise}. \cr
\end{cases}
\]
[Notice that big `$\bigwedge$' has higher priority over the usual `$\land$' and hence applies only to the immediate formula that follows.] Then define
\[
\bigwedge_{0 \leq i < j \leq n} \varphi_{ij} = \bigwedge_{0 \leq i < n} \bigwedge_{i < j \leq n} \varphi_{ij} .
\]
The case for `$\bigvee$' is similar.\\
\ 
\\At this moment, $\varphi_{\mathfrak{A}}$ merely says that all its models have \textit{exactly} $n+1$ elements. The critical part is introduced in the following.\\
\ 
\\We add to $\varphi$ a formula $\psi$ \textit{in conjuction} for each relational, functional, and constant symbol according to the following cases:
\begin{enumerate}[(i)]
\item $c^{\mathfrak{A}} = a_k$, where $c \in S$: $\psi = v_k \equiv c$.
%%
\item $f^{\mathfrak{A}} (a_{i_1}, \ldots , a_{i_m}) = a_k$, where $f \in S$ is $m$-ary: $\psi = fv_{i_1} \ldots v_{i_m} \equiv v_k$.
%%
\item $R^{\mathfrak{A}} (a_{i_1}, \ldots , a_{i_m})$, where $R \in S$ is $m$-ary: $\psi = Rv_{i_1} \ldots v_{i_m}$.
\end{enumerate}
\ 
\\Now, the construction is complete and it is clear that all models of $\varphi_{\mathfrak{A}}$ are precisely those isomorphic to $\mathfrak{A}$.\nolinebreak\hfill$\talloblong$
%End of III.5.9----------------------------------------------------------------------------------------------
%
%III.5.10----------------------------------------------------------------------------------------------------
\item \textbf{Solution to Exercise 5.10.}
\begin{enumerate}[(a)]
\item $\varphi = \exists v_0 \exists v_1 (v_0 + a \equiv 0 \land \neg v_1 \equiv 0 \land v_0 + b \equiv v_1 \cdot v_1 )$.
%%
\item Let $\pi:\mathbb{R} \to \mathbb{R}:\pi(x) = -x$ be an automorphism onto $\mathbb{R}$. If there were a formula $\varphi \in L_2^{\{+, 0\}}$ such that for all $a, b \in \mathbb{R}$, $(\mathbb{R}, +, 0) \models \varphi[a, b]$ iff $a < b$, then we would get successively:
\[
\begin{array}{l}
(\mathbb{R}, +, 0) \models \varphi[0, 1] \mbox{\ (since 0 $<$ 1);} \\
\, \\
(\mathbb{R}, +, 0) \models \varphi[0, -1] \mbox{\ (by Corollary 5.3);} \\
\, \\
0 < -1 \mbox{\ (by hypothesis),}
\end{array}
\]
a contradiction.\nolinebreak\hfill$\talloblong$
\end{enumerate}
%End of III.5.10---------------------------------------------------------------------------------------------
%
%III.5.11----------------------------------------------------------------------------------------------------
\item \textbf{Solution to Exercise 5.11.}
\begin{enumerate}[(a)]
\item We only show that the negation of a universal formula $\varphi$ is logically equivalent to an existential formula by induction on universal formulas, the other part is similar.\\
\ 
\\$\varphi$ is quantifier-free: $\neg \varphi$ is also quantifier-free and hence existential.\\
\ 
\\$\varphi = (\psi \ast \chi)$, where $\psi$ and $\chi$ are both universal and $\ast = \land, \lor$: We only show the case for `$\land$': $\neg ( \psi \land \chi )$ is logically equivalent to $(\neg \psi \lor \neg \chi)$, an existential formula (by induction, $\neg \psi$ and $\neg \chi$ are both existential).\\
\ 
\\$\varphi = \forall x \psi$, where $\psi$ is universal: $\neg \forall x \psi$ is logically equivalent to $\exists x \neg \psi$, an existential formula (by induction, $\neg \psi$ is existential).
%%
\item Let $\mathfrak{A} \subset \mathfrak{B}$, then by Corollary 5.8, for every universal formula $\neg \varphi$, where $\varphi$ is existential,
\[
\mbox{if $\mathfrak{B} \models \neg \varphi$ then $\mathfrak{A} \models \neg \varphi$.}
\]
By (a) we have that: For every existential formula $\varphi$,
\[
\mbox{if $\mathfrak{A} \models \varphi$ then $\mathfrak{B} \models \varphi$}.
\]
\end{enumerate} \begin{flushright}$\talloblong$\end{flushright}
%End of III.5.11---------------------------------------------------------------------------------------------
\end{enumerate}
%End of Section III.5----------------------------------------------------------------------------------------
\ 
\\
\\
%Section III.6-----------------------------------------------------------------------------------------------
{\large \S6. Some Simple Formalizations}
\begin{enumerate}[1.]
%III.6.7-----------------------------------------------------------------------------------------------------
\item \textbf{Solution to Exercise 6.7.}
\begin{enumerate}[(a)]
\item $\forall x ( 0 < x \rightarrow \exists t ( 0 < t \land t \cdot t \equiv x ) )$.
%%
\item $( \forall x \forall y ( x < y \rightarrow fx < fy ) \lor \forall x \forall y ( x < y \rightarrow fy < fx ) ) \rightarrow \forall x \forall y ( fx \equiv fy \rightarrow x \equiv y)$.
%%
\item $\forall u ( 0 < u \rightarrow \exists v ( 0 < v \land \forall x \forall y ( dxy < v \rightarrow dfxfy < u ) ) )$.
%%
\item $\forall x ( \exists c \forall u ( 0 < u \rightarrow \exists v ( 0 < v \land \forall y ( ( 0 < dxy \land dxy < v ) \rightarrow d+fy \cdot cx+fx \cdot cy < \cdot udxy ) ) ) \rightarrow \forall u ( 0 < u \rightarrow \exists v ( 0 < v \land \forall y ( dxy < v \rightarrow dfxfy < u ) ) ) )$.\nolinebreak\hfill$\talloblong$
\end{enumerate}
%End of III.6.7----------------------------------------------------------------------------------------------
%
%III.6.8----------------------------------------------------------------------------------------------
\item \textbf{Solution to Exercise 6.8.}
\begin{enumerate}[(a)]
\item $\Phi_{\mbox{\scriptsize{eq}}} \cup \{ \exists x \exists y \neg Rxy \}$.
%%
\item $\Phi_{\mbox{\scriptsize{eq}}} \cup \{ \exists x \exists y ( \neg x \equiv y \land Rxy ) \}$.\nolinebreak\hfill$\talloblong$
\end{enumerate}
%End of III.6.8----------------------------------------------------------------------------------------------
%
%III.6.9-----------------------------------------------------------------------------------------------------
\item \textbf{Solution to Exercise 6.9.}
\begin{enumerate}[(a)]
\item Each formula in $\Phi_{\mbox{\scriptsize{gr}}}$ is a Horn sentence.
%%
\item ``$\forall x \forall y ( x < y \lor x \equiv y \lor y < x )$'' from $\Phi_{\mbox{\scriptsize{ord}}}$ and ``$\forall x ( \neg x \equiv 0 \rightarrow \exists y \, x \cdot y \equiv 1 )$'' from $\Phi_{\mbox{\scriptsize{fd}}}$ are neither Horn sentences.\nolinebreak\hfill$\talloblong$
\end{enumerate}
%End of III.6.9----------------------------------------------------------------------------------------------
%
%End of III.6.10---------------------------------------------------------------------------------------------
\item \textbf{Solution to Exercise 6.10.} (INCOMPLETE)
\begin{enumerate}[(a)]
\item Choose $S = \emptyset$. Let
\[
\psi_1 \colonequals \forall x_0 \forall x_1 \, x_0 \equal x_1,
\]
and for $n > 1$ let
\[
\psi_n \colonequals \enump{\exists x_0}{\exists x_{n - 1}} \parenadj{\parenadj{\bigwedge\limits_{0 \leq i < j \leq n - 1} \neg x_i \equal x_j} \land \forall x \bigvee\limits_{0 \leq i \leq n - 1} x \equal x_i}.
\]
$\psi_n$ states that ``the domain contains exactly $n$ elements''.\bigskip\\
For finite $M \subset \setenum{1, 2, 3, \ldots}$, we take the characterizing sentence
\[
\varphi_1 \colonequals
\begin{cases}
\exists x \neg x \equal x & \mbox{if \(M = \emptyset\)} \cr
\bigvee\limits_{n \in M} \psi_n & \mbox{otherwise.}
\end{cases}
\]
%%
\item If $m = 1$, then we choose $S = \emptyset$ and take the characterizing sentence
\[
\varphi_2 \colonequals \exists x \, x \equal x.
\]
If $m > 1$, then we choose $S = \setenum{f}$, where $f$ is a unary function symbol. And we take the characterizing sentence
\[
\varphi_2 \colonequals \forall x \parenadj{x \equal \underbrace{\enump{f}{f}}_{\mbox{\scriptsize\(m\)-times}} x \land \bigwedge\limits_{1 \leq i \leq m - 1} \neg x \equal \underbrace{\enump{f}{f}}_{\scriptsize\mbox{\(i\)-times}} x}.
\]
%%
\item Choose $S \colonequals S_\ar^< \cup \setenum{c}$. And let $\chi$ be the conjunction of the following sentences:
\begin{enumerate}[(1)]
%%%
\item The sentences in $\Phi_\ord$ \quad ($<$ is an ordering relation);
%%%
\item $\forall x (0 < x \lor 0 \equal x)$ \quad ($0$ is the least element);
%%%
\item $\forall x (x < c \lor x \equal c)$ \quad ($c$ is the greatest element);
%%%
\item $\forall x \, x + 0 \equal x$;
%%%
\item $c + 1 \equal c \land \forall x (x < c \limply (x < x + 1 \land \neg\exists y (x < y \land y < x + 1)))$ \quad (for $x < c$, $x + 1$ is the element immediately greater than $x$, whereas $c + 1$ is identical to $c$);
%%%
\item $\forall x \forall y \, x + (y + 1) \equal (x + y) + 1$;
%%%
\item $0 + 1 \equal 1$;
%%%
\item $\forall x \, x \mul 0 \equal 0$;
%%%
\item $\forall x \forall y \, x \mul (y + 1) \equal (x \mul y) + x$.
%%%
\end{enumerate}
Intuitively, $\chi$ says that there are exactly $c + 1$ elements $\seqp{0}{c}$ in the domain ($c \geq 1$), where $<$ is the usual ordering relation and $+$ and $\mul$ are usual addition and multiplication, respectively, in which $c$ acts like the infinity $\infty$ in the sense that $c + x = c$ for any $x$ and $c \mul x = c$ for any $x \neq 0$.\medskip\\
By choosing $\psi_1$ as in part (a), we take the characterizing sentence
\[
\varphi_3 \colonequals \psi_1 \lor (\chi \land \exists x (x + 1 < c \land x \mul (x + (1 + 1)) \equal c)).
\]
%%
\item Choose $S$ and $\chi$ as in part (c), and choose $\psi_1$ as in part (a).
Also, denote by $\theta$ the conjunction of the following sentence
\[
\begin{array}{l}
\exists x \exists y (0 < x \land 0 < y \land x + 1 < c \land y + 1 < c \land x \mul (y + 1) + y \equal c).
\end{array}
\]
We take the characterizing sentence
\[
\varphi_4 \colonequals \psi_1 \lor (\chi \land \theta).
\]
%%
\item Choose $S$, $\chi$ and $\theta$ as in part (d). We take the characterizing sentence
\[
\varphi_5 \colonequals \chi \land \neg\theta.
\]
%%
\end{enumerate}
%End of III.6.10---------------------------------------------------------------------------------------------
\end{enumerate}
%End of Section III.6----------------------------------------------------------------------------------------
\ 
\\
\\
%Section III.7-----------------------------------------------------------------------------------------------
{\large \S7. Some Remarks on Formalizability}
\begin{enumerate}[1.]
\item \textbf{Note to Part (2) of 7.3.} The induction axiom $(\gamma)$ basically states that every \emph{inductive set} (for the concept of inductive set, cf. Section VII.3) defined over the domain (that is, the set $\mathbb{N}$ of natural numbers) coincides with it. Hence $\mathbb{N}$ is the smallest inductive set.
%
\item \textbf{Note to the Proof of 7.4 Dedekind's Theorem.} Let us briefly verify the validity of the isomorphism $\pi$. We show by induction on $n$ that $\pi$ is defined for every $n \in \mathbb{N}$ according to the two defining clauses (i)$^\prime$ and (ii)$^\prime$, and hence that $\pi$ is well-defined.\\
\\
$n = 0$: This is trivial from (i)$^\prime$.\\
\\
Induction step: Suppose that $\pi$ is defined for $n$, i.e. that $\pi(n) \in A$. Then by (ii)$^\prime$ $\pi$ is also defined for $n + 1$ as $\pi(n + 1) = \mbf{\sigma}^A(\pi(n))$.   
%III.7.5-----------------------------------------------------------------------------------------------------
\item \textbf{Solution to Exercise 7.5.}
\begin{enumerate}[(a)]
\item We only show that $(A, \mbf{\sigma}^A, 0^A)$ satisfies (P1), the other two cases are similar. Since $\mathfrak{A}$ is a model of $\Pi$, $\mathfrak{A} \models \forall x \neg x + 1 \equiv 0$, i.e.,
\[
\mbox{for all $a \in A$, not $a +^A 1^A = 0^A$,}
\]
which means
\[
\mbox{for all $a \in A$, not $\mbf{\sigma}^A(a) = 0^A$,}
\]
which in definition is
\[
\mathfrak{A} \models \forall x \neg \mathbf{\sigma} x \equiv 0.
\]
%%
\item Let $\mathfrak{A} = (A, +^A, \cdot^A, 0^A, 1^A)$ be a model of $\Pi$. From (a), there is an isomorphism $\pi : \mathfrak{N}_\sigma \cong (A, \sigma^A, 0^A)$ with
\[
\pi(0^\mathbb{N}) = 0^A
\]
and for all $n \in \mathbb{N}$,
\[
\pi(\sigma^\mathbb{N}(n)) = \sigma^A(\pi(n)).
\]
\\
We argue that $\pi$ is also an isomorphism between $\mathfrak{N}$ and $\mathfrak{A}$ ($\pi: \mathfrak{N} \cong \mathfrak{A}$). Parts (1) - (4) in the following establish this fact:\\
\begin{enumerate}[1)]
\item First, we already have
\[
\pi(0^\mathbb{N}) = 0^A.
\]
%%%
\item Next, let $X^A := \{ a \ | \ 0^A +^A a = a \}$. We shall show by induction on $a$ that $X^A = A$.\\
\\
$a = 0^A$: $0^A +^A 0^A = 0^A$ (cf. $\Pi$), so $0^A \in X^A$.\\
Induction step: Suppose $a \in X^A$, i.e. $0^A +^A a = a$, then
\[
\begin{array}{lll}
\ & 0^A +^A (a +^A 1^A) & \ \cr
= & (0^A +^A a) +^A 1^A & \mbox{(cf. $\Pi$)} \cr
= & a +^A 1^A & \mbox{(by induction hypothesis)}, \cr
\end{array}
\]
i.e. $(a +^A 1^A) \in X^A$. From (the ``induction axiom'' of) $\Pi$, we have that for every element $a$ of $A$, $a \in X^A$, i.e. $A \subset X^A$.\\
\\
Furthermore, since $X^A \subset A$ by definition, it follows that $X^A = A$. So, $\mathfrak{A} \models \forall x \ 0 + x \equiv x$.\\
\\
It follows that
\[
\begin{array}{llll}
\pi(1^\mathbb{N}) & = & \pi(\sigma^\mathbb{N}(0^\mathbb{N})) & \mbox{(by definition of $\sigma^\mathbb{N}$)} \cr
\ & = & \sigma^A(\pi(0^\mathbb{N})) & \mbox{(cf. $\pi: \mathfrak{N}_\sigma \cong (A, \sigma^A, 0^A)$)} \cr
\ & = & \sigma^A(0^A) & \mbox{(cf. $\pi: \mathfrak{N}_\sigma \cong (A, \sigma^A, 0^A)$)} \cr
\ & = & 0^A +^A 1^A & \mbox{(by definition of $\sigma^A$)} \cr
\ & = & 1^A.
\end{array}
\]
%%%
\item Then, let $m$ be an arbitrary but fixed integer, and let $Y^\mathbb{N} := \{ n \ |$\\$\pi(m +^\mathbb{N} n) = \pi(m) +^A \pi(n) \}$. We shall show by induction on $n$ that $Y^\mathbb{N} = \mathbb{N}$.\\
\\
$n = 0^\mathbb{N}$:
\[
\begin{array}{llll}
\pi(m +^\mathbb{N} 0^\mathbb{N}) & = & \pi(m) & \mbox{(cf. $\Pi$)} \cr
\ & = & \pi(m) +^A 0^A & \mbox{(cf. $\Pi$)} \cr
\ & = & \pi(m) +^A \pi(0^\mathbb{N}) & \mbox{(cf. $\pi: \mathfrak{N}_\sigma \cong (A, \sigma^A, 0^A)$)}.
\end{array}
\]
Induction step: Suppose $n \in Y^\mathbb{N}$, i.e. $\pi(m +^\mathbb{N} n) = \pi(m) +^A \pi(n)$, then
\[
\begin{array}{lll}
\ & \pi(m +^\mathbb{N} (n +^\mathbb{N} 1^\mathbb{N})) & \ \cr
= & \pi((m +^\mathbb{N} n) +^\mathbb{N} 1^\mathbb{N}) & \mbox{(cf. $\Pi$)} \cr
= & \pi(\sigma^\mathbb{N}(m +^\mathbb{N} n)) & \mbox{(by definition of $\sigma^\mathbb{N}$)} \cr
= & \sigma^A(\pi(m +^\mathbb{N} n)) & \mbox{(cf. $\pi: \mathfrak{N}_\sigma \cong (A, \sigma^A, 0^A)$)} \cr
= & \pi(m +^\mathbb{N} n) +^A 1^A & \mbox{(by definition of $\sigma^A$)} \cr
= & (\pi(m) +^A \pi(n)) +^A 1^A & \mbox{(by induction hypothesis)} \cr
= & \pi(m) +^A (\pi(n) +^A 1^A) & \mbox{(cf. $\Pi$)} \cr
= & \pi(m) +^A \sigma^A(\pi(n)) & \mbox{(by definition of $\sigma^A$)} \cr
= & \pi(m) +^A \pi(\sigma^\mathbb{N}(n)) & \mbox{(cf. $\pi: \mathfrak{N}_\sigma \cong (A, \sigma^A, 0^A)$)} \cr
= & \pi(m) +^A \pi(n +^\mathbb{N} 1^\mathbb{N}) & \mbox{(by definition of $\sigma^\mathbb{N}$)},
\end{array}
\]
i.e. $(n +^\mathbb{N} 1^\mathbb{N}) \in Y^\mathbb{N}$. From (the ``induction axiom'' of) $\Pi$, we have that for every integer $n$ of $\mathbb{N}$, $n \in Y^\mathbb{N}$, i.e. $\mathbb{N} \subset Y^\mathbb{N}$.\\
\\
Furthermore, since $Y^\mathbb{N} \subset \mathbb{N}$ by definition, it follows that $Y^\mathbb{N} = \mathbb{N}$. Also note that the above argument applies to all arbitrarily chosen integer $m$, hence we have that
\[
\mbox{for all $m$, $n \in \mathbb{N}$, $\pi(m +^\mathbb{N} n) = \pi(m) +^A \pi(n)$}.
\]
%%%
\item Finally, let $m$ be an arbitrary but fixed integer, and let $Z^\mathbb{N} := \{ n \ |$\\$\pi(m \cdot^\mathbb{N} n) = \pi(m) \cdot^A \pi(n) \}$. We shall show by induction on $n$ that $Z^\mathbb{N} = \mathbb{N}$.\\
\\
$n = 0^\mathbb{N}$:
\[
\begin{array}{llll}
\pi(m \cdot^\mathbb{N} 0^\mathbb{N}) & = & \pi(0^\mathbb{N}) & \mbox{(cf. $\Pi$)} \cr
\ & = & 0^A & \mbox{(cf. $\pi: \mathfrak{N}_\sigma \cong (A, \sigma^A, 0^A)$)} \cr
\ & = & \pi(m) \cdot^A 0^A & \mbox{(cf. $\Pi$)} \cr
\ & = & \pi(m) \cdot^A \pi(0^\mathbb{N}) & \mbox{(cf. $\pi: \mathfrak{N}_\sigma \cong (A, \sigma^A, 0^A)$)}.
\end{array}
\]
Induction step: Suppose $n \in Z^\mathbb{N}$, i.e. $\pi(m \cdot^\mathbb{N} n) = \pi(m) \cdot^A \pi(n)$, then
\[
\begin{array}{lll}
\ & \pi(m \cdot^\mathbb{N} (n +^\mathbb{N} 1^\mathbb{N})) & \ \cr
= & \pi((m \cdot^\mathbb{N} n) +^\mathbb{N} m) & \mbox{(cf. $\Pi$)} \cr
= & \pi(m \cdot^\mathbb{N} n) +^A \pi(m) & \mbox{(by (3))} \cr
= & (\pi(m) \cdot^A \pi(n)) +^A \pi(m) & \mbox{(by induction hypothesis)} \cr
= & \pi(m) \cdot^A (\pi(n) +^A 1^A) & \mbox{(cf. $\Pi$)} \cr
= & \pi(m) \cdot^A \sigma^A(\pi(n)) & \mbox{(by definition of $\sigma^A$)} \cr
= & \pi(m) \cdot^A \pi(\sigma^\mathbb{N}(n)) & \mbox{(cf. $\pi: \mathfrak{N}_\sigma \cong (A, \sigma^A, 0^A)$)} \cr
= & \pi(m) \cdot^A \pi(n +^\mathbb{N} 1^\mathbb{N}) & \mbox{(by definition of $\sigma^\mathbb{N}$)},
\end{array}
\]
i.e. $(n +^\mathbb{N} 1^\mathbb{N}) \in Z^\mathbb{N}$. From (the ``induction axiom'' of) $\Pi$, we have that for every integer $n$ of $\mathbb{N}$, $n \in Z^\mathbb{N}$, i.e. $\mathbb{N} \subset Z^\mathbb{N}$.\\
\\
Furthermore, since $Z^\mathbb{N} \subset \mathbb{N}$ by definition, it follows that $Z^\mathbb{N} = \mathbb{N}$. Also note that the above argument applies to all arbitrarily chosen integer $m$, hence we have that
\[
\mbox{for all $m$, $n \in \mathbb{N}$, $\pi(m \cdot^\mathbb{N} n) = \pi(m) \cdot^A \pi(n)$}.
\]
\end{enumerate}
\end{enumerate} \begin{flushright}$\talloblong$\end{flushright}
%End of III.7.5----------------------------------------------------------------------------------------------
\end{enumerate}
%End of Section III.7----------------------------------------------------------------------------------------
\ 
\\
\\
%Section III.8-----------------------------------------------------------------------------------------------
{\large \S8. Substitution}
\begin{enumerate}[1.]
\item \textbf{Note to 8.3 Substitution Lemma.} Often in mathematical arguments that the whole problem can be divided into several \emph{symmetric} cases we found the phrase ``without loss of generality'' or equivalently the phrase ``with no loss of generality'' that accompanies certain presumptions. As is clear, the validity of such arguments is based on the Substitution Lemma.\\
\\
More precisely, in this kind of situation, we restrict ourselves to a certain case. And when we finish this case, we are able to claim that we have settled other cases as well because, they can be reduced to this case by means of the Substitution Lemma.
%
\item \textbf{Proposition}: \textit{$\mathfrak{I} \frac{ab}{xy} \models \varphi \frac{y}{x}$ iff $\mathfrak{I} \frac{b}{y} \models \varphi \frac{y}{x}$.}\\
\textit{Proof.} If $x$ does not appear in $\varphi$, then the statement is vacuously true. Suppose $x$ appears in $\varphi$, if it occurs free, then it disappears in $\varphi \frac{y}{x}$, the statement is true; otherwise it is a bound variable, and the mapping of $x$ becomes irrelevant in both interpretations, thus the statement is still true.\nolinebreak\hfill$\talloblong$
%
\item \textbf{Proposition}: \textit{Let $\varphi$ be a formula, $x$ a variable, and $y$ the first variable which is different from $x$ and does not occur free in $\varphi$. Then
\[
\mathfrak{I} \models \exists^{=1} x \varphi \mbox{ iff there is exactly one $a \in A$ such that $\mathfrak{I} \frac{a}{x} \models \varphi$}.
\]
}
\textit{Proof.}
\[
\begin{array}{ll}
\, & \mathfrak{I} \models \exists^{=1} x \varphi \\
\mbox{iff} & \mbox{there is an $a \in A$ such that $\mathfrak{I} \frac{a}{x} \models ( \varphi \land \forall y ( \varphi \frac{y}{x} \rightarrow x \equiv y ) )$} \\
\mbox{iff} & \mbox{there is an $a \in A$ such that} \\
\, & \mbox{( $\mathfrak{I} \frac{a}{x} \models \varphi$ and for all $b \in A$, $\mathfrak{I} \frac{ab}{xy} \models ( \varphi \frac{y}{x} \rightarrow x \equiv y )$ )} \\
\mbox{iff} & \mbox{there is an $a \in A$ such that ( $\mathfrak{I} \frac{a}{x} \models \varphi$ and} \\
\, & \mbox{for all $b \in A$, if $\mathfrak{I} \frac{b}{y} \models \varphi \frac{y}{x}$ then $a = b$ ) \,\, (by the conjecture above)} \\
\mbox{iff} & \mbox{there is an $a \in A$ such that ( $\mathfrak{I} \frac{a}{x} \models \varphi$ and for all $b \in A$,} \\
\, & \mbox{if $(\mathfrak{I} \frac{b}{y}) \frac{\mathfrak{I} \frac{b}{y} (y) }{x} \models \varphi$ then $a = b$ ) \,\, (by substitution lemma)} \\
\mbox{iff} & \mbox{there is an $a \in A$ such that ( $\mathfrak{I} \frac{a}{x} \models \varphi$ and} \\
\, & \mbox{for all $b \in A$, if $\mathfrak{I} \frac{b}{x} \models \varphi$ then $a = b$ ) \,\, (since by premise $y$ does not} \\
\, & \mbox{occur free in $\varphi$, see the proof of the above conjecture)} \\
\mbox{iff} & \mbox{there is an $a \in A$ such that ( $\mathfrak{I} \frac{a}{x} \models \varphi$ and for all $b \in A \setminus \{ a \}$,} \\
\, & \mbox{not $\mathfrak{I} \frac{b}{x} \models \varphi$ )} \\
\mbox{iff} & \mbox{there is exactly one $a \in A$ such that $\mathfrak{I} \frac{a}{x} \models \varphi$.}
\end{array}
\] \begin{flushright}$\talloblong$\end{flushright}
%
%III.8.8-----------------------------------------------------------------------------------------------------
\item \textbf{Solution to Exercise 8.8.}
\begin{itemize}
\item ``there exist at most $n$'': $\displaystyle \exists^{\leq n} x \varphi := \exists v_0 \ldots \exists v_{n-1}(\bigwedge_{0 \leq i < j \leq n - 1} \neg v_i \equiv v_j \land \forall v_n (\varphi \frac{v_n}{x} \rightarrow \bigvee_{0 \leq i \leq n - 1} x \equiv v_i))$.
%%
\item ``there exist exactly $n$'': $\displaystyle \exists^{=n} x \varphi := \exists^{\leq n } x \varphi \land \neg \exists^{\leq n - 1} x \varphi$.\nolinebreak\hfill$\talloblong$
\end{itemize}
%End of III.8.8------------------------------------------------------------------------------------
%
%III.8.9-----------------------------------------------------------------------------------------------------
\item \textbf{Solution to Exercise 8.9.}
\begin{enumerate}[(a)]
\item
\[
\begin{array}{ll}
\, & [ \exists x \exists y ( Pxu \land Pyu ) ] \frac{uuu}{xyv} \\
= & \exists x [ \exists y ( Pxu \land Pyv ) \frac{ux}{vx} ] \\
= & \exists x \exists y [ ( Pxu \land Pyv ) \frac{uy}{vy} ] \\
= & \exists x \exists y ( Pxu \frac{u}{v} \land Pyv \frac{u}{v} ) \\
= & \exists x \exists y ( Pxu \land Pyu )
\end{array}
\]
%%
\item
\[
\begin{array}{ll}
\, & [ \exists x \exists y ( Pxu \land Pyv ) ] \frac{v\,fuv}{u\,\phantom{f}v\phantom{u}} \\
= & \exists x [ \exists y ( Pxu \land Pyv ) \frac{v\,fuv\,x}{u\,\phantom{f}v\phantom{u}\,x} ] \\
= & \exists x \exists y [ ( Pxu \land Pyv ) \frac{v\,fuv\,y}{u\,\phantom{f}v\phantom{u}\,y} ] \\
= & \exists x \exists y ( Pxu \frac{v\,fuv}{u\,\phantom{f}v\phantom{u}} \land Pyv \frac{v\,fuv}{u\,\phantom{f}v\phantom{u}} ) \\
= & \exists x \exists y ( Pxv \land Pyfuv )
\end{array}
\]
%%
\item
\[
\begin{array}{ll}
\, & [ \exists x \exists y ( Pxu \land Pyv ) ] \frac{u\,x\,fuv}{x\,u\,\phantom{f}v\phantom{u}} \\
= & \exists w [ \exists y ( Pxu \land Pyv ) \frac{x\,fuv\,w}{u\,\phantom{f}v\phantom{u}\,x} ] \\
= & \exists w \exists y [ ( Pxu \land Pyv ) \frac{x\,fuv\,w\,y}{u\,\phantom{f}v\phantom{u}\,x\,y} ] \\
= & \exists w \exists y ( Pxu \frac{x\,fuv\,w}{u\,\phantom{f}v\phantom{u}\,x} \land Pyv \frac{x\,fuv\,w}{u\,\phantom{f}v\phantom{u}\,x} ) \\
= & \exists w \exists y ( Pwx \land Pyfuv )
\end{array}
\]
%%
\item
\[
\begin{array}{ll}
\, & [ \forall x \exists y ( Pxy \land Pxu ) \lor \exists u \, fuu \equiv x ] \frac{x\,fxy}{x\,\phantom{f}u\phantom{y}} \\
= & [ \forall x \exists y ( Pxy \land Pxu ) ] \frac{x\,fxy}{x\,\phantom{f}u\phantom{y}} \lor [ \exists u \, fuu \equiv x ] \frac{x\,fxy}{x\,\phantom{f}u\phantom{y}} \\
= & \forall v [ \exists y ( Pxy \land Pxu ) \frac{fxy\,v}{\phantom{f}u\phantom{y}\,x} ] \lor \exists u \, fuu \equiv x \\
= & \forall v \exists w [ ( Pxy \land Pxu ) \frac{fxy\,v\,w}{\phantom{f}u\phantom{y}\,x\,y} ] \lor \exists u \, fuu \equiv x \\
= & \forall v \exists w ( Pxy \frac{fxy\,v\,w}{\phantom{f}u\phantom{y}\,x\,y} \land Pxu \frac{fxy\,v\,w}{\phantom{f}u\phantom{y}\,x\,y} ) \lor \exists u \, fuu \equiv x \\
= & \forall v \exists w ( Pvw \land Pvfxy ) \lor \exists u \, fuu \equiv x.
\end{array}
\]
\end{enumerate} \begin{flushright}$\talloblong$\end{flushright}
%End of III.8.9----------------------------------------------------------------------------------------------
%
%III.8.10----------------------------------------------------------------------------------------------------
\item \textbf{Solution to Exercise 8.10.} For every interpretation $\mathfrak{I}$,
\[
\begin{array}{ll}
\, & \mathfrak{I} \models \varphi \frac{t_0 \ldots t_r}{x_0 \ldots x_r} \\
\mbox{iff} & \mbox{for all $a_0 \in A$, \ldots , for all $a_r \in A$, $\mathfrak{I} \frac{a_0 \ldots a_r}{x_0 \ldots x_r} \models \varphi \frac{t_0 \ldots t_r}{x_0 \ldots x_r}$ \,\, (since by} \\
\, & \mbox{premise that $x_0, \ldots , x_r \not \in \var(t_0) \cup \ldots \cup \var(t_r)$, $\mathfrak{I} \frac{a_0 \ldots a_r}{x_0 \ldots x_r} (t_0) = \mathfrak{I} (t_0)$,} \\
\, & \mbox{i.e. the mapping of $x_0, \ldots , x_r$ is irrelevant to the satisfiability and,} \\
\, & \mbox{in addition, the appearances of $x_0, \ldots , x_r$ are gone in the resulting} \\
\, & \mbox{formula $\varphi \frac{t_0 \ldots t_r}{x_0 \ldots x_r}$ after the substitution in $\varphi$} \\
\mbox{iff} & \mbox{for all $a_0 \in A$, \ldots , for all $a_r \in A$, $(\mathfrak{I} \frac{a_0 \ldots a_r}{x_0 \ldots x_r}) \frac{\mathfrak{I}(t_0) \ldots \mathfrak{I}(t_r)}{x_0 \ldots x_r} \models \varphi$ \,\, (by} \\
\, & \mbox{Substitution Lemma and $\mathfrak{I} \frac{a_0 \ldots a_r}{x_0 \ldots x_r} (t_0) = \mathfrak{I} (t_0)$, for the same reason} \\
\, & \mbox{explained above)} \\
\mbox{iff} & \mbox{for all $a_0 \in A$, \ldots , for all $a_r \in A$, if $\mathfrak{I} \frac{a_0 \ldots a_r}{x_0 \ldots x_r}(x_0) = \mathfrak{I}(t_0)$, \ldots , and} \\
\, & \mbox{$\mathfrak{I} \frac{a_0 \ldots a_r}{x_0 \ldots x_r}(x_r) = \mathfrak{I}(t_r)$, then $\mathfrak{I} \frac{a_0 \ldots a_r}{x_0 \ldots x_r} \models \varphi$} \\
\mbox{iff} & \mbox{for all $a_0 \in A$, \ldots , for all $a_r \in A$,} \\
\, & \mbox{$\mathfrak{I} \frac{a_0 \ldots a_r}{x_0 \ldots x_r} \models ((x_0 \equiv t_0 \land \ldots \land x_r \equiv t_r) \rightarrow \varphi)$ \,\, (since} \\
\, & \mbox{$\mathfrak{I} \frac{a_0 \ldots a_r}{x_0 \ldots x_r}(t_0) = \mathfrak{I}(t_0)$, \ldots , $\mathfrak{I} \frac{a_0 \ldots a_r}{x_0 \ldots x_r}(t_r) = \mathfrak{I}(t_r)$)} \\
\mbox{iff} & \mathfrak{I} \models \forall x_0 \ldots \forall x_r (( x_0 \equiv t_0 \land \ldots \land x_r \equiv t_r ) \rightarrow \varphi )
\end{array}
\]
%End of III.8.10---------------------------------------------------------------------------------------------
%
%III.8.11----------------------------------------------------------------------------------------------------
\item \textbf{Solution to Exercise 8.11.}
\[
\begin{array}{l}
\displaystyle \frac{\,}{x \;\;\; x_0 \ldots x_r \;\;\; t_0 \ldots t_r \;\;\; x} \mbox{ if $x \neq x_0, \ldots , x \neq x_r$}; \\
\, \\
\displaystyle \frac{\,}{x \;\;\; x_0 \ldots x_r \;\;\; t_0 \ldots t_r \;\;\; t_i} \mbox{ if $x = x_i$}; \\
\, \\
\displaystyle \frac{\,}{c \;\;\; x_0 \ldots x_r \;\;\; t_0 \ldots t_r \;\;\; c}; \\
\, \\
\displaystyle \frac{{{\displaystyle  t_1^\prime \;\;\; x_0 \ldots x_r \;\;\; t_0 \ldots t_r \;\;\; s_1^\prime 
\atop 
\displaystyle  \;\; \vdots  \;\;\;\;\;  \vdots  \;\;\;\;\;\;\;\;\;\;\;\;\;\;\;  \vdots  \;\;\;\;\;\;\;\;\;\;\;\;\; \vdots \;\;\;\; } 
\atop 
\displaystyle  t_1^\prime \;\;\; x_0 \ldots x_r \;\;\; t_0 \ldots t_r \;\;\; s_1^\prime}}{ft_1^\prime \ldots t_n^\prime \;\;\; x_0 \ldots x_r \;\;\; t_0 \ldots t_r \;\;\; fs_1^\prime \ldots s_n^\prime} \mbox{ if $f \in S$ and $f$ is $n$-ary}; \\
\, \\
\, \\
\, \\
\displaystyle \frac{ {\displaystyle t_1^\prime \;\;\; x_0 \ldots x_r \;\;\; t_0 \ldots t_r \;\;\; s_1^\prime \atop 
\displaystyle t_2^\prime \;\;\; x_0 \ldots x_r \;\;\; t_0 \ldots t_r \;\;\; s_2^\prime} }{t_1^\prime \equiv t_2^\prime \;\;\; x_0 \ldots x_r \;\;\; t_0 \ldots t_r \;\;\; s_1^\prime \equiv s_2^\prime} \mbox{ if $t_1^\prime , t_2^\prime \in T^S$}; \\
\, \\
\displaystyle \frac{{{\displaystyle  t_1^\prime \;\;\; x_0 \ldots x_r \;\;\; t_0 \ldots t_r \;\;\; s_1^\prime 
\atop 
\displaystyle  \;\; \vdots  \;\;\;\;\;  \vdots  \;\;\;\;\;\;\;\;\;\;\;\;\;\;\;  \vdots  \;\;\;\;\;\;\;\;\;\;\;\;\; \vdots \;\;\;\; } 
\atop 
\displaystyle  t_1^\prime \;\;\; x_0 \ldots x_r \;\;\; t_0 \ldots t_r \;\;\; s_1^\prime}}{Rt_1^\prime \ldots t_n^\prime \;\;\; x_0 \ldots x_r \;\;\; t_0 \ldots t_r \;\;\; Rs_1^\prime \ldots s_n^\prime} \mbox{ if $R \in S$ is $n$-ary and } \\
\phantom{Rt_1^\prime \ldots t_n^\prime \;\;\; x_0 \ldots x_r \;\;\; t_0 \ldots t_r \;\;\; Rs_1^\prime \ldots s_n^\prime} \;\;\; t_1^\prime, \ldots, t_n^\prime \in T^S; \\
\, \\
\displaystyle \frac{ \phantom{\neg} \varphi \;\;\; x_0 \ldots x_r \;\;\; t_0 \ldots t_r \;\;\; \phantom{\neg} \varphi \frac{t_0 \ldots t_r}{x_0 \ldots x_r} }{\neg \varphi \;\;\; x_0 \ldots x_r \;\;\; t_0 \ldots t_r \;\;\; \neg \varphi \frac{t_0 \ldots t_r}{x_0 \ldots x_r} } \mbox{ if $\varphi \in L^S$}; \\
\, \\
\displaystyle \frac{ \displaystyle { 
\phantom{( \lor \psi)} \varphi \;\;\; x_0 \ldots x_r \;\;\; t_0 \ldots t_r \;\;\; \varphi \frac{t_0 \ldots t_r}{x_0 \ldots x_r} \phantom{( \lor \psi \frac{t_0 \ldots t_r}{x_0 \ldots x_r} )}
\atop
\phantom{( \lor \varphi)} \psi \;\;\; x_0 \ldots x_r \;\;\; t_0 \ldots t_r \;\;\;  \psi \frac{t_0 \ldots t_r}{x_0 \ldots x_r} \phantom{( \lor \varphi \frac{t_0 \ldots t_r}{x_0 \ldots x_r} )}
 } }
{(\varphi \lor \psi) \;\;\; x_0 \ldots x_r \;\;\; t_0 \ldots t_r \;\;\; (\varphi \frac{t_0 \ldots t_r}{x_0 \ldots x_r} \lor \psi \frac{t_0 \ldots t_r}{x_0 \ldots x_r} )} 
\mbox{ if $\varphi, \psi \in L^S$}; \\
\, \\
\displaystyle 
\frac{\phantom{\exists x} \varphi \;\;\; x_{i_1} \ldots x_{i_s} \  x \;\;\; t_{i_1} \ldots t_{i_s} \  u \;\;\; \phantom{\exists u} \varphi \frac{t_{i_1} \ldots t_{i_s} \  u }{ x_{i_q} \ldots x_{i_s} \  x }}{\exists x \varphi \;\;\; x_0 \ldots x_r \phantom{\;\;\; x} \;\;\; t_0 \ldots t_r \phantom{\;\;\; u} \;\;\; \exists u \varphi \frac{t_{i_1} \ldots t_{i_s} \  u }{ x_{i_1} \ldots x_{i_s} \  x }} \mbox{ if $\varphi \in L^S$ and}
\end{array}
\]
$x_{i_1} , \ldots , x_{i_s} \; (i_1 < \ldots < i_s)$ are those $x_i$ among $x_0 , \ldots , x_r$ such that $x_i \in \free (\exists x \varphi)$ and $x_i \neq t_i$. And $u$ is the variable $x$ if $x$ does not occur in $t_{i_1}, \ldots , t_{i_s}$, otherwise it is the first variable in the list $v_0 , v_1 , v_2 , \ldots$ which does not occur in $\varphi , t_{i_1} , \ldots , t_{i_s}$.\nolinebreak\hfill$\talloblong$
%End of III.8.11-----------------------------------------------------------------------------------
\end{enumerate}
%End of Section III.8----------------------------------------------------------------------------------------
%End of Chapter III------------------------------------------------------------------------------------------
%%Chapter IV----------------------------------------------------------------------------------------
{\LARGE \bfseries IV \\ \\ A Sequent Calculus}
\\
\\
\\
%Section IV.1--------------------------------------------------------------------------------------
{\large \S1. Sequent Rules}
\begin{enumerate}[1.]
%
\item \textbf{Note on the Paragraph after Definition 1.1.} The reader should bear in mind that, sequent rules themselves are only syntactic operations: we can speak of the \emph{correctness} of $\mathfrak{S}$ only if we interpret it as a tool for reasoning.
%
\item \textbf{Note on Sequents.} Implicitly implied by the context, a sequent is finite in length.
%
\end{enumerate}
%End of Section IV.1-------------------------------------------------------------------------------
\ 
\\
\\
%Section IV.2--------------------------------------------------------------------------------------
{\large \S2. Structural Rules and Connective Rules}
\begin{enumerate}[1.]
\item \textbf{Note to the Correctness of 2.4, Contradiction Rule (Ctr).} For every interpretation (more specifically, one that satisfies $\Gamma$), either it satisfies $\varphi$ or not, i.e. it satisfies either $\varphi$ or $\neg \varphi$. Furthermore, since there is no interpretation satisfying the sequence $\Gamma \neg \varphi$ of formulas (because of both $\Gamma \neg \varphi \psi$ and $\Gamma \neg \varphi \neg \psi$ being derivable sequents), it turns out that, for every interpretation that satisfies $\Gamma$, there is only one possibility --- it satisfies $\varphi$.
%
%IV.2.7--------------------------------------------------------------------------------------------
\item \textbf{Solution to Exercise 2.7.}
\begin{enumerate}[(a)]
\item Correct. \\
\textit{Justification.}
\[
\begin{array}{lllll}
1. & \Gamma & \varphi_1 & \psi_1 & \mbox{premise} \\
2. & \Gamma & \varphi_2 & \psi_2 & \mbox{premise} \\
3. & \Gamma & \varphi_1 & (\psi_1 \lor \psi_2) & \mbox{($\lor$S) applied to 1.} \\
4. & \Gamma & \varphi_2 & (\psi_1 \lor \psi_2) & \mbox{($\lor$S) applied to 2.} \\
5. & \Gamma & (\varphi_1 \lor \varphi_2) & (\psi_1 \lor \psi_2) & \mbox{($\lor$A) applied to 3. and 4.} 
\end{array}
\]
%%
\item Incorrect. \\
\textit{Counterexample.} Let $\Gamma = \emptyset$, $\psi_1 = \varphi_1$, and $\psi_2 = \varphi_2 = \neg \varphi_1$. 
\end{enumerate} \begin{flushright}$\talloblong$\end{flushright}
%End of IV.2.7-------------------------------------------------------------------------------------
\end{enumerate}
%End of Section IV.2-------------------------------------------------------------------------------
\ 
\\
\\
%Section IV.3--------------------------------------------------------------------------------------
{\large \S3. Derivable Connective Rules}
\begin{enumerate}[1.]
\item \textbf{Note to the First Paragraph in \S3.} The basic form of tautologies $(\varphi \lor \neg \varphi)$ are derivable by applying the rule (TND) (see page 63).\\
Note that other forms of tautologies are derivable from this basic form using logical connectives, $\neg$ and $\lor$, and hence are derivable in the sequent calculus $\mathfrak{S}$.
%
\item \textbf{Solution to Exercise 3.6.}
\begin{itemize}
\item[(a1)]
\[
\begin{array}{lllll}
1. & \Gamma & \, & \varphi & \mbox{premise} \\
2. & \Gamma & \neg \varphi & \neg \varphi & \mbox{(Assm)} \\
3. & \Gamma & \neg \varphi & (\neg \varphi \lor \neg \neg \varphi) & \mbox{($\lor$S) applied to 2.} \\
4. & \Gamma & \neg \neg \varphi & \neg \neg \varphi & \mbox{(Assm)} \\
5. & \Gamma & \neg \neg \varphi & (\neg \varphi \lor \neg \neg \varphi) & \mbox{($\lor$S) applied to 4.} \\
6. & \Gamma & \, & (\neg \varphi \lor \neg \neg \varphi) & \mbox{(PC) applied to 3. and 5.} \\
7. & \Gamma & \, & \neg \neg \varphi & \mbox{3.5 applied to 6. and 1.}
\end{array}
\]
%%
\item[(a2)]
\[
\begin{array}{lllll}
1. & \Gamma & \, & \neg \neg \varphi & \mbox{premise} \\
2. & \Gamma & \neg \varphi & \neg \varphi & \mbox{(Assm)} \\
3. & \Gamma & \neg \varphi & \neg \neg \varphi & \mbox{(Ant) applied to 1.} \\
4. & \Gamma & \, & \varphi & \mbox{(Ctr) applied to 2. and 3.}
\end{array}
\]
%%
\item[(b)]
\[
\begin{array}{lllll}
1. & \Gamma & \, & \varphi & \mbox{premise} \\
2. & \Gamma & \, & \psi & \mbox{premise} \\
3. & \Gamma & (\neg \varphi \lor \neg \psi) & \varphi & \mbox{(Ant) applied to 1.} \\
4. & \Gamma & (\neg \varphi \lor \neg \psi) & \psi & \mbox{(Ant) applied to 2.} \\
5. & \Gamma & (\neg \varphi \lor \neg \psi) & (\neg \varphi \lor \neg \psi) & \mbox{(Assm)} \\
6. & \Gamma & (\neg \varphi \lor \neg \psi) & \neg \psi & \mbox{3.5 applied to 5. and 3.} \\
7. & \Gamma & \, & \neg (\neg \varphi \lor \neg \psi) & \mbox{(Ctr) applied to 4. and 7.}
\end{array}
\]
%%
\item[(c)]
\[
\begin{array}{lllll}
1. & \Gamma & \varphi & \psi & \mbox{premise} \\
2. & \Gamma & \varphi & (\neg \varphi \lor \psi) & \mbox{($\lor$S) applied to 1.} \\
3. & \Gamma & \neg \varphi & \neg \varphi & \mbox{(Assm)} \\
4. & \Gamma & \neg \varphi & (\neg \varphi \lor \psi) & \mbox{($\lor$S) applied to 3.} \\
5. & \Gamma & \, & (\neg \varphi \lor \psi) & \mbox{(PC) applied to 2. and 4.}
\end{array}
\]
%%
\item[(d1)]
\[
\begin{array}{lllll}
1. & \Gamma & \, & \neg (\neg \varphi \lor \neg \psi) & \mbox{premise} \\
2. & \Gamma & \neg \varphi & \neg \varphi & \mbox{(Assm)} \\
3. & \Gamma & \neg \varphi & (\neg \varphi \lor \neg \psi) & \mbox{($\lor$S) applied to 2.} \\
4. & \Gamma & \neg \varphi & \neg (\neg \varphi \lor \neg \psi) & \mbox{(Ant) applied to 1.} \\
5. & \Gamma & \, & \varphi & \mbox{(Ctr) applied to 3. and 4.}
\end{array}
\]
%%
\item[(d2)]
\[
\begin{array}{lllll}
1. & \Gamma & \, & \neg (\neg \varphi \lor \neg \psi) & \mbox{premise} \\
2. & \Gamma & \neg \psi & \neg \psi & \mbox{(Assm)} \\
3. & \Gamma & \neg \psi & (\neg \varphi \lor \neg \psi) & \mbox{($\lor$S) applied to 2.} \\
4. & \Gamma & \neg \psi & \neg (\neg \varphi \lor \neg \psi) & \mbox{(Ant) applied to 1.} \\
5. & \Gamma & \, & \psi & \mbox{(Ctr) applied to 3. and 4.}
\end{array}
\]
\end{itemize} \begin{flushright}$\talloblong$\end{flushright}
%End of IV.3.6-------------------------------------------------------------------------------------
%
\item \textbf{Derivable Rules:}
\begin{enumerate}[(a)]
\item \textbf{Commutativity Rule for $\lor$ (Comm).}
\[
\begin{array}{ll}
\Gamma & (\varphi \lor \psi) \cr \hline
\Gamma & (\psi \lor \varphi)
\end{array}
\]
\\
\textit{Justification.}
\[
\begin{array}{llll}
1. & \Gamma & (\varphi \lor \psi) & \mbox{premise} \\
2. & \Gamma \;\; \varphi & \varphi & \mbox{(Assm)} \\
3. & \Gamma \;\; \varphi & (\psi \lor \varphi) & \mbox{($\lor$S) applied to 2.} \\
4. & \Gamma \;\; \psi & \psi & \mbox{(Assm)} \\
5. & \Gamma \;\; \psi & (\psi \lor \varphi) & \mbox{($\lor$S) applied to 4.} \\
6. & \Gamma \;\; (\varphi \lor \psi) & (\psi \lor \varphi) & \mbox{($\lor$A) applied to 3. and 5.} \\
7. & \Gamma & (\psi \lor \varphi) & \mbox{(Ch) applied to 1. and 6.}
\end{array}
\]
%%
\item
\[
\begin{array}{ll}
\Gamma & (\varphi \rightarrow \psi) \cr
\Gamma & (\psi \rightarrow \varphi) \cr \hline
\Gamma & (\varphi \leftrightarrow \psi)
\end{array}
\]
\\
\textit{Justification.}
\[
\begin{array}{llll}
1. & \Gamma & (\varphi \rightarrow \psi) & \mbox{premise} \\
2. & \Gamma & (\psi \rightarrow \varphi) & \mbox{premise} \\
3. & \Gamma \;\; \varphi & (\varphi \rightarrow \psi) & \mbox{(Ant) applied to 1.} \\
4. & \Gamma \;\; \varphi & \varphi & \mbox{(Assm)} \\
5. & \Gamma \;\; \varphi & \psi & \mbox{3.5 applied to 3. and 4.} \\
6. & \Gamma \;\; \varphi & \neg (\neg \varphi \lor \neg \psi) & \mbox{3.6(b) applied to 4.} \\
\  & \                   & \                                  & \mbox{and 5.} \\
7. & \Gamma \;\; \varphi & (\neg (\varphi \lor \psi) \lor \neg (\neg \varphi \lor \neg \psi)) & \mbox{($\lor$S) applied to 6.} \\
8. & \Gamma \;\; \psi & (\psi \rightarrow \varphi) & \mbox{(Ant) applied to 2.} \\
9. & \Gamma \;\; \psi & \psi & \mbox{(Assm)} \\
10.& \Gamma \;\; \psi & \varphi & \mbox{3.5 applied to 8. and 9.} \\
11.& \Gamma \;\; (\varphi \lor \psi) & \varphi & \mbox{($\lor$A) applied to 4.} \\
\  & \                               & \       & \mbox{and 10.} \\
12.& \Gamma \;\; \neg \varphi & \neg (\varphi \lor \psi) & \mbox{3.3(a) applied to 11.} \\
13.& \Gamma \;\; \neg \varphi & (\neg (\varphi \lor \psi) \lor \neg (\neg \varphi \lor \neg \psi)) & \mbox{($\lor$S) applied to 12.} \\
14.& \Gamma & (\varphi \leftrightarrow \psi) & \mbox{(PC) applied to 7.} \\
\  & \      & \                              & \mbox{and 13.}
\end{array}
\]
%%
\item 
\[
\begin{array}{ll}
\Gamma & (\varphi \leftrightarrow \psi) \cr \hline
\Gamma & (\varphi \rightarrow \psi)
\end{array}
\]
\\
\textit{Justification.}
\[
\begin{array}{llll}
1. & \Gamma & (\varphi \leftrightarrow \psi) & \mbox{premise} \\
2. & \Gamma \;\; \varphi & (\varphi \leftrightarrow \psi) & \mbox{(Ant) applied to 1.} \\
3. & \Gamma \;\; \varphi & \varphi & \mbox{(Assm)} \\
4. & \Gamma \;\; \varphi & (\varphi \lor \psi) & \mbox{($\lor$S) applied to 3.} \\
5. & \Gamma \;\; \varphi & \neg (\neg \varphi \lor \neg \psi) & \mbox{3.5 applied to 2. and 4.} \\
6. & \Gamma \;\; \varphi & \psi & \mbox{3.6(d2) applied to 5.} \\
7. & \Gamma & (\varphi \rightarrow \psi) & \mbox{3.6(c) applied to 6.}
\end{array}
\]
%%
\item 
\[
\begin{array}{ll}
\Gamma & (\varphi \leftrightarrow \psi) \cr \hline
\Gamma & (\psi \rightarrow \varphi)
\end{array}
\]
\\
\textit{Justification.}
\[
\begin{array}{llll}
1. & \Gamma & (\varphi \leftrightarrow \psi) & \mbox{premise} \\
2. & \Gamma \;\; \psi & (\varphi \leftrightarrow \psi) & \mbox{(Ant) applied to 1.} \\
3. & \Gamma \;\; \psi & \psi & \mbox{(Assm)} \\
4. & \Gamma \;\; \psi & (\varphi \lor \psi) & \mbox{($\lor$S) applied to 3.} \\
5. & \Gamma \;\; \psi & \neg (\neg \varphi \lor \neg \psi) & \mbox{3.5 applied to 2. and 4.} \\
6. & \Gamma \;\; \psi & \varphi & \mbox{3.6(d1) applied to 5.} \\
7. & \Gamma & (\psi \rightarrow \varphi) & \mbox{3.6(c) applied to 6.}
\end{array}
\]
\end{enumerate} \begin{flushright}$\talloblong$\end{flushright}
\end{enumerate}
%End of Section IV.3-------------------------------------------------------------------------------
\ 
\\
\\
%Section IV.4--------------------------------------------------------------------------------------
{\large \S4. Quantifier and Equality Rules}
\begin{enumerate}[1.]
\item \textbf{Note on 4.2.} The additional condition follows from the fact that the validity of $\psi$ may be based on $\varphi \frac{y}{x}$, the particular substitution of $y$ into $x$ in $\varphi$, without which $\psi$ may not be derivable.
%
\item \textbf{Note on 4.2.} This rule only applies to the case in which $y$ is a \textit{variable}, but not to the case in which it is a term other than a variable. The \textit{incorrect} sequent below serves as a counterexample:
\[
\begin{array}{ll}
0 \equiv z & 0 \equiv z \cr \hline
\exists x \; x \equiv z & 0 \equiv z
\end{array}.
\]
%
%IV.4.5--------------------------------------------------------------------------------------------
\item \textbf{Solution to Exercise 4.5.} (INCOMPLETE) All three rules are correct. For the first two, we provide justifications:
\[
\begin{seqrule}{ll}
\varphi & \psi \cr \hline
\exists x \varphi & \exists x \psi
\end{seqrule}
\]
\textit{Justification.}\\
\centerline{
\begin{derivation}
1. & $\varphi$ & $\psi$ & premise \\
2. & $\varphi$ & $\exists x \psi$ & $\es$ applied to 1. with $t = x$ \\
3. & $\exists x \varphi$ & $\exists x \psi$ & $\ea$ applied to 2. with $y = x$
\end{derivation}}
\[
\begin{seqrule}{lll}
\Gamma & \varphi & \psi \cr \hline
\Gamma & \forall x \varphi & \exists x \psi
\end{seqrule}
\]
\textit{Justification.}\\
\centerline{
\begin{derivation}
1. & $\Gamma \ \varphi$ & $\psi$ & premise \\
2. & $\Gamma \ \neg \psi$ & $\neg \varphi$ & (Cp) applied to 1. \\
3. & $\Gamma \ \neg \psi$ & $\exists x \neg \varphi$ & $\es$ applied to 2. with $t = x$ \\
4. & $\Gamma \ \neg \exists x \neg \varphi$ & $\psi$ & (Cp) applied to 3. \\
5. & $\Gamma \ \neg \exists x \neg \varphi$ & $\exists x \psi$ & $\es$ applied to 4. with $t = x$
\end{derivation}}
\medskip\\
As for the last rule,\medskip\\
\begin{bquoteno}{60ex}{($\ast$)}
$\begin{seqrule}{ll}
\Gamma & \varphi \sbst{fy}{x} \cr \hline
\Gamma & \forall x \varphi
\end{seqrule}$ \quad if $f$ is unary, and $f$ and $y$ do not occur in $\Gamma \ \forall x \varphi$,
\end{bquoteno}\medskip\\
we argue its correctness below:\medskip\\
\textit{Correctness.} We show ($\ast$) by induction on \emph{the sequent rule} by which $\Gamma \ \varphi\sbst{fy}{x}$ is obtained in a derivation.\smallskip\\
If the sequent $\Gamma \ \varphi\sbst{fy}{x}$ is obtained by applying $\assm$: Then $\varphi\sbst{fy}{x} \in \Gamma$. By the premise that $f$ and $y$ do not occur in $\Gamma \ \forall x \varphi$, we have $x \not\in \free{\varphi}$. Thus, $\varphi\sbst{z}{x} = \varphi = \varphi\sbst{fy}{x} \in \Gamma$, where $z$ is a variable not occurring in $\Gamma \ \forall x \varphi$. The following derivation yields that $\Gamma \derives \forall x \varphi$:\smallskip\\
\begin{derivation}
1. & $\Gamma$ & $\varphi\sbst{z}{x}$ & $\assm$\cr
2. & \ & $v_0 \equal v_0$ & $\eq$\cr
3. & \ & $\exists v_0 \, v_0 \equal v_0$ & $\es$ applied to 2. with $t = v_0$\cr
4. & $\Gamma$ & $\exists v_0 \, v_0 \equal v_0$ & $\ant$ applied to 3.\cr
5. & $\Gamma \ \exists v_0 \, v_0 \equal v_0$ & $\varphi\sbst{z}{x}$ & $\ant$ applied to 1.\cr
6. & $\Gamma \ \neg\varphi\sbst{z}{x}$ & $\neg\exists v_0 \, v_0 \equal v_0$ & (Cp)(a) applied to 5.\cr
7. & $\Gamma \ \exists x \neg\varphi$ & $\neg\exists v_0 \, v_0 \equal v_0$ & $\ea$ applied to 6.\cr
8. & $\Gamma \ \exists v_0 \, v_0 \equal v_0$ & $\forall x \varphi$ & (Cp)(d) applied to 7.\cr
9. & $\Gamma$ & $\forall x \varphi$ & (Ch) applied 4. and 8.
\end{derivation}\medskip\\
If the sequent $\Gamma \ \varphi\sbst{fy}{x}$ is obtained by applying $\ant$ to a derivable sequent $\Gamma' \ \varphi\sbst{fy}{x}$ where $\Gamma' \subset \Gamma$: Then $f$ and $y$ do not occur in $\Gamma'$, either. By induction hypothesis, we have $\Gamma' \derives \forall x \varphi$. The following derivation yields that $\Gamma \derives \forall x \varphi$:\smallskip\\
\begin{derivation}
1. & $\Gamma'$ & $\forall x \varphi$ & premise\cr
2. & $\Gamma$  & $\forall x \varphi$ & $\ant$ applied to 1.
\end{derivation}\medskip\\
If the sequent $\Gamma \ \varphi\sbst{fy}{x}$ is obtained by applying $\pc$ to two derivable sequents $\Gamma \ \psi \ \varphi\sbst{fy}{x}$ and $\Gamma \ \neg\psi \ \varphi\sbst{fy}{x}$: Then
HERE\medskip\\
{[OLD TEXT STARTS HERE]} \textit{Correctness.} First of all, let us assume that $\Gamma \ \forall x \varphi$ is a sequent over $L^S$. Suppose $\Gamma \models \varphi\frac{fy}{x}$ and $f$ is unary, and $f$ and $y$ do not occur in $\Gamma \  \forall x \varphi$. Let the $S$-interpretation $\mathfrak{I} = (\mathfrak{A}, \beta)$ be a model of $\Gamma$. Then by premise and the Coincidence Lemma, it follows that the value of $\beta(y)$ can be arbitrarily chosen; moreover, for \emph{any} $S \cup \{ f \}$-interpretation $\mathfrak{I}^\prime = (\mathfrak{A}^\prime, \beta)$ with $\mathfrak{A}^\prime$ an $S \cup \{ f \}$-expansion of $\mathfrak{A}$, $\mathfrak{I}^\prime \models \Gamma$. Without loss of generality, let us pick the one with $f^A$ being the \emph{identity} (i.e. $f^A(a) = a$ for all $a \in A$). Thus, we have that $\mathfrak{I}^\prime \models \varphi\frac{fy}{x}$, and further that $\mathfrak{I}^\prime \frac{f^A(\beta(y))}{x} \models \varphi$ by the Substitution Lemma. Hence $\mathfrak{I}^\prime \frac{\beta(y)}{x} \models \varphi$, since $f^A$ is identity. Yet at this point any random value can be assigned to $\beta(y)$, as is stated previously. In other words,\newline
\ \newline
for all $a \in A$, $(\mathfrak{I}^\prime\frac{a}{y})\frac{\mathfrak{I}^\prime\frac{a}{y}(y)}{x} \models \varphi$, and successively:\\
for all $a \in A$, $(\mathfrak{I}^\prime\frac{a}{x})\frac{\mathfrak{I}^\prime\frac{a}{x}(x)}{y} \models \varphi$;\\
for all $a \in A$, $(\mathfrak{I}^\prime\frac{a}{x}) \models \varphi$ (since $y$ does not occur in $\varphi$, and using Coincidence Lemma\footnote{Alternatively, this can be argued by Substitution Lemma, and using the fact that $\varphi\frac{x}{y} = \varphi$.}),\\
\ \newline
namely, $\mathfrak{I}^\prime \models \forall x \varphi$. Again, using the Coincidence Lemma, we obtain $\mathfrak{I} \models \forall x \varphi$.\nolinebreak\hfill$\talloblong$\newline

\textit{Remarks.}
\begin{enumerate}[(a)]
\item The last sequent rule $(*)$ seems not to be derivable as the first two are, though correct --- the reader may notice that we verified its correctness (in a semantical flavor) instead of providing a justification (which is syntactical in contrast).\\
\ \\
In light of this, I conjecture that there are some correct rules which are \emph{not} derivable,  as $(*)$ suggests. In some cases, a derivation fragment for the passage from one derivable sequent to another ($\Gamma \, \varphi\frac{fy}{x}$ to $\Gamma \, \forall x \varphi$ in this example) may vary according to the actual context in their common antecedent ($\Gamma$ in this example). That is to say, it is not always possible to carry out a derivation fragment for a correct sequent rule; a general form (one in which the antecedent is represented merely by a symbol such as $\Gamma$, which is too inconclusive in this kind of situation) for it may not exist.\\
\ \\
Nevertheless, it does not contradict the \emph{completeness} of $\mathfrak{S},$\footnote{For more details see Chapter V, which is entirely devoted to developing the Completeness Theorem.} which guarantees a derivation for a correct \emph{sequent} instead of a correct \emph{sequent rule}.
%%
\item
We shall investigate more about $(*)$. Consider the \emph{correct} rule below,\\ \newline
$(**)$ \hfill $\begin{array}{lll}
\Gamma & \varphi\frac{fy}{x} & \psi \cr \hline
\Gamma & \exists x \varphi & \psi
\end{array}$ if $f$ is unary, and $f$ and $y$ do not occur in $\Gamma \, \exists x \varphi \, \psi$, \hfill \ \\ \newline
which also seems not to be derivable. (The reader is encouraged to verify its correctness.) Interestingly, it is \emph{equivalent} to $(*)$. Here we say that two rules are equivalent if one is derivable after the addition of the other into $\mathfrak{S}$. The two derivations below confirm this:
\begin{enumerate}[(1)]
\item \textit{$(*)$ is given.}
\[
\begin{array}{lllll}
1. & \Gamma & \varphi\frac{fy}{x} & \psi & \mbox{premise} \cr
2. & \Gamma & \neg\psi & \neg\varphi\frac{fy}{x} & \mbox{(Cp) applied to 1.} \cr
3. & \Gamma & \neg\psi & \forall x \neg\varphi & \mbox{$(*)$ applied to 2., note that $f$ and $y$ do not} \cr
\ & \ & \ & \ & \mbox{occur in $\Gamma \, \neg\psi \, \forall x \neg\varphi$} \cr
4. & \Gamma & \neg\psi & \neg\varphi\frac{u}{x} & \mbox{5.5(a1) applied to 3., with $u$ not free in} \cr
\ & \ & \ & \ & \Gamma \, \exists x \varphi \, \psi \cr
5. & \Gamma & \varphi\frac{u}{x} & \psi & \mbox{(Cp) applied to 4.} \cr
6. & \Gamma & \exists x \varphi & \psi & \mbox{($\exists$A) applied to 5.}
\end{array}
\]
%%%
\item \textit{$(**)$ is given.}
\[
\begin{array}{lllll}
1. & \ & \ & v \equiv v & \mbox{$(\equiv)$, with $v \neq y$} \cr
2. & \Gamma & \ & v \equiv v & \mbox{(Ant) applied to 1.} \cr
3. & \Gamma & \ & \varphi\frac{fy}{x} & \mbox{premise} \cr
4. & \Gamma & v \equiv v & \varphi\frac{fy}{x} & \mbox{(Ant) applied to 3.} \cr
5. & \Gamma & \neg\varphi\frac{fy}{x} & \neg v \equiv v & \mbox{(Cp) applied to 4.} \cr
6. & \Gamma & \exists x \neg \varphi & \neg v \equiv v & \mbox{$(**)$ applied to 5., note that $f$ and $y$ do} \cr
\  & \ & \ & \ & \mbox{not occur in $\Gamma \, \exists x \neg \varphi \, \neg v \equiv v$} \cr
7. & \Gamma & v \equiv v & \forall x \varphi & \mbox{(Cp) applied to 6.} \cr
8. & \Gamma & \ & \forall x \varphi & \mbox{(Ch) applied to 2. and 7.}
\end{array}
\]
\end{enumerate}
The rules $(+)$ and $(++)$ below are obtained by replacing $fy$ by a suitable constant symbol $c$ in $(*)$ and $(**)$, respectively:\\
\begin{tabular}{lll}
$(+)$  & $\begin{array}{ll}
\Gamma & \varphi \frac{c}{x} \cr \hline
\Gamma & \forall x \varphi
\end{array}$ & if $c$ does not occur in $\Gamma \, \forall x \varphi$; \cr
$(++)$ & $\begin{array}{lll}
\Gamma & \varphi \frac{c}{x} & \psi \cr \hline
\Gamma & \exists x \varphi & \psi
\end{array}$ & if $c$ does not occur in $\Gamma \, \exists x \varphi \, \psi$
\end{tabular}
\ \\
Both $(+)$ and $(++)$ are correct and can be justified similarly,\footnote{We may apply \emph{extensions by definitions} (a technique which will be introduced in Chapter VIII) to verify it.} except a little easier.\footnote{Alternatively, there is a syntactic way to verify this: Take the constant symbol $c$ as a variable , say $y$; since by assumption $c$ does not occur in the conclusion sequents, there is only syntactic difference between $c$ and $y$. Now it should be easy to see that both $(+)$ and $(++)$ are correct.} Likewise, they are equivalent; the derivations establishing the equivalence can be obtained by slightly modifying those for the equivalence between $(*)$ and $(**)$.\\
\ \\
Also, the rules below turn out to be correct and equivalent:\\
\begin{tabular}{lll}
$(\circ)$      & $\begin{array}{ll}
\Gamma & \varphi \frac{t}{x} \cr \hline
\Gamma & \forall x \varphi
\end{array}$       & if $t$ is a term containing no symbols or \cr
\              & \ & variables occurring in $\Gamma \ \forall x \varphi$; \cr
$(\circ\circ)$ & $\begin{array}{lll}
\Gamma & \varphi \frac{t}{x} & \psi \cr \hline
\Gamma & \exists x \varphi   & \psi
\end{array}$       & if $t$ is a term containing no symbols or \cr
\              & \ & variables occurring in $\Gamma \ \exists x \varphi \ \psi$.
\end{tabular}
\ \\
The correctness of $(\circ)$ can be established in a similar way as the final rule of the exercise: Let $S$ be a symbol set such that $\Gamma$ is a sequent over $S$, and $S^\prime \supset S$ a symbol set such that $t$ is an $S^\prime$-term. For every $S$-interpretation $\INT = (\struct{A}, \beta)$, let $\INT^\prime$ be the $S^\prime$-interpretation in which
\begin{enumerate}[1)]
\item $\struct{A}^\prime |_S = \struct{A}$;
%%%
\item for every unary $f \in S^\prime \setminus S$ (if any) $f^{\struct{A}^\prime}$ is the identity function over $A$;
%%%
\item for every $n$-ary ($n > 1$) $f \in S^\prime \setminus S$ (if any) $f^{\struct{A}^\prime}$ is the function that projects its first argument. For example, if $n = 2$ then $f^{\struct{A}^\prime} (a, b) = a$ for all $a, b \in A$;
%%%
\item constant symbols $c \in S^\prime \setminus S$ (if any) are interpreted arbitrarily.
\end{enumerate}
It can be verified by induction that $\INT^\prime(t)$ equals either $\beta(v_i)$ for some $v_i$ occurring in $t$ or $c^{\struct{A}^\prime}$ for some $c$ occurring in $t$.\footnote{More precisely, it equals the value of the first non-function symbol (which is obviously either a variable or a constant symbol) in $t$ under $\INT^\prime$.} In such a way the remaining of the argument can be likewise conducted.\\
\ \\
The equivalence between $(\circ)$ and $(\circ\circ)$ can be verified as was the one between $(*)$ and $(**)$ above; hence the correctness of $(\circ\circ)$ is established.\\
\ \\
Notably the rules $(\circ)$ and $(\circ\circ)$ can be generalized to \emph{multiple quantifications}. For instance, by succesively applying $(\circ)$, we obtain\\
\begin{tabular}{lll}
$(\cdot)$ & $\begin{array}{ll}
\Gamma & \varphi \frac{t_1 \ldots t_n}{x_1 \ldots x_n} \cr \hline
\Gamma & \forall x_1 \ldots \forall x_n \varphi
\end{array}$  & if the symbols and variables occurring \cr
\         & \ & in $t_1, \ldots, t_n$ do not occur in $\Gamma \ \forall x_1 \ldots x_n \varphi$.
\end{tabular}
%%
\item Note that the sequent rules in $\mathfrak{S}$ are much like \emph{definitions}, while in contrast the derivable rules are much like \emph{theorems}, in the sense that the former all together \emph{defines} a set of sequents whereas the latter \emph{guarantees} some sequents are in that set with the fact that derivable rules all consist of (finite) applications of sequent rules in $\mathfrak{S}$. Thus, derivable rules are somewhat like \emph{shortcuts} in a certain viewpoint.\\
\\
We conclude these remarks by giving another correct rule (which also seems not to be derivable):
\[
\begin{array}{ll}
\ & \ \cr\hline
\Gamma & \varphi
\end{array},
\]
where
\[
\varphi := \begin{cases}
v_0 \equiv v_0, & \mbox{if \(\Gamma\) contains an even number of formulas}; \cr
v_1 \equiv v_1, & \mbox{otherwise}.
\end{cases}
\]
A general derivation is unlikely to exist, as the succedent varies according to the parity of the number of formulas in the antecedent.\nolinebreak\hfill$\talloblong$
\end{enumerate}
%End of IV.4.5-------------------------------------------------------------------------------------
%
\item \textbf{Note to Exercise 4.5.} Similar to the first derivable rule, the following one is also derivable:
\[
\begin{array}{ll}
\varphi & \psi \cr \hline
\forall x \varphi & \forall x \psi
\end{array}
\]
\\
\textit{Justification.}
\[
\begin{array}{llll}
1. & \varphi & \psi & \mbox{premise} \\
2. & \forall x \varphi & \psi & \mbox{5.5(b3) applied to 1.} \\
3. & \forall x \varphi & \forall x \psi & \mbox{5.5(b4) applied to 2.}
\end{array}
\]
\end{enumerate}
%End of Section IV.4-----------------------------------------------------------------------------------------
\ 
\\
\\
%Section IV.5------------------------------------------------------------------------------------------------
{\large \S5. Further Derivable Rules and Sequents}
\begin{enumerate}[1.]
%IV.5.5--------------------------------------------------------------------------------------------
\item \textbf{Solution to Exercise 5.5.}
\begin{itemize}
\item[(a1)]
\[
\begin{array}{lllll}
1. & \Gamma & \, & \neg \exists x \neg \varphi & \mbox{premise} \\
2. & \Gamma & \neg \varphi \frac{t}{x} & \neg \varphi \frac{t}{x} & \mbox{(Assm)} \\
3. & \Gamma & \neg \varphi \frac{t}{x} & \exists x \neg \varphi & \mbox{($\exists$S) applied to 2.} \\
4. & \Gamma & \neg \varphi \frac{t}{x} & \neg \exists x \neg \varphi & \mbox{(Ant) applied to 1.} \\
5. & \Gamma & \, & \varphi \frac{t}{x} & \mbox{(Ctr) applied to 3. and 4.}
\end{array}
\]
%%
\item[(a2)] Immediately follows from (a1).
%%
\item[(b1)]
\[
\begin{array}{lllll}
1. & \Gamma & \varphi \frac{t}{x} & \psi & \mbox{premise} \\
2. & \Gamma & \neg \psi & \neg \varphi \frac{t}{x} & \mbox{(Cp) applied to 1.} \\
3. & \Gamma & \neg \psi & \exists x \neg \varphi & \mbox{($\exists$S) applied to 2.} \\
4. & \Gamma & \neg \exists x \neg \varphi & \psi & \mbox{(Cp) applied to 3.}
\end{array}
\]
%%
\item[(b2)]
\[
\begin{array}{lllll}
1. & \Gamma & \, & \varphi \frac{y}{x} & \mbox{premise} \\
2. & \Gamma & \exists x \neg \varphi & \varphi \frac{y}{x} & \mbox{(Ant) applied to 1.} \\
3. & \Gamma & \neg \varphi \frac{y}{x} & \neg \exists x \neg \varphi & \mbox{(Cp) applied to 2.} \\
4. & \Gamma & \exists x \neg \varphi & \neg \exists x \neg \varphi & \mbox{($\exists$A) applied to 3.} \\
\, & \, & \, & \, & \mbox{(note that $y$ is not free in $\exists x \neg \varphi$)} \\
5. & \Gamma & \neg \exists x \neg \varphi & \neg \exists x \neg \varphi & \mbox{(Assm)} \\
6. & \Gamma & \, & \neg \exists x \neg \varphi & \mbox{(PC) applied to 4. and 5.}
\end{array}
\]
\textit{Remark.} Generally we cannot derive $\Gamma \ \forall x \varphi$ from $\Gamma \ \varphi\frac{t}{x}$ where $t$ is a \emph{term}, even if all variables in $\var(t)$ do not appear free in $\Gamma$. For a counter example, consider the case: $\Gamma = c \equiv y$, $\varphi = x \equiv y$ and $t = c$. However, if all the \emph{symbols} in $t$ do not occur in $\Gamma$, then $\Gamma \ \forall x \varphi$ can be derived. (See the third derivable rules in Exercise 4.5.)
%%
\item[(b3)] Immediately follows from (b1).
%%
\item[(b4)] Immediately follows from (b2).
\end{itemize} \begin{flushright}$\talloblong$\end{flushright}
%End of IV.5.5-------------------------------------------------------------------------------------
%
\item \textbf{Derivable Rules.}
\begin{enumerate}[(a)]
\item
\[
\begin{array}{ll}
\Gamma & \exists x \varphi \cr\hline
\Gamma & \exists x \neg\neg\varphi
\end{array}
\]
\textit{Justification.}
\[
\begin{array}{lllll}
1. & \Gamma & \ & \exists x \varphi & \mbox{premise} \cr
2. & \ & \varphi & \varphi & \mbox{(Assm)} \cr
3. & \ & \varphi & \neg\neg\varphi & \mbox{3.6(a1) applied to 2.} \cr
4. & \ & \exists x \varphi & \exists x \neg\neg\varphi & \mbox{4.5 applied to 3.} \cr
5. & \Gamma & \exists x \varphi & \exists x \neg\neg\varphi & \mbox{(Ant) applied to 4.} \cr
6. & \Gamma & \ & \exists x \neg\neg\varphi & \mbox{(Ch) applied to 1. and 5.}
\end{array}
\]
%%
\item
\[
\begin{array}{ll}
\Gamma & \exists x \neg\neg\varphi \cr\hline
\Gamma & \exists x \varphi
\end{array}
\]
\textit{Justification.}
\[
\begin{array}{lllll}
1. & \Gamma & \ & \exists x \neg\neg\varphi & \mbox{premise} \cr
2. & \ & \neg\neg\varphi & \neg\neg\varphi & \mbox{(Assm)} \cr
3. & \ & \neg\neg\varphi & \varphi & \mbox{3.6(a2) applied to 2.} \cr
4. & \ & \exists x \neg\neg\varphi & \exists x \varphi & \mbox{4.5 applied to 3.} \cr
5. & \Gamma & \exists x \neg\neg\varphi & \exists x \varphi & \mbox{(Ant) applied to 4.} \cr
6. & \Gamma & \ & \exists x \varphi & \mbox{(Ch) applied to 1. and 5.}
\end{array}
\]
%%
\item
\[
\begin{array}{ll}
\Gamma & \forall x \varphi \cr\hline
\Gamma & \forall x \neg\neg\varphi
\end{array}
\]
\textit{Justification.}
\[
\begin{array}{lllll}
1. & \Gamma & \ & \forall x \varphi & \mbox{premise} \cr
2. & \ & \varphi & \varphi & \mbox{(Assm)} \cr
3. & \ & \varphi & \neg\neg\varphi & \mbox{3.6(a1) applied to 2.} \cr
4. & \ & \forall x \varphi & \neg\neg\varphi & \mbox{5.5(b3) applied to 3.} \cr
5. & \ & \forall x \varphi & \forall x \neg\neg\varphi & \mbox{5.5(b4) applied to 4.} \cr
6. & \Gamma & \forall x \varphi & \forall x \neg\neg\varphi & \mbox{(Ant) applied to 5.} \cr
7. & \Gamma & \ & \forall x \neg\neg\varphi & \mbox{(Ch) applied to 1. and 6.}
\end{array}
\]
%%
\item
\[
\begin{array}{ll}
\Gamma & \forall x \neg\neg\varphi \cr\hline
\Gamma & \forall x \varphi
\end{array}
\]
\textit{Justification.}
\[
\begin{array}{lllll}
1. & \Gamma & \ & \forall x \neg\neg\varphi & \mbox{premise} \cr
2. & \ & \neg\neg\varphi & \neg\neg\varphi & \mbox{(Assm)} \cr
3. & \ & \neg\neg\varphi & \varphi & \mbox{3.6(a2) applied to 2.} \cr
4. & \ & \forall x \neg\neg\varphi & \varphi & \mbox{5.5(b3) applied to 3.} \cr
5. & \ & \forall x \neg\neg\varphi & \forall x \varphi & \mbox{5.5(b4) applied to 4.} \cr
6. & \Gamma & \forall x \neg\neg\varphi & \forall x \varphi & \mbox{(Ant) applied to 5.} \cr
7. & \Gamma & \ & \forall x \varphi & \mbox{(Ch) applied to 1. and 6.}
\end{array}
\]
\end{enumerate}
%
\item \textbf{Note to Exercise III.4.11(c).} Here we give a derivation for this. In the following, let $x \not \in \free(\varphi)$.
\begin{enumerate}[(i)]
\item $\forall x (\varphi \lor \psi) \derives (\varphi \lor \forall x \psi)$:
\[
\begin{array}{llll}
1. & \forall x (\varphi \lor \psi) \;\; \varphi & \varphi & \mbox{(Assm)} \\
2. & \forall x (\varphi \lor \psi) \;\; \varphi & (\varphi \lor \forall x \psi) & \mbox{($\lor$S) applied to 1.} \\
3. & \forall x (\varphi \lor \psi) \;\; \neg \varphi & \forall x (\varphi \lor \psi) & \mbox{(Assm)} \\
4. & \forall x (\varphi \lor \psi) \;\; \neg \varphi & (\varphi \lor \psi) & \mbox{5.5(a2) applied to 3.} \\
5. & \forall x (\varphi \lor \psi) \;\; \neg \varphi & \neg \varphi & \mbox{(Assm)} \\
6. & \forall x (\varphi \lor \psi) \;\; \neg \varphi & \psi & \mbox{3.4 applied to 4. and 5.} \\
7. & \forall x (\varphi \lor \psi) \;\; \neg \varphi & \forall x \psi & \mbox{5.5(b4) applied to 6.} \\
8. & \forall x (\varphi \lor \psi) \;\; \neg \varphi & (\varphi \lor \forall x \psi) & \mbox{($\lor$S) applied to 7.} \\
9. & \forall x (\varphi \lor \psi) & (\varphi \lor \forall x \psi) & \mbox{(PC) applied to 2. and 8.}
\end{array}
\]
%%%
\item $(\varphi \lor \forall x \psi) \derives \forall x (\varphi \lor \psi)$:
\[
\begin{array}{llll}
1. & (\varphi \lor \forall x \psi) \;\; \varphi & \varphi & \mbox{(Assm)} \\
2. & (\varphi \lor \forall x \psi) \;\; \varphi & (\varphi \lor \psi) & \mbox{($\lor$S) applied to 1.} \\
3. & (\varphi \lor \forall x \psi) \;\; \neg \varphi & (\varphi \lor \forall x \psi) & \mbox{(Assm)} \\
4. & (\varphi \lor \forall x \psi) \;\; \neg \varphi & \neg \varphi & \mbox{(Assm)} \\
5. & (\varphi \lor \forall x \psi) \;\; \neg \varphi & \forall x \psi & \mbox{3.4 applied to 3. and 4.} \\
6. & (\varphi \lor \forall x \psi) \;\; \neg \varphi & \psi & \mbox{5.5(a2) applied to 5.} \\
7. & (\varphi \lor \forall x \psi) \;\; \neg \varphi & (\varphi \lor \psi) & \mbox{($\lor$S) applied to 6.} \\
8. & (\varphi \lor \forall x \psi) & (\varphi \lor \psi) & \mbox{(PC) applied to 2. and 7.} \\
9. & (\varphi \lor \forall x \psi) & \forall x (\varphi \lor \psi) & \mbox{5.5(b4) applied to 8.}
\end{array}
\]
\end{enumerate}
\end{enumerate}
%End of Section IV.5-----------------------------------------------------------------------------------------
\ 
\\
\\
%Section IV.6------------------------------------------------------------------------------------------------
{\large \S6. Summary and Example}
\begin{enumerate}[1.]
\item \textbf{Note to the Rule (Assm).} We are going to show that (Assm) is not \emph{redundant}, i.e. it cannot be derived from other rules in $\mathfrak{S}$. Assume that $S = \{ R \}$, where $R$ is a unary relation symbol, and let $\mathfrak{S}^- := \mathfrak{S} \setminus \{ \mbox{(Assm)} \}$. We restrict ourselves to show that the correct sequent
\[
Rx \ Rx
\]
is not derivable in $\mathfrak{S}^-$. (Note that $Rx$ is not valid.)\newline
\\
We assert this by proving the following claim:\newline
\\
\textbf{Claim.} \textit{For every derivable sequent in $\mathfrak{S}^-$, there is at least one subformula of the form $t \equiv t$ not within the scope of $\neg$ \footnote{We say that a subformula $t \equiv t$ of $\varphi$ is within the scope of $\neg$ iff $t \equiv t$ is contained in some formula $\neg\chi$ which in turn is also a subformula of $\varphi$. For example, $x \equiv y$ is within the scope of $\neg$ in $(Rz \land x \equiv y)$ since $(Rz \land x \equiv y)$ is taken to be an abbreviation for $\neg(\neg Rz \lor \neg x \equiv y)$; however, it is not so in $(x \equiv y \lor Ry)$.} in the succedent, where $t$ is a term.}\newline
\\
\textit{Proof.} We show by induction on the rules of $\mathfrak{S}^-$ that the succedent of the sequent thus generated has at least one subformula of the form $t \equiv t$ that is not within the scope of $\neg$.
\begin{enumerate}[(1)]
\item For ($\equiv$), it is trivial.
%%
\item For (Ant), (PC), ($\lor\mathrm{A}$), ($\lor\mathrm{S}$), ($\exists\mathrm{A}$), ($\exists\mathrm{S}$) and (Sub), it is trivial, as the succedent of the conclusion sequent is the same as, or a subformula of, that of the premise sequent. For (Ctr), note that the succedent of the second premise sequent must not have such a subformula whether or not the one in the first does, therefore the statement ``if the premise has this property, then so does the conclusion'' holds.\footnote{As a result, the rule (Ctr) is never used in any derivation carried out in $\mathfrak{S}^-$ since it demands as the second premise sequent a sequent of which the succedent contains no subformula $t \equiv t$ outside the scope of $\neg$, which is impossible.}\nolinebreak\hfill$\talloblong$
\end{enumerate}
%
\item \textbf{Note to the Concept of $\derives$.} Let $S$ be a symbol set, $\Phi$ a set of $S$-formulas, and $\varphi$ an $S$-formula. The problem of telling whether $\Phi \derives \varphi$ is undecidable, i.e. there is no algorithm for derivability.
%
\item \textbf{Note to the Correctness Theorem.} It is termed the \emph{Soundness Theorem} in many other textbooks.
%
\item \textbf{Note to the Derivable Rule (Comm).} In part (a) of \textbf{Derivable Rules} in the notes to Section 3, we proved $(\mathrm{Comm})$, the Commutativity Rule for $\lor$:
\[
\begin{array}{ll}
\Gamma & (\varphi \lor \psi) \cr\hline
\Gamma & (\psi \lor \varphi)
\end{array}
\]
We shall show that it is \emph{equivalent} to both parts (a) and (b) of $(\lor\mathrm{S})$, in which by `equivalent' here we mean that one rule can be derived from the other (and of course, possibly with the help of other rules in $\mathfrak{S}$). More specifically, if two among $(\mathrm{Comm})$ and parts (a) and (b) of $(\lor\mathrm{S})$ are given, then the other can be derived.\newline
\\
For the case of deriving $(\mathrm{Comm})$ from other two rules, we have already done this before. As for the case of deriving part (b) of $(\lor\mathrm{S})$ given other two, the derivation is provided:
\[
\begin{array}{llll}
1. & \Gamma & \varphi & \mbox{premise} \cr
2. & \Gamma & (\varphi \lor \psi) & \mbox{part $(\mathrm{a})$ of $(\lor\mathrm{S})$ applied to 1.} \cr
3. & \Gamma & (\psi \lor \varphi) & \mbox{$(\mathrm{Comm})$ applied to 2.}
\end{array}
\]
The case for deriving part (a) is symmetric.
%
\item \textbf{Note to the Sequent Rules $(\lor\mathrm{S})$ in $\mathfrak{S}$.} After being introduced to $\mathfrak{S}$, the reader may notice that $(\lor\mathrm{S})$ is different from other rules in $\mathfrak{S}$: It is the only rule that consists of two parts. In this setting, $(\mathrm{Comm})$ can be derived (see part (a) of \textbf{Derivable Rule} in the notes to Section 3), with both parts of $(\lor\mathrm{S})$ playing critical roles. Conversely, both parts (a) and (b) of $(\lor\mathrm{S})$ can be derived from $(\mathrm{Comm})$ provided that the other is given, as we discussed in \textbf{Note to the Derivable Rule (Comm)}.\newline
\\
Some natural questions thus arise: Can part $(\mathrm{a})$ be derived from other rules in $\mathfrak{S}$? How about part $(\mathrm{b})$? Can $(\mathrm{Comm})$ still be derived with either part of $(\lor\mathrm{S})$ omitted? If otherwise both parts of $(\lor\mathrm{S})$ are replaced by $(\mathrm{Comm})$ alone, can they be derived from it? All the answers are ``No'', by the following argument.\newline
\\
Assume $S$ to be fixed. We shall denote by $\mathfrak{S}^-$ a set of sequent rules, with its contents varying according to the case.
\begin{enumerate}[(1)]
\item \emph{Part $(\mathrm{b})$ is omitted.} \begin{math}\mathfrak{S}^- := \mathfrak{S} \setminus \{\mbox{part (b) of $(\lor\mathrm{S})$}\}\end{math}. Observe that $\lor$ is merely a syntactic object, which carries no internal semantics, and hence is not necessarily interpreted as \emph{disjunction}: Within $\mathfrak{S}^-$, $\lor$ can otherwise be interpreted as \emph{projection1}.\footnote{\emph{projection of the first argument}, for which we adopt the symbol $\mathbf{proj}_1$: For all $\varphi$, $\psi$ in $L^S$, $(\varphi \ \mathbf{proj}_1 \ \psi) \bimodels \varphi$.} The reader is encouraged to verify that disjunction and projection1 are the only two among 16 possible interpretations for $\lor$ that establishes the correctness of $\mathfrak{S}^-$:
\begin{center}
For all $\Phi$ and $\varphi$, if $\Phi \derives_{\mathfrak{S}^-} \varphi$\footnote{Here we use subscript $_{\mathfrak{S}^-}$ to clarify the notion of \emph{derivability within $\mathfrak{S}^-$}.} then $\Phi \models_{\lor : \mbox{\scriptsize disjunction}} \varphi$,\footnote{Likewise, we use the subscripts $_{\lor : \mbox{\tiny disjunction}}$ and $_{\lor : \mbox{\tiny projection1}}$ to distinguish the notions of \emph{consequence relation with respect to the disjunction-} and \emph{projection1-interpretations of $\lor$}.}
\end{center}
and
\begin{center}
For all $\Phi$ and $\varphi$, if $\Phi \derives_{\mathfrak{S}^-} \varphi$ then $\Phi \models_{\lor : \mbox{\scriptsize projection1}} \varphi$.
\end{center}
\ \newline
Since $\mathfrak{S}^-$ consists of nothing but syntactic operations (as does $\mathfrak{S}$), it is clear that whether $\lor$ is interpreted as disjunction or projection1 is irrelevant as far as those rules in $\mathfrak{S}^-$ are concerned.\newline
\\
Let us focus on the projection1-interpretation of $\lor$: Part (b) of $(\lor\mathrm{S})$ from $\mathfrak{S}$ is \emph{not} correct , because introducing an arbitrary formula $\psi$ to $\varphi$ as the first component of projection1 given $\varphi$ is generally not valid; likewise, $(\mathrm{Comm})$ is also not correct. (In fact, since the equivalence between them was established in \textbf{Note to the Derivable Sequent (Comm)}, we know that they are correct or not correct at the same time.)\newline
\\
Therefore, neither rules are derivable within $\mathfrak{S}^-$, since $\mathfrak{S}^-$ is correct under the projection1-interpretation of $\lor$, as we just claimed.
%%
\item \emph{Part $(\mathrm{a})$ is omitted.} \begin{math}\mathfrak{S}^- := \mathfrak{S} \setminus \{\mbox{part (a) of $(\lor\mathrm{S})$}\}\end{math}. The argument is symmetric to the previous case.
%%
\item \emph{Both parts $(\mathrm{a})$ and $(\mathrm{b})$ are replaced by $(\mathrm{Comm})$.} \begin{math}\mathfrak{S}^- := (\mathfrak{S} \cup \{(\mathrm{Comm})\}) \setminus \{(\lor\mathrm{S})\}\end{math}. In this case, $\lor$ can be interpreted as \emph{absurdity},\footnote{Or \emph{falsum}, \emph{contradiction}, for which we adopt the symbol $\bot$: For all $\varphi$, $\psi$, $\chi$ in $L^S$, $(\varphi \bot \psi) \bimodels (\chi \land \neg \chi)$.} \emph{conjunction} or \emph{exclusive disjunction}, in addition to disjunction. Neither $(\mathrm{a})$ nor $(\mathrm{b})$ is correct under these interpretations, except under the disjunction-interpretation. The argument is similar.
\end{enumerate}
\emph{Remark}. The completeness of $\mathfrak{S}^-$ obviously does not hold in all three cases, under the usual disjunction-interpretation of $\lor$. On the other hand, $\lor$ can only be interpreted as disjunction in $\mathfrak{S}$ regarding the correctness.
%
%%BEGIN: the item below is subjected to doubtedness, hence commented by now
%\item \textbf{Note to the Theorem on the Correctness of $\mathfrak{S}$ 6.2.} A better statement for this theorem might be
%\[
%\mbox{``\emph{For all $\Gamma$ and $\varphi$, if $\Gamma \derives \varphi$ then $\Gamma \models \varphi$.}''}
%\]
%For the original one in text, namely
%\[
%\mbox{``\emph{For all $\Phi$ and $\varphi$, if $\Phi \derives \varphi$ then $\Phi \models \varphi$}''}
%\]
%is indeed the \emph{Theorem on the Soundness of First-Order Logic}. And the proof given in text is essentially the one for the former statement, except for the premise $\Phi \derives \varphi$ and the conclusion $\Phi \models \varphi$.\\
%\\
%However, there is only a slight difference between them, as $\Phi \derives \varphi$ means $\Gamma \derives \varphi$ for some $\Gamma \subset \Phi$ by definition and meanwhile $\Gamma \models \varphi$ obviously implies $\Phi \models \varphi$. (It turns out that the Correctness Theorem implies the Soundness Theorem.) Therefore this problem can be ignored.
%END
%
\item \textbf{A Derivation Over $\mathfrak{S}$ May Be Obtained in Various Ways.} For some derivations, there is no unique sequence of applications of sequent rules from $\mathfrak{S}$. For example, we can obtain the derivation
\[
\begin{array}{lll}
1. & \ & 0 \equiv 0 \cr
2. & 0 \equiv 0 & 0 \equiv 0
\end{array}
\]
by either (1) Apply $\eq$, and then apply $\ant$ to the sequent just derived; or (2) Apply $\eq$, and then apply $\assm$.\\
\ \\
\textit{Remark.} One may notice that in contrast, terms and formulas are obtained in unique ways. Such discrepancy arises from the fact that the rules of $\mathfrak{S}$ are not set up according to disjoint cases, unlike those of calculi of terms and formulas.
\end{enumerate}
%End of Section IV.6-----------------------------------------------------------------------------------------
\ 
\\
\\
\newpage\noindent
%Section IV.7--------------------------------------------------------------------------------------
{\large \S7. Consistency}
\begin{enumerate}[1.]
\item \textbf{Note to the Concept of Consistency.} As an immediate result of \textbf{Note to the Concept of $\derives$} mentioned in the annotations of last section, the problem of telling whether a set $\Phi$ of formulas is consistent is undecidable.
%
\item \textbf{Proposition:} \textit{Let $S \subset S^\prime$ and $\Phi \subset L^S$. \textit{If $\con_{S^\prime} \Phi$ then $\con_S \Phi$.}}\\
\textit{Proof.} Since $\inc_S \Phi$ implies $\inc_{S^\prime} \Phi$.\nolinebreak\hfill$\talloblong$
%
\item \textbf{Note to Lemma 7.6.} In the proof of (a), note that in the case that $\Phi \cup \{ \neg \varphi \}$ is inconsistent, there is some suitable $\Gamma$ such that the sequent $\Gamma \; \varphi$ is derivable. Such a $\Gamma$ may or may not contain $\neg \varphi$. However, in either cases the sequent $\Gamma \; \neg \varphi \; \varphi$ is derivable (by the rule (Ant)), as stated in text.\\
\\
On the other hand, the implication stated in (c) should, in fact, be \emph{bi-directional}, as the implication ``If $\con \Phi \cup \{ \varphi \}$ or $\con \Phi \cup \{ \neg \varphi \}$, then $\con \Phi$'' is trivial: Its contraposition is ``If $\inc \Phi$, then $\inc \Phi \cup \{ \varphi \}$ and $\inc \Phi \cup \{ \neg \varphi \}$''.
%
\item \textbf{Note to Lemma 7.5.} In fact, it is an equivalent statement of the Correctness Theorem. The proof of this lemma given in text shows that it follows from the Correctness Theorem. Here we prove the converse, namely that the Correctness Theorem holds assuming this lemma: For $\Phi$ and $\varphi$, we have
\begin{center}
\begin{tabular}{lll}
\    & $\Phi \derives \varphi$ & \ \cr
iff  & $\inc \Phi \cup \{ \neg\varphi \}$ & (by 7.6(a)) \cr
then & not $\sat \Phi \cup \{ \neg\varphi \}$ & (by premise) \cr
iff  & $\Phi \models \varphi$ & (by III.4.4).
\end{tabular}
\end{center}
%
\item \textbf{Proposition:} \textit{Let $S$ and $S^\prime$ be symbol sets with $S \subset S^\prime$, and $\Phi \subset L^S$ a set of formulas. If $\inc_S \, \Phi$ then $\inc_{S^\prime} \, \Phi$.}\\
\textit{Proof.} Since for an $S$-formula $\varphi$, $\Phi \derives_S \varphi$ implies $\Phi \derives_{S^\prime} \varphi$. \begin{flushright}$\talloblong$\end{flushright}
%
%IV.7.8--------------------------------------------------------------------------------------------
\item \textbf{Solution to Exercise 7.8.}
\begin{enumerate}[(a)]
\item No.\\
\textit{Reason.} Suppose that ($\exists\forall$) is a derivable rule. Then it makes no changes to the sequent calculus $\mathfrak{S}$ if we add this rule to it, except introducing some convenience for deriving sequents (cf. IV.3 in text).\\
\\
However, the following derivation showing that $\derives \forall x \forall y \ x \equiv y$ (assuming the symbol set $S$ to be fixed) and hence (by the Theorem on the Correctness of $\mathfrak{S}$) that $\models \forall x \forall y \ x \equiv y$ is a contradiction, as that sentence is indeed \emph{not} valid (it is not satisfied by any structure whose domain is not a singleton set):\\
\[
\begin{array}{llll}
1. & \ & x \equiv x & \mbox{($\equiv$)}\\
2. & \ & \exists y \ x \equiv y & \mbox{($\exists$S) applied to 1. with $t = x$}\\
3. & \exists y \ x \equiv y & \forall y \ x \equiv y & \mbox{($\exists\forall$)}\\
4. & \ & \forall y \ x \equiv y & \mbox{(Ch) applied to 2. and 3.}\\
5. & \ & \forall x \forall y \ x \equiv y & \mbox{5.5(b4) applied to 4.}
\end{array}
\]
It turns out that ($\exists\forall$) is not a derivable rule.
%%
\item No.\\
\textit{Reason.} Suppose that every sequent is derivable in $\mathfrak{S}^\prime$ (again, assuming $S$ to be fixed); in particular, the sequent $\forall x \forall y \  x \equiv y \ \neg \forall x \forall y \ x \equiv y$ is also derivable in $\mathfrak{S}^\prime$.\\
\\
Assume that we are given a derivation of the sequent mentioned above in $\mathfrak{S}^\prime$. We shall show below that we can construct in two stages a derivation of that sequent using only the rules of $\mathfrak{S}$ by modifying the given one. This, when appended to with the fragment of derivation below (assuming there are $p$ steps in the previous derivation), yields a derivation for $\derives \ \neg\forall x \forall y \ x \equiv y$ in $\mathfrak{S}$, which contradicts the correctness of $\mathfrak{S}$:
\[
\begin{array}{llll}
(p + 1). & \neg\forall x \forall y \ x \equiv y & \neg\forall x \forall y \ x \equiv y & \mbox{(Assm)} \cr
(p + 2). & \ & \neg\forall x \forall y \ x \equiv y & \mbox{(PC) applied to $p.$ and} \cr
\ & \ & \ & \mbox{$(p + 1).$}
\end{array}
\]
\ \\
At the first stage, we \emph{prepend} the sentence $\forall x \forall y \ x \equiv y$ to the antecedent $\Gamma$ of the sequent $\Gamma \varphi$ in each step of the derivation that is not resulted by the rule ($\equiv$) (which has no antecedent) or that does not contain that sentence (i.e. $\forall x \forall y \ x \equiv y \not \in \Gamma$). More specifically, if in the $m$th step the sequent is
\[
\begin{array}{lll}
m. & \Gamma & \varphi
\end{array}
\]
and satisfies the aforementioned conditions, then it becomes
\[
\begin{array}{lll}
m. & \Gamma^\prime & \varphi,
\end{array}
\]
in which $\Gamma^\prime := \forall x \forall y \ x \equiv y \ \Gamma$, after this modification. The reader could verify that it remains a derivation of $\forall x \forall y \ x \equiv y \ \neg \forall x \forall y \ x \equiv y$ (in $\mathfrak{S}^\prime$) by examining the rules of $\mathfrak{S}^\prime$.\\
\\
At the second stage, we replace in the derivation just obtained all the steps resulted by ($\exists\forall$) (assuming the sequents therein are of the form $\Gamma^\prime \ \exists x \varphi \ \forall x \varphi$, where $\Gamma^\prime$ contains $\forall x \forall y \ x \equiv y$ as stated in the previous stage) with the following derivation:\\ \ \\
\(
\begin{array}{lllll}
n.     & \forall x \forall y \ x \equiv y & \ & \ & \forall x \forall y \ x \equiv y \cr
\      & \multicolumn{4}{l}{\mbox{(Assm)}} \cr
(n+1). & \forall x \forall y \ x \equiv y & \ & \ & \forall y \ u \equiv y \cr
\      & \multicolumn{4}{l}{\mbox{5.5(a1) applied to $n.$, } u \not \in \free(\varphi)} \cr
(n+2). & \forall x \forall y \ x \equiv y & \ & \ & u \equiv v \cr
\      & \multicolumn{4}{l}{\mbox{5.5(a1) applied to $(n+1).$, } u \neq v \not \in \free(\varphi)} \cr
(n+3). & \forall x \forall y \ x \equiv y & \varphi\frac{u}{x} & \ & u \equiv v \cr
\      & \multicolumn{4}{l}{\mbox{(Ant) applied to $(n+2).$}} \cr
(n+4). & \forall x \forall y \ x \equiv y & \varphi\frac{u}{x} & \ & \varphi\frac{u}{x} \cr
\      & \multicolumn{4}{l}{\mbox{(Assm)}} \cr
(n+5). & \forall x \forall y \ x \equiv y & \varphi\frac{u}{x} & u \equiv v & \varphi\frac{v}{x} \cr
\      & \multicolumn{4}{l}{\mbox{(Sub) applied to $(n+4).$}} \cr
(n+6). & \forall x \forall y \ x \equiv y & \varphi\frac{u}{x} & \ & \varphi\frac{v}{x} \cr
\      & \multicolumn{4}{l}{\mbox{(Ch) applied to $(n+3).$ and $(n+5).$}} \cr
(n+7). & \forall x \forall y \ x \equiv y & \varphi\frac{u}{x} & \ & \forall x \varphi \cr
\      & \multicolumn{4}{l}{\mbox{5.5(b2) applied to $(n+6).$}} \cr
(n+8). & \forall x \forall y \ x \equiv y & \exists x \varphi & \ & \forall x \varphi \cr
\      & \multicolumn{4}{l}{\mbox{($\exists$A) applied to $(n+7).$}}
\end{array}
\)\\ \ \\
and additionally the sequent:\\ \ \\
\(
\begin{array}{lllll}
(n+9). & \Gamma^\prime\phantom{\forall y \ x \equiv y } \, & \exists x \varphi \, & \phantom{u \equiv v} & \forall x \varphi \cr
\      & \multicolumn{4}{l}{\mbox{(Ant) applied to $(n+8).$}}
\end{array}
\)\\ \ \\
if $\Gamma^\prime$ contains sequents other than $\forall x \forall y \ x \equiv y$. The derivation thus obtained uses only the rules of $\mathfrak{S}$.
\end{enumerate} \begin{flushright}$\talloblong$\end{flushright}
%End of IV.7.8-----------------------------------------------------------------------------------------------
\end{enumerate}
%End of Section IV.7-----------------------------------------------------------------------------------------
%End of Chapter IV-------------------------------------------------------------------------------------------
%%Chapter V---------------------------------------------------------------------------------------------------
{\LARGE \bfseries V \\ \\ The Completeness Theorem}
\\
\\
\\
%Section V.0-------------------------------------------------------------------------------------------------
{\large \S \ Prolog}
\begin{enumerate}[1.]
\item \textbf{Note to the Prolog of Chapter V in Page 75.} An alternative \emph{proof} for that $(*)$ follows from $(**)$ in page 75 is provided here: Assuming $(**)$. If $\Phi \models \varphi$, then $\Phi \cup \{ \neg \varphi \}$ is not satisfiable by III.4.4. Thus by hypothesis we have that $\Phi \cup \{ \neg \varphi \}$ is inconsistent, i.e. $\Phi \vdash \varphi$ from IV.7.6(a). \begin{flushright}$\talloblong$\end{flushright}
%
\item \textbf{Proposition:} \textit{$(**)$ also follows from $(*)$.}\\
\emph{Proof.} Assuming $(*)$. If $\Phi$ is not satisfiable, then clearly $\Phi \models \varphi$ for all $\varphi$ since it has \emph{no} models. It follows from the hypothesis that $\Phi \vdash \varphi$ for all $\varphi$, i.e. $\Phi$ is inconsistent. \begin{flushright}$\talloblong$\end{flushright}
\end{enumerate}
%End of Section V.0------------------------------------------------------------------------------------------
\ 
\\
\\
%Section V.1-------------------------------------------------------------------------------------------------
{\large \S1. Henkin's Theorem}
\begin{enumerate}[1.]
\item \textbf{Note to the Proof of Lemma 1.7.} The third sufficient and necessary condition (``iff'') in (b) follows from the property of equivalence relation. In case (ii) of (c),
\[
\forall x_1 \ldots \forall x_n \varphi
\]
can be regarded as
\[
\neg \exists x_1 \ldots \exists x_n \neg \varphi
\]
(cf. III.4) and hence the result immediately follows from (i).
%
\item \textbf{Note to the Discussion before Definition 1.8 in Page 78.} Note that $S = \{ R \}$ contains neither function symbols nor constant symbols, so there are only variables $\in \mathcal{A}$ in the domain $T^{\Phi}$ of $\mathfrak{I}^{\Phi}$.
%
\item \textbf{Proof to Corollary 1.11.} Apply Henkin's Theorem 1.10 to formulas in $\Phi$ and its term interpretation $\mathfrak{I}^\Phi$ (notice that $\Phi \models \varphi$ for all $\varphi \in \Phi$). \begin{flushright}$\talloblong$\end{flushright}
%
%V.1.12---------------------------------------------------------------------------------------------------
\item \textbf{Solution to Exercise 1.12.}
\begin{enumerate}[(a)]
\item
\begin{enumerate}[(i)]
\item Let $\mathfrak{I} = (\mathfrak{A}, \beta)$ be an interpretation of $\Phi$ such that $A = \{ a_0, a_1 \}$, and for all $x \in \mathcal{A}$,
\[
\beta(x) := a_1,
\]
and
\[
\mbox{$R^{\mathfrak{A}}a_0$ and not $R^{\mathfrak{A}}a_1$}.
\]
It is easy to see that $\mathfrak{I} \models \Phi$ and hence $\Phi$ is satisfiable.\\
\ 
\\As it turns out, $\Phi$ is also consistent, for otherwise there would be a formula $\varphi$ such that $\Phi \vdash \varphi$ and $\Phi \vdash \neg \varphi$ (by IV.7.1(b)), which by IV.6.2 implies that $\Phi \models \varphi$ and $\Phi \models \neg \varphi$, contrary to the result that $\Phi$ is satisfiable.
%%%
\item $T^S$ contains only variables because there are no function symbols and constant symbols in $S$.\\
Since $\Phi$ is consistent from (i), and $\Phi \vdash \neg Ry$ for all variables $y$ by definition, there are no \textit{terms} $t \in T^S$ (more precisely, \textit{variables}, in this case) such that $\Phi \vdash Rt$.
%%%
\item If $\mathfrak{I} \models \Phi$, then $\mathfrak{I} \models \exists x Rx$, i.e. there is an $a \in A$ such that $\mathfrak{I} \frac{a}{x} \models Rx$. Furthermore, $a \not \in \{ \mathfrak{I}(t) | t \in T^S \}$, since otherwise it would be the case that $\mathfrak{I} \frac{\mathfrak{I}(t)}{x} \models Rx$, where $\mathfrak{I}(t) = a$ for some $t \in T^S$, which by the Substitution Lemma implies that $\mathfrak{I} \models Rt$, or $\mathfrak{I} \models Ry$ for some variable $y$ in this case.
\end{enumerate}
%%
\item
\begin{enumerate}[(i)]
\item Let $\mathfrak{I} = (\mathfrak{A}, \beta)$ be an interpretation of $\Phi$ with $R^\mathfrak{A} \beta(y)$. Then $\mathfrak{I} \models Ry$ and furthermore $\mathfrak{I} \models (Rx \lor Ry)$, i.e. $\mathfrak{I}$ is a model of $\Phi$, whether $R^\mathfrak{A} \beta(x)$ or not. Therefore, neither $Rx$ nor $\neg Rx$ is a consequence of $\Phi$, i.e. not $\Phi \models Rx$ and not $\Phi \models \neg Rx$, which implies that not $\Phi \vdash Rx$ and not $\Phi \vdash \neg Rx$ by the correctness of $\mathfrak{S}$ (cf. IV.6.2).
%%%
\item From (i) we have not $\Phi \vdash Rx$ and not $\Phi \vdash Ry$. From this and 1.7(b) we successively get:
\[
\begin{array}{l}
\mbox{not $\mathfrak{I}^{\Phi} \models Rx$ and not $\mathfrak{I}^{\Phi} \models Ry$}, \cr 
\mbox{$\mathfrak{I}^{\Phi} \models \neg Rx$ and $\mathfrak{I}^{\Phi} \models \neg Ry$}, \cr
\mathfrak{I}^{\Phi} \models (\neg Rx \land \neg Ry), \cr
\mbox{$\mathfrak{I}^{\Phi} \models \neg (Rx \lor Ry)$ (cf. III.4)}, \cr
\mbox{not $\mathfrak{I}^{\Phi} \models (Rx \lor Ry)$}.
\end{array}
\]
\end{enumerate}
\end{enumerate} \begin{flushright}$\talloblong$\end{flushright}
%End of V.1.12-----------------------------------------------------------------------------------------------
%
%V.1.13---------------------------------------------------------------------------------------------------
\item \textbf{Solution to Exercise 1.13.}
Let $\Phi$ be inconsistent. Then for all $\varphi \in L^S$, $\Phi \vdash \varphi$. In particular, for all terms $t_1, t_2 \in T^S$,
\[
\Phi \vdash t_1 \equiv t_2
\]
and for all $n$-ary relation symbols $R$ and for all terms $t_1, \ldots, t_n \in T^S$,
\[
\Phi \vdash Rt_1 \ldots t_n.
\]
Hence by 1.7(b), for $\mathfrak{I}^{\Phi}$ we have: for all terms $t_1, t_2 \in T^S$,
\[
\overline{t_1} = \overline{t_2},
\]
and for all terms $t_1, \ldots, t_n \in T^S$,
\[
R^{\mathfrak{T}^{\Phi}} \overline{t_1} \ldots \overline{t_n}.
\]
Finally, for all variables $x$,
\[
\beta^{\Phi}(x) := \overline{x}.
\]
From this we perceive that $\mathfrak{I}^{\Phi}$ is independent of the inconsistent set $\Phi$. \begin{flushright}$\talloblong$\end{flushright}
%End of V.1.13-----------------------------------------------------------------------------------------------
%End of Section V.1------------------------------------------------------------------------------------------
\end{enumerate}
\ 
\\
\\
%Section V.2-------------------------------------------------------------------------------------------------
{\large \S2. Satisfiability of Consistent Sets of Formulas (the Countable Case)}
\begin{enumerate}[1.]
\item \textbf{Note to the Proof of Lemma 2.1.} In the hypothesis of this theorem, we assume that $\free(\Phi)$ is finite, so as to guarantee an infinite supply of free variables for witnesses. The reader may be tempted to drop this assumption without much care, as he may think that even though all variables appear free in $\Phi$, he can renumber them so that only, say, odd-numbered variables appear in $\Phi$, thus preserving all even-numbered variables for witnesses. This strategy, however, is not always feasible, as illustrated by Exercise 2.5.\newline
\\
On the other hand, a derivation of $\vdash \, \Gamma \, \varphi_n \frac{y_n}{x_n} \, \varphi$ missing in textbook is provided here:
\[
\begin{array}{lllll}
(m+4). & \Gamma & \varphi_n \frac{y_n}{x_n} & \varphi_n \frac{y_n}{x_n} & \mbox{(Assm)} \cr
(m+5). & \Gamma & \varphi_n \frac{y_n}{x_n} & (\neg \exists x_n \varphi_n \lor \varphi_n \frac{y_n}{x_n}) & \mbox{($\lor$S) applied to $(m+4).$} \cr
(m+6). & \Gamma & \varphi_n \frac{y_n}{x_n} & \varphi & \mbox{(Ch) applied to $(m+5).$} \cr
\ & \ & \ & \ & \mbox{and $m.$ (with (Ant))}
\end{array}
\]
where we set $l = m+6$.
%
\item \textbf{Note to the Proof of Lemma 2.2.} An alternative definition for $\Theta_{n + 1}$ is:
\[
\Theta_{n + 1} := \begin{cases}
\Theta_n \cup \{ \varphi_n \} & \mbox{if \(\con \Theta_n \cup \{ \varphi_n \}\)}; \cr
\Theta_n \cup \{ \neg\varphi_n \} & \mbox{otherwise}.
\end{cases}
\]
By IV.7.6 (c), it is obvious that $\con \Theta_n$ for all $n \in \mathbb{N}$. 
%
\item \textbf{Note to the Proof of Theorem 2.4.}
\begin{enumerate}[(1)]
\item The term interpretation $\mathfrak{I}^{\Phi^\prime}$ of $\Phi^\prime$ is an $S^\prime$-expansion (cf. III.4.7) of the term interpretation $\mathfrak{I}^\Phi$ of $\Phi$.
%%
\item By the Coincidence Lemma, it is no harm to choose such $\beta^\prime$, since each variable $v_n$ is bound.
%%
\item By IV.7.7, every finite subset $\Phi_0^\prime$ of $\Phi^\prime$ is consistent (with respect to $S^\prime$) implies that $\Phi^\prime$ is consistent (with respect to $S^\prime$).
\end{enumerate}
%
%V.2.5-------------------------------------------------------------------------------------------------------
\item \textbf{Solution to Exercise 2.5.} First we show that $\sat \Phi$ by giving a model $\mathfrak{I} = (\mathfrak{A}, \beta)$, hence $\con \Phi$ by IV.7.5: Let $A := \{ a_0, a_1 \}$ (where $a_0 \neq a_1$), and
\[
\begin{array}{rl}
\beta(v_i) := a_0 & \mbox{for all $i \in \mathbb{N}$,} \cr
c^\mathfrak{A} := a_0 & \mbox{for all constant $c$, and} \cr
f^\mathfrak{A}(u_0, \ldots, u_{n - 1}) := a_0 & \mbox{for all $n$-ary function symbol $f$ and for all} \cr
\  & \mbox{arguments $u_0, \ldots, u_{n - 1} \in A$.}
\end{array}
\]
Thus we have that $\mathfrak{I}(t) = a_0$ for all terms $t \in T^S$. It immediately follows that $\mathfrak{I} \models \Phi$.\\
\\
Next we show that there is no consistent set in $L^S$ which includes $\Phi$ and contains witnesses: Suppose for the sake of contradiction that $\Phi \subset \Psi \subset L^S$, where $\Psi$ is consistent and \emph{contains witnesses}. Therefore there exist $t_0, t_1 \in T^S$ and $\Gamma_0, \Gamma_1 \subset \Psi$ such that the derivations\\ \ \\ \phantom{an}
\begin{math}
\begin{array}{c}
\vdots \cr
\Gamma_0 \  (\exists v_0 \exists v_1 \neg v_0 \equiv v_1 \rightarrow \exists v_1 \neg t_0 \equiv v_1)
\end{array}
\end{math}
\\ \ \\and\\ \ \\ \phantom{an}
\begin{math}
\begin{array}{c}
\mbox{\rotatebox{90}{$\shortmid\,\shortmid\,\shortmid$}} \cr
\Gamma_1 \  (\exists v_1 \neg t_0 \equiv v_1 \rightarrow \neg t_0 \equiv t_1)
\end{array}
\end{math}
\\ \ \\can be carried out. Then let us consider the two derivations below:\\ \ \\ \phantom{an}
\begin{math}
\begin{array}{lllll}
\multicolumn{5}{c}{\vdots} \cr
m. & \Gamma_0 & \ & \ & (\exists v_0 \exists v_1 \neg v_0 \equiv v_1 \rightarrow \exists v_1 \neg t_0 \equiv v_1) \cr
\multicolumn{5}{c}{\mbox{\rotatebox{90}{$\shortmid\,\shortmid\,\shortmid$}}} \cr
n. & \Gamma_1 & \ & \ & (\exists v_1 \neg t_0 \equiv v_1 \rightarrow \neg t_0 \equiv t_1) \cr
(n+1). & \Gamma_0 & \Gamma_1 & \exists v_0 \exists v_1 \neg v_0 \equiv v_1 & (\exists v_0 \exists v_1 \neg v_0 \equiv v_1 \rightarrow \exists v_1 \neg t_0 \equiv v_1) \cr
\ & \ & \ & \ & \mbox{(Ant) applied to $m$.} \cr
(n+2). & \Gamma_0 & \Gamma_1 & \exists v_0 \exists v_1 \neg v_0 \equiv v_1 & (\exists v_1 \neg t_0 \equiv v_1 \rightarrow \neg t_0 \equiv t_1) \cr
\ & \ & \ & \ & \mbox{(Ant) applied to $n$.} \cr
(n+3). & \Gamma_0 & \Gamma_1 & \exists v_0 \exists v_1 \neg v_0 \equiv v_1 & \exists v_0 \exists v_1 \neg v_0 \equiv v_1 \cr
\ & \ & \ & \ & \mbox{(Assm)} \cr
(n+4). & \Gamma_0 & \Gamma_1 & \exists v_0 \exists v_1 \neg v_0 \equiv v_1 & \exists v_1 \neg t_0 \equiv v_1 \cr
\ & \ & \ & \ & \mbox{IV.3.5 applied to $(n+1).$ and} \cr
\ & \ & \ & \ & (n+3). \cr
(n+5). & \Gamma_0 & \Gamma_1 & \exists v_0 \exists v_1 \neg v_0 \equiv v_1 & \neg t_0 \equiv t_1 \cr
\ & \ & \ & \ & \mbox{IV.3.5 applied to $(n+2).$ and} \cr
\ & \ & \ & \ & (n+4).
\end{array}
\end{math}
\\ \ \\and\\ \ \\ \phantom{an}
\begin{math}
\begin{array}{lllll}
1. & v_0 \equiv t_0 & v_0 \equiv t_1 & v_0 \equiv t_0 & \mbox{(Assm)} \cr
2. & v_0 \equiv t_0 & v_0 \equiv t_1 & t_0 \equiv v_0 & \mbox{IV.5.3(a) applied to 1.} \cr
3. & v_0 \equiv t_0 & v_0 \equiv t_1 & v_0 \equiv t_1 & \mbox{(Assm)} \cr
4. & v_0 \equiv t_0 & v_0 \equiv t_1 & t_0 \equiv t_1 & \mbox{IV.5.3(b) applied to 2. and 3.}
\end{array}
\end{math}
\\ \ \\which reveal that $\Psi \vdash \neg t_0 \equiv t_1$ and $\Psi \vdash t_0 \equiv t_1$, respectively, and hence contradictory to the consistency of $\Psi$.\begin{flushright}$\talloblong$\end{flushright}
%End of V.2.5--------------------------------------------------------------------------------------
%End of Section V.2--------------------------------------------------------------------------------
\end{enumerate}
\ 
\\
\\
%Section V.3---------------------------------------------------------------------------------------
{\large \S3. Satisfiability of Consistent Sets of Formulas (the General Case)}
\begin{enumerate}[1.]
\item \textbf{Note to the Prolog of \S3. in Page 82.} The cardinality of the symbol set $S$ is equal to the cardinality of the set of $S$-formulas $L^S$ for infinite $S$. (See Set Theory \cite{Thomas_Jech}).
%
\item \textbf{Note to Corollary 3.3.} From 3.1 and 3.2, we \emph{only} have\\
\ \\
If $\Phi \subset L^S$ and $\con_S \, \Phi$, then there is an $S^\prime \supset S$ and a set $\Phi^\prime$ such that $\Phi \subset \Phi^\prime \subset L^{S^\prime}$ and $\con_{S^\prime} \, \Phi^\prime$, and $\Phi^\prime$ contains witnesses and is negation complete with respect to $S^\prime$.\\
\ \\
And then, from 1.11 we get\\
\ \\
If $\Phi \subset L^S$ and $\con_S \, \Phi$, then there is an $S^\prime \supset S$ and a set $\Phi^\prime$ such that $\Phi \subset \Phi^\prime \subset L^{S^\prime}$ and $\sat \, \Phi^\prime$.\\
\ \\
Finally, by Coincidence Lemma we obtain\\
\ \\
If $\Phi \subset L^S$ and $\con_S \, \Phi$, then $\sat \, \Phi$.\\
\ \\
That is, the result stated in 3.3; let $(\mathfrak{A}, \beta)$ be a model of $\Phi^\prime$, then $(\mathfrak{A}|_S, \beta) \models \Phi$. (For example, $(\mathfrak{T}^{\Phi^\prime}|_S, \beta^{\Phi^\prime}) \models \Phi$, where $\mathfrak{I}^{\Phi^\prime} = (\mathfrak{T}^{\Phi^\prime}, \beta^{\Phi^\prime})$ is the term interpretation associated with $\Phi^\prime$.)\\
The term interpretation $\mathfrak{I}^\Theta$ associated with $\Theta$ which is a model of $\Theta$ is also a model of $\Phi$.
%
\item \textbf{Note to the Proof of 3.4.} Since $\Phi_0$ is a \textit{finite} subset of $\Phi_0^\ast$, $\free(\Phi_0)$ is finite.
%
\item \textbf{Note to the Proof of Lemma 3.1.}
\begin{enumerate}[(1)]
\item Actually, $W(S_n) \subset L^{S_{n+1}}$ for $n \in \mathbb{N}$.
%%
\item By definition, $S^\ast$ and $W(S)$ have the common cardinality, i.e. that of $L^S$.  (Notice that the size of $S$ is no greater than that of $L^S$.) Furthermore, since $S^\ast$ is infinite, $L^{S^\ast}$ has the same cardinality as $S^\ast$ (cf. \cite{Thomas_Jech, Kenneth_Kunen}) and hence as $L^S$; likewise, this is also true for $W(S^\ast)$.\newline
\ 
\\In light of this, one can show by induction on $n$ that, if $n > 0$, then $S_n$ and $\Phi_n$\footnote{Note that the size of $\Phi_0 = \Phi$ is no greater than that of $L^S$.} are of the same size as $L^S$. As a result, $S^\prime = \bigcup_{n \in \mathbb{N}} S_n$ and $\Psi = \bigcup_{n \in \mathbb{N}} \Phi_n$ are of the same size as $L^S$.\newline
\ 
\\Moreover, the set $T^{S^\prime}$ of $S^\prime$-terms has the same cardinality as $S^\prime$ and hence as $L^S$.
%%
\item \textit{$\con_{S_n} \Phi_n$ for $n \in \mathbb{N}$.}\\
\textit{Proof.} $\con_{S_0} \Phi_0$ holds by hypothesis. For the inductive step suppose $\con_{S_n} \Phi_n$, then by 3.4 $\con_{S_{n+1}} \Phi_{n+1}$ holds. \begin{flushright}$\talloblong$\end{flushright}
\end{enumerate}
%
\item \textbf{Note to the Proof of Lemma 3.2.}
\begin{enumerate}[(1)]
\item It is without loss of generality to assume that $\Phi_1 \subset \Phi_2 \subset \ldots \subset \Phi_n$, since the indices on $\varphi_i$'s is irrelevant.
%%
\item From the observation that $\con_S \, \Theta_0$ for every finite subset $\Theta_0$ of $\Theta_1$ and IV.7.4, we conclude that $\con_S \, \Theta_1$. Note that here it may be temptating to apply IV.7.7 to $\Theta_1$ and therefore conclude that $\con_S \, \Theta_1$, as $\mathfrak{V}$ is a chain in $\mathfrak{U}$ and $\Theta_1 := \bigcup_{\Phi \in \mathfrak{V}} \Phi$; but we do not because $\mathfrak{V}$ is \emph{not} necessarily countable.
\end{enumerate}
\end{enumerate}
%End of Section V.3--------------------------------------------------------------------------------
\ 
\\
\\
\newpage\noindent
%Section V.4---------------------------------------------------------------------------------------
{\large \S4. The Completeness Theorem}
\begin{enumerate}[1.]
\item \textbf{Note to the Correctness (Theorem), the Completeness (Theorem), and the Adequacy (Theorem) of a Sequent Calculus.} One may be confused with these terms, so let us clarify these notions. Let $S$ be fixed. Given a \emph{logical system}\footnote{This term will first appear in IX.1, and will be defined formally in XIII.1.} $\mathcal{L}$, we say that a sequent calculus $\mathfrak{S}_\mathcal{L}$ for it is \emph{correct} iff every derivable sequent $\Gamma\varphi$ is correct (cf. IV.1):
\begin{quote}
\emph{For all $\Gamma$ and $\varphi$, if $\Gamma \vdash \varphi$ then $\Gamma \models \varphi$.}\footnote{Recall that $\Gamma$ is required to be \emph{finite}, cf. IV.1}
\end{quote}
As a basic requirement, $\mathfrak{S}_\mathcal{L}$ must be correct, as is obvious that an incorrect sequent calculus (one that derives incorrect sequents) is worthless, since its \emph{raison d'\^{e}tre} is its being correct.\newline
\ 
\\Suppose $\mathfrak{S}_\mathcal{L}$ is correct. And let $\Phi$ be a (possibly infinite) set of formulas, and $\varphi$ a formula. Since by definition $\Phi \vdash \varphi$ means that there exists some (finite) $\Gamma \subset \Phi$ such that $\Gamma \vdash \varphi$, which by the correctness entails that $\Gamma \models \varphi$ and further that $\Phi \models \varphi$, we directly obtain the \emph{Correctness Theorem}:
\begin{quote}
\emph{For all $\Phi$ and $\varphi$, if $\Phi \vdash \varphi$ then $\Phi \models \varphi$.}
\end{quote}
Thus, what we proved essentially in IV.6.2 is indeed the correctness, instead of the Correctness Theorem. But the gap between them can be ignored as they can be bridged readily (the transition from $\Gamma \models \varphi$ to $\Phi \models \varphi$ is trivial).\newline
\ 
\\Worth noting is that, as we have just explained, the correctness is supposed to hold. So it is almost insignificant to refer to the correctness as well as the Correctness Theorem, when it comes to analyzing a sequent calculus for a logical system, in contrast to the completeness/Completeness Theorem, which we now discuss.\newline
\ 
\\We say that $\mathfrak{S}_\mathcal{L}$ is \emph{complete} iff every correct sequent $\Gamma\varphi$ is derivable:
\begin{quote}
\emph{For all $\Gamma$ and $\varphi$, if $\Gamma \models \varphi$ then $\Gamma \vdash \varphi$.}
\end{quote}
One can see that it is the converse of correctness.\newline
\ 
\\In this chapter, we are aimed at proving the \emph{Completeness Theorem}:
\begin{quote}
\emph{For all $\Phi$ and $\varphi$, if $\Phi \models \varphi$ then $\Phi \vdash \varphi$.}
\end{quote}
Likewise, it is the converse of the Correctness Theorem. Notably, however, the relation between the correctness and the Correctness Theorem is reversed for this situation: The Comleteness Theorem entails the completeness. This fact can be easily verified by taking of $\Gamma$ as some finite $\Phi$.\newline
\ 
\\Finally, we say that $\mathfrak{S}_\mathcal{L}$ is \emph{adequate} iff it is both correct and complete, i.e. iff it derives exactly the correct sequents. On the other hand, the \emph{Adequacy Theorem} is taken to be the conjunction of the Correctness Theorem and the Completeness Theorem.\newline
\ 
\\
\textit{Remark.} In some other textbooks, the Adequacy Theorem here is termed the Completeness Theorem, of which the Correctness Theorem and the Completeness Theorem (here), respectively, are in turn referred to as the \emph{trivial} and \emph{nontrivial} parts.
\end{enumerate}
%End of Section V.4--------------------------------------------------------------------------------
%End of Chapter V----------------------------------------------------------------------------------
%%Chapter VI--------------------------------------------------------------------------------------------------
{\LARGE \bfseries VI \\ \\ The L\"{o}wenheim-Skolem Theorem\\ \\and the Compactness Theorem}
\\
\\
\\
%Section VI.1--------------------------------------------------------------------------------------
{\large \S1. The L\"{o}wenheim-Skolem Theorem}
\begin{enumerate}[1.]
\item \textbf{Note to the Proof of L\"{o}wenheim-Skolem Theorem.} The proof is divided into cases concerning whether $\Phi$ contains formulas other than sentences: it is done so in order to apply the results in V.1 and V.2 directly.\newline
\\
Notice that in the second case, the term interpretation $\mathfrak{I}^{\Phi^\prime}$ (the one also discussed in V.2.4) only applies to set $\Phi^\prime$ of \textit{sentences}. However, it is easy to settle this problem (by the Coincidence Lemma): just define $\beta$ accordingly as in the proof of V.2.4, then it implies from the Substitution Lemma that $\mathfrak{I}^{\Phi^\prime} \models \Phi$.
%
\item \textbf{Note to the ``Downward'' L\"{o}wenheim-Skolem Theorem.} The proof for this theorem follows from (part of) the discussion in \textbf{Note to the Proof of Lemma 3.1} in the annotations to Chapter V, and additionally from the fact: The domain $A$ of the interpretation constructed out of the proofs in V.1 and V.3 consists of classes $\overline{t}$ of terms, where $t$ ranges over $T^{S^\prime}$. Since $T^{S^\prime}$ is of the same size as $L^S$ (cf. the note mentioned above), it follows that the cardinality of $A$ is not greater than that of $L^S$.\newline
\ 
\\Recall that this is all concerned with Henkin's \emph{constructive} method, which we applied to prove the Completeness Theorem in Chapter V. A \emph{by-product} for this method is the Downward L\"{o}wenheim-Skolem Theorem. In other words, this method proves the \emph{combination} of them:
\begin{quote}
\emph{If $\Phi \subset L^S$ is consistent, then it is satisfiable over a domain of cardinality not greater than the cardinality of $L^S$.}
\end{quote}
This can be illustrated by
\begin{quote}
Henkin's method $\Rightarrow$ the Completeness Theorem $+$ the Downward L\"{o}wenheim-Skolem Theorem.
\end{quote}
From this point it is clear that the (Downward) L\"{o}wenheim-Skolem Theorem itself has nothing to do with the Completeness Theorem. It will be manifested when we encounter extensions of first-order logic in Chapter IX.
%
\item \textbf{Note to the Paragraph after the ``Downward'' L\"{o}wenheim-Skolem Theorem in Page 88.} Similar to $\mathfrak{R}^<$, the field $\mathfrak{R}$ of real numbers cannot be characterized by a set $\Phi$ of $S_{\mbox{\scriptsize ar}}$-sentences.
%
\item \textbf{Solution to Exercise 1.3.} Let $\Phi$ be an at most countable set of formulas which is satisfiable over an infinite domain $A$. Let
\[
\Psi := \Phi \cup \{ \varphi_{\geq n} | 2 \leq n \}.
\]
Then $\sat \Psi$ (the infinite model $(\mathfrak{A}, \beta)$ of $\Phi$ described earlier is also a model of $\Psi$). By 1.2, $\Psi$ is satisfiable over an at most countable domain. But since $\Psi$ is only satisfiable over infinite models, the at most countable domain must be countable. Let $\mathfrak{I}$ be a model of $\Psi$ with this countable domain, then $\mathfrak{I} \models \Phi$. \begin{flushright}$\talloblong$\end{flushright}
\end{enumerate}
%End of Section VI.1-------------------------------------------------------------------------------
\ 
\\
\\
%VI.2----------------------------------------------------------------------------------------------
{\large \S2. The Compactness Theorem}
\begin{enumerate}[1.]
\item \textbf{Note to the Compactness Theorem.} In all situations, it is a direct consequence of the Completeness Theorem together with the Correctness Theorem (or equivalently the Adequacy Theorem). Furthermore, recall that in \textbf{Note to the Correctness (Theorem), the Completeness (Theorem), and the Adequacy (Theorem) of a Sequent Calculus} in the annotations to Chapter V, there we stated that the Correctness Theorem is usually insignificant. Thus whenever the Completeness Theorem holds for a sequent calculus for a logical system, the Compactness Theorem holds as well.\footnote{This fact is often useful when proving the Completeness Theorem does \emph{not} hold, by showing the Compactness Theorem does not hold, cf. Chapter IX.} In this sense, therefore, the Completeness Theorem is logically equivalent to the conjunction of the Compactness Theorem and the \emph{completeness}, which might be described by an ``equation'' of mathematical flavor:
\begin{center}
Completeness Theorem $=$ Compactness Theorem $+$ completeness.
\end{center}
To verify the above statement holds, we need only to check the right implies the the left, as follows: For all $\Phi$ and $\varphi$,\\
\begin{tabular}{ll}
if   & $\Phi \models \varphi$ \cr
then & there is a finite $\Phi_0 \subset \Phi$ such that $\Phi_0 \models \varphi$ \cr
\    & (by the Compactness Theorem) \cr
iff  & there is a finite $\Phi_0 \subset \Phi$ such that $\Gamma \models \varphi$, where $\Gamma$ is a sequent \cr
\    & that consists of the formulas in $\Phi_0$ \cr
then & there is a finite $\Phi_0 \subset \Phi$ such that $\Gamma \vdash \varphi$, where $\Gamma$ is a sequent \cr
\    & that consists of the formulas in $\Phi_0$ (by the completeness) \cr
iff  & there is a finite $\Phi_0 \subset \Phi$ such that $\Phi_0 \vdash \varphi$ \cr
then & $\Phi \vdash \varphi$.
\end{tabular}
%
\item \textbf{Note to Theorem 2.2.} A direct consequence of it is:
\begin{quote}
\emph{There is no set of first-order formulas $\Phi$ that characterizes interpretations of finite domain, i.e.\ there is no $\Phi$ such that for all interpretations $\intp = \intpp{\struct{A}}{\beta}$, ($\intp \models \Phi$ \quad iff \quad $A$ is finite).}
\end{quote}
\begin{proof}
If $\Phi$ were such a set, then by 2.2 it would have an \emph{infinite} model $\intp$, a contradiction.
\end{proof}
%
\item \textbf{Note to the Theorem of L\"{o}wenheim, Skolem and Tarski 2.4.} An immediate consequence of this theorem is that \emph{any negation complete set of formulas does not characterize a structure up to isomorphism}.
%
\item \textbf{Solution to Exercise 2.5.} Let
\[
\begin{array}{ll}
\mathcal{T} := \{T \subset \Sigma | &\mbox{there is a set $\Phi_{\Sigma \setminus T} \subset L_0^S$}\\
\  &\mbox{\ \ \ such that } \Sigma \setminus T = \{ \mathfrak{A} \in \Sigma | \mathfrak{A} \models \Phi_{\Sigma \setminus T} \} \}.
\end{array}
\]
We verify that $\mathcal{T}$ is a topology on $\Sigma$ as follows:
\begin{enumerate}[(1)]
\item $\emptyset \in \mathcal{T}$: Let $\Phi_{\Sigma \setminus \emptyset} := \emptyset$ since $\Sigma \setminus \emptyset = \Sigma$ and for every $\mathfrak{A} \in \Sigma$, $\mathfrak{A} \models \emptyset$.\\
$\Sigma \in \mathcal{T}$: Let $\Phi_{\Sigma \setminus \Sigma} := \{\exists x \neg x \equiv x \}$ since $\Sigma \setminus \Sigma = \emptyset$ and for every $\mathfrak{A} \in \Sigma$, not $\mathfrak{A} \models \exists x \neg x \equiv x$.
%%
\item Suppose that $T_1$ and $T_2$ are members of $\mathcal{T}$. Then $T_1 \cap T_2$ is also a member of $\mathcal{T}$: Let $\Phi_{\Sigma \setminus (T_1 \cap T_2)} := \{ (\varphi_1 \lor \varphi_2 | \varphi_1 \in \Phi_{\Sigma \setminus T_1}, \varphi_2 \in \Phi_{\Sigma \setminus T_2} \}$.
%%
\item Suppose that $\{ T_i | i \in I \}$ is a family of members of $\mathcal{T}$. Then $\bigcup_{i \in I} T_i$ is also a member of $\mathcal{T}$: Let $\Phi_{\Sigma \setminus \bigcup_{i \in I} T_i} := \bigcup_{i \in I} \Phi_{\Sigma \setminus T_i}$ (since $\Sigma \setminus \bigcup_{i \in I} T_i = \bigcap_{i \in I} (\Sigma \setminus T_i)$).
\end{enumerate}
\ 
\\
\begin{enumerate}[(a)]
\item Let $T \in \mathcal{T}$. Then there is a set $\Phi_{\Sigma \setminus T}$ of $S$-sentences such that
\[
\Sigma \setminus T = \{ \mathfrak{A} \in \Sigma | \mathfrak{A} \models \Phi_{\Sigma \setminus T} \}.
\]
Therefore, $\Sigma \setminus T = \bigcap_{\varphi \in \Phi_{\Sigma \setminus T}} X_\varphi$. And we have that
\[
\begin{array}{lll}
T & = & \Sigma \setminus \bigcap_{\varphi \in \Phi_{\Sigma \setminus T}} X_\varphi \\
\ & = & \bigcup_{\varphi \in \Phi_{\Sigma \setminus T}} (\Sigma \setminus X_\varphi) \\
\ & = & \bigcup_{\varphi \in \Phi_{\Sigma \setminus T}} X_{\neg \varphi} .
\end{array}
\]
(Note that
\[
\begin{array}{lll}
\Sigma \setminus X_\varphi & = & \Sigma \setminus \{\mathfrak{A} \in \Sigma | \mathfrak{A} \models \varphi \} \\
\ & = & \{\mathfrak{A} \in \Sigma | \mbox{ not $\mathfrak{A} \models \varphi$}\} \\
\ & = & \{\mathfrak{A} \in \Sigma | \mathfrak{A} \models \neg \varphi \} \\
\ & = & X_{\neg \varphi} \mbox{\ \ .)}
\end{array}
\]
Hence $\{ X_\varphi | \varphi \in L_0^S \}$ is a basis for $\mathcal{T}$.
%%
\item Since $\Sigma \setminus X_\varphi = X_{\neg \varphi}$ (see above) is a member of $\mathcal{T}$ (actually both $X_\varphi$ and $X_{\neg \varphi}$ are members of $\mathcal{T}$), i.e. $\Sigma \setminus X_\varphi$ is open, $X_\varphi$ is closed.
%%
\item Let $\{T_i \in \mathcal{T} | i \in I \}$ be an open covering of $\Sigma$, i.e. $\Sigma \subset \bigcup_{i \in I} T_i$. From the fact that $\mathcal{T}$ is a topology we have that $\bigcup_{i \in I} T_i \in \mathcal{T}$. Therefore $\bigcup_{i \in I} T_i \subset \Sigma$. Hence
\[
\bigcup_{i \in I} T_i = \Sigma,
\]
and we have that
\[
\bigcap_{i \in I} (\Sigma \setminus T_i) = \emptyset,
\]
i.e. $\bigcup_{i \in I} \Phi_{\Sigma \setminus T_i}$ is inconsistent.\\
\\
From IV.7.2(b), there is a $\varphi \in L_0^S$ such that $\bigcup_{i \in I} \Phi_{\Sigma \setminus T_i} \vdash (\varphi \land \neg \varphi)$, or equivalently $\bigcup_{i \in I} \Phi_{\Sigma \setminus T_i} \models (\varphi \land \neg \varphi)$ (by the Adequacy Theorem V.4.2). From the Compactness Theorem (VI.2.1(a)), there is a finite subset $\Phi_0$ of $\bigcup_{i \in I} \Phi_{\Sigma \setminus T_i}$ such that $\Phi_0 \models (\varphi \land \neg \varphi)$. Clearly there is a finite subset $I_0$ of $I$ such that
\[
\Phi_0 \subset \bigcup_{i \in I_0} \Phi_{\Sigma \setminus T_i},
\]
and hence $\bigcup_{i \in I_0} \Phi_{\Sigma \setminus T_i} \models (\varphi \land \neg \varphi)$, or equivalently
\[
\bigcap_{i \in I_0} (\Sigma \setminus T_i) = \emptyset,
\]
i.e. $\bigcup_{i \in I_0} T_i = \Sigma$. Then $\{ T_i \in \mathcal{T} | i \in I_0 \}$ is a finite subcovering, i.e. $\Sigma$ is (quasi-)compact. \begin{flushright}$\talloblong$\end{flushright}
\end{enumerate}
%End of VI.2.5-----------------------------------------------------------------------------------------------
\end{enumerate}
%End of Section VI.2-----------------------------------------------------------------------------------------
\ 
\\
\\
%Section VI.3------------------------------------------------------------------------------------------------
{\large \S3. Elementary Classes}
\begin{enumerate}[1.]
%VI.3.7------------------------------------------------------------------------------------------------------
\item \textbf{Solution to Exercise 3.7.} Let $\mathfrak{K} := \modelclass{S}{\Phi}$. Then $\mathfrak{K}^\infty := \modelclass{S}{(\Phi \cup \{\varphi_{\geq n} | 2 \leq n \})}$. \begin{flushright}$\talloblong$\end{flushright}
%End of VI.3.7-----------------------------------------------------------------------------------------------
%
%VI.3.8------------------------------------------------------------------------------------------------------
\item \textbf{Solution to Exercise 3.8.}
\begin{enumerate}[(a)]
\item Suppose $\mathfrak{K}$ is elementary. Let $\mathfrak{K} := \modelclass{S}{\varphi}$. Then take $\Phi := \{ \varphi \}$. Conversely, suppose $\mathfrak{K} := \modelclass{S}{\Phi}$ where $\Phi$ is finite. Let $\Phi := \{ \varphi_n | n \leq n_0 \}$ for some $n_0 \in \mathbb{N}$. Then $\Phi \models \bigwedge_{n \leq n_0} \varphi_n$ and $\bigwedge_{n \leq n_0} \varphi_n \models \varphi_n$ for all $n \leq n_0$. Therefore $\mathfrak{K} = \modelclass{S}{\bigwedge_{n \leq n_0} \varphi_n}$.
%%
\item If $\Phi$ is finite, then the proof is complete from (a). So let us assume that $\Phi$ is infinite. Since $\mathfrak{K}$ is elementary, $\mathfrak{K} = \modelclass{S}{\varphi}$ for some sentence $\varphi$. Therefore $\Phi \models \varphi$ and $\varphi \models \varphi^\prime$ for all $\varphi^\prime \in \Phi$.\\
On the one hand, since $\Phi \models \varphi$, there is a finite subset $\Phi_0 \subset \Phi$ such that $\Phi_0 \models \varphi$ by the Compactness Theorem. Hence $\modelclass{S}{\Phi_0} \subset \modelclass{S}{\varphi}$.\\
On the other hand, since $\varphi \models \varphi^\prime$ for all $\varphi^\prime \in \Phi$, $\varphi \models \varphi^\prime$ for all $\varphi^\prime \in \Phi_0$ (note that $\Phi_0 \subset \Phi$). Hence $\modelclass{S}{\varphi} \subset \modelclass{S}{\Phi_0}$.\\
We have that $\modelclass{S}{\varphi} = \modelclass{S}{\Phi_0}$, i.e. $\mathfrak{K} = \modelclass{S}{\Phi_0}$.
\end{enumerate} \begin{flushright}$\talloblong$\end{flushright}
%End of VI.3.8-----------------------------------------------------------------------------------------------
%
%VI.3.9------------------------------------------------------------------------------------------------------
\item \textbf{Solution to Exercise 3.9.} Let $\mathfrak{K} := \modelclass{S}{\varphi_0}$ and $\mathfrak{K}_1 := \modelclass{S}{\Phi_1}$.
\begin{enumerate}[(a)]
\item \textit{If $\mathfrak{K}_1$ is elementary then $\mathfrak{K}_2$ is elementary}: Let $\mathfrak{K}_1 := \modelclass{S}{\varphi_1}$, then $\mathfrak{K}_2 = \modelclass{S}{\varphi_0 \land \neg\varphi_1}$.\\
\\
\textit{If $\mathfrak{K}_2$ is elementary then $\mathfrak{K}_2$ is $\Delta$-elementary}: By definition 3.1. (See the discusssion below it.)\\
\\
\textit{If $\mathfrak{K}_2$ is $\Delta$-elementary then $\mathfrak{K}_1$ is elementary}: If $\mathfrak{K}_1 = \emptyset$ or $\mathfrak{K}_2 = \emptyset$, then $\mathfrak{K}_1$ and $\mathfrak{K}_2$ are trivially elementary. For example, if $\mathfrak{K}_1 = \emptyset$, then  $\mathfrak{K}_1 = \modelclass{S}{(\varphi \land \neg \varphi)}$ for arbitrary $\varphi$, and $\mathfrak{K}_2 = \modelclass{S}{\varphi_0}$. The case that $\mathfrak{K}_2 = \emptyset$ is similar.\\
\\
Suppose that $\mathfrak{K}_1 \not = \emptyset$ and $\mathfrak{K}_2 \not = \emptyset$. Let $\mathfrak{K}_2 := \modelclass{S}{\Phi_2}$. Since $\modelclass{S}{(\Phi_1 \cup \Phi_2)} = \mathfrak{K}_1 \cap \mathfrak{K}_2 = \emptyset$, i.e. $\Phi_1 \cup \Phi_2$ is not satisfiable, there is a finite subset $\Phi_0 \subset \Phi_1 \cup \Phi_2$ such that $\Phi_0$ is not satisfiable, by the Compactness Theorem, 2.1(b).\\
\\
Moreover, $\Phi_0 \cap \Phi_1 \not = \emptyset$ and $\Phi_0 \cap \Phi_2 \not = \emptyset$. Because if $\Phi_0 \cap \Phi_1 = \emptyset$ then $\Phi_0 \subset \Phi_2$, which implies that $\Phi_2$ is not satisfiable (since $\Phi_0$ is not satisfiable), which in turn implies that $\mathfrak{K}_2 = \emptyset$, contrary to the assumption. The argument for $\Phi_0 \cap \Phi_2 \not = \emptyset$ is similar.\\
\\
\setcounter{equation}{0}
Hence, let $\Phi_1^\prime := \Phi_0 \cap \Phi_1$, $\Phi_2^\prime := \Phi_0 \cap \Phi_2$. Then
\begin{equation}
(\modelclass{S}{\Phi_1^\prime}) \cap \mathfrak{K}_2 = \emptyset \label{eq1}
\end{equation}
(otherwise it would be that $\Phi_1^\prime \cup \Phi_2$ is satisfiable, which implies that $\Phi_1^\prime \cup \Phi_2^\prime = \Phi_0$ is also satisfiable, contrary to the previous result) and $(\modelclass{S}{\Phi_2^\prime}) \cap \mathfrak{K}_1 = \emptyset$ (the argument is similar).\\
\\
It is clear that
\begin{equation}
\modelclass{S}{\Phi_1} \subset \modelclass{S}{\Phi_1^\prime}. \label{eq2}
\end{equation}\\
\\
From (\ref{eq1}) and (\ref{eq2}) we have that
\[
\mathfrak{K}_1 = \modelclass{S}{(\Phi_1^\prime \cup \{ \varphi_0 \})}.
\]
Since $\Phi_1^\prime$ is finite (note that $\Phi_0 \supset \Phi_1^\prime$ is finite), $\Phi_1^\prime \cup \{ \varphi_0 \}$ is also finite. As it turns out, $\mathfrak{K}_1$ is elementary by Exercise 3.8(a). Similarly, $\mathfrak{K}_2$ is elementary.
%%
\item The class of fields is elementary by 3.2; the class of fields of characteristic $0$ is $\Delta$-elementary but not elementary from the discussions below 3.3 and below 3.4. From (a) we conclude that the class of fields whose characteristic is a prime is not $\Delta$-elementary.
\end{enumerate} \begin{flushright}$\talloblong$\end{flushright}
%End of VI.3.9-----------------------------------------------------------------------------------------------
%
%VI.3.10-----------------------------------------------------------------------------------------------------
\item \textbf{Solution to Exercise 3.10.}
\begin{enumerate}[(a)]
\item Let $\Phi := \{ \varphi_0, \ldots, \varphi_{n-1} \}$, $n \in \mathbb{N}$. Next, let $\Phi_n := \Phi$, and for each $0 \leq i < n$,
\[
\Phi_i := \begin{cases} \Phi_{i+1} \setminus \{ \varphi_i \}, & \mbox{if \(\Phi_{i+1} \setminus \{ \varphi_i \} \models \varphi_i\);} \cr
\Phi_{i+1}, & \mbox{otherwise.}
\end{cases}
\]
Then for $0 \leq i \leq n$, $\Phi_i \subset \Phi$ and $\modelclass{S}{\Phi_i} = \modelclass{S}{\Phi}$. In particular, $\Phi_0$ is independent.
%%
\item Since $S$ is at most countable, $L^S$ is countable (cf. II.3.3). Let $\mathfrak{K} := \modelclass{S}{\Psi}$ be a $\Delta$-elementary class of $S$-structures, where
\[
\Psi = \{ \psi_n | n \in \mathbb{N} \} \subset L^S
\]
is a system of axioms for $\mathfrak{K}$. (If originally $\Psi$ was finite, i.e. it had maximally indexed sentence $\psi_{n_0}$, then let $\psi_n := (\forall v_n \, v_n \equiv v_n \lor \neg \forall v_n \, v_n \equiv v_n)$ for $n > n_0$.)\\
\\
Then for each $n \in \mathbb{N}$ let
\[
\varphi_n := \begin{cases}
\psi_0, & \mbox{if \(n = 0\);} \cr
(\varphi_{n-1} \land \psi_n), & \mbox{otherwise.} \cr
\end{cases}
\]
Obviously $\models \varphi_{i+1} \rightarrow \varphi_i$ for $i \in \mathbb{N}$.\\
\\
Next let $\varphi_0^\prime := \varphi_0$, and for $n \geq 1$ let
\[
\varphi_n^\prime := \begin{cases}
\varphi_{n-1}^\prime, & \mbox{if \(\models \varphi_{n-1} \rightarrow \varphi_n\);} \cr
(\varphi_{n-1} \rightarrow \varphi_n), & \mbox{otherwise.}
\end{cases}
\]
Denote
\[
\Phi := \{ \varphi_n^\prime | n \in \mathbb{N} \}
\]
and $\Phi_n := \{ \varphi_i^\prime | i \leq n \}$ for $n \in \mathbb{N}$. (Notice that if $\varphi_i^\prime = \varphi_j^\prime$ for some $i < j < n$, then $\Phi_n$ contains exactly one of them, since there is no repetition of an element in a set.) $\Phi_n$'s have the following remarkable properties:
\begin{enumerate}[(1)]
\item $\modelclass{S}{\Phi_n} = \modelclass{S}{\{ \psi_i | i \leq n \}}$ for $n \in \mathbb{N}$.
%%
\item $\Phi_n$ is independent for $n \in \mathbb{N}$.
\end{enumerate}
Property (1) immediately follows from the definitions of $\varphi_n$ and $\varphi_n^\prime$. As to property (2), $\Phi_0$ is obviously independent; the other cases can be easily shown by using mathematical induction.\\
\\
We can apply these two properties to show that $\Phi$ is independent by investigating the classes of $S$-structures $\modelclass{S}{\varphi_n^\prime}$ and $\modelclass{S}{(\Phi \setminus \{\varphi_n^\prime\})}$ (it must be the case that neither $(\modelclass{S}{(\Phi \setminus \{\varphi_n^\prime\})}) \subset \modelclass{S}{\varphi_n^\prime}$ nor $(\modelclass{S}{\varphi_n^\prime})$ $\subset \modelclass{S}{(\Phi \setminus \{\varphi_n^\prime\})}$, except the case that $\Phi = \Phi_0$, in which $\Phi$ is automatically independent) for each $n \in \mathbb{N}$. (Though we do not commit ourselves to do this since it is a tedious work \ldots ) Finally, it is clear that $\modelclass{S}{\Phi} = \modelclass{S}{\Psi}$. Hence $\Phi$ is an independent system of axioms for $\mathfrak{K}$.
\end{enumerate} \begin{flushright}$\talloblong$\end{flushright}
%End of VI.3.10----------------------------------------------------------------------------------------------
%
%VI.3.11-----------------------------------------------------------------------------------------------------
\item \textbf{Solution to Exercise 3.11.}
\begin{enumerate}[(a)]
\item Let $n \in \mathbb{Z}^+$ and
\[
\begin{array}{ll}
\varphi_n := & \phantom{\land} \exists v_0 \ldots \exists v_{n-1}(Vv_0 \land \ldots \land Vv_{n-1} \cr
\ & \land \forall u(Vu \rightarrow \exists^{=1}c_0 \ldots \exists^{=1}c_{n-1}(Fc_0 \land \ldots \land Fc_{n-1} \cr
\ & \land u \equiv ( \underbrace{(\ldots (}_{\mbox{\scriptsize $(n-2)$ times}} (c_0 * v_0) \underbrace{\circ \_\,\_\,\_ ) \ldots  \circ \_\,\_\,\_ )}_{\mbox{\scriptsize $(n-2)$ terms}} \circ (c_{n-1} * v_{n-1})) ))),
\end{array}
\]
where $\_\,\_\,\_$ stands for $(c_i * v_i)$ for $0 < i < n-1$. ($\exists^{=1}$ was introduced in III.8, see the paragraph under III.8.7.) Then $\modelclass{S}{(\Phi \cup \{ \varphi_n \})}$ is the class of $n$-dimensional vector spaces.
%%
\item The class of infinite-dimensional vector spaces is $\modelclass{S}{(\Phi \cup \{ \neg \varphi_n | n \in \mathbb{Z}^+ \})}$ hence is $\Delta$-elementary.
%%
\item Let $\varphi$ be an $S$-sentence which is valid in all infinite-dimensional vector spaces, i.e.
\[
\Phi \cup \{ \neg \varphi_n | n \in \mathbb{Z}^+ \} \models \varphi.
\]\ 
\\
By the Compactness Theorem there is an $n_0 \in \mathbb{N}$ (depending on $\varphi$) such that
\[
\Phi \cup \{ \neg \varphi_n | n < n_0 \} \models \varphi.
\]
Hence $\varphi$ is valid in all vector spaces of dimension $\geq n_0$.\\
\\
Thus there is no $S$-sentence $\varphi$ such that $\modelclass{S}{(\Phi \cup \{ \varphi \})}$ is the class of all infinite-dimensional vector spaces, i.e. it is not elementary (cf. Exercise 3.8(a)). As it turns out, the class of finite-dimensional vector spaces is not $\Delta$-elementary (cf. Exercise 3.9(a)).
\end{enumerate} \begin{flushright}$\talloblong$\end{flushright}
%End of VI.3.11----------------------------------------------------------------------------------------------
\end{enumerate}
%End of Section VI.3-----------------------------------------------------------------------------------------
\ 
\\
\\
%Section VI.4------------------------------------------------------------------------------------------------
{\large \S4. Elementarily Equivalent Structures}
\begin{enumerate}[1.]
\item \textbf{Note on Definition 4.1 and Lemma 4.2.} For part (a) of Definition 4.1, by definition the statement $\struct{A} \equiv \struct{B}$ is equivalent to\smallskip\\
\begin{quoteno}{($\ast$)}
for every $S$-sentence, if $\struct{A} \models \varphi$ then $\struct{B} \models \varphi$.
\end{quoteno}\smallskip\\
We verify this: It suffices to show\\
\centerline{for every $S$-sentence, if not $\struct{A} \models \varphi$ then not $\struct{B} \models \varphi$}
provided that ($\ast$) holds: Let $\varphi$ be an $S$-sentence, suppose $\struct{A}$ does not satisfy $\varphi$, i.e. $\struct{A} \models \neg\varphi$, then by ($\ast$) we have $\struct{B} \models \neg\varphi$, namely $\varphi$ does not hold in $\struct{B}$.\\
\ \\
Note that $\equiv$ is obviously an equivalence relation. Furthermore, if we define a binary relation $\sim$ over $S$-structures in such a way that $\struct{A} \sim \struct{B}$ if $\struct{A} \in \theoarg{\struct{B}}$, then by Lemma 4.2 $\sim$ is also an equivalence relation.
%
\item \textbf{Note to Corollary 4.4 in Page 95.} In III.7 we mentioned that there is no set of first-order $\{ \mbf{\sigma}, 0 \}$-sentences has (up to isomorphism) just $\mathfrak{N}_\sigma$ as a model (cf. the first paragraph in page 51). Now it is an obvious result of corollary 4.4.
%
\item \textbf{Note to the Paragraph Discussing the System of Axioms $\Pi$ That Characterizes $\mathfrak{N}$ Up to Isomorphism in Page 96.} From the discussion in textbook, we know that the induction axiom, which is the only second-order axiom of $\Pi$, cannot be formulated as a first-order formula or as a set of first-order formulas.\\
\\
The reason for this should be clear: The second-order variable $X$ in this axiom is a unary relation variable (cf. IX.1.1), which is interpreted as a \textit{subset} of the domain of an interpretation. The only alternative for the induction axiom by a set of first-order formulas is the axiom of induction schema for subsets of $\mathbb{N}$, i.e. an induction axiom for each subset of $\mathbb{N}$. However, the set of all subsets of $\mathbb{N}$ is uncountable (cf. Cantor's theorem), while there is only a coutable supply of induction axioms since $L^{S_{\mbox{\tiny ar}}}$ is countable (cf. II.1.2). Therefore the alternative is infeasible and hence the induction axiom can only be formulated in second-order logic.
%
\item \label{VI_4_1} \textbf{Note to the Nonstandard Model $\mathfrak{A}$ of $\thr{\mathfrak{N}^<)}$.} First, in $\mathfrak{N}^<$ the sentences
\[
\forall x \forall y ( x \equiv y \lor ( x < y \lor y < x ) )
\]
and for all $\mbf{n}^{\mathbb{N}} = n \in \mathbb{Z}^+$,
\[
\forall x ( x < \mbf{n} \leftrightarrow ( x \equiv \mbf{0} \lor ( \ldots \lor x \equiv \mbf{n-1}) ) )
\]
hold. It then follows that $\mbf{n}^A <^A a$ for $\mbf{n}^{\mathbb{N}} = n \in \mathbb{N}$ where $a$ is a ``further element.''\\
\\
Next, for all $\mbf{m}^{\mathbb{N}} = m, \mbf{n}^{\mathbb{N}} = n \in \mathbb{N}$, the sentences
\[
\forall x ( x + \mbf{m} \equiv \mbf{n} \leftrightarrow x \equiv \mbf{n-m})
\]
where $m \leq n$,
\[
\forall x \neg ( x + \mbf{n} \equiv \mbf{m})
\]
where $m < n$,
\[
\forall x ( x + \mbf{m} < \mbf{n} \leftrightarrow x < \mbf{n-m} )
\]
where $m < n$, and
\[
\forall x \neg ( x + \mbf{n} < \mbf{m} )
\]
where $m \leq n$ all hold. As it turns out, $\mbf{n}^A <^A a +^A \mbf{m}^A$ for all $\mbf{m}^{\mathbb{N}} = m, \mbf{n}^{\mathbb{N}} = n \in \mathbb{N}$, since otherwise it would be that $a$ is an element of $\mathbb{N}$. Similarly, $a +^A \mbf{n}^A <^A a +^A a +^A \mbf{m}^A$ for all $\mbf{m}^{\mathbb{N}} = m, \mbf{n}^{\mathbb{N}} = n \in \mathbb{N}$.\\
\\
On the other hand, $A$ has infinitely many ``further elements'' such as $a$. Further, the sentence
\[
\forall x \exists^{=1} y ( ( x \equiv \mbf{2} \cdot y \lor x \equiv ( \mbf{2} \cdot y ) + \mbf{1} ) \land (\neg x \equiv \mbf{2} \cdot y \lor \neg x \equiv ( \mbf{2} \cdot y ) + \mbf{1} ) )
\]
is in $\thr{\mathfrak{N}^<}$, i.e. parity is well-defined. As a consequence, for all pairs of ``further elements'' $a, b \in A$ such that $a <^A b$,
\[
c := \begin{cases}
\displaystyle\frac{a +^A b}{\mbf{2}^A} & \mbox{if \(a +^A b\) is even;} \cr
\displaystyle\frac{a +^A b +^A 1}{\mbf{2}^A} & \mbox{otherwise}
\end{cases}
\]
(where by $\displaystyle\frac{d}{\mbf{2}^A}$ we mean the element $e$ such that $d = \mbf{2}^A \cdot^A e$) is also a ``further element'' and $a <^A c <^A b$, $a +^A \mbf{n}^A <^A c +^A \mbf{m}^A$, and $c +^A \mbf{n}^A <^A b +^A \mbf{m}^A$. (The argument is similar.)
%
%VI.4.8------------------------------------------------------------------------------------------------------
\item \textbf{Solution to Exercise 4.8.} First note that for any ordered field $\struct{A}$, it contains an infinite ascending chain\\
\begin{quoteno}{($\ast$)}
$\intpted{0}{A} \mathrel{\intpted{<}{A}} \intpted{1}{A} \mathrel{\intpted{<}{A}} \intpted{1}{A} \mathbin{\intpted{+}{A}} \intpted{1}{A} \mathrel{\intpted{<}{A}} \ldots$
\end{quoteno}\\
because $\formal{n} < \formal{n} + 1$ is valid for $n \in \nat$, where $\formal{n}$ denotes $\underbrace{1 + \cdots + 1}_{\scripttext{\begin{math}n\end{math}-times}}$ if $n > 0$ and $0$ otherwise.\\
\ \\
Let $S \defas \symbarord \union \setenum{c}$ and $\Phi \defas \axiomsofd \union \setm{\formal{n} < c}{n \in \nat}$. Then for any $\symbarord$-structure $\struct{A}$, $\struct{A}$ is a non-archimedian ordered fields if and only if there is an $S$-expansion $\pair{\struct{A}}{\intpted{c}{A}}$ that is a model of $\Phi$. In other words, the class of non-archimedian ordered fields is exactly the class of the $\symbarord$-reducts of the structures in $\modelclass{S}{\Phi}$. Obviously, structures in $\modelclass{S}{\Phi}$ are themselves non-archimedian ordered fields.\\
\ \\
Therefore, if $\varphi \in \fstordlang[0]{\symbarord}$ is valid in all non-archimedian ordered fields then $\Phi \consq \varphi$. By the Compactness Theorem, there is a finite subset $\Phi_0$ of $\Phi$ such that $\Phi_0 \consq \varphi$; without loss of generality we assume $\axiomsofd \subset \Phi_0$. Let $n_0$ be the largest among the numbers $n \in \nat$ such that the sentences $\formal{n} < c$ occur in $\Phi_0$ (note that $\Phi_0$ is finite), $n_0 = 0$ if $\Phi_0$ contains no such sentence.\\
\ \\
Clearly, for any $\symbarord$-structure $\struct{A}$, if $\struct{A} \satis \axiomsofd$ then, by suitably choosing $\intpted{c}{A}$ according to ($\ast$) the $S$-expansion $\pair{\struct{A}}{\intpted{c}{A}}$ is a model of $\Phi_0$, and $\pair{\struct{A}}{\intpted{c}{A}} \satis \varphi$ because $\Phi_0 \consq \varphi$, which yields $\struct{A} \satis \varphi$ by the Coincidence Lemma. We thus conclude that $\varphi$ is valid for all non-archimedian ordered fields.
%End of VI.4.8--------------------------------------------------------------------------------
%
%VI.4.9------------------------------------------------------------------------------------------------------
\item \textbf{Solution to Exercise 4.9.} Let $\varphi := \forall x \forall y (x < y \liff (\neg x \equal y \land \exists z \, x + z \equal y))$. In fact, $\varphi$ is an extension by definitions for the new binary relation symbol $<$ in terms of the symbols in $\arsymb$ (cf.\ VIII.3).\\
\ \\
To each $\chi \in \fstordlang{\symbarord}$ we associate an $\arsymb$-formula $\chi^\prime$ that is obtained by replacing in it all subformulas of the form $t_1 < t_2$ (if any), where $t_1, t_2 \in \term{\arsymb}$, with $(\neg t_1 \equal t_2 \land \exists z \, t_1 + z \equal t_2)$. Then it can easily be verified that for every $\symbarord$-structure $\struct{B}$ that satisfies $\varphi$, every assignment $\beta$ in $\struct{B}$, and every $\chi \in \fstordlang{\symbarord}$,
\begin{medcenter}
$\intparg{\struct{B}}{\beta} \satis \chi$ \ \ \ iff \ \ \ $\intparg{\reduct{\struct{B}}{\arsymb}}{\beta} \satis \chi^\prime$.
\end{medcenter}
In particular, if $\chi$ is an $\symbarord$-sentence then by the Coincidence Lemma we have\\
\begin{quoteno}{($\ast$)}
$\pair{\struct{A}}{\intpted{<}{A}} \satis \chi$ \ \ \ iff \ \ \ $\struct{A} \satis \chi^\prime$
\end{quoteno}\\
and\\
\begin{quoteno}{($\ast\ast$)}
$\natstrord \satis \chi$ \ \ \ iff \ \ \ $\natstr \satis \chi^\prime$
\end{quoteno}\\
because $\pair{\struct{A}}{\intpted{<}{A}}$ and $\natstrord$ are both models of $\varphi$.\\
\ \\
Therefore we have, using Lemma 4.2 in text, that for every $\symbarord$-sentence $\chi$,
\begin{medcenter}
\begin{tabular}{lll}
\   & $\pair{\struct{A}}{\intpted{<}{A}} \satis \chi$ & \ \cr
iff & $\struct{A} \satis \chi^\prime$ & (by ($\ast$)) \cr
iff & $\natstr \satis \chi^\prime$ & (by premise that $\struct{A} \satis \theoarg{\natstr}$) \cr
iff & $\natstrord \satis \chi$ & (by ($\ast\ast$))
\end{tabular}
\end{medcenter}
i.e.\ $\pair{\struct{A}}{\intpted{<}{A}} \equiv \natstrord$ and thus $\pair{\struct{A}}{\intpted{<}{A}}$ is a model of $\theoarg{\natstrord}$.
%End of VI.4.9--------------------------------------------------------------------------------
%
%VI.4.10-----------------------------------------------------------------------------------------------------
\item \textbf{Solution to Exercise 4.10.} For brevity $\formal{n}$ denotes $\underbrace{1 + \ldots + 1}_{\scripttext{\begin{math}n\end{math}-times}}$ for $n \in \nat$ below.\\
\ \\
Let $S \defas \arsymb \union \setenum{k}$, $\Phi^+ \defas \setm{(\exists x \, \formal{p} \cdot x \equal k \land \forall x \neg (\formal{p} \cdot \formal{p}) \cdot x \equal k)}{p \in Q}$, $\Phi^- \defas \setm{\forall x \neg\formal{p} \cdot x \equal k}{p \not\in Q}$, and let $\Phi \defas \theoarg{\natstr} \union \setenum{\neg k \equal 0} \union \Phi^+ \union \Phi^-$.\\
\ \\
Then every finite subset $\Phi_0$ of $\Phi$ is satisfiable: $\Phi_0$ is a subset of $\Psi \defas \theoarg{\natstr} \union \setenum{\neg k \equal 0} \union (\Phi^+ \intsec \Phi_0) \union \Phi^-$, and $\pair{\natstr}{\intpted{k}{\nat}}$ is a model of $\Psi$ where $\intpted{k}{\nat}$ is the product of primes $p$ such that the formulas $(\exists x \, \formal{p} \cdot x \equal k \land \forall x \neg (\formal{p} \cdot \formal{p}) \cdot x \equal k)$ occur in $\Phi^+ \intsec \Phi_0$ if $\Phi^+ \intsec \Phi_0$ is nonempty and $\intpted{k}{\nat} = 1$ otherwise. Thus, from the Compactness Theorem it follows that $\Phi$ is satisfiable, i.e.\ there is an $S$-structure $\struct{A}$ that is a model of arithmetic that contains an element ($\intpted{k}{A}$ here) whose prime divisors are just the members of $Q$; in fact, $\intpted{k}{A}$ contains these prime divisors without multiplicity.\\
\ \\
There are as many different sets $Q$ as there are subsets of $\nat$ (which is uncountable), and so are there different sets $\Phi$. Each $\Phi$ is countable and thus has a countable model by the L\"{o}wenheim-Skolem Theorem (a model of $\Phi$ must be infinite). It follows that there are at least as many pairwise nonisomorphic countable models of arithmetic as there are subsets of $\nat$.
%End of VI.4.10-------------------------------------------------------------------------------
%
%VI.4.11-----------------------------------------------------------------------------------------------------
\item \textbf{Solution to Exercise 4.11.}
\begin{enumerate}[(a)]
\item Since $\pair{\nat}{\intpted{<}{\nat}}$ is a (total) ordering, $\field{\nat} = \nat$. For $n \in \nat$, there are only finitely many numbers $n - 1, \ldots, 0$ that are smaller (in the sense $\intpted{<}{\nat}$) than $0$, so there is no infinite descending chain.\\
\ \\
Now let the $\symbarord$-structure $\struct{A}$ be a model of $\theoarg{\natstrord}$, then $\struct{A}$ is an ordering because $\forall x \forall y (x < y \lor x \equal y \lor y < x) \in \theoarg{\natstrord}$, therefore $\field{A} = A$. If $\pair{A}{\intpted{<}{A}}$ contains no infinite descending chain, then every nonempty subset of $A$ has a $\intpted{<}{A}$-minimum; in other words, $\pair{\struct{A}}{\intpted{<}{A}}$ and hence $\struct{A}$ satisfy the so-called \emph{well-ordering principle}, which is equivalent to (strong) induction principle that can be formulated (since $\struct{A} \satis \theoarg{\natstrord}$) as the second-order $\arsymb$-sentence
\[
\forall X ((X0 \land \forall x (Xx \limply Xx + 1)) \limply \forall y Xy).
\]
Therefore, $\reduct{\struct{A}}{\arsymb}$ is a model $\sndordpeanoarith$ and hence is isomorphic to $\natstr$ (cf.\ part (b) of Exercise III.7.5).\\
\ \\
We shall conclude that also $\struct{A}$ is isomorphic to $\natstrord$ and as a result is a standard model of $\theoarg{\natstrord}$. For this purpose, let $\pi : \reduct{\struct{A}}{\arsymb} \iso \natstr$. It remains to show: for $a, b \in A$,
\begin{medcenter}
$a \mathrel{\intpted{<}{A}} b$ \ \ \ iff \ \ \ $\pi(a) \mathrel{\intpted{<}{\nat}} \pi(b)$.
\end{medcenter}
Using the fact that $\forall v_0 \forall v_1 (v_0 < v_1 \liff (\neg v_0 \equal v_1 \land \exists v_2 \, v_0 + v_2 \equal v_1)) \in \theoarg{\natstrord}$, the equivalence is proved below: for $a, b \in A$,\\
\begin{tabular}{ll}
\   & $a \mathrel{\intpted{<}{A}} b$ \cr
iff & $\struct{A} \satis v_0 < v_1 [a, b]$ \cr
iff & $\struct{A} \satis (\neg v_0 \equal v_1 \land \exists v_2 \, v_0 + v_2 \equal v_1)[a, b]$ \; (since $\struct{A} \satis \theoarg{\natstrord}$)\cr
iff & $\reduct{\struct{A}}{\arsymb} \satis (\neg v_0 \equal v_1 \land \exists v_2 \, v_0 + v_2 \equal v_1)[a, b]$ \cr
\   & \multicolumn{1}{r}{(by the Coincidence Lemma)} \cr
iff & $\natstr \satis (\neg v_0 \equal v_1 \land \exists v_2 \, v_0 + v_2 \equal v_1)[\pi(a), \pi(b)]$ \cr
\   & \multicolumn{1}{r}{(since $\pi : \reduct{\struct{A}}{\arsymb} \iso \natstr$)} \cr
iff & $\natstrord \satis (\neg v_0 \equal v_1 \land \exists v_2 \, v_0 + v_2 \equal v_1)[\pi(a), \pi(b)]$ \cr
\   & \multicolumn{1}{r}{(by the Coincidence Lemma)} \cr
iff & $\natstrord \satis v_0 < v_1 [\pi(a), \pi(b)]$ \cr
iff & $\pi(a) \mathrel{\intpted{<}{\nat}} \pi(b)$. \cr
\end{tabular}
%%
\item Let $S^\prime \defas S \union \setm{c_n}{n \in \nat}$ (assuming $c_n \not\in S$) and for $m \geq 2$ let $\Psi_m \defas \Phi \union \axiomspord \union \setm{c_n < c_{n - 1}}{0 < n < m}$ (cf.\ III.6.4 for the definition of $\axiomspord$).\\
\ \\
It follows that, for $m \geq 2$, $\Psi_m$ has an $S^\prime$-structure as a model: By premise there is an $S$-structure $\struct{A}$ satisfying $\Phi \union \axiomspord$ where $\field{A}$ (a linear ordering) contains at least $m$ elements $\seq{a}{m - 1}$ with
\[
a_{m - 1} \intpted{<}{A} \ldots \intpted{<}{A} a_0.
\]
By setting $\intpted{c_n}{A} \defas a_n$ for $n < m$, we have $\tuple{\struct{A}, \seq{c}{m - 1}} \models \Psi_m$; by the Coincidence Lemma, any $S^\prime$-expansion of $\tuple{\struct{A}, \seq{c}{m - 1}}$ is a model of $\Psi_m$.\\
\ \\
Now set $\Psi \defas \bunion_{m \geq 2} \Psi_m$. Then every finite subset of $\Psi$ is also a subset of $\Psi_m$ for a suitable $m$ and hence is satisfiable (in the sense of $S^\prime$). By the Compactness Theorem $\Psi$ itself has an $S^\prime$-structure $\struct{B}^\prime$ as a model, where in particular
\[
\ldots \mathrel{\intpted{<}{B}} \intpted{c_2}{B} \mathrel{\intpted{<}{B}} \intpted{c_1}{B} \mathrel{\intpted{<}{B}} \intpted{c_0}{B}.
\]
Taking $\struct{B} \defas \reduct{\struct{B}^\prime}{S}$, then, again, by the Coincidence Lemma $\struct{B}$ is a model of $\Phi$ such that $\pair{B}{\intpted{<}{B}}$ is a partially defined ordering containing an infinite descending chain.
\end{enumerate}
%End of VI.4.11----------------------------------------------------------------------------------------------
\end{enumerate}
%End of Section VI.4-----------------------------------------------------------------------------------------
%End of Chapter VI-------------------------------------------------------------------------------------------
%%Chapter VII-------------------------------------------------------------------------------------------------
{\LARGE \bfseries VII \\ \\ The Scope of First-Order Logic}
\\
\\
\\
%Sectin VII.2------------------------------------------------------------------------------------------------
{\large \S2. Mathematics Within the Framework of First-Order Logic}
\begin{enumerate}[1.]
\item \textbf{Note to the Second Paragraph in Page 103.} The fact that the structure $\mathfrak{N}_\sigma = (\mathbb{N}, \sigma, 0)$ cannot be characterized up to isomorphism in $L^{\{ \mbf{\sigma}, 0 \}}$ immediately follows from Corollary VI.4.4. (Actually, we have mentioned this in notes to Section VI.4.)
%
\item \textbf{Note to the Mathematical Universe in Page 103.} Note that although the (mathematical) universe is characterized by the system $\Phi_0$ (or ZFC, as is introduced in Section 3) as a whole entity for mathematics, it is, however, \emph{not} a mathematical object itself, and hence not a set: If otherwise it were, then it would be the \emph{set of all sets},\footnote{or the \emph{set of all urelements and sets}, if $\Phi_0$ is adopted rather than ZFC.} the concept of which leads to a contradiction.\footnote{To my knowledge, there are three arguments (in the framework of ZFC) for this contradictory concept: One is to apply the \emph{separation axiom} to it with the predicate ``$x \not \in x$'', which then introduces \emph{Russell's Paradox}; another is to use the \emph{axiom of regularity} directly to refuse such set; and finally, the remaining one is to apply the \emph{axiom of powerset} to it, and argue that it includes its powerset as a subset, which violates \emph{Cantor's Theorem}.}\newline
\ \\
In this regard, we should not treat it as a \emph{model} of $\Phi_0$ (or ZFC) since, again, a model is still a mathematical object.
%
\item \textbf{Note to the Second Paragraph in Page 104.} The sentence
\[
\forall x \exists y \ x < y
\]
in $L^{\{ < \}}$ formalizes the proposition ``there is no largest real number'' (actually the proposition ``for every real number there is a larger one'') about the structure $(\mathbb{R}, <^\mathbb{R})$.
%
\item \textbf{Note to the First Paragraph in Page 105.} We provide a proof for the proposition:
\[
\mbox{$(x,y) = (x^\prime, y^\prime)$ iff $x = x^\prime$ and $y = y^\prime$.}
\]
In fact, we need only show that
\[
\mbox{if $(x, y) = (x^\prime, y^\prime)$ then $x = x^\prime$ and $y = y^\prime$},
\]
since the other direction is trivial.\\
\\
Suppose $(x,y) = (x^\prime, y^\prime)$, then
\[
(\mbox{$\{x, x\} = \{x^\prime, x^\prime\}$ and $\{x, y\} = \{x^\prime, y^\prime\}$})
\]
or
\[
(\mbox{$\{x, x\} = \{x^\prime, y^\prime\}$ and $\{x, y\} = \{x^\prime, y^\prime\}$}),
\]
since by (A3) every set is uniquely determined by its elements and hence by (A4) every pair set is uniquely determined given both its elements. By applying similar arguments, it follows that
\[
(\mbox{$x = x^\prime$ and $y = y^\prime$})
\]
or
\[
(\mbox{$x = y = x^\prime = y^\prime$}),
\]
i.e. $x = x^\prime$ and $y = y^\prime$.
%
\item \textbf{Note to the Abbreviations for Ease of Formalizing in $L^S$ in Page 105.} As mentioned in text, both approaches of introducing abbreviations to $L^S$ or introducing new symbols to $S$ together with expanding $\Phi_0$ by adding corresponding new axiom are equivalent.\\
\\
We illustrate this by taking the symbol `$\mbf{\subset}$' as an example. (The following discussion makes much use of the results in Section VIII.3, so the reader is suggested to read through that section before going ahead.)\\
\\
Let $I$ be the associated syntactic interpretation of $S \cup \{ \mbf{\subset} \}$ in $S$, with
\[
\varphi_\mbfs{\subset}(x, y) := (\mathbf{M}x \land \mathbf{M}y \land \forall z (z \in x \rightarrow z \in y)).
\]
Since `$\mbf{\subset}$' is a binary relation symbol, it is straightforward to define (cf. Definition VIII.3.1(a))
\[
\delta_\mbfs{\subset} := \forall x \forall y (x \mbf{\subset} y \leftrightarrow \varphi_\mbfs{\subset}(x, y)).
\]
\\
Then parts (b) and (c) of Theorem on Definitions VIII.3.2 give
\begin{enumerate}[(1)]
\item For all $\varphi \in L^S$, $\Phi_0 \cup \{ \delta_\mbfs{\subset} \} \models \varphi$ iff $\Phi_0 \models \varphi$.
%%
\item For all $\varphi \in L^{S \cup \{ \mbfs{\subset} \}}$, $\Phi_0 \cup \{ \delta_\mbfs{\subset} \} \models \varphi$ iff $\Phi_0 \models \varphi^I$.
\end{enumerate}
\end{enumerate}
%End of Section VII.2----------------------------------------------------------------------------------------
\ 
\\
\\
%Section VII.3-----------------------------------------------------------------------------------------------
{\large \S3. The Zermelo-Fraenkel Axioms for Set Theory}
\begin{enumerate}[1.]
\item \textbf{Note to the Second to Last Paragraph in Page 107.} Replacing urelements by suitable sets is achieved by the \emph{axiom of regularity} REG, as will be introduced in later note (\textbf{Note to ZFC, As Is Introduced in Page 108}). It essentially states that every set has \emph{well foundations}, which play the role of urelements, i.e. there is no infinite chain of `$\mbf{\in}$'.
%
\item \textbf{Note to ZFC, As Is Introduced in Page 108.} The following axiom, REG (\emph{the axiom of regularity}, also known as \emph{the axiom of foundation}), is missing in text\footnote{However, it is not generally accepted as an axiom of ZFC. Indeed, some textbooks do not contain it.}:\\
\\
REG: $\forall x (\neg \mbf{\emptyset} \equiv x \rightarrow \exists y (y \mbf{\in} x \land x \mbf{\cap} y \equiv \mbf{\emptyset}))$\\
``Given a nonempty set $x$, there exists a set $y$ in $x$ such that $x$ and $y$ are disjoint.''\\
\\
Now, we are going to transform ZFC into its counterpart $\Phi$ ($\subset L_0^{\{ \mbfs{\in} \}}$), so as to demonstrate the fact that all axioms in ZFC can be formalized in the first-order language involving only the symbol `$\mbf{\in}$'. (The following discussion makes much use of the results in Section VIII.3, so the reader is suggested to read through that section before going ahead.)\\
\\
Let $I$ be the associated syntactic interpretation of $\{ \mbf{\in}, \mbf{\emptyset}, \ \mbf{\subset}, \ \mbf{ \{ , \} }, \ \mbf{\cup}, \ \mbf{\cap}, \ \mathbf{P} \}$ in $\{ \mbf{\in} \}$, with
\[
\begin{array}{lll}
\varphi_\mbfs{\emptyset}(x)   & := & \forall y \neg y \mbf{\in} x, \cr
\varphi_\mbfs{\subset}(x,y)     & := & \forall z (z \mbf{\in} x \rightarrow z \mbf{\in} y), \cr
\varphi_{\mbfs{ \{ , \} }}(x,y,z)    & := & \forall w (w \mbf{\in} z \leftrightarrow (w \equiv x \lor w \equiv y)), \cr
\varphi_\mbfs{\cup}(x,y,z)        & := & \forall w (w \mbf{\in} z \leftrightarrow (w \mbf{\in} x \lor w \mbf{\in} y)), \cr
\varphi_\mbfs{\cap}(x,y,z)        & := & \forall w (w \mbf{\in} z \leftrightarrow (w \mbf{\in} x \land w \mbf{\in} y)), \mbox{ and} \cr
\varphi_\mathbf{P}(x,y)  & := & \forall z (z \mbf{in} y \leftrightarrow \forall w (w \mbf{\in} z \rightarrow w \mbf{\in} x)).
\end{array}
\]
\\
In addition, define
\[
\Delta := \{ \delta_\mbfs{\emptyset}, \delta_\mbfs{\subset}, \delta_{ \mbfs{ \{ , \} } }, \delta_\mbfs{\cup}, \delta_\mbfs{\cap}, \delta_{\mathbf{P}} \},
\]
where $\delta_\mbfs{\emptyset}$, $\delta_\mbfs{\subset}$, $\delta_{ \mbfs{ \{ , \} } }$, $\delta_\mbfs{\cup}$, $\delta_\mbfs{\cap}$, and $\delta_{\mathbf{P}}$ are defined accordingly (cf. Definition VIII.3.1). Later we shall see that all those $\delta$'s are $\{ \mbf{\in} \}$-definitions in $\Phi$ of the corresponding symbols, i.e. all the $\varphi$'s just mentioned satisfy the requirements in Definition VIII.3.1.\\
\\
Next, let $\Psi \subset L_0^{ \{ \mbfs{\in}, \mbfs{\emptyset}, \ \mbfs{\subset}, \ \mbfs{ \{ , \} }, \ \mbfs{\cup}, \ \mbfs{\cap}, \ \mathbf{P} \} }$ consist of the axioms SEP through REG. Note that in this way, $\mbox{ZFC} = \Psi \cup \Delta$.\\
\\
Finally, we set
\[
\Phi := \{ \psi^I | \psi \in \Psi \}.
\]
\\
We are now ready to illustrate that the $\varphi$'s satisfy the requirements in Definition VIII.3.1 (instead of giving a rigorous proof, which is not pursued here). Firstly, $\mbox{INF}^I$\footnote{By $\mbox{INF}^I$ we mean the sentence in $L_0^{\{ \mbft{\in} \}}$ corresponding to $\mbox{INF}$ in $L_0^{ \{ \mbft{\in}, \mbft{\emptyset}, \ \mbft{\subset}, \ \mbft{ \{, \} }, \ \mbft{\cup}, \ \mbft{\cap}, \ \mathbf{P} \} }$ given the syntactic interpretation $I$. The cases for other axioms are similar.} establishes the existence of such a set $x$ that $\varphi_\mbfs{\emptyset}(x)$ holds, since the infinite set mentioned in $\mbox{INF}^I$ contains such an $x$ as its member. As for the uniqueness, it is guaranteed by $\mbox{EXT}$ ($= \mbox{EXT}^I$). Next, there is no need to verify $\delta_\mbfs{\subset}$ because `$\mbf{\subset}$' is a relation symbol (cf. Definition VIII.3.1). Finally, the existence property related to each of $\varphi_{\mbfs{ \{,\} }}$, $\varphi_\mbfs{\cup}$, $\varphi_\mbfs{\cap}$, and $\varphi_\mathbf{P}$ is established by $\mbox{PAIR}^I$, $\mbox{PAIR}^I$ together with $\mbox{SUM}$\footnote{By $\mbox{PAIR}^I$, \emph{a} pair set exists given two sets $x$ and $y$. Notice that we use the indefinite article \emph{a} instead of the definite article \emph{the}, for we have not provided a proof of its uniqueness. (We shall keep this style throughout.) And then by $\mbox{SUM}$, \emph{a} union of all sets in such a pair set (i.e. $x$ and $y$) exists.}, $\mbox{SEP}$\footnote{By $\mbox{SEP}$, a set $\{z \in x | z \in y\}$ exists (by setting $\varphi(z,y) := z \mbff{\in} y$).}, and $\mbox{POW}$, respectively; while the uniqueness property related to each of them is, again, confirmed by $\mbox{EXT}$. We briefly summarize, as follows:\\
\ 
\\
\begin{tabular}{ccc}
\hline
\textsc{symbol} & \textsc{existence} & \textsc{uniqueness} \cr
\hline\hline
$\mbf{\emptyset}$ & $\mbox{INF}^I$ & $\mbox{EXT}$ ($= \mbox{EXT}^I$) \cr
$\mbf{\subset}$ & - & - \cr
$\mbf{ \{, \} }$ & $\mbox{PAIR}^I$ & $\mbox{EXT}$ \cr
$\mbf{\cup}$ & $\mbox{PAIR}^I$, $\mbox{SUM}$ & $\mbox{EXT}$ \cr
$\mbf{\cap}$ & $\mbox{SEP}$ & $\mbox{EXT}$ \cr
$\mathbf{P}$ & $\mbox{POW}$ & $\mbox{EXT}$ \cr
\hline
\end{tabular}
\\
\\
\\
The following two results serve as the basis of the proof:
\begin{enumerate}[(1)]
\item Let $\chi \in \Psi$. By Theorem VIII.3.2(b) (and Exercise VIII.3.3 also),
\[
(\Phi \cup \Delta) \models (\chi \leftrightarrow \chi^I).
\]
Hence $(\Phi \cup \Delta) \models \chi$, since $(\Phi \cup \Delta) \models \chi^I$ (because $\chi^I \in \Phi$). Therefore,
\[
\modelclass{ \{ \mbfs{\in}, \mbfs{\emptyset}, \ \mbfs{\subset}, \ \mbfs{\{, \}}, \ \mbfs{\cup}, \ \mbfs{\cap}, \ \mathbf{P} \} }{(\Phi \cup \Delta)} \subset \modelclass{ \{ \mbfs{\in}, \mbfs{\emptyset}, \ \mbfs{\subset}, \ \mbfs{\{, \}}, \ \mbfs{\cup}, \ \mbfs{\cap}, \ \mathbf{P} \} }{\zfc}.
\]
%%
\item By Theorem VIII.3.2(b), we have that for all $\chi \in \Psi$ (hence all $\chi^I \in \Phi$),
\[
\Delta (= \emptyset \cup \Delta) \models (\chi \leftrightarrow \chi^I).
\]
In particular,
\[
\mbox{ZFC} (= \Psi \cup \Delta) \models (\chi \leftrightarrow \chi^I).
\]
But $\mbox{ZFC} \models \chi$ (because $\chi \in \mbox{ZFC}$), we have that $\mbox{ZFC} \models \chi^I$. Therefore,
\[
\modelclass{ \{ \mbfs{\in}, \mbfs{\emptyset}, \ \mbfs{\subset}, \ \mbfs{\{, \}}, \ \mbfs{\cup}, \ \mbfs{\cap}, \ \mathbf{P} \} }{\zfc} \subset \modelclass{ \{ \mbfs{\in}, \mbfs{\emptyset}, \ \mbfs{\subset}, \ \mbfs{\{, \}}, \ \mbfs{\cup}, \ \mbfs{\cap}, \ \mathbf{P} \} }{(\Phi \cup \Delta)}.
\]
\end{enumerate}
\ 
\\
Thus, (1) and (2) together yield
\[
\modelclass{ \{ \mbfs{\in}, \mbfs{\emptyset}, \ \mbfs{\subset}, \ \mbfs{\{, \}}, \ \mbfs{\cup}, \ \mbfs{\cap}, \ \mathbf{P} \} }{(\Phi \cup \Delta)} = \modelclass{ \{ \mbfs{\in}, \mbfs{\emptyset}, \ \mbfs{\subset}, \ \mbfs{\{, \}}, \ \mbfs{\cup}, \ \mbfs{\cap}, \ \mathbf{P} \} }{\zfc}.
\]
\\
Also note that, by Theorem VIII.3.2(c), it follows that for all $\varphi \in L_0^{ \{ \mbfs{\in} \} }$ ($\subset L_0^{ \{ \mbfs{\in}, \mbfs{\emptyset}, \ \mbfs{\subset}, \ \mbfs{\{, \}}, \ \mbfs{\cup}, \ \mbfs{\cap}, \ \mathbf{P} \} }$),
\[
\mbox{$\Phi \models \varphi$ iff $(\Phi \cup \Delta) \models \varphi$}
\]
(since $\varphi^I = \varphi$ in this case).\\
\\
Finally, we obtain: for all $\varphi \in L_0^{ \{ \mbfs{\in} \} }$,
\[
\mbox{$\Phi \models \varphi$ iff $\mbox{ZFC} \models \varphi$}.
\]
Also notice that, for all $\varphi \in L_0^{ \{ \mbfs{\in} \} }$ ($\subset L_0^{ \{ \mbfs{\in}, \mbfs{\emptyset}, \ \mbfs{\subset}, \ \mbfs{\{, \}}, \ \mbfs{\cup}, \ \mbfs{\cap}, \ \mathbf{P} \} }$),
\[
\mbox{$\Phi \cup \Delta \models \varphi$ iff $\mbox{ZFC} \models \varphi$}.
\]
It is clear that $\Phi$ is the counterpart of ZFC in $L_0^{ \{ \mbfs{\in} \} }$.\\
\\
A subtlety is in order: In some textbooks of set theory, EXS (\emph{the axiom of the existence of a set})
\[
\exists x \; x \equiv x
\]
is included in ZFC, though it can be derived:
\[
\begin{array}{lll}
1. & x \equiv x & \mbox{$(\equiv)$} \cr
2. & \exists x \ x \equiv x & \mbox{IV.5.1(a) applied to 1.}
\end{array}
\]
For more details for this, go to
\[
\mbox{http://en.wikipedia.org/wiki/Axiom$\underline{\ }$of$\underline{\ }$empty$\underline{\ }$set}.
\]
%
\item \textbf{Note to the Fourth to Last Paragraph in Page 109.} The counterparts (in $L^{\{ \mbfs{\in} \}}$) of the abbreviations for ordered pair, ordered triple and others presented in Section 2 can be easily obtained, just by omitting in each of the formulas ($\mathbf{OP}$), ($\mathbf{OT}$), etc. the parts of the form `$\mathbf{M}x$', as well as the `$\land$' (if any) that immediately follows it.\\
%
\item \textbf{Note to the Structure $(\omega, \mbf{\nu}, \tilde{0})$ Mentioned in Page 109.} As mentioned in text, INF ensures an inductive set $x$, whereas by SEP with
\[
\varphi(z) := \forall y ((\mbf{\emptyset} \mbf{\in} y \land  \forall u (u \mbf{\in} y \rightarrow u \mbf{\cup} \mbf{\{ } u \mbf{ \}} \mbf{\in} y)) \rightarrow z \mbf{\in} y)
\]
we conclude that the set $\omega$ exists. Following the definitions of $\omega$ and $(\omega, \mbf{\nu}, \tilde{0})$ in page 109, we first show that $\omega$ is the smallest inductive set, and then that $(\omega, \mbf{\nu}, \tilde{0})$ is a Peano structure.\\
\\
Note that since $\emptyset \in x$ and for all inductive $y$, $\emptyset \in y$, we have that $\emptyset \in \omega$. And next, suppose that $t \in \omega$, namely $t \in x$ and $t \in y$ for all inductive $y$. Then it follows that $t \cup \{ t \} \in x$ because $x$ is inductive. Similarly, $t \cup \{ t \} \in y$ for all inductive $y$. Therefore $t \cup \{ t \} \in \omega$, i.e. $\omega$ is inductive. As for $\omega$ being the smallest, let $y$ be an arbitrary inductive set. If $t \in \omega$, then by definition $t \in y$. Thus $\omega \subset y$, i.e. $\omega$ is the smallest inductive set in the sense that \emph{every} inductive set includes it as a subset.\\
\\
It remains to show that $(\omega, \mbf{\nu}, \tilde{0})$ satisfies (P1)-(P3) (cf. III.7). First, for all sets $x$ in $\omega$, we have $x \in x \cup \{ x \}$, i.e. $\mbf{\nu}(x) = x \cup \{ x \} \neq \emptyset = \tilde{0}$ and (P1) is satisfied. As for (P2), let $x$, $y$ be both in $\omega$, and $x \cup \{ x \} = y \cup \{ y \}$. For the sake of contradiction, we assume that $x \neq y$. Then $x \in y$ since $\{ x \} \subset y \cup \{ y \}$. Similarly, we have that $y \in x$. But this (both $x \in y$ and $y \in x$ hold) contradicts REG, which implies that no cycle of `$\in$' exists. Hence $x = y$, and (P2) holds for $(\omega, \mbf{\nu}, \tilde{0})$. To show that (P3) also holds for $(\omega, \mbf{\nu}, \tilde{0})$, let $X \subset \omega$. If $X$ is inductive, then $\omega \subset X$ since $\omega$ is the smallest inductive set (as we have showed earlier). So $X = \omega$, and obviously $X$ contains all the elements in $\omega$. The proof is complete. 
%
\item \textbf{Note to the Third Paragraph in Page 110.} The formalizations of the definitions of $\mathbb{R}$, $\mathbf{Fin}$, etc. are not pursued in this text, but can be found in textbooks of set theory (e.g. \cite{Thomas_Jech}) or any books that deal with foundational problems in mathematics  (e.g. \cite{Edmund_Landau}).
%
\item \textbf{Note to the Sixth Paragraph in Page 110.} 3.1 and 3.2 together imply that CH together with ZFC are independent (for the notion of independent, cf. Exercises III.4.14 or VI.3.10), by the Adequacy Theorem V.4.2.
\end{enumerate}
%End of Section VII.3----------------------------------------------------------------------------------------
\ 
\\
\\
%Section VII.4-----------------------------------------------------------------------------------------------
{\large \S4. Set Theory as a Basis for Mathematics}
\begin{enumerate}[1.]
\item \textbf{Note to 4.2.} Note that a formula $\varphi$ in $L^{\{ \mbfs{\in} \}}$ is likely to have \emph{different} meanings in object set theory and background set theory, thus we must carefully distinguish them.
\item \textbf{Note to 4.3.} As will be introduced in VIII.1, the special case in which $S = \{ P^1, P^2, \ldots \}$ (that is, $S$ is a \emph{relational symbol set}) does not pose any restrictions in the sense that other cases can be easily transformed into it. Hence the arguments in 4.3 (indirectly) apply to other caes as well.\\
\\
A misprint is found in line 2, page 113: The definition of $\tilde{P}^x$ should be
\[
\mbox{``$\tilde{P}^x := (\tilde{1}, x)$ where $x \in \omega \setminus \{ \tilde{0} \}$.''}
\]
(The tilde `$\sim$' of $\tilde{0}$ is missing in textbook.)\\
\\
By the way, the symbol $At^{\equiv}$ shown up in page 113 stands for the set of \underline{at}omic formulas involving $\equiv$, whereas $At^R$ for the set of \underline{at}omic formulas involving relation symbols.
%
%VII.4.4-----------------------------------------------------------------------------------------------------
\item \textbf{Solution to Exercise 4.4.} First-order logic and set theory are both appropriate for the development of mathematical theories.\\
\\
Actually, mathematicians inevitably need to use a language, which is more precise than daily languages such as English and without ambiguities, as a background language (so-called \textit{metalanguage}) to describe their investigations. Upon it they build up their theories (so-called \textit{object languages}).\\
\\
As a consequence, both first-order logic and set theory are suitable for the role of background language, with one playing this role and the other playing the role of object language. \begin{flushright}$\talloblong$\end{flushright}
%VII.4.4-----------------------------------------------------------------------------------------------------
\end{enumerate}
%End of Section VII.4----------------------------------------------------------------------------------------
%End of Chapter VII------------------------------------------------------------------------------------------
%%Chapter VIII------------------------------------------------------------------------------------------------
{\LARGE \bfseries VIII \\ \\ Syntactic Interpretations and\\ \\Normal Forms}
\\
\\
\\
%Section VIII.1------------------------------------------------------------------------------------
{\large \S1. Term-Reduced Formulas and Relational Symbol Sets}
\begin{enumerate}[1.]
\item \textbf{Note to Theorem 1.3.} In the proof of part (a) in text, the following statement is missing:\smallskip\\
\begin{tabular}{lll}
$\relational{[c \equiv x]}$ & $\colonequals$ & $Cx$.
\end{tabular}\smallskip\\
In part (b), it is more appropriate to set\smallskip\\
\begin{tabular}{lll}
$\invrelational{[F\enum[1]{y}{n}x]}$ & $\colonequals$ & $f\enum[1]{y}{n} \equal x$, \cr
$\invrelational{[Cx]}$ & $\colonequals$ & $c \equal x$
\end{tabular}\smallskip\\
because $\relational{S}$ has no function or constant symbols.\bigskip\\
Also, this theorem can be specialized to the case of sentences and structures (using Coincidence Lemma).\bigskip\\
In the following we investigate relational symbol sets in more depth concerning validity. Again, let $S$ be an arbitrary symbol set, and $S^r$ the corresponding relational symbol set. More precisely,
\begin{enumerate}[(1)]
\item If $f \in S$ is an $n$-ary function symbol, then we assign to it $F \in S^r$ the corresponding $(n + 1)$-ary relation symbol;
%%
\item If $c \in S$ is a constant symbol, then we assign to it $C \in S^r$ the corresponding unary relation symbol.
\end{enumerate}
Let $\psi$ be an $S$-sentence, and $\psi^r$ an $S^r$-sentence defined as in the proof. Without loss of generality, we may assume that $S$ is finite, and hence so is $S^r$. Furthermore, let $\chi$ be the $S^r$-sentence which is the conjunction of the $S^r$-sentences in the set below:
\[
\begin{array}{r}
\{ \forall v_0 \ldots \forall v_{n - 1} \exists^{=1} v_n F v_0 \ldots v_n \ | \ \mbox{$F \in S^r \setminus S$ is $(n + 1)$-ary, where $n > 0$} \} \cup \cr
\{ \exists^{=1} v_0 C v_0 \ | \ \mbox{$C \in S^r \setminus S$ is unary} \}.
\end{array}
\]
That is, $\chi$ states that $F$ stands for a function if it does not appear in $S$ and its arity is at least two, and that $C$ stands for a constant if it does not appear in $S$ and its arity is one. Then we have\\
\ \\
\textbf{Corollary.} \hfill \emph{$\models \psi$ \ \ \ iff \ \ \ $\models (\chi \rightarrow \psi^r)$.} \hfill \phantom{Corollary.}\\
\textit{Proof.} Suppose $\models \psi$. Then for every $S^r$-structure $\mathfrak{B}$, if $\mathfrak{B} \models \chi$, there is an $S$-structure $\mathfrak{A}$ such that $\mathfrak{A}^r = \mathfrak{B}$; more precisely,
\begin{enumerate}[(1)]
\item $A = B$;
%%
\item For all $a_1, \ldots, a_n \in A (= B)$, $R^\mathfrak{A} a_1 \ldots a_n$ iff $R^\mathfrak{B} a_1 \ldots a_n$;
%%
\item For all $a_0, \ldots, a_n \in A (= B)$, $f^\mathfrak{A} (a_0, \ldots, a_{n - 1}) = a_n$ iff $F^\mathfrak{B} a_0 \ldots a_n$;
%%
\item For all $a \in A (= B)$, $c^\mathfrak{A} = a$ iff $C^\mathfrak{B} a$.
\end{enumerate}
From (the specialized case of) this theorem it follows that
\begin{center}
$\mathfrak{A} \models \psi$ \ \ \ iff \ \ \ $\mathfrak{B} (= \mathfrak{A}^r) \models \psi^r$.
\end{center}
As $\psi$ is valid, however, it turns out that the righthand-side condition above holds. Hence $\models (\chi \rightarrow \psi^r)$.\\
\ \\
Conversely, suppose $\models (\chi \rightarrow \psi^r)$. Then from (the specialized case of) this theorem it follows that for every $S$-structure $\mathfrak{A}$,
\begin{center}
$\mathfrak{A} \models \psi$ \ \ \ iff \ \ \ $\mathfrak{A}^r \models (\chi \rightarrow \psi^r)$.
\end{center}
Symmetrically, as $(\chi \rightarrow \psi^r)$ is valid, it turns out that the lefthand-side condition above holds. Hence $\models \psi$.\nolinebreak\hfill$\talloblong$
%
\item \textbf{A Parallel to Corollary 1.4.} We state and prove:\medskip\\
\begin{theorem}{Corollary}
For two $S$-structures $\struct{A}$ and $\struct{B}$,\smallskip\\
\centerline{$\struct{A} \iso \struct{B}$ \quad iff \quad $\relational{\struct{A}} \iso \relational{\struct{B}}$.}
\end{theorem}
\begin{proof}
For constant symbol $c \in S$ and $n$-ary function symbol $f \in S$, respectively, we assign unary relation symbol $C \in \relational{S}$ and $(n + 1)$-ary relation symbol $F \in \relational{S}$.\\
\ \\
Assume $\pi : \struct{A} \iso \struct{B}$. Then: For $a \in A$,\smallskip\\
\begin{tabular}[b]{lll}
\   & $\intpted{C}{\relational{\struct{A}}} a$ & \cr
iff & $\intpted{c}{\struct{A}} = a$ & (by definition) \cr
iff & $\intpted{c}{\struct{B}} = \pi (a)$ & (by $\pi: \struct{A} \iso \struct{B}$) \cr
iff & $\intpted{C}{\relational{\struct{B}}} \pi (a)$ & (by definition). \cr
\end{tabular}\medskip\\
For $\seq[1]{a}{n}, a \in A$,\smallskip\\
\begin{tabular}[b]{lll}
\   & $\intpted{F}{\relational{\struct{A}}} \enum[1]{a}{n}a$ & \cr
iff & $\intpted{f}{\struct{A}} (\seq[1]{a}{n}) = a$ & (by definition) \cr
iff & $\intpted{f}{\struct{B}} (\seqp{\pi(a_1)}{\pi(a_n)}) = \pi(a)$ & (by $\pi: \struct{A} \iso \struct{B}$) \cr
iff & $\intpted{F}{\relational{\struct{B}}} \enump{\pi(a_1)}{\pi(a_n)}\pi(a)$ & (by definition). \cr
\end{tabular}\smallskip\\
So we have $\relational{\struct{A}} \iso \relational{\struct{B}}$.\\
\ \\
Conversely, suppose that $\rho: \relational{\struct{A}} \iso \relational{\struct{B}}$. Then:\smallskip\\
\begin{tabular}[b]{llll}
$\rho(\intpted{c}{\struct{A}})$ & $=$ & $\rho(a)$ where $a \in A$ and $\intpted{C}{\relational{\struct{A}}}a$ & (by definition) \cr
\ & $=$ & $\rho(a)$ where $a \in A$ and $\intpted{C}{\relational{\struct{B}}}\rho(a)$ & (by $\rho: \relational{\struct{A}} \iso \relational{\struct{B}}$) \cr
\ & $=$ & $\intpted{c}{\struct{B}}$ & (by definition).
\end{tabular}\medskip\\
For $\seq[1]{a}{n} \in A$, $\rho(\intpted{f}{\struct{A}}(\seq[1]{a}{n}))$\smallskip\\
\begin{tabular}[b]{llll}
$=$ & $\rho(a)$ where $a \in A$ and $\intpted{F}{\relational{\struct{A}}}\enum[1]{a}{n}a$ & (by definition) \cr
$=$ & $\rho(a)$ where $a \in A$ and $\intpted{F}{\relational{\struct{B}}}\enump{\rho(a_1)}{\rho(a_n)}\rho(a)$ & (by $\rho: \relational{\struct{A}} \iso \relational{\struct{B}}$) \cr
$=$ & $\intpted{f}{\struct{B}}(\seqp{\rho(a_1)}{\rho(a_n)})$ & (by definition).
\end{tabular}\smallskip\\
It follows that $\struct{A} \iso \struct{B}$.
\end{proof}
\end{enumerate}
%End of Section VIII.1-----------------------------------------------------------------------------
\
\\
\\
%Section VIII.2------------------------------------------------------------------------------------
{\large \S2. Syntactic Interpretations}
\begin{enumerate}[1.]
\item \textbf{Note to Paragraph E. Syntactic Interpretations on Page 120.} Note that $S$ is not necessarily included in $S^\prime$.
%
\item \textbf{Note to Definition 2.1.} Note that $\varphi_c(v_0) \in L^S_1$. On the other hand,
\[
\varphi_{S^\prime}(v_0) = v_0 \equiv v_0
\]
is the case in which the domain of an $S$-structure coincides with that of an induced $S^\prime$-structure.
%
\item \textbf{Note to the Paragraph Discussing \textit{Identity} on Page 121.} Note that only when $f \in S \cap S^\prime$ and $c \in S \cap S^\prime$ is it meaningful that we talk about the identity $I$ on them. Hence $f \in S^\prime$ and $c \in S^\prime$ should be replaced by ``$f \in S \cap S^\prime$'' and ``$c \in S \cap S^\prime$,'' respectively.
%
\item \textbf{Note to the Syntactic Interpretation $I$ of $S_{\mbox{\scriptsize gr}}$ in $S_{\mbox{\scriptsize ar}}$ on Page 122.} The item
\[
I(\circ) := x \cdot y = z
\]
should be replaced by
\[
I(\circ) := x \cdot y \equiv z.
\]
Moreover, the item
\[
I(e) := x \equiv 1
\]
is missing in textbook.\\
\\
On the other hand, $\Phi_I$ is, by definition,
\[
\{\exists x \varepsilon (x), \, \forall x \forall y (\varepsilon (x) \land \varepsilon (y) \rightarrow \exists^{=1} z (\varepsilon (z) \land x \cdot y \equiv z)), \, \exists^{=1} x (\varepsilon (x) \land x \equiv 1)\}.
\]
And it is equivalent to
\[
\{\exists x \varepsilon (x)\},
\]
provided that the underlying structure $\mathfrak{A}$ is a ring. (The other two sentences are derivable from this one, see the discussion in paragraph C.)
%
\item \textbf{Note to the Paragraph Discussing the Syntactic Interpretation $I$ of $S^\prime = \{<, \leq\}$ in $S = \{<\}$ on Page 122.} By definition, $\Phi_I = \{ \exists v_0 \, v_0 \equiv v_0 \}$. Hence it is equivalent to the empty set.\\
\\On the other hand, that $\mathfrak{A} \models \Phi_{\mbox{\scriptsize ord}}$ implies $\mathfrak{A}^{-I} \models \Phi^\prime_{\mbox{\scriptsize ord}}$ arises from the definition of $\varphi_\leq$.\\
\\
Furthermore, $(\mathfrak{B}|_S)^{-I} \models \Phi_{\mbox{\scriptsize ord}}^\prime$ implies $\mathfrak{B}|_S \models \Phi_{\mbox{\scriptsize ord}}$. Thus there is a bijective map between $\modelclass{S^\prime}{}{\Phi^\prime_{\mbox{\scriptsize ord}}}$ and $\modelclass{S}{}{\Phi_{\mbox{\scriptsize ord}}}$.
%
\item \textbf{Note to the Paragraph Discussing the Syntactic Interpretation $I$ of $S_{\mbox{\scriptsize grp}}$ in $S_{\mbox{\scriptsize g}}$ in Page 122.} $\mathfrak{A} = (A, \circ^A)$ is a group means that $\mathfrak{A} \models \Phi_{\mbox{\scriptsize g}}$. Also, $\Phi_I$ is equivalent to
\[
\{ \forall x \forall y \exists^{=1}z \; x \circ y \equiv z, \; \exists^{=1}x \forall y \; y \circ x \equiv y \},
\]
which in turn is equivalent to
\[
\{ \exists^{=1} x \forall y \; y \circ x \equiv y \}.
\]
%
\item \textbf{Note to the Proof of Theorem 2.2.} Note that for all variables $x$, $\beta (x) \in A^{-I}$.
%
\item \textbf{Note to the Equation (+) in Page 123.} Note that $P^A$ is a subset of $A$.
%
\item \textbf{Note to the First Paragraph in Page 124.} The reason for \textit{$P^A$ being $S$-closed implies that it is $S \cup \{ P \}$-closed} is that $P$ is a relation symbol and we ignore relation symbols when it comes to talking about the concept of $S$-closed.
%
\item \textbf{Note to Lemma 2.3.} Following the definition of the relativization $\psi^P$ of $\psi \in L^S$ to $P$ given in page 124, we give a direct proof of 2.3 here.\\
\\
First note that, by Theorem 1.2, it suffices to consider those term-reduced $\psi \in L^S$.
\begin{enumerate}[1.]
\item $\psi$ is atomic, thus $\psi^P = \psi$:
\begin{enumerate}[(1)]
\item $\psi = x \equiv y$:
\[
\begin{array}{ll}
\    & ([P^A]^{\mathfrak{A}}, \beta) \models x \equiv y \\
\Iff & ([P^A]^{\mathfrak{A}}, \beta)(x) = ([P^A]^{\mathfrak{A}}, \beta)(y) \\
\Iff & \beta(x) = \beta(y) \\
\Iff & (\mathfrak{A}, \beta)(x) = (\mathfrak{A}, \beta)(y) \\
\Iff & (\mathfrak{A}, \beta) \models x \equiv y.
\end{array}
\]
%%%
\item $\psi = fx_1 \ldots x_n \equiv x$:
\[
\begin{array}{ll}
\    & ([P^A]^{\mathfrak{A}}, \beta) \models fx_1 \ldots x_n \equiv x \\
\Iff & f^{[P^A]^{\mathfrak{A}}}(([P^A]^{\mathfrak{A}}, \beta)(x_1), \ldots, ([P^A]^{\mathfrak{A}}, \beta)(x_n)) = ([P^A]^{\mathfrak{A}}, \beta)(x) \\
\Iff & f^{[P^A]^{\mathfrak{A}}}(\beta(x_1), \ldots, \beta(x_n)) = \beta(x) \\
\Iff & f^{\mathfrak{A}}(\beta(x_1), \ldots, \beta(x_n)) = \beta(x) \\
\    & \mbox{(since $P^A$ is $S$-closed)} \\
\Iff & f^{\mathfrak{A}}((\mathfrak{A}, \beta)(x_1), \ldots, (\mathfrak{A}, \beta)(x_n)) = (\mathfrak{A}, \beta)(x) \\
\Iff & (\mathfrak{A}, \beta) \models fx_1 \ldots x_n \equiv x.
\end{array}
\]
%%%
\item $\psi = c \equiv x$:
\[
\begin{array}{ll}
\    & ([P^A]^{\mathfrak{A}}, \beta) \models c \equiv x \\
\Iff & c^{[P^A]^{\mathfrak{A}}} = ([P^A]^{\mathfrak{A}}, \beta)(x) \\
\Iff & c^{[P^A]^{\mathfrak{A}}} = \beta(x) \\
\Iff & c^{\mathfrak{A}} = \beta(x) \\
\    & \mbox{(since $P^A$ is $S$-closed)} \\
\Iff & c^{\mathfrak{A}} = (\mathfrak{A}, \beta)(x) \\
\Iff & (\mathfrak{A}, \beta) \models c \equiv x.
\end{array}
\]
\end{enumerate}
%%
\item $\psi = \neg \varphi$, thus $\psi^P = \neg \varphi^P$:
\[
\begin{array}{ll}
\    & ([P^A]^{\mathfrak{A}}, \beta) \models \neg \varphi \\
\Iff & \mbox{not $([P^A]^{\mathfrak{A}}, \beta) \models \varphi$} \\
\Iff & \mbox{not $(\mathfrak{A}, \beta) \models \varphi^P$} \\
\    & \mbox{(by induction hypothesis)} \\
\Iff & (\mathfrak{A}, \beta) \models \neg \varphi^P.
\end{array}
\]
%%
\item $\psi = (\varphi_1 \lor \varphi_2)$, thus $\psi^P = (\varphi_1^P \lor \varphi_2^P)$:
\[
\begin{array}{ll}
\    & ([P^A]^{\mathfrak{A}}, \beta) \models (\varphi_1 \lor \varphi_2) \\
\Iff & \mbox{$([P^A]^{\mathfrak{A}}, \beta) \models \varphi_1$ or $([P^A]^{\mathfrak{A}}, \beta) \models \varphi_2$} \\
\Iff & \mbox{$(\mathfrak{A}, \beta) \models \varphi_1^P$ or $(\mathfrak{A}, \beta) \models \varphi_2^P$} \\
\    & \mbox{(by induction hypothesis)} \\
\Iff & \mathfrak{A} \models (\varphi_1^P \lor \varphi_2^P).
\end{array}
\]
%%
\item $\psi = (\exists x \varphi)$, thus $\psi^P = \exists x (Px \land \varphi^P)$):
\[
\begin{array}{ll}
\    & ([P^A]^{\mathfrak{A}}, \beta) \models \exists x \varphi \\
\Iff & \mbox{there is some $a \in P^A$ such that $([P^A]^{\mathfrak{A}}, \beta\frac{a}{x}) \models \varphi$} \\
\    & \mbox{(note that $([P^A]^{\mathfrak{A}}, \beta)\frac{a}{x} := ([P^A]^{\mathfrak{A}}, \beta\frac{a}{x})$)} \\
\Iff & \mbox{there is some $a \in P^A$ such that $(\mathfrak{A}, \beta\frac{a}{x}) \models \varphi^P$} \\
\    & \mbox{(by induction hypothesis)} \\
\Iff & \mbox{there is some $a \in A$ such that $P^{\mathfrak{A}}a$ and $(\mathfrak{A}, \beta\frac{a}{x}) \models \varphi^P$} \\
\Iff & \mbox{there is some $a \in A$ such that $(\mathfrak{A}, \beta\frac{a}{x}) \models Px$ and $(\mathfrak{A}, \beta\frac{a}{x}) \models \varphi^P$} \\
\Iff & \mbox{there is some $a \in A$ such that $(\mathfrak{A}, \beta\frac{a}{x}) \models (Px \land \varphi^P)$} \\
\Iff & (\mathfrak{A}, \beta) \models \exists x (Px \land \varphi^P).
\end{array}
\]
\end{enumerate} \begin{flushright}$\talloblong$\end{flushright}
%
%VIII.2.4----------------------------------------------------------------------------------------------------
\item \textbf{Solution to Exercise 2.4.} First note that the following, as a corollary to 2.3 (the Relativization Lemma), is clear:

\textit{Assume the premises of 2.3, and additionally let $Q \not \in S \cup \{ P \}$ be unary such that $Q^A \subset P^A$ is $S \cup \{ P \}$-closed in $[P^A]^{\mathfrak{A}}$ (and hence in $\mathfrak{A}$). Then for all $\psi \in L^S_0$,
\[
\mbox{$[P^A]^{\mathfrak{A}} \models \psi^Q$ iff $\mathfrak{A} \models [\psi^P]^Q$}.
\]
}\\
Therefore, we have for all $\psi \in L^S_0$,
\[
\begin{array}{ll}
\    & (\mathfrak{A}, U^A, V^A) \models [\psi^V]^U \\
\Iff & [V^A]^{(\mathfrak{A}, U^A, V^A)} \models \psi^U \\
\    & \mbox{(since $U^A \subset V^A$ and is $S$-closed (and hence $S \cup \{ V \}$-closed) in} \\
\    & \mbox{$[V^A]^{(\mathfrak{A}, U^A, V^A)}$, and by the discussion above)} \\
\Iff & \mbox{$[U^A]^{(\mathfrak{A}, U^A, V^A)} \models \psi$ (by Lemma 2.3)} \\
\Iff & \mbox{$(\mathfrak{A}, U^A, V^A) \models \psi^U$ (by Lemma 2.3)},
\end{array}
\]
i.e. $(\mathfrak{A}, U^A, V^A) \models ([\psi^V]^U \leftrightarrow \psi^U)$. \begin{flushright}$\talloblong$\end{flushright}
%End of VIII.2.4---------------------------------------------------------------------------------------------
%
%VIII.2.5----------------------------------------------------------------------------------------------------
\item \textbf{Solution to Exercise 2.5.}
\begin{enumerate}[(a)]
\item Let $I$ be the syntactic interpretation of $\{ < \}$ in $\{ \leq \}$ such that
\[
\begin{array}{lll}
\varphi_{\{ < \}} & := & v_0 \equiv v_0; \\
\varphi_<         & := & v_0 \leq v_1.
\end{array}
\]
Then $\Phi_I$ is equivalent to the empty set. Define $\psi := \varphi^I$, then it follows from Theorem 2.2 that for every $\varphi \in L_0^{\{ < \}}$ there is a $\psi \in L_0^{\{ \leq \}}$ such that
\[
\mbox{$(A, <^A) \models \varphi$ iff $(A, \leq^A) \models \psi$}.
\]
\ 
\\
Conversely, let $I^\prime$ be the syntactic interpretation of $\{ \leq \}$ in $\{ < \}$ such that
\[
\begin{array}{lll}
\psi_{\{ \leq \}} & := & v_0 \equiv v_0; \\
\psi_\leq         & := & v_0 < v_1.
\end{array}
\]
Then $\Phi_{I^\prime}$ is equivalent to the empty set. Define $\varphi := \psi^{I^\prime}$, then it follows from Theorem 2.2 that for every $\psi \in L_0^{\{ \leq \}}$ there is a $\varphi \in L_0^{\{ < \}}$ such that
\[
\mbox{$(A, <^A) \models \varphi$ iff $(A, \leq^A) \models \psi$}.
\]
%%
\item The set of axioms for orderings involving only $\leq$ (in the sense of ``$\leq$'') is
\[
\Phi^{\prime\prime}_{\mbox{\scriptsize ord}} := \left\{
\begin{array}{l}
\forall x \;x \leq x \\
\forall x \forall y ((x \leq y \land y \leq x) \rightarrow x \equiv y) \\
\forall x \forall y \forall z ((x \leq y \land y \leq z) \rightarrow x \leq z) \\
\forall x \forall y \forall z ((x \leq y \lor y \leq x) \land (\neg x \leq y \lor \neg y \leq x))
\end{array} \right. .
\]
It is not hard to verify that a $\{ \leq \}$-structure which is a model of $\Phi^{\prime\prime}_{\mbox{\scriptsize ord}}$ is an ordering.\\
\\
Let $I$ be the syntactic interpretation of $\{ \leq \}$ in $\{ < \}$ such that
\[
\begin{array}{lll}
\psi_{\{ \leq \}} & := & v_0 \equiv v_0; \\
\psi_\leq         & := & (v_0 < v_1 \lor v_0 \equiv v_1).
\end{array}
\]
Then $\Phi_I$ is equivalent to the empty set. Define $\varphi := \psi^I$, then it follows from Theorem 2.2 that for every $\psi \in L_0^{\{ \leq \}}$ there is a $\varphi \in L_0^{\{ < \}}$ such that
\[
\mbox{$(A, \leq^A)$ in the sense of ``$\leq$'' $\models \psi$ iff $(A, <^A) \models \varphi$}.
\]
\ 
\\
Conversely, let $I^\prime$ be the syntactic interpretation of $\{ < \}$ in $\{ \leq \}$ such that
\[
\begin{array}{lll}
\varphi_{\{ < \}} & := & v_0 \equiv v_0; \\
\psi_<            & := & (v_0 \leq v_1 \land \neg v_0 \equiv v_1).
\end{array}
\]
Then $\Phi_{I^\prime}$ is equivalent to the empty set. Define $\psi := \varphi^{I^\prime}$, then it follows from Theorem 2.2 that for every $\varphi \in L_0^{\{ < \}}$ there is a $\psi \in L_0^{\{ \leq \}}$ such that
\[
\mbox{$(A, \leq^A)$ in the sense of ``$\leq$'' $\models \psi$ iff $(A, <^A) \models \varphi$}.
\]
\end{enumerate} \begin{flushright}$\talloblong$\end{flushright}
%End of VIII.2.5---------------------------------------------------------------------------------------------
%
%VIII.2.6----------------------------------------------------------------------------------------------------
\item \textbf{Solution to Exercise 2.6.} Define $I$ to be the syntactic interpretation of $S_{\mbox{\scriptsize g}}$ in $S_{\mbox{\scriptsize grp}}$ as follows:
\[
\begin{array}{lll}
\varphi_{S_{\mbox{\tiny g}}} & := & x \equiv x \\
\varphi_\circ & := & x \circ y \equiv z.
\end{array}
\]
Then $\Phi_I$ is equivalent to $\{ \forall x \forall y \exists z \; x \circ y \equiv z \}$.\\
\\
If an $S_{\mbox{\scriptsize grp}}$-structure $\mathfrak{A} = (A, \circ^A, ^{-1^A}, e^A)$ is a model of $\Phi_{\mbox{\scriptsize grp}}$ (i.e. a group), then it is natural that $\mathfrak{A} \models \Phi_I$. In particular,
\[
\mathfrak{A}^{-I} = \mathfrak{A} |_{S_{\mbox{\tiny g}}} = (A, \circ^A).
\]
Theorem 2.2 yields for every $\varphi \in L_0^{S_{\mbox{\tiny g}}}$,
\[
\mbox{$\mathfrak{A}^{-I} \models \varphi$ iff $\mathfrak{A} \models \varphi^I$}
\]
and $\mathfrak{A} \models \Phi_{\mbox{\scriptsize grp}}$ implies $\mathfrak{A}^{-I} \models \Phi_{\mbox{\scriptsize g}}$. (See page 118 for the definition of $\Phi_{\mbox{\scriptsize g}}$.)\\
\\
On the other hand, if an $S_{\mbox{\scriptsize g}}$-structure $\mathfrak{B}$ is a model of $\Phi_{\mbox{\scriptsize g}}$ (i.e. a group) with identity element $e^B$ and inverse function $^{-1^B}$, then
\[
\mathfrak{B} = (\mathfrak{B}, ^{-1^B}, e^B) |_{S_{\mbox{\tiny g}}} = (\mathfrak{B}, ^{-1^B}, e^B)^{-I}
\]
and for every $\varphi \in L_0^{S_{\mbox{\tiny g}}}$,
\[
\mbox{$\mathfrak{B} \models \varphi$ iff $(\mathfrak{B}, ^{-1^B}, e^B) \models \varphi^I$}.
\]
Therefore, $\mathfrak{B} \models \Phi_{\mbox{\scriptsize g}}$ implies $(\mathfrak{B}, ^{-1^B}, e^B) \models \Phi_{\mbox{\scriptsize grp}}$.\\
\\
Thus for every $\varphi \in L_0^{S_{\mbox{\tiny g}}}$,
\[
\mbox{$\Phi_{\mbox{\scriptsize g}} \models \varphi$ iff $\Phi_{\mbox{\scriptsize grp}} \models \varphi^I$}.
\] \begin{flushright}$\talloblong$\end{flushright}
%End of VIII.2.6---------------------------------------------------------------------------------------------
%
%VIII.2.7----------------------------------------------------------------------------------------------------
\item \textbf{Solution to Exercise 2.7.}
\begin{enumerate}[(a)]
\item In the following we write $x$, $y$, \ldots for $v_0$, $v_1$, \ldots. Let $I : S_{\mbox{\scriptsize ar}} \rightarrow S_{\mbox{\scriptsize ar}}$ be the following syntactic interpretation of $S_{\mbox{\scriptsize ar}}$ in $S_{\mbox{\scriptsize ar}}$:
\[
\begin{array}{lll}
\varphi_{S_{\mbox{\scriptsize ar}}} & := & \exists x_1 \exists x_2 \exists x_3 \exists x_4 \; x \equiv x_1 \cdot x_1 + x_2 \cdot x_2 + x_3 \cdot x_3 + x_4 \cdot x_4; \\
\varphi_+ & := & x + y \equiv z; \\
\varphi_\cdot & := & x \cdot y \equiv z; \\
\varphi_0 & := & \forall x \; x + z \equiv x; \\
\varphi_1 & := & \forall x \; x \cdot u \equiv x.
\end{array}
\]
Then
\[
\begin{array}{lll}
\Phi_I & := & \{ \exists x \varphi_{S_{\mbox{\scriptsize ar}}}, \\
\      & \  & \phantom{\{}\forall x \forall y ((\varphi_{S_{\mbox{\scriptsize ar}}}(x) \land \varphi_{S_{\mbox{\scriptsize ar}}}(y)) \rightarrow \exists^{=1}z(\varphi_{S_{\mbox{\scriptsize ar}}}(z) \land x + y \equiv z)), \\
\      & \  & \phantom{\{}\forall x \forall y ((\varphi_{S_{\mbox{\scriptsize ar}}}(x) \land \varphi_{S_{\mbox{\scriptsize ar}}}(y)) \rightarrow \exists^{=1}z(\varphi_{S_{\mbox{\scriptsize ar}}}(z) \land x \cdot y \equiv z)), \\
\      & \  & \phantom{\{}\exists^{=1}z(\varphi_{S_{\mbox{\scriptsize ar}}}(z) \land \forall x \; x + z \equiv x), \\
\      & \  & \phantom{\{}\exists^{=1}u(\varphi_{S_{\mbox{\scriptsize ar}}}(u) \land \forall x \; x \cdot u \equiv x)\}
\end{array}
\]
and $(\mathbb{Z}, +, \cdot, 0, 1) \models \Phi_I$.\\
\\
From the hint it follows that $(\mathbb{Z}, +, \cdot, 0, 1)^{-I} = (\mathbb{N}, +, \cdot, 0, 1)$, and by the Theorem on Syntacic Interpretations we obtain for all $\varphi \in L_0^{S_{\mbox{\tiny ar}}}$,\\
\ \\
\phantom{a} \hfill $(\mathbb{N}, +, \cdot, 0, 1) \models \varphi$ iff $(\mathbb{Z}, +, \cdot, 0, 1) \models \varphi^I$. \hfill $\talloblong$\\
\ \\
\textit{Remark.} The statement in the hint, ``every natural number can be written as the sum of four squares of integers'', is known as \emph{Lagrange's Four-Square Theorem}.
%%
\item First note that, there is a bijection $\pi: \mathbb{N} \mapsto \mathbb{Z}$ between $\mathbb{N}$ and $\mathbb{Z}$,
\[
\pi(n) := \begin{cases}
\displaystyle -\frac{n + 1}{2}, & \mbox{if \(n\) is odd};\cr
\displaystyle \frac{n}{2}, & \mbox{if \(n\) is even},
\end{cases}
\]
of which we list some initial values below:
\[
\begin{tabular}{c||cccccc}
$n$ & $0$ & $1$ & $2$ & $3$ & $4$ & $\cdots$ \\ \hline
$\pi(n)$ & $0$ & $-1$ & $1$ & $-2$ & $2$ & $\cdots$
\end{tabular}.
\]
It is straightforward to define functions $+^\prime$, $\cdot^\prime$ over $\mathbb{N}$, and to assign elements in $\mathbb{N}$ to constants $0^\prime$, $1^\prime$ such that $(\mathbb{N}, +^\prime, \cdot^\prime, 0^\prime, 1^\prime)$ and $(\mathbb{Z}, +, \cdot, 0, 1)$, as two $S_{\mbox{\scriptsize ar}}$-structures, are isomorphic ($(\mathbb{N}, +^\prime, \cdot^\prime, 0^\prime, 1^\prime) \cong (\mathbb{Z}, +, \cdot, 0, 1)$), as is done below: (Note that we use superscripts to avoid ambiguities.)
\[
\begin{array}{lllll}
0^\prime & := & \pi^{-1}(0^\mathbb{Z}) & = & 0^\mathbb{N}, \cr
1^\prime & := & \pi^{-1}(1^\mathbb{Z}) & = & 2^\mathbb{N}, \cr
\end{array}
\]
and for all $m, n \in \mathbb{N}$,
\[
\begin{array}{lll}
m +^\prime n & := & \pi^{-1}(\pi(m) +^\mathbb{Z} \pi(n)), \cr
m \cdot^\prime n & := & \pi^{-1}(\pi(m) \cdot^\mathbb{Z} \pi(n)).
\end{array}
\]
More precisely, the ``behavior'' of $+^\prime$ and $\cdot^\prime$ are presented as follows: for all $x, y \in \mathbb{N}$,
\[
x +^\prime y := \begin{cases}
x + y, & \mbox{if \(x\) and \(y\) are both even}; \cr
x + y + 1, & \mbox{if \(x\) and \(y\) are both odd}; \cr
y - x, & \mbox{if \(x\) is even, \(y\) is odd, and \(x < y\)}; \cr
x - y - 1, & \mbox{if \(x\) is even, \(y\) is odd, and \(y < x\)}; \cr
y +^\prime x, & \mbox{otherwise},
\end{cases}
\]
and
\[
x \cdot^\prime y := \begin{cases}
\displaystyle \frac{x \cdot y}{2}, & \mbox{if \(x\) and \(y\) are both even}; \cr
\displaystyle \frac{(x+1) \cdot (y+1)}{2}, & \mbox{if \(x\) and \(y\) are both odd}; \cr
\displaystyle \frac{x \cdot (y+1)}{2} - 1, & \mbox{if \(x\) is even and \(y\) is odd}; \cr
y \cdot^\prime x, & \mbox{otherwise}.
\end{cases}
\]
\\
By the Isomorphism Lemma, it follows that for all $\varphi \in L_0^{S_{\mbox{\tiny ar}}}$,
\[
\mbox{$(\mathbb{Z}, +, \cdot, 0, 1) \models \varphi$ \  iff \  $(\mathbb{N}, +^\prime, \cdot^\prime, 0^\prime, 1^\prime) \models \varphi$}.
\]
(Note that if the assignment in $(\mathbb{N}, +^\prime, \cdot^\prime, 0^\prime, 1^\prime)$ corresponds to the one in $(\mathbb{Z}, +, \cdot, 0, 1)$ according to $\pi$, then the above statement can be generalized to all $\varphi \in L^{S_{\mbox{\tiny ar}}}$.)\\
\\
The proof is complete if we can provide a syntactic interpretation $I$ of $S_{\mbox{\scriptsize ar}}$ in $S_{\mbox{\scriptsize ar}}$ such that, for all $\varphi \in L_0^{S_{\mbox{\tiny ar}}}$:
\[
\mbox{$\;\;\;$ $(\mathbb{N}, +^\prime, \cdot^\prime, 0^\prime, 1^\prime) \models \varphi$ \  iff \  $(\mathbb{N}, +, \cdot, 0, 1) \models \varphi^I$}.
\]
This is not hard to achieve, as is done below:
\\
Consider the following syntactic interpretation $I$ of $S_{\mbox{\scriptsize ar}}$ in $S_{\mbox{\scriptsize ar}}$:
\[
\begin{array}{lll}
\varphi_{S_{\mbox{\tiny ar}}}(v_0) & := & v_0 \equiv v_0, \\
\varphi_+(v_0, v_1, v_2) & := & \phantom{\land}(((\varphi_{\mbox{\scriptsize even}}(v_0) \land \varphi_{\mbox{\scriptsize even}}(v_1)) \rightarrow v_0 + v_1 \equiv v_2) \\
\  & \  & \land ((\varphi_{\mbox{\scriptsize odd}}(v_0) \land \varphi_{\mbox{\scriptsize odd}}(v_1)) \rightarrow (v_0 + v_1) + 1 \equiv v_2) \\
\  & \  & \land ((\varphi_{\mbox{\scriptsize even}}(v_0) \land \varphi_{\mbox{\scriptsize odd}}(v_1) \land \varphi_< (v_0, v_1)) \\
\  & \  & \phantom{\land(} \rightarrow v_0 + v_2 \equiv v_1) \\
\  & \  & \land ((\varphi_{\mbox{\scriptsize even}}(v_0) \land \varphi_{\mbox{\scriptsize odd}}(v_1) \land \varphi_< (v_1, v_0)) \\
\  & \  & \phantom{\land(} \rightarrow (v_1 + v_2) + 1 \equiv v_0) \\
\  & \  & \land ((\varphi_{\mbox{\scriptsize odd}}(v_0) \land \varphi_{\mbox{\scriptsize even}}(v_1) \land \varphi_< (v_1, v_0)) \\
\  & \  & \phantom{\land(} \rightarrow v_1 + v_2 \equiv v_0) \\
\  & \  & \land ((\varphi_{\mbox{\scriptsize odd}}(v_0) \land \varphi_{\mbox{\scriptsize even}}(v_1) \land \varphi_< (v_0, v_1)) \\
\  & \  & \phantom{\land(} \rightarrow (v_0 + v_2) + 1 \equiv v_1)), \\
\varphi_\cdot(v_0, v_1, v_2) & := & \phantom{\land}(((\varphi_{\mbox{\scriptsize even}}(v_0) \land \varphi_{\mbox{\scriptsize even}}(v_1)) \\
\  & \  & \phantom{\land(} \rightarrow v_0 \cdot v_1 \equiv (1 + 1) \cdot v_2) \\
\  & \  & \land ((\varphi_{\mbox{\scriptsize odd}}(v_0) \land \varphi_{\mbox{\scriptsize odd}}(v_1)) \\
\  & \  & \phantom{\land(} \rightarrow (v_0 + 1) \cdot (v_1 + 1) \equiv (1 + 1) \cdot v_2) \\
\  & \  & \land ((\varphi_{\mbox{\scriptsize even}}(v_0) \land \varphi_{\mbox{\scriptsize odd}}(v_1)) \\
\  & \  & \phantom{\land(} \rightarrow v_0 \cdot (v_1 + 1) \equiv (1 + 1) \cdot (v_2 + 1)) \\
\  & \  & \land ((\varphi_{\mbox{\scriptsize odd}}(v_0) \land \varphi_{\mbox{\scriptsize even}}(v_1)) \\
\  & \  & \phantom{\land(} \rightarrow (v_0 + 1) \cdot v_1 \equiv (1 + 1) \cdot (v_2 + 1)), \\
\varphi_0(v_0) & := & 0 \equiv v_0, \\
\varphi_1(v_0) & := & (1 + 1) \equiv v_0,
\end{array}
\]
where
\[
\begin{array}{lll}
\varphi_{\mbox{\scriptsize even}}(v_0) & := & \exists v_0^\prime (1 + 1) \cdot v_0^\prime \equiv v_0, \\
\varphi_{\mbox{\scriptsize odd}}(v_0)  & := & \exists v_0^\prime ((1 + 1) \cdot v_0^\prime) + 1 \equiv v_0, \\
\varphi_<(v_0, v_1)                    & := & \exists v_2 (\neg v_2 \equiv 0 \land v_0 + v_2 \equiv v_1).
\end{array}
\]
\\
It is easy to verify that $(\mathbb{N}, +, \cdot, 0, 1) \models \Phi_I$ and hence $(\mathbb{N}, +^\prime, \cdot^\prime, 0^\prime, 1^\prime) = (\mathbb{N}, +, \cdot, 0, 1)^{-I}$. Therefore, by the Theorem on Syntactic Interpretations, we have for all $\varphi \in L_0^{S_{\mbox{\tiny ar}}}$,
\[
\mbox{$\;\;\;$ $(\mathbb{N}, +^\prime, \cdot^\prime, 0^\prime, 1^\prime) \models \varphi$ \  iff \  $(\mathbb{N}, +, \cdot, 0, 1) \models \varphi^I$}.
\]
\end{enumerate} \begin{flushright}$\talloblong$\end{flushright}
%End of VIII.2.7---------------------------------------------------------------------------------------------
%
%VIII.2.8----------------------------------------------------------------------------------------------------
\item \textbf{Solution to Exercise 2.8.}
\begin{enumerate}
\item \textit{For every $\psi \in L^S$ there is $\psi^r \in L^{S^r}$ such that for all $S$-interpretation $\mathfrak{I} = (\mathfrak{A}, \beta)$,
\[
\mbox{$(\mathfrak{A}, \beta) \models \psi$ iff $(\mathfrak{A}^r, \beta) \models \psi^r$}.
\]}
\\
\textit{Proof.} We define the syntactic interpretation $I : S \rightarrow S^r$ as follows:
\[
\begin{array}{llll}
\varphi_S(v_0) & := & v_0 \equiv v_0; & \  \\
\varphi_P(v_0, \ldots, v_{n-1}) & := & Pv_0 \ldots v_{n-1} & \mbox{for $n$-ary relation symbol $P$}; \\
\varphi_f(v_0, \ldots, v_n) & := & Fv_0 \ldots v_n & \mbox{for $n$-ary function symbol $f$,} \\
\                           & \  & \               & \mbox{where $F \in S^r$ is its} \\
\                           & \  & \               & \mbox{counterpart}; \\
\varphi_c(v_0) & := & Cv_0 & \mbox{for constant $c$, where $C \in S^r$} \\
\                           & \  & \               & \mbox{is its counterpart}.
\end{array}
\]
Then $\Phi_I$ is equivalent to
\[
\{ \forall v_0 \ldots \forall v_{n-1} \exists^{=1} v_n Fv_0 \ldots v_n,\; \exists^{=1}v_0Cv_0\}.
\]
Furthermore, for every $S$-structure $\mathfrak{A}$, its counterpart---the $S^r$-structure $\mathfrak{A}^r$---is a model of $\Phi_I$ and $(\mathfrak{A}^r)^{-I} = \mathfrak{A}$. After using Theorem 2.2, we immediately get the result.
%%
\item \textit{For every $\psi \in L^{S^r}$ there is $\psi^{-r} \in L^S$ such that for all $S$-interpretations $\mathfrak{I} = (\mathfrak{A}, \beta)$,
\[
\mbox{$(\mathfrak{A}, \beta) \models \psi^{-r}$ iff $(\mathfrak{A}^r, \beta) \models \psi$}.
\]}
\\
\textit{Proof.} We define the syntactic interpretation $I : S^r \rightarrow S$ as follows:
\[
\begin{array}{llll}
\varphi_{S^r} & := & v_0 \equiv v_0; & \  \\
\varphi_P     & := & Pv_0 \ldots v_{n-1} & \mbox{for $n$-ary relation symbol $P \in S$}; \\
\varphi_F     & := & fv_0 \ldots v_{n-1} \equiv v_n & \mbox{for $(n+1)$-ary relation symbol} \\
\             & \  & \                              & \mbox{$F \in S^r \setminus S$, whose counterpart} \\
\             & \  & \                              & \mbox{is $f \in S$}; \\
\varphi_C     & := & c \equiv v_0 & \mbox{for unary relation symbol $C \in S^r \setminus S$,}\\
\             & \  & \            & \mbox{whose counterpart is $c \in S$}.
\end{array}
\]
Then $\Phi_I$ is equivalent to the empty set. Furthermore, for every $S$-structure $\mathfrak{A}$, $\mathfrak{A} \models \Phi_I$ and $\mathfrak{A}^{-I} = \mathfrak{A}^r$ (the counterpart of $\mathfrak{A}$, an $S^r$-structure). On the other hand, an assignment $\beta$ in $\mathfrak{A}$ is also one in $\mathfrak{A}^r$. After applying Theorem 2.2, we obtain the result.
\end{enumerate} \begin{flushright}$\talloblong$\end{flushright}
%End of VIII.2.8---------------------------------------------------------------------------------------------
\end{enumerate}
%End of Section VIII.2---------------------------------------------------------------------------------------
\
\\
\\
%Section VIII.3----------------------------------------------------------------------------------------------
{\large \S3. Extensions by Definitions}
\begin{enumerate}[1.]
\item \textbf{Note to ``$\Phi_{\mbox{\scriptsize g}} \models \exists^{=1}x \forall y \; y \circ x \equiv y$'' in Page 125.} Here is a derivation of ``$\Phi_{\mbox{\scriptsize g}} \vdash \exists^{=1}x \forall y \; y \circ x \equiv y$,'' i.e. ``$\Phi_{\mbox{\scriptsize g}} \vdash \exists x (\forall y \; y \circ x \equiv y \land \forall u (\forall y \; y \circ u \equiv y \rightarrow x \equiv u))$.''\\
\\
In the following, let
\[
\begin{array}{lll}
\varphi & := & (\forall x \; x \circ z \equiv x \land \forall x \exists y \; x \circ y \equiv z), \\
\psi    & := & \forall y \; y \circ u \equiv y,
\end{array}
\]
and further,
\[
\begin{array}{lll}
\chi & := & \forall x \forall y \forall z (x \circ y) \circ z \equiv x \circ (y \circ z),\\
\eta & := & \neg (\neg \varphi\frac{x}{z} \lor (\neg \psi \lor x \equiv u)),
\end{array}
\]
where
\[
\varphi\frac{x}{z} = (\forall y \; y \circ x \equiv y \land \forall y \exists y^\prime \; y \circ y^\prime \equiv x).
\]
\[
\begin{array}{lll}
1. & \neg \varphi\frac{x}{z} & \neg \varphi\frac{x}{z} \\
\  & \               & \mbox{(Assm)} \\
2. & \neg \varphi\frac{x}{z} & (\neg \varphi\frac{x}{z} \lor (\neg \psi \lor x \equiv u)) \\
\  & \               & \mbox{($\lor$S) applied to 1.} \\
3. & \eta & \varphi\frac{x}{z} \\
\  & \ & \mbox{(Cp) applied to 2.} \\
4. & \eta & \forall y \; y \circ x \equiv y \\
\  & \ & \mbox{IV.3.6(d1) applied to 3.} \\
5. & \eta & u \circ x \equiv u \\
\  & \ & \mbox{IV.5.5(a1) applied to 4. with $t = u$} \\
6. & \eta \;\; u \circ u^\prime \equiv x & u \circ x \equiv u \\
\  & \ & \mbox{(Ant) applied to 5.} \\
7. & \eta \;\; u \circ u^\prime \equiv x & u \circ u^\prime \equiv x \\
\  & \ & \mbox{(Assm)} \\
8. & \eta \;\; u \circ u^\prime \equiv x & x \equiv u \circ u^\prime \\
\  & \ & \mbox{IV.5.3(a) to 7.} \\
9. & \eta \;\; u \circ u^\prime \equiv x \;\; x \equiv u \circ u^\prime & u \circ (u \circ u^\prime) \equiv u \\
\  & \                                & \mbox{(Sub) applied to 6.} \\
10. & \eta \;\; u \circ u^\prime \equiv x & u \circ (u \circ u^\prime) \equiv u \\
\  & \                                & \mbox{(Ch) applied to 8. and 9.} \\
11. & \chi & \chi \\
\  & \ & \mbox{(Assm)} \\
12. & \chi & \forall y \forall z (u \circ y) \circ z \equiv u \circ (y \circ z) \\
\  & \ & \mbox{IV.5.5(a1) applied to 11. with} \\
\  & \ & \mbox{$t = u$} \\
13. & \chi & \forall z (u \circ u) \circ z \equiv u \circ (u \circ z) \\
\  & \ & \mbox{IV.5.5(a1) applied to 12. with} \\
\  & \ & \mbox{$t = u$} \\
14. & \chi & (u \circ u) \circ u^\prime \equiv u \circ (u \circ u^\prime) \\
\   & \ & \mbox{IV.5.5(a1) applied to 13. with} \\
\   & \ & \mbox{$t = u^\prime$} \\
15. & \chi & u \circ (u \circ u^\prime) \equiv (u \circ u) \circ u^\prime \\
\   & \ & \mbox{IV.5.3(a) applied to 14.} \\
16. & \chi \;\; \eta \;\; u \circ u^\prime \equiv x & u \circ (u \circ u^\prime) \equiv (u \circ u) \circ u^\prime \\
\   & \ & \mbox{(Ant) applied to 15.} \\
17. & \chi \;\; \eta \;\; u \circ u^\prime \equiv x & u \circ (u \circ u^\prime) \equiv u \\
\   & \                                       & \mbox{(Ant) applied to 10.} \\
18. & \chi \;\; \eta \;\; u \circ u^\prime \equiv x  \;\; u \circ (u \circ u^\prime) \equiv u & u \equiv (u \circ u) \circ u^\prime \\
\   & \ & \mbox{(Sub) applied to 16.} \\
19. & \chi \;\; \eta \;\; u \circ u^\prime \equiv x & u \equiv (u \circ u) \circ u^\prime \\
\   & \                                       & \mbox{(Ch) applied to 17. and 18.} \\
20. & \neg \psi & \neg \psi \\
\   & \               & \mbox{(Assm)} \\
21. & \neg \psi & (\neg \psi \lor x \equiv u) \\
\   & \               & \mbox{($\lor$S) applied to 20.}
\end{array}
\]
\[
\begin{array}{lll}
22. & \neg \psi & (\neg \varphi\frac{x}{z} \lor (\neg \psi \lor x \equiv u)) \\
\   & \               & \mbox{($\lor$S) applied to 21.} \\
23. & \eta & \psi \\
\   & \ & \mbox{(Cp) applied to 22.} \\
24. & \eta & u \circ u \equiv u \\
\   & \ & \mbox{IV.5.5(a1) applied to 23. with} \\
\   & \ & \mbox{$t = u$} \\
25. & \chi \;\; \eta \;\; u \circ u^\prime \equiv x & u \circ u \equiv u \\
\   &                                         & \mbox{(Ant) applied to 24.} \\
26. & \chi \;\; \eta \;\; u \circ u^\prime \equiv x \;\; u \circ u \equiv u & u \equiv u \circ u^\prime \\
\   & \                                       & \mbox{(Sub) applied to 19.} \\
27. & \chi \;\; \eta \;\; u \circ u^\prime \equiv x & u \equiv u \circ u^\prime \\
\   & \                                       & \mbox{(Ch) applied to 25. and 26.} \\
28. & \chi \;\; \eta \;\; u \circ u^\prime \equiv x & u \circ u^\prime \equiv x \\
\   & \                                       & \mbox{(Assm)} \\
29. & \chi \;\; \eta \;\; u \circ u^\prime \equiv x \;\; u \circ u^\prime \equiv x & u \equiv x \\
\   & \                                       & \mbox{(Sub) applied to 27.} \\
30. & \chi \;\; \eta \;\; u \circ u^\prime \equiv x & u \equiv x \\
\   & \                                       & \mbox{(Ch) applied to 28. and 29.} \\
31. & \chi \;\; \eta \;\; u \circ u^\prime \equiv x & x \equiv u \\
\   & \                                       & \mbox{IV.5.3(a) applied to 30.} \\
32. & \chi \;\; \eta \;\; \exists y^\prime \; u \circ y^\prime \equiv x & x \equiv u \\
\   & \                                       & \mbox{($\exists$A) applied to 31.} \\
33. & \exists y^\prime \; u \circ y^\prime \equiv x & \exists y^\prime \; u \circ y^\prime \equiv x \\
\   & \                                       & \mbox{(Assm)} \\
34. & \forall y \exists y^\prime \; y \circ y^\prime \equiv x & \exists y^\prime \; u \circ y^\prime \equiv x \\
\   & \                                       & \mbox{IV.5.5(b1) applied to 33. with $t = u$} \\
35. & \varphi\frac{x}{z} \;\; \forall y \exists y^\prime \; y \circ y^\prime \equiv x & \exists y^\prime \; u \circ y^\prime \equiv x \\
\   & \                                       & \mbox{(Ant) applied to 34.} \\
36. & \varphi\frac{x}{z} & \varphi\frac{x}{z} \\
\   & \          & \mbox{(Assm)} \\
37. & \varphi\frac{x}{z} & \forall y \exists y^\prime \; y \circ y^\prime \equiv x \\
\   & \          & \mbox{IV.3.6(d2) applied to 36.} \\
38. & \varphi\frac{x}{z} & \exists y^\prime \; u \circ y^\prime \equiv x \\
\   & \                                       & \mbox{(Ch) applied to 37. and 35.} \\
39. & \eta \;\; \varphi\frac{x}{z} & \exists y^\prime \; u \circ y^\prime \equiv x \\
\   & \                                       & \mbox{(Ant) applied to 38.} \\
40. & \eta & \exists y^\prime \; u \circ y^\prime \equiv x \\
\   & \                                       & \mbox{(Ch) applied to 3. and 39.} \\
41. & \chi \;\; \eta & \exists y^\prime \; u \circ y^\prime \equiv x \\
\   & \                                       & \mbox{(Ant) applied to 40.} \\
42. & \chi \;\; \eta & x \equiv u \\
\   & \                                       & \mbox{(Ch) applied to 41. and 32.} \\
43. & \chi \;\; \eta & (\neg \psi \lor x \equiv u) \\
\   & \                                       & \mbox{($\lor$S) applied to 42.} \\
\end{array}
\]
\[
\begin{array}{lll}
44. & \chi \;\; \eta & (\neg \varphi\frac{x}{z} \lor (\neg \psi \lor x \equiv u)) \\
\   & \                                       & \mbox{($\lor$S) applied to 43.} \\
45. & \chi \;\; \eta & \eta \\
\   & \                                       & \mbox{(Assm)} \\
46. & \chi & (\neg \varphi\frac{x}{z} \lor (\neg \psi \lor x \equiv u)) \\
\   & \                                       & \mbox{(Ctr) applied to 44. and 45.} \\
47. & \chi \;\; \varphi\frac{x}{z} & (\neg \varphi\frac{x}{z} \lor (\psi \rightarrow x \equiv u)) \\
\   & \                                       & \mbox{(Ant) applied to 46.} \\
48. & \chi \;\; \varphi\frac{x}{z} & \varphi\frac{x}{z} \\
\   & \                                       & \mbox{(Assm)} \\
49. & \chi \;\; \varphi\frac{x}{z} & (\psi \rightarrow x \equiv u) \\
\   & \                                       & \mbox{IV.3.5 applied to 47. and 48.} \\
50. & \chi \;\; \varphi\frac{x}{z} & \forall u (\psi \rightarrow x \equiv u) \\
\   & \                                       & \mbox{IV.5.5(b4) applied to 49.} \\
51. & \chi \;\; \varphi\frac{x}{z} & \forall y \; y \circ x \equiv y \\
\   & \                         & \mbox{IV.3.6(d1) applied to 48.} \\
52. & \chi \;\; \varphi\frac{x}{z} & (\forall y \; y \circ x \equiv y \land \forall u (\psi \rightarrow x \equiv u)) \\
\   & \                                       & \mbox{IV.3.6(b) applied to 51. and 50.} \\
53. & \chi \;\; \varphi\frac{x}{z} & \exists x (\forall y \; y \circ x \equiv y \land \forall u (\psi \rightarrow x \equiv u)) \\
\   & \                                       & \mbox{IV.5.1(a) applied to 52.} \\
54. & \chi \;\; \exists z \varphi & \exists x (\forall y \; y \circ x \equiv y \land \forall u (\psi \rightarrow x \equiv u)) \\
\   & \                                       & \mbox{($\exists$A) applied to 53.}
\end{array}
\]
By the Adequacy Theorem V.4.2, we have $\Phi_{\mbox{\scriptsize g}} \models \exists^{=1}x \forall y \; y \circ x \equiv y$.
%
\item \textbf{Note to the Equivalence between the Sets $\Phi_{\mbox{\scriptsize g}} \cup \{ \delta_e \}$ and $\Phi_{\mbox{\scriptsize gr}}$ of $S_{\mbox{\scriptsize gr}}$-Sentences.} Here we shall show that
\[
\mbox{for every $\varphi \in \Phi_{\mbox{\scriptsize g}} \cup \{ \delta_e \}$, $\Phi_{\mbox{\scriptsize gr}} \models \varphi$,}
\]
and
\[
\mbox{for every $\varphi \in \Phi_{\mbox{\scriptsize gr}}$, $\Phi_{\mbox{\scriptsize g}} \cup \{ \delta_e \} \models \varphi$.}
\]
Also note that in the case of $\Phi_{\mbox{\scriptsize gr}}$, we write $x$, $y$, $z$ for $v_0$, $v_1$, $v_2$.
\\
\begin{enumerate}[(i)]
\item For every $\varphi \in \Phi_{\mbox{\scriptsize g}} \cup \{ \delta_e \}$, $\Phi_{\mbox{\scriptsize gr}} \models \varphi$: Let $\Gamma$ be the sequent that consists of all three sentences from $\Phi_{\mbox{\scriptsize gr}}$.
\begin{enumerate}
\item $\forall x \forall y \forall z (x \circ y) \circ z \equiv x \circ (y \circ z)$:
\[
\begin{array}{llll}
1. & \Gamma & \forall x \forall y \forall z (x \circ y) \circ z \equiv x \circ (y \circ z) & \mbox{(Assm)}
\end{array}
\]
%%%
\item $\exists z (\forall x \; x \circ z \equiv x \land \forall x \exists y \; x \circ y \equiv z)$:
\[
\begin{array}{lll}
1. & \Gamma & \forall x \; x \circ e \equiv x \\
\  & \      & \mbox{(Assm)} \\
2. & \Gamma & \forall x \exists y \; x \circ y \equiv e \\
\  & \      & \mbox{(Assm)} \\
3. & \Gamma & (\forall x \; x \circ e \equiv x \land \forall x \exists y \; x \circ y \equiv e) \\
\  & \      & \mbox{IV.3.6(b) applied to 1. and 2.} \\
4. & \Gamma & \exists z (\forall x \; x \circ z \equiv x \land \forall x \exists y \; x \circ y \equiv z) \\
\  & \      & \mbox{($\exists$S) applied to 3.}
\end{array}
\]
%%%
\item $\forall x (e \equiv x \leftrightarrow \forall y \; y \circ x \equiv y)$: In the following, let
\[
\varphi := \forall y \; y \circ x \equiv y,
\]
and
\[
\psi := (\varphi \land \forall u (\varphi\frac{u}{x} \rightarrow x \equiv u)).
\]
(We shall use $\forall y \; y \circ x \equiv y$ and $\varphi$ interchangeably.)
\[
\begin{array}{lll}
1. & \Gamma & \exists x \psi \\
\  & \      & \mbox{Since the succedent is} \\
\  & \      & \mbox{derivable from $\Phi_{\mbox{\scriptsize g}}$} \\
\  & \      & \mbox{(cf. the previous note),} \\
\  & \      & \mbox{which in turn is derivable} \\
\  & \      & \mbox{from $\Gamma$ (cf. the previous} \\
\  & \      & \mbox{two derivations).} \\
2. & \Gamma \;\; \neg x \equiv e \;\; \varphi & \varphi\frac{e}{x} \\
\  & \      & \mbox{(Assm)} \\
3. & \Gamma \;\; \neg x \equiv e \;\; \neg \varphi\frac{e}{x} & \neg \varphi \\
\  & \      & \mbox{(Cp) applied to 2.} \\
4. & \Gamma \;\; \neg x \equiv e \;\; \varphi & \neg x \equiv e \\
\  & \      & \mbox{(Assm)} \\
5. & \Gamma \;\; \neg x \equiv e \;\; x \equiv e & \neg \varphi \\
\  & \      & \mbox{(Cp) applied to 4.} \\
6. & \Gamma \;\; \neg x \equiv e \;\; (\neg \varphi\frac{e}{x} \lor x \equiv e) & \neg \varphi \\
\  & \      & \mbox{($\lor$A) applied to 3. and 5.} \\
7. & \Gamma \;\; \neg x \equiv e \;\; \varphi & \neg (\varphi\frac{e}{x} \rightarrow x \equiv e) \\
\  & \      & \mbox{(Cp) applied to 6.} \\
8. & \Gamma \;\; \neg x \equiv e \;\; \varphi & \exists u \neg (\varphi\frac{u}{x} \rightarrow x \equiv u) \\
\  & \      & \mbox{($\exists$S) applied to 7.} \\
9. & \Gamma \;\; \neg x \equiv e \;\; \varphi & \neg \forall u (\varphi\frac{u}{x} \rightarrow x \equiv u) \\
\  & \      & \mbox{IV.3.6(a1) applied to 8.} \\
10.& \Gamma \;\; \neg x \equiv e & (\neg \varphi \lor \neg \forall u (\varphi\frac{u}{x} \rightarrow x \equiv u)) \\
\  & \      & \mbox{IV.3.6(c) applied to 9.} \\
11.& \Gamma \;\; \psi & x \equiv e \\
\  & \      & \mbox{(Cp) applied to 10.} \\
12.& \Gamma \;\; \psi & \psi \\
\  & \      & \mbox{(Assm)} \\
13.& \Gamma \;\; \psi \;\; x \equiv e & (\varphi\frac{e}{x} \land \forall u (\varphi\frac{u}{x} \rightarrow e \equiv u)) \\
\  & \      & \mbox{(Sub) applied to 12.} \\
14.& \Gamma \;\; \psi & (\varphi\frac{e}{x} \land \forall u (\varphi\frac{u}{x} \rightarrow e \equiv u)) \\
\  & \      & \mbox{(Ch) applied to 11. and 13.} \\
15.& \Gamma \;\; \exists x \psi & (\varphi\frac{e}{x} \land \forall u (\varphi\frac{u}{x} \rightarrow e \equiv u)) \\
\  & \      & \mbox{($\exists$A) applied to 14.} \\
16.& \Gamma & (\varphi\frac{e}{x} \land \forall u (\varphi\frac{u}{x} \rightarrow e \equiv u)) \\
\  & \      & \mbox{(Ch) applied to 1. and 15.} \\
17.& \Gamma & \forall u (\varphi\frac{u}{x} \rightarrow e \equiv u) \\
\  & \      & \mbox{IV.3.6(d2) applied to 16.} \\
18.& \Gamma & (\varphi \rightarrow e \equiv x) \\
\  & \      & \mbox{IV.5.5(a1) applied to 17. with} \\
\  & \      & \mbox{$t = x$} \\
19.& \Gamma & \varphi\frac{e}{x} \\
\  & \      & \mbox{(Assm)}
\end{array}
\]
\[
\begin{array}{lll}
20.& \Gamma \;\; e \equiv x & \varphi \\
\  & \      & \mbox{(Sub) applied to 19. with $t = e$} \\
\  & \      & \mbox{and $t^\prime = x$} \\
21.& \Gamma & (e \equiv x \rightarrow \varphi) \\
\  & \      & \mbox{IV.3.6(c) applied to 20.} \\
22. & \Gamma \;\; \varphi & (\varphi \rightarrow e \equiv x) \\
\  & \      & \mbox{(Ant) applied to 18.} \\
23. & \Gamma \;\; \varphi & \varphi \\
\  & \      & \mbox{(Assm)} \\
24. & \Gamma \;\; \varphi & e \equiv x \\
\  & \      & \mbox{IV.3.5 applied to 22. and 23.} \\
25. & \Gamma \;\; \varphi & (e \equiv x \land \varphi) \\
\  & \      & \mbox{IV.3.6(b) applied to 24. and 23.} \\
26. & \Gamma \;\; \varphi & (\neg (e \equiv x \lor \varphi) \lor (e \equiv x \land \varphi)) \\
\  & \      & \mbox{($\lor$S) applied to 25.} \\
27. & \Gamma \;\; e \equiv x & (e \equiv x \rightarrow \varphi) \\
\  & \      & \mbox{(Ant) applied to 21.} \\
28. & \Gamma \;\; e \equiv x & e \equiv x \\
\  & \      & \mbox{(Assm)} \\
29.& \Gamma \;\; e \equiv x & \varphi \\
\  & \      & \mbox{IV.3.5 applied to 27. and 28.} \\
30.& \Gamma \;\; (e \equiv x \lor \varphi) & \varphi \\
\  & \      & \mbox{($\lor$A) applied to 29. and 23.} \\
31.& \Gamma \;\; \neg \varphi & \neg (e \equiv x \lor \varphi) \\
\  & \      & \mbox{IV.3.3(a) applied to 30.} \\
32.& \Gamma \;\; \neg \varphi & (\neg (e \equiv x \lor \varphi) \lor (e \equiv x \land \varphi)) \\
\  & \      & \mbox{($\lor$S) applied to 31.} \\
33.& \Gamma & (e \equiv x \leftrightarrow \varphi) \\
\  & \      & \mbox{(PC) applied to 26. and 32.} \\
34.& \Gamma & \forall x (e \equiv x \leftrightarrow \forall y \; y \circ x \equiv y) \\
\  & \      & \mbox{IV.5.5(b4) applied to 33.}
\end{array}
\]
\end{enumerate}
%%
\item For every $\varphi \in \Phi_{\mbox{\scriptsize gr}}$, $\Phi_{\mbox{\scriptsize g}} \cup \{ \delta_e \} \models \varphi$: Let $\Gamma$ be the sequent that consists of both sentences from $\Phi_{\mbox{\scriptsize g}}$ together with $\delta_e$.
\begin{enumerate}
\item $\forall x \forall y \forall z (x \circ y) \circ z \equiv x \circ (y \circ z)$:
\[
\begin{array}{llll}
1. & \Gamma & \forall x \forall y \forall z (x \circ y) \circ z \equiv x \circ (y \circ z) & \mbox{(Assm)}
\end{array}
\]
%%%
\item $\forall x \; x \circ e \equiv x$: In the following, let
\[
\varphi := (\forall x \; x \circ z \equiv x \land \forall x \exists y \; x \circ y \equiv z).
\]
\[
\begin{array}{lll}
1. & \Gamma \;\; \varphi & \varphi \\
\  & \                   & \mbox{(Assm)} \\
2. & \Gamma \;\; \varphi & \forall x \; x \circ z \equiv x \\
\  & \                   & \mbox{IV.3.6(d1) applied to 1.} \\
3. & \Gamma \;\; \varphi & y \circ z \equiv y \\
\  & \                   & \mbox{IV.5.5(a1) applied to 2. with $t = y$} \\
4. & \Gamma \;\; \varphi & \forall y \; y \circ z \equiv y \\
\  & \                   & \mbox{IV.5.5(b4) applied to 3.} \\
5. & \Gamma              & \forall x (e \equiv x \leftrightarrow \forall y \; y \circ x \equiv y) \\
\  & \                   & \mbox{(Assm)} \\
6. & \Gamma              & (e \equiv z \leftrightarrow \forall y \; y \circ z \equiv y) \\
\  & \                   & \mbox{IV.5.5(a1) applied to 5. with $t = z$} \\
7. & \Gamma \;\; \forall y \; y \circ z \equiv y & (e \equiv z \leftrightarrow \forall y \; y \circ z \equiv y) \\
\  & \                   & \mbox{(Ant) applied to 6.} \\
8. & \Gamma \;\; \forall y \; y \circ z \equiv y & \forall y \; y \circ z \equiv y \\
\  & \                   & \mbox{(Assm)} \\
9. & \Gamma \;\; \forall y \; y \circ z \equiv y & (e \equiv z \lor \forall y \; y \circ z \equiv y) \\
\  & \                   & \mbox{($\lor$S) applied to 8.} \\
10.& \Gamma \;\; \forall y \; y \circ z \equiv y & (e \equiv z \land \forall y \; y \circ z \equiv y) \\
\  & \                   & \mbox{IV.3.5 applied to 9. and 7.} \\
11.& \Gamma \;\; \forall y \; y \circ z \equiv y & e \equiv z \\
\  & \                   & \mbox{IV.3.6(d1) applied to 10.} \\
12.& \Gamma \;\; \varphi \;\; \forall y \; y \circ z \equiv y & e \equiv z \\
\  & \                   & \mbox{(Ant) applied to 11.} \\
13.& \Gamma \;\; \varphi & e \equiv z \\
\  & \                   & \mbox{(Ch) applied to 4. and 12.} \\
14.& \Gamma \;\; \varphi & z \equiv e \\
\  & \                   & \mbox{IV.5.3(a) applied to 13.} \\
15.& \Gamma \;\; \varphi \;\; z \equiv e & \forall y \; y \circ e \equiv y \\
\  & \                   & \mbox{(Sub) applied to 4. with $t = z$} \\
\  & \                   & \mbox{and $t^\prime = e$} \\
16.& \Gamma \;\; \varphi & \forall y \; y \circ e \equiv y \\
\  & \                   & \mbox{(Ch) applied to 14. and 15.} \\
17.& \Gamma \;\; \varphi & x \circ e \equiv x \\
\  & \                   & \mbox{IV.5.5(a1) applied to 16. with $t = x$} \\
18.& \Gamma \;\; \varphi & \forall x \; x \circ e \equiv x \\
\  & \                   & \mbox{IV.5.5(b4) applied to 17.} \\
19.& \Gamma \;\; \exists z \varphi & \forall x \; x \circ e \equiv x \\
\  & \                   & \mbox{IV.5.1(b) applied to 18.} \\
20.& \Gamma              & \forall x \; x \circ e \equiv x \\
\  & \                   & \mbox{(Ant) applied to 19.}
\end{array}
\]
%%%
\item $\forall x \exists y \; x \circ y \equiv e$: In the following, let
\[
\varphi := (\forall x \; x \circ z \equiv x \land \forall x \exists y \; x \circ y \equiv z).
\]
\[
\begin{array}{lll}
1. & \Gamma \;\; \varphi & \varphi \\
\  & \                   & \mbox{(Assm)} \\
2. & \Gamma \;\; \varphi & \forall x \; x \circ z \equiv x \\
\  & \                   & \mbox{IV.3.6(d1) applied to 1.} \\
3. & \Gamma \;\; \varphi & y \circ z \equiv y \\
\  & \                   & \mbox{IV.5.5(a1) applied to 2. with $t = y$} \\
4. & \Gamma \;\; \varphi & \forall y \; y \circ z \equiv y \\
\  & \                   & \mbox{IV.5.5(b4) applied to 3.} \\
5. & \Gamma              & \forall x (e \equiv x \leftrightarrow \forall y \; y \circ x \equiv y) \\
\  & \                   & \mbox{(Assm)} \\
6. & \Gamma              & (e \equiv z \leftrightarrow \forall y \; y \circ z \equiv y) \\
\  & \                   & \mbox{IV.5.5(a1) applied to 5. with $t = z$} \\
7. & \Gamma \;\; \forall y \; y \circ z \equiv y & (e \equiv z \leftrightarrow \forall y \; y \circ z \equiv y) \\
\  & \                   & \mbox{(Ant) applied to 6.} \\
8. & \Gamma \;\; \forall y \; y \circ z \equiv y & \forall y \; y \circ z \equiv y \\
\  & \                   & \mbox{(Assm)} \\
9. & \Gamma \;\; \forall y \; y \circ z \equiv y & (e \equiv z \lor \forall y \; y \circ z \equiv y) \\
\  & \                   & \mbox{($\lor$S) applied to 8.} \\
10.& \Gamma \;\; \forall y \; y \circ z \equiv y & (e \equiv z \land \forall y \; y \circ z \equiv y) \\
\  & \                   & \mbox{IV.3.5 applied to 9. and 7.} \\
11.& \Gamma \;\; \forall y \; y \circ z \equiv y & e \equiv z \\
\  & \                   & \mbox{IV.3.6(d1) applied to 10.} \\
12.& \Gamma \;\; \varphi \;\; \forall y \; y \circ z \equiv y & e \equiv z \\
\  & \                   & \mbox{(Ant) applied to 11.} \\
13.& \Gamma \;\; \varphi & e \equiv z \\
\  & \                   & \mbox{(Ch) applied to 4. and 12.} \\
14.& \Gamma \;\; \varphi & z \equiv e \\
\  & \                   & \mbox{IV.5.3(a) applied to 13.} \\
15.& \Gamma \;\; \varphi & \forall x \exists y \; x \circ y \equiv z \\
\  & \                   & \mbox{IV.3.6(d2) applied to 1.} \\
16.& \Gamma \;\; \varphi \;\; z \equiv e & \forall x \exists y \; x \circ y \equiv e \\
\  & \                   & \mbox{(Sub) applied to 15. with $t = z$} \\
\  & \                   & \mbox{and $t^\prime = e$} \\
17.& \Gamma \;\; \varphi & \forall x \exists y \; x \circ y \equiv e \\
\  & \                   & \mbox{(Ch) applied to 14. and 16.} \\
18.& \Gamma \;\; \exists z \varphi & \forall x \exists y \; x \circ y \equiv e \\
\  & \                   & \mbox{IV.5.1(b) applied to 17.} \\
19.& \Gamma              & \forall x \exists y \; x \circ y \equiv e \\
\  & \                   & \mbox{(Ant) applied to 18.}
\end{array}
\]
\end{enumerate}
\end{enumerate}
%
%VIII.3.3----------------------------------------------------------------------------------------------------
\item \textbf{Solution to Exercise 3.3.} Let $S_0 \neq \emptyset$ be a set of new symbols not in $S$, $\Phi_0 := \{ \delta_s | s \in S_0 \}$ the set of extensions of definitions of symbols in $S_0$ in $\Phi$, and $I$ the associated syntactic interpretation of $S \cup S_0$ in $S$. ($I$ is defined accordingly.)\\
Then $\Phi_I$ is logically equivalent to $\{ \psi_s | s \in S_0 \}$, where
\[
\psi_s := \begin{cases}
\forall v_0 \ldots \forall v_{n-1} \exists^{=1}v_n \varphi_f(v_0, \ldots, v_{n-1}, v_n), & \begin{minipage}[t]{8em}if \(s\) is an \(n\)-ary function symbol \(f\);\end{minipage} \cr
\exists^{=1}v_0 \varphi_c(v_0), & \mbox{if \(s\) is a constant \(c\)}.
\end{cases}
\]
And we have for every $S$-structure $\mathfrak{A}$ with $\mathfrak{A} \models \Phi$:
\[
\begin{array}{ll}
\mbox{(+)} & \mathfrak{A} \models \Phi_I \\
\mbox{(++)}& \mbox{For all $s^A$: $(\mathfrak{A}, s^A) \models \delta_s$ iff $\mathfrak{A}^{-I}|_{S \cup \{ s \}} = (\mathfrak{A}, s^A)$.}
\end{array}
\]
\\
Now we are ready to prove:
\begin{enumerate}[(a)]
\item For all $\varphi \in L^S_0$:
\[
\mbox{$\Phi \cup \Phi_0 \models \varphi$ iff $\Phi \models \varphi$}.
\]
%%
\item For all $\chi \in L_0^{S \cup S_0}$:
\[
\Phi \cup \Phi_0 \models \chi \leftrightarrow \chi^I.
\]
%%
\item For all $\varphi \in L_0^{S \cup S_0}$:
\[
\mbox{$\Phi \cup \Phi_0 \models \varphi$ iff $\Phi \models \varphi^I$}.
\]
\end{enumerate}
\ 
\\
\textit{Proof.}
\begin{enumerate}[(a)]
\item For the non-trivial direction of the proof, let $\varphi \in L^S_0$ and $\mathfrak{A}$ be an $S$-structure with $\mathfrak{A} \models \Phi$. Then by (+) $\mathfrak{A}^{-I}$ is defined, let it be such that $\mathfrak{A}^{-I}|_{S \cup \{ s \}} = (\mathfrak{A}, s^A)$. By (++) it follows that $\mathfrak{A}^{-I} \models \Phi \cup \Phi_0$ since $\mathfrak{A}^{-I}|_{S \cup \{ s \}} \models \delta_s$ for every $s \in S_0$. Hence $\mathfrak{A}^{-I} \models \varphi$, and by the Coincidence Lemma we have $\mathfrak{A} \models \varphi$.
%%
\item Let $\chi \in L_0^{S \cup S_0}$ and let $\mathfrak{A}$ be an $(S \cup S_0)$-structure such that
\[
\mathfrak{A} \models \Phi \cup \Phi_0.
\]
Then for every $s \in S_0$, $\mathfrak{A}|_{S \cup \{ s \}} \models \delta_s$, and by (++) we have
\[
(\mathfrak{A}|_S)^{-I}|_{S \cup \{ s \}} = \mathfrak{A}|_{S \cup \{ s \}}.
\]
So $(\mathfrak{A}|_S)^{-I} = \mathfrak{A}$.
\\
By the Theorem on Syntactic Interpretations, the following holds for $\mathfrak{A}$ ($= (\mathfrak{A}|_S)^{-I}$):
\[
\begin{array}{lll}
\mathfrak{A} \models \chi & \mbox{iff} & \mathfrak{A}|_S \models \chi^I \\
\                         & \mbox{iff} & \mathfrak{A} \models \chi^I \mbox{ (by the Coincidence Lemma).}
\end{array}
\]
%%
\item Let $\varphi \in L_0^{S \cup S_0}$, then
\[
\begin{array}{lll}
\Phi \cup \Phi_0 \models \varphi & \mbox{iff} & \mbox{$\Phi \cup \Phi_0 \models \varphi^I$ (from (b))} \\
\                                & \mbox{iff} & \mbox{$\Phi \models \varphi^I$ (from (a))}.
\end{array}
\]
\end{enumerate}
%End of VIII.3.3---------------------------------------------------------------------------------------------
%
%VIII.3.4----------------------------------------------------------------------------------------------------
\item \textbf{Solution to Exercise 3.4.} Let $S$ be a symbol set, $s_1 \neq s_2$ two new symbols not in $S$, and $\Phi$ a set of $S$-sentences.
\\
Define $\delta_{s_1}$ to be the $S$-definition of $s_1$ in $\Phi$, $\delta_{s_2}$ the $(S \cup \{ s_1 \})$-definition of $s_2$ in $\Phi \cup \{ \delta_{s_1} \}$, and $\delta^\prime_{s_2}$ the $S$-definition of $s_2$ in $\Phi \cup \{ \delta_{s_1} \}$.
\ 
\\
\\
We formulate the statement in this exercise as follows:
\[
\modelclass{(S \cup \{ s_1 \}) \cup \{ s_2 \}}{((\Phi \cup \{ \delta_{s_1} \}) \cup \{ \delta_{s_2} \})} = \modelclass{(S \cup \{ s_1 \}) \cup \{ s_2 \}}{((\Phi \cup \{ \delta_{s_1} \}) \cup \{ \delta_{s_2}^\prime \})}.
\]
\\
\textit{Proof.} Let $I_0$ be the associated syntactic interpretation of $S \cup \{ s_1 \}$ in $S$. We extend $I_0$ to $I$ in such a way that
\[
I((S \cup \{ s_1 \}) \cup \{ s_2 \}) := v_0 \equiv v_0, \;\; I(s_2) := \mbox{the identity on $s_2$},
\]
and
\[
\mbox{$I(s) := I_0(s)$ for $s \in S \cup \{ s_1 \}$}.
\]
Define $\delta^\prime_{s_2}$ to be $\delta^I_{s_2}$.
\ 
\\
\\
First, by part (b) of Theorem on Definitions, it follows that
\[
(\Phi \cup \{ \delta_{s_1} \}) \cup \{ \delta_{s_2} \} \models \delta_{s_2} \leftrightarrow \delta^I_{s_2}.
\]
And obviously,
\[
(\Phi \cup \{ \delta_{s_1} \}) \cup \{ \delta_{s_2} \} \models \delta_{s_2}.
\]
Hence $(\Phi \cup \{ \delta_{s_1} \}) \cup \{ \delta_{s_2} \} \models \delta^I_{s_2}$ and
\[
\modelclass{(S \cup \{ s_1 \}) \cup \{ s_2 \}}{((\Phi \cup \{ \delta_{s_1} \}) \cup \{ \delta_{s_2} \})} \subset \modelclass{(S \cup \{ s_1 \}) \cup \{ s_2 \}}{((\Phi \cup \{ \delta_{s_1} \}) \cup \{ \delta_{s_2}^I \})}.
\]
\ 
\\
Next, by the same discusssion we have
\[
(\Phi \cup \{ \delta_{s_1} \}) \cup \{ \delta^I_{s_2} \} \models \delta_{s_2} \leftrightarrow \delta^I_{s_2}.
\]
and
\[
(\Phi \cup \{ \delta_{s_1} \}) \cup \{ \delta_{s_2} \} \models \delta^I_{s_2}.
\]
And we also have
\[
\modelclass{(S \cup \{ s_1 \}) \cup \{ s_2 \}}{((\Phi \cup \{ \delta_{s_1} \}) \cup \{ \delta^I_{s_2} \})} \subset \modelclass{(S \cup \{ s_1 \}) \cup \{ s_2 \}}{((\Phi \cup \{ \delta_{s_1} \}) \cup \{ \delta_{s_2} \})}.
\]
\ 
\\
From the above discussion, we obtain the result. \begin{flushright}$\talloblong$\end{flushright}
%End of VIII.3.4---------------------------------------------------------------------------------------------
%
%VIII.3.5----------------------------------------------------------------------------------------------------
\item \textbf{Solution to Exercise 3.5.} Assume the premise. Let $I$ be the associated syntactic interpretation of $S \cup \{ P \}$ in $S$, in which
\[
I(P) := \varphi_P(v_0, \ldots, v_{k - 1}).
\]
(The existence of such $\varphi_P$ is guaranteed by Beth's Definability Theorem. See hint.)
\\
\\
Define
\[
\Phi := \{ \chi^I | \chi \in \Phi^\prime \}
\]
and
\[
\delta_P := \forall v_0 \ldots \forall v_{k - 1}(Pv_0 \ldots v_{k - 1} \leftrightarrow \varphi_P(v_0, \ldots, v_{k - 1})).
\]
\\
In the following, we shall show that $\modelclass{S \cup \{ P \}}{}{(\Phi \cup \{ \delta_P \})} = \modelclass{S \cup \{ P \}}{}{\Phi^\prime}$.
\begin{enumerate}[1)]
\item $\modelclass{S \cup \{ P \}}{}{(\Phi \cup \{ \delta_P \})} \subset \modelclass{S \cup \{ P \}}{}{\Phi^\prime}$: By part (b) of the Theorem on Definitions,
\[
\mbox{$\Phi \cup \{ \delta_P \} \models \chi \leftrightarrow \chi^I$ for $\chi \in \Phi^\prime$}.
\]
And by definition,
\[
\mbox{$\Phi \cup \{ \delta_P \} \models \chi^I$ for $\chi \in \Phi^\prime$}.
\]
Therefore, $\Phi \cup \{ \delta_P \} \models \chi$ for $\chi \in \Phi^\prime$.
%%
\item $\modelclass{S \cup \{ P \}}{}{\Phi^\prime} \subset \modelclass{S \cup \{ P \}}{}{(\Phi \cup \{ \delta_P \})}$: We shall argue that $\Phi^\prime \models \chi \leftrightarrow \chi^\prime$ for all $\chi \in L_0^{S \cup \{ P \}}$, in particular, for all those $\chi \in \Phi^\prime$: First, $\delta_P$ is an $S$-definition of $P$ in $\emptyset$, since $\emptyset \subset L_0^S$. Next, let $I$ be the associated syntactic interpretation of $S \cup \{ P \}$ in $S$. Then the associated $\Phi_I$ is equivalent to $\emptyset$. (Note that $P$ is a relation symbol.)\\
\\
We have essentially for \textit{every} $S$-structure $\mathfrak{A}$ (i.e. $\mathfrak{A} \models \emptyset$):
\begin{itemize}
\item $\mathfrak{A} \models \Phi_I (= \emptyset)$,
%%
\item For all $s^A$: $(\mathfrak{A}, s^A) \models \delta_P$ iff $\mathfrak{A}^{-I} = (\mathfrak{A}, s^A)$.
\end{itemize}
\ 
\\
After arguing as in the proof of part (b) of the Theorem on Definitions, we obtain 
\[
\mbox{$\{ \delta_P \} (= \emptyset \cup \{ \delta_P \}) \models \chi \leftrightarrow \chi^I$ for all $\chi \in L_0^{S \cup \{ P \}}$}.
\]
And it is obvious that $\Phi^\prime \cup \{ \delta_P \} \models \chi \leftrightarrow \chi^I$ for all $\chi \in \Phi^\prime$, which implies that
\[
\mbox{$\Phi^\prime \models \chi \leftrightarrow \chi^I$ for all $\chi \in \Phi^\prime$},
\]
since $\Phi^\prime \models \delta_P$ by the premise.\\
\\
By definition, $\Phi^\prime \models \chi$ for all $\chi \in \Phi^\prime$. So
\[
\mbox{$\Phi^\prime \models \chi^I$ for all $\chi \in \Phi^\prime$}.
\]
Since $\Phi^\prime \models \delta_P$ also, it follows that
\[
\mbox{$\Phi^\prime \models \psi$ for all $\psi \in \Phi \cup \{ \delta_P \}$}.
\]
\end{enumerate} \begin{flushright}$\talloblong$\end{flushright}
%End of VIII.3.5---------------------------------------------------------------------------------------------
\end{enumerate}
%End of Section VIII.3---------------------------------------------------------------------------------------
\
\\
\\
%Section VIII.4----------------------------------------------------------------------------------------------
{\large \S4. Normal Forms}
\begin{enumerate}[1.]
\item \textbf{Note to $\langle \Phi \rangle$ on Page 128.} The formulas in $\langle \Phi \rangle$ are defined in a recursive way: For every $\varphi \in \langle \Phi \rangle$, either
\[
\varphi \in \Phi,
\]
or
\[
\mbox{$\varphi = \neg \psi$ for some $\psi \in \langle \Phi \rangle$},
\]
or
\[
\mbox{$\varphi = (\psi \lor \chi)$ for some $\psi, \chi \in \langle \Phi \rangle$}.
\]
%
\item \textbf{Note to Lemma 4.1.} The converse of this lemma is trivially true since $\Phi \subset \langle \Phi \rangle$.\\
\ \\
As for the proof given in text, notice the set of $\varphi$ for which ($*$) holds includes $\Phi$ by the premise. And it obviously includes $\neg \psi$ and $(\psi \lor \chi)$ with $\psi$ and $\chi$ being its elements. Thus, this set must includes $\langle \Phi \rangle$ since by definition, $\langle \Phi \rangle$ is defined to be the smallest such set of formulas $\subset L^S_r$.
%
\item \textbf{Note to Lemma 4.2.} First note that, here and later on at the Theorem on the Disjuctive Normal Form and at Exercise 4.7, we speak of disjuctions and conjunctions in a somewhat general sense: A disjunction (or conjunction) may have only one disjuct (or conjuct, respectively).\\
\ \\
On the other hand, since all formulas in $\langle \Phi \rangle$ have the form presented in the lemma, they must be finite. On the other hand, all unsatisfiable formulas are logically equivalent to, say, $\neg v_0 \equiv v_0$. Therefore, there are finitely many pairwise logically nonequivalent formulas in $\langle \Phi \rangle$.\\
\ \\
As for the proof given in text, note the following:
\begin{enumerate}[(1)]
\item There might be unsatisfiable such formulas of the form $\psi_0 \land \ldots \land \psi_n$, hence the set
\[
\{ \psi_{(\mathfrak{A}, \stackrel{r}{a})} | \mbox{$\mathfrak{A}$ is an $S$-structure and $\stackrel{r}{a} \in A^r$} \}
\]
has ``at most'' instead of ``exactly'' $2^{n + 1}$ elements.
%%
\item For $\varphi \in \langle \Phi \rangle$, the set
\[
\{ \psi_{(\mathfrak{A}, \stackrel{r}{a})} | \mbox{$\mathfrak{A}$ is an $S$-structure and $\stackrel{r}{a} \in A^r$, $\mathfrak{A} \models \varphi[a_0, \ldots, a_{r - 1}]$} \}
\]
is a subset of
\[
\{ \psi_{(\mathfrak{A}, \stackrel{r}{a})} | \mbox{$\mathfrak{A}$ is an $S$-structure and $\stackrel{r}{a} \in A^r$} \}.
\]
\end{enumerate}
%
\item \textbf{Note to the Properties of Logical Equivalence Mentioned in the Proof of the Theorem on the Prenex Normal Form.} By the definition of logical equivalence, $\varphi \sim \psi$ is equivalent to say $\varphi \bimodels \psi$, which in turn is equivalent to say $\varphi \vdash \psi$ and $\psi \vdash \varphi$ (by the Adequacy Theorem V.4.2).\\
\\
We shall give proofs (more precisely, derivations) for each each of those properties:
\begin{enumerate}[(1)]
\item If $\varphi \sim \psi$, then $\neg \psi \sim \neg \varphi$.
\[
\begin{array}{llll}
1. & \varphi & \psi & \mbox{premise} \\
2. & \psi & \varphi & \mbox{premise} \\
3. & \neg \varphi & \neg \psi & \mbox{IV.3.3(a) applied to 2.} \\
4. & \neg \psi & \neg \varphi & \mbox{IV.3.3(a) applied to 1.}
\end{array}
\]
%%
\item If $\varphi_0 \sim \psi_0$ and $\varphi_1 \sim \psi_1$, then $(\varphi_0 \lor \varphi_1) \sim (\psi_0 \lor \psi_1)$.
\[
\begin{array}{llll}
1. & \varphi_0 & \psi_0 & \mbox{premise} \\
2. & \psi_0 & \varphi_0 & \mbox{premise} \\
3. & \varphi_1 & \psi_1 & \mbox{premise} \\
4. & \psi_1 & \varphi_1 & \mbox{premise} \\
5. & \varphi_0 & (\psi_0 \lor \psi_1) & \mbox{($\lor$S) applied to 1.} \\
6. & \varphi_1 & (\psi_0 \lor \psi_1) & \mbox{($\lor$S) applied to 3.} \\
7. & \psi_0 & (\varphi_0 \lor \varphi_1) & \mbox{($\lor$S) applied to 2.} \\
8. & \psi_1 & (\varphi_0 \lor \varphi_1) & \mbox{($\lor$S) applied to 4.} \\
9. & (\varphi_0 \lor \varphi_1) & (\psi_0 \lor \psi_1) & \mbox{($\lor$A) applied to 5. and 6.} \\
10. & (\psi_0 \lor \psi_1) & (\varphi_0 \lor \varphi_1) & \mbox{($\lor$A) applied to 7. and 8.}
\end{array}
\]
%%
\item
\begin{itemize}
\item If $\varphi \sim \psi$, then $\exists x \varphi \sim \exists x \psi$.
\[
\begin{array}{llll}
1. & \varphi & \psi & \mbox{premise} \\
2. & \psi & \varphi & \mbox{premise} \\
3. & \exists x \varphi & \exists x \psi & \mbox{Exercise IV.4.5 applied to 1.} \\
4. & \exists x \psi & \exists x \varphi & \mbox{Exercise IV.4.5 applied to 2.}
\end{array}
\]
%%%
\item If $\varphi \sim \psi$, then $\forall x \varphi \sim \forall x \psi$.
\[
\begin{array}{llll}
1. & \varphi & \psi & \mbox{premise} \\
2. & \psi & \varphi & \mbox{premise} \\
3. & \neg \psi & \neg \varphi & \mbox{IV.3.3(a) applied to 1.} \\
4. & \neg \varphi & \neg \psi & \mbox{Iv.3.3(a) applied to 2.} \\
5. & \exists x \neg \psi & \exists x \neg \varphi & \mbox{Exercise IV.4.5 applied to 3.} \\
6. & \exists x \neg \varphi & \exists x \neg \psi & \mbox{Exercise IV.4.5 applied to 4.} \\
7. & \forall x \varphi & \forall x \psi & \mbox{IV.3.3(a) applied to 5.} \\
8. & \forall x \psi & \forall x \varphi & \mbox{IV.3.3(a) applied to 6.}
\end{array}
\]
\end{itemize}
%%
\item
\begin{itemize}
\item $\neg \exists x \varphi \sim \forall x \neg \varphi$.
\[
\begin{array}{llll}
1. & \neg \neg \varphi & \neg \neg \varphi & \mbox{(Assm)} \\
2. & \neg \neg \varphi & \varphi & \mbox{IV.3.6(a2) applied to 1.} \\
3. & \exists x \neg \neg \varphi & \exists x \varphi & \mbox{Exercise IV.4.5 applied to 2.} \\
4. & \varphi & \varphi & \mbox{(Assm)} \\
5. & \varphi & \neg \neg \varphi & \mbox{IV.3.6(a1) applied to 4.} \\
6. & \exists x \varphi & \exists x \neg \neg \varphi & \mbox{Exercise IV.4.5 applied to 5.} \\
7. & \neg \exists x \varphi & \forall x \neg \varphi & \mbox{IV.3.3(a) applied to 3.} \\
8. & \forall x \neg \varphi & \neg \exists x \varphi & \mbox{IV.3.3(a) applied to 6.}
\end{array}
\]
%%%
\item $\neg \forall x \varphi \sim \exists x \neg \varphi$.
\[
\begin{array}{llll}
1. & \forall x \varphi & \forall x \varphi & \mbox{(Assm)} \\
2. & \neg \forall x \varphi & \exists x \neg \varphi & \mbox{IV.3.3(c) applied to 1.} \\
3. & \exists x \neg \varphi & \neg \forall x \varphi & \mbox{IV.3.3(d) applied to 1.}
\end{array}
\]
\end{itemize}
%%
\item Suppose $x \not \in \free(\psi)$.
\begin{itemize}
\item $(\exists x \varphi \lor \psi) \sim \exists x (\varphi \lor \psi)$.
\[
\begin{array}{llll}
1. & \varphi & \varphi & \mbox{(Assm)} \\
2. & \varphi & (\varphi \lor \psi) & \mbox{($\lor$S) applied to 1.} \\
3. & \exists x \varphi & \exists x (\varphi \lor \psi) & \mbox{Exercise IV.4.5 applied to 2.} \\
4. & \psi & \psi & \mbox{(Assm)} \\
5. & \psi & (\varphi \lor \psi) & \mbox{($\lor$S) applied to 4.} \\
6. & \psi & \exists x (\varphi \lor \psi) & \mbox{IV.5.1(a) applied to 5.} \\
7. & \varphi & \exists x \varphi & \mbox{IV.5.1(a) applied to 1.} \\
8. & \varphi & (\exists x \varphi \lor \psi) & \mbox{($\lor$S) applied to 7.} \\
9. & \psi & (\exists x \varphi \lor \psi) & \mbox{($\lor$S) applied to 4.} \\
10. & (\varphi \lor \psi) & (\exists x \varphi \lor \psi) & \mbox{($\lor$A) applied to 8. and 9.} \\
11. & (\exists x \varphi \lor \psi) & \exists x (\varphi \lor \psi) & \mbox{($\lor$A) applied to 3. and 6.} \\
12. & \exists x (\varphi \lor \psi) & (\exists x \varphi \lor \psi) & \mbox{IV.5.1(b) applied to 10.}
\end{array}
\]
%%%
\item $(\forall x \varphi \lor \psi) \sim \forall x (\varphi \lor \psi)$.
\[
\begin{array}{llll}
1. & \forall x \varphi & \forall x \varphi & \mbox{(Assm)} \\
2. & \forall x \varphi & \varphi & \mbox{IV.5.5(a2) applied to 1.} \\
3. & \forall x \varphi & (\varphi \lor \psi) & \mbox{($\lor$S) applied to 2.} \\
4. & \forall x \varphi & \forall x (\varphi \lor \psi) & \mbox{IV.5.5(b4) applied to 3.} \\
5. & \psi & \psi & \mbox{(Assm)} \\
6. & \psi & (\varphi \lor \psi) & \mbox{($\lor$S) applied to 5.} \\
7. & \psi & \forall x (\varphi \lor \psi) & \mbox{IV.5.5(b4) applied to 6.} \\
8. & \forall x (\varphi \lor \psi) \;\; \psi & \psi & \mbox{(Assm)} \\
9. & \forall x (\varphi \lor \psi) \;\; \psi & (\forall x \varphi \lor \psi) & \mbox{($\lor$S) applied to 8.} \\
10. & (\varphi \lor \psi) & (\varphi \lor \psi) & \mbox{(Assm)} \\
11. & \forall x (\varphi \lor \psi) & (\varphi \lor \psi) & \mbox{IV.5.5(b3) applied to 10.}\\
12. & \forall x (\varphi \lor \psi) \;\; \neg \varphi & (\varphi \lor \psi) & \mbox{(Ant) applied to 11.} \\
13. & \forall x (\varphi \lor \psi) \;\; \neg \varphi & \neg \varphi & \mbox{(Assm)} \\
14. & \forall x (\varphi \lor \psi) \;\; \neg \varphi & \psi & \mbox{IV.3.4 applied to 12. and 13.} \\
15. & \forall x (\varphi \lor \psi) \;\; \neg \psi & \varphi & \mbox{IV.3.3(c) applied to 14.} \\
16. & \forall x (\varphi \lor \psi) \;\; \neg \psi & \forall x \varphi & \mbox{IV.5.5(b4) applied to 15.} \\
17. & \forall x (\varphi \lor \psi) \;\; \neg \psi & (\forall x \varphi \lor \psi) & \mbox{($\lor$S) applied to 16.} \\
18. & (\forall x \varphi \lor \psi) & \forall x (\varphi \lor \psi) & \mbox{($\lor$A) applied to 4. and 7.} \\
19. & \forall x (\varphi \lor \psi) & (\forall x \varphi \lor \psi) & \mbox{(PC) applied to 9. and 17.}
\end{array}
\]
%%%
\item $(\psi \lor \exists x \varphi) \sim \exists x (\varphi \lor \psi)$.
\[
\begin{array}{llll}
1. & \psi & \psi & \mbox{(Assm)} \\
2. & \psi & (\varphi \lor \psi) & \mbox{($\lor$S) applied to 1.} \\
3. & \psi & \exists x (\varphi \lor \psi) & \mbox{IV.5.1(a) applied to 2.} \\
4. & \varphi & \varphi & \mbox{(Assm)} \\
5. & \varphi & (\varphi \lor \psi) & \mbox{($\lor$S) applied to 4.} \\
6. & \exists x \varphi & \exists x (\varphi \lor \psi) & \mbox{Exercise IV.4.5 applied to 5.} \\
7. & \varphi & \exists x \varphi & \mbox{IV.5.1(a) applied to 4.} \\
8. & \varphi & (\psi \lor \exists x \varphi) & \mbox{($\lor$S) applied to 7.} \\
9. & \psi & (\psi \lor \exists x \varphi) & \mbox{($\lor$S) applied to 1.} \\
10. & (\varphi \lor \psi) & (\psi \lor \exists x \varphi) & \mbox{($\lor$A) applied to 8. and 9.} \\
11. & (\psi \lor \exists x \varphi) & \exists x (\varphi \lor \psi) & \mbox{($\lor$A) applied to 3. and 6.} \\
12. & \exists x (\varphi \lor \psi) & (\psi \lor \exists x \varphi) & \mbox{IV.5.1(b) applied to 10.}
\end{array}
\]
%%%
\item $(\psi \lor \forall x \varphi) \sim \forall x (\varphi \lor \psi)$.
\[
\begin{array}{llll}
1. & \psi & \psi & \mbox{(Assm)} \\
2. & \psi & (\varphi \lor \psi) & \mbox{($\lor$S) applied to 2.} \\
3. & \psi & \forall x (\varphi \lor \psi) & \mbox{IV.5.5(b4) applied to 3.} \\
4. & \forall x \varphi & \forall x \varphi & \mbox{(Assm)} \\
5. & \forall x \varphi & \varphi & \mbox{IV.5.5(a2) applied to 4.} \\
6. & \forall x \varphi & (\varphi \lor \psi) & \mbox{($\lor$S) applied to 5.} \\
7. & \forall x \varphi & \forall x (\varphi \lor \psi) & \mbox{IV.5.5(b4) applied to 6.} \\
8. & \forall x (\varphi \lor \psi) \;\; \psi & \psi & \mbox{(Assm)} \\
9. & \forall x (\varphi \lor \psi) \;\; \psi & (\psi \lor \forall x \varphi) & \mbox{($\lor$S) applied to 8.} \\
10. & (\varphi \lor \psi) & (\varphi \lor \psi) & \mbox{(Assm)} \\
11. & \forall x (\varphi \lor \psi) & (\varphi \lor \psi) & \mbox{IV.5.5(b3) applied to 10.}\\
12. & \forall x (\varphi \lor \psi) \;\; \neg \varphi & (\varphi \lor \psi) & \mbox{(Ant) applied to 11.} \\
13. & \forall x (\varphi \lor \psi) \;\; \neg \varphi & \neg \varphi & \mbox{(Assm)} \\
14. & \forall x (\varphi \lor \psi) \;\; \neg \varphi & \psi & \mbox{IV.3.4 applied to 12. and 13.} \\
15. & \forall x (\varphi \lor \psi) \;\; \neg \psi & \varphi & \mbox{IV.3.3(c) applied to 14.} \\
16. & \forall x (\varphi \lor \psi) \;\; \neg \psi & \forall x \varphi & \mbox{IV.5.5(b4) applied to 15.} \\
17. & \forall x (\varphi \lor \psi) \;\; \neg \psi & (\psi \lor \forall x \varphi) & \mbox{($\lor$S) applied to 16.} \\
18. & (\psi \lor \forall x \varphi) & \forall x (\varphi \lor \psi) & \mbox{($\lor$A) applied to 3. and 7.} \\
19. & \forall x (\varphi \lor \psi) & (\psi \lor \forall x \varphi) & \mbox{(PC) applied to 9. and 17.}
\end{array}
\]
\end{itemize}
(For a proof of semantical flavor, cf. Exercise III.4.11.)
\end{enumerate}
%
\item \textbf{More on Logical Equivalence.} Note that, by definition, $\varphi \sim \varphi$ for all $\varphi$. Moreover, $\varphi \sim \neg \neg \varphi$:
\[
\begin{array}{llll}
1. & \varphi & \varphi & \mbox{(Assm)} \\
2. & \neg \neg \varphi & \neg \neg \varphi & \mbox{(Assm)} \\
3. & \varphi & \neg \neg \varphi & \mbox{IV.3.6(a1) applied to 1.} \\
4. & \neg \neg \varphi & \varphi & \mbox{IV.3.6(a2) applied to 2.}
\end{array}
\]
\\
On the other hand, if $y \not \in \free(\varphi)$, then $\forall x \varphi \sim \forall y \varphi \frac{y}{x}$. The following is a derivation for it:
\[
\begin{array}{llll}
1. & \forall x \varphi & \forall x \varphi & \mbox{(Assm)} \\
2. & \forall x \varphi & \varphi\frac{y}{x} & \mbox{IV.5.5(a1) applied to 1. with $t = y$} \\
3. & \forall y \varphi\frac{y}{x} & \forall y \varphi\frac{y}{x} & \mbox{(Assm)} \\
4. & \forall y \varphi\frac{y}{x} & \varphi\frac{y}{x} & \mbox{IV.5.5(a2) applied to 3.} \\
5. & \forall x \varphi & \forall y \varphi\frac{y}{x} & \mbox{IV.5.5(b2) applied to 2.} \\
6. & \forall y \varphi\frac{y}{x} & \forall x \varphi & \mbox{IV.5.5(b2) applied to 4.}
\end{array}
\]
Note that if we drop the premise, then the statement above does not hold in general, as illustrated by the following counterexample:
\[
\forall x \neg Ryy \not \sim \forall y \neg Ryy.
\]
\\
Also, that $\exists x \varphi \sim \exists y \varphi \frac{y}{x}$ for $y \not \in \free(\varphi)$ follows from the above argument: First, we have $\forall x \neg \varphi \sim \forall y \neg \varphi \frac{y}{x}$. Next, since by property (4), $\forall x \neg \varphi \sim \neg \exists x \varphi$ and $\forall y \neg \varphi\frac{y}{x} \sim \neg \exists y \varphi\frac{y}{x}$, it immediately follows that $\neg \exists x \varphi \sim \neg \exists y \varphi\frac{y}{x}$. This in turn implies that $\neg \neg \exists x \varphi \sim \neg \neg \exists y \varphi\frac{y}{x}$ (by property (1)). And finally, we arrive at $\exists x \varphi \sim \exists y \varphi\frac{y}{x}$ because $\neg \neg \exists x \varphi \sim \exists x \varphi$ and $\neg \neg \exists y \varphi\frac{y}{x} \sim \exists y \varphi\frac{y}{x}$ (by the previous argument).
%
\item \textbf{Note to the Main Part of the Proof of the Theorem on the Prenex Normal Form.} The induction utilized in the proof actually proceeds in two directions, one in the quantifier number, and the other the structure of a formula (encapsulated in the inductive step of the former). In fact, the base case for the direction of the latter consists of atomic formulas, which is included in both base case ($n = 0$) and inductive step ($n > 0$) in the direction of quantifier number.\\
\\
There is a subtlety here: According to the original statement of $(*)_n$, our goal is to prove such a ``$\psi$'' exists. However, in text this symbol is misused in the first two cases in the inductive step (see page 131), i.e. it stands for only part of the objective formula. Hence, it is reasonable to replace those $\psi$'s by an unused symbol (say, $\psi^\prime$, which is perfectly suitable here), and let $\psi$ still represent the objective formula. (We shall adopt this minor change in the argument below.)\\
\\
We complement the proof by appending the arguments ``$\free(\varphi) = \free(\psi)$'' to each part:
\begin{enumerate}[(1)]
\item Base case, $n = 0$: Since $\psi := \varphi$, obviously $\free(\varphi) = \free(\psi)$.
%%
\item Inductive step, $n > 0$:
\begin{enumerate}[(i)]
\item $\varphi = \neg \varphi^\prime$:
\[
\begin{array}{llll}
\free(\varphi) & = & \free(\psi) & \mbox{(by induction hypothesis, in the} \\
\              & \ & \           & \mbox{direction of formula structure)} \\
\              & = & \free(Qx \chi) & \  \\
\              & = & \free(\chi) \setminus \{ x \} & \  \\
\              & = & \free(\psi^\prime) \setminus \{ x \} & \mbox{(by induction hypothesis, in the} \\
\              & \ & \           & \mbox{direction of quantifier number)} \\
\              & = & \free(Q^{-1}x \psi^\prime) \\
\              & = & \free(\psi). & \ 
\end{array}
\]
%%%
\item $\varphi = (\varphi^\prime \lor \varphi^{\prime\prime})$:
\[
\begin{array}{llll}
\free(\varphi) & = & \free(\varphi^\prime) \cup \free(\varphi^{\prime\prime}) \\
\              & = & \free(Qx \chi) \cup \free(\varphi^{\prime\prime}) & \mbox{(by induction hypothesis,}\\
\              & \ & \                                                 & \mbox{in the direction of} \\
\              & \ & \                                                 & \mbox{formula structure)} \\
\              & = & \free(Qy \chi\frac{y}{x}) \cup \free(\varphi^{\prime\prime}) & \mbox{(The reader is encouraged} \\
\              & \ & \ & \mbox{to verify this)} \\
\              & = & \free((\chi\frac{y}{x} \lor \varphi^{\prime\prime})) \setminus \{ y \} & \mbox{(since $y$ does not occur} \\
\              & \ & \ & \mbox{in $\varphi^{\prime\prime}$)} \\
\              & = & \free(\psi^\prime) \setminus \{ y \} & \mbox{(by induction hypothesis,} \\
\              & \ & \ & \mbox{in the direction of} \\
\              & \ & \ & \mbox{quantifier number)} \\
\              & = & \free(Qy \psi^\prime) & \ \\
\              & = & \free(\psi). & \ 
\end{array}
\]
%%%
\item $\varphi = \exists x \varphi^\prime$:
\[
\begin{array}{llll}
\free(\varphi) & = & \free(\exists x \varphi^\prime) & \ \\
\              & = & \free(\varphi^\prime) \setminus \{ x \} & \ \\
\              & = & \free(\psi^\prime) \setminus \{ x \} & \mbox{(by induction hypothesis, in the} \\
\              & \ & \ & \mbox{direction of quantifier number)} \\
\              & = & \free(\exists x \psi^\prime) & \ \\
\              & = & \free(\psi). & \ 
\end{array}
\]
\end{enumerate}
\end{enumerate}
\item \textbf{Note to Equivalence for Satisfaction Mentioned in Page 131.} Let $\varphi$ and $\psi$ be two $S$-formulas such that $\psi$ is independent of $\varphi$, i.e.
\[
\mbox{neither $\varphi \models \psi$ nor $\varphi \models \neg \psi$}.
\]
Then $\varphi$ and $(\varphi \land \psi$) are obviously equivalent for satisfaction. (But note that clearly they are \emph{not} logically equivalent.)\\
\\
Also, if two formulas are logically equivalent, then they are equivalent for satisfaction by definition. But the converse is generally not true. (As the above argument provides a counterexample.)
%
\item \textbf{Note to the Paragraph in Page 131 before the Theorem on the Skolem Normal Form.} Note that $\psi \models \varphi$ implies that
\[
\mbox{If $\sat \psi$, then $\sat \varphi$}.
\]
%
\item \textbf{Note to the Proof of the Theorem on the Skolem Normal Form.}
\begin{enumerate}[(i)]
\item For the formula $\psi^\prime$ discussed in this proof,
\[
\begin{array}{lll}
\free(\psi^\prime) & = & \free(\varphi_1) \setminus \{x_i | 1 \leq i \leq k + 1 \} \\
\                  & = & \free(\varphi_0) \setminus \{x_i | 1 \leq i \leq m \} \\
\                  & = & \free(\varphi).
\end{array}
\]
%%
\item In property (2), again, $\psi^\prime \models \varphi$ implies that
\[
\mbox{``if $\sat \psi^\prime$, then $\sat \varphi$}.
\]
This, together with property (1), implies that $\varphi$ and $\psi^\prime$ are equivalent for satisfaction.
%%
\item In property (3), note that $\free(\psi^\prime)$ remains the same (i.e. an \emph{invariant}) after each induction step.
%%
\item In property (4), the previous $\psi^\prime$ is a consequence of the new $\psi^\prime$ generated in each step, i.e.
\[
\mbox{new } \psi^\prime \models \mbox{previous } \psi^\prime.
\]
%%
\item In the part of the proof for property (1), note that $\mathfrak{I} = (\mathfrak{A}, \beta) = ((\mathfrak{A}, f^A), \beta) |_S$, so it follows that for all $a_1, \ldots a_k \in A$:
\[
\mathfrak{I}\displaystyle\frac{a_1 \ldots a_k f^A(a_1, \ldots, a_k)}{x_1 \ldots x_k \phantom{f^Aa_1} x_{k+1} \phantom{,a_k)}} \models \varphi_1
\]
iff
\[
((\mathfrak{A}, f^A), \beta)\displaystyle\frac{a_1 \ldots a_k f^A(a_1, \ldots, a_k)}{x_1 \ldots x_k \phantom{f^Aa_1} x_{k+1} \phantom{,a_k)}} \models \varphi_1
\]
(cf. the argument after Definition III.4.7 in page 38). And further, for all $a_1, \ldots, a_k \in A$,
\[
((\mathfrak{A}, f^A), \beta)\displaystyle\frac{a_1 \ldots a_k f^A(a_1, \ldots, a_k)}{x_1 \ldots x_k \phantom{f^Aa_1} x_{k+1} \phantom{,a_k)}} \models \varphi_1
\]
holds, since we already have, for all $a_1, \ldots, a_k \in A$,
\[
\mathfrak{I}\displaystyle\frac{a_1 \ldots a_k f^A(a_1, \ldots, a_k)}{x_1 \ldots x_k \phantom{f^Aa_1} x_{k+1} \phantom{,a_k)}} \models \varphi_1.
\]
Then it is straightforward to apply the Substitution Lemma to obtain the result.
%%
\item In the part of the proof for property (2), we provide in the following an alternative for it, i.e. a derivation of $\psi^\prime \vdash \varphi$ (hence $\psi^\prime \models \varphi$):
\[
\begin{array}{lll}
1. & \varphi_1\frac{fx_1 \ldots x_k}{x_{k + 1}} & \varphi_1\frac{fx_1 \ldots x_k}{x_{k + 1}} \\
\  & \ & \mbox{(Assm)} \\
2. & \varphi_1\frac{fx_1 \ldots x_k}{x_{k + 1}} & \exists x_{k + 1} \varphi_1 \\
\  & \ & \mbox{($\exists$S) applied to 1.} \\
3. & \forall x_k \varphi_1\frac{fx_1 \ldots x_k}{x_{k + 1}} & \exists x_{k + 1} \varphi_1 \\
\  & \ & \mbox{IV.5.5(b3) applied to 2.} \\
4. & \forall x_k \varphi_1\frac{fx_1 \ldots x_k}{x_{k + 1}} & \forall x_k \exists x_{k + 1} \varphi_1 \\
\  & \ & \mbox{IV.5.5(b4) applied to 3.} \\
\multicolumn{3}{c}{\vdots} \\
(2k + 1). & \forall x_1 \ldots \forall x_k \varphi_1\frac{fx_1 \ldots x_k}{x_{k + 1}} & \forall x_2 \ldots \forall x_k \exists x_{k + 1} \varphi_1 \\
\  & \ & \mbox{IV.5.5(b3) applied to ($2k$).} \\
(2k + 2). & \forall x_1 \ldots \forall x_k \varphi_1\frac{fx_1 \ldots x_k}{x_{k + 1}} & \forall x_1 \ldots \forall x_k \exists x_{k + 1} \varphi_1 \\
\  & \ & \mbox{IV.5.5(b4) applied to ($2k + 1$).} \\
\end{array}
\]
\end{enumerate}
%
%VIII.4.6------------------------------------------------------------------------------------------
\item \textbf{Solution to Exercise 4.6.} Note that this is a special case of the proof of the Theorem on the Skolem Normal Form, for which all we have to do to complete this exercise is to modify the proof by dropping (and hence disregarding) the assignment $\beta$.\nolinebreak\hfill$\talloblong$
%End of VIII.4.6-----------------------------------------------------------------------------------
%
%VIII.4.7------------------------------------------------------------------------------------------
\item \textbf{Solution to Exercise 4.7.} By Exercise IV.3.6(a1) and (a2) together with the Correctness Theorem, every formula $\chi$ is logically equivalent to its double negation $\neg\neg\chi$. More generally, using III.4.12(b), for every formula $\delta$ we can replace in it all double negations $\neg\neg\chi$ by the equivalent formulas $\chi$ to obtain a formula $\delta^\prime$ logically equivalent to $\delta$, and vice versa. Below we shall use these two facts to prove the Theorem on the Conjunctive Normal Form.\\
\ \\
Let $\varphi$ be a quantifier-free formula, here we describe a procedure to get a formula $\varphi^\prime$ in conjunctive normal form that is logically equivalent to $\varphi$:
\begin{enumerate}[(1)]
\item Take its negation $\neg\varphi$, and then apply the Theorem on the Disjunctive Normal Form to get an equivalent formula $\psi$ in disjunctive normal form.
%%
\item Replace in $\psi$ each disjunct that is an atomic formula $\chi$ by its equivalent double negation $\neg\neg\chi$ to get an equivalent formula $\psi^\prime$.
%%
\item The formula $\neg\psi^\prime$ is thus a conjunction\footnote{We take $\land$ as an abbreviation, cf. the discussion at the bottom of page 35.} of disjunctions of atomic formulas or single- or double-negated atomic formulas. Note that $\neg\psi^\prime$ is logically equivalent to $\varphi$. Replace in $\neg\psi^\prime$ all double-negated atomic formulas $\neg\neg\chi$ in disjunctions by their equivalent atomic formulas $\chi$ to obtain a formula $\varphi^\prime$ in conjunctive normal form. $\varphi^\prime$ is logically equivalent to $\neg\psi^\prime$ and hence to $\varphi$.
\end{enumerate}
\ \\
Here we illustrate the procedure by an example. Let $S \colonequals \{ P \}$ with unary relation symbol $P$, and let $\varphi \colonequals ((Pv_0 \rightarrow v_1 \equiv v_2) \land \neg v_0 \equiv v_1 \land Pv_2)$.\\
\ \\
In step (1), we apply the Theorem on the Disjunctive Normal Form to $\neg\varphi$ to obtain $\psi \colonequals ((Pv_0 \land \neg v_1 \equiv v_2) \lor v_0 \equiv v_1 \lor \neg Pv_2)$.\\
\ \\
Then step (2) yields $\psi^\prime \colonequals ((Pv_0 \land \neg v_1 \equiv v_2) \lor \neg\neg v_0 \equiv v_1 \lor \neg Pv_2)$ for $\psi$. Note that $\psi^\prime = (\neg (\neg Pv_0 \lor \neg\neg v_1 \equiv v_2 ) \lor \neg\neg v_0 \equiv v_1 \lor \neg Pv_2)$.\\
\ \\
And finally in step (3), we have $\neg\psi^\prime = \neg (\neg (\neg Pv_0 \lor \neg\neg v_1 \equiv v_2 ) \lor \neg\neg v_0 \equiv v_1 \lor \neg Pv_2) = ((\neg Pv_0 \lor \neg\neg v_1 \equiv v_2) \land \neg v_0 \equiv v_1 \land Pv_2)$, since we regard $\land$ as an abbreviation mentiond  at the bottom of page 35. By replacing in $\neg\psi^\prime$ the double negation $\neg\neg v_1 \equiv v_2$ by the equivalent $v_1 \equiv v_2$, we get $\varphi^\prime \colonequals ((\neg Pv_0 \lor v_1 \equiv v_2) \land \neg v_0 \equiv v_1 \land Pv_2)$, a formula in conjunctive normal form that is logically equivalent to $\varphi$.
%End of VIII.4.7-----------------------------------------------------------------------------------
%
%VIII.4.8------------------------------------------------------------------------------------------
\item \textbf{Solution to Exercise 4.8.} Let $\mathfrak{A}$ be an $S$-structure such that $\mathfrak{A} \models \varphi$. By Exercise 4.6, there is an $S^\prime$-expansion $\mathfrak{A}^\prime$ of $\mathfrak{A}$ such that $\mathfrak{A}^\prime \models \varphi^\prime$, where
\[
\varphi^\prime \colonequals \forall y_0 \ldots \forall y_m \psi\sbst{c_0 \ldots c_n}{x_0 \ldots x_n} \in L_0^{S^\prime}
\]
is universal.\\
\\
Let $\mathfrak{A}_0^\prime \subset \mathfrak{A}^\prime$ with $A_0^\prime = \{ c_i^{\mathfrak{A}^\prime} | 0 \leq i \leq n \}$. Note that
\[
| A_0^\prime | = | \{ c_i^{\mathfrak{A}^\prime} | 0 \leq i \leq n \} | \leq \sum_{0 \leq i \leq n} | \{ c_i^{\mathfrak{A}^\prime} \} | = n + 1,
\]
and $\mathfrak{A}_0^\prime \models \varphi^\prime$ (cf. Corollary III.5.8).\\
\\
Since $\varphi^\prime \models \varphi$ by the Theorem on the Skolem Normal Form, we have that $\mathfrak{A}_0^\prime \models \varphi$. By the discussion after Definition III.4.7 on page 38, it follows that $\mathfrak{A}_0^\prime |_S \models \varphi$. Also note that $\mathfrak{A}_0^\prime |_S \subset \mathfrak{A}$, thus $\mathfrak{A}_0^\prime |_S$ meets all the requirements.\\
\\
On the other hand, consider $S = \{ < \}$. Then $(\mathbb{N}, <^\mathbb{N}) \models \forall x \exists y \; x < y$. However, $(\mathbb{N}, <^\mathbb{N})$ contains no \emph{finite} substructure which is also a model of $\forall x \exists y \; x < y$, as in such a substructure, there is no element ``greater'' than the maximum element. Thus, in general, the sentence $\forall x \exists y Rxy$ cannot be logically equivalent to a sentence of the same shape as $\varphi$. \nolinebreak\hfill$\talloblong$\\
\ \\
\textit{Remark.} There is a typo in the formula
\[
\exists x_0 \ldots \exists x_n \forall y_0 \ldots y_m \psi
\]
in line 2 should be replaced by
\[
\exists x_0 \ldots \exists x_n \forall y_0 \ldots \forall y_m \psi,
\]
i.e. the `$\forall$' proceeding $y_m$ is missing.
%End of VIII.4.8-----------------------------------------------------------------------------------
\end{enumerate}
%End of Section VIII.4---------------------------------------------------------------------------------------
%End of Chapter VIII
%%Chapter IX----------------------------------------------------------------------------------------
{\LARGE \bfseries IX \\ \\ Extensions of First-Order Logic}
\\
\\
\\
%Section IX.1--------------------------------------------------------------------------------------
{\large \S1. Second-Order Logic}
\begin{enumerate}[1.]
\item \textbf{Note to Second-Order Logic $\SOL$.} $\SOL$ may be characterized by the following ``equation'':
\[
\SOL \colonequals \bigcup\limits_S \LII^S + (\mbox{satisfaction relation for $\bigcup\limits_S \LII^S$}).
\]
\ \\
Let $S$ be given. To be compatible with $\FOL$, we shall call $S$-formulas of the form
\[
Xt_1 \ldots t_n
\]
\emph{atomic}, where $X$ is an $n$-ary relation variable and $t_1$, \ldots, $t_n$ are $S$-terms.\\
\ \\
When it comes to the satisfaction relation ``$\models$'', on the other hand, we can also dispense with the connectives $\land$, $\rightarrow$, $\leftrightarrow$ and the quantifier $\forall$ in second-order languages, as we did in $\FOL$. This can be justified by the same arguments in III.4.
%
\item \textbf{Note to Remarks and Examples 1.3.} As introduced in text, the formation rules of second-order formulas are composed of those of first-order counterparts and two additional rules (for second-order). Thus when using inductions on second-order formulas or providing recursive definitions for them, there are indeed only two more cases (formulas of the form $X t_1 \ldots t_n$ or of the form $\exists X \varphi$) to be considered, campared with those for first-order ones.\\
\\
But sometimes, at least when it comes to discussing properties that are solely pertained to second-order logic, it is convenient to combine cases within the scope of first-order into one case. The definition of the free occurrences of relation variables given in (1) below exemplifies this argument.\\
\\
Here we present notes to all of the parts (1) - (7) of 1.3.
\begin{enumerate}[(1)]
\item The definition of the set of free (ordinary) \emph{variables} in a second-order formula $\varphi$ is obtained by extending that for first-order formulas (cf. II.5.1) by the following two cases:
\[
\begin{array}{lll}
\free{X t_1 \ldots t_n}  & := & \enumpop{\var{t_1}}{\cup}{\var{t_n}} \cr
\free{\exists X \varphi} & := & \free{\varphi}
\end{array}
\]
\ \\
As for the definition of the set of free \emph{relation variables} in a second-order formula $\varphi$, in which this set is denoted by $\sndordfree{\varphi}$, we have:\\
\ \\
\textbf{Definition of Free Occurrences of Relation Variables.}
\[
\begin{array}{lll}
\sndordfree{\varphi}           & \colonequals & \emptyset \quad \mbox{if \(\varphi\) is a first-order formula} \cr
\sndordfree{X\enum[1]{t}{n}}   & \colonequals & \{ X \} \cr
\sndordfree{\neg\varphi}       & \colonequals & \sndordfree{\varphi} \cr
\sndordfree{\varphi\lor \psi}  & \colonequals & \sndordfree{\varphi} \cup \sndordfree{\psi} \cr
\sndordfree{\exists x\varphi}  & \colonequals & \sndordfree{\varphi} \cr
\sndordfree{\exists X\varphi}  & \colonequals & \sndordfree{\varphi} \setminus \{ X \}
\end{array}
\]
\ \\
Now we are ready to state and prove the Coincidence Lemma for $\mathcal{L}_\mathrm{II}$:\\
\\
\textbf{Coincidence Lemma for $\mathcal{L}_\mathrm{II}$.} \emph{Let $\mathfrak{I}_1 = (\mathfrak{A}_1, \gamma_1)$ be a second-order $S_1$-interpretation and $\mathfrak{I}_2 = (\mathfrak{A}_2, \gamma_2)$ be a second-order $S_2$-interpretation, both with the same domain, i.e. $A_1 = A_2$. Put $S := S_1 \cap S_2$.}
\begin{enumerate}[(a)]
\item \emph{Let $t$ be an $S$-term. If $\mathfrak{I}_1$ and $\mathfrak{I}_2$ agree on the $S$-symbols occurring in $t$ and on the variables occurring in $t$, then $\mathfrak{I}_1(t) = \mathfrak{I}_2(t)$.}
%%%
\item \emph{Let $\varphi$ be a second-order $S$-formula. If $\mathfrak{I}_1$ and $\mathfrak{I}_2$ agree on the $S$-symbols and on the variables and relation variables occurring free in $\varphi$, then $\mathfrak{I}_1 \models \varphi$ iff $\mathfrak{I}_2 \models \varphi$.}
\end{enumerate}
\begin{proof} (a) cf. the proof of III.4.6.\\
\ \\
(b) We use induction on second-order $S$-formulas:
\begin{description}
\item $\varphi$ is a first-order $S$-formula: cf. the proof of III.4.6.
%%%
\item $\varphi = Xt_1 \ldots t_n$: $\mathfrak{I}_1 \models \varphi$\\
\begin{tabular}{ll}
iff & $\gamma_1(X)$ holds for $\mathfrak{I}_1(t_1)$, \ldots, $\mathfrak{I}_1(t_n)$ \cr
iff & $\gamma_1(X)$ holds for $\mathfrak{I}_2(t_1)$, \ldots, $\mathfrak{I}_2(t_n)$ \quad (by (a)) \cr
iff & \begin{minipage}[t]{59ex} $\gamma_2(X)$ holds for $\mathfrak{I}_2(t_1)$, \ldots, $\mathfrak{I}_2(t_n)$\\(by hypothesis, $\gamma_1(X) = \gamma_2(X)$)\end{minipage} \cr
iff & $\mathfrak{I}_2 \models \varphi$.
\end{tabular}
%%%
\item $\varphi = \neg \psi$: $\mathfrak{I}_1 \models \neg \psi$\\
\begin{tabular}{ll}
iff & not $\mathfrak{I}_1 \models \psi$\cr
iff & not $\mathfrak{I}_2 \models \psi$ (by induction hypothesis)\cr
iff & $\mathfrak{I}_2 \models \neg \psi$.
\end{tabular}
%%%
\item $\varphi = (\psi \lor \chi)$: $\mathfrak{I}_1 \models (\psi \lor \chi)$\\
\begin{tabular}{ll}
iff & $\mathfrak{I}_1 \models \psi$ or $\mathfrak{I}_2 \models \chi$\cr
iff & $\mathfrak{I}_2 \models \psi$ or $\mathfrak{I}_2 \models \chi$ (by induction hypothesis)\cr
iff & $\mathfrak{I}_2 \models (\psi \lor \chi)$.
\end{tabular}
%%%
\item $\varphi = \exists x \psi$: $\mathfrak{I}_1 \models \exists x \psi$\\
\begin{tabular}{ll}
iff & there is an $a \in A_1$ such that $\mathfrak{I}_1\frac{a}{x} \models \psi$\cr
iff & \begin{minipage}[t]{59ex} there is an $a \in A_2 (= A_1)$ such that $\mathfrak{I}_2\frac{a}{x} \models \psi$ (by induction hypothesis applied to $\psi$, $\mathfrak{I}_1\frac{a}{x}$ and $\mathfrak{I}_2\frac{a}{x}$; note that, because $\free(\psi) \subset \free(\varphi) \cup \{ x \}$ and $\sndordfree{\psi} = \sndordfree{\varphi}$, the interpretations $\mathfrak{I}_1\frac{a}{x}$ and $\mathfrak{I}_2\frac{a}{x}$ agree on all symbols occurring in $\psi$ and all variables as well as all relation variables occurring free in $\psi$)\end{minipage} \cr
iff & $\mathfrak{I}_2 \models \exists x \psi$.
\end{tabular}
%%%
\item $\varphi = \exists X^n \psi$: $\mathfrak{I}_1 \models \exists X^n \psi$\\
\begin{tabular}{ll}
iff & there is a $C \subset A_1^n$ such that $\mathfrak{I}_1\frac{C}{X} \models \psi$\cr
iff & \begin{minipage}[t]{59ex} there is a $C \subset A_2^n (= A_1^n)$ such that $\mathfrak{I}_2\frac{C}{X} \models \psi$ (by induction hypothesis applied to $\psi$, $\mathfrak{I}_1\frac{C}{X}$ and $\mathfrak{I}_2\frac{C}{X}$; similarly, because $\free(\psi) = \free(\varphi)$ and $\sndordfree{\psi} \subset \sndordfree{\varphi} \cup \{ X \}$, the interpretations $\mathfrak{I}_1\frac{C}{X}$ and $\mathfrak{I}_2\frac{C}{X}$ agree on all symbols occurring in $\psi$ and all variables as well as all relation variables occurring free in $\psi$)\end{minipage} \cr
iff & $\mathfrak{I}_2 \models \exists X^n \psi$.\qedhere
\end{tabular}
\end{description}
\end{proof}
%%
\item We justify the use of $\forall X \varphi$ as an abbreviation for $\neg \exists X \neg \varphi$:
\begin{center}
\begin{tabular}{ll}
\ & $\mathfrak{I} \models \forall X^n \varphi$, namely $\mathfrak{I} \models \neg \exists X^n \neg \varphi$ \cr
iff & not $\mathfrak{I} \models \exists X^n \neg \varphi$\cr
iff & there is no $C \subset A^n$ such that $\mathfrak{I}\frac{C}{X} \models \neg \varphi$\cr
iff & there is no $C \subset A^n$ such that not $\mathfrak{I}\frac{C}{X} \models \varphi$\cr
iff & for all $C \subset A^n$, $\mathfrak{I}\frac{C}{X} \models \varphi$\cr
\   & (since for any interpretation $\mathfrak{I}$, either $\mathfrak{I} \models \varphi$ or not $\mathfrak{I} \models \varphi$).
\end{tabular}
\end{center}
%%
\item A unary relation over a domain is a subset of it. On the other hand, the result
\begin{quote}
``No first-order axioms can characterize the structure $(\mathbb{N}, \mathbf{\sigma}, 0)$ up to isomorphism''
\end{quote}
follows from Corollary VI.4.4.
%%
\item Notice that we have discussed in VI.1 that $\mathfrak{R}^<$ cannot be characterized by any set of $S^<_{\mbox{\scriptsize ar}}$-sentences. A misprint is found in remark (4): '$S_{\mbox{\scriptsize ar}}$' in line 3 should be replaced by '$S^<_{\mbox{\scriptsize ar}}$'.
%%
\item We justify the validity of $(+)$. Given an arbitrary $S$-interpretation $\mathfrak{I} = (\mathfrak{A}, \gamma)$, we have for all variables $x$, $y$:
\begin{enumerate}[1)]
\item If $x \equiv y$ holds, i.e. $\mathfrak{I}(x) = \mathfrak{I}(y)$, then clearly for every subset $S$ of $A$, whenever it contains $\mathfrak{I}(x)$ it must also contain $\mathfrak{I}(y)$ and vice versa, hence $\forall X (Xx \leftrightarrow Xy)$ holds.
%%%
\item Conversely, if $\forall X (Xx \leftrightarrow Xy)$ holds, then for every subset $C$ of $A$, $\mathfrak{I} \displaystyle\frac{C}{X} \models (Xx \leftrightarrow Xy)$. In particular,
\[
\mathfrak{I}\frac{\{ \mathfrak{I}(y) \}}{X} \models (Xx \leftrightarrow Xy)
\]
(notice that $\{ \mathfrak{I}(y) \} = \{ a \in A | a = \mathfrak{I}(y) \}$). By definition, $\mathfrak{I} \displaystyle\frac{\{ \mathfrak{I}(y) \}}{X} \models Xy$. Thus we have $\mathfrak{I} \displaystyle\frac{\{ \mathfrak{I}(y) \}}{X} \models Xx$, or equivalently $\mathfrak{I}(x) = \mathfrak{I}(y)$, i.e. $x \equiv y$ holds.
\end{enumerate}
%%
\item Recall that in VIII.1 we introduced term-reduced formulas (cf. Definition VIII.1.1). Likewise in $\mathcal{L}_\mathrm{II}$, we can define logically equivalent\footnote{The notion of logical equivalence in $\mathcal{L}_\mathrm{II}$ is defined analogously.} \emph{term-reduced second-order formulas} $\psi^\ast$ for arbitrary second-order formulas $\psi$, by extending the definition in the proof of Theorem VIII.1.2 by two more cases below: For $n$-ary relation variables $X$,
\[
\begin{array}{lll}
[X t_1 \ldots t_n]^\ast & := & \exists x_1 \ldots \exists x_n ([t_1 \equiv x_1]^\ast \land \ldots \land [t_n \equiv x_n]^\ast \land \cr
\                       & \  & \ X x_1 \ldots x_n); \cr
[\exists X \psi]^\ast   & := & \exists X \psi^\ast.
\end{array}
\]
It is easy to prove the logical equivalence between $\psi$ and $\psi^\ast$ by induction and hence obtain\\
\ \\
\textbf{Theorem.} \emph{For every $\psi \in \LII^S$ there is a logically equivalent, term-reduced $\psi^\ast \in \LII^S$ with $\free(\psi) = \free(\psi^\ast)$ and $\freeII(\psi) = \freeII(\psi^\ast)$.}\nolinebreak\hfill$\talloblong$\\
\ \\
Following the result for term-reduced second-order formulas, we show how to turn a second-order formula $\varphi$ with ``function variables'' into an ordinary one, $\varphi^+$, that is equivalent to $\varphi$ \emph{in principle}. (In fact, $\varphi$ and $\varphi^+$ are logically equivalent for those interpretations $\INT{I} = (\struct{A}, \gamma)$ with
\begin{center}
$\gamma(g) (a_0, \ldots, a_{n - 1}) = a_n$ iff $\gamma(G) a_0 \ldots a_n$,
\end{center}
where $G$ is the $(n + 1)$-ary relation variable in $\varphi^+$ corresponding to the $n$-ary function variable in $\varphi$.)
\\
\ \\
Let $S$ be given. It suffices to define for term-reduced $\varphi$:\footnote{If $\varphi$ is not term-reduced, then we can turn it into a term-reduced one that is logically equivalent to $\varphi$ by earlier results. We get stucked, however, when recursively applying the definition to $\varphi$, if $\varphi$ does involve function variables; we did not take this case into account. A way around this problem is simple: Suppose the function variables occurring in $\varphi$ are among $g_0, \ldots, g_n$, then we ``pretend'' as if $S \cup \{ g_0, \ldots, g_n \}$ were a symbol set, i.e. as if $g_0, \ldots, g_n$ were function symbols, and we regard $\varphi$ as a second-order $S \cup \{ g_0, \ldots, g_n \}$-formula.}
\[
\begin{array}{lll}
[x \equiv y]^+ & := & x \equiv y; \cr
[f y_1 \ldots y_n \equiv x]^+ & := & f y_1 \ldots y_n \equiv x, \mbox{ if $f \in S$ is $n$-ary}; \cr
[c \equiv x]^+ & := & c \equiv x; \cr
[R x_1 \ldots x_n]^+ & := & R x_1 \ldots x_n, \mbox{ if $R \in S$ is $n$-ary}; \cr
[g y_1 \ldots y_n \equiv x]^+ & := & (G y_1 \ldots y_n x \land \forall y_1 \ldots \forall y_n \exists^{=1} x G y_1 \ldots y_n x), \cr
\ & \ & \mbox{ where $G$ is an $(n + 1)$-ary relation variable} \cr
\ & \ & \mbox{ currently not used}; \cr
[X x_1 \ldots x_n]^+ & := & X x_1 \ldots x_n; \cr
[\neg \varphi]^+ & := & \neg \varphi^+; \cr
(\varphi \lor \psi)^+ & := & (\varphi^ \lor \psi^+); \cr
[\exists x \varphi]^+ & := & \exists x \varphi^+; \cr
[\exists g \varphi]^+ & := & \exists G \varphi^+, \ \mbox{where $G$ is the relation variable that} \cr
\ & \ & \ \mbox{replaces $g$ during the transition from $\varphi$} \cr
\ & \ & \ \mbox{to $\varphi^+$ if $g$ occurs in $\varphi$; otherwise it is an} \cr
\ & \ & \ \mbox{arbitrary relation variable}; \cr
[\exists X \varphi]^+ & := & \exists X \varphi^+.
\end{array}
\]
(We will introduce (simultaneous) substitutions in the next part.)\\
\ \\
Next, we provide a \textit{proof} for the claim
\begin{quote}
``Every injective function from $A$ to $A$ is surjective if and only if $A$ is finite.''
\end{quote}
\ \\
Firstly, suppose $A$ is finite. Let $f: A \to A$ be a function, then by definition the range of $f$ is a subset of $A$. If $f$ is injective, then such a subset cannot be proper since, if it were, by the Pigeonhole Principle there would be two distinct elements $a_1$, $a_2$ of $A$ such that $f(a_1) = f(a_2)$ (notice that $A$ is \emph{finite}), which contradicts the premise that $f$ is injective.\\
\\
Conversely, suppose $A$ is not finite, then it must include a countable subset $B$. Let $g_1$ be a bijection from $B$ to $\mathbb{N}$. We choose a $b \in B$ and define $B^\prime := B \setminus \{ b \}$. Then obviously $B^\prime$ is a countable \emph{proper} subset of $A$. Let $g_2$ be a bijection from $B^\prime$ to $\mathbb{N}$. It is clear that $g_0 = g_2^{-1} \circ g_1$ is a bijection from $B$ to $B^\prime$. Let $g$ be a function from $A$ to $A$ such that
\[
g(a) := \begin{cases}
a, & \mbox{if \(a \in A \setminus B\)};\cr
g_0(a), & \mbox{otherwise},
\end{cases}
\]
then $g$ is injective but not surjective, since its range $A \setminus \{ b \}$ is a proper subset of $A$ as we mentioned earlier.\nolinebreak\hfill$\talloblong$\\
\ \\
Finally, we close this part by an argument about $\varphi_{\mbox{\scriptsize fin}}$. The formula $\varphi_0 := \forall x \exists^{=1} y Xxy$ states that the binary relation $X$ is a (unary) function, and when conjuncted with it the formula $\varphi_1 := \forall x \forall y \forall z((Xxz \land Xyz) \rightarrow x \equiv y)$ states that the function $X$ is injective. However, the formula $\varphi_2 := \forall y \exists x Xxy$ alone does by no means state that ``the function $X$ is surjective'' as we might mistakenly think so; it merely says that for every $y$ there exists some $x$ such that the \emph{relation} $X$ holds (that is, $Xxy$ holds).\\
\\
Then, how can we claim that $\varphi_{\mbox{\scriptsize fin}}$ works correctly? Let us reveal the subtlety here. The statement ``If $X$ is an injective function then it is a surjective function'' can be formalized using $\varphi_0$, $\varphi_1$ and $\varphi_2$, as follows:
\[
(\varphi_0 \land \varphi_1) \rightarrow (\varphi_0 \land \varphi_2).
\]
It is easy to see that the above formula is logically equivalent to
\[
(\varphi_0 \land \varphi_1) \rightarrow \varphi_2.
\]
Hence, eliminating the second $\varphi_0$ in the formula above the previous one leaves the semantics unaltered.
%%
\item We introduce in $\mathcal{L}_\mathrm{II}$ some syntactic operations (substitutions and relativizations) and develop results analogous to $\mathcal{L}_\mathrm{I}$. Assume $S$ fixed.\\
\ \\
Before we proceed, let us generalize the notion of \emph{reassigning} variables in interpretations: Let $\xi_0, \ldots, \xi_r$ be a list of pairwise distinct variables (ordinary or relation) and $\mathfrak{I} = (\mathfrak{A}, \gamma)$ a second-order interpretation; and for $0 \leq i \leq r$,
\begin{enumerate}[1)]
\item $\alpha_i \in A$ if $\xi_i$ is an ordinary variable;
%%%
\item $\alpha_i \subset A^n$ if $\xi_i$ is an $n$-ary relation variable.
\end{enumerate}
Then let $\gamma \displaystyle\frac{\alpha_0 \ldots \alpha_r}{\xi_0 \ldots \xi_r}$ be the second-order assignment in $\mathfrak{A}$ with
\[
\gamma \frac{\alpha_0 \ldots \alpha_r}{\xi_0 \ldots \xi_r} (\upsilon) := \begin{cases}
\gamma (\upsilon) & \mbox{if \(\upsilon \not\in \{ \xi_0, \ldots, \xi_r \}\)} \cr
\alpha_i & \mbox{if \(\upsilon = \xi_i\)}
\end{cases}
\]
and
\[
\mathfrak{I} \frac{\alpha_0 \ldots \alpha_r}{\xi_0 \ldots \xi_r} := \left( \mathfrak{A}, \gamma\frac{\alpha_0 \ldots \alpha_r}{\xi_0 \ldots \xi_r} \right).
\]
\ \\
The substitution operation for formulas in $\mathcal{L}_{\mbox{\scriptsize II}}$ is defined analogously:\\
\ \\
\textbf{Definition of Simultaneous Substitution.} Suppose $\xi_0$, \ldots, $\xi_r$ are a list of pairwise distinct variables (each of which is ordinary or relation), and $\tau_0$, \ldots, $\tau_r$ are a list in which each $\tau_i$ is either a term or a relation variable. Furthermore, $\tau_i$ is a relation variable if and only if $\xi_i$ is.\\
\ \\
Moreover, let $\xi_{i_1}$, \ldots, $\xi_{i_s}$ ($i_1 < \ldots < i_s$) be exactly the ordinary variables among $\xi_0$, \ldots, $\xi_r$.
\begin{enumerate}[(a)]
\item $\varphi \displaystyle \frac{\tau_0 \ldots \tau_r}{\xi_0 \ldots \xi_r} := \varphi\frac{\tau_{i_1} \ldots \tau_{i_s}}{\xi_{i_1} \ldots \xi_{i_s}}$ for $\varphi \in L^S$
%%%
\item \begin{math} [X t_1 \ldots t_n] \displaystyle \frac{\tau_0 \ldots \tau_r}{\xi_0 \ldots \xi_r} := \left\{
\begin{tabular}{l}
$X t_1 \displaystyle \frac{\tau_{i_1} \ldots \tau_{i_s}}{\xi_{i_1} \ldots \xi_{i_s}} \ldots t_n \frac{\tau_{i_1} \ldots \tau_{i_s}}{\xi_{i_1} \ldots \xi_{i_s}}$, \cr
\ \ \ if $X \not \in \{ \xi_0 \ldots \xi_r \}$ \cr
$\xi_k t_1 \displaystyle \frac{\tau_{i_1} \ldots \tau_{i_s}}{\xi_{i_1} \ldots \xi_{i_s}} \ldots t_n \frac{\tau_{i_1} \ldots \tau_{i_s}}{\xi_{i_1} \ldots \xi_{i_s}}$, \cr
\ \ \ if $X = \xi_k$
\end{tabular}
\right.
\end{math}
%%%
\item $[\neg \varphi] \displaystyle \frac{\tau_0 \ldots \tau_r}{\xi_0 \ldots \xi_r} := \neg \left[\varphi \frac{\tau_0 \ldots \tau_r}{\xi_0 \ldots \xi_r} \right]$
%%%
\item $(\varphi \lor \psi) \displaystyle \frac{\tau_0 \ldots \tau_r}{\xi_0 \ldots \xi_r} := \left(\varphi \displaystyle\frac{\tau_0 \ldots \tau_r}{\xi_0 \ldots \xi_r} \lor \psi \displaystyle \frac{\tau_0 \ldots \tau_r}{\xi_0 \ldots \xi_r} \right)$
%%%
\item Suppose $\xi_{j_1}, \ldots, \xi_{j_p}$ ($j_1 < \ldots < j_p$) are exactly those variables $\xi_j$ (either ordinary or relation) among $\xi_0, \ldots, \xi_r$, such that
\begin{center}
$\xi_j \in \free(\exists x \varphi) \cup \freeII(\exists x \varphi)$ and $\xi_j \neq \tau_j$.
\end{center}
In particular, $x \neq \xi_{j_1}, \ldots, x \neq \xi_{j_p}$. Then set
\[
[\exists x \varphi] \frac{\tau_0 \ldots \tau_r}{\xi_0 \ldots \xi_r} := \exists u \left[ \varphi \frac{\tau_{j_1} \ldots \tau_{j_p} u}{\xi_{j_1} \ldots \xi_{j_p} x} \right],
\]
where $u$ is the variable $x$ if $x$ does not occur in $\tau_{j_1}, \ldots, \tau_{j_p}$; otherwise $u$ is the first variable in the list $v_0, v_1, v_2, \ldots$ which does not occur in $\varphi, \tau_{j_1}, \ldots, \tau_{j_p}$.
%%%
\item Let $\xi_{j_1}, \ldots, \xi_{j_p}$ ($j_1 < \ldots < j_p$) be exactly those variables (either ordinary or relation) among $\xi_0$, \ldots, $\xi_r$, such that each $\xi_i$ other than $\xi_{j_1}$, \ldots, $\xi_{j_p}$ is a relation variable with
\[
\mbox{$\xi_i \not \in \free(\exists X^n \varphi)$ or $\xi_i = \tau_i$}.
\]
In particular, $X \neq \xi_{j_1}$, \ldots, $X \neq \xi_{j_p}$. Then set
\[
[\exists X^n \varphi] \displaystyle \frac{\tau_0 \ldots \tau_r}{\xi_0 \ldots \xi_r} := \exists U^n \left[ \varphi \displaystyle \frac{\tau_{j_1} \ldots \tau_{j_p} U^n}{\xi_{j_1} \ldots \xi_{j_p} X^n} \right],
\]
where $U^n$ is the relation variable $X^n$ if $X^n \neq \tau_{j_1}$, \ldots, $X^n \neq \tau_{j_p}$; otherwise $U^n$ is the first relation variable in the list $V^n_0$, $V^n_1$, $V^n_2$, \ldots which does not occur in $\varphi$, $\tau_{j_1}$, \ldots, $\tau_{j_p}$.
\end{enumerate}\ 
\\
We show that the Substitution Lemma holds for $\mathcal{L}_\mathrm{II}$:\\
\ \\
\textbf{Substitution Lemma for $\mathcal{L}_\mathrm{II}$.} (a) \emph{For every term $t$,
\[
\mathfrak{I} \left( t \frac{\tau_0 \ldots \tau_r}{\xi_0 \ldots \xi_r} \right) = \mathfrak{I} \frac{\mathfrak{I}(\tau_0) \ldots \mathfrak{I}(\tau_r)}{\xi_0 \ldots \xi_r} (t).
\]}
(b) \emph{For every second-order formula $\varphi$,
\begin{center}
$\mathfrak{I} \models \varphi \displaystyle\frac{\tau_0 \ldots \tau_r}{\xi_0 \ldots \xi_r}$ \ \ \ iff \ \ \ $\mathfrak{I} \displaystyle\frac{\mathfrak{I}(\tau_0) \ldots \mathfrak{I}(\tau_r)}{\xi_0 \ldots \xi_r} \models \varphi$.
\end{center}
}
\textit{Proof.} (a) Similar to the proof of part (a) of Substitution Lemma for $\mathcal{L}_\mathrm{I}$, and using Coincidence Lemma for $\mathcal{L}_\mathrm{II}$.\\
\ \\
(b) This is achieved by induction on $\varphi$. In the following, we assume that $\xi_{i_1}, \ldots, \xi_{i_s}$ ($i_1 < \ldots < i_s$) are exactly those ordinary variables among $\xi_0, \ldots, \xi_r$. Let $\mathfrak{I} = (\mathfrak{A}, \gamma)$.\\
\ \\
$\varphi$ is a first-order formula: Then $\varphi \displaystyle\frac{\tau_0 \ldots \tau_r}{\xi_0 \ldots \xi_r} = \varphi \displaystyle\frac{\tau_{i_1} \ldots \tau_{i_s}}{\xi_{i_1} \ldots \xi_{i_s}}$. Thus
\begin{center}
\begin{tabular}{ll}
\ & $\mathfrak{I} \models \varphi \displaystyle\frac{\tau_0 \ldots \tau_r}{\xi_0 \ldots \xi_r}$ \cr
iff & $\mathfrak{I} \models \varphi \displaystyle\frac{\tau_{i_1} \ldots \tau_{i_s}}{\xi_{i_1} \ldots \xi_{i_s}}$ \cr
iff & $\mathfrak{I} \displaystyle\frac{\mathfrak{I}(\tau_{i_1}) \ldots \mathfrak{I}(\tau_{i_s})}{\xi_{i_1} \ldots \xi_{i_s}} \models \varphi$ (by the Substitution Lemma for $\mathcal{L}_\mathrm{I}$) \cr
iff & $\mathfrak{I} \displaystyle\frac{\mathfrak{I}(\tau_0) \ldots \mathfrak{I}(\tau_r)}{\xi_0 \ldots \xi_r} \models \varphi$ (by the Coincidence Lemma for $\mathcal{L}_\mathrm{II}$).
\end{tabular}
\end{center}
$\varphi = X t_1 \ldots t_n$: Let $[X t_1 \ldots t_n] \displaystyle\frac{\tau_0 \ldots \tau_r}{\xi_0 \ldots \xi_r} = X^\prime t_1 \displaystyle\frac{\tau_{i_1} \ldots \tau_{i_s}}{\xi_{i_1} \ldots \xi_{i_s}} \ldots t_n \displaystyle\frac{\tau_{i_1} \ldots \tau_{i_s}}{\xi_{i_1} \ldots \xi_{i_s}}$. Thus
\begin{center}
\begin{tabular}{ll}
\ & $\mathfrak{I} \models [X t_1 \ldots t_n] \displaystyle\frac{\tau_0 \ldots \tau_r}{\xi_0 \ldots \xi_r}$ \cr
iff & $\mathfrak{I} \models X^\prime t_1 \displaystyle\frac{\tau_{i_1} \ldots \tau_{i_s}}{\xi_{i_1} \ldots \xi_{i_s}} \ldots t_n \displaystyle\frac{\tau_{i_1} \ldots \tau_{i_s}}{\xi_{i_1} \ldots \xi_{i_s}}$ \cr
iff & $\mathfrak{I}(X^\prime)$ holds for $\mathfrak{I}(t_1 \displaystyle\frac{\tau_{i_1} \ldots \tau_{i_s}}{\xi_{i_1} \ldots \xi_{i_s}}), \ldots, \mathfrak{I}(t_n \displaystyle\frac{\tau_{i_1} \ldots \tau_{i_s}}{\xi_{i_1} \ldots \xi_{i_s}})$ \cr
iff & $\mathfrak{I}(X^\prime)$ holds for $\mathfrak{I} \displaystyle\frac{\mathfrak{I}(\tau_{i_1}) \ldots \mathfrak{I}(\tau_{i_s})}{\xi_{i_1} \ldots \xi_{i_s}}(t_1), \ldots, \mathfrak{I} \displaystyle\frac{\mathfrak{I}(\tau_{i_1}) \ldots \mathfrak{I}(\tau_{i_s})}{\xi_{i_1} \ldots \xi_{i_s}}(t_n)$ \cr
\ & (by (a)) \cr
iff & $\mathfrak{I}(X^\prime)$ holds for $\mathfrak{I} \displaystyle\frac{\mathfrak{I}(\tau_0) \ldots \mathfrak{I}(\tau_r)}{\xi_0 \ldots \xi_r}(t_1), \ldots, \mathfrak{I} \displaystyle\frac{\mathfrak{I}(\tau_0) \ldots \mathfrak{I}(\tau_r)}{\xi_0 \ldots \xi_r}(t_n)$ \cr
\ & (by Coincidence Lemma for $\mathcal{L}_\mathrm{II}$) \cr
iff & $\mathfrak{I} \displaystyle\frac{\mathfrak{I}(\tau_0) \ldots \mathfrak{I}(\tau_r)}{\xi_0 \ldots \xi_r} (X)$ holds for $\mathfrak{I} \displaystyle\frac{\mathfrak{I}(\tau_0) \ldots \mathfrak{I}(\tau_r)}{\xi_0 \ldots \xi_r}(t_1), \ldots,$ \cr
\ & $\mathfrak{I} \displaystyle\frac{\mathfrak{I}(\tau_0) \ldots \mathfrak{I}(\tau_r)}{\xi_0 \ldots \xi_r}(t_n)$ \cr
\ & ($\mathfrak{I}(X^\prime) = \gamma(X^\prime) = \gamma \displaystyle\frac{\mathfrak{I}(\tau_0) \ldots \mathfrak{I}(\tau_r)}{\xi_0 \ldots \xi_r}(X) = \mathfrak{I} \displaystyle\frac{\mathfrak{I}(\tau_0) \ldots \mathfrak{I}(\tau_r)}{\xi_0 \ldots \xi_r} (X)$) \cr
iff & $\mathfrak{I} \displaystyle\frac{\mathfrak{I}(\tau_0) \ldots \mathfrak{I}(\tau_r)}{\xi_0 \ldots \xi_r} \models X t_1 \ldots t_n$. \cr
\end{tabular}
\end{center}
$\varphi = \neg \psi$: Then $[\neg \psi] \displaystyle\frac{\tau_0 \ldots \tau_r}{\xi_0 \ldots \xi_r} = \neg \left[ \psi \displaystyle\frac{\tau_0 \ldots \tau_r}{\xi_0 \ldots \xi_r} \right]$. Thus
\begin{center}
\begin{tabular}{ll}
\ & $\mathfrak{I} \models [\neg \psi] \displaystyle\frac{\tau_0 \ldots \tau_r}{\xi_0 \ldots \xi_r}$ \cr
iff & $\mathfrak{I} \models \neg \left[ \psi \displaystyle\frac{\tau_0 \ldots \tau_r}{\xi_0 \ldots \xi_r} \right]$ \cr
iff & not $\mathfrak{I} \models \psi \displaystyle\frac{\tau_0 \ldots \tau_r}{\xi_0 \ldots \xi_r}$ \cr
iff & not $\mathfrak{I} \displaystyle\frac{\mathfrak{I}(\tau_0) \ldots \mathfrak{I}(\tau_r)}{\xi_0 \ldots \xi_r} \models \psi$ (by induction hypothesis) \cr
iff & $\mathfrak{I} \displaystyle\frac{\mathfrak{I}(\tau_0) \ldots \mathfrak{I}(\tau_r)}{\xi_0 \ldots \xi_r} \models \neg \psi$.
\end{tabular}
\end{center}
$\varphi = (\psi_0 \lor \psi_1)$: Then $(\psi_0 \lor \psi_1) \displaystyle\frac{\tau_0 \ldots \tau_r}{\xi_0 \ldots \xi_r} = \left( \psi_0 \displaystyle\frac{\tau_0 \ldots \tau_r}{\xi_0 \ldots \xi_r} \lor \psi_1 \displaystyle\frac{\tau_0 \ldots \tau_r}{\xi_0 \ldots \xi_r} \right)$. Thus
\begin{center}
\begin{tabular}{ll}
\ & $\mathfrak{I} \models (\psi_0 \lor \psi_1) \displaystyle\frac{\tau_0 \ldots \tau_r}{\xi_0 \ldots \xi_r}$ \cr
iff & $\mathfrak{I} \models \left( \psi_0 \displaystyle\frac{\tau_0 \ldots \tau_r}{\xi_0 \ldots \xi_r} \lor \psi_1 \displaystyle\frac{\tau_0 \ldots \tau_r}{\xi_0 \ldots \xi_r} \right)$ \cr
iff & $\mathfrak{I} \models \psi_0 \displaystyle\frac{\tau_0 \ldots \tau_r}{\xi_0 \ldots \xi_r}$ or $\mathfrak{I} \models \psi_1 \displaystyle\frac{\tau_0 \ldots \tau_r}{\xi_0 \ldots \xi_r}$ \cr
iff & $\mathfrak{I} \displaystyle\frac{\mathfrak{I}(\tau_0) \ldots \mathfrak{I}(\tau_r)}{\xi_0 \ldots \xi_r} \models \psi_0$ or $\mathfrak{I} \displaystyle\frac{\mathfrak{I}(\tau_0) \ldots \mathfrak{I}(\tau_r)}{\xi_0 \ldots \xi_r} \models \psi_1$ \cr
\ & (by induction hypothesis) \cr
iff & $\mathfrak{I} \displaystyle\frac{\mathfrak{I}(\tau_0) \ldots \mathfrak{I}(\tau_r)}{\xi_0 \ldots \xi_r} \models (\psi_0 \lor \psi_1)$.
\end{tabular}
\end{center}
$\varphi = \exists x \psi$: Let $[\exists x \psi] \displaystyle\frac{\tau_0 \ldots \tau_r}{\xi_0 \ldots \xi_r} = \exists u \left[\psi\displaystyle\frac{\tau_{j_1} \ldots \tau_{j_p} u}{\xi_{j_1} \ldots \xi_{j_p} x} \right]$. Thus
\begin{center}
\begin{tabular}{ll}
\ & $\mathfrak{I} \models [\exists x \psi] \displaystyle\frac{\tau_0 \ldots \tau_r}{\xi_0 \ldots \xi_r}$ \cr
iff & $\mathfrak{I} \models \exists u \left[ \psi\displaystyle\frac{\tau_{j_1} \ldots \tau_{j_p} u}{\xi_{j_1} \ldots \xi_{j_p} x} \right]$ \cr
iff & there is an $a \in A$ such that $\mathfrak{I} \displaystyle\frac{a}{u} \models \psi\displaystyle\frac{\tau_{j_1} \ldots \tau_{j_p} u}{\xi_{j_1} \ldots \xi_{j_p} x}$ \cr
iff & there is an $a \in A$ such that $\mathfrak{I} \displaystyle\frac{\mathfrak{I}\displaystyle\frac{a}{u}(\tau_{j_1}) \ldots \mathfrak{I}\displaystyle\frac{a}{u}(\tau_{j_p}) a a}{\xi_{j_1} \ldots \xi_{j_p} u x} \models \psi$ \cr
\ & (by induction hypothesis) \cr
iff & there is an $a \in A$ such that $\mathfrak{I} \displaystyle\frac{\mathfrak{I}(\tau_{j_1}) \ldots \mathfrak{I}(\tau_{j_p}) a a}{\xi_{j_1} \ldots \xi_{j_p} u x} \models \psi$ \cr
\ & ($u$ does not occur in $\tau_{j_1}, \ldots, \tau_{j_p}$; using Coincidence Lemma for \cr
\ & $\mathcal{L}_\mathrm{II}$) \cr
iff & there is an $a \in A$ such that $\mathfrak{I} \displaystyle\frac{\mathfrak{I}(\tau_{j_1}) \ldots \mathfrak{I}(\tau_{j_p}) a}{\xi_{j_1} \ldots \xi_{j_p} x} \models \psi$ \cr
\ & (if $u \neq x$, then $u$ does not occur in $\psi$; using Coincidence Lemma \cr
\ & for $\mathcal{L}_\mathrm{II}$) \cr
iff & there is an $a \in A$ such that $\mathfrak{I} \displaystyle\frac{\mathfrak{I}(\tau_0) \ldots \mathfrak{I}(\tau_r) a}{\xi_0 \ldots \xi_r x} \models \psi$ \cr
\ & (by definition and Coincidence Lemma for $\mathcal{L}_\mathrm{II}$) \cr
iff & $\mathfrak{I} \displaystyle\frac{\mathfrak{I}(\tau_0) \ldots \mathfrak{I}(\tau_r) a}{\xi_0 \ldots \xi_r x} \models \exists x \psi$
\end{tabular}
\end{center}
$\varphi = \exists X^n \psi$: Let $[\exists X^n \psi] \displaystyle\frac{\tau_0 \ldots \tau_r}{\xi_0 \ldots \xi_r} = \exists U^n \left[ \psi \displaystyle\frac{\tau_{j_1} \ldots \tau_{j_p} U^n}{\xi_{j_1} \ldots \xi_{j_p} X^n} \right]$. Thus,
\begin{center}
\begin{tabular}{ll}
\ & $\mathfrak{I} \models [\exists X^n \psi] \displaystyle\frac{\tau_0 \ldots \tau_r}{\xi_0 \ldots \xi_r}$ \cr
iff & $\mathfrak{I} \models \exists U^n \left[ \psi \displaystyle\frac{\tau_{j_1} \ldots \tau_{j_p} U^n}{\xi_{j_1} \ldots \xi_{j_p} X^n} \right]$ \cr
iff & there is $C \subset A^n$ such that $\mathfrak{I} \displaystyle\frac{C}{U^n} \models \psi \displaystyle\frac{\tau_{j_1} \ldots \tau_{j_p} U^n}{\xi_{j_1} \ldots \xi_{j_p} X^n}$ \cr
iff & there is $C \subset A^n$ such that $\mathfrak{I} \displaystyle\frac{\mathfrak{I}\displaystyle\frac{C}{U^n}(\tau_{j_1}) \ldots \mathfrak{I}\displaystyle\frac{C}{U^n}(\tau_{j_p}) C C}{\xi_{j_1} \ldots \xi_{j_p} U^n X^n} \models \psi$ \cr
iff & there is $C \subset A^n$ such that $\mathfrak{I} \displaystyle\frac{\mathfrak{I}(\tau_{j_1}) \ldots \mathfrak{I}(\tau_{j_p}) C C}{\xi_{j_1} \ldots \xi_{j_p} U^n X^n} \models \psi$ \cr
\ & ($U^n \not\in \{ \tau_{j_1}, \ldots, \tau_{j_p} \}$) \cr
iff & there is $C \subset A^n$ such that $\mathfrak{I} \displaystyle\frac{\mathfrak{I}(\tau_{j_1}) \ldots \mathfrak{I}(\tau_{j_p}) C}{\xi_{j_1} \ldots \xi_{j_p} X^n} \models \psi$ \cr
\ & (if $U^n \neq X^n$, then $U^n$ does not occur in $\psi$; using Coincidence \cr
\ & Lemma for $\mathcal{L}_\mathrm{II}$) \cr
iff & there is $C \subset A^n$ such that $\mathfrak{I} \displaystyle\frac{\mathfrak{I}(\tau_0) \ldots \mathfrak{I}(\tau_r) C}{\xi_0 \ldots \xi_r X^n} \models \psi$ \cr
\ & (by definition and Coincidence Lemma for $\mathcal{L}_\mathrm{II}$) \cr
iff & $\mathfrak{I} \displaystyle\frac{\mathfrak{I}(\tau_0) \ldots \mathfrak{I}(\tau_r)}{\xi_0 \ldots \xi_r} \models \exists X^n \psi$.
\end{tabular}
\end{center}
\ \hfill$\talloblong$
\ \\
Let us proceed to Isomorphism Lemma for $\SOL$:\\
\ \\
\textbf{Isomorphism Lemma for $\SOL$.} \emph{If $\struct{A}$ and $\struct{B}$ are isomorphic structures then for all second-order sentences $\varphi$
\begin{center}
$\struct{A} \models \varphi$ \ \ \ iff \ \ \ $\struct{B} \models \varphi$.
\end{center}}
Before giving a proof for this lemma, let us consider the extension 
\[
\pi^\prime : A \cup (\bigcup\limits_{m \in \mathbb{Z}^+} \mathcal{P}(A^m)) \to B \cup (\bigcup\limits_{m \in \mathbb{Z}^+} \mathcal{P}(B^m))
\]
of $\pi$, where by $\mathcal{P}(C)$ we mean the \emph{power set of a set $C$}:
\[
\pi^\prime (\alpha) := \left\{
\begin{array}{ll}
\pi (\alpha)                                                   & \mbox{if $\alpha \in A$} \cr
\{ ( \pi (a_1), \ldots, \pi (a_n) ) \ | \ \mbox{$a_1, \ldots, a_n \in A$} & \cr
\multicolumn{1}{r}{\mbox{and $\alpha a_1 \ldots a_n$} \}}      & \mbox{if $\alpha \subset A^n$}
\end{array} \right.
\]
Trivially $\pi^\prime$ is bijective, from the bijectiveness of $\pi$.\\
\\
Suppose $P$ is an $n$-ary relation over $A$. Then by definition, for all $a_1, \ldots, a_n \in A$,
\begin{center}
$Pa_1 \ldots a_n$ \ \ \ iff \ \ \ $\pi^\prime (P) \pi (a_1) \ldots \pi (a_n)$.
\end{center}
In particular, if $X$ is a relation variable and $\gamma$ is a second-order assignment in $\struct{A}$, then for all $a_1, \ldots, a_n \in A$,
\begin{center}
(+) \hfill $\gamma (X) a_1 \ldots a_n$ \ \ \ iff \ \ \ $\gamma^\pi (X) \pi (a_1) \ldots \pi (a_n)$, \hfill \phantom{(+)}
\end{center}
where $\gamma^\pi := \pi^\prime \circ \gamma$.\\
\ \\
Let $\INT^\pi := (\struct{B}, \gamma^\prime)$. Then we have
\begin{center}
$(*)$ \hfill $\pi(\INT(t)) = \INT{I}^\pi(t)$. \hfill \phantom{(*)}
\end{center}
This can be verified easily by induction on $t$:\\
$t = x$:
\[
\begin{array}{ll}
\ & \pi(\INT(x)) \cr
= & \pi(\gamma(x)) \cr
= & \pi^\prime(\gamma(x)) \cr
= & \gamma^\pi(x) \cr
= & \INT^\pi(x).
\end{array}
\]
$t = c$:
\[
\begin{array}{ll}
\ & \pi(\INT(c)) \cr
= & \pi(c^\struct{A}) \cr
= & c^\struct{B} \cr
= & \INT^\pi(c).
\end{array}
\]
$t = f t_1 \ldots t_n$:
\[
\begin{array}{ll}
\ & \pi(\INT(f t_1 \ldots t_n)) \cr
= & \pi(f^\struct{A} (\INT(t_1), \ldots, \INT(t_n))) \cr
= & f^\struct{B} (\pi(\INT(t_1)), \ldots, \pi(\INT(t_n))) \cr
= & f^\struct{B} (\INT^\pi(t_1), \ldots, \INT^\pi(t_n)) \ \mbox{(by induction hypothesis)} \cr
= & \INT^\pi(f t_1 \ldots t_n).
\end{array}
\]
Finally, notice that for any variable $\xi$ (ordinary or relation) and any $\alpha \in A \cup \powerset{A^n}$ with $n > 0$ such that \emph{$\alpha$ can be assigned to $\xi$ appropriately},\footnote{That is, $\alpha \in A$ if $\xi$ is an ordinary variable; $\alpha \in \powerset{A^n}$ if $\xi$ is an $n$-ary relation variable.}
\begin{center}
$(**)$ \hfill $\left( \INT \df{\alpha}{\xi} \right)^\pi = \INT^\pi \df{\pi^\prime(\alpha)}{\xi}$. \hfill \phantom{(**)}
\end{center}
This can be verified easily.\\
\ \\
Now we are ready to provide a \textit{proof} for the above lemma: It is similar to the one of Isomorphism Lemma for $\FOL$, with slight modifications. (Say, by replacing all appearances of $\beta$ with $\gamma$.) And consider two more cases below:\\
$\varphi = X t_1 \ldots t_n$:
\begin{center}
\begin{tabular}{ll}
\   & $\struct{I} \models X t_1 \ldots t_n$ \cr
iff & $\gamma (X) \INT(t_1) \ldots \INT(t_n)$ \cr
iff & $\gamma^\pi (X) \pi(\INT(t_1)) \ldots \pi(\INT(t_n))$ \ (by $(+)$) \cr
iff & $\gamma^\pi (X) \INT^\pi (t_1) \ldots \INT^\pi (t_n)$ \ (by $(*)$) \cr
iff & $\INT^\pi \models X t_1 \ldots t_n$.
\end{tabular}
\end{center}
$\varphi = \exists X^n \psi$:
\begin{center}
\begin{tabular}{ll}
\   & $\INT \models \exists X^n \psi$ \cr
iff & there is a $C \subset A^n$ such that $\INT \df{C}{X^n} \models \psi$ \cr
iff & there is a $C \subset A^n$ such that $\left( \INT \df{C}{X^n} \right)^\pi \models \psi$ \cr
\   & (by induction hypothesis) \cr
iff & there is a $C \subset A^n$ such that $\INT^\pi \df{\pi^\prime (C)}{X^n} \models \psi$ \cr
\   & (by $(**)$) \cr
iff & there is a $D \subset B^n$ such that $\INT^\pi \df{D}{X^n} \models \psi$ \cr
\   & (since $A = B$ and $\pi^\prime$ is surjective) \cr
iff & $\INT^\pi \models \exists X^n \psi$.
\end{tabular}\\
\phantom{a} \hfill $\talloblong$
\end{center}
Similarly, from this proof we infer \\
\\
\textbf{Corollary.} \emph{Suppose $\pi: \mathfrak{A} \cong \mathfrak{B}$. Let $a_0, \ldots, a_{p - 1} \in A$, $C_0 \subset A^{n_0}$, \ldots, $C_{q - 1} \subset A^{n_{q - 1}}$, and $V_0$, \ldots, $V_{q - 1}$ are relation variables such that each $V_i$ is $n_i$-ary. Furthermore, if $\gamma$ is a second-order assignment in $\mathfrak{A}$ with $\gamma(v_0) = a_0$, \ldots, $\gamma(v_{p - 1}) = a_{p - 1}$, $\gamma(V_0) = C_0$, \ldots, $\gamma(V_{q - 1}) = C_{q - 1}$, $\gamma^\prime$ a second-order assignment in $\mathfrak{B}$ with $\gamma^\prime(v_0) = \pi^\prime(a_0)$, \ldots, $\gamma^\prime(v_{p - 1}) = \pi^\prime(a_{p - 1})$, $\gamma^\prime(V_0) = \pi^\prime(C_0)$, \ldots, $\gamma^\prime(V_{q - 1}) = \pi^\prime(C_{q - 1})$ (cf. the above proof for the description of $\pi^\prime$), then for $\varphi \in L^S_{\mbox{\scriptsize\upshape II}}$ with $\free(\varphi) \subset \{ v_0, \ldots, v_{p - 1} \}$ and with $\freeII(\varphi) \subset \{ V_0, \ldots, V_{q - 1} \}$,
\begin{center}
\phantom{$\talloblong$} \hfill $(\mathfrak{A}, \gamma) \models \varphi$ \ \ \ iff \ \ \ $(\mathfrak{B}, \gamma^\prime) \models \varphi$. \hfill $\talloblong$
\end{center}}
\ \\
Finally, we introduce relativizations in $\sndordlog$.\\
\ \\
\textbf{Definition of Relativization for $\sndordlog$.} Let $P$ be a unary relation symbol not contained in $S$. The \emph{relativization $\psi^P \in \LII^{S \cup \{ P \}}$ of second-order formulas $\psi \in \LII^S$ to $P$} is defined inductively by
\begin{center}
\begin{tabular}{lll}
$\psi^P$                   & $:=$ & $\psi$, if $\psi$ is atomic \cr
$[\neg \psi]^P$            & $:=$ & $\neg \psi^P$ \cr
$(\psi_1 \lor \psi_2)^P$   & $:=$ & $(\psi_1^P \lor \psi_2^P)$ \cr
$[\exists x \psi]^P$       & $:=$ & $\exists x (Px \land \psi^P)$ \cr
$[\exists X^n \psi]^P$     & $:=$ & $\exists X^n (\forall v_0 \ldots \forall v_{n - 1} (X v_0 \ldots v_{n - 1} \rightarrow Pv_0 \land \ldots$ \cr
\                          & \    & \ \ \ $\land Pv_{n - 1}) \land \psi^P)$.
\end{tabular}
\end{center}
By induction on the formula $\psi$ one can easily prove that for all assignments $\gamma : (\{ v_n \ | \ n \in \nat \} \to P^A) \cup (\bigcup\limits_{m \in \zah^+} \{ V^m_n \ | \ n \in \nat \} \to \powerset{(P^A)^m})$,
\begin{center}
$(\left[ P^A \right]^\struct{A}, \gamma) \models \psi$ \ \ \ iff \ \ \ $(\struct{A}, \gamma) \models \psi^P$;
\end{center}
and hence obtains:\smallskip\\
\begin{theorem}{Relativization Lemma for $\sndordlog$} Let $\struct{A}$ be an $S \cup \{ P \}$-structure such that $P \not\in S$ and $P$ is unary. Suppose the set $P^A \subset A$ is $S$-closed in $\struct{A}$. Then for all second-order $S$-sentences $\psi$,\smallskip\\
$\left[ P^A \right]^\struct{A} \models \psi$ \quad iff \quad $\struct{A} \models \psi^P$.\qed
\end{theorem}
\end{enumerate}
%
\item \textbf{More Syntactic Operations on $\mathcal{L}_\mathrm{II}$.} (INCOMPLETE) Recall that we have investigated many syntactic properties of $\mathcal{L}_\mathrm{I}$ in Chapter VIII. Many of them can be transferred to $\mathcal{L}_\mathrm{II}$. We mention a few.
\begin{enumerate}[(1)]
\item A symbol set is called relational if it contains only relation symbols. (cf. VIII.1.)\\
\ \\
Given an arbitrary symbol set $S$ and an $S$-structure $\mathfrak{A}$, let $S^r$ and the $S^r$-structure $\mathfrak{A}^r$ be defined as in VIII.1. Then by extending the definition of $\psi^r$ for term-reduced formulas $\psi$ in the proof of part (a) of Theorem VIII.1.3 by two more cases:
\[
\begin{array}{lll}
[X x_1 \ldots x_n]^r & := & X x_1 \ldots x_n; \cr
[\exists X \psi ]^r  & := & \exists X \psi^r,
\end{array}
\]
and extending the definition of $\psi^{-r}$ for $\psi$ in the proof of part (b) of the same theorem by two more cases:
\[
\begin{array}{lll}
[X t_1 \ldots t_n]^{-r} & := & X t_1 \ldots t_n; \cr
[\exists X \psi ]^{-r}  & := & \exists X \psi^{-r},
\end{array}
\]
respectively, we obtain:\medskip\\
\begin{theorem}{Theorem on Replacement Operation on $\sndordlog$}
\begin{enumerate}[\rm(a)]
\item For every $\psi \in \sndordlang{S}$ there is $\relational{\psi} \in \sndordlang{\relational{S}}$ such that for all (second-order) $S$-interpretations $\intp = \intparg{\struct{A}}{\sndordassgn}$,\\
\centerline{$\intparg{\struct{A}}{\sndordassgn} \models \psi$ \quad iff \quad $\intparg{\relational{\struct{A}}}{\sndordassgn} \models \relational{\psi}$.}
%%%
\item For every $\psi \in \LII^{S^r}$ there is $\psi^{-r} \in \LII^S$ such that for all $S$-inter-pretations $\mathfrak{I} = (\mathfrak{A}, \gamma)$,\\
\centerline{$(\mathfrak{A}, \gamma) \models \psi^{-r}$ \quad iff \quad $(\mathfrak{A}^r, \gamma) \models \psi$.}
\end{enumerate}
\end{theorem}
\begin{proof}
By induction.
\end{proof}
\ \\
As in \textbf{Note to Theorem 1.3} in notes to Chapter VIII, the above theorem can likewise be directly specialized to second-order sentences and structures (using, analogously, Coincidence Lemma for $\mathcal{L}_\mathrm{II}$). Moreover, if we take the $S^r$-sentence $\chi$ thereof then we obtain\\
\ \\
\textbf{Corollary.} \hfill \emph{$\models \psi$ \ \ \ iff \ \ \ $\models (\chi \rightarrow \psi^r)$.} \hfill \phantom{Corollary.}$\talloblong$
%%
\item In paragraph E of VIII.2 we introduced syntactic interpretations. By adopting the same definitions, we likewise obtain\\
\ \\
\textbf{Theorem on Syntactic Interpretations in $\sndordlog$.} \emph{Let $I$ be a syntactic interpretation on $S^\prime$ in $S$. Then, to every $\psi \in \LII^{S^\prime}$ one can associate a $\psi^I \in \LII^S$ with $\free(\psi^I) \subset \free(\psi)$ and $\freeII(\psi^I) = \freeII(\psi)$ such that for all $S$-structures $\mathfrak{A}$ with $\mathfrak{A} \models \Phi_I$ and all second-order assignments $\gamma$ in $\mathfrak{A}^{-I}$,
\begin{center}
$(\mathfrak{A}, \gamma) \models \psi^I$ \ \ \ iff \ \ \ $(\mathfrak{A}^{-I}, \gamma) \models \psi$.
\end{center}
In particular, for $\LII^{S^\prime}$-sentences $\psi$,
\begin{center}
$\mathfrak{A} \models \psi^I$ \ \ \ iff \ \ \ $\mathfrak{A}^{-I} \models \psi$.
\end{center}}
For the \emph{proof} of this theorem, extend the definition of $\varphi^I$ by two more cases: For $n$-ary relation variables $X$,
\[
\begin{array}{lll}
[X x_0 \ldots x_{n - 1}]^I & := & X x_0 \ldots x_{n - 1}; \cr
[\exists X \varphi]^I & := & \exists X (\varphi^I \land \forall x_0 \ldots \forall x_{n - 1} (X x_0 \ldots x_{n - 1} \rightarrow \cr
\ & \ & \multicolumn{1}{r}{(\varphi_{S^\prime}(x_0) \land \ldots \land \varphi_{S^\prime}(x_{n - 1}))));}
\end{array}
\]
and use induction.\nolinebreak\hfill$\talloblong$
%%
\item (INCOMPLETE HERE) Theorem on Prenex Normal Form for $\sndordlog$.
\end{enumerate}
%
\item \textbf{Note to the Proof of Theorem 1.4.} For a complete description of $\varphi_{\geq n}$, see III.6.3 (on page 46).
%
\item \textbf{Note to the Proof of Theorem 1.5.}
\begin{enumerate}[(1)]
\item $\varphi_{\mbox{\scriptsize unc}}$ is satisfiable becasue $\mathbb{R} \models \varphi_{\mbox{\scriptsize unc}}$. Note that it makes no difference here to regard the set $\mathbb{R}$ of real numbers as a ``structure'' since we are speaking of $L^\emptyset_{\mbox{\scriptsize II}}$. On the other hand, the concept of a set's being countable or uncountable that we are mentioning in this chapter is in the background sense (cf. VII.2).
%%
\item Define
\[
\begin{array}{lll}
\psi_{\mbox{\scriptsize fin}}(X) & := & \forall F ((\forall x (Xx \rightarrow \exists^{=1} y (Xy \land Fxy)) \\
\                             & \  & \phantom{\forall F} \land \forall x \forall y \forall z ((Xx \land Xy \land Xz \land Fxz \land Fyz) \rightarrow x \equiv y)) \\
\                             & \  & \phantom{\forall (} \rightarrow \forall y (Xy \rightarrow \exists x (Xx \land Fxy))).
\end{array}
\]
The idea behind this is that ``Every injective function defined over the set $X$ is surjective if and only if $X$ is finite'' (cf. 1.3 (6)). Also note that in this idea, we consider the graph of a function (hence the relation variable $F$) instead of the function itself.
%%
\item We show that a set $A$ is at most countable if and only if there is an ordering relation on $A$ such that every element has only finitely many predecessors.\\
\\
Firstly, suppose $A$ is at most countable and hence let $f: A \to \mathbb{N}$ be injective (cf. Lemma II.1.1). We define $<^A$ to be the ordering preserved under $f$, i.e. for any $a_1$, $a_2 \in A$,
\[
\mbox{$a_1 <^A a_2$ :iff $f(a_1) <^\mathbb{N} f(a_2)$}.
\]
Then for any $a \in A$, $g(a)$ has finitely many predecessors (in the sense of $<^\mathbb{N}$), and so does $a$ (in the sense of $<^A$).\\
\\
Conversely, suppose there is such an ordering relation $<^A$. Let $g: A \to \mathbb{N}$ be the function such that for any $a \in A$, $g(a)$ denotes the number of predecessors of $a$ (in the sense of $<^A$). Then $g$ is injective since for any $a_1$, $a_2 \in A$ with $a_1 \neq a_2$, either $a_1 <^A a_2$ or $a_2 <^A a_1$ (trichotomy, cf. $\Phi_{\mbox{\scriptsize ord}}$ in III.6.4), and thus $g(a_1) \neq g(a_2)$. By Lemma II.1.1, we have that $A$ is at most countable. \begin{flushright}$\talloblong$\end{flushright}
\end{enumerate}
%
\item \textbf{Note to 1.6.} Adhering to the terms referred to in our discussion in \textbf{Note to the Correctness (Theorem), the Completeness (Theorem), and the Adequacy (Theorem) of a Sequent Calculus} in the annotations to Chapter V, it is actually the absence of the Completeness Theorem for any possible sequent calculus suitable for $\mathcal{L}_\mathrm{II}$ that leads to the failure of the Compactness Theorem. Nevertheless, there is indeed no correct and complete sequent calculus for $\mathcal{L}_\mathrm{II}$ (cf. the remarks below X.5.5).\\
\ \\
To verify some correct sequent rules for $\SOL$, we may have to make explicit use of set-theoretical assumptions such as $\zfc$ (thus reflecting the relationship between second-order logic and set theory). For example, the rule (given a fixed symbol set $S$)\\
\centerline{$\calrule{ }{\Gamma \ \ \ \exists X \enump{\forall v_0}{\forall v_{n - 1}} (X \enum{v}{n - 1} \leftrightarrow \varphi (\seq{v}{n - 1}))}$, \begin{minipage}[t]{2.5cm}
$X$ is $n$-ary\\and $\varphi \in L^S_n$
\end{minipage}}\\
reflects the separation axioms of $\zfc$.\footnote{The domain of any structure is a set, to which separation axioms are applicable (cf. Section VII.3).} Also note that this rule cannot be generalized to $\varphi \in \LII^S$, since for $\varphi = \neg X v_0$ the sentence $\exists X \forall v_0 (X v_0 \leftrightarrow \neg X v_0)$ is not satisfiable.
%
%the following item is referenced in notes to Chapter XIII
\item \textbf{Note to the Discussions Concerning the Second-Order Logic and the Continuum Hypothesis on Pages 141 and 142.} Note that $\varphi_{\mbox{\scriptsize CH}} \in L^\emptyset_{\mbox{\scriptsize II}}$. Therefore $\varphi_{\mbox{\scriptsize CH}}$ is a basic semantic question about $\mathcal{L}_{\mbox{\scriptsize II}}$, since it refers to no particular symbol set ($S = \emptyset$ here). Recall that its first-order counterpart, $\mbox{CH}$, can be formulated in $L^{\{ \mbfs{\in} \}}$ (see page 110), cf. VII.3. (We shall later show that $\models \varphi_{\mbox{\scriptsize CH}}$ if and only if CH holds.) Also in that section we mentioned that ZFC is sufficient for mathematics but unfortunately it cannot decide CH, hence we conclude that some semantical properties of $\mathcal{L}_{\mbox{\scriptsize II}}$ are beyond the scope of the ZFC framework.\\
\\
Now we are ready to provide a \emph{proof} for the statement:
\[
\mbox{$\varphi_{\mbox{\scriptsize CH}}$ iff CH holds.}
\]
To start, we equip ourselves by formulating $\chi_{\leq \mbox{\scriptsize ctbl}}(X)$ as follows:\\
\begin{tabular}{lll}
$\chi_{\leq \mbox{\scriptsize ctbl}}(X)$ & $:=$ & $\exists Z (\forall x (Xx \rightarrow \neg Zxx)$ \cr
\ & \ & $\phantom{\exists Z (} \land \forall x \forall y \forall z ((Xx \land Xy \land Xz \land Zxy \land Zyz) \rightarrow Zxz)$ \cr
\ & \ & $\phantom{\exists Z (} \land \forall x \forall y ((Xx \land Xy) \rightarrow (Zxy \lor x \equiv y \lor Zyx))$ \cr
\ & \ & $\phantom{\exists Z (} \land \forall x (Xx \rightarrow \exists Y (\psi_{\mbox{\scriptsize fin}}(Y) \land \forall y (Yy \leftrightarrow Zyx))))$.
\end{tabular}\\
\\
Let us proceed into a digression for a while. Observe how we transfer $\varphi_{\mbox{\scriptsize fin}}$ and $\varphi_{\leq \mbox{\scriptsize ctbl}}$, respectively, into $\psi_{\mbox{\scriptsize fin}}$ and $\chi_{\leq \mbox{\scriptsize ctbl}}$: The resulting formulations are obtained by replacing all subformulas of the form $\forall x \varphi$ by the corresponding formulas of the form $\forall x (Xx \rightarrow \varphi)$, where $X$ stands for the set to be described. The idea behind this is that we restrict the case from the domain to a subset (of it), hence the string `$Xx \rightarrow$' preceding $\varphi$.\\
\\
On the other hand, $\varphi_{\mathfrak{R}^<}$ in line 9 is erroneously typed, the correct one is $\psi_{\mathfrak{R}^<}$.\\
\\
Alright, let us return to our proof. To continue, we formulate the second-order sentence $\varphi_\mathbb{R}$ as
\[
\varphi_\mathbb{R} := \exists F_+ \exists F_\cdot \exists c_0 \exists c_1 \exists R_< (\chi_{\mbox{\scriptsize binary func}}(F_+) \land \chi_{\mbox{\scriptsize binary func}}(F_\cdot) \land \psi),
\]
where $\psi$ is obtained from $\psi_{\mathfrak{R}^<}$ by replacing each of the symbols $+$, $\cdot$, $0$, $1$ and $<$ by $F_+$, $F_\cdot$, $c_0$, $c_1$ and $R_<$, respectively, as is clear by the choice of the symbols.\\
\\
Well, the critical part starts. Suppose $\models \varphi_{\mbox{\scriptsize CH}}$, hence in particular $\mathbb{R} \models \varphi_{\mbox{\scriptsize CH}}$. Then since $\mathbb{R} \models \varphi_\mathbb{R}$,  it follows that every subset of $\mathbb{R}$ is either at most countable or else of the same cardinality as $\mathbb{R}$ (i.e. there is a bijection of $\mathbb{R}$ onto this subset), which means that CH holds.\\
\\
Conversely, suppose CH holds. Let $\mathfrak{A}$ be an arbitrarily chosen second-order structure (with respect to some symbol set $S$). If $A$ is of the same cardinality as $\mathbb{R}$, then by definition there is a bijection $f$ of $A$ onto $\mathbb{R}$. Consider any subset $B$ of $A$, it must be of the same cardinality as $f(B)$ (which is a subset of $\mathbb{R}$), with the restriction of $f$ on $B$ serving as a bijection for this. Since by hypothesis CH holds, $f(B)$ is either at most countable or there is a bijection of $\mathbb{R}$ onto $f(B)$. Thus we have that $B$ is either at most countable or there is a bijection of $\mathbb{R}$ onto $B$, and further that $B$ is either at most countable or else of the same cardinality as $A$. Therefore $\mathfrak{A} \models \varphi_{\mbox{\scriptsize CH}}$. Because $\mathfrak{A}$ is arbitrarily chosen (as mentioned earlier), it follows that $\models \varphi_{\mbox{\scriptsize CH}}$. The proof is complete. \begin{flushright}$\talloblong$\end{flushright}
%
\item \textbf{Solution to Exercise 1.7 with Remark.} (INCOMPLETE) Note that in $\mathcal{L}^w_{\mbox{\scriptsize II}}$, by the way that `$\exists X^n$' is interpreted, `$\forall X^n$' is interpreted as \emph{for all finite $C \subset A^n$}.\\
\\
On the other hand, the notion of satisfaction $\models_w$ for $\mathcal{L}^w_{\mbox{\scriptsize II}}$ is, formally speaking, defined to coincide with that for $\mathcal{L}_{\mbox{\scriptsize II}}$, except for the case that for $\mathfrak{I} = (\mathfrak{A}, \gamma)$:
\[
\mbox{$\mathfrak{I} \models_w \exists X^n \varphi$ iff there is a \emph{finite} $C \subset A^n$ such that $\mathfrak{I} \displaystyle \frac{C}{X^n} \models_w \varphi$.}
\]
(Notice that we have slightly modified the statement in text: We changed `$\models$' to `$\models_w$', since the original one seems a bit inconsistent with its intended meaning...)\\
\\
We provide solutions to parts (a) - (c).
\begin{enumerate}[(a)]
\item Here we formulate as a second-order $\{ \leq \}$-sentence $\varphi$ the statement ``for every nonempty set $X$, there is a maximum element in the sense of the relation $\leq$'':
\[
\varphi := \forall X (\exists z Xz \rightarrow \exists x ( Xx \land \forall y (Xy \rightarrow y \leq x))).
\]
Let $\mathfrak{A} = (\mathbb{N}, \leq)$. Then clearly $\mathfrak{A} \models_w \varphi$ but not $\mathfrak{A} \models \varphi$.
%%
\item We define a map $^+$ by induction on formulas, which associates with every $L^{w, S}_{\mbox{\scriptsize II}}$-formula $\varphi$ an $L^S_{\mbox{\scriptsize II}}$-formula $\varphi^+$, as follows:\\
\begin{tabular}{lll}
$\varphi$ is atomic & : & $\varphi^+ := \varphi$ \cr
$\varphi = \neg \chi$ & : & $\varphi^+ := \neg \chi^+$ \cr
$\varphi = (\chi_1 \lor \chi_2)$ & : & $\varphi^+ := (\chi_1^+ \lor \chi_2^+)$ \cr
$\varphi = \exists x \chi$ & : & $\varphi^+ := \exists x \chi^+$ \cr
$\varphi = \exists X \chi$ & : & $\varphi^+ := \exists X (\psi_{\mbox{\scriptsize fin}}(X) \land \chi^+)$
\end{tabular}\\
(For the formulation of $\psi_{\mbox{\scriptsize fin}}(X)$, cf. part (2) of \textbf{Note to the Proof of Theorem 1.5}.)\\
\\
We show by induction on formulas that for each $L^{w, S}_{\mbox{\scriptsize II}}$-formula $\varphi$, the $L^S_{\mbox{\scriptsize II}}$-formula $\varphi^+$ satisfies: For all $S$-interpretations $\mathfrak{I} = (\mathfrak{A}, \gamma)$,
\[
\mbox{$\mathfrak{I} \models_w \varphi$ iff $\mathfrak{I} \models \varphi^+$.}
\]
But before doing so, note that the only difference between $\mathcal{L}_{\mbox{\scriptsize II}}$ and $\mathcal{L}^w_{\mbox{\scriptsize II}}$ lies in the definitions of the notions of satisfaction for formulas of the form $\exists X \chi$. So the proof can be done by only giving the $\exists X$-step: Let $\varphi = \exists X^n \chi$, then $\varphi^+ = \exists X^n (\psi_{\mbox{\scriptsize fin}}(X^n) \land \chi^+)$. It follows that for all $S$-interpretations $\mathfrak{I} = (\mathfrak{A}, \gamma)$:\\
\begin{tabular}{lll}
$\mathfrak{I} \models_w \exists X^n \chi$ & iff & there is a finite $C \subset A^n$ such that $\mathfrak{I} \displaystyle \frac{C}{X^n} \models_w \chi$ \cr
\ & iff & there is a $C \subset A^n$ such that $C$ is finite and \cr
\ & \   & $\mathfrak{I} \displaystyle \frac{C}{X^n} \models_w \chi$ \cr
\ & iff & there is a $C \subset A^n$ such that $C$ is finite and \cr
\ & \   & $\mathfrak{I} \displaystyle \frac{C}{X^n} \models \chi^+$ (by induction hypothesis) \cr
\ & iff & there is a $C \subset A^n$ such that $\mathfrak{I} \displaystyle \frac{C}{X^n} \models \psi_{\mbox{\scriptsize fin}}(X^n)$ \cr
\ & \   & and $\mathfrak{I} \displaystyle \frac{C}{X^n} \models \chi^+$ \cr
\ & iff & there is a $C \subset A^n$ such that \cr
\ & \   & $\mathfrak{I} \displaystyle \frac{C}{X^n} \models (\psi_{\mbox{\scriptsize fin}}(X^n) \land \chi^+)$ \cr
\ & iff & $\mathfrak{I} \models \exists X^n (\psi_{\mbox{\scriptsize fin}}(X^n) \land \chi^+)$,
\end{tabular}\\
namely $\mathfrak{I} \models (\exists X^n \chi)^+$. In particular, if $\varphi$ is an $L^{w, S}_{\mbox{\scriptsize II}}$-sentence, then $\varphi^+$ is an $L^S_{\mbox{\scriptsize II}}$-sentence and for all $S$-structures $\mathfrak{A}$, $\mathfrak{A} \models_w \varphi$ if and only if $\mathfrak{A} \models \varphi^+$ (by the Coincidence Lemma for $\mathcal{L}_{\mbox{\scriptsize II}}$, the argument is similar to that given at the bottom in page 37 in text). It is easy to see that $\psi := \varphi^+$ meets the requirement.
%%
\item The following set of $L^{w, \emptyset}_{\mbox{\scriptsize II}}$-sentences is a counterexample:
\[
\{ \exists X \forall x Xx \} \cup \{ \varphi_{\geq n} | n \geq 2 \}.
\]
\end{enumerate}
\textit{Remark.} (INCOMPLETE HERE) Notice that $\weaksndordlog$ differs with $\sndordlog$ only in the satisfaction relation. In $\weaksndordlog$ we tacitly adopt syntactic and semantic notions (such as term-reduced formulas and logical equivalence, respectively) from $\sndordlog$. Then we have the following results:\medskip\\
\begin{theorem}{Coincidence Lemma for $\weaksndordlog$}
(INCOMPLETE)
\end{theorem}
\begin{proof}
(INCOMPLETE)
\end{proof}
\begin{theorem}{Theorem on Term-Reduced Formulas in $\weaksndordlog$}
For every $\psi \in \weaksndordlang{S}$ there is a logically equivalent, term-reduced formula $\psi^\ast \in \weaksndordlang{S}$ with $\free{\psi} = \free{\psi^\ast}$ and $\sndordfree{\psi} = \sndordfree{\psi^\ast}$.
\end{theorem}
\begin{proof}
By induction.
\end{proof}
\begin{theorem}{Relativization Lemma for $\weaksndordlog$}
Let $\struct{A}$ be an $S \cup \{ P \}$-structure such that $P \not\in S$ and $P$ is unary. Suppose the set $\intpted{P}{A} \subset A$ is $S$-closed in $\struct{A}$. Then for all $\weaksndordlang{S}$-sentences $\psi$,\smallskip\\
\centerline{$\substr{{\intpted{P}{A}}}{\struct{A}} \models \psi$ \quad iff \quad $\struct{A} \models \relativize{\psi}{P}$.}
\end{theorem}
\begin{proof}
By induction.
\end{proof}
\begin{theorem}{Theorem on Replacement Operation on $\weaksndordlog$}
\begin{enumerate}[\rm(a)]
\item For every $\psi \in \weaksndordlang{S}$ there is $\relational{\psi} \in \weaksndordlang{\relational{S}}$ such that for all (second-order) $S$-interpretations $\intp = \intparg{\struct{A}}{\sndordassgn}$,\\
\centerline{$\intparg{\struct{A}}{\sndordassgn} \models \psi$ \quad iff \quad $\intparg{\relational{\struct{A}}}{\sndordassgn} \models \relational{\psi}$.}
%%%
\item For every $\psi \in \weaksndordlang{{\relational{S}}}$ there is $\invrelational{\psi} \in \weaksndordlang{S}$ such that for all $S$-interpretations $\intp = \intparg{\struct{A}}{\sndordassgn}$,\\
\centerline{$\intparg{\struct{A}}{\sndordassgn} \models \invrelational{\psi}$ \quad iff \quad $\intparg{\relational{\struct{A}}}{\sndordassgn} \models \psi$.}
\end{enumerate}
\end{theorem}
\begin{proof}
By induction.
\end{proof}

\end{enumerate}
%End of Section IX.1-----------------------------------------------------------------------------------------
\
\\
\\
%Section IX.2----------------------------------------------------------------------------------------------
{\large \S2. The System $\infinlog$}
\begin{enumerate}[1.]
\item \textbf{Note to 2.1 Definition of $\mathcal{L}_{\omega_1\omega}$.} Note that the formulas $\bigvee \{ \varphi, \psi \}$ and $(\varphi \lor \psi)$, though different in their syntactic characteristic, are logically equivalent.\footnote{The notion of logical equivalence in $\mathcal{L}_{\omega_1\omega}$ is defined analogously.}\\
\\
Also notice that the Theorem on the Prenex Normal Form does not hold for $\mathcal{L}_{\omega_1\omega}$, as is demonstrated by the counterexample below:
\[
\bigvee \{ \exists v_i \varphi_i | i \in \mathbb{N} \}.
\]
%
\item \textbf{Note to the $\mathcal{L}_{\omega_1\omega}$-Sentence $\bigvee \{ \underbrace{1 + \ldots + 1}_{\mbox{\scriptsize\begin{math}n\end{math}-times}} \equiv 0 | n \geq 2 \}$ in Page 143.} Actually $\bigvee \{ \underbrace{1 + \ldots + 1}_{\mbox{\scriptsize \begin{math}p\end{math}-times}} \equiv 0 | \mbox{\begin{math}p\end{math} is a prime} \}$ is sufficient for the purpose, since if $\underbrace{1 + \ldots + 1}_{\mbox{\scriptsize\begin{math}p\end{math}-times}} \equiv 0$ holds for some prime $p$ then $\underbrace{1 + \ldots + 1}_{\mbox{\scriptsize\begin{math}n\end{math}-times}} \equiv 0$ also holds for all multiples $n$ of $p$.
%
\item \textbf{Note to the $L^{ \{ \mbfs{\sigma}, 0 \} }_{\omega_1\omega}$-Sentence Characterizing $(\mathbb{N}, \sigma, 0)$ Up to Isomorphism in Page 143.} Let (P3)$^\prime$ denote the following $L^{ \{ \mbfs{\sigma}, 0 \} }_{\omega_1\omega}$-sentence:
\[
\forall x \bigvee \{ x \equiv \underbrace{\mbf{\sigma} \ldots \mbf{\sigma}}_{\mbox{\scriptsize $n$-times}} 0 | n \geq 0 \}.
\]
We show that (P1), (P2) (cf. III.7.3) and (P3)$^\prime$ characterize $\mathfrak{N}_\sigma = (\mathbb{N}, \sigma, 0)$ up to isomorphism.\\
\\
To start, assume that $\mathfrak{A} = (A, \mbf{\sigma}^A, 0^A)$ is a $\{ \mbf{\sigma}, 0 \}$-structure satisfying (P1), (P2) and (P3)$^\prime$. Define the isomorphism $\pi: \mathfrak{N}_\sigma \cong \mathfrak{A}$ by the clause:
\[
\mbox{$\pi ( ( \mbf{\sigma}^\mathbb{N} )^n ( 0^\mathbb{N} ) ) := (\mbf{\sigma}^A)^n (0^A)$ \ \ \ for all $n \in \mathbb{N}$,}
\]
that is,
\[
\mbox{$\pi ( (\sigma)^n ( 0 ) ) = (\mbf{\sigma}^A)^n (0^A)$ \ \ \ for all $n \in \mathbb{N}$,}
\]
where by $(f)^n$ we mean $n$ iterative applications of a function $f$. From (P3)$^\prime$ it is clear that $\pi$ is well defined by the clause above. And it turns out that
\begin{equation}
\pi ( 0^\mathbb{N} ) = 0^A, \label{equation1}
\end{equation}
and again by (P3)$^\prime$ that
\begin{equation}
\mbox{$\pi( \mbf{\sigma}^\mathbb{N} ( n ) ) = \mbf{\sigma}^A ( \pi ( n ) )$ \ \ \ for all $n \in \mathbb{N}$.} \label{equation2}
\end{equation}
At this point, both the compatibility conditions ((\ref{equation1}) and (\ref{equation2}) above) for an isomorphism are already satisfied, and it remains to show that $\pi$ is bijective.\\
\\
Surjectivity: Immediately follows from (P3)$^\prime$ and the defining clause of $\pi$.\\
\\
Injectivity: Suppose $m$, $n \in \mathbb{N}$, $m \neq n$. Without loss of generality, we assume that $m < n$. Naturally, $m = (\sigma)^m ( 0 )$ and $n = (\sigma)^n ( 0 )$. Furthermore, from the defining clause of $\pi$ we have that $\pi (m) = (\mbf{\sigma}^A)^m (0^A)$ and $\pi (n) = (\mbf{\sigma}^A)^n (0^A)$. Here we claim that $\pi ( m ) \neq \pi ( n )$, for otherwise we could apply (P2) $m$ times to $(\pi (m) = ) \  (\sigma)^m (0) = (\sigma)^n (0) \  ( = \pi (n))$
to obtain that $(\sigma)^{n - m} (0) = \sigma ( (\sigma)^{n - m - 1} (0) ) = 0$,
which contradicts (P1).
%
\item \textbf{More Syntactic Operations on $\infinlog$.} We introduce several syntactic operations on $\infinlog$. In the following arguments, let a symbol set $S$ be given.\\
\ \\
First, we extend the definition of the set $\free{\varphi}$ of free variables occurring in $\varphi$ introduced in \reftitle{II.5} to obtain the $\infinlog$-version by considering one more case:
\[
\begin{array}{lll}
\free{\bigvee\Phi} & \colonequals & \bigcup_{\varphi \in \Phi} \free{\varphi}.
\end{array}
\]
Next, by adding the following clause
\[
\begin{array}{lll}
\parenadj{\bigvee \Phi}^\ast & \colonequals & \bigvee \setm{\psi^\ast}{\psi \in \Phi}
\end{array}
\]
to the definition of term-reduced formulas $\psi^\ast$ introduced in \reftitle{VIII.1} we obtain the $\infinlog$-version and\medskip\\
\begin{theorem}{Theorem on Term-Reduced Formulas in $\infinlog$}
For every $\psi \in \infinlang{S}$ there is a logically equivalent, term-reduced formula $\psi^\ast \in \infinlang{S}$ with $\free{\psi} = \free{\psi^\ast}$.
\end{theorem}
\begin{proof}
By induction.
\end{proof}
Then, let $P \not\in S$ be unary. By adding the following clause
\[
\begin{array}{lll}
\relativize{\parenadj{\bigvee\Phi}}{P} & \colonequals & \bigvee\setm{\relativize{\psi}{P}}{\psi \in \Phi}
\end{array}
\]
to the definition of the relativization $\relativize{\psi}{P}$ of $\psi$ to $P$ introduced in \reftitle{VIII.2} we obtain the $\infinlog$-version and\medskip\\
\begin{theorem}{Relativization Lemma for $\infinlog$}
Let $\struct{A}$ be an $S \cup \{ P \}$-structure such that $P \not\in S$ and $P$ is unary. Suppose the set $P^A \subset A$ is $S$-closed in $\struct{A}$, and moreover $\assgn$ is an assignment in $\struct{A}$ with $\assgn(v_n) \in P^A$ for $n \in \nat$. Then for $\psi \in \infinlang{S}$,\\
\centerline{$\intparg{\substr{P^A}{\struct{A}}}{\assgn} \models \psi$ \quad iff \quad $\intparg{\struct{A}}{\assgn} \models \relativize{\psi}{P}$.}
\end{theorem}
\begin{proof}
By induction.
\end{proof}
Finally, let $\relational{S}$ and $\relational{\struct{A}}$ in which $\struct{A}$ is an $S$-structure be defined as in \reftitle{VIII.1}. Furthermore, we extend the definition of $\relational{\psi}$ mentioned there to obtain the $\infinlog$-version by adding the clause:
\[
\begin{array}{lll}
\relational{\parenadj{\bigvee\Phi}} & \colonequals & \bigvee\setm{\relational{\psi}}{\psi \in \Phi};
\end{array}
\]
and likewise extend the definition of $\invrelational{\psi}$ by adding the clause:
\[
\begin{array}{lll}
\invrelational{\parenadj{\bigvee\Phi}} & \colonequals & \bigvee\setm{\invrelational{\psi}}{\psi \in \Phi}.
\end{array}
\]
Then as in \reftitle{VIII.1.3}, we obtain:\medskip\\
\begin{theorem}{Theorem on the Replacement Operation on $\infinlog$}
\emph{\begin{enumerate}[(a)]
\item \emph{For every $\psi \in \infinlang{S}$ there is $\relational{\psi} \in \infinlang{\relational{S}}$ such that for all $S$-interpretations $\intp = \intparg{\struct{A}}{\assgn}$,\\
\centerline{$\intparg{\struct{A}}{\assgn} \models \psi$ \quad iff \quad $\intparg{\relational{\struct{A}}}{\assgn} \models \relational{\psi}$.}}
%%
\item \emph{For every $\psi \in \infinlang{\relational{S}}$ there is $\invrelational{\psi} \in \infinlang{S}$ such that for all $S$-interpretations $\intp = \intparg{\struct{A}}{\assgn}$,\\
\centerline{$\intparg{\struct{A}}{\assgn} \models \invrelational{\psi}$ \quad iff \quad $\intparg{\relational{\struct{A}}}{\assgn} \models \psi$.}}
\end{enumerate}}
\end{theorem}
\begin{proof}
By induction.
\end{proof}
%
\item \textbf{Note to 2.2 Remarks.} The items below correspond to the remarks listed in 2.2 in text.
\begin{enumerate}
\item The following argument serves as a verification of the assertion mentioned in Remark 2.2 (a):\\
\ \\
\begin{tabular}{ll}
\ & $\mathfrak{I} \models \bigwedge \Phi$, namely $\mathfrak{I} \models \neg \bigvee \{ \neg \varphi | \varphi \in \Phi \}$ \cr
\Iff & not $\mathfrak{I} \models \bigvee \{ \neg \varphi | \varphi \in \Phi \}$ \cr
\Iff & not ($\mathfrak{I} \models \neg \varphi$ for some $\varphi \in \Phi$) \cr
\Iff & for all $\varphi \in \Phi$, not $\mathfrak{I} \models \neg \varphi$ \cr
\Iff & for all $\varphi \in \Phi$, $\mathfrak{I} \models \neg \neg \varphi$ \cr
\Iff & for all $\varphi \in \Phi$, $\mathfrak{I} \models \varphi$.
\end{tabular}\\
\\
First notice that the last (logical) equivalence follows from \textbf{Proposition}: \textit{For every formula $\varphi$, $\varphi \bimodels \neg \neg \varphi$} in notes to Section III.4. There is a word of caution: We cannot claim this equivalence from the derivable sequent rules
\[
\mbox{$\displaystyle \frac{\Gamma \;\;\; \varphi \phantom{\neg\neg}}{\Gamma \;\;\; \neg \neg \varphi}$ \ \ \ and
\ \ \ $\displaystyle \frac{\Gamma \;\;\; \neg \neg \varphi}{\Gamma \;\;\; \varphi \phantom{\neg\neg}}$}
\]
presented in Section IV.3 (cf. parts (a1) and (a2) of Exercise 3.6) together with the Adequacy Theorem, as these results apply to $\mathcal{L}_\mathrm{I}$ only.
%%
\item As an extention of $\mathcal{L}_\mathrm{I}$, the system $\mathcal{L}_{\omega_1\omega}$ possesses slightly different properties. For example,
\[
\bigvee \{ v_n \equiv v_{n + 1} | n \geq 1 \} \not \in \SF(\bigvee \{ v_n \equiv v_{n + 1} | n \geq 0 \}).
\]
\\
Next, the claim that an at most countable union of at most countable sets is at most countable basically follows from part (a) of Exercise II.1.4 (for the case of \emph{finite} union of at most countable sets, consider $M_p = \emptyset$ for all $p \geq n_0$, for some $n_0 \in \mathbb{N}$).\\
\\
Finally, let $\varphi$ be an arbitrary formula in $L^S_{\omega_1\omega}$. Note that it is obtained by \emph{finite} applications of the formution rules of $\mathcal{L}_{\omega_1\omega}$ (cf. Definition 2.1). Furthermore, $\SF(\varphi)$ is at most countable (shown in text). It turns out that $\varphi$ can be eventually decomposed into at most countably many subformulas (of $\varphi$) of finite lengths. Thus there are only at most countably many symbols involved in $\varphi$, showing that there exists an at most countable $S^\prime \subset S$ such that $\varphi \in L^{S^\prime}_{\omega_1\omega}$.
%%
\item The assertion mentioned in text is shown below. We confine ourselves to the $\bigvee$-step, as other cases are trivial.\\
\\
Let $\varphi := \bigvee \Phi$, and suppose $\free(\varphi) := \bigcup_{\psi \in \Phi} \free(\psi)$ is finite. Then clearly for every $\psi \in \Phi$, $\free(\psi)$ is also finite, which by induction hypothesis implies that $\free(\chi)$ is finite for every subformula $\chi$ of $\psi$. Therefore, $\free(\psi)$ is also finite for every subformula $\psi$ of $\varphi$. (Recall that $\SF(\bigvee \Phi) := \{ \bigvee \Phi \} \cup \bigcup_{\psi \in \Phi} \SF(\psi)$.)\\
\\
In particular, for every $\mathcal{L}_{\omega_1\omega}$-sentence $\varphi$ we have that $\free(\varphi) = \emptyset$, hence that $\varphi$ has only finitely many free variables.
\end{enumerate}
%
\item \textbf{Note to Theorem 2.3.} The following discussion provides a counterexample of the Completeness as well as the Compactness Theorems for $\mathcal{L}_{\omega_1\omega}$:
\begin{center}
Let $S := \{ c_i | i \in \mathbb{N} \}$, and $\Phi := \{ \neg c_i \equiv c_j | i, j \in \mathbb{N}, i < j \}$.
\end{center}
We see that $\Phi \models \bigwedge \Phi$, but not $\Phi \vdash \bigwedge \Phi$,\footnote{There is no sequent $\Gamma \subset \Phi$ with $\Gamma \vdash \bigwedge \Phi$, as $\Gamma$ is finite in length.} even if we allow \emph{infinitely long} derivations. As a result, there is no finite subset $\Phi_0$ of $\Phi$ such that $\Phi_0 \models \bigwedge \Phi$; furthermore, there exist consistent but unsatisfiable sets of $\mathcal{L}_{\omega_1\omega}$-formulas (e.g. $\{ \psi_\mathrm{fin} \} \cup \{ \varphi_{\geq n} | n \geq 2 \}$ introduced before Theorem 2.3).
%
\item \textbf{Conjecture.} Note that in property (2) of $\mathcal{L}_{\omega_1\omega}$ stated in page 145, the correctness and the completeness of the sequent calculus for $\mathcal{L}_{\omega_1\omega}$ may be generalized to the case in which the sequent consists of formulas: Given \emph{$\mathcal{L}_{\omega_1\omega}$-formulas} $\varphi_1, \ldots, \varphi_n, \varphi$, the sequent $\varphi_1 \ldots \varphi_n \varphi$ is derivable iff it is correct. We argue this by reducing the original sequent into another in whicn new constants take the place where free variables originally occur, and apply the result to it and then restore to the original sequent. The key point is that in such case \emph{the free variables behave as constants}.
%
\item \textbf{Note to the Proof of Lemma 2.5.} First, the set $B_0$ mentioned in the preliminary analysis must incluce the set
\[
\{ t^\mathfrak{B} | \mbox{$t$ is a variable-free term} \}
\]
as a subset.\\
\\
Second, the definition of the sequence $A_0$, $A_1$, $A_2$, \ldots is actually a procedure for the domain $A$ of the countable substructure $\mathfrak{A}$ of $\mathfrak{B}$ (enlarging $A_0$ gradually to $A$). Also notice that this definition works as $\SF(\varphi)$ together with the set of formulas of the form $fx_1 \ldots x_n \equiv x$ are at most countable on the one hand, and $\free(\psi)$ is finite for all formulas $\psi \in \SF(\varphi)$ (since $\varphi$ is a \emph{sentence}, see part (c) of Remarks 2.2) on the other. Next, in the first paragraph in page 146 in text, the set $A_m$ is at most countable, hence so is $\{ (a_1, \ldots, a_n) | a_1, \ldots, a_n \in A_m \}$. And besides $\SF(\varphi)$ and $A_m$, the set of formulas of the form $fx_1 \ldots x_n \equiv x$ is also at most countable (this is a missing statement in text). Therefore, the $A_m^\prime$ is at most countable. (The previous two statements serve as a supplement to the reason for this result.)\\
\\
Third, in part (2) of the analysis of the domain $A$, note that
\[
\mathfrak{B} \models \exists x \; fx_1 \ldots x_n \equiv x[a_1, \ldots, a_n]
\]
definitely holds, as $f^\mathfrak{B}$ is defined over $B$.\\
\\
Finally, in the $\exists$-case of the inductive proof of ($\ast\ast$), the assumption that $\psi(x_1, \ldots, x_n) = \exists x \chi(x_1, \ldots, x_n, x)$ is no loss of generality since, if $x$ is among the $n$ variables $x_1$, \ldots, $x_n$, by applying substitution operation we can transform this formula into an equivalent one in which $x$ follows $x_1$, \ldots, $x_n$ in the variable list. Also note that ($\ast$) follows from ($\ast\ast$) because $\varphi \in \SF(\varphi)$ is a sentence.\\
\\
We close this note by the concluding remark: The L\"{o}wenheim-Skolem Theorem for the case of $\mathcal{L}_{\omega_1\omega}$-formulas $\varphi$ with only \emph{finitely} many free variables, i.e. that every satisfiable $\mathcal{L}_{\omega_1\omega}$-formula $\varphi$ involving only finitely many free variables is satisfiable over an at most countable domain, immediately follows from this proof.\newline
\ 
\\\textit{Remark.} Notice that we cannot directly generalize this lemme to the case of formulas in which infinitely many free variables occur: in the construction process given in the proof, uncountably many elements might be added if we allowed the formula $\psi \in \SF (\varphi)$ therein to contain infinitely many free variables. However, there is a way around this problem, see the \textbf{Claim} below.
%
\item \textbf{Claim.} \emph{Let $S$ be at most countable, $\varphi$ an $L^S_{\omega_1\omega}$-formula and $\mathfrak{B}$ an $S$-structure such that $(\mathfrak{B}, \beta) \models \varphi$. Then there is an at most countable substructure $\mathfrak{A} \subset \mathfrak{B}$ such that $(\mathfrak{A}, \beta) \models \varphi$, where $\beta(v_i) \in A$ for $i \in \mathbb{N}$.}\newline
\ 
\\Hence the complete statement given in the L\"{o}wenheim-Skolem Theorem VI.1.1 absolutely holds for $\mathcal{L}_{\omega_1\omega}$, as given any satisfiable and at most countable set $\Phi$ of formulas we can take the formula $\varphi := \bigwedge \Phi$ and apply the result in this conjecture to it. (If $\Phi$ consists of only \emph{sentences}, then 2.5 is sufficient for the purpose.)\newline
\ 
\\On the other hand, this claim is a generalization of 2.5, which in turn entails a generalization of 2.4:
\begin{quote}
\emph{Every satisfiable $\mathcal{L}_{\omega_1\omega}$-formula has a model over an at most countable domain.}
\end{quote}
\ 
\\\textit{Proof.} First, we add countably many new constants $c_0$, $c_1$, \ldots to $S$ to comprise a new symbol set $S^\prime$. It is clear that $S^\prime$ is countable.\\
\\
Second, let $c_i$ correspond to $v_i$ for $i \in \mathbb{N}$, and define for every $L^S_{\omega_1\omega}$-\emph{formula} $\psi$ the $L^{S^\prime}_{\omega_1\omega}$-\emph{sentence} $\psi^\prime$ to be obtained by replacing all free occurrences of variables (if any) by occurrences of the corresponding constants. For example, if
\[
\psi = (\forall v_0 \exists v_1 v_0 \equiv f(v_1) \land \neg v_2 \equiv v_3),
\]
then
\[
\psi^\prime := (\forall v_0 \exists v_1 v_0 \equiv f(v_1) \land \neg c_2 \equiv c_3).
\]
\ \\
Third, one should agree that for every $\psi \in L^S_{\omega_1\omega}$ and every $S$-structure $\mathfrak{C}$ with $(\mathfrak{C}, \beta_0) \models \psi$,
\[
(\mathfrak{C}, \beta_0) \models \psi \; \Iff \; \mathfrak{C}^\prime \models \psi^\prime,
\]
where $\mathfrak{C}^\prime$ is the $S^\prime$-expansion of $\mathfrak{C}$ with $c_i^{\mathfrak{C}^\prime} := \beta_0(v_i)$ for $i \in \mathbb{N}$. From this observation we have that $\mathfrak{B}^\prime \models \varphi^\prime$ since $(\mathfrak{B}, \beta) \models \varphi$ by premise.\\
\\
Finally, by Lemma 2.5 there is an at most countable substructure $\mathfrak{A}^\prime \subset \mathfrak{B}^\prime$ such that $\mathfrak{A}^\prime \models \varphi^\prime$. Then, by the observation made above again, we have that $(\mathfrak{A}, \beta) \models \varphi$, where $\mathfrak{A}$ is, symmetrically, the $S$-reduct of $\mathfrak{A}^\prime$. This is so because for $i \in \mathbb{N}$,
\[
\begin{tabular}{lll}
$\beta(v_i)$ & $=$ $c_i^{\mathfrak{B}^\prime}$ \cr
\            & $=$ $c_i^{\mathfrak{A}^\prime}$ (since $\mathfrak{A}^\prime \subset \mathfrak{B}^\prime$) \cr
\            & $\in$ $A^\prime = A$ (note that $\mathfrak{A}^\prime$ is an $S^\prime$-expansion of $\mathfrak{A}$).
\end{tabular}
\]
\begin{flushright}$\talloblong$\end{flushright}
%
\item \textbf{Note to the Paragraph Immediately Following the Proof of Lemma 2.5.} By applying 2.5 to $\varphi := \bigwedge \Phi$, we have that every model of $\Phi$ has an at most countable substructure which is also a model of $\Phi$. (Every model of $\varphi$ is a model of $\Phi$, and vice versa.)\\
\\
\textbf{Conjecture.} Instead of resorting to the generalization of 2.5 mentioned earlier, we can prove that of 2.4 directly: \textit{Suppose $\Phi$ is an at most countable set of first-order formulas, and $(\mathfrak{B}, \beta)$ a model of $\Phi$. Furthermore, let $\Phi^\prime$ be obtained by replacing every formula $\psi(x_1, \ldots, x_n) \in \Phi$ by $\exists x_1 \ldots \exists x_n \psi$ and $\varphi := \bigwedge \Phi^\prime$, then $\mathfrak{B} \models \Phi^\prime$ and hence $\mathfrak{B} \models \varphi$. By 2.5 there is an at most countable substructure $\mathfrak{A} \subset \mathfrak{B}$ such that $\mathfrak{A} \models \varphi$ and hence $\mathfrak{A} \models \Phi^\prime$, where in the procedure given in the proof of 2.5 we additionally dictate that $\beta(v_i) \in A_0$ for $i \in \mathbb{N}$. Also, in this procedure we will certainly encounter the case $\exists x \psi[\beta(x_1), \ldots, \beta(x_n)]$ in which $x \not \in \free(\psi)$ at some stage (since $\beta(x_i)$ is kept in $A_0$ as mandated). Therefore $(\mathfrak{A}, \beta) \models \exists x \psi$ and further $(\mathfrak{A}, \beta) \models \psi$ as $x \not \in \free(\psi)$. As it turns out, $(\mathfrak{A}, \beta) \models \Phi$. This serves as a demonstration of the L\"{o}wenheim-Skolem Theorem for first-order logic which does not rely on the proof of the Completeness Theorem. Note that the result stated in this conjecture also follows from the previous conjecture.}\begin{flushright}$\talloblong$\end{flushright}
%
\item \textbf{Note to Page 147.}
\begin{enumerate}[(1)]
\item In the third paragraph, the statement ``A group $\mathfrak{G}$ is said to be \emph{simple} if $\{ e^G \}$ and $G$ are the only normal subgroups of $\mathfrak{G}$.'' should be modified into ``A group $\mathfrak{G}$ is said to be \emph{simple} if $\{ e^G \}$ and $G$ are the domains of the only normal subgroups of $\mathfrak{G}$.'' (A set by itself is \emph{not} a group unless a binary function $\circ$ is defined over it and this set together with $\circ$ satisfy the group axioms.)
%%
\item The sentence
\[
\begin{array}{r}
\forall x (\neg x \equiv e \rightarrow \forall y \bigvee \{ \exists u_0 \ldots \exists u_n \bigvee \{ y \equiv u_0 x^{z_0} u_0^{-1} \ldots u_n x^{z_n} u_n^{-1} | \phantom{n \in \mathbb{N} \} } \cr
z_0, \ldots, z_n \in \mathbb{Z} \} | n \in \mathbb{N} \},
\end{array}
\]
given in the middle of page 147, should be modified into
\[
\begin{array}{r}
\forall x (\neg x \equiv e \rightarrow \forall y \bigvee \{ \exists u_0 \ldots \exists u_n \bigvee \{ y \equiv u_0 x^{z_0} u_0^{-1} \ldots u_n x^{z_n} u_n^{-1} | \phantom{n \in \mathbb{N} \} } \cr
z_0, \ldots, z_n \in \mathbb{Z} \} | n \in \mathbb{N} \} ),
\end{array}
\]
i.e. the right parenthesis `$)$' is missing in the original one.
\end{enumerate}
%
\item \textbf{Note to 2.6.} In the premise of the statement given in 2.6, the simple group $\mathfrak{G}$ should furthermore be \emph{infinite}, otherwise it would be impossible for it to have a countable subgroup. Also note that in the proof of 2.6, applying 2.5 to $\mathfrak{G}^\prime$ and $\varphi_s$ only promises that the simple subgroup thus obtained is \emph{at most countable}. It is the fact that the subgroup contains the countable set $M$ that implies its being \emph{countable}.
%
\item \textbf{Solution to Exercise 2.7} Let a symbol set $S$ be given. We shall assume a countable set $U \colonequals \setm{v^{n, k}_{i, j}}{k \in \nat, n > 0, i > 0, 1 \leq j \leq n}$ of (first-order) variables not appearing in those $\weaksndordlang{S}$-sentences we consider in this exercise\footnote{This is possible since, say, we may multiply by $2$ all the indices of (first-order) variables occurring in each $\weaksndordlang{S}$-formula (for example, we obtain $\forall X \exists v_2 Xv_2$ from $\forall X \exists v_1 Xv_1$) and thus use only variables of even indices, yet having a countable set of unused variables, namely variables of odd indices.} (given $n$ and $k$, the variables $v^{n, k}_{i, j}$ will be associated with the second-order variable $V^n_k$). This is not an essential restriction for our purpose because for every $\weaksndordlang{S}$-sentence that contains variables from $U$ there is a logically equivalent one that contains no variables from $U$. For example, $\forall X \exists v^{1, 0}_{3, 1} X v^{1, 0}_{3, 1}$ is logically equivalent to $\forall X \exists v X v$, where $v$ is any variable other than those from $U$.\bigskip\\
\emph{Notice that we are considering $\weaksndordlang{S}$-\emph{sentences} in this exercise, thus every second-order variable therein stands for a \emph{finite} set when we talk about satisfaction in a structure. Without loss of generality, therefore, we assume that for any structure $\struct{A}$ and any second-order assignment $\sndordassgn$ in $\struct{A}$, the set $\sndordassgn(V^n_k)$ is finite for every second-order variable $V^n_k$ (according to the Coincidence Lemma for $\weaksndordlog$). Moreover, if $\sndordassgn(V^n_k)$ contains $m > 0$ elements, then $\setm{\tuple{\seqp{\sndordassgn(v^{n, k}_{i, 1})}{\sndordassgn(v^{n, k}_{i, n})}}}{1 \leq i \leq m} = \sndordassgn(V^n_k)$; there is no restriction on the values of $\sndordassgn(v^{n, k}_{i, j})$ if $\sndordassgn(V^n_k) = \emptyset$. In order to be compatible, we assume additionally that these properties also hold for $\sndordassgn\sbst{C}{V^n_k}$ (definitely $C$ will be chosen to be finite in this situation).}\bigskip\\
For $m \in \nat$ we say the pair $\pair{V^n_k}{m}$ is a \emph{substitute for $V^n_k$}. If $F$ is a set of substitutes for second-order variables, then we write\smallskip\\
\centerline{$F \leftarrow \pair{V^n_k}{m} \colonequals (F \setminus \setm{\pair{V^n_k}{m^\prime}}{m^\prime \in \nat}) \cup \{ \pair{V^n_k}{m} \}$.}\bigskip\\
Now, with every $\weaksndordlang{S}$-formula $\varphi$ and every set $F$ of substitutes for second-order variables in which $F \cap \sett{\pair{V^n_k}{m}}{\(V^n_k \in \sndordfree{\varphi}\) and \(m \in \nat\)}$ defines a map from $\sndordfree{\varphi}$ to $\nat$ if $\sndordfree{\varphi} \neq \emptyset$, we associate an $\infinlang{S}$-formula $\varphi^F$, which is defined by induction below:\medskip\\
\begin{tabular}{lll}
$(t_1 \equal t_2)^F$ & $\colonequals$ & $t_1 \equal t_2$ \cr
$(R \enum[1]{t}{n})^F$ & $\colonequals$ & $R \enum[1]{t}{n}$ \cr
$(V^n_k \enum[1]{t}{n})^F$ & $\colonequals$ &
\begin{minipage}[t]{48ex}
$\begin{cases}
\bigvee\limits_{1 \leq i \leq m}\parenadj{\bigwedge\limits_{1 \leq j \leq n} t_j \equal v^{n, k}_{i, j}} & \mbox{if \(m > 0\)} \cr
\exists v_0 \neg v_0 \equal v_0 & \mbox{otherwise,}
\end{cases}$\smallskip\\
in which $\pair{V^n_k}{m} \in F$
\end{minipage} \cr
$(\neg\varphi)^F$ & $\colonequals$ & $\neg\varphi^F$ \cr
$(\varphi \lor \psi)^F$ & $\colonequals$ & $\varphi^F \lor \psi^F$ \cr
$(\exists x \varphi)^F$ & $\colonequals$ & $\exists x \varphi^F$ \cr
$(\exists V^n_k \varphi)^F$ & $\colonequals$ & $\bigvee \setm{ \exists^m \varphi^F_{V^n_k}}{m \in \nat}$,
\end{tabular}\medskip\\
where we adopt the following abbreviations throughout this exercise:\smallskip\\
\begin{tabular}{lll}
$\exists^0 \varphi^F_{V^n_k}$ & $\colonequals$ & $\varphi^{F \leftarrow \pair{V^n_k}{0}}$, \cr
$\exists^1 \varphi^F_{V^n_k}$ & $\colonequals$ & $\enump{\exists v^{n, k}_{1, 1}}{\exists v^{n, k}_{1, n}} \varphi^{F \leftarrow \pair{V^n_k}{1}}$,
\end{tabular}\smallskip\\
and for $m \geq 2$,\smallskip\\
\begin{tabular}{lll}
$\exists^m \varphi^F_{V^n_k}$ & $\colonequals$ & $\enump{\enump{\exists v^{n, k}_{1, 1}}{\exists v^{n, k}_{1, n}}}{\enump{\exists v^{n, k}_{m, 1}}{\exists v^{n, k}_{m, n}}} \varphi^{F, m}_{V^n_k}$ \cr
$\varphi^{F, m}_{V^n_k}$ & $\colonequals$ & $\bigwedge\limits_{1 \leq i < j \leq m} \parenadj{\bigvee\limits_{1 \leq l \leq n} \neg v^{n, k}_{i, l} \equal v^{n, k}_{j, l}} \land \varphi^{F \leftarrow \pair{V^n_k}{m}}$.
\end{tabular}\bigskip\\
For every (second-order) $S$-interpretation $\intparg{\struct{A}}{\sndordassgn}$ and every $\varphi \in \weaksndordlang{S}$, we say the (possibly empty) set $F$ of substitutes for second-order variables is a \emph{substitutor of $\intparg{\struct{A}}{\sndordassgn}$ and $\varphi$} if $F \cap \sett{\pair{V^n_k}{m}}{\(V^n_k \in \sndordfree{\varphi}\) and \(m \in \nat\)} = \setm{\pair{V^n_k}{\card{\sndordassgn(V^n_k)}}}{V^n_k \in \sndordfree{\varphi}}$ \quad ($\sndordassgn(V^n_k)$ is finite by assumption).\bigskip\\
Finally, given an $S$-interpretation $\intparg{\struct{A}}{\sndordassgn}$ we denote by $\assgn^\sndordassgn$ the (first-order) assignment in $\struct{A}$ that is a subset of $\sndordassgn$ (identifying the maps $\sndordassgn$ and $\assgn^\sndordassgn$ with their graphs).\bigskip\\
We then have:\medskip\\
\begin{theorem}{Claim}
For every $\weaksndordlang{S}$-formula $\varphi$, every $S$-interpretation $\intparg{\struct{A}}{\sndordassgn}$, and every substitutor $F$ of $\intparg{\struct{A}}{\sndordassgn}$ and $\varphi$,\smallskip\\
\begin{quoteno}{\rm($\ast$)}
$\intparg{\struct{A}}{\sndordassgn} \models_\weak \varphi$ \quad iff \quad $\intparg{\struct{A}}{\assgn^\sndordassgn} \models \varphi^F$.
\end{quoteno}
\end{theorem}
\begin{proof}
We prove ($\ast$) by induction on $\varphi$.\medskip\\
For first-order atomic formulas $\varphi$, ($\ast$) is trivially true.\medskip\\
For $\varphi = V^n_k \enum[1]{t}{n}$: If $\sndordassgn(V^n_k) = \emptyset$ then ($\ast$) is trivially true. So let $m \colonequals \card{\sndordassgn(V^n_k)} > 0$ below.\smallskip\\
\begin{tabular}[b]{ll}
\   & $\intparg{\struct{A}}{\sndordassgn} \models_\weak \varphi$ \cr
iff & $\tuple{\seqp{\intparg{\struct{A}}{\sndordassgn}(t_1)}{\intparg{\struct{A}}{\sndordassgn}(t_n)}} \in \sndordassgn(V^n_k)$ \cr
iff & $\seqp{\intparg{\struct{A}}{\sndordassgn}(t_1) = \sndordassgn(v^{n, k}_{i, 1})}{\intparg{\struct{A}}{\sndordassgn}(t_n) = \sndordassgn(v^{n, k}_{i, n})}$ \quad for some $1 \leq i \leq m$ \cr
iff &
\begin{minipage}[t]{64ex}
$\seqp{\intparg{\struct{A}}{\assgn^\sndordassgn}(t_1) = \assgn^\sndordassgn(v^{n, k}_{i, 1})}{\intparg{\struct{A}}{\assgn^\sndordassgn}(t_n) = \assgn^\sndordassgn(v^{n, k}_{i, n})}$\\for some $1 \leq i \leq m$\\(since $\intparg{\struct{A}}{\sndordassgn}(t_j) = \intparg{\struct{A}}{\assgn^\sndordassgn}(t_j)$ and $\sndordassgn(v^{n, k}_{i, j}) = \assgn^\sndordassgn(v^{n, k}_{i, j})$)
\end{minipage} \cr
iff & $\seqp{\intparg{\struct{A}}{\assgn^\sndordassgn} \models t_1 \equal v^{n, k}_{i, 1}}{\intparg{\struct{A}}{\assgn^\sndordassgn} \models t_n \equal v^{n, k}_{i, n}}$ \quad for some $1 \leq i \leq m$ \cr
iff & $\intparg{\struct{A}}{\assgn^\sndordassgn} \models \bigwedge\limits_{1 \leq j \leq n} t_j \equal v^{n, k}_{i, j}$ \quad for some $1 \leq i \leq m$ \cr
iff & $\intparg{\struct{A}}{\assgn^\sndordassgn} \models \varphi^F$.
\end{tabular}\bigskip\\
For $\varphi = \neg\psi$: $\intparg{\struct{A}}{\sndordassgn} \models_\weak \varphi$\smallskip\\
\begin{tabular}[b]{ll}
iff & not $\intparg{\struct{A}}{\sndordassgn} \models_\weak \psi$ \cr
iff & not $\intparg{\struct{A}}{\assgn^\sndordassgn} \models \psi^F$ \quad (by induction hypothesis) \cr
iff & $\intparg{\struct{A}}{\assgn^\sndordassgn} \models \varphi^F$.
\end{tabular}\bigskip\\
For $\varphi = \psi \lor \chi$: $\intparg{\struct{A}}{\sndordassgn} \models_\weak \varphi$\smallskip\\
\begin{tabular}[b]{ll}
iff & $\intparg{\struct{A}}{\sndordassgn} \models_\weak \psi$ or $\intparg{\struct{A}}{\sndordassgn} \models_\weak \chi$ \cr
iff & $\intparg{\struct{A}}{\assgn^\sndordassgn} \models \psi^F$ or $\intparg{\struct{A}}{\assgn^\sndordassgn} \models \chi^F$ \quad (by induction hypothesis) \cr
iff & $\intparg{\struct{A}}{\assgn^\sndordassgn} \models \varphi^F$.
\end{tabular}\bigskip\\
For $\varphi = \exists x \psi$ ($x \not\in U$!): $\intparg{\struct{A}}{\sndordassgn} \models_\weak \varphi$\smallskip\\
\begin{tabular}[b]{ll}
iff & there is an $a \in A$ such that $\intparg{\struct{A}}{\sndordassgn\sbst{a}{x}} \models_\weak \psi$ \cr
iff &
\begin{minipage}[t]{62ex}
there is an $a \in A$ such that $\intparg{\struct{A}}{\assgn^{\sndordassgn\scriptsbst{a}{x}}} \models \psi^F$\\(by induction hypothesis)
\end{minipage} \cr
iff &
\begin{minipage}[t]{62ex}
there is an $a \in A$ such that $\intparg{\struct{A}}{\assgn^\sndordassgn\sbst{a}{x}} \models \psi^F$\\(since $\assgn^{\sndordassgn\scriptsbst{a}{x}} = \assgn^\sndordassgn\sbst{a}{x}$)
\end{minipage} \cr
iff & $\intparg{\struct{A}}{\assgn^\sndordassgn} \models \varphi^F$.
\end{tabular}\bigskip\\
For $\varphi = \exists V^n_k \psi$: $\intparg{\struct{A}}{\sndordassgn} \models_\weak \varphi$\smallskip\\
\begin{tabular}[b]{ll}
iff & there is a finite $C \subset A^n$ such that $\intparg{\struct{A}}{\sndordassgn\sbst{C}{V^n_k}} \models_\weak \psi$ \cr
iff &
\begin{minipage}[t]{62ex}
$\intparg{\struct{A}}{\sndordassgn\sbst{\emptyset}{V^n_k}} \models_\weak \psi$; or\smallskip\\
$\intparg{\struct{A}}{\sndordassgn\sbst{C}{V^n_k}} \models_\weak \psi$ for some $C \subset A^n$ with $\card{C} = 1$; or\smallskip\\
$\intparg{\struct{A}}{\sndordassgn\sbst{C}{V^n_k}} \models_\weak \psi$ for some $C \subset A^n$ with $\card{C} = 2$; or\smallskip\\
\ldots
\end{minipage} \cr
iff &
\begin{minipage}[t]{62ex}
$\intparg{\struct{A}}{\assgn^{\sndordassgn\scriptsbst{\emptyset}{V^n_k}}} \models \psi^{F \leftarrow \pair{V^n_k}{0}}$; or\smallskip\\
$\intparg{\struct{A}}{\assgn^{\sndordassgn\scriptsbst{C}{V^n_k}}} \models \psi^{F \leftarrow \pair{V^n_k}{1}}$ for some $C \subset A^n$ with $\card{C} = 1$; or\smallskip\\
$\intparg{\struct{A}}{\assgn^{\sndordassgn\scriptsbst{C}{V^n_k}}} \models \psi^{F \leftarrow \pair{V^n_k}{2}}$ for some $C \subset A^n$ with $\card{C} = 2$; or\smallskip\\
\ldots\\(by induction hypothesis)
\end{minipage} \cr
iff &
\begin{minipage}[t]{62ex}
$\intparg{\struct{A}}{\assgn^{\sndordassgn\scriptsbst{\emptyset}{V^n_k}}} \models \psi^{F \leftarrow \pair{V^n_k}{0}}$; or\smallskip\\
$\intparg{\struct{A}}{\assgn^{\sndordassgn\scriptsbst{C}{V^n_k}}} \models \psi^{F \leftarrow \pair{V^n_k}{1}}$ for some $\seqp{a_{1, 1}}{a_{1, n}} \in A$ where $C = \{ \tuple{\seqp{a_{1, 1}}{a_{1, n}}} \}$; or\smallskip\\
$\intparg{\struct{A}}{\assgn^{\sndordassgn\scriptsbst{C}{V^n_k}}} \models \psi^{F \leftarrow \pair{V^n_k}{2}}$ for some $\seqp{a_{1, 1}}{a_{1, n}}, \seqp{a_{2, 1}}{a_{2, n}} \in A$ where $C = \{ \tuple{\seqp{a_{1, 1}}{a_{1, n}}}, \tuple{\seqp{a_{2, 1}}{a_{2, n}}} \}$ and at least one of $\seqp{a_{1, 1} \neq a_{2, 1}}{a_{1, n} \neq a_{2, n}}$ is the case; or\smallskip\\
\ldots\\(by the extensionality of sets)
\end{minipage}
\end{tabular}\\
\begin{tabular}[b]{lll}
iff &
\begin{minipage}[t]{62ex}
$\intparg{\struct{A}}{\assgn^\sndordassgn} \models \psi^{F \leftarrow \pair{V^n_k}{0}}$; or\smallskip\\
$\intparg{\struct{A}}{\assgn^\sndordassgn\sbst{\enump{a_{1, 1}}{a_{1, n}}}{\enump{v^{n, k}_{1, 1}}{v^{n, k}_{1, n}}}} \models \psi^{F \leftarrow \pair{V^n_k}{1}}$ for some $\seqp{a_{1, 1}}{a_{1, n}} \in A$; or\smallskip\\
$\intparg{\struct{A}}{\assgn^\sndordassgn\sbst{\enump{a_{1, 1}}{a_{1, n}}\enump{a_{2, 1}}{a_{2, n}}}{\enump{v^{n, k}_{1, 1}}{v^{n, k}_{1, n}}\enump{v^{n, k}_{2, 1}}{v^{n, k}_{2, n}}}} \models \bigwedge\limits_{1 \leq i < j \leq 2} \parenadj{\bigvee\limits_{1 \leq l \leq n} \neg v^{n, k}_{i, l} \equal v^{n, k}_{j, l}}$\\and $\intparg{\struct{A}}{\assgn^\sndordassgn\sbst{\enump{a_{1, 1}}{a_{1, n}}\enump{a_{2, 1}}{a_{2, n}}}{\enump{v^{n, k}_{1, 1}}{v^{n, k}_{1, n}}\enump{v^{n, k}_{2, 1}}{v^{n, k}_{2, n}}}} \models \psi^{F \leftarrow \pair{V^n_k}{2}}$\\for some $\seqp{a_{1, 1}}{a_{1, n}}, \seqp{a_{2, 1}}{a_{2, n}} \in A$; or\smallskip\\
\ldots\\(by the fact that $\pair{V^n_k}{\sndordassgn(V^n_k)}$ is not in the map $\assgn^\sndordassgn$, and by the Coincidence Lemma for $\infinlog$: none of the variables $v^{n, k}_{i, j}$ occurs free in $\psi^{F \leftarrow \pair{V^n_k}{0}}$; none of the variables $v^{n, k}_{m + 1, j}$ occurs free in $\psi^{F \leftarrow \pair{V^n_k}{m}}$ for $m > 0$)
\end{minipage} \cr
iff &
\begin{minipage}[t]{62ex}
$\intparg{\struct{A}}{\assgn^\sndordassgn} \models \psi^{F \leftarrow \pair{V^n_k}{0}}$; or\smallskip\\
$\intparg{\struct{A}}{\assgn^\sndordassgn\sbst{\enump{a_{1, 1}}{a_{1, n}}}{\enump{v^{n, k}_{1, 1}}{v^{n, k}_{1, n}}}} \models \psi^{F \leftarrow \pair{V^n_k}{1}}$ for some $\seqp{a_{1, 1}}{a_{1, n}} \in A$; or\smallskip\\
$\intparg{\struct{A}}{\assgn^\sndordassgn\sbst{\enump{a_{1, 1}}{a_{1, n}}\enump{a_{2, 1}}{a_{2, n}}}{\enump{v^{n, k}_{1, 1}}{v^{n, k}_{1, n}}\enump{v^{n, k}_{2, 1}}{v^{n, k}_{2, n}}}} \models \psi^{F, 2}_{V^n_k}$\\for some $\seqp{a_{1, 1}}{a_{1, n}}, \seqp{a_{2, 1}}{a_{2, n}} \in A$; or\smallskip\\
\ldots
\end{minipage} \cr
iff & $\intparg{\struct{A}}{\assgn^\sndordassgn} \models \exists^m \psi^F_{V^n_k}$ \quad for some $m \in \nat$ \cr
iff & $\intparg{\struct{A}}{\assgn^\sndordassgn} \models \varphi^F$.
\end{tabular}
\end{proof}
For every $\weaksndordlang{S}$-sentence $\varphi$, we choose $\varphi^\emptyset$ for $\psi$. The desired result immediately follows from the above claim.
%
\item \textbf{Solution to Exercise 2.8}
\begin{enumerate}[(a)]
\item A group $\mathfrak{G}$ is said to be \emph{finitely generated} if there exists a finite subset $G_0$ of the domain $G$ such that every element of $G$ can be expressed as a finite product of the elements in $G_0$ and their inverses. Therefore the class of finitely generated groups can be directly axiomatlzed by the $L^{S_\mathrm{grp}}_{\omega_1\omega}$-sentence $\varphi_a$, the conjuction of the group axioms and the following sentence:
\[
\begin{array}{r}
\bigvee \{ \exists u_0 \ldots \exists u_m \forall x \bigvee \{ \bigvee \{ x \equiv v_0 \ldots v_n | \, \mbox{$v_i \in \{ u_0, u_0^{-1}, \ldots, u_m, u_m^{-1} \}$} \;\;\; \cr
\mbox{for $1 \leq i \leq n$} \} | \, n \in \mathbb{N} \} | \, m \in \mathbb{N} \}.
\end{array}
\]
Notice that in the sentence shown above, the formula inside the outermost braces with prefix $\exists u_0 \ldots \exists u_m$ states that $G_0$ contains \emph{at most} $m + 1$ elements; the case in which $G_0 = \emptyset$ has the same effect as the case in which $G_0 = \{ e \}$, which has been included.
%%
\item A structure isomorphic to $(\mathbb{Z}, <)$ is an ordering in whicn there are smaller and larger elements for every element and furthermore, for every pair of distinct elements there are only \emph{finitely many} elements (possibly none) between them. Hence the class of structures isomorphic to $(\mathbb{Z}, <)$ can be axiomatized by the $L^{ \{ < \} }_{\omega_1\omega}$-sentence $\varphi_b$, which is obtained by taking the conjuction of the sentences in the axioms of the theory of orderings $\Phi_\mathrm{ord}$ and the following three sentences:
\[
\forall x \exists y \, x < y
\]
(There is a larger one for every element),
\[
\forall x \exists y \, y < x
\]
(There is a smaller one for every element), and
\[
\begin{array}{r}
\forall x \forall y ( \exists z (x < z \land z < y) \rightarrow \bigvee \{ \exists v_0 \ldots \exists v_n \forall z ( (x < z \land z < y) \rightarrow \;\;\; \cr
(\bigvee\limits_{0 \leq m \leq n} z \equiv v_m ) ) | \, n \in \mathbb{N} \} )
\end{array}
\]
(For every pair of distinct elements, if there is some element between them, then there are only finitely many such elements).\\
\\
Notice that in the last sentence shown above, the formula inside the braces with prefix $\exists v_0 \ldots v_n$ states that there are \emph{at most} $n + 1$ elements between $x$ and $y$.\\
\\
To show that every structure $(\mathfrak{A}, <^A)$ satisfying $\varphi_b$ is really isomorphic to $(\mathbb{Z}, <)$, one can define the mapping $\pi: \mathbb{Z} \to A$ such that
\begin{enumerate}[(i)]
\item $\pi(z) = a$, where $z \in \mathbb{Z}$ and $a \in A$ are both arbitrarily chosen
%%%
\item $\pi(w) = b$, where $b$ is the $n$th successor (or predecessor) of $a$ and similarly $w$ the $n$th successor (or predecessor) of $z$,
\end{enumerate}
and verify that $\pi$ is an isomorphism.
\end{enumerate}
%
\item \textbf{Solution to Exercise 2.9}
\begin{enumerate}[(a)]
\item The set
\[
\{ \bigvee \{ v_n \equiv v_0 | \, n \in M \} | \, M \subset \mathbb{N} \} \subset L^S_{\omega_1\omega}
\]
is uncountable.
%%
\item Let $S := \{ c_n \, | \, n \in \mathbb{N} \} \cup \{ R \}$, $\mathfrak{B}$ an uncountable $S$-structure such that
\begin{enumerate}[(i)]
\item $B := \mathbb{N} \cup (2^\mathbb{N} \setminus \emptyset)$
%%%
\item For every $n \in \mathbb{N}$, $c_n^\mathfrak{B} := n$
%%%
\item For all $a, b \in B$, $R^\mathfrak{B}ab$ :iff $a \in b$.
\end{enumerate}\ 
\\
Consider the following \emph{uncountable} set of $L^S_{\omega_1\omega}$-sentences:
\[
\Phi := \{ \neg c_i \equiv c_j \, | \, i, j \in \mathbb{N}, i \neq j \} \cup \{ \exists^{=1} d \bigwedge \{ Rc_kd \, | \, k \in M \} \, | \, \emptyset \neq M \subset \mathbb{N} \}.
\]
From this setting it is clear that $\mathfrak{B} \models \Phi$ and there is no countable $S$-structure $\mathfrak{A} \models \Phi$ as all of the $S$-structures satisfying $\Phi$ must be \emph{uncountable}.
\end{enumerate}
\end{enumerate}
%End of Section IX.2-----------------------------------------------------------------------------------------
\
\\
\\
%Section IX.3----------------------------------------------------------------------------------------------
{\large \S3. The System $\qlog$}
\begin{enumerate}[1.]
\item \textbf{More Syntactic Operations on $\qlog$.} (INCOMPLETE) We obtain the $\qlog$-version of $\sbfmlabase$ and $\freebase$ by considering one more case
\[
\sbfmla{\qexist x \varphi} := \{ \qexist x \varphi \} \cup \sbfmla{\varphi}
\]
and
\[
\free{\qexist x \varphi} := \free{\varphi} \setminus \{ x \},
\]
respectively, in each of \reftitle{Definitions II.4.5(b) and II.5.1}.\\
\ \\
Then, we introduce several syntactic operations on $\qlog$ below (assuming a fixed symbol set $S$ has been given):
\begin{enumerate}[(1)]
\item By adding the following clause
\[
\begin{array}{lll}
(\qexist x \psi)^\ast & \colonequals & \qexist x \psi^\ast
\end{array}
\]
to the definition of term-reduced formulas $\psi^\ast$ introduced in \reftitle{VIII.1} we obtain the $\qlog$-version and\medskip\\
\begin{theorem}{Theorem on Term-Reduced Formulas in $\qlog$}
For every $\psi \in \qlang{S}$ there is a logically equivalent, term-reduced formula $\psi^\ast \in \qlang{S}$ with $\free{\psi} = \free{\psi^\ast}$.
\end{theorem}
\begin{proof}
By induction.
\end{proof}
%%
\item Then, let $P \not\in S$ be unary. By adding the following clause
\[
\begin{array}{lll}
\relativize{(\qexist x \psi)}{P} & \colonequals & \qexist x \relativize{\psi}{P}
\end{array}
\]
to the definition of the relativization $\relativize{\psi}{P}$ of $\psi$ to $P$ introduced in \reftitle{VIII.2} we obtain the $\qlog$-version and\medskip\\
\begin{theorem}{Relativization Lemma for $\qlog$}
Let $\struct{A}$ be an $S \cup \{ P \}$-structure such that $P \not\in S$ and $P$ is unary. Suppose the set $P^A \subset A$ is $S$-closed in $\struct{A}$, and moreover $\assgn$ is an assignment in $\struct{A}$ with $\assgn(v_n) \in P^A$ for $n \in \nat$. Then for $\psi \in \qlang{S}$, $\intparg{\substr{P^A}{\struct{A}}}{\assgn} \models \psi$ \quad iff \quad $\intparg{\struct{A}}{\assgn} \models \relativize{\psi}{P}$.
\end{theorem}
\begin{proof}
By induction.
\end{proof}
%%
\item Finally, let $\relational{S}$ and $\relational{\struct{A}}$ in which $\struct{A}$ is an $S$-structure be defined as in \reftitle{VIII.1}. Furthermore, we extend the definition of $\relational{\psi}$ mentioned there to obtain the $\qlog$-version by adding the clause:
\[
\begin{array}{lll}
\relational{(\qexist x \psi)} & \colonequals & \qexist x \relational{\psi};
\end{array}
\]
and likewise extend the definition of $\invrelational{\psi}$ by adding the clause:
\[
\begin{array}{lll}
\invrelational{(\qexist x \psi)} & \colonequals & \qexist x \invrelational{\psi}.
\end{array}
\]
Then as in \reftitle{VIII.1.3}, we obtain\medskip\\
\begin{theorem}{Theorem on the Replacement Operation on $\qlog$}
\emph{\begin{enumerate}[(a)]
\item \emph{For every $\psi \in \qlang{S}$ there is $\relational{\psi} \in \qlang{\relational{S}}$ such that for all $S$-interpretations $\intp = \intparg{\struct{A}}{\assgn}$,\smallskip\\
$\intparg{\struct{A}}{\assgn} \models \psi$ \quad iff \quad $\intparg{\relational{\struct{A}}}{\assgn} \models \relational{\psi}$.}
%%
\item \emph{For every $\psi \in \qlang{\relational{S}}$ there is $\invrelational{\psi} \in \qlang{S}$ such that for all $S$-interpretations $\intp = \intparg{\struct{A}}{\assgn}$,\smallskip\\
$\intparg{\struct{A}}{\assgn} \models \invrelational{\psi}$ \quad iff \quad $\intparg{\relational{\struct{A}}}{\assgn} \models \psi$.}
\end{enumerate}}
\end{theorem}
\begin{proof}
By induction.
\end{proof}

\end{enumerate}
%
\item $^\star$\textbf{Note to the $L_Q^{ \{ < \} }$-Sentence $\varphi_0$ Mentioned in Page 148.} The ordinal structure $\omega_1$ is a model of $\varphi_0$.
%
\item \textbf{Some Derivable Rules.}
\begin{enumerate}[(a)]
\item
\[
\begin{array}{ll}
\Gamma & Qx\varphi \cr\hline
\Gamma & QxQx \varphi
\end{array}
\]
\textit{Justification.}
\[
\begin{array}{lllll}
1. & \Gamma & \ & Qx\varphi & \mbox{premise} \cr
2. & \Gamma & \varphi\frac{u}{x} & Qx\varphi & \mbox{(Ant) applied to 1. with $u$ not} \cr
\ & \ & \ & \ & \mbox{occuring free in $\Gamma \ Qx\varphi$} \cr
3. & \Gamma & \ & (\varphi\frac{u}{x} \rightarrow Qx\varphi) & \mbox{IV.3.6(c) applied to 2.} \cr
4. & \Gamma & \ & \forall x (\varphi \to Qx\varphi) & \mbox{IV.5.5(b2) applied to 3.} \cr
5. & \Gamma & \ & Qx\varphi \rightarrow QxQx\varphi & \mbox{third rule of $\mathcal{L}_Q$ applied to 4.} \cr
6. & \Gamma & \ & QxQx\varphi & \mbox{IV.3.5 applied to 5. and 1.}
\end{array}
\]
%%
\item
\[
\begin{array}{ll}
\Gamma & Qx\varphi \cr\hline
\Gamma & Qx \ x \equiv x
\end{array}
\]
\textit{Justification.}
\[
\begin{array}{lllll}
1. & \Gamma & \ & Qx\varphi & \mbox{premise} \cr
2. & \ & \ & y \equiv y & \mbox{($\equiv$) with $y \neq x$ not occuring free} \cr
\  & \ & \ & \ & \mbox{in $\Gamma\varphi$} \cr
3. & \Gamma & \varphi\frac{y}{x} & y \equiv y & \mbox{(Ant) applied to 2.} \cr
4. & \Gamma & \ & (\varphi\frac{y}{x} \rightarrow y \equiv y) & \mbox{IV.3.6(c) applied to 3.} \cr
5. & \Gamma & \ & \forall x (\varphi \rightarrow x \equiv x) & \mbox{IV.5.5(b2) applied to 4.} \cr
6. & \Gamma & \ & Qx\varphi \rightarrow Qx \ x \equiv x & \mbox{third rule of $\mathcal{L}_Q$ applied to 5.} \cr
7. & \Gamma & \ & Qx \ x \equiv x & \mbox{IV.3.5 applied to 6. and 1.}
\end{array}
\]
\end{enumerate}
\textit{Remark.} In general, the rule
\[
\begin{array}{ll}
\Gamma & QxQy\varphi \cr\hline
\Gamma & QyQx\varphi
\end{array}
\]
is \emph{not} correct, i.e. the quantifier $Q$ is not commutative. For a counterexample, let $(\mathbb{C}, \abs^\mathbb{C})$ be an $\{ \abs \}$-structure with
\[
\mbox{ $\abs^\mathbb{C} (a, b)$ :iff $a$ is the absolute value of $b$ (i.e. $a = |b|$)}.
\]
We see that $(\mathbb{C}, \abs^\mathbb{C}) \models QxQy \ \abs \, x \, y$ but \emph{not} $(\mathbb{C}, \abs^\mathbb{C}) \models QyQx \ \abs \, x \, y$ though obviously $QxQy \ \abs \, x \, y \models QxQy \ \abs \, x \, y$ (we take $\Gamma := QxQy \ \abs \, x \, y$ here).
%
\item \textbf{Note to the Paragraph Immediately after the Introduction of Four Additional Sequent Rules.} That the calculus resulting from adding the four rules allows us to derive exactly the correct sequents does \emph{not} mean we obtain its completeness: Recall that a sequent calculus is complete if for every ``$\Phi$'' and $\varphi$, $\Phi \models \varphi$ implies that there is $\Gamma \subset \Phi$ such that $\Gamma \vdash \varphi$.
%
\item \textbf{Solution to Exercise 3.3.} Let $\varphi$ be an $L^S_Q$-sentence, and $\mathfrak{B}$ an $S$-structure with $\mathfrak{B} \models \varphi$. Since $\varphi$ is finite in length, we may assume without loss of generality that $S$ is finite, hence $L^S_Q$ is countable. We have to show that there is a substructure $\mathfrak{A} \subset \mathfrak{B}$ of cardinality at most $\aleph_1$ such that $\mathfrak{A} \models \varphi$.\newline
\ 
\\As in the proof of Lemma 2.5, for pairwise distinct variables $x_1, \ldots, x_n$ we write $\psi(x_1, \ldots, x_n)$ to denote a formula $\psi$ with $\free(\psi) \subset \{ x_1, \ldots, x_n \}$; $\mathfrak{D} \models \psi[a_1, \ldots, a_n]$ says that $\psi$ holds in $\mathfrak{D}$ if the variables $x_i$ get the assignment $a_i$ for $1 \leq i \leq n$.\newline
\ 
\\We define a sequence $A_0, A_1, A_2,\ldots$ of subsets of $B$ with their cardinalities $\leq \aleph_1$ such that for $m \in \mathbb{N}$:
\begin{enumerate}[(a)]
\item $A_m \subset A_{m + 1}$;
%%
\item for $a_1, \ldots, a_n \in A_m$:
\begin{enumerate}[i)]
\item and for $\psi(x_1, \ldots, x_n, x)$ a subformula of $\varphi$: if $\mathfrak{B} \models Qx \psi[a_1, \ldots, a_n]$, then there are $\aleph_1$ $a$'s in $A_{m + 1}$ such that $\mathfrak{B} \models \psi[a_1, \ldots, a_n, a]$; or else if $\mathfrak{B} \models Qx \psi[a_1, \ldots, a_n]$ fails to hold but $\mathfrak{B} \models \exists x \psi[a_1, \ldots, a_n]$ does, then there is an $a \in A_{m + 1}$ such that $\mathfrak{B} \models \psi[a_1, \ldots, a_n, a]$;
%%%
\item and for $f \in S$ an $n$-ary function symbol, $f^{\mathfrak{B}}(a_1, \ldots, a_n) \in A_{m + 1}$.
\end{enumerate}
\end{enumerate}
\ 
\\
Let $A_0$ be a nonempty subset of $B$ with cardinality $\leq \aleph_1$ which contains $\{ c^\mathfrak{B} | c \in S \}$. Inductively, suppose $A_m$ with cardinality $\leq \aleph_1$ is already defined. To define $A_{m + 1}$, for $a_1, \ldots, a_n \in A_m$:
\begin{enumerate}[i)]
\item and for every subformula $\psi(x_1, \ldots, x_n, x)$: if $\mathfrak{B} \models Qx \psi[a_1, \ldots, a_n]$ then we choose $\aleph_1$ $b$'s in $B$ such that $\mathfrak{B} \models \psi[a_1, \ldots, a_n, b]$; otherwise, if $\mathfrak{B} \models \exists x \psi[a_1, \ldots, a_n]$, then we choose a $b \in B$ such that $\mathfrak{B} \models \psi[a_1, \ldots, a_n, b]$; 
%%%
\item for $f$ an $n$-ary function symbol, choose the element $b = f^{\mathfrak{B}}(a_1, \ldots, a_n) \in B$.
\end{enumerate}
Let $A_m^\prime$ be the set of $b$'s chosen in this way. Since the set of subformulas of $\varphi$ and the set of function symbols are both finite, and the cardinality of $A_m$ is at most $\aleph_1$, the cardinality of $A_m^\prime$ is also at most $\aleph_1$. We set $A_{m + 1} := A_m \cup A_m^\prime$. Then the cardinality of $A_{m + 1}$ is at most $\aleph_1$, and (a) and (b) are satisfied.\newline
\ 
\\Set
\[
A := \bigcup_{m \in \mathbb{N}} A_m.
\]
We have:
\begin{enumerate}[(1)]
\item The cardinality of $A$ is at most $\aleph_1$.
%%
\item $A$ is $S$-closed. By choice of $A_0$, we need only show that $A$ is closed under $f^{\mathfrak{B}}$ for $n$-ary $f \in S$. Let $a_0, \ldots, a_n \in A$. Since the sets $A_m$ form an ascending chain, $a_1, \ldots, a_n$ lie in some $A_k$. According to part ii) of (b), the element $f^{\mathfrak{B}}(a_1, \ldots, a_n)$ lies in $A_{k + 1}$, hence in $A$. 
\end{enumerate}
By (1) and (2), $A$ is the domain of a substructure $\mathfrak{A} \subset \mathfrak{B}$ with cardinality $\leq \aleph_1$. The proof is complete if we can show that $\mathfrak{A} \models \varphi$, which follows from the claim:\newline
\ 
\\For all $a_1, \ldots, a_n \in A$ and all subformulas $\psi(x_1, \ldots, x_n)$,\newline
\phantom{a}\hfill $\mathfrak{A} \models \psi[a_1, \ldots, a_n]$ iff $\mathfrak{B} \models \psi[a_1, \ldots, a_n]$.\hfill\phantom{a}\newline
\ 
\\We prove this claim by induction on $\psi$, but restrict ourselves to the $Q$-case.\newline
\ 
\\Let $\psi(x_1, \ldots, x_n) = Qx \chi(x_1, \ldots, x_n, x)$, and suppose $a_1, \ldots, a_n \in A$.\linebreak[2] If $\mathfrak{A} \models Qx \chi[a_1, \ldots, a_n]$ then we obtain successively:\newline\ \\
\begin{tabular}{l}
$\{ a \in A \ | \ \mathfrak{A} \models \chi[a_1, \ldots, a_n, a] \}$ is uncountable;\ \ \ \ \ \ \ \ \ \ \cr
$\{ a \in A \ | \ \mathfrak{B} \models \chi[a_1, \ldots, a_n, a] \}$ is uncountable\cr
\multicolumn{1}{r}{(by induction hypothesis);}\cr
$\mathfrak{B} \models Qx \chi[a_1, \ldots, a_n]$.
\end{tabular}\newline\ \\
Conversely, if $\mathfrak{B} \models Qx \chi[a_1, \ldots, a_n]$, we choose $k$ such that $a_1, \ldots, a_n \in A_k$, and we obtain successively:\newline
\ \\
\begin{tabular}{l}
$\{ a \in A_{k + 1} \ | \ \mathfrak{B} \models \chi[a_1, \ldots, a_n, a] \}$ is uncountable\ \ \ \ \ \ \ \ \ \ \ \ \ \ \ \ \ \ \ \ \ \cr
\multicolumn{1}{r}{(more precisely, its cardinality is $\aleph_1$, by part i) of (b));}\cr
$\{ a \in A_{k + 1} \ | \ \mathfrak{A} \models \chi[a_1, \ldots, a_n, a] \}$ is uncountable\cr
\multicolumn{1}{r}{(by induction hypothesis);}\cr
$\mathfrak{A} \models Qx \chi[a_1, \ldots, a_n]$.
\end{tabular}\newline
\ 
\\\textit{Remark.} Alternatively, we could consider \emph{all} formulas in $L^S_Q$ instead of only subformulas of $\varphi$ in part i) of the process. And then prove that for $a_1, \ldots, a_n \in A$ and for $\psi(x_1, \ldots, x_n) \in L^S_Q$, $\mathfrak{A} \models \psi[a_1, \ldots, a_n]$ iff $\mathfrak{B} \models \psi[a_1, \ldots, a_n]$. In particular, thus, we have that $\mathfrak{A} \models \varphi$.\footnote{Since all formulas are considered with this alternative method, the ``shape'' of the structure $\mathfrak{A}$ constructed in this way is independent of any particular $\varphi$, a situation different from the original method.}\newline
\ 
\\\textbf{Conjecture.} \emph{The result in this exercise can be generalized to the case of $\mathcal{L}_Q$-formulas: Every satisfiable $\mathcal{L}_Q$-formulas has a model over a domain of cardinality at most $\aleph_1$.}\newline
\ 
\\\textit{Proof.} Let $\varphi \in L^S_Q$, and $(\mathfrak{B}, \beta)$ an $S$-interpretation satisfying $\varphi$. Let $\free(\varphi) \subset \{ v_0, \ldots, v_n \}$. We add new constants $c_0, \ldots, c_n$ to $S$ to comprise a larger symbol set $S^\prime$. Then let $\mathfrak{B}^\prime$ be an $S^\prime$-expansion of $\mathfrak{B}$ with $c_i^{\mathfrak{B}^\prime} := \beta(v_i)$ for $0 \leq i \leq n$. Thus we get successively:\newline
\ \\
\begin{tabular}{l}
$(\mathfrak{B}, \beta) \models \varphi$;\cr
$(\mathfrak{B}^\prime, \beta) \models \varphi$\ \ \ (by the Coincidence Lemma);\cr
$(\mathfrak{B}^\prime, \beta) \models \varphi\frac{c_0 \ldots c_n}{v_0 \ldots v_n}$\ \ \ (by the Substitution Lemma);\cr
$\mathfrak{B}^\prime \models \varphi\frac{c_0 \ldots c_n}{v_0 \ldots v_n}$\ \ \ (by the Coincidence Lemma);\cr
There is an $\mathfrak{A}^\prime \subset \mathfrak{B}^\prime$ with cardinality $\leq \aleph_1$ such that $\mathfrak{A}^\prime \models \varphi\frac{c_0 \ldots c_n}{v_0 \ldots v_n}$\cr
\multicolumn{1}{r}{(apply the result of this exercise);}\cr
There is an $\mathfrak{A}^\prime \subset \mathfrak{B}^\prime$ with cardinality $\leq \aleph_1$  such that $(\mathfrak{A}^\prime, \beta) \models \varphi\frac{c_0 \ldots c_n}{v_0 \ldots v_n}$\cr
\multicolumn{1}{r}{(by the Coincidence Lemma);}\cr
There is an $\mathfrak{A}^\prime \subset \mathfrak{B}^\prime$ with cardinality $\leq \aleph_1$ such that $(\mathfrak{A}^\prime, \beta) \models \varphi$\cr
\multicolumn{1}{r}{(by the Substitution Lemma);}\cr
There is an $\mathfrak{A} (= \mathfrak{A}^\prime |_S) \subset \mathfrak{B}$ with cardinality $\leq \aleph_1$ such that $(\mathfrak{A}, \beta) \models \varphi$\cr
\multicolumn{1}{r}{(by the Coincidence Lemma).}
\end{tabular}\\ \phantom{a}\hfill$\talloblong$
%
\item \textbf{Solution to Exercise 3.4.} The following example shows that the Compactness Theorem does not hold for $\mathcal{L}_Q^\circ$:
\[
\{ \neg Qx \ x \equiv x \} \cup \{ \varphi_{\geq n} | n \geq 2 \}.
\]
As for proving the L\"{o}wenheim-Skolem Theorem:
\begin{quote}
\emph{Every satisfiable and at most countable set $\Phi$ of $L_Q^{\circ, S}$-formulas is satisfiable over a domain which is at most countable};
\end{quote}
We assume as in the proof of VI.1.1 that $S$ is at most countable with no loss of generality. The rest is similar to the method employed in the previous exercise:\footnote{In fact, this method can also be applied to prove the Downward L\"{o}wenheim-Skolem Theorem.} Construct a substructure $\mathfrak{A}$ of $\mathfrak{B}$ where $\mathfrak{B} \models \Phi$, except that here we consider subformulas $\psi$ of the formulas $\varphi \in \Phi$ in the process of construction, and that for $a_1, \ldots, a_n \in A_m$ if $\mathfrak{B} \models Qx \psi [a_1, \ldots, a_n]$ then we add countably many $b \in B$ into $A_m^\prime$ during the progress. As noted in the remark to that exercise, we may alternatively consiser all formulas in $L_Q^{\circ, S}$ also.\footnote{Correspondingly, with this alternative method the ``shape'' of $\mathfrak{A}$ is independent of any particular $\Phi$.}\nolinebreak\hfill$\talloblong$
\end{enumerate}
%End of Section IX.3-------------------------------------------------------------------------------
%End of Chapter IX---------------------------------------------------------------------------------
%%Chapter X-----------------------------------------------------------------------------------------
{\LARGE \bfseries X \\ \\ Limitations of the Formal Method}
\\
\\
\\
%Section X.1---------------------------------------------------------------------------------------
{\large \S1. Decidability and Enumerability}
\begin{enumerate}[1.]
\item \textbf{Solution to Exercise 1.2.}
\begin{enumerate}[(a)]
\item $W \cup W^\prime$. Given $\zeta \in \alphabet^\ast$, decide whether $\zeta \in W$. If so, then the answer is ``yes''. Otherwise we have to further decide whether $\zeta \in W^\prime$, the answer is ``yes'' if $\zeta \in W^\prime$ and ``no'' otherwise.
%%
\item $\alphabet^\ast \setminus W$. Given $\zeta \in \alphabet^\ast$, decide whether $\zeta \in W$. If so the answer is ``no'' and ``yes'' otherwise.
%%
\item $W \cap W^\prime$. From the results we just obtained above, it follows that $\alphabet^\ast \setminus W$ and $\mathcal{A}^\ast \setminus W^\prime$ are both decidable, and thus so is $(\mathcal{A}^\ast \setminus W) \cup (\mathcal{A}^\ast \setminus W^\prime) = \mathcal{A}^\ast \setminus (W \cap W^\prime)$. Since $W \cap W^\prime = \mathcal{A}^\ast \setminus (\mathcal{A}^\ast \setminus (W \cap W^\prime))$, we have that $W \cap W^\prime$ is also decidable.\nolinebreak\hfill$\talloblong$
\end{enumerate}
%
\item \textbf{Solution to Exercise 1.3.}
\begin{enumerate}[(a)]
\item Given $\zeta \in \mathcal{A}_0^\ast$, we treat it as a string over $\mathcal{A}_\infty$. If $\zeta \neq \boxempty$, then we proceed to check the first symbol of $\zeta$ and the substring of $\zeta$ excluding the first symbol, respectively, against the set of variables (cf. II.2.1(a)) and the formation rules for formulas (cf. II.3.2, possibly with the help of II.3.1). Assume that both pass the check, then let us denote the first symbol by $x$, the substring by $\varphi$. We generate the set $\free (\varphi)$\footnote{This set may be implemented as a \emph{list} for practical concerns.} according to II.4.5(a) and II.5.1. Finally, we check whether $x$ occurs in thie set: if it does then the answer is ``yes''. If at any stage stated above the process fails the check, then the answer is ``no''.
%%
\item As in the former, we treat a given $\zeta \in \mathcal{A}_0^\ast$ as a string over $\mathcal{A}_\infty$. Then decide whether $\zeta$ is a formula. If it is some formula $\varphi$, then go on to generate the set $\free (\varphi)$. If $\free (\varphi) = \emptyset$ then $\varphi$ is a sentence and the answer is ``yes''. If at any stage stated above the process fails the check, then the answer is ``no''.\nolinebreak\hfill$\talloblong$
\end{enumerate}
%
\item \textbf{Note to Decision and Enumeration Procedures.} Given an alphabet $\mathcal{A}$, a decision procedure $\mathfrak{P}$ may be seen as a function $f: \mathcal{A}^* \to \mathcal{A}^*$ mapping from strings to strings over $\mathcal{A}$; while this may not be true for enumeration procedures, which may not halt.
%
\item \textbf{Note to the Proof of 1.5.} The set $\mathcal{A}^\ast$ can be enumerated in lexicographic order by successively listing all strings of length $n$ over $\mathcal{A}$ in lexicographic order for all $n \in \mathbb{N}$.
%
\item \textbf{One More Enumerability Property.} If $W$ and $W^\prime$ are both enumerable, then so is $W \cup W^\prime$: List the elements in $W$ and the ones in $W^\prime$ in an interleaving fashion.
%
\item \textbf{Solution to Exercise 1.9.} Given a string $\zeta \in \mathcal{A}^\ast$, we show how to decide whether $\zeta \in W$. First decide whether $\zeta \in U$. If not, then the answer is ``no''; if $\zeta \in U$, then simultaneously run the enumeration procedures for $W$ and $U \setminus W$. Since these two procedures together enumerate all the strings in $U$, $\zeta$ will be on either output list of them. If $\zeta$ appears on the output list produced by the enumeration procedure for $W$, then the answer is ``yes'' and ``no'' otherwise.\nolinebreak\hfill$\talloblong$
%
\item \textbf{Solution to Exercise 1.10.} Suppose $W$ is decidable with respect to $\mathcal{A}_1$, then by definition there is a decision procedure $\mathfrak{P}$ for $W$. We claim that $W$ is decidable with respect to $\mathcal{A}_2$: For strings $\zeta \in \mathcal{A}_2^*$, it is easy to decide whether $\zeta \in \mathcal{A}_1^*$; if $\zeta \in \mathcal{A}_1^*$, then invoke $\mathfrak{P}$ to decide whether $\zeta \in W$.\\
\ \\
Conversely, if $W$ is decidable with respect to $\mathcal{A}_2$, then it is decidable with respect to $\mathcal{A}_1$: Since $\mathcal{A}_1 \subset \mathcal{A}_2$, a decision procedure with respect to $\mathcal{A}_2$ (that is, one which expects input from $\mathcal{A}_2^*$) of course accepts input from $\mathcal{A}_1^*$.\\
\ \\
On the other hand, given $\mathcal{A}_1 \subset \mathcal{A}_2$, a set $W \subset \mathcal{A}_1^*$ is obviously a subset both of $\mathcal{A}_1^*$ and of $\mathcal{A}_2^*$. Hence an enumeration procedure for $W$ with respect to $\mathcal{A}_1$ (that is, one which regards output as a subset of $\mathcal{A}_1^*$) is one with resepct to $\mathcal{A}_2$; conversely, an enumeration procedure for $W$ with respect to $\mathcal{A}_2$ is one with respect to $\mathcal{A}_1$.\nolinebreak\hfill$\talloblong$
%
\item \textbf{Solution to Exercise 1.11.} Define for every $n \in \mathbb{Z}^+$ the collection $P_n$ of polynomials which take the form:
\[
\sum_{i = 0}^{m} \mbox{term}_i,
\]
and satisfy:
\begin{enumerate}[1)]
\item $0 \leq m < n$. That is, the number of terms therein is lowerbounded by $1$ and upperbounded by $n$.
%%
\item The variables occurring in each term are indexed at most $n - 1$.
%%
\item Each term has \emph{degree} at most $n$. That is, there are, possibly with repetitions, at most $n$ variables occurring in each term.
%%
\item For $i \neq j$, $\mbox{term}_i$ and $\mbox{term}_j$ differ by at least one variable.
%%
\item The coefficients are integers ranging from $-n$ to $n$, and are nonzero. A constant term (that is, one with no variables) is itself a coefficient.
%%
\item There is at least one variable that occurs. That is, we do not concern constant polynomials here.
\end{enumerate}
For example,
\[
3 x_0^2 x_1 -3 x_1^2 -4
\]
and
\[
x_0^3 - 3 x_2^2
\]
are such polynomials in $P_3$, while
\[
\begin{array}{rl}
x_0^3 + x_1^2 - 2 x_2^2 + 3 & \mbox{(violating 1))}, \cr
x_3^3 + x_1^2 - 1 & \mbox{(violating 2))}, \cr
x_2^4 + x_0^2 + 2 x_0^2 & \mbox{(violating 3) and 4))}, \cr
-4 x_1^2 - 7 x_0 + 2 & \mbox{(violating 5))}, \mbox{ and} \cr
2 & \mbox{(violating 6))}
\end{array}
\]
are not.\\
\ \\
From this setting it is conceivable that each of the polynomials we shall concern in this exercise falls into $P_n$ for some $n$. Furthermore, notice that $P_n \subset P_{n + 1}$ for all $n \in \mathbb{Z}^+$. We say that a polynomial $p$ has \emph{degree} $k$ if the maximum degree among those of the terms is $k$.\\
\ \\
Below we will provide a procedure for each part.
\begin{enumerate}[(a)]
\item For every given $n \in \mathbb{Z}^+$, and for every polynomial $p \in P_n$, check whether there is an $n$-tuple from $\{-n, \ldots, -1, 1, \ldots, n\}^n$ such that it is a root of $p$. If so, list $p$ in the output list.
%%
\item Let us agree that the polynomials $p$ of degree $n$ ($n \in \mathbb{Z}^+$) with one variable ($x$ here) take the form:
\[
p(x) = \sum_{k = 1}^n a_k x^k + a_0,
\]
where for $0 \leq k \leq n$, $a_k \in \mathbb{Z}^+$, and in particular, $a_n \neq 0$.\\
\ \\
Let
\[
m := \max \left\{ b_k \ \left| \  0 \leq k \leq n \right\}\right.,
\]
where for $0 \leq k \leq n$,
\[
b_k := \begin{cases}
1, & \mbox{if \(k = 0\)}; \cr
\left\lceil \,\sqrt[k]{ n \left| \displaystyle\frac{a_{n - k}}{a_n} \right| }\, \right\rceil, & \mbox{otherwise}.
\end{cases}
\]
Then we have\\
(1) \hfill $m \geq 1$ \hfill \phantom{(1)}\\
and\\
(2) \hfill $\displaystyle m \geq \left\lceil \,\sqrt[k]{ n \left| \displaystyle\frac{a_{n - k}}{a_n} \right| }\, \right\rceil$ \hfill \ \phantom{(2)}\\
for $1 \leq k \leq n$.\\
\ \\
Let us consider the two cases below concerning the sign of $a_n$:
\begin{enumerate}[1)]
\item $a_n \geq 1$. For $x \geq m + 1$,
\[
\begin{array}{lll}
p(x) & = & \displaystyle\sum_{k = 1}^n a_k x^k + a_0 \cr
\    & \geq & \displaystyle\sum_{k = 1}^n a_k (m + 1)^k + a_0 \cr
\multicolumn{3}{r}{\mbox{(note that $m + 1 > 0$ by (1))}} \cr
\    & = & \displaystyle\sum_{k = 1}^n \left[ \displaystyle\frac{1}{n} a_n (m + 1)^k + a_{n - k} \right] (m + 1)^{n - k} \cr
\    & \geq & \displaystyle\sum_{k = 1}^n \left[ \displaystyle\frac{1}{n} a_n (m^k + 1) + a_{n - k} \right] (m + 1)^{n - k} \cr
\multicolumn{3}{r}{\mbox{(by (1))}} \cr
\    & \geq & \displaystyle\sum_{k = 1}^n \left[ \displaystyle\frac{1}{n} a_n \left(\left\lceil \,\sqrt[k]{ n \left| \displaystyle\frac{a_{n - k}}{a_n} \right| }\, \right\rceil^k + 1\right) + a_{n - k} \right] (m + 1)^{n - k} \cr
\multicolumn{3}{r}{\mbox{(by (2))}} \cr
\    & \geq & \displaystyle\sum_{k = 1}^n \left[ \displaystyle\frac{1}{n} a_n \left(\left( \,\sqrt[k]{ n \left| \displaystyle\frac{a_{n - k}}{a_n} \right| }\, \right)^k + 1\right) + a_{n - k} \right] (m + 1)^{n - k} \cr
\    & = & \displaystyle\sum_{k = 1}^n \left[ \displaystyle\frac{1}{n} a_n \left( n \left| \displaystyle\frac{a_{n - k}}{a_n} \right| + 1\right) + a_{n - k} \right] (m + 1)^{n - k} \cr
\    & = & \displaystyle\sum_{k = 1}^n \left[ \left| a_{n - k} \right| + a_{n - k} + \displaystyle\frac{1}{n} a_n \right] (m + 1)^{n - k} \cr
\multicolumn{3}{r}{\mbox{(since $a_n > 0$)}} \cr
\    & \geq & \displaystyle\sum_{k = 1}^n \displaystyle\frac{1}{n} a_n (m + 1)^{n - k} \cr
\multicolumn{3}{r}{\mbox{(since $\left|a_{n - k}\right| + a_{n - k} \geq 0$)}} \cr
\ & > & 0.
\end{array}
\]
Similarly, for $x \leq -m - 1$, $p(x) > 0$ if $n$ is even and $p(x) < 0$ if $n$ is odd.
%%
\item $a_n \leq -1$. By applying the result in the above case to $-p(x)$, we immediately obtain for $x \geq m + 1$, $p(x) < 0$ and for $x \leq -m - 1$, $p(x) < 0$ if $n$ is even and $p(x) > 0$ is $n$ is odd.
\end{enumerate}
\ \\
It follows that all integer roots of $p(x) = 0$ (if any) lie in the interval $[-m, m]$. And it is easy to check for each integer $i$ with $-m \leq i \leq m$ whether $p(i) = 0$.\nolinebreak\hfill$\talloblong$
\end{enumerate}
%
\item \textbf{Solution to Exercise 1.12.} Below we show that (i) implies (ii) in (a), that (ii) implies (iii) in (b), and finally that (iii) implies (i) in (c), hence the equivalence between them:
\begin{enumerate}[(a)]
\item We provide an enumeration procedure for $\{ \zeta\# f(\zeta) \,|\, \zeta \in \mathcal{A}^\ast \}$: Produce all strings over $\mathcal{A}$ in alphabetic order. For each string $\zeta \in \mathcal{A}^\ast$, compute $f(\zeta)$, and then print the string $\zeta\#f(\zeta)$ in the output list.
%%
\item We show how to decide the set $\{ \zeta\#f(\zeta) \,|\, \zeta \in \mathcal{A}^\ast \}$ given an enumeration procedure for it. Given a string $\chi \in (\mathcal{A} \cup \mathcal{B} \cup \{ \# \})^\ast$, first decide whether $\chi = \zeta\#\xi$ for some $\zeta \in \mathcal{A}^\ast$ and $\xi \in \mathcal{B}^\ast$. If not, then answer is simply ``no''; otherwise, run the enumeration procedure for $\{ \zeta\#f(\zeta) \,|\, \zeta \in \mathcal{A}^\ast \}$. Search the output list for \emph{the} string $\theta$ prefixed by $\zeta\#$ (note there is exactly one such $\theta$ on the list since $f$ is a function). If $\theta = \chi$, then the answer is ``yes''; if at any stage stated above the check fails, then the answer is ``no''.
%%
\item Provided that the set $\{ \zeta\#f(\zeta) \,|\, \zeta \in \mathcal{A}^\ast \}$ is decidable, we show how to compute $f(\zeta)$ for each $\zeta \in \mathcal{A}^\ast$: Decide for each $\xi \in \mathcal{B}^\ast$ in alphabetic order whether $\zeta\#\xi$ is in the set just mentioned. If the answer is ``yes'', then we are done, $\xi = f(\zeta)$. If the answer is ``no'', then go on to check the next to $\xi$ until the answer ``yes'' is obtained.\nolinebreak\hfill$\talloblong$
\end{enumerate}
\end{enumerate}
%End of Section X.1--------------------------------------------------------------------------------
\
\\
\\
%Section X.2---------------------------------------------------------------------------------------
{\large \S2. Register Machines}
\begin{enumerate}[1.]
\item \textbf{The Program $\mathrm{P}$ as an Exercise in Page 160 Suggested by the Authors.} Below we provide an instance:
\[
\begin{array}{rl}
0 & \IF \ \R_0 = \boxempty \ \THEN \ 7 \ \ELSE \ 7 \ \OR \ 7 \ \OR \ 1 \cr
1 & \PS{0}{a_2} \cr
2 & \IF \ \R_0 = \boxempty \ \THEN \ 7 \ \ELSE \ 3 \ \OR \ 7 \ \OR \ 7 \cr
3 & \PS{0}{a_0} \cr
4 & \IF \ \R_0 = \boxempty \ \THEN \ 7 \ \ELSE \ 5 \ \OR \ 7 \ \OR \ 7 \cr
5 & \PS{0}{a_0} \cr
6 & \IF \ \R_0 = \boxempty \ \THEN \ 8 \ \ELSE \ 7 \ \OR \ 7 \ \OR \ 7 \cr
7 & \GOTO \ 7 \cr
8 & \HALT
\end{array}
\]
%
\item \textbf{Note to the Paragraph Below Defintion 2.6.} The program
\[
\begin{array}{rl}
0 & \PRINT \cr
1 & \HALT
\end{array}
\]
is one that decides $\{ \Box \}$.\\
\ \\
Also note that a program that enumerates a finite set does not necessarily halt. For example, the program referred to in text
\[
\begin{array}{rl}
0 & \PA{1}{a_0} \cr
1 & \GOTO \ 0 \cr
2 & \HALT
\end{array}
\]
does not halt although it enumerates $\emptyset$.
%
\item \textbf{Corollary to 2.8.} Since for $\varphi \in L_0^{S_\infty}$,
\begin{center}
not $\sat \varphi$ iff $\neg \varphi$ is valid,
\end{center}
it is easy to provide a program that enumerates the set of unsatisfiable $S_\infty$-sentences, and hence that set is R-enumerable.
%
\item \textbf{Solution to Exercise 2.9.} For intuitive treatment (a \emph{procedure}) of this exercise, refer to \textbf{Solution to Exercise 1.2}. As for the precise counterpart (a \emph{program}), see below.\\
\ \\
Let $\mathcal{A} = \{ a_0, \ldots, a_r \}$, and let $\p_W$ and $\p_{W^\prime}$ be programs that decide $W$ and $W^\prime$, respectively. Furthermore, assume that there are $L + 2$ and $L^\prime + 2$ instructions, respectively, in $\p_W$ and $\p_{W^\prime}$,\footnote{One for the print-instruction, and one for halt-instruction.} where $L \geq 0$ and $L^\prime \geq 0$. In addition, there is exactly one print-instruction in each of $\p_W$ and $\p_{W^\prime}$.\\
\ \\
We prove $\mathcal{A}^\ast \setminus W$, $W \cap W^\prime$ and $W \cup W^\prime$ are all R-decidable by showing how to construct the programs $\p_{\mathcal{A}^\ast \setminus W}$, $\p_{W \cap W^\prime}$ and $\p_{W \cup W^\prime}$, that decide them, respectively.
\begin{enumerate}[(a)]
\item $\p_{\mathcal{A}^\ast \setminus W}$. This program is obtained from $\p_W$ by replacing the print-instruction
\[
\begin{array}{rl}
L_0 & \PRINT
\end{array}
\]
by
\[
\begin{array}{rl}
L_0 & \PI{0}{L_0 + 1}{L_0 + 3}{L_0 + 3} \cr
L_0 + 1 & \PA{0}{a_0} \cr
L_0 + 2 & \GOTO \ L_0 + r + 5 \cr
L_0 + 3 & \PS{0}{a_0} \cr
\multicolumn{2}{c}{\vdots} \cr
L_0 + r + 3 & \PS{0}{a_r} \cr
L_0 + r + 4 & \IF \ \R_0 \ \THEN \ L_0 + r + 5 \ \ELSE \ L_0 + 3 \ldots \cr
\ & \OR \ L_0 + 3 \cr
L_0 + r + 5 & \PRINT
\end{array}
\]
and increase all labels in $\p_W$ that are greater than $L_0$ by the amount of $r + 5$.\\
\ \\
(If in $\p_W$ the print-instruction is to be executed with $\R_0 = \Box$, which means that $\zeta \in W$, then in $\p_{\mathcal{A}^\ast \setminus W}$ we add $a_0$ to $\R_0$; otherwise, $\zeta \not\in W$, then we clear $\R_0$ before the content in it is printed out.)
%%
\item $\p_{W \cap W^\prime}$. Assume that the registers used in $\p_W$ are among $\R_0, \ldots, \R_s$. Then there are $L + L^\prime + (r + 2)s + 11r + 17$ instructions in $\p_{W \cap W^\prime}$, where
\begin{enumerate}[(1)]
\item Instructions $0$ - $7r + 8$ are
\[
\begin{array}{rl}
0 & \IF \ \R_0 = \Box \ \THEN \ 4r + 5 \ \ELSE \ 1 \cr
\ & \OR \ldots \ 4k + 1 \ldots \ \OR \ 4r + 1 \cr
1 & \PS{0}{a_0} \cr
2 & \PA{1}{a_0} \cr
3 & \PA{s + 1}{a_0} \cr
4 & \GOTO \ 0 \cr
\multicolumn{2}{c}{\vdots} \cr
4r + 1 & \PS{0}{a_r} \cr
4r + 2 & \PA{1}{a_r} \cr
4r + 3 & \PA{s + 1}{a_r} \cr
4r + 4 & \GOTO \ 0 \cr
4r + 5 & \IF \ \R_1 = \Box \ \THEN \ 7r + 9 \ \ELSE \ 4r + 6 \cr
\ & \OR \ldots \ 4r + 3k + 6 \ldots \ \OR \ 7r + 6 \cr
4r + 6 & \PA{0}{a_0} \cr
4r + 7 & \PS{1}{a_0} \cr
4r + 8 & \GOTO \ 4r + 5 \cr
\multicolumn{2}{c}{\vdots} \cr
7r + 6 & \PA{0}{a_r} \cr
7r + 7 & \PS{1}{a_r} \cr
7r + 8 & \GOTO \ 4r + 5 \cr
\end{array}
\]
(Copy the input $\zeta$ in $\R_0$ to $\R_{s + 1}$ (in reverse order), using $\R_1$ as a temporary storage. Upon completion, $\R_0$ contains $\zeta$, whereas $\R_{s + 1}$ contains $\zeta$ in reverse order.)
%%%
\item Instructions $7r + 9$ - $L + 7r + 9$ are derived from $\p_W$ with the last instruction
\[
\begin{array}{rl}
L + 1 & \HALT
\end{array}
\]
left out. More precisely, for $0 \leq l \leq L$, instruction at label $l + 7r + 9$ is the one at label $l$ in $\p_W$ with all labels therein (if any) increased by $7r + 9$. Suppose the print-instruction in $\p_W$ is at label $L_0$:
\[
\begin{array}{rl}
L_0 & \PRINT
\end{array}
\]
Then the instruction at label $L_0 + 7r + 9$ is replaced by
\[
\begin{array}{rl}
L_0 & \PI{0}{L + 7r + 10}{L_1}{L_1}
\end{array}
\]
(If in $\p_W$ the content in $\R_0$ is to be printed out, then in $\p_{W \cap W^\prime}$ we jump to the print-instruction (see part (4)) if $\R_0 \neq \Box$ (which means that $\zeta \not\in W$), and proceed to check whether $\zeta \in W^\prime$ otherwise.)
%%%
\item Instructions $L + 7r + 10$ - $L + (r + 2)s + 11r + 15$ are
\[
\begin{array}{rl}
L + 7r + 10 & \PS{0}{a_0} \cr
\multicolumn{2}{c}{\vdots} \cr
L + 8r + 10 & \PS{0}{a_r} \cr
L + 8r + 11 & \IF \ \R_0 = \Box \ \THEN \ L + 8r + 12 \cr
\ & \ELSE \ L + 7r + 10 \ldots \ \OR \ L + 7r + 10 \cr
\multicolumn{2}{c}{\vdots} \cr
\multicolumn{2}{c}{\vdots} \cr
L + (r + 2)s + 7r + 10 & \PS{s}{a_0} \cr
\multicolumn{2}{c}{\vdots} \cr
L + (r + 2)s + 8r + 10 & \PS{s}{a_r} \cr
L + (r + 2)s + 8r + 11 & \IF \ \R_s = \Box \cr
\ & \THEN \ L + (r + 2)s + 8r + 12 \cr
\ & \ELSE \ L + (r + 2)s + 7r + 10 \ldots \cr
\ & \OR \ L + (r + 2)s + 7r + 10 \cr
L + (r + 2)s + 8r + 12 & \IF \ \R_{s + 1} = \Box \cr
\ & \THEN \ L + (r + 2)s + 11r + 16 \cr
\ & \ELSE \ L + (r + 2)s + 8r + 13 \cr
\ & \OR \ldots \ L + (r + 2)s + 8r + 3k + 13 \ldots \cr
\ & \OR \ L + (r + 2)s + 11r + 13 \cr
L + (r + 2)s + 8r + 13 & \PS{s + 1}{a_0} \cr
L + (r + 2)s + 8r + 14 & \PA{0}{a_0} \cr
L + (r + 2)s + 8r + 15 & \GOTO \ L + (r + 2)s + 8r + 12 \cr
\multicolumn{2}{c}{\vdots} \cr
L + (r + 2)s + 11r + 13 & \PS{s + 1}{a_r} \cr
L + (r + 2)s + 11r + 14 & \PA{0}{a_r} \cr
L + (r + 2)s + 11r + 15 & \GOTO \ L + (r + 2)s + 8r + 12
\end{array}
\]
(On entrance into this part, we have $\zeta \in W$. Therefore, clear $\R_0, \ldots, \R_s$, then recover $\zeta$ in $\R_0$ using $\R_{s + 1}$, and check whether $\zeta \in W^\prime$ in the next part. Note that instructions $L + 7r + 10$ - $L + 8r + 11$ are actually redundant since $\R_0$ is already empty, we put them here for compatibility issues.)
%%%
\item Instructions $L + (r + 2)s + 11r + 16$ - $L + L^\prime + (r + 2)s + 11r + 16$ are derived from $\p_{W^\prime}$ by increasing all labels in it by $L + (r + 2)s + 11r + 16$. Assume the print-instruction is at label $L_1$.\\
\ \\
(This final part checks whether $\zeta \in W^\prime$ and prints the contents in $\R_0$ as output accordingly.)
\end{enumerate}
%%
\item $W \cup W^\prime$. It is obtained from the previous program by replacing the instruction at label $L_0$ with
\[
\begin{array}{rl}
L_0 & \PI{0}{L_1}{L + 7r + 10}{L + 7r + 10}
\end{array}
\]
(Check whether $\zeta \in W^\prime$ only if $\zeta \not\in W$.)\nolinebreak\hfill$\talloblong$\\
\ \\
\textit{Remark.} Alternatively, this part can be argued in the same way as is done in Exercise 1.2.
\end{enumerate}
%
\item \textbf{Solution to Exercise 2.10.} Assume that $\mathcal{A} = \{ a_0, \ldots, a_r \}$.
\begin{enumerate}[(a)]
\item Intuitively, we can enumerate the set $\mathcal{A}^\ast$ by successively enumerating all strings of length $n$ (in lexicographic order) over $\mathcal{A}$ for all $n \in \mathbb{N}$.\\
\ \\
Below is a program that enumerates $\mathcal{A}^\ast$:
\[
\begin{array}{rl}
0 & \PI{1}{4}{1}{1} \cr
1 & \PS{1}{a_0} \cr
2 & \PA{0}{a_0} \cr
3 & \GOTO \ 0 \cr
4 & \PRINT \cr
5 & \PII{0}{6}{8}{3k + 8}{3r + 8} \cr
6 & \PA{0}{a_0} \cr
7 & \GOTO \ 0 \cr
8 & \PS{0}{a_0} \cr
9 & \PA{0}{a_1} \cr
10 & \GOTO \ 0 \cr
\multicolumn{2}{c}{\vdots} \cr
3k + 8 & \PS{0}{a_k} \cr
3k + 9 & \PA{0}{a_{k + 1}} \cr
3k + 10 & \GOTO \ 0 \cr
\multicolumn{2}{c}{\vdots} \cr
3r + 8 & \PS{0}{a_r} \cr
3r + 9 & \PA{1}{a_0} \cr
3r + 10 & \GOTO \ 5 \cr
3r + 11 & \HALT
\end{array}
\]
In case that $r = 0$, there are $11$ instructions in this program, where instructions $8$ - $10$ are
\[
\begin{array}{rl}
8 & \PS{0}{a_0} \cr
9 & \PA{1}{a_0} \cr
10 & \GOTO \ 5 \cr
\end{array}
\]
\ \\
This program enumerates $\mathcal{A}^\ast$ by listing
\[
\Box, a_0, \ldots, a_r, a_0a_0, \ldots, a_0a_r, \ldots, a_ra_0, \ldots, a_ra_r, a_0a_0a_0, \ldots
\]
\ \\
Instructions $0$ - $3$ serve to append the content in $\R_1$ to $\R_0$. Instruction $4$ prints out the content in $\R_0$. Instruction $5$ checks whether $\R_0$ is empty:
\begin{enumerate}[(1)]
\item If $\R_0$ is empty, then an $a_0$ is added to it at instruction $6$, and subsequently all strings of length $n + 1$ are printed out, assuming the content in $\R_1$ is $\underbrace{a_0 \ldots a_0}_{n\mbox{\scriptsize-times}}$, at label $6$. (So, after jumping at label $7$ and later executing instructions $0$ - $3$, the content in $\R_0$ is $\underbrace{a_0 \ldots a_0}_{(n + 1)\mbox{\scriptsize-times}}$.)
%%%
\item If otherwise $\R_0$ terminates with $a_r$, that is, the content in $\R_0$ is $\zeta a_r$ for some $\zeta \in \mathcal{A}^\ast$, then $\R_0$ becomes $\zeta$ and an $a_0$ is added to $\R_1$ ($\R_1$ serves to record the suffix appened later to $\R_0$, hence the instructions $0$ - $3$), at instructions $3r + 8$ - $3r + 9$. The program then jumps at label $3r + 10$ to label $5$ for further checks.
%%%
\item If $\R_0$ terminates with some $a_i$ (that is, the content in $\R_0$ is $\zeta a_i$ for some $\zeta \in \mathcal{A}^\ast$) with $i < r$, then it will terminate with $a_{i + 1}$ (the content in $\R_0$ becomes $\zeta a_{i + 1}$) after executing instructions $3i + 8$ - $3i + 9$, and then the program jumps at label $3i + 10$ to instruction $0$ to append the content in $\R_1$ to $\R_0$.
\end{enumerate}
Note that initially all registers are empty. Furthermore, after executing instructions $0$ - $3$, $\R_0$ must either be empty (the first time they are executed) or terminate with $\underbrace{a_0 \ldots a_0}_{\mbox{\scriptsize\begin{math}n\end{math}-times}}$ where $n \in \mathbb{Z}^+$, while $\R_1$ becomes empty.
%%
\item For an intuitive treatment, see Theorem 1.8. Here we provide programs for this argument.\\
\ \\
Suppose $W$ is R-decidable, i.e. there is a program $\p$ that decides it, then we show below how to construct the programs $\p_W$ and $\p_{\mathcal{A}^\ast \setminus W}$, respectively, that enumerate $W$ and $\mathcal{A}^\ast \setminus W$. For convenience, let us assume that there is exactly one print-instruction in $\p$, that there are $L + 2$ instructions therein with $L \geq 0$, and that the registers used in it are among $\R_0, \ldots, \R_s$ with $s \geq 1$.\\
\ \\
\emph{The program $\p_W$.} There are $L + (r + 2)s + 13r + 32$ instructions in it, where
\begin{enumerate}[(1)]
\item Instructions $0$ - $3r + 10$ are derived from the program in part (a), with the last instruction
\[
\begin{array}{rl}
3r + 11 & \HALT
\end{array}
\]
left out, and with the instruction at label 4 replaced by
\[
\begin{array}{rl}
4 & \GOTO \ 3r + 11
\end{array}
\]
(Produce a string $\zeta \in \mathcal{A}^\ast$ in $\R_0$.)
%%%
\item Instructions $3r + 11$ - $10r + 19$ are
\[
\begin{array}{rl}
3r + 11 & \IF \ \R_0 = \Box \ \THEN \ 7r + 16 \ \ELSE \ 3r + 12 \cr
\ & \OR \ldots \ 3r + 4k + 12 \ldots \ \OR \ 7r + 12 \cr
3r + 12 & \PS{0}{a_0} \cr
3r + 13 & \PA{1}{a_0} \cr
3r + 14 & \PA{s + 1}{a_0} \cr
3r + 15 & \GOTO \ 3r + 11 \cr
\multicolumn{2}{c}{\vdots} \cr
7r + 12 & \PS{0}{a_r} \cr
7r + 13 & \PA{1}{a_r} \cr
7r + 14 & \PA{s + 1}{a_r} \cr
7r + 15 & \GOTO \ 3r + 11 \cr
7r + 16 & \IF \ \R_1 = \Box \ \THEN \ 10r + 20 \ \ELSE \ 7r + 17 \cr
\ & \OR \ldots \ 7r + 3k + 17 \ldots \ \OR \ 10r + 17 \cr
7r + 17 & \PA{0}{a_0} \cr
7r + 18 & \PS{1}{a_0} \cr
7r + 19 & \GOTO \ 7r + 16 \cr
\multicolumn{2}{c}{\vdots} \cr
10r + 17 & \PA{0}{a_r} \cr
10r + 18 & \PS{1}{a_r} \cr
10r + 19 & \GOTO \ 7r + 16 \cr
\end{array}
\]
(Copy the content in $\R_0$ to $\R_{s + 1}$ (in reverse order), using $\R_1$ as a temporary repository. (Note that on entrance into this part, $\R_1 = \Box$.) Upon completion, $\R_0$ contains a string $\zeta$ produced previously, while $\R_{s + 1}$ contains $\zeta$ in reverse order and all other registers are empty.)
%%%
\item Instructions $10r + 20$ - $L + 10r + 20$ are derived from $\p$, with the last instruction
\[
\begin{array}{rl}
L + 1 & \HALT
\end{array}
\]
left out, and with all labels increased by $10r + 20$. Suppose in the resulting program fragment the print-instruction is located at label $L_0$:
\[
\begin{array}{rl}
L_0 & \PRINT
\end{array}
\]
replace it with
\[
\begin{array}{rl}
L_0 & \GOTO \ L + 10r + 21
\end{array}
\]
(Apply $\p$ to $\zeta$, the content in $\R_0$. Whenever in $\p$ the print-instruction is to be issued, check whether $\R_0 = \Box$, i.e. whether $\zeta \in W$, in the next part.)
%%%
\item Instructions $L + 10r + 21$ - $L + (r + 2)s + 13r + 31$ are
\[
\begin{array}{rl}
L + 10r + 21 & \IF \ \R_0 = \Box \ \THEN \ L + 10r + 23 \cr
\ & \ELSE \ L + 10r + 22 \ldots \cr
\ & \OR \ L + 10r + 22 \cr
L + 10r + 22 & \PA{s + 2}{a_0} \cr
L + 10r + 23 & \IF \ \R_{s + 1} = \Box \ \THEN \ L + 13r + 27 \cr
\ & \ELSE \ L + 10r + 24 \cr
\ & \OR \ldots \ L + 10r + 3k + 24 \ldots \cr
\ & \OR \ L + 13r + 24 \cr
L + 10r + 24 & \PS{s + 1}{a_0} \cr
L + 10r + 25 & \PA{0}{a_0} \cr
L + 10r + 26 & \GOTO \ L + 10r + 23 \cr
\multicolumn{2}{c}{\vdots} \cr
L + 13r + 24 & \PS{s + 1}{a_r} \cr
L + 13r + 25 & \PA{0}{a_r} \cr
L + 13r + 26 & \GOTO \ L + 10r + 23 \cr
L + 13r + 27 & \IF \ \R_{s + 2} = \Box \ \THEN \ L + 13r + 28 \cr
\ & \ELSE \ L + 13r + 29 \ldots \cr
\ & \OR \ L + 13r + 29 \cr
L + 13r + 28 & \PRINT \cr
L + 13r + 29 & \PS{s + 2}{a_0} \cr
L + 13r + 30 & \PS{1}{a_0} \cr
\multicolumn{2}{c}{\vdots} \cr
L + 14r + 30 & \PS{1}{a_r} \cr
L + 14r + 31 & \IF \ \R_1 = \Box \ \THEN \ L + 14r + 32 \cr
\ & \ELSE \ L + 13r + 30 \ldots \cr
\ & \OR \ L + 13r + 30 \cr
\multicolumn{2}{c}{\vdots} \cr
\multicolumn{2}{c}{\vdots} \cr
L + (r + 2)s + 12r + 28 & \PS{s}{a_0} \cr
\multicolumn{2}{c}{\vdots} \cr
L + (r + 2)s + 13r + 28 & \PS{s}{a_r} \cr
L + (r + 2)s + 13r + 29 & \IF \ \R_s = \Box \cr
\ & \THEN \ L + (r + 2)s + 13r + 30 \cr
\ & \ELSE \ L + (r + 2)s + 12r + 28 \ldots \cr
\ & \OR \ L + (r + 2)s + 12r + 28 \cr
L + (r + 2)s + 13r + 30 & \GOTO \ 5 \cr
L + (r + 2)s + 13r + 31 & \HALT
\end{array}
\]
(At labels $L + 10r + 21$ - $L + 10r + 22$, $\R_{s + 2}$ is used to indicate whether $\zeta \in W$: the content is $\Box$ if $\zeta \in W$, and is $a_0$ otherwise. And then $\R_0$ is recovered from $\R_{s + 1}$. If $\zeta \in W$, then it is printed out, at label $L + 13r + 28$. Registers $\R_1, \ldots, \R_{s + 2}$ are all cleared before going back to label 5 to begin the next iteration.)
%%%
\end{enumerate}
\ \\
\emph{The program $\p_{\mathcal{A}^\ast \setminus W}$.} It is obtained from $\p_W$ by replacing the instruction at label $L + 13r + 27$ with
\[
\begin{array}{rl}
L + 13r + 27 & \IF \ \R_{s + 2} = \Box \ \THEN \ L + 13r + 29 \cr
\ & \ELSE \ L + 13r + 28 \ldots \cr
\ & \OR \ L + 13r + 28 \cr
\end{array}
\]
(Interchange the labels corresponding to cases whether $\R_{s + 2} = \Box$.)\\
\ \\
On the other hand, given programs $\p_W$ and $\p_{\mathcal{A}^\ast \setminus W}$ that enumerate $W$ and $\mathcal{A}^\ast \setminus W$, respectively, we show how to construct the program $\p$, that decides $W$.\\
\ \\
The main idea is that, we set a timer for each computation of $\p_W$ and of $\p_{\mathcal{A}^\ast \setminus W}$, that is, the number of steps is no more than a given time bound in each computation. Everytime a print-instruction is to be executed, we compare the content in $\R_0$ against the input $\zeta$. If there is a match, we will know whether $\zeta \in W$ according to which of $\p_W$ and $\p_{\mathcal{A}^\ast \setminus W}$ that issued the print-instruction. If the timer set for $\p_W$ decreases to zero during the computation of $\p_W$, then $\p_W$ is aborted and it is the turn of $\p_{\mathcal{A}^\ast \setminus W}$ to begin computing, and vice versa. Finally, whenever a program is aborted, a greater time bound is given to the computation of the other.\\
\ \\
Likewise, for convenience we assume that there is exactly one print-instruction in $\p_W$ and in $\p_{\mathcal{A}^\ast \setminus W}$, that there are $L + 2$ and $L^\prime + 2$ instructions in $\p_W$ and in $\p_{\mathcal{A}^\ast \setminus W}$ respectively,\footnote{One for $\PRINT$, and one for $\HALT$.} and that the registers used in $\p_W$ and in $\p_{\mathcal{A}^\ast \setminus W}$ collectively are among $\R_0, \ldots, \R_s$.\\
\ \\
Let us pick the following registers for special purposes:
\begin{enumerate}[1)]
\item $\R_{s + 1}$ is used to record the input $\zeta$.
%%%
\item $\R_{s + 2}$ contains a copy of $\zeta$, and serves to recover $\R_{s + 1}$ after comparison (if necessary).
%%%
\item $\R_{s + 3}$ serves as a time bound, and is used to recover $\R_{s + 4}$ before every computation of $\p_W$ and of $\p_{\mathcal{A}^\ast \setminus W}$.
%%%
\item $\R_{s + 4}$ is the timer: It decreases by one after each step during every computation of $\p_W$ and of $\p_{\mathcal{A}^\ast \setminus W}$.
%%%
\item $\R_{s + 5}$ is used as a temporary storage for $\xi$, the content to be printed out by the print-instruction issued by $\p_W$ or $\p_{\mathcal{A}^\ast \setminus W}$.
%%%
\item $\R_{s + 6}$ is used as a temporary storage for $\xi$ to recover $\R_{s + 5}$ after comparison (if necessary).
%%%
\item $\R_{s + 7}$ indicates which of $\p_W$ and $\p_{\mathcal{A}^\ast \setminus W}$ is the running program: If $\R_{s + 7} = \Box$ then the running program is $\p_W$; if the content in it is $a_0$ then the running program is $\p_{\mathcal{A}^\ast \setminus W}$.
\end{enumerate}
\ \\
There are $3(L + L^\prime) + (r + 2)s + 26r + 67$ instructions in it, where
\begin{enumerate}[(1)]
\item Instructions $0$ - $4r + 4$ are
\[
\begin{array}{rl}
0 & \IF \ \R_0 = \Box \ \THEN \ 4r + 5 \ \ELSE \ 1 \ \OR \ldots \ 4k + 1 \ldots \cr
\ & \OR \ 4r + 1 \cr
1 & \PS{0}{a_0} \cr
2 & \PA{s + 1}{a_0} \cr
3 & \PA{s + 2}{a_0} \cr
4 & \GOTO \ 0 \cr
\multicolumn{2}{c}{\vdots} \cr
4r + 1 & \PS{0}{a_r} \cr
4r + 2 & \PA{s + 1}{a_r} \cr
4r + 3 & \PA{s + 2}{a_r} \cr
4r + 4 & \GOTO \ 0 \cr
\end{array}
\]
(Move the content $\zeta$ in $\R_0$ to $\R_{s + 1}$ and $\R_{s + 2}$, in reverse order. Upon completion, $\R_0$ contains $\Box$, while $\R_{s + 1}$ and $\R_{s + 2}$ both contain $\zeta$ in reverse order.)
%%%
\item Instructions $4r + 5$ - $4r + 14$ are
\[
\begin{array}{rl}
4r + 5 & \PA{s + 3}{a_0} \cr
4r + 6 & \IF \ \R_{s + 3} = \Box \ \THEN \ 4r + 11 \ \ELSE \ 4r + 7 \ldots \cr
\ & \OR \ 4r + 7 \cr
4r + 7 & \PS{s + 3}{a_0} \cr
4r + 8 & \PA{s + 4}{a_0} \cr
4r + 9 & \PA{0}{a_0} \cr
4r + 10 & \GOTO \ 4r + 6 \cr
4r + 11 & \IF \ \R_0 = \Box \ \THEN \ 4r + 15 \ \ELSE \ 4r + 12 \ldots \cr
\ & \OR \ 4r + 12 \cr
4r + 12 & \PS{0}{a_0} \cr
4r + 13 & \PA{s + 3}{a_0} \cr
4r + 14 & \GOTO \ 4r + 11
\end{array}
\]
(Increase the time bound (the content in $\R_{s + 3}$) by one. And then copy it, using $\R_0$ as a temporary storage, into $\R_{s + 4}$, which is the timer.)
%%%
\item Instruction $4r + 15$ is
\[
\begin{array}{rl}
4r + 15 & \IF \ \R_{s + 7} = \Box \ \THEN \ 4r + 16 \ \ELSE \ 3L + 4r + 21 \ldots \cr
\ & \OR \ 3L + 4r + 21
\end{array}
\]
(If $\R_{s + 7} = \Box$, which means the program to run is $\p_W$, then jump to the starting point of $\p_W$ (the next part); otherwise, the program to run is $\p_{\mathcal{A}^\ast \setminus W}$, and jump to the starting point of $\p_{\mathcal{A}^\ast \setminus W}$.)
%%%
\item Instructions $4r + 16$ - $3L + 4r + 20$ are derived from $\p_W$: For $0 \leq l \leq L$,
\begin{enumerate}[1)]
\item the instruction at label $3l + 4r + 16$ is the one at label $l$ in $\p_W$ with all labels $l^\prime$ therein changed to $3l^\prime + 4r + 16$;
%%%%
\item instructions $3l + 4r + 17$ - $3l + 4r + 18$ are
\[
\begin{array}{rl}
3l + 4r + 17 & \PS{s + 4}{a_0} \cr
3l + 4r + 18 & \IF \ \R_{s + 4} = \Box \ \THEN \ 3(L + L^\prime) + 25r + 58 \cr
\ & \ELSE \ 3l + 4r + 19 \ldots \cr
\ & \OR \ 3l + 4r + 19
\end{array}
\]
\end{enumerate}
(Decrease the timer by one after each step of $\p_W$.)\\
\ \\
Instructions $3L + 4r + 19$ - $3L + 4r + 20$ are
\[
\begin{array}{rl}
3L + 4r + 19 & \PA{0}{a_0} \cr
3L + 4r + 20 & \GOTO \ 3(L + L^\prime) + (r + 2)s + 26r + 65
\end{array}
\]
($\p_W$ is to halt, thus $\zeta \not\in W$.)\\
\ \\
Suppose in the resulting program fragment, the print-instruction is at label $L_0$, then it is replaced by
\[
\begin{array}{rl}
L_0 & \GOTO \ 3(L + L^\prime) + 5r + 26
\end{array}
\]
%%%
\item Instructions $3L + 4r + 21$ - $3(L + L^\prime) + 5r + 25$ are derived from $\p_{\mathcal{A}^\ast \setminus W}$: For $0 \leq l \leq L^\prime$,
\begin{enumerate}[1)]
\item the instruction at label $3l + 3L + 4r + 21$ is the one at label $l$ in $\p_{\mathcal{A}^\ast \setminus W}$ with all labels $l^\prime$ therein changed to $3l^\prime + 3L + 4r + 21$;
%%%%
\item instructions $3l + 3L + 4r + 22$ - $3l + 3l + 4r + 23$ are
\[
\begin{array}{rl}
3l + 3L + 4r + 22 & \PS{s + 4}{a_0} \cr
3l + 3L + 4r + 23 & \IF \ \R_{s + 4} = \Box \ \THEN \ 3(L + L^\prime) + 25r + 58 \cr
\ & \ELSE \ 3l + 3L + 4r + 24 \ldots \cr
\ & \OR \ 3l + 3L + 4r + 24
\end{array}
\]
\end{enumerate}
(Decrease the timer by one after each step of $\p_{\mathcal{A}^\ast \setminus W}$.)\\
\ \\
Instructions $3(L + L^\prime) + 4r + 24$ - $3(L + L^\prime) + 5r + 25$ are
\[
\begin{array}{rl}
3(L + L^\prime) + 4r + 24 & \PS{0}{a_0} \cr
\multicolumn{2}{c}{\vdots} \cr
3(L + L^\prime) + 5r + 24 & \PS{0}{a_r} \cr
3(L + L^\prime) + 5r + 25 & \IF \ \R_0 = \Box \cr
\ & \THEN \ 3(L + L^\prime) + (r + 2)s + 26r + 65 \cr
\ & \ELSE \ 3(L + L^\prime) + 4r + 24 \ldots \cr
\ & \OR \ 3(L + L^\prime) + 4r + 24
\end{array}
\]
($\p_{\mathcal{A}^\ast \setminus W}$ is to halt, thus $\zeta \in W$.)\\
\ \\
Suppose in the resulting program fragment, the print-instruction is at label $L_1$, then it is replaced by
\[
\begin{array}{rl}
L_1 & \GOTO \ 3(L + L^\prime) + 5r + 26
\end{array}
\]
%%%
\item $3(L + L^\prime) + 5r + 26$ - $3(L + L^\prime) + 9r + 30$ are
\[
\begin{array}{rl}
3(L + L^\prime) + 5r + 26 & \IF \ \R_0 = \Box \ \THEN \ 3(L + L^\prime) + 9r + 31 \cr
\ & \ELSE \ 3(L + L^\prime) + 5r + 27 \cr
\ & \OR \ 3(L + L^\prime) + 5r + 4k + 27 \ldots \cr
\ & \OR \ 3(L + L^\prime) + 9r + 27 \cr
3(L + L^\prime) + 5r + 27 & \PS{0}{a_0} \cr
3(L + L^\prime) + 5r + 28 & \PS{s + 5}{a_0} \cr
3(L + L^\prime) + 5r + 29 & \PS{s + 6}{a_0} \cr
3(L + L^\prime) + 5r + 30 & \GOTO \ 3(L + L^\prime) + 5r + 26 \cr
\multicolumn{2}{c}{\vdots} \cr
3(L + L^\prime) + 9r + 27 & \PS{0}{a_r} \cr
3(L + L^\prime) + 9r + 28 & \PS{s + 5}{a_r} \cr
3(L + L^\prime) + 9r + 29 & \PS{s + 6}{a_r} \cr
3(L + L^\prime) + 9r + 30 & \GOTO \ 3(L + L^\prime) + 5r + 26
\end{array}
\]
(Move the content $\xi$ in $\R_0$, which is to be printed out by $\p_W$ or $\p_{\mathcal{A}^\ast \setminus W}$, to $\R_{s + 5}$ and $\R_{s + 6}$, in reverse order. Upon completion, $\R_0 = \Box$, whereas $\R_{s + 5}$ and $\R_{s + 6}$ both contain $\xi$ in reverse order.)
%%%
\item Instructions $3(L + L^\prime) + 9r + 31$ - $3(L + L^\prime) + 13r + 39$ are
\[
\begin{array}{rl}
3(L + L^\prime) + 9r + 31 & \IF \ \R_{s + 1} = \Box \cr
\ & \THEN \ 3(L + L^\prime) + 9r + 32 \cr
\ & \ELSE \ 3(L + L^\prime) + 9r + 36 \cr
\ & \OR \ldots \ 3(L + L^\prime) + 9r + 4k + 36 \ldots \cr
\ & \OR \ 3(L + L^\prime) + 13r + 36 \cr
3(L + L^\prime) + 9r + 32 & \IF \ \R_{s + 5} = \Box \cr
\ & \THEN \ 3(L + L^\prime) + 9r + 33 \cr
\ & \ELSE \ 3(L + L^\prime) + 13r + 40 \ldots \cr
\ & \OR \ 3(L + L^\prime) + 13r + 40 \cr

3(L + L^\prime) + 9r + 33 & \IF \ \R_{s + 7} = \Box \cr
\ & \THEN \ 3(L + L^\prime) + (r + 2)s + 26r + 65 \cr
\ & \ELSE \ 3(L + L^\prime) + 9r + 34 \ldots \cr
\ & \OR \ 3(L + L^\prime) + 9r + 34 \cr
3(L + L^\prime) + 9r + 34 & \PA{0}{a_0} \cr
3(L + L^\prime) + 9r + 35 & \GOTO \ 3(L + L^\prime) + (r + 2)s + 26r + 65 \cr
\end{array}
\]
\[
\begin{array}{rl}
3(L + L^\prime) + 9r + 36 & \IF \ \R_{s + 5} = \Box \cr
\ & \THEN \ 3(L + L^\prime) + 13r + 40 \cr
\ & \ELSE \ 3(L + L^\prime) + 9r + 37 \cr
\ & \mbox{\scriptsize$r$-times}\left\{\begin{array}{l}
\OR \ 3(L + L^\prime) + 13r + 40 \cr
\multicolumn{1}{c}{\vdots} \cr
\OR \ 3(L + L^\prime) + 13r + 40
\end{array}\right. \cr
3(L + L^\prime) + 9r + 37 & \PS{s + 1}{0} \cr
3(L + L^\prime) + 9r + 38 & \PS{s + 5}{0} \cr
3(L + L^\prime) + 9r + 39 & \GOTO \ 3(L + L^\prime) + 9r + 31 \cr
\multicolumn{2}{c}{\vdots} \cr
3(L + L^\prime) + 9r + 4k + 36 & \IF \ \R_{s + 5} = \Box \cr
\ & \THEN \ 3(L + L^\prime) + 13r + 40 \cr
\ & \mbox{\scriptsize $k$-times}\left\{\begin{array}{l}
\ELSE \ 3(L + L^\prime) + 13r + 40 \cr
\multicolumn{1}{c}{\vdots} \cr
\OR \ 3(L + L^\prime) + 13r + 40
\end{array}\right. \cr
\ & \OR \ 3(L + L^\prime) + 9r + 4k + 37 \cr
\ & \mbox{\scriptsize $(r - k)$-times}\left\{\begin{array}{l}
\OR \cr
3(L + L^\prime) + 13r + 40 \cr
\multicolumn{1}{c}{\vdots} \cr
\OR \cr
3(L + L^\prime) + 13r + 40
\end{array}\right. \cr
3(L + L^\prime) + 9r + 4k + 37 & \PS{s + 1}{a_k} \cr
3(L + L^\prime) + 9r + 4k + 38 & \PS{s + 5}{a_k} \cr
3(L + L^\prime) + 9r + 4k + 39 & \GOTO \ 3(L + L^\prime) + 9r + 31 \cr
\multicolumn{2}{c}{\vdots} \cr
3(L + L^\prime) + 13r + 36 & \IF \ \R_{s + 5} = \Box \cr
\ & \THEN \ 3(L + L^\prime) + 13r + 40 \cr
\ & \mbox{\scriptsize $r$-times}\left\{\begin{array}{l}
\ELSE \ 3(L + L^\prime) + 13r + 40 \cr
\multicolumn{1}{c}{\vdots} \cr
\OR \ 3(L + L^\prime) + 13r + 40
\end{array}\right. \cr
3(L + L^\prime) + 13r + 37 & \PS{s + 1}{a_r} \cr
3(L + L^\prime) + 13r + 38 & \PS{s + 5}{a_r} \cr
3(L + L^\prime) + 13r + 39 & \GOTO \ 3(L + L^\prime) + 9r + 31
\end{array}
\]
(Compare the content in $\R_{s + 1}$ against that in $\R_{s + 5}$, i.e. $\zeta$ in reverse order against $\xi$ in reverse order. If they match, then we have $\zeta \in W$ if $\R_{s + 7} = \Box$ and $\zeta \not\in W$ otherwise, and set $\R_0$ accordingly, then jump to the print-instruction. If they do not match, then both $\R_{s + 1}$ and $\R_{s + 5}$ are cleared in the next part.)
%%%
\item Instructions $3(L + L^\prime) + 13r + 40$ - $3(L + L^\prime) + 15r + 43$ are
\[
\begin{array}{rl}
3(L + L^\prime) + 13r + 40 & \PS{s + 1}{a_0} \cr
\multicolumn{2}{c}{\vdots} \cr
3(L + L^\prime) + 14r + 40 & \PS{s + 1}{a_r} \cr
3(L + L^\prime) + 14r + 41 & \IF \ \R_{s + 1} = \Box \ \THEN \ 3(L + L^\prime) + 14r + 42 \cr
\ & \ELSE \ 3(L + L^\prime) + 13r + 40 \ldots \cr
\ & \OR \ 3(L + L^\prime) + 13r + 40 \cr
3(L + L^\prime) + 14r + 42 & \PS{s + 5}{a_0} \cr
\multicolumn{2}{c}{\vdots} \cr
3(L + L^\prime) + 15r + 42 & \PS{s + 5}{a_r} \cr
3(L + L^\prime) + 15r + 43 & \IF \ \R_{s + 5} = \Box \ \THEN \ 3(L + L^\prime) + 15r + 44 \cr
\ & \ELSE \ 3(L + L^\prime) + 14r + 42 \ldots \cr
\ & \OR \ 3(L + L^\prime) + 14r + 42
\end{array}
\]
(Clear $\R_{s + 1}$ and $\R_{s + 5}$.)
%%%
\item Instructions $3(L + L^\prime) + 15r + 44$ - $3(L + L^\prime) + 22r + 52$ are
\[
\begin{array}{rl}
3(L + L^\prime) + 15r + 44 & \IF \ \R_{s + 2} = \Box \ \THEN \ 3(L + L^\prime) + 18r + 48 \cr
\ & \ELSE \ 3(L + L^\prime) + 15r + 45 \cr
\ & \OR \ldots \ 3(L + L^\prime) + 15r + 3k + 45 \ldots \cr
\ & \OR \ 3(L + L^\prime) + 18r + 45 \cr
3(L + L^\prime) + 15r + 45 & \PS{s + 2}{a_0} \cr
3(L + L^\prime) + 15r + 46 & \PA{0}{a_0} \cr
3(L + L^\prime) + 15r + 47 & \GOTO \ 3(L + L^\prime) + 15r + 44 \cr
\multicolumn{2}{c}{\vdots} \cr
3(L + L^\prime) + 18r + 45 & \PS{s + 2}{a_r} \cr
3(L + L^\prime) + 18r + 46 & \PA{0}{a_r} \cr
3(L + L^\prime) + 18r + 47 & \GOTO \ 3(L + L^\prime) + 15r + 44 \cr
3(L + L^\prime) + 18r + 48 & \IF \ \R_0 = \Box \ \THEN \ 3(L + L^\prime) + 22r + 53 \cr
\ & \ELSE \ 3(L + L^\prime) + 18r + 49 \cr
\ & \OR \ldots \ 3(L + L^\prime) + 18r + 4k + 49 \ldots \cr
\ & \OR \ 3(L + L^\prime) + 22r + 49 \cr
3(L + L^\prime) + 18r + 49 & \PS{0}{a_0} \cr
3(L + L^\prime) + 18r + 50 & \PA{s + 1}{a_0} \cr
3(L + L^\prime) + 18r + 51 & \PA{s + 2}{a_0} \cr
3(L + L^\prime) + 18r + 52 & \GOTO \ 3(L + L^\prime) + 18r + 48 \cr
\multicolumn{2}{c}{\vdots} \cr
3(L + L^\prime) + 22r + 49 & \PS{0}{a_r} \cr
3(L + L^\prime) + 22r + 50 & \PA{s + 1}{a_r} \cr
3(L + L^\prime) + 22r + 51 & \PA{s + 2}{a_r} \cr
3(L + L^\prime) + 22r + 52 & \GOTO \ 3(L + L^\prime) + 18r + 48 \cr
\end{array}
\]
(Recover $\R_{s + 1}$ from $\R_{s + 2}$, using $\R_0$ as a temporary storage. Upon completion, $\R_0 = \Box$, and both $\R_{s + 1}$ and $\R_{s + 2}$ contain $\zeta$ in reverse order.)
%%%
\item Instructions $3(L + L^\prime) + 22r + 53$ - $3(L + L^\prime) + 25r + 56$ are
\[
\begin{array}{rl}
3(L + L^\prime) + 22r + 53 & \IF \ \R_{s + 6} = \Box \ \THEN \ 3(L + L^\prime) + 25r + 57 \cr
\ & \ELSE \ 3(L + L^\prime) + 22r + 54 \cr
\ & \OR \ldots \ 3(L + L^\prime) + 22r + 3k + 54 \ldots \cr
\ & \OR \ 3(L + L^\prime) + 25r + 54 \cr
3(L + L^\prime) + 22r + 54 & \PS{s + 6}{a_0} \cr
3(L + L^\prime) + 22r + 55 & \PA{0}{a_0} \cr
3(L + L^\prime) + 22r + 56 & \GOTO \ 3(L + L^\prime) + 22r + 53 \cr
\multicolumn{2}{c}{\vdots} \cr
3(L + L^\prime) + 25r + 54 & \PS{s + 6}{a_r} \cr
3(L + L^\prime) + 25r + 55 & \PA{0}{a_r} \cr
3(L + L^\prime) + 25r + 56 & \GOTO \ 3(L + L^\prime) + 22r + 53
\end{array}
\]
(Recover $\R_0$ from $\R_{s + 6}$. Upon completion, $\R_0$ contains $\xi$, while $\R_{s + 5} = \Box$ and $\R_{s + 6} = \Box$.)
%%%
\item Instruction $3(L + L^\prime) + 25r + 57$ is
\[
\begin{array}{rl}
3(L + L^\prime) + 25r + 57 & \IF \ \R_{s + 7} = \Box \ \THEN \ L_0 + 1 \cr
\ & \ELSE \ L_1 + 1 \ldots \ \OR \ L_1 + 1 \cr
\end{array}
\]
(Resume the program.)
%%%
\item Instructions $3(L + L^\prime) + 25r + 58$ - $3(L + L^\prime) + (r + 2)s + 26r + 64$ are
\[
\begin{array}{rl}
3(L + L^\prime) + 25r + 58 & \PS{0}{a_0} \cr
\multicolumn{2}{c}{\vdots} \cr
3(L + L^\prime) + 26r + 58 & \PS{0}{a_r} \cr
3(L + L^\prime) + 26r + 59 & \IF \ \R_0 = \Box \cr
\ & \THEN \ 3(L + L^\prime) + 26r + 60 \cr
\ & \ELSE \ 3(L + L^\prime) + 25r + 58 \ldots \cr
\ & \OR \ 3(L + L^\prime) + 25r + 58 \cr
\multicolumn{2}{c}{\vdots} \cr
\multicolumn{2}{c}{\vdots} \cr
3(L + L^\prime) + (r + 2)s + 25r + 58 & \PS{s}{a_0} \cr
\multicolumn{2}{c}{\vdots} \cr
3(L + L^\prime) + (r + 2)s + 26r + 58 & \PS{s}{a_r} \cr
3(L + L^\prime) + (r + 2)s + 26r + 59 & \IF \ \R_s = \Box \cr
\ & \THEN \cr
\ & 3(L + L^\prime) + (r + 2)s + 26r + 60 \cr
\ & \ELSE \cr
\ & 3(L + L^\prime) + (r + 2)s + 25r + 58 \cr
\ & \ldots \cr
\ & \OR \cr
\ & 3(L + L^\prime) + (r + 2)s + 25r + 58 \cr
3(L + L^\prime) + (r + 2)s + 26r + 60 & \IF \ \R_{s + 7} = \Box \cr
\ & \THEN \cr
\ & 3(L + L^\prime) + (r + 2)s + 26r + 61 \cr
\ & \ELSE \cr
\ & 3(L + L^\prime) + (r + 2)s + 26r + 63 \cr
\ & \ldots \cr
\ & \OR \cr
\ & 3(L + L^\prime) + (r + 2)s + 26r + 63 \cr
3(L + L^\prime) + (r + 2)s + 26r + 61 & \PA{s + 7}{a_0} \cr
3(L + L^\prime) + (r + 2)s + 26r + 62 & \GOTO \ 4r + 5 \cr
3(L + L^\prime) + (r + 2)s + 26r + 63 & \PS{s + 7}{a_0} \cr
3(L + L^\prime) + (r + 2)s + 26r + 64 & \GOTO \ 4r + 5
\end{array}
\]
(Time is up. Clear registers $\R_0, \ldots, \R_s$, and run the other program and change $\R_{s + 7}$ accordingly.)
%%%
\item Instructions $3(L + L^\prime) + (r + 2)s + 26r + 65$ - $3(L + L^\prime) + (r + 2)s + 26r + 66$ are
\[
\begin{array}{rl}
3(L + L^\prime) + (r + 2)s + 26r + 65 & \PRINT \cr
3(L + L^\prime) + (r + 2)s + 26r + 66 & \HALT
\end{array}
\]\nolinebreak\hfill$\talloblong$
\end{enumerate}
\end{enumerate}
\textit{Remark.} $\mathcal{A}^\ast$ is also decidable, as shown by the program:
\[
\begin{array}{rl}
0 & \PS{0}{a_0} \cr
\multicolumn{2}{c}{\vdots} \cr
r & \PS{0}{a_r} \cr
r + 1 & \PII{0}{r + 2}{0}{k}{r} \cr
r + 2 & \PRINT \cr
r + 3 & \HALT
\end{array}
\]
%
\item \textbf{Solution to Exercise 2.11.} We prove that (a) and (b) are equivalent by showing that (a) implies (b) and that (b) implies (a). Let $\mathcal{A} = \{ a_0, \ldots, a_r \}$. In the following, the program that enumerates $W$ is denoted by $\p_W$.\\
\ \\
\emph{(a) implies (b).} Given the program $\p_W$, we show how to construct $\p$. The idea behind this construction is that, everytime the content in $\R_0$ is to be printed by $\p_W$, we compare it with the input $\zeta$: If they match, then we have that $\zeta \in W$ and $\p$ halts; otherwise, we go on to test the next string to be printed. It follows that if $\zeta \not\in W$, then $\p$ never halts.\\
\ \\
For convenience, let us assume that there are $L + 2$ instructions in $\p_W$ with $L \geq 0$,\footnote{One for $\PRINT$, and one for $\HALT$.} that there is exactly one print-instruction, and that the registers used in it are among $\R_0, \ldots, \R_s$.\\
\ \\
Then there are $L + 24r + 37$ instructions in $\p$, where
\begin{enumerate}[(1)]
\item Instructions $0$ - $4r + 4$ are
\[
\begin{array}{rl}
0 & \IF \ \R_0 = \Box \ \THEN \ 4r + 5 \cr
\ & \ELSE \ 1 \ \OR \ldots \ 4k + 1 \ldots \ \OR \ 4r + 1 \cr
1 & \PS{0}{a_0} \cr
2 & \PA{s + 1}{a_0} \cr
3 & \PA{s + 2}{a_0} \cr
4 & \GOTO \  0 \cr
\multicolumn{2}{c}{\vdots} \cr
4r + 1 & \PS{0}{a_r} \cr
4r + 2 & \PA{s + 1}{a_r} \cr
4r + 3 & \PA{s + 2}{a_r} \cr
4r + 4 & \GOTO \ 0
\end{array}
\]
(Move the content in $\R_0$, that is, the input $\zeta$, to $\R_{s + 1}$ and $\R_{s + 2}$ in reverse order. Upon completion, $\R_0$ contains $\Box$, whereas $\R_{s + 1}$ and $\R_{s + 2}$ both contain $\zeta$ in reverse order.)
%%
\item Instructions $4r + 5$ - $L + 4r + 6$ are derived from $\p_W$: For $0 \leq l \leq L + 1$, the instruction at label $l + 4r + 5$ is the one at label $l$ in $\p_W$, with all labels therein increased by $4r + 5$. Furthermore, suppose in the resulting program fragment the print-instruction is at label $L_0$, then it is replaced by
\[
\begin{array}{rl}
L_0 & L + 4r + 7
\end{array}
\]
(whenever the content $\xi$ in $\R_0$ is to be printed out, compare it with $\zeta$, in the next part)
and the instruction at label $L + 4r + 6$ is replaced by
\[
\begin{array}{rl}
L + 4r + 6 & \GOTO \ L + 4r + 6
\end{array}
\]
($\zeta \not\in W$, thus loop forever)
%%
\item Instructions $L + 4r + 7$ - $L + 8r + 11$ are
\[
\begin{array}{rl}
L + 4r + 7 & \IF \ \R_0 = \Box \ \THEN \ L + 8r + 12 \cr
\ & \ELSE \ L + 4r + 8 \ \OR \ldots \ L + 4r + 4k + 8 \ldots \cr
\ & \OR \ L + 8r + 8 \cr
L + 4r + 8 & \PS{0}{a_0} \cr
L + 4r + 9 & \PA{s + 3}{a_0} \cr
L + 4r + 10 & \PA{s + 4}{a_0} \cr
L + 4r + 11 & \GOTO \ L + 4r + 7 \cr
\multicolumn{2}{c}{\vdots} \cr
L + 8r + 8 & \PS{0}{a_r} \cr
L + 8r + 9 & \PA{s + 3}{a_r} \cr
L + 8r + 10 & \PA{s + 4}{a_r} \cr
L + 8r + 11 & \GOTO \ L + 4r + 7
\end{array}
\]
(Move the content $\xi$ in $\R_0$ to $\R_{s + 3}$ and $\R_{s + 4}$ in reverse order. Upon completion, $\R_0 = \Box$, while $\R_{s + 3}$ and $\R_{s + 4}$ both contain $\xi$ in reverse order.)
%%
\item Instructions $L + 8r + 12$ - $L + 12r + 17$ are
\[
\begin{array}{rl}
L + 8r + 12 & \IF \ \R_{s + 1} = \Box \ \THEN \ L + 8r + 13 \cr
\ & \ELSE \ L + 8r + 14 \ \OR \ldots \ L + 8r + k + 14 \ldots \cr
\ & \OR \ L + 9r + 14 \cr
L + 8r + 13 & \IF \ \R_{s + 3} = \Box \ \THEN \ L + 24r + 35 \cr
\ & \ELSE \ L + 12r + 18 \ldots \ \OR \ L + 12r + 18 \cr
L + 8r + 14 & \IF \ \R_{s + 3} = \Box \cr
\ & \THEN \ L + 12r + 18 \ \ELSE \ L + 9r + 15 \cr
\ & \underbrace{\OR \ L + 12r + 18 \ \OR \ldots \OR \ L + 12r + 18}_{\mbox{\scriptsize$r$-times}} \cr
\multicolumn{2}{c}{\vdots} \cr
L + 8r + k + 14 & \IF \ \R_{s + 3} = \Box \ \THEN \ L + 12r + 18 \cr
\ & \underbrace{\ELSE \ L + 12r + 18 \ \OR \ldots \OR \ L + 12r + 18}_{\mbox{\scriptsize$k$-times}} \cr
\ & \OR \ L + 9r + 3k + 15 \cr
\ & \underbrace{\OR \ L + 12r + 18 \ \OR \ldots \OR \ L + 12r + 18}_{\mbox{\scriptsize $(r - k)$-times}} \cr
\multicolumn{2}{c}{\vdots} \cr
L + 9r + 14 & \IF \ \R_p = \Box \ \THEN \ L + 12r + 18 \cr
\ & \underbrace{\ELSE \ L + 12r + 18 \ \OR \ldots \OR \ L + 12r + 18}_{\mbox{\scriptsize$r$-times}} \cr
\ & \OR \ L + 12r + 15 \cr
L + 9r + 15 & \PS{s + 1}{a_0} \cr
L + 9r + 16 & \PS{s + 3}{a_0} \cr
L + 9r + 15 & \GOTO \ L + 8r + 12 \cr
\multicolumn{2}{c}{\vdots} \cr
L + 12r + 15 & \PS{s + 1}{a_r} \cr
L + 12r + 16 & \PS{s + 3}{a_r} \cr
L + 12r + 17 & \GOTO \ L + 8r + 12
\end{array}
\]
(Compare the content in $\R_{s + 1}$ against that in $\R_{s + 3}$, i.e. $\zeta$ in reverse order against $\xi$ in reverse order: If they match, then $\zeta \in W$, so jump to the print-instruction; otherwise, $\R_{s + 1}$ and $\R_{s + 3}$ are both cleared in the next part.)
%%
\item Instructions $L + 12r + 18$ - $L + 14r + 21$ are
\[
\begin{array}{rl}
L + 12r + 18 & \PS{s + 1}{a_0} \cr
\multicolumn{2}{c}{\vdots} \cr
L + 13r + 18 & \PS{s + 1}{a_r} \cr
L + 13r + 19 & \IF \ \R_{s + 1} = \Box \ \THEN \ L + 13r + 20 \cr
\ & \ELSE \ L + 12r + 18 \ldots \ \OR \ L + 12r + 18 \cr
L + 13r + 20 & \PS{s + 3}{a_0} \cr
\multicolumn{2}{c}{\vdots} \cr
L + 14r + 20 & \PS{s + 3}{a_r} \cr
L + 14r + 21 & \IF \ \R_{s + 3} = \Box \ \THEN \ L + 14r + 22 \cr
\ & \ELSE \ L + 13r + 20 \ldots \ \OR \ L + 13r + 20 \cr
\end{array}
\]
(Clear $\R_{s + 1}$ and $\R_{s + 3}$.)
%%
\item Instructions $L + 14r + 22$ - $L + 17r + 25$ are
\[
\begin{array}{rl}
L + 14r + 22 & \IF \ \R_{s + 2} = \Box \ \THEN \ L + 17r + 26 \cr
\ & \ELSE \ L + 14r + 23 \ \OR \ldots \ L + 14r + 3k + 23 \ldots \cr
\ & \OR \ L + 17r + 23 \cr
L + 14r + 23 & \PS{s + 2}{a_0} \cr
L + 14r + 24 & \PA{0}{a_0} \cr
L + 14r + 25 & \GOTO \ L + 14r + 22 \cr
\multicolumn{2}{c}{\vdots} \cr
L + 17r + 23 & \PS{s + 2}{a_r} \cr
L + 17r + 24 & \PA{0}{a_r} \cr
L + 17r + 25 & \GOTO \ L + 14r + 22 \cr
L + 17r + 26 & \IF \ \R_0 = \Box \ \THEN L + 21r + 31 \cr
\ & \ELSE \ L + 17r + 27 \ \OR \ldots \ L + 17r + 4k + 27 \ldots \cr
\ & \OR \ L + 21r + 27 \cr
L + 17r + 27 & \PS{0}{a_0} \cr
L + 17r + 28 & \PA{s + 1}{a_0} \cr
L + 17r + 29 & \PA{s + 2}{a_0} \cr
L + 17r + 30 & \GOTO \ L + 17r + 26 \cr
\multicolumn{2}{c}{\vdots} \cr
L + 21r + 27 & \PS{0}{a_r} \cr
L + 21r + 28 & \PA{s + 1}{a_r} \cr
L + 21r + 29 & \PA{s + 2}{a_r} \cr
L + 21r + 30 & \GOTO \ L + 17r + 26
\end{array}
\]
(Copy the content in $\R_{s + 2}$, that is, $\zeta$ in reverse order, into $\R_{s + 1}$, using $\R_0$ as a temporary register. Note that both on entrance to and exit from this part, $\R_0 = \Box$.)
%%
\item Instructions $L + 21r + 31$ - $L + 24r + 36$ are
\[
\begin{array}{rl}
L + 21r + 31 & \IF \ \R_{s + 4} = \Box \ \THEN \ L_0 + 1 \cr
\ & \ELSE \ L + 21r + 32 \ \OR \ldots \ L + 21r + 3k + 32 \ldots \cr
\ & \OR \ L + 24r + 32 \cr
L + 21r + 32 & \PS{s + 4}{a_0} \cr
L + 21r + 33 & \PA{0}{a_0} \cr
L + 21r + 34 & \GOTO \ L + 21r + 31 \cr
\multicolumn{2}{c}{\vdots} \cr
L + 24r + 32 & \PS{s + 4}{a_r} \cr
L + 24r + 33 & \PA{0}{a_r} \cr
L + 24r + 34 & \GOTO \ L + 21r + 31
\end{array}
\]
(Recover $\xi$ in $\R_0$ from $\R_{s + 4}$. Upon completion, $\R_0$ contains $\xi$, whereas $\R_{s + 1}$ and $\R_{s + 2}$ both contain $\zeta$ in reverse order, and $\R_{s + 3} = \Box$ and $\R_{s + 4} = \Box$. Then go back to label $L_0 + 1$ to begin the next iteration.)
%%
\item Instructions $L + 24r + 35$ - $L + 24r + 36$ are
\[
\begin{array}{rl}
L + 24r + 35 & \PRINT \cr
L + 24r + 36 & \HALT
\end{array}
\]
(Print the content $\Box$ in $\R_0$, and then halt.)
\end{enumerate}
\ \\
\emph{(b) implies (a).} Conversely, given $\p$, we show how to construct the program $\p_W$ that enumerates the set $W$. The idea is that, we set a timer, and run $\p$ with each of the strings of length $\leq n$ over $\mathcal{A}$ in lexicographic order within the time bound $n$, for all $n \in \mathbb{N}$. If $\p : \zeta \to \Box$, then by definition $\zeta \in W$ and we print it out. Note that there will be infinitely many duplicates for each $\zeta \in W$, and hence $\p_W$ does not halt.\\
\ \\
Again, for convenience we assume that there are $L + 2$ instructions in $\p$ with $L \geq 0$,\footnote{One for $\PRINT$, and one for $\HALT$.} that there is exactly one print-instruction in it, and that the registers used in $\p$ are among $\R_0, \ldots, \R_s$ with $s \geq 0$. Furthermore, we keep the following registers for special purposes:
\begin{enumerate}[1)]
\item $\R_{s + 1}$ indicates the time bound.
%%
\item $\R_{s + 2}$ helps limit the number of strings with which $\p$ executes with each given time bound.
%%
\item $\R_{s + 3}$ is the timer; it can also tell whether $\zeta \in W$ after applying $\p$ to $\zeta$.
%%
\item $\R_{s + 4}$ is used as a temporary storage.
\end{enumerate}
Then there are $3L + (r + 2)s + 17r + 60$ instructions in $\p_W$, where
\begin{enumerate}[(1)]
\item Instructions $0$ - $9$ are
\[
\begin{array}{rl}
0 & \PI{s + 1}{4}{1}{1} \cr
1 & \PS{s + 1}{a_0} \cr
2 & \PA{0}{a_0} \cr
3 & \GOTO \ 0 \cr
4 & \PI{0}{10}{5}{5} \cr
5 & \PS{0}{a_0} \cr
6 & \PA{s + 1}{a_0} \cr
7 & \PA{s + 2}{a_0} \cr
8 & \PA{s + 3}{a_0} \cr
9 & \GOTO \ 4 \cr
\end{array}
\]
(Copy the content in $\R_{s + 1}$, $\underbrace{a_0 \ldots a_0}_{n\mbox{\scriptsize-times}}$ for some $n \in \mathbb{N}$, to $\R_{s + 2}$ and $\R_{s + 3}$, using $\R_0$ as a temporary storage. Also note that on entrance to this part, $\R_0 = \Box$.)
%%
\item Instructions $10$ - $3r + 21$ are
\[
\begin{array}{rl}
10 & \PI{1}{5r + 28}{11}{11} \cr
11 & \PS{1}{a_0} \cr
12 & \PA{0}{a_0} \cr
13 & \GOTO \ 10 \cr
14 & \IF \ \R_0 = \Box \ \THEN \ 15 \ \ELSE \ 19 \ \OR \ldots \ 3k + 19 \ldots \cr
\  & \OR \ 3r + 19 \cr
15 & \PI{s + 2}{3r + 22}{16}{16} \cr
16 & \PS{s + 2}{a_0} \cr
17 & \PA{0}{a_0} \cr
18 & \GOTO \ 10 \cr
19 & \PS{0}{a_0} \cr
20 & \PA{0}{a_1} \cr
21 & \GOTO \ 10 \cr
\multicolumn{2}{c}{\vdots} \cr
3k + 19 & \PS{0}{a_k} \cr
3k + 20 & \PA{0}{a_{k + 1}} \cr
3k + 21 & \GOTO \ 10 \cr
\multicolumn{2}{c}{\vdots} \cr
3r + 19 & \PS{0}{a_r} \cr
3r + 20 & \PA{1}{a_0} \cr
3r + 21 & \GOTO \ 14 \cr
\end{array}
\]
(Apply $\p$ to all strings of length $\leq n$ with the timer $\R_{s + 1}$, the content in which is $\underbrace{a_0 \ldots a_0}_{n\mbox{\scriptsize-times}}$: For each such string $\zeta$, jump to part (5) to check whether $\p : \zeta \to \Box$ within the time bound.)
%%
\item Instructions $3r + 22$ - $5r + 25$ are
\[
\begin{array}{rl}
3r + 22 & \PS{1}{a_0} \cr
\multicolumn{2}{c}{\vdots} \cr
4r + 22 & \PS{1}{a_r} \cr
4r + 23 & \PI{1}{4r + 24}{3r + 22}{3r + 22} \cr
4r + 24 & \PS{s + 3}{a_0} \cr
\multicolumn{2}{c}{\vdots} \cr
5r + 24 & \PS{s + 3}{a_r} \cr
5r + 25 & \PI{s + 3}{5r + 26}{4r + 24}{4r + 24}
\end{array}
\]
(Clear $\R_1$, $\R_{s + 3}$.)
%%
\item Instructions $5r + 26$ - $5r + 27$ are
\[
\begin{array}{rl}
5r + 26 & \PA{s + 1}{a_0} \cr
5r + 27 & \GOTO \ 0
\end{array}
\]
(Increase $\R_{s + 1}$ by one, and then start a new iteration with a greater time bound and, of course, with more strings. This part and the previous ones together constitude the main body of $\p_W$.)
%%
\item Instructions $5r + 28$ - $12r + 36$ are
\[
\begin{array}{rl}
5r + 28 & \IF \ \R_0 = \Box \ \THEN \ 9r + 33 \ \ELSE \ 5r + 29 \cr
\       & \OR \ldots \ 5r + 4k + 29 \ldots \ \OR \ 9r + 29 \cr
5r + 29 & \PS{0}{a_0} \cr
5r + 30 & \PA{1}{a_0} \cr
5r + 31 & \PA{s + 4}{a_0} \cr
5r + 32 & \GOTO \ 5r + 28 \cr
\multicolumn{2}{c}{\vdots} \cr
9r + 29 & \PS{0}{a_r} \cr
9r + 30 & \PA{1}{a_r} \cr
9r + 31 & \PA{s + 4}{a_r} \cr
9r + 32 & \GOTO \ 5r + 28 \cr
9r + 33 & \IF \ \R_1 = \Box \ \THEN \ 12r + 37 \ \ELSE \ 9r + 34 \cr
\       & \OR \ldots \ 9r + 3k + 34 \ldots \ \OR \ 12r + 34 \cr
9r + 34 & \PS{1}{a_0} \cr
9r + 35 & \PA{0}{a_0} \cr
9r + 36 & \GOTO \ 9r + 33 \cr
\multicolumn{2}{c}{\vdots} \cr
12r + 34 & \PS{1}{a_r} \cr
12r + 35 & \PA{0}{a_r} \cr
12r + 36 & \GOTO \ 9r + 33
\end{array}
\]
(Copy the content $\zeta$ in $\R_0$ to $\R_{s + 4}$ in reverse order, using $\R_1$ as a temporary storage. Upon completion, the content in $\R_0$ is still $\zeta$, and the content in $\R_{s + 4}$ is $\zeta$ in reverse order. $\R_{s + 4}$ will be used to restore $\R_0$ later. Also note that on entrance to this part, $\R_1 = \Box$. This part and the remaining ones, except the last, serve to check whether $\p : \zeta \to \Box$ within the time bound.)
%%
\item Instructions $12r + 37$ - $3L + 12r + 39$ are derived from $\p$: For $0 \leq l \leq L$,
\begin{enumerate}[1)]
\item the instruction at label $3l + 12r + 37$ is the one at label $l$ in $\p$ with all labels $l^\prime$ therein replaced by $3l^\prime + 12r + 37$;
%%%
\item instruction $3l + 12r + 38$ - $3l + 12r + 39$ are
\[
\begin{array}{rl}
3l + 12r + 38 & \PS{s + 3}{a_0} \cr
3l + 12r + 39 & \IF \ \R_{s + 3} = \Box \ \THEN \ 3L + 12r + 40 \cr
\             & \ELSE \ 3l + 12r + 40 \ldots \ \OR \ 3l + 12r + 40
\end{array}
\]
\end{enumerate}
Suppose in the resulting program fragment the print-instruction is at label $L_0$:
\[
\begin{array}{rl}
L_0 & \PRINT
\end{array}
\]
It is now replaced by a dummy operation:
\[
\begin{array}{rl}
L_0 & \GOTO \ L_0 + 1
\end{array}
\]
(This part checks whether $\p : \zeta \to \Box$ within the time bound. Actually, if $\zeta \in W$ then $\p : \zeta \to \halt$, otherwise $\p : \zeta \to \infty$. Therefore, it suffices to check whether $\p$ with input $\zeta$ reaches the halt-instrution within the time bound. That is why the print-instruction is \emph{disabled} and the halt-instruction is missing. If we arrive at the label where $\HALT$ should have been, which means that $\zeta \in W$, then $\R_{s + 3} \neq \Box$, and vice versa; conversely, if $\R_{s + 3} = \Box$ then we are not sure whether $\zeta \in W$ but only that \emph{time is up}, and the next iteration is to begin after all has been settled. Hence later in part (9), $\R_{s + 3}$ will be useful to decide whether to print out $\zeta$.)
%%
\item Instructions $3L + 12r + 40$ - $3L + (r + 2)s + 13r + 41$ are
\[
\begin{array}{rl}
3L + 12r + 40 & \PS{0}{a_0} \cr
\multicolumn{2}{c}{\vdots} \cr
3L + 13r + 40 & \PS{0}{a_r} \cr
3L + 13r + 41 & \IF \ \R_0 = \Box \ \THEN \ 3L + 13r + 42 \cr
\ & \ELSE \ 3L + 12r + 40 \ldots \cr
\ & \OR \ 3L + 12r + 40 \cr
\multicolumn{2}{c}{\vdots} \cr
\multicolumn{2}{c}{\vdots} \cr
3L + (r + 2)s + 12r + 40 & \PS{s}{a_0} \cr
\multicolumn{2}{c}{\vdots} \cr
3L + (r + 2)s + 13r + 40 & \PS{s}{a_r} \cr
3L + (r + 2)s + 13r + 41 & \IF \ \R_s = \Box \cr
\ & \THEN \ 3L + (r + 2)s + 13r + 42 \cr
\ & \ELSE \ 3L + (r + 2)s + 12r + 40 \ldots \cr
\ & \OR \ 3L + (r + 2)s + 12r + 40 \cr
\end{array}
\]
(Clear $\R_0, \ldots, \R_s$.)
%%
\item Instructions $3L + (r + 2)s + 13r + 42$ - $3L + (r + 2)s + 16r + 45$ are
\[
\begin{array}{rl}
3L + (r + 2)s + 13r + 42 & \IF \ \R_{s + 4} = \Box \cr
\ & \THEN \ 3L + (r + 2)s + 16r + 46 \cr
\ & \ELSE \ 3L + (r + 2)s + 13r + 43 \cr
\ & \OR \ldots \ 3L + (r + 2)s + 13r + 3k + 43 \ldots \cr
\ & \OR \ 3L + (r + 2)s + 16r + 43 \cr
3L + (r + 2)s + 13r + 43 & \PS{s + 4}{a_0} \cr
3L + (r + 2)s + 13r + 44 & \PA{0}{a_0} \cr
3L + (r + 2)s + 13r + 45 & \GOTO \ 3L + (r + 2)s + 13r + 42 \cr
\multicolumn{2}{c}{\vdots} \cr
3L + (r + 2)s + 16r + 43 & \PS{s + 4}{a_r} \cr
3L + (r + 2)s + 16r + 44 & \PA{0}{a_r} \cr
3L + (r + 2)s + 16r + 45 & \GOTO \ 3L + (r + 2)s + 13r + 42 \cr
\end{array}
\]
(Recover $\R_0$ from $\R_{s + 4}$. Upon completion, the content in $\R_0$ is $\zeta$, and $\R_{s + 4} = \Box$.)
%%
\item Instructions $3L + (r + 2)s + 16r + 46$ - $3L + (r + 2)s + 16r + 47$ are
\[
\begin{array}{rl}
3L + (r + 2)s + 16r + 46 & \IF \ \R_{s + 3} = \Box \cr
\ & \THEN \ 3L + (r + 2)s + 16r + 48 \cr
\ & \ELSE \ 3L + (r + 2)s + 16r + 47 \ldots \cr
\ & \OR \ 3L + (r + 2)s + 16r + 47 \cr
3L + (r + 2)s + 16r + 47 & \PRINT
\end{array}
\]
(If $\R_{s + 3} \neq \Box$, which means that $\zeta \in W$, as discussed earlier in part (6), then print it out.)
%%
\item Instruction $3L + (r + 2)s + 16r + 48$ - $3L + (r + 2)s + 17r + 49$ are
\[
\begin{array}{rl}
3L + (r + 2)s + 16r + 48 & \PS{s + 3}{a_0} \cr
\multicolumn{2}{c}{\vdots} \cr
3L + (r + 2)s + 17r + 48 & \PS{s + 3}{a_r} \cr
3L + (r + 2)s + 17r + 49 & \IF \ \R_{s + 3} = \Box \cr
\ & \THEN \ 3L + (r + 2)s + 17r + 50 \cr
\ & \ELSE \ 3L + (r + 2)s + 16r + 48 \ldots \cr
\ & \OR \ 3L + (r + 2)s + 16r + 48
\end{array}
\]
(Clear $\R_{s + 3}$.)
%%
\item Instruction $3L + (r + 2)s + 17r + 50$ - $3L + (r + 2)s + 17r + 58$ are
\[
\begin{array}{rl}
3L + (r + 2)s + 17r + 50 & \IF \ \R_{s + 1} = \Box \cr
\ & \THEN \ 3L + (r + 2)s + 17r + 55 \cr
\ & \ELSE \ 3L + (r + 2)s + 17r + 51 \ldots \cr
\ & \OR \ 3L + (r + 2)s + 17r + 51 \cr
3L + (r + 2)s + 17r + 51 & \PS{s + 1}{a_0} \cr
3L + (r + 2)s + 17r + 52 & \PA{1}{a_0} \cr
3L + (r + 2)s + 17r + 53 & \PA{s + 3}{a_0} \cr
3L + (r + 2)s + 17r + 54 & \GOTO \ 3L + (r + 2)s + 17r + 50 \cr
3L + (r + 2)s + 17r + 55 & \IF \ \R_1 = \Box \ \THEN \ 14 \cr
\ & \ELSE \ 3L + (r + 2)s + 17r + 56 \ldots \cr
\ & \OR \ 3L + (r + 2)s + 17r + 56 \cr
3L + (r + 2)s + 17r + 56 & \PS{1}{a_0} \cr
3L + (r + 2)s + 17r + 57 & \PA{s + 1}{a_0} \cr
3L + (r + 2)s + 17r + 58 & \GOTO \ 3L + (r + 2)s + 17r + 55
\end{array}
\]
(Restore the timer: Copy the content in $\R_{s + 1}$ to $\R_{s + 3}$, using $\R_1$ as a temporary storage. Upon completion, go back to label $14$ in part (2) to start a new iteration with the string next to $\zeta$ in lexicographic order.)
%%
\item Instruction $3L + (r + 2)s + 17r + 59$ is
\[
\begin{array}{rl}
3L + (r + 2)s + 17r + 59 & \HALT
\end{array}
\]\nolinebreak\hfill$\talloblong$
\end{enumerate}
%
\item* \textbf{Solution to Exercise 2.12.} Let $\p$ be a program that decides the set $W$, and let $\p_W$ be a program that lexicographically enumerates $W$.\\
\ \\
We prove that $W$ is R-decidable if and only if $W$ is lexicographically R-enumerable by showing how to construct $\p_W$ from $\p$ and then $\p$ from $\p_W$ in the following.\\
\ \\
\emph{Construct $\p_W$ from $\p$.} The construction of $\p_W$ from $\p$ there in part (b) of Exercise 2.10 already proved this claim.\\
\ \\
\emph{Construct $\p$ from $\p_W$.} The idea is that, for $\zeta \in \mathcal{A}^\ast$, we decide whether $\zeta \in W$ by scanning the output list of $\p_W$. Since $\p_W$ lexicographically enumerates $W$, it is easy to check whether $\zeta$ occurs on the list: if there is some $\xi \in \mathcal{A}^\ast$ occurring on the list with $l(\xi) > l(\zeta)$ (cf. the proof of 1.5 for the definition of $l$) but $\zeta$ has not yet showed up, then we conclude that $\zeta$ does not occur on the list and hence $\zeta \not\in W$.\\
\ \\
Assume that in $\p_W$ there are $L + 2$ instructions where $L \geq 0$,\footnote{One for $\PRINT$, and one for $\HALT$.} that there is exactly one print-instruction in it, and that all registers used it are among $\R_0, \ldots, \R_s$.\\
\ \\
Let us preserve the following registers for special purposes:
\begin{enumerate}[1)]
\item $\R_{s + 1}$ contains a copy of the input $\zeta$, and is used to compare to the content $\xi$ in $\R_0$ when $\p_W$ issues $\PRINT$.
%%
\item $\R_{s + 2}$ serves as a temporary storage for $\zeta$ during comparisons of $\zeta$ against $\xi$, and is used to recover $\zeta$ if necessary.
%%
\item $\R_{s + 3}$ serves as a temporary storage for $\xi$ during comparisons of $\zeta$ against $\xi$, and is used to recover $\xi$ if necessary.
\end{enumerate}
\ \\
There are $L + 30r + 45$ instructions in $\p$, where
\begin{enumerate}[(1)]
\item Instructions $0$ - $6r + 7$ are
\[
\begin{array}{rl}
0 & \IF \ \R_0 = \Box \ \THEN \ 3r + 4 \ \ELSE \ 1 \ \OR \ldots \ 3k + 1 \ldots \cr
\ & \OR \ 3r + 1 \cr
1 & \PS{0}{a_0} \cr
2 & \PA{1}{a_0} \cr
3 & \GOTO \ 0 \cr
\multicolumn{2}{c}{\vdots} \cr
3r + 1 & \PS{0}{a_r} \cr
3r + 2 & \PA{1}{a_r} \cr
3r + 3 & \GOTO \ 0 \cr
3r + 4 & \IF \ \R_1 = \Box \ \THEN \ 6r + 8 \ \ELSE \ 3r + 5 \cr
\ & \OR \ldots \ 3r + 3k + 5 \ldots \ \OR \ 6r + 5 \cr
3r + 5 & \PS{1}{a_0} \cr
3r + 6 & \PA{s + 1}{a_0} \cr
3r + 7 & \GOTO \ 3r + 4 \cr
\multicolumn{2}{c}{\vdots} \cr
6r + 5 & \PS{1}{a_r} \cr
6r + 6 & \PA{s + 1}{a_r} \cr
6r + 7 & \GOTO \ 3r + 4
\end{array}
\]
(Move the input $\zeta$ into $\R_{s + 1}$, using $\R_1$ as a temporary storage. Upon completion, the content in $\R_0$ is $\Box$, and that in $\R_{s + 1}$ is $\zeta$.)
%%
\item Instructions $6r + 8$ - $L + 6r + 10$ are derived from $\p_W$: For $0 \leq l \leq L$, the instruction at label $l + 6r + 8$ is obtained from the one at label $l$ in $\p_W$ by increasing all labels in it by $6r + 8$.\\
\ \\
Instructions $L + 6r + 9$ - $L + 6r + 10$ are
\[
\begin{array}{rl}
L + 6r + 9 & \PA{0}{a_0} \cr
L + 6r + 10 & \GOTO \ L + 30r + 43
\end{array}
\]
Suppose in the resulting program fragment the print-instruction is at label $L_0$:
\[
\begin{array}{rl}
L_0 & \PRINT
\end{array}
\]
Then it is replaced by
\[
\begin{array}{rl}
L_0 & \GOTO \ L + 6r + 11
\end{array}
\]
(Whenever $\p_W$ is about to print the content $\xi$ in $\R_0$, compare it against $\zeta$, which is the content in $\R_{s + 1}$, in the next part. If $\p_W$ is to halt, then $\zeta \not \in W$, hence the last two instructions in this part.)
%%
\item Instructions $L + 6r + 11$ - $L + 12 + 17$ are
\[
\begin{array}{rl}
L + 6r + 11 & \IF \ \R_{s + 1} = \Box \ \THEN \ L + 30r + 43 \cr
\ & \ELSE \ L + 6r + 12 \cr
\ & \OR \ldots \ L + 6r + 6k + 12 \ldots \ \OR \ L + 12r + 12 \cr
L + 6r + 12 & \IF \ \R_0 = \Box \ \THEN \ L + 24r + 36 \cr
\ & \ELSE \ L + 6r + 13 \cr
\ & \underbrace{\OR \ L + 18r + 26 \ldots \ \OR \ L + 18r + 26}_{r\mbox{\scriptsize-times}} \cr
L + 6r + 13 & \PS{0}{a_0} \cr
L + 6r + 14 & \PA{s + 2}{a_0} \cr
L + 6r + 15 & \PS{s + 1}{a_0} \cr
L + 6r + 16 & \PA{s + 3}{a_0} \cr
L + 6r + 17 & \GOTO \ L + 6r + 11 \cr
\multicolumn{2}{c}{\vdots} \cr
L + 6r + 6k + 12 & \IF \ \R_0 = \Box \ \THEN \ L + 24r + 36 \cr
\ & \ELSE \ \underbrace{L + 12r + 18 \ldots \OR \ L + 12r + 18}_{k\mbox{\scriptsize-times}} \cr
\ & \OR \ L + 6r + 6k + 13 \cr
\ & \underbrace{\OR \ L + 18r + 26 \ldots \ \OR \ L + 18r + 26}_{(r - k)\mbox{\scriptsize-times}} \cr
L + 6r + 6k + 13 & \PS{0}{a_k} \cr
L + 6r + 6k + 14 & \PA{s + 2}{a_k} \cr
L + 6r + 6k + 15 & \PS{s + 1}{a_k} \cr
L + 6r + 6k + 16 & \PA{s + 3}{a_k} \cr
L + 6r + 6k + 17 & \GOTO \ L + 6r + 11 \cr
\multicolumn{2}{c}{\vdots} \cr
L + 12r + 12 & \IF \ \R_0 = \Box \ \THEN \ L + 24r + 36 \cr
\ & \ELSE \ \underbrace{L + 12r + 18 \ldots \OR \ L + 12r + 18}_{r\mbox{\scriptsize-times}} \cr
\ & \OR \ L + 12r + 13 \cr
L + 12r + 13 & \PS{0}{a_r} \cr
L + 12r + 14 & \PA{s + 2}{a_r} \cr
L + 12r + 15 & \PS{s + 1}{a_r} \cr
L + 12r + 16 & \PA{s + 3}{a_r} \cr
L + 12r + 17 & \GOTO \ L + 6r + 11 \cr
\end{array}
\]
(On entrance to this part, the content in $\R_{s + 1}$ is $\zeta$, while the content in $\R_0$ is, say, some $\xi \in \mathcal{A}^\ast$. Let us consider the following cases:
\begin{enumerate}[1)]
\item $\zeta = \Box$.
\begin{enumerate}[1$^\circ$]
\item $\xi = \Box$. The answer is ``yes'', so print out the content in $\R_0$.
%%%%
\item $\xi \neq \Box$. The answer is ``no'', so print out the content in $\R_0$.
\end{enumerate}
%%%
\item $\zeta$ ends with some $a_i$.
\begin{enumerate}[1$^\circ$]
\item $\xi = \Box$. Go on to compare the next string on the output list against $\zeta$.
%%%%
\item $\xi$ ends with $a_i$. Go on to compare the substrings of $\xi$ and of $\zeta$ with their common last character $a_i$ deleted.
%%%%
\item $\xi$ ends with $a_j$, where $j < i$.
\begin{itemize}
\item[(a)] $l(\zeta) < l(\xi)$. The answer is ``no'', so print out the content in $\R_0$.
%%%%%
\item[(b)] $l(\zeta) \geq l(\xi)$. Go on to compare the next string on the output list against $\zeta$.
\end{itemize}
See part (4).
%%%%
\item $\xi$ ends with $a_j$, where $j > i$.
\begin{itemize}
\item[(a)] $l(\zeta) \leq l(\xi)$. The answer is ``no'', so print out the content in $\R_0$.
%%%%%
\item[(b)] $l(\zeta) > l(\xi)$. Go on to compare the next string on the output list against $\zeta$.
\end{itemize}
See part (5).
\end{enumerate}
\end{enumerate})
%%
\item Instructions $L + 12r + 18$ - $L + 18r + 25$ are
\[
\begin{array}{rl}
L + 12r + 18 & \IF \ \R_{s + 1} = \Box \ \THEN \ L + 12r + 19 \cr
\ & \ELSE \ L + 12r + 20 \cr
\ & \OR \ldots \ L + 12r + 3k + 20 \ldots \cr
\ & \OR \ L + 15r + 20 \cr
L + 12r + 19 & \IF \ \R_0 = \Box \ \THEN \ L + 24r + 36 \cr
\ & \ELSE \ L + 30r + 43 \ldots \cr
\ & \OR \ L + 30r + 43 \cr
L + 12r + 20 & \PS{s + 1}{a_0} \cr
L + 12r + 21 & \PA{s + 3}{a_0} \cr
L + 12r + 22 & \GOTO \ L + 15r + 22 \cr
\multicolumn{2}{c}{\vdots} \cr
L + 12r + 3k + 20 & \PS{s + 1}{a_k} \cr
L + 12r + 21 & \PA{s + 3}{a_k} \cr
L + 12r + 22 & \GOTO \ L + 15r + 22 \cr
\multicolumn{2}{c}{\vdots} \cr
L + 15r + 20 & \PS{s + 1}{a_r} \cr
L + 15r + 21 & \PA{s + 3}{a_r} \cr
L + 15r + 22 & \IF \ \R_0 = \Box \ \THEN \ L + 24r + 36 \cr
\ & \ELSE \ L + 15r + 23 \cr
\ & \OR \ldots \ L + 15r + 3k + 23 \ldots \cr
\ & \OR \ L + 18r + 23 \cr
L + 15r + 23 & \PS{0}{a_0} \cr
L + 15r + 24 & \PA{s + 2}{a_0} \cr
L + 15r + 25 & \GOTO \ L + 12r + 18 \cr
\multicolumn{2}{c}{\vdots} \cr
L + 18r + 23 & \PS{0}{a_r} \cr
L + 18r + 24 & \PA{s + 2}{a_r} \cr
L + 18r + 25 & \GOTO \ L + 12r + 18 \cr
\end{array}
\]
(Compare the lengths of $\zeta$ and of $\xi$, where $\R_{s + 1} = \zeta$ and $\R_0 = \xi$. If $l(\zeta) < l(\xi)$, then go to the print-instruction with the answer ``no''; otherwise, recover the contents in $\R_0$ and in $\R_{s + 1}$ (see part (6)) before the next comparison.)
%%
\item Instructions $L + 18r + 26$ - $L + 24r + 34$ are
\[
\begin{array}{rl}
L + 18r + 26 & \IF \ \R_{s + 1} = \Box \ \THEN \ L + 18r + 27 \cr
\ & \ELSE \ L + 18r + 29 \cr
\ & \OR \ldots \ L + 18r + 3k + 29 \ldots \cr
\ & \OR \ L + 21r + 29 \cr
L + 18r + 27 & \PA{0}{a_0} \cr
L + 18r + 28 & \GOTO \ L + 30r + 43 \cr
L + 18r + 29 & \PS{s + 1}{a_0} \cr
L + 18r + 30 & \PA{s + 3}{a_0} \cr
L + 18r + 31 & \GOTO \ L + 21r + 31 \cr
\multicolumn{2}{c}{\vdots} \cr
L + 18r + 3k + 29 & \PS{s + 1}{a_k} \cr
L + 18r + 3k + 30 & \PA{s + 3}{a_k} \cr
L + 18r + 3k + 31 & \GOTO \ L + 21r + 31 \cr
\multicolumn{2}{c}{\vdots} \cr
L + 21r + 29 & \PS{s + 1}{a_r} \cr
L + 21r + 30 & \PA{s + 3}{a_r} \cr
L + 21r + 31 & \IF \ \R_0 = \Box \ \THEN \ L + 24r + 35 \cr
\ & \ELSE \ L + 21r + 32 \cr
\ & \OR \ldots \  L + 21r + 3k + 32 \ldots \cr
\ & \OR \ L + 24r + 32 \cr
L + 21r + 32 & \PS{0}{a_0} \cr
L + 21r + 33 & \PA{s + 2}{a_0} \cr
L + 21r + 34 & \GOTO \ L + 18r + 26 \cr
\multicolumn{2}{c}{\vdots} \cr
L + 24r + 32 & \PS{0}{a_r} \cr
L + 24r + 33 & \PA{s + 2}{a_r} \cr
L + 24r + 34 & \GOTO \ L + 18r + 26 \cr
\end{array}
\]
(Compare the lengths of $\zeta$ and of $\xi$, where $\R_{s + 1} = \zeta$ and $\R_0 = \xi$. If $l(\zeta) \leq l(\xi)$ then go to the print-instruction with the answer ``no''; otherwise, recover the contents in $\R_0$ and in $\R_{s + 1}$ (see next part) before the next comparison.)
%%
\item Instructions $L + 24r + 35$ - $L + 30r + 42$ are
\[
\begin{array}{rl}
L + 24r + 35 & \IF \ \R_{s + 2} = \Box \ \THEN \ L + 27r + 39 \cr
\ & \ELSE \ L + 24r + 36 \cr
\ & \OR \ldots \ L + 24r + 3k + 36 \ldots \cr
\ & \OR \ L + 27r + 36 \cr
L + 24r + 36 & \PS{s + 2}{a_0} \cr
L + 24r + 37 & \PA{0}{a_0} \cr
L + 24r + 38 & \GOTO \ L + 24r + 35 \cr
\multicolumn{2}{c}{\vdots} \cr
L + 27r + 36 & \PS{s + 2}{a_r} \cr
L + 27r + 37 & \PA{0}{a_r} \cr
L + 27r + 38 & \GOTO \ L + 24r + 35 \cr
L + 27r + 39 & \IF \ \R_{s + 3} = \Box \ \THEN \ L_0 + 1 \cr
\ & \ELSE \ L + 27r + 40 \cr
\ & \OR \ldots \ L + 27r + 3k + 40 \ldots \cr
\ & \OR \ L + 30r + 40 \cr
L + 27r + 40 & \PS{s + 3}{a_0} \cr
L + 27r + 41 & \PA{s + 1}{a_0} \cr
L + 27r + 42 & \GOTO \ L + 27r + 39 \cr
\multicolumn{2}{c}{\vdots} \cr
L + 30r + 40 & \PS{s + 3}{a_r} \cr
L + 30r + 41 & \PA{s + 1}{a_r} \cr
L + 30r + 42 & \GOTO \ L + 27r + 39 \cr
\end{array}
\]
(Recover the contents of $\R_0$ and of $\R_{s + 1}$ from $\R_{s + 2}$ and from $\R_{s + 3}$, respectively. And then go back to $\p_W$.)
%%
\item Instruction $L + 30r + 43$ - $L + 30r + 44$ are
\[
\begin{array}{rl}
L + 30r + 43 & \PRINT \cr
L + 30r + 44 & \HALT
\end{array}
\]\nolinebreak\hfill$\talloblong$
\end{enumerate}
%
\item \textbf{Solution to Exercise 2.13.} We will reduce this \emph{computability} problem to the following \emph{decision} one:
\begin{quote}
There is no register program $\p^\prime$ over $\{ a_0, a_1 \}$, in this new setting, that decides the set $\{ \Box, a_0, a_1 \}$, i.e. the set of strings of length less than $2$.
\end{quote}
First we prove the above claim, and then use it to prove the one mentioned in this exercise.\\
\ \\
But before we argue that, it is helpful to define some term explicitly. Let us say that, for $k \in \mathbb{Z}^+$, a program, possibly with input, is \emph{at the $k$th step} if it has executed $k - 1$ instructions since it was started with that input (if any) and, has not yet reached the halt-instruction.\\
\ \\
Now let us return to our decision problem. Suppose the program $\p^\prime$ decides whether the length of a string is less than $2$. Then consider the four cases of inputs with which $\p^\prime$ is started: $a_0$, $a_1$, $a_0a_1$ and $a_1a_0$. By definition, we have
\begin{center}
\begin{tabular}{l}
$\p^\prime : a_0 \to \Box$, \cr
$\p^\prime : a_1 \to \Box$, \cr
$\p^\prime : a_0a_1 \to \eta_0$ and \cr
$\p^\prime : a_1a_0 \to \eta_1$,
\end{tabular}
\end{center}
where $\eta_0 \neq \Box$ and $\eta_1 \neq \Box$. In particular, the $a_0$ (and $a_1$) in $\R_0$ must be deleted at some time during the computation of $\p^\prime$ with the input $a_0$ (and the input $a_1$, respectively).
Thus, in $\p^\prime$ there must occur an instruction of the form
\[
\begin{array}{rl}
L_0 & \PS{0}{a_0}
\end{array}
\]
as it is the only way to delete an $a_0$ from $\R_0$. Likewise, in $\p^\prime$ there must be an instruction of the form
\[
\begin{array}{rl}
L_1 & \PS{0}{a_1}
\end{array}
\]
\ \\
Let us say that $\p^\prime$ with input $a_0$ (and $a_1$) executes $L_0 \ \PS{0}{a_0}$ (and $L_1 \ \PS{0}{a_1}$, respectively) at the $k_0$th (and $k_1$th, respectively) step and, just prior to this step, the content in $\R_0$ is $a_0$ (and $a_1$, respectively); hence after executing that instruction, $\R_0 = \Box$. Pick $k := \min \{ k_0, k_1 \}$. It can be easily shown by induction that for all four cases, the program $\p^\prime$ executes the same instruction at each step before the $k$th and, the register machine suffers the same changes until $\p^\prime$ reaches the $k$th step.\\
\ \\
An absurdity is near: If $k = k_0$, then just after the $k$th step $\R_0 = a_1$ for both cases in which $\p^\prime$ is started with input $a_1$ and with input $a_1a_0$. Similarly, if $k = k_1$, then $\R_0 = a_0$ for both cases of $a_0$ and $a_0a_1$. In other words, $\p^\prime$ \emph{cannot} distinguish $a_0$ from $a_0a_1$, neither can it distinguish $a_1$ from $a_1a_0$, contrary to our previous assumption.\\
\ \\
Now we are ready to show that there is no program $\p$ with the behavior requested in this exercise. Suppose there were a program $\p$ such that $\p : \zeta \to \zeta\zeta$ for all $\zeta \in \{ a_0, a_1 \}^\ast$, then $\p^\prime$ could be obtained from $\p$: Without loss of generality, let us assume that there is exactly one print-instruction in $\p$.\footnote{It is easy to convert a program with multiple $\PRINT$'s to one with exactly one $\PRINT$.} If the print-instruction is at label $L$:
\[
\begin{array}{rl}
L & \PRINT
\end{array}
\]
Then replace it with
\[
\begin{array}{rl}
L & \PS{0}{a_0} \cr
L + 1 & \PS{0}{a_1} \cr
L + 2 & \PS{0}{a_1} \cr
L + 3 & \PS{0}{a_0} \cr
L + 4 & \PRINT
\end{array}
\]
and, of course, increase all labels $l$ in $\p$ which are greater than $L$ by the amount of $4$.\\
\ \\
Consider the 7 cases of input with which $\p^\prime$ is started: $\Box$, $a_0$, $a_1$, $a_0a_0$, $a_0a_1$, $a_1a_0$ and $a_1a_1$. Just before the execution of instruction $L$, the contents of $\R_0$ are, by definition, $\Box$, $a_0a_0$, $a_1a_1$, $a_0a_0a_0a_0$, $a_0a_1a_0a_1$, $a_1a_0a_1a_0$ and $a_1a_1a_1a_1$, respectively. (Recall the behavior of $\p$.) After executing instruction $L + 3$, they become $\Box$, $\Box$, $\Box$, $a_0a_0$, $a_0a_1$, $a_1$ and $a_1a_1$. In detail, we list below the contents of $\R_0$ after executing each of instructions $L$ - $L + 3$ for each case.
\begin{center}
\begin{tabular}{l||lllll}
7 cases of input & just before $L$   & $L$            & $L + 1$     & $L + 2$     & $L + 3$  \cr\hline
$\Box$           & $\Box$         & $\Box$         & $\Box$      & $\Box$      & $\Box$   \cr
$a_0$            & $a_0a_0$       & $a_0$          & $a_0$       & $a_0$       & $\Box$   \cr
$a_1$            & $a_1a_1$       & $a_1a_1$       & $a_1$       & $\Box$      & $\Box$   \cr
$a_0a_0$         & $a_0a_0a_0a_0$ & $a_0a_0a_0$    & $a_0a_0a_0$ & $a_0a_0a_0$ & $a_0a_0$ \cr
$a_0a_1$         & $a_0a_1a_0a_1$ & $a_0a_1a_0a_1$ & $a_0a_1a_0$ & $a_0a_1a_0$ & $a_0a_1$ \cr
$a_1a_0$         & $a_1a_0a_1a_0$ & $a_1a_0a_1$    & $a_1a_0$    & $a_1a_0$    & $a_1$    \cr
$a_1a_1$         & $a_1a_1a_1a_1$ & $a_1a_1a_1a_1$ & $a_1a_1a_1$ & $a_1a_1$    & $a_1a_1$
\end{tabular}
\end{center}
For all $\zeta \in \mathcal{A}^\ast$ of length no less than $2$, on the other hand, it is clear that $\p^\prime : \zeta \to \eta$ for some $\eta \neq \Box$. It follows that $\p^\prime$ decides whether the length of a given string is less than $2$, a contradiction.\nolinebreak\hfill$\talloblong$\\
\ \\
\textit{Remark.} If $\mathcal{A} = \{ a_0 \}$, however, the function $f : \mathcal{A}^\ast \to \mathcal{A}^\ast$, $f(\zeta) = \zeta\zeta$ is actually R-computable:
\[
\begin{array}{rl}
0 & \IF \ \R_0 = \Box \ \THEN \ 5 \ \ELSE \ 1 \cr
1 & \PS{0}{a_0} \cr
2 & \PA{1}{a_0} \cr
3 & \PA{2}{a_0} \cr
4 & \GOTO \ 0 \cr
5 & \IF \ \R_1 = \Box \ \THEN \ 9 \ \ELSE \ 6 \cr
6 & \PS{1}{a_0} \cr
7 & \PA{0}{a_0} \cr
8 & \GOTO \ 5 \cr
9 & \IF \ \R_2 = \Box \ \THEN \ 13 \ \ELSE \ 10 \cr
10 & \PS{2}{a_0} \cr
11 & \PA{0}{a_0} \cr
12 & \GOTO \ 9 \cr
13 & \PRINT \cr
14 & \HALT
\end{array}
\]
\end{enumerate}
%End of Section X.2--------------------------------------------------------------------------------
\
\\
\\
%Section X.3---------------------------------------------------------------------------------------
{\large \S3. The Halting Problem for Register Machines}
\begin{enumerate}[1.]
\item \textbf{Note to the 5th Paragraph on Page 166.} A brief explanation for ``the set $\Pi$ of register programs lies in $\mathcal{P}$'' is given as follows: Recall that $\mathcal{A} = \{ a_0, \ldots, a_r \}$ and
\[
\mathcal{B} := \mathcal{A} \cup \{ A, B, C, \ldots, X, Y, Z \} \cup \{ 0, 1, \ldots, 8, 9 \} \cup \{ =, +, -, \Box, \S \}.
\]
Hence the size of $\mathcal{B}$ is $(r + 42)$.\\
\ \\
Given $\zeta = \underbrace{a_0 \ldots a_0}_{n\mbox{\scriptsize-times}}$, if $n = 0$ then $\zeta \not\in \Pi$ (trivial). If $n > 0$, then we determine the integer $m$ such that
\[
\frac{(r + 42)^{m + 1} - 1}{r + 41} \leq n < \frac{(r + 42)^{m + 2} - 1}{r + 41},
\]
where the leftmost term in the above inequality is the sum of the summation
\[
1 + (r + 42) + \ldots + (r + 42)^m,
\]
let us denote it by $s$. Then identify the $(n - s)$th word in the lexicographic ordering on the set of strings of length $(m + 1)$ over $\mathcal{B}$. It is straightforward to check whether that word is the encoding (in $\mathcal{B}^\ast$) of a register program over $\mathcal{A}$. In summary, it is conceivable that the above procedure can be carried out in polynomial time.
%
\item \textbf{Note to the First Paragraph on Page 167.} ``$\mathrm{SAT} \not\in \mathcal{P}$ if and only if $\mathcal{P} \neq \mathcal{NP}$'' is the famous \emph{Cook's Theorem}.
%
\item \textbf{Solution to Exercise 3.5.}
\begin{enumerate}[(a)]
\item Suppose there is an $a \in M$ such that $D = M_a$. Then consider the membership of $a$ to $D$: If $a \in D$, i.e. not $Raa$, then by the definition of $M_a$ we have $a \not\in M_a = D$; however, if $a \not\in D$, i.e. $Raa$, then we have $a \in M_a = D$. Either way we arrive at a contradiction.
%%
\item For $\xi \in M$ let $M_\xi := \{ \eta \ | \ R\xi\eta \}$. If there is a program $\p$, of which the G\"{o}del number is $\xi_\p$, that enumerates $D$, then we have $D = M_{\xi_\p}$ by definition, but that contradicts the result we obtained from part (a).
%%
\item If $W \subset \mathcal{A}^\ast$ is R-decidable, then from part (b) of Exercise 2.10 it follows that $\mathcal{A}^\ast \setminus W$ is R-enumerable; and furthermore by part (b) of Exercise 2.11, there is a program $\p$ such that $\p: \eta \to \infty$ if and only if $\eta \in W$. Take the G\"{o}del number $\xi_\p$ of $\p$, we have
\begin{center}
for all $\eta \in \mathcal{A}^\ast$, $\eta \in W$ if and only if $R\xi_\p\eta$,
\end{center}
i.e. $W = M_{\xi_\p}$.\\
\ \\
On the other hand,
\[
\begin{array}{lll}
D & = & \{ \xi \in M \ |\ \mbox{not $R\xi\xi$} \} \cr
\ & = & \{ \xi \in \mathcal{A}^\ast \ |\ \mbox{$\xi$ is the G\"{o}del number of a program $\p$}\cr
\ & \ & \ \ \ \ \ \ \ \ \ \ \ \ \ \ \mbox{with $\p: \xi \to \halt$} \} \cr
\ & = & \{ \xi_\p \ |\ \mbox{$\p$ is a program over $\mathcal{A}$ and $\p: \xi_\p \to \halt$} \} \cr
\ & = & \Pi_\scripttext{halt}^\prime.
\end{array}
\]\nolinebreak\hfill$\talloblong$
\end{enumerate}
\textit{Remark.} In part (c), for $\xi \in \mathcal{A}^\ast$ that is not the G\"{o}del number of any program,
\begin{center}
$R\xi\eta$ for all $\eta \in \mathcal{A}^\ast$,
\end{center}
and thus $M_\xi = \mathcal{A}^\ast$.\\
\ \\
In fact, we could have defined the binary relation $R$ as
\begin{center}
\begin{tabular}{lll}
$R\xi\eta$ & :iff & $\xi$ is the G\"{o}del number of a program $\p$ with $\p : \eta \to \halt$,
\end{tabular}
\end{center}
and we would obtain the same result: Let $W \subset \mathcal{A}^\ast$ be R-decidable. By part (b) of Exercise 2.10, $W$ is R-enumerable; further by part (b) of Exercise 2.11, there is a program $\p$ such that $\p : \eta \to \halt$ if and only if $\eta \in W$. Thus we have
\[
W = M_{\xi_\p},
\]
where $\xi_\p$ is the G\"{o}del number of $\p$.\\
\ \\
On the other hand,
\[
\begin{array}{lll}
D & = & \{ \xi \in M \ | \ \mbox{not $R\xi\xi$} \} \cr
\ & = & \{ \xi \in \mathcal{A}^\ast \ | \ \mbox{$\xi$ is \emph{not} the G\"{o}del number of a program $\p$} \cr
\ & \ & \ \ \ \ \ \ \ \ \ \ \ \ \ \ \mbox{with $\p: \xi \to \halt$} \} \cr
\ & = & \mathcal{A}^\ast \setminus \{ \xi \in \mathcal{A}^\ast \ | \ \mbox{$\xi$ is the G\"{o}del number of a program $\p$} \cr
\ & \ & \ \ \ \ \ \ \ \ \ \ \ \ \ \ \ \ \ \ \ \ \mbox{with $\p : \xi \to \halt$} \} \cr
\ & = & \mathcal{A}^\ast \setminus \{ \xi_\p \ |\ \mbox{$\p$ is a program over $\mathcal{A}$ and $\p: \xi_\p \to \halt$} \} \cr
\ & = & \mathcal{A}^\ast \setminus \Pi^\prime_\scripttext{halt}
\end{array}
\]
From part (a) we have that $\mathcal{A}^\ast \setminus \Pi^\prime_\scripttext{halt}$ is R-undecidable and, much more importantly,
\begin{center}
$\Pi^\prime_\scripttext{halt}$ is R-undecidable,
\end{center}
cf. Exercise 2.9.
\end{enumerate}
%End of Section X.3--------------------------------------------------------------------------------
\
\\
\\
%Section X.4---------------------------------------------------------------------------------------
{\large \S4. The Undecidability of First-Order Logic}
\begin{enumerate}[1.]
\item \textbf{Note to the Proof of 4.1 on Page 169.} Note that if $\p : \Box \to \halt$, then it runs for $(s_\p + 1)$ steps, where $s_\p \geq 0$. Furthermore, when we talk about ``the configuration of $\p$ after $s$ steps'', note that $0 \leq s \leq s_\p$; on the other hand, its definition should be refined such that $\p$ runs for at least \emph{$(s + 1)$ steps}, instead of $s$ steps, after started with $\Box$. \\
\ \\
Also note that $\mathfrak{A}_\p \models R \overline{s_\p} \overline{k} \overline{m_0} \ldots \overline{m_n}$ if and only if $\p : \Box \to \halt$. However, for an $S$-structure $\mathfrak{A}$ with $\mathfrak{A} \models \psi_\p$, $\mathfrak{A} \models R \overline{s} \overline{L} \overline{m_0} \ldots \overline{m_n}$ does \emph{not} necessarily imply that $(L, m_0, \ldots, m_n)$ is the configuration of $\p$ after $s$ steps.\\
\ \\
On the other hand, in $\psi_0$, ``$<$ is an ordering'' is the conjunction of sentences in $\Phi_\scripttext{ord}$ from III.6.4.
%
\item \textbf{Corollaries to 4.1.} 
\begin{enumerate}[(a)]
\item Since for $\varphi \in L_0^{S_\infty}$,
\begin{center}
$\models \varphi$ \ \ \ iff \ \ \ not $\sat \neg \varphi$,
\end{center}
we have that the set of unsatisfiable $S_\infty$-sentences is R-undecidable, for otherwise we could decide whether $\models \varphi$ by deciding whether $\neg \varphi$ is unsatisfiable.
%%
\item The problem ``Given $\Phi \subset L_0^{S_\infty}$ and $\varphi \in L_0^{S_\infty}$, whether $\Phi \models \varphi$'' is R-undecidable.
%%
\item The problem ``Given $\Phi \subset L_0^{S_\infty}$, whether $\con \Phi$'' is R-undecidable, because
\begin{center}
$\Phi \models \varphi$ \ \ \ iff \ \ \ $\inc \Phi \cup \{ \neg\varphi \}$.
\end{center}
\end{enumerate}
%
\item \textbf{Solution to Exercise 4.2.} We first prove the base case, in which $s = 0$: By definition, $(0 , \ldots, 0)$ is the configuration after $0$ steps or, the initial configuration. Let us suppose that $\mathfrak{A}$ is an $S$-structure with $\mathfrak{A} \models \psi_\p$. Then it is vacuously true that $\overline{0}^\mathfrak{A}$ \emph{is pairwise distinct}, and by definition $\mathfrak{A} \models R\overline{0}\ldots \overline{0}$ holds.\\
\ \\
Induction step: Suppose that (2)(b) has been proved for $s = r$. It then suffices to show that $\mathfrak{A} \models R \, \overline{r + 1} \, \overline{L^+} \, \overline{m_0^+} \ldots \overline{m_n^+}$, and that $\overline{0}^\mathfrak{A}, \overline{1}^\mathfrak{A}, \ldots, \overline{r}^\mathfrak{A}, \overline{r + 1}^\mathfrak{A}$ are pairwise distinct, provided that $\mathfrak{A}$ is an $S$-structure with $\mathfrak{A} \models \psi_\p$, that $(L^+, m_0^+, \ldots, m_n^+)$ is the configuration after $(r + 1)$ steps, and the induction hypothesis:\\
\begin{tabular}{ll}
(+) & \begin{tabular}{l}
if $(L, m_0, \ldots, m_n)$ is the configuration after $r$ steps, then \cr
$\overline{0}^\mathfrak{A}, \overline{1}^\mathfrak{A}, \ldots, \overline{r}^\mathfrak{A}$ are pairwise distinct and $\mathfrak{A} \models R \overline{r} \overline{L} \overline{m_0} \ldots \overline{m_n}$.
\end{tabular}
\end{tabular}\\
\ \\
For this purpose, let us consider the instruction executed at step $(r + 1)$:
\begin{enumerate}[1)]
\item If it is an add-instruction, say $L \ \PA{i}{|}$, then we have $L = L^+ - 1$, and the configuration after $r$ steps is $(L^+ - 1, m_0^+, \ldots, m_i^+ - 1, \ldots, m_n^+)$. By hypothesis, $\mathfrak{A} \models R \, \overline{r} \, \overline{L^+ - 1} \, \overline{m_0^+} \ldots \overline{m_i^+ - 1} \ldots \overline{m_n^+}$ and $\overline{0}^\mathfrak{A}, \overline{1}^\mathfrak{A}, \ldots, \overline{r}^\mathfrak{A}$ are pairwise distinct. Note that the notations $\overline{L^+ - 1}$ and $\overline{m_i^+ - 1}$ are both well-defined: Since in this case $L^+ > 0$ and $m_i^+ > 0$, $\overline{L^+ - 1}$ and $\overline{m_i^+ - 1}$ are just, respectively, $\overline{L^+}$ and $\overline{m^+}$ with $f$ applied once less.\\
\ \\
As $\mathfrak{A} \models \psi_\p$ and, in particular, $\mathfrak{A} \models \psi_{\alpha_{L^+ - 1}}$, we have
\[
\overline{r}^\mathfrak{A} <^\mathfrak{A} \overline{r + 1}^\mathfrak{A}
\]
and
\[
\mathfrak{A} \models R \, \overline{r + 1} \, \overline{L^+} \, \overline{m_0^+} \ldots \overline{m_n^+}.
\]
Notice that the first result leads to that $\overline{0}^\mathfrak{A}, \overline{1}^\mathfrak{A}, \ldots, \overline{r}^\mathfrak{A}, \overline{r + 1}^\mathfrak{A}$ are pairwise distinct.
%%
\item If it is the instruction $L \ \PS{i}{|}$, then we have $L = L^+ - 1$, and the configuration after $r$ steps is $(L^+ - 1, m_0^+, \ldots, m_i^\prime, \ldots, m_n^+)$, where $m_i^\prime = m_i^+ + 1$ if $m_i^+ > 0$, and if $m_i^+ = 0$ then $m_i^\prime$ may be either $0$ or $1$. By hypothesis, $\mathfrak{A} \models R \, \overline{r} \, \overline{L^+ - 1} \, \overline{m_0^+} \ldots \overline{m_i^\prime} \ldots \overline{m_n^+}$ and $\overline{0}^\mathfrak{A}, \overline{1}^\mathfrak{A}, \ldots, \overline{r}^\mathfrak{A}$ are pairwise distinct.\\
\ \\
From $\mathfrak{A} \models \psi_\p$ and hence $\mathfrak{A} \models \psi_{\alpha_{L^+ - 1}}$, it follows that
\[
\overline{r}^\mathfrak{A} <^\mathfrak{A} \overline{r + 1}^\mathfrak{A}
\]
and
\[
\mathfrak{A} \models R \, \overline{r + 1} \, \overline{L^+} \, \overline{m_0^+} \ldots \overline{m_n^+}.
\]
The first result leads to that $\overline{0}^\mathfrak{A}, \overline{1}^\mathfrak{A}, \ldots, \overline{r}^\mathfrak{A}, \overline{r + 1}^\mathfrak{A}$ are pairwise distinct.
%%
\item If it is the instruction $L \ \IF \ \R_i = \Box \ \THEN \ L^\prime \ \ELSE \ L_0$, then we have
\[
L^+ = \begin{cases}
L^\prime,  & \mbox{if \(m_i^+ = 0\)}; \cr
L_0,       & \mbox{otherwise}.
\end{cases}
\]
And the configuration after $r$ steps is $(L, m_0^+, \ldots, m_n^+)$. By hypothesis, $\mathfrak{A} \models R \, \overline{r} \, \overline{L} \, \overline{m_0^+} \ldots \overline{m_n^+}$ and $\overline{0}^\mathfrak{A}, \overline{1}^\mathfrak{A}, \ldots, \overline{r}^\mathfrak{A}$ are pairwise distinct.\\
\ \\
As $\mathfrak{A} \models \psi_\p$ and furthermore $\mathfrak{A} \models \psi_{\alpha_L}$, we have
\[
\overline{r}^\mathfrak{A} <^\mathfrak{A} \overline{r + 1}^\mathfrak{A}
\]
and
\[
\mathfrak{A} \models R \, \overline{r + 1} \, \overline{L^+} \, \overline{m_0^+} \ldots \overline{m_n^+}.
\]
We also have that $\overline{0}^\mathfrak{A}, \overline{1}^\mathfrak{A}, \ldots, \overline{r}^\mathfrak{A}, \overline{r + 1}^\mathfrak{A}$ are pairwise distinct.
%%
\item Finally, if it is $L \ \PRINT$, then we have $L = L^+ - 1$, and the configuration after $r$ steps is $(L^+ - 1, m_0^+, \ldots, m_n^+)$. By hypothesis, $\mathfrak{A} \models R \, \overline{r} \, \overline{L^+ - 1} \, \overline{m_0^+} \ldots \overline{m_n^+}$ and $\overline{0}^\mathfrak{A}, \overline{1}^\mathfrak{A}, \ldots, \overline{r}^\mathfrak{A}$ are pairwise distinct.\\
\ \\
Since $\mathfrak{A} \models \psi_\p$, and hence $\mathfrak{A} \models \psi_{\alpha_{L^+ - 1}}$, it follows that
\[
\overline{r}^\mathfrak{A} <^\mathfrak{A} \overline{r + 1}^\mathfrak{A}
\]
and
\[
\mathfrak{A} \models R \, \overline{r + 1} \, \overline{L^+} \, \overline{m_0^+} \ldots \overline{m_n^+}.
\]
And from the first result we obtain that $\overline{0}^\mathfrak{A}, \overline{1}^\mathfrak{A}, \ldots, \overline{r}^\mathfrak{A}, \overline{r + 1}^\mathfrak{A}$ are pairwise distinct.\nolinebreak\hfill$\talloblong$
\end{enumerate}
%
\item \textbf{Solution to Exercise 4.3.} First note that 
\[
\{ \varphi \in L_0^{S_\infty} \ | \ \models \neg\varphi \}
\]
is the set of \emph{unsatisfiable} $S_\infty$-sentences, every element in which is the negation of a valid $S_\infty$-sentence. Hence it is enumerable and undecidable, as is the set of valid $S_\infty$-sentences (cf. 1.6 and the remark below the statement of 4.1).\\
\ \\
Suppose that the set
\[
L_0^{S_\infty} \setminus \{ \varphi \in L_0^{S_\infty} \ | \ \models \neg\varphi \}
\]
of satisfiable $S_\infty$-sentences were R-enumerable, then by Church's Thesis it would be enumerable. As the set $L_0^{S_\infty}$ of $S_\infty$-sentences is decidable (cf. part (b) of Exercise 1.3), the set of unsatisfiable $S_\infty$-sentences would be decidable (cf. Exercise 1.9), a contradiction.\nolinebreak\hfill$\talloblong$
%
\item \textbf{Solution to Exercise 4.4.} In the proof of 4.1, leave out $<$ entirely. So $S$ becomes $\{ R, f, c \}$. Note that (1) also holds in this new setting. Similarly, we provide an $S$-sentence $\psi_\p$ which describes the operations of $\p$ on $\Box$; we abbreviate $c, fc, ffc, \ldots$ by $\overline{0}, \overline{1}, \overline{2}, \ldots$, respectively.\\
\ \\
In contrary to (2) in the proof, here the following holds instead:\\
\begin{tabular}{ll}
(2)$^\prime$ & (a) \ $\mathfrak{A}_\p \models \psi_\p$. \cr
\ & (b) \ If $\mathfrak{A}$ is an $S$-structure with $\mathfrak{A} \models \psi_\p$ and $(L, m_0, \ldots, m_n)$ is \cr
\ & the configuration of $\p$ after $s$ steps, then $\mathfrak{A} \models R \overline{s} \overline{L} \overline{m_0} \ldots \overline{m_n}$.
\end{tabular}\\
Notice now in (2)$^\prime$(b) we are no longer sure of whether $\overline{0}^\mathfrak{A}, \overline{1}^\mathfrak{A}, \ldots, \overline{s}^\mathfrak{A}$ are pairwise distinct even if the premise holds.\\
\ \\
We set
\[
\psi_\p := R \overline{0} \ldots \overline{0} \land \psi_{\alpha_0} \land \ldots \land \psi_{\alpha_{k - 1}}.
\]
For $\alpha = \alpha_0, \ldots, \alpha_{k - 1}$, the sentence $\psi_\alpha$ describes the operation corresponding to instruction $\alpha$. $\psi_\alpha$ is defined in the following:\\
If $\alpha$ is the instruction $L \ \PA{i}{|}$, then let
\[
\begin{array}{rr}
\psi_\alpha := & \multicolumn{1}{l}{\forall x \forall y_0 \ldots \forall y_n (R x \overline{L} y_0 \ldots y_n \rightarrow R fx \overline{L + 1} y_0 \ldots y_{i - 1} fy_i y_{i + 1} \ldots y_n).}
\end{array}
\]
If $\alpha$ is the instruction $L \ \PS{i}{|}$, then let
\[
\begin{array}{rr}
\psi_\alpha := & \multicolumn{1}{l}{\forall x \forall y_0 \ldots \forall y_{i - 1} \forall y_{i + 1} \ldots \forall y_n (R x \overline{L} y_0 \ldots y_{i - 1} \overline{0} y_{i + 1} \ldots y_n \rightarrow} \cr
\ & R fx \overline{L + 1} y_0 \ldots y_{i - 1} \overline{0} y_{i + 1} \ldots y_n) \land \cr
\ & \multicolumn{1}{l}{\forall x \forall y_0 \ldots \forall y_n (R x \overline{L} y_0 \ldots y_{i - 1} fy_i y_{i + 1} \ldots y_n \rightarrow R fx \overline{L + 1} y_0 \ldots y_n).}
\end{array}
\]
If $\alpha$ is the instruction $L \ \IF \ \R_i = \Box \ \THEN \ L^\prime \ \ELSE \ L_0$, then let
\[
\begin{array}{rr}
\psi_\alpha := & \multicolumn{1}{l}{\forall x \forall y_0 \ldots \forall y_{i - 1} \forall y_{i + 1} \ldots \forall y_n (R x \overline{L} y_0 \ldots y_{i - 1} \overline{0} y_{i + 1} \ldots y_n \rightarrow \ \ \ } \cr
\ & R fx \overline{L^\prime} y_0 \ldots y_{i - 1} \overline{0} y_{i + 1} \ldots y_n) \land \cr
\ & \multicolumn{1}{l}{\forall x \forall y_0 \ldots \forall y_n (R x \overline{L} y_0 \ldots y_{i - 1} fy_i y_{i + 1} \ldots y_n \rightarrow} \cr
\ & R fx \overline{L_0} y_0 \ldots y_{i - 1} fy_i y_{i + 1} \ldots y_n).
\end{array}
\]
Finally for $\alpha = L \ \PRINT$, let
\[
\begin{array}{rr}
\psi_\alpha := & \multicolumn{1}{l}{\forall x \forall y_0 \ldots \forall y_n (R x \overline{L} y_0 \ldots y_n \rightarrow R fx \overline{L + 1} y_0 \ldots y_n).}
\end{array}
\]
Now set
\[
\varphi_\p := \psi_\p \rightarrow \exists x \exists y_0 \ldots \exists y_n \, R x \overline{k} y_0 \ldots y_n.
\]
Then we can argue that
\begin{center}
$\models \varphi_\p$ \ iff \ $\p : \Box \to \halt$
\end{center}
in the same way as we did in the proof.\\
\ \\
Note that if we take $\psi := \psi_\p$ and $\chi := \exists x \exists y_0 \ldots \exists y_n \, R x \overline{k} y_0 \ldots y_n$, then $\psi, \chi \in L_0^{S_\infty}$, they do not contain the equality symbol, and $\psi$ is a universal Horn sentence. Hence the set mentioned in this exercise is not R-decidable, for otherwise we could decide whether $\models \psi \rightarrow \chi$ (i.e. whether $\models \varphi_\p$), by deciding whether $(\psi, \chi)$ is in this set, and hence whether $\p : \Box \to \halt$.\\
\ \\
It remains to show (2)$^\prime$(b) holds. This can be done by induction on $s$:\\
$s = 0$: By definition, $(0, \ldots, 0)$ is the configuration after $0$ steps; and if $\mathfrak{A}$ is an $S$-structure with $\mathfrak{A} \models \psi_\p$ then, again by definition, we have $R \overline{0} \ldots \overline{0}$.\\
\ \\
It then suffices to show $\mathfrak{A} \models R \, \overline{r + 1} \, \overline{L^+} \, \overline{m_0^+} \ldots \overline{m_n^+}$, provided that $\mathfrak{A}$ is an $S$-structure with $\mathfrak{A} \models \psi_\p$, that $(L^+, m_0^+, \ldots, m_n^+)$ is the configuration after $(r + 1)$ steps, and the induction hypothesis:
\begin{center}
\begin{tabular}{ll}
(+) & if $(L, m_0, \ldots, m_n)$ is the configuration after $r$ steps, \cr
\ & $\mathfrak{A} \models R \overline{r} \overline{L} \overline{m_0} \ldots \overline{m_n}$.
\end{tabular}
\end{center}
\ \\
For this purpose, let us consider the instruction executed at step $(r + 1)$:
\begin{enumerate}[1)]
\item If it is an add-instruction, say $L \ \PA{i}{|}$, then we have $L = L^+ - 1$, and the configuration after $r$ steps is $(L^+ - 1, m_0^+, \ldots, m_i^+ - 1, \ldots, m_n^+)$. By hypothesis, $\mathfrak{A} \models R \, \overline{r} \, \overline{L^+ - 1} \, \overline{m_0^+} \ldots \overline{m_i^+ - 1} \ldots \overline{m_n^+}$. Note that the notations $\overline{L^+ - 1}$ and $\overline{m_i^+ - 1}$ are both well-defined: Since in this case $L^+ > 0$ and $m_i^+ > 0$, $\overline{L^+ - 1}$ and $\overline{m_i^+ - 1}$ are just, respectively, $\overline{L^+}$ and $\overline{m^+}$ with $f$ applied once less.\\
\ \\
As $\mathfrak{A} \models \psi_\p$ and, in particular, $\mathfrak{A} \models \psi_{\alpha_{L^+ - 1}}$, we have
\[
\mathfrak{A} \models R \, \overline{r + 1} \, \overline{L^+} \, \overline{m_0^+} \ldots \overline{m_n^+}.
\]
%%
\item If it is the instruction $L \ \PS{i}{|}$, then we have $L = L^+ - 1$, and the configuration after $r$ steps is $(L^+ - 1, m_0^+, \ldots, m_i^\prime, \ldots, m_n^+)$, where $m_i^\prime = m_i^+ + 1$ if $m_i^+ > 0$, and if $m_i^+ = 0$ then $m_i^\prime$ may be either $0$ or $1$. By hypothesis, $\mathfrak{A} \models R \, \overline{r} \, \overline{L^+ - 1} \, \overline{m_0^+} \ldots \overline{m_i^\prime} \ldots \overline{m_n^+}$.\\
\ \\
From $\mathfrak{A} \models \psi_\p$ and hence $\mathfrak{A} \models \psi_{\alpha_{L^+ - 1}}$, it follows that
\[
\mathfrak{A} \models R \, \overline{r + 1} \, \overline{L^+} \, \overline{m_0^+} \ldots \overline{m_n^+}.
\]
%%
\item If it is the instruction $L \ \IF \ \R_i = \Box \ \THEN \ L^\prime \ \ELSE \ L_0$, then we have
\[
L^+ = \begin{cases}
L^\prime,  & \mbox{if \(m_i^+ = 0\)}; \cr
L_0,       & \mbox{otherwise}.
\end{cases}
\]
And the configuration after $r$ steps is $(L, m_0^+, \ldots, m_n^+)$. By hypothesis, $\mathfrak{A} \models R \, \overline{r} \, \overline{L} \, \overline{m_0^+} \ldots \overline{m_n^+}$.\\
\ \\
As $\mathfrak{A} \models \psi_\p$ and furthermore $\mathfrak{A} \models \psi_{\alpha_L}$, we have
\[
\mathfrak{A} \models R \, \overline{r + 1} \, \overline{L^+} \, \overline{m_0^+} \ldots \overline{m_n^+}.
\]
%%
\item Finally, if it is $L \ \PRINT$, then we have $L = L^+ - 1$, and the configuration after $r$ steps is $(L^+ - 1, m_0^+, \ldots, m_n^+)$. By hypothesis, $\mathfrak{A} \models R \, \overline{r} \, \overline{L^+ - 1} \, \overline{m_0^+} \ldots \overline{m_n^+}$.\\
\ \\
Since $\mathfrak{A} \models \psi_\p$, and hence $\mathfrak{A} \models \psi_{\alpha_{L^+ - 1}}$, it follows that
\begin{center}
\phantom{a} \hfill $\mathfrak{A} \models R \, \overline{r + 1} \, \overline{L^+} \, \overline{m_0^+} \ldots \overline{m_n^+}.$ \hfill $\talloblong$
\end{center}
\end{enumerate}
\textit{Remark.} This result also leads to the undecidability of first-order logic, that is to say, we have just proved Theorem 4.1 in another way: If the set $\{ \varphi \in L_0^{S_\infty} \ | \ \models \varphi \}$ were decidable, then we could decide the set mentioned in this exercise by deciding whether $\models \psi \to \chi$ for a given pair $(\psi, \chi)$, where $\psi, \chi \in L_0^{S_\infty}$ do not contain the equality symbol, and where $\psi$ is a universal Horn sentence and $\chi$ is of the form $\exists x_1 \ldots \exists x_n \chi_0$ with atomic $\chi_0$. Note that, as hint suggested, we leave out the ordering $<$ in the proof here, since it is unnecessary for this purpose (cf. the foot note on page 168).
\end{enumerate}
%End of Section X.4--------------------------------------------------------------------------------
\
\\
\\
%Section X.5---------------------------------------------------------------------------------------
{\large \S5. Trahtenbrot's Theorem and the Incompleteness of Second-Order Logic}
\begin{enumerate}[1.]
\item \textbf{Note to Definition 5.1.} From the definition, we immdiately have
\[
\{ \varphi \in L_0^{S_\infty} \ | \ \models \varphi \} \subset \Phi_\scripttext{fv} \subset \Phi_\scripttext{fs} \subset \{ \varphi \in L_0^{S_\infty} \ | \ \sat \varphi \} \subset L_0^{S_\infty}
.
\]
%
\item \textbf{Note to the Paragraph Below Theorem 5.5.} From the failure of the Compactness Theorem, what we claimed to fail as well for any correct sequent calculus for $\mathcal{L}_\scripttext{II}$ in IX.1 is indeed the \emph{Completeness Theorem}: For all sets $\Phi \subset L^S_\scripttext{II}$ and all second-order $S$-formulas $\varphi$, 
\begin{center}
if $\Phi \models \varphi$ then $\Phi \vdash \varphi$;
\end{center}
instead of the \emph{completeness} for those calculi: For all sequents $\Gamma \subset L^S_\scripttext{II}$ and all second-order $S$-formulas $\varphi$,
\begin{center}
if $\Gamma \models \varphi$ then $\Gamma \vdash \varphi$.
\end{center}
There we were not sure whether it holds. However, Theorem 5.5 also leads to the failure of the completeness just stated: If the completeness did hold for a correct proof calculus, then we could apply it (with the help of the correctness) to the case in which $\Gamma = \emptyset$ (i.e. the empty sequent) and thus enumerate all valid second-order $S$-formulas, contrary to Theorem 5.5.
%
\item \textbf{Incompleteness of Weak Second-Order Logic.} To apply the argument for Theorem 5.5 given in text to weak second-order logic, we choose $\varphi_\mathrm{fin}$ to be the second-order sentence
\[
\varphi_\mathrm{fin} \colonequals \exists X \forall x \, Xx,
\]
where $X$ is a unary relation variable. Then it is clear (cf. Exercise XI.1.7) that for all $\struct{A}$,
\begin{center}
$\struct{A} \models \varphi_\mathrm{fin}$ \ \ \ iff \ \ \ $A$ is finite.
\end{center}
%
\item \textbf{Note to the Paragraph before Exercise 5.6.} (INCOMPLETE) Recall that to obtain Theorem 4.1 and other results in X.4 and X.5, we only used the 4 symbols $R$ ($(n + 3)$-ary), $<$ (binary), $f$ (unary) and $c$. It is conceivable that to obtain those results, it suffices for a symbol set $S$ to be effectively given and contain the 4 symbols.\\
\ \\
$[$INCOMPLETE: Show that it is even sufficient for $S$ to contain only one binary relation symbol to describe the execution of register programs.$]$
%
\item \textbf{Solution to Exercise 5.6.} Recall that a symbol set is called relational if it contains only relation symbols (cf. VIII.1). And we obtained in part (1) of \textbf{More on Syntactic Properties of $\mathcal{L}_\mathrm{II}$} in notes to Chapter IX that for any $\varphi \in \LII^S$,\footnote{Without loss of generality, $S$ is assumed to be \emph{finite}.}
\begin{center}
$\models \varphi$ \ \ \ iff \ \ \ $\models (\chi \rightarrow \varphi^r)$,
\end{center}
where $S^r$ is the corresponding relational symbol set and $\chi \in L^{S^r}$ states that the new relation symbols are the counterparts of the function and constant symbols in $S$, and where $\varphi^r$ essentially states in $\LII^{S^r}$ the same as does $\varphi$ in $\LII^S$.\\
\ \\
Therefore, it suffices to show that for every second-order $S$-sentence $\varphi$ with $S$ relational and finite, there is a second-order $\emptyset$-sentence $\varphi^\prime$ such that
\begin{center}
$\models \varphi$ \ \ \ iff \ \ \ $\models \varphi^\prime$.
\end{center}
Then we are done: If the set of valid second-order $\emptyset$-sentences were R-enumerable, we could enumerate the set of valid second-order $S_\infty$-sentences (contrary to 5.5) as follows: For a string $\zeta$ over $\mathcal{A}_0$, decide whether $\zeta$ is a second-order $S_\infty$-sentence.\footnote{This can achieved by extending the decision procedure given in part (b) of Exercise 1.3 according to the formation rules for $\mathcal{L}_\scripttext{II}$ in IX.1.1.} If so, let $S_0$ be a \emph{finite} symbol set such that $\zeta$ is a second-order $S_0$-sentence $\varphi$. Effectively take the corresponding second-order $S^r_0$-sentence $(\chi \rightarrow \varphi^r)$, and then generate the second-order $\emptyset$-sentence $(\chi \rightarrow \varphi^r)^\prime$. Print out $\zeta$ if $(\chi \rightarrow \varphi^r)^\prime$ ever appears as an output of an enumeration procedure for the set of valid second-order $\emptyset$-sentences.\\
\ \\
Let $S = \{ R_0, \ldots, R_n \}$ be a relational symbol set in which the arity of $R_i$ is $r_i$ for $0 \leq i \leq n$, and $\varphi \in L^S_\scripttext{II}$ a sentence. We shall denote by $(A, C_0, \ldots, C_n)$ an $S$-structure with domain $A$ and with $R_i^A = C_i$ for $0 \leq i \leq n$, and denote by $(A, C_0, \ldots, C_n, \gamma)$ the corresponding $S$-interpretation with the second-order assignment $\gamma$. We designate the second-order \emph{$\emptyset$-formula} $\psi$ with its free \emph{relation} variables among $X_0, \ldots, X_n$ such that $\varphi = \psi \displaystyle\frac{R_0 \ldots R_n}{X_0 \ldots X_n}$. Then
\begin{center}
\begin{tabular}{lll}
$\models \varphi$ & iff & For all $S$-interpretations $\mathfrak{I}$, $\mathfrak{I} \models \varphi$ \cr
\ & iff & For all nonempty sets $A$, all $C_0 \subset A^{r_0}, \ldots, C_n \subset A^{r_n}$, \cr
\ & \ & and all second-order assignments $\gamma$, the $S$-interpretation \cr
\ & \ & $(A, C_0, \ldots, C_n, \gamma) \models \psi \displaystyle\frac{R_0 \ldots R_n}{X_0 \ldots X_n}$ \cr
\ & \ & (here we ``expand'' the $S$-interpretation $\mathfrak{I}$) \cr
\ & iff & For all nonempty sets $A$, all $C_0 \subset A^{r_0}, \ldots, C_n \subset A^{r_n}$, \cr
\ & \ & and all second-order assignments $\gamma$, \cr
\ & \ & $(A, C_0, \ldots, C_n, \gamma) \displaystyle\frac{C_0 \ldots C_n}{X_0 \ldots X_n} \models \psi$ \cr
\ & \ & (by the Substitution Lemma for $\mathcal{L}_\scripttext{II}$; note that for $0 \leq i \leq n$, \cr
\ & \ & $(A, C_0, \ldots, C_n, \gamma)(R_i) = C_i$) \cr
\ & iff & For all $S$-interpretations $\mathfrak{I}$ with domain $A$, \cr
\ & \ & and all $C_0 \subset A^{r_0}, \ldots, C_n \subset A^{r_n}$, $\mathfrak{I} \displaystyle\frac{C_0 \ldots C_n}{X_0 \ldots X_n} \models \psi$ \cr
\ & \ & (by the Coincidence Lemma for $\mathcal{L}_\scripttext{II}$) \cr
\ & iff & For all $S$-interpretations $\mathfrak{I}$, $\mathfrak{I} \models \forall X_0 \ldots \forall X_n \psi$ \cr
\ & iff & $\models \forall X_0 \ldots \forall X_n \psi$.
\end{tabular}
\end{center}
Clearly $\forall X_0 \ldots \forall X_n \psi$ is a second-order \emph{$\emptyset$-sentence.}\\
\ \\
Finally, take\\
\phantom{a} \hfill $\varphi^\prime := \forall X_0 \ldots \forall X_n \psi.$ \hfill $\talloblong$
\end{enumerate}
%End of Section X.5--------------------------------------------------------------------------------
\
\\
\\
%Section X.6---------------------------------------------------------------------------------------
{\large \S6. Theories and Decidability}
\begin{enumerate}[1.]
\item \textbf{Note on Definition 6.1 and the Remark Thereof.} In short, $T \subset L^S_0$ is a theory if and only if
\[
\sat T
\]
and for all $\varphi \in L^S_0$,
\begin{center}
$\varphi \in T$ \ \ \ iff \ \ \ $T \models \varphi$.
\end{center}
Naturally, in the latter condition, the lefthand side implies the right. Hence when examining a set of sentences whether it is a theory, it suffices to provide a model for it and show that every sentence which is a consequence of it is a member of it.\\
\ \\
For every $S$-structure $\mathfrak{A}$, the set $\thr{\struct{A}}$ is a theory: First, $\struct{A} \models \thr{\struct{A}}$ by defintion. Next, suppose $\varphi \in L^S_0$ and $\thr{\struct{A}} \models \varphi$, then by definition we have $\struct{A} \models \varphi$ as $\struct{A} \models \thr{\struct{A}}$; therefore $\varphi \in \thr{\struct{A}}$.\\
\ \\
There is a typo in the second paragraph below 6.1: ``$\Phi \in L^S_0$'' should be replaced with ``$\Phi \subset L^S_0$''.\\
\ \\
One should note that, $\Phi_\pa$ does \emph{not} characterize $\mathfrak{N}$ up to isomorphism (cf. VI.2.4): Intuitively, the induction schema ($*$) contains only properties definable in first-order logic, which are countable in total; whereas the induction axiom in Peano axioms applies to all properties of $\mathfrak{N}$, i.e. all $n$-ary relations over $\mathbb{N}$ for every $n \in \mathbb{Z}^+$, which are, in contrast, uncountable in total.
%
\item \textbf{Note on Consequence Closures Concerning Relativization.} Recall the relativization operation introduced in \reftitle{VIII.1}, for a symbol set $S$ and an $S$-formula $\varphi$ we write $\relational{S}$ and $\relational{\varphi}$ for their respective relational counterparts. Also, given $\Psi \subset \fstordlang[0]{S}$ we shall denote $\relational{\Psi} \colonequals \setm{\relational{\psi}}{\psi \in \Psi}$.
Then we have:\medskip\\
\begin{theorem}{Claim}
If $\consqn{\Psi} = \Psi$, then $\consqn{(\relational{\Psi})} = \relational{\Psi}$.
\end{theorem}
\begin{proof}
If $S$ is itself relational, then the claim trivially holds. Thus, we shall assume that $S$ is not relational.\bigskip\\
Assuming the hypothesis, it suffices to prove the nontrivial direction: $\consqn{(\relational{\Psi})} \subset \relational{\Psi}$. We prove it in two steps below.\bigskip\\
First, we have\smallskip\\
\begin{bquoteno}{60ex}{($\ast$)}
for every $\relational{S}$-sentence $\varphi$: if $\relational{\Psi} \models \varphi$, then $\Psi \models \invrelational{\varphi}$.
\end{bquoteno}\smallskip\\
This can be argued: Suppose $\relational{\Psi} \models \varphi$ and $\struct{A}$ is an $S$-structure with $\struct{A} \models \Psi$. By \reftitle{VIII.1.3(a)} we obtain $\relational{\struct{A}} \models \relational{\Psi}$; so $\relational{\struct{A}} \models \varphi$ by premise. Using \reftitle{VIII.1.3(b)}, we further get $\struct{A} \models \invrelational{\varphi}$. It follows that $\Psi \models \invrelational{\varphi}$.\bigskip\\
Then, we conclude by showing\smallskip\\
\centerline{for every $\relational{S}$-sentence $\varphi$: if $\varphi \in \consqn{(\relational{\Psi})}$, then $\varphi \in \relational{\Psi}$.}\smallskip\\
Suppose $\varphi \in \consqn{(\relational{\Psi})}$, i.e.\ $\relational{\Psi} \models \varphi$. By ($\ast$) we have $\Psi \models \invrelational{\varphi}$ and hence $\invrelational{\varphi} \in \Psi$ because $\consqn{\Psi} = \Psi$. It follows that $\varphi \in \relational{\Psi}$ (note that $\relational{(\invrelational{\varphi})} = \varphi$ because $\invrelational{\varphi}$ is already term-reduced, cf.\ \reftitle{VIII.1}).
\end{proof}
%
\item \textbf{Note to the Remark after 6.2.} (INCOMPLETE) Show that $\Th_\pa$ and $\Th_\zfc$ are not finitely axiomatizable.
%
\item \textbf{Note to Definition 6.4.} If $T$ is complete, then by definition for every sentence $\varphi$, either $\varphi \in T$ or $\neg\varphi \in T$, and hence either $T \models \varphi$ or $T \models \neg\varphi$.
%
\item \textbf{Solution to Exercise 6.6.} Let $\varphi_0, \varphi_1, \ldots$ be an enumeration of $\Phi$. For $n \in \nat$, pick
\[
\psi_n := \begin{cases}
\varphi_0                      & \mbox{if \(n = 0\)} \cr
(\psi_{n - 1} \land \varphi_n) & \mbox{if \(n > 0\)};
\end{cases}
\]
and let $\Psi := \{ \psi_n \ | \ n \in \nat \}$. Obviously the list $\psi_0, \psi_1, \ldots$ enumerates $\Psi$ in increasing length and hence \emph{in lexicographic order}. As a result, $\Psi$ is R-decidable (cf. Exercise 2.12). It remains to show $T = \Psi^{\models}$.\\
\ \\
For a sentence $\chi \in T$, we have $\Phi \models \chi$; then by the Compactness Theorem there is a finite subset $\Phi_n := \{ \varphi_0, \ldots, \varphi_n \} \subset \Phi$ such that $\Phi_n \models \chi$. Clearly $\Psi_n := \{ \psi_0, \ldots, \psi_n \} \models \chi$ (as $\Phi_n$ and $\Psi_n$ are logically equivalent) and further $\Psi \models \chi$, or $\chi \in \Psi^{\models}$. Thus $T \subset \Psi^{\models}$.\\
\ \\
On the other hand, for a sentence $\chi \in \Psi^{\models}$, we have $\Psi \models \chi$. Again by the Compactness Theorem, there is a finite subset $\Psi_n \subset \Psi$ such that $\Psi_n \models \chi$. Trivially $\Phi_n \models \chi$ and also $\Phi \models \chi$, or $\chi \in \Phi^{\models} = T$. Therefore, $\Psi^{\models} \subset T$.\nolinebreak\hfill$\talloblong$\\
\ \\
\textit{Remark.} According to the argument above, we have that for every R-enumerable $\Phi$, there is an R-decidable $\Psi$ such that, for every formula $\varphi$,
\begin{center}
$\Phi \vdash \varphi$ \ \ \ iff \ \ \ $\Psi \vdash \varphi$.
\end{center}
%
\item* \textbf{Solution to Exercise 6.7.} (a) Suppose $T$ is incomplete, i.e. there is an $S$-sentence $\varphi$ such that neither $\varphi \in T$ nor $\neg\varphi \in T$; in other words, neither $T \models \varphi$ nor $T \models \neg\varphi$ holds, and further by III.4.4,
\begin{center}
$\sat T \cup \{ \varphi \}$ \ \ \ and \ \ \ $\sat T \cup \{ \neg\varphi \}$.
\end{center}
As $T$ is countable (since $\{ \varphi \in L^S_0 \ | \ \models \varphi \} \subset T \subset L^S_0$), so are $T \cup \{ \varphi \}$ and $T \cup \{ \neg\varphi \}$. By premise, both $T \cup \{ \varphi \}$ and $T \cup \{ \neg\varphi \}$ have only infinite models, for otherwise $T$ would have finite models.\\
\ \\
From the Theorem of L\"{o}wenheim, Skolem and Tarski (cf. VI.2.4) and the premise, it follows that there are two $S$-structures $\struct{A}$ and $\struct{B}$ with cardinality $\kappa$ (hence $\struct{A} \cong \struct{B}$) such that $\struct{A} \models T \cup \{ \varphi \}$ and $\struct{B} \models T \cup \{ \neg\varphi \}$, respectively. In particular,
\begin{center}
$\struct{A} \models \varphi$ \ \ \ and \ \ \ $\struct{B} \models \neg\varphi$.
\end{center}
On the other hand, however, since $\struct{A} \cong \struct{B}$ we have by Isomorphism Lemma (cf. III.5.2) that
\begin{center}
$\struct{A} \models \varphi$ \ \ \ iff \ \ \ $\struct{B} \models \varphi$,
\end{center}
a contradiction.\\
\ \\
(b) ??
%
\item \textbf{Note to Lemma 6.8.} According to the remarks after 2.7, we can think of $\chi_\p
 [l_0, \ldots, l_n, L, m_0, \ldots, m_n]$ as stating
\begin{quote}
``$\p$ reaches the configuration $(L, m_0, \ldots, m_n)$ in finitely many steps after \emph{being started with the the $n$ inputs $l_0, \ldots, l_n$}''
\end{quote}
\ \\
As for the proof, in statement (2) the length of the sequence is $(s + 1) \cdot (n + 2)$. However, what $\chi_\p$ actually formalizes is instead the statement\\
\ \\
``There is an $s \in \nat$ and an (arbitrary) sequence of which the \emph{prefix} is
$(a_0, \ldots, a_{n + 1}, a_{n + 2}, \ldots, a_{(n + 2) + (n + 1)}, \ldots, a_{s \cdot (n + 2)}, \ldots, a_{s \cdot (n + 2) + (n + 1)}),$ where $a_0 = 0, a_1 = x_0, \ldots, a_{n + 1} = x_n$,\\ $a_{s \cdot (n + 2)} = z, a_{s \cdot (n + 2) + 1} = y_0, \ldots, a_{s \cdot (n + 2) + (n + 1)} = y_n$,\\
and for all $i < s$:\\
$(a_{i \cdot (n + 2)}, \ldots, a_{i \cdot (n + 2) + (n + 1)}) \begin{array}{c} \ \cr \rightarrow \cr \p \end{array} (a_{(i + 1) \cdot (n + 2)}, \ldots, a_{(i + 1) \cdot (n + 2) + (n + 1)})$.''\\
\ \\
Here we acquire no information about the length of the sequence; only the particular prefix is specified. It should be clear that, if the sequence of the aforementioned statement does exist, then the particular prefix must satisfy statement (2); conversely, if (2) holds, then the sequence thereof must satisfy the statement here.\\
\ \\
In other words, for our purpose of formalizing statement (2) it suffices to \emph{embed} the computation of the register program $\p$ (more precisely, beginning with the configuration $(0, x_0, \ldots, x_n)$, and reaching the configuration $(z, y_0, \ldots, y_n)$ in finitely many steps) into the prefix of any sufficient long sequence, and hence the statement above.\\
\ \\
Because of this, the two numbers $t$ and $p$ mentioned in 6.11 are \emph{not} unique, for they could encode a longer sequence in which $(a_0, \ldots, a_r)$ is a \emph{proper} prefix.\footnote{We say the sequence $s_1$ is a proper prefix of the sequence $s_2$ if $s_1$ is a prefix of $s_2$ and $s_1$ is shorter than $s_2$.} However, the $t$ and $p$ in the proof of 6.11 do encode $(a_0, \ldots, a_r)$ and thus they are the smallest suitable pairs of numbers that serve the purpose.\\
\ \\
Now, let us complete the definition of the $S_\ar$-formulas $\psi_j$:\\
If $\alpha_j$ is an add-instruction
\[
\begin{array}{ll}
j & \PA{e}{|}
\end{array}
\]
then
\[
\begin{array}{lll}
\psi_j & := & u \equiv \mbf{j} \rightarrow (u^\prime \equiv u + 1 \cr
\ & \ & \land u^\prime_0 \equiv u_0 \land \ldots \land u^\prime_{e - 1} \equiv u_{e - 1} \cr
\ & \ & \land u^\prime_e \equiv u_e + 1 \cr
\ & \ & \land u^\prime_{e + 1} \equiv u_{e + 1} \land \ldots \land u^\prime_n \equiv u_n).
\end{array}
\]
If $\alpha_j$ is a sub-instruction
\[
\begin{array}{ll}
j & \PS{e}{|}
\end{array}
\]
then
\[
\begin{array}{lll}
\psi_j & := & u \equiv \mbf{j} \rightarrow (u^\prime \equiv u + 1 \cr
\ & \ & \land u^\prime_0 \equiv u_0 \land \ldots \land u^\prime_{e - 1} \equiv u_{e - 1} \cr
\ & \ & \land (u_e \equiv 0 \rightarrow u^\prime_e \equiv 0) \land (\neg u_e \equiv 0 \rightarrow u_e \equiv u_e^\prime + 1) \cr
\ & \ & \land u^\prime_{e + 1} \equiv u_{e + 1} \land \ldots \land u^\prime_n \equiv u_n).
\end{array}
\]
If $\alpha_j$ is a jump-instruction
\[
\begin{array}{ll}
j & \IF \ \R_e = \Box \ \THEN \ j^\prime \ \ELSE \ j_0
\end{array}
\]
then
\[
\begin{array}{lll}
\psi_j & := & u \equiv \mbf{j} \rightarrow ((u_e \equiv 0 \rightarrow u^\prime \equiv \mbf{j^\prime}) \land (\neg u_e \equiv 0 \rightarrow u^\prime \equiv \mbf{j_0}) \cr
\ & \ & \land u^\prime_0 \equiv u_0 \land \ldots \land u^\prime_n \equiv u_n).
\end{array}
\]
If $\alpha_j$ is a print-instruction
\[
\begin{array}{ll}
j & \PRINT
\end{array}
\]
then
\[
\begin{array}{lll}
\psi_j & := & u \equiv \mbf{j} \rightarrow (u^\prime \equiv u + 1 \cr
\ & \ & \land u_0^\prime \equiv u_0 \land \ldots \land u^\prime_0 \equiv u_n).
\end{array}
\]
\ \\
Finally, note that the binary relation symbol `$<$' is not in $S_\ar$. Therefore $\chi_\p$ given in text is not an $S_\ar$-formula because of the fragment ``$i < s$''; nevertheless, we may regard that fragment as an abbreviation for
\[
\exists h (\neg h \equiv 0 \land i + h \equiv s),
\]
and hence $\chi_\p$ as an $S_\ar$-formula. Also, there is a typo in the definition of $\chi_\p$ on page 179: The third argument of the second $\varphi_\beta$ in line 3 is $s \cdot (\mbf{n + 2}) + 1$ instead of $s \cdot ((\mbf{n + 2}) + 1)$.
%
\item \textbf{Note to Corollary 6.10.} A direct consequence of $\Phi_\pa^{\models} \subsetneq \thr{\natstr}$ is that $\Phi_\pa^{\models}$ is not complete.
%
\item \textbf{Note to the Proof of $\beta$-Function Lemma 6.11.} First let us verify that $b_0 < b_1$ with
\[
\begin{array}{lll}
b_0 & := & 1 \cdot p^0 + \ldots + a_{i - 1} p^{2i - 1}, \cr
b_1 & := & p^{2i}.
\end{array}
\]
Since the prime $p$ is chosen to be larger than $a_0, \ldots, a_r, r + 1$, if we pick
\[
c := \max \{ a_0, \ldots, a_r, r + 1 \},
\]
then we immediately have $p \geq c + 1$, or $c \leq p - 1$. Therefore,
\[
\begin{array}{ll}
\    & b_0 \cr
=    & \displaystyle \sum_{j = 0}^{i - 1} \left[ (j + 1)p^{2j} + a_j p^{2j + 1} \right] \cr
\leq & c (1 + p \ldots + p^{2i - 1}) \cr
=    & \displaystyle c \cdot \frac{p^{2i} - 1}{p - 1} \cr
<    & \displaystyle \frac{cp^{2i}}{p - 1} \cr
\leq & \displaystyle \frac{(p - 1)p^{2i}}{p - 1} \cr
=    & p^{2i} \cr
=    & b_1.
\end{array}
\]
\ \\
On the other hand, according to the proof of this lemma, we may define $\varphi_\beta (v_0, v_1, v_2, v_3)$ as follows:
\[
\begin{array}{lll}
\varphi_\beta (v_0, v_1, v_2, v_3) & := & (\exists v_4 \psi (v_0, v_1, v_2, v_4) \rightarrow \cr
\ & \ & \phantom{(} (\psi (v_0, v_1, v_2, v_3) \land \cr
\ & \ & \phantom{)(} \forall v_5 (\exists v_6 (\neg v_6 \equiv 0 \land v_5 + v_6 \equiv v_3) \rightarrow \cr
\ & \ & \phantom{)(\forall v_5)} \neg\psi (v_0, v_1, v_2, v_5) \cr
\ & \ & \phantom{(\forall v_5)} ) \cr
\ & \ & \phantom{(} ) \cr
\ & \ & )\land \cr
\ & \ & (\forall v_4 \neg\psi (v_0, v_1, v_2, v_4) \rightarrow v_3 \equiv 0),
\end{array}
\]
in which $\psi (v_0, v_1, v_2, v_4)$ says that ``(i)-(iii) and (iv)$^\prime$ hold for $u = v_0$, $q = v_1$, $j = v_2$ and $a = v_4$'':
\[
\begin{array}{lll}
\psi (v_0, v_1, v_2, v_4) & := & \exists b_0 \exists b_1 \exists b_2 ( v_0 \equiv b_0 + \cr
\ & \ & \phantom{\exists b_0 \exists b_1 \exists b_2 (} b_1 \cdot ((v_2 + 1) + v_4 \cdot v_1 + b_2 \cdot v_1 \cdot v_1) \land \cr
\ & \ & \phantom{\exists b_0 \exists b_1 \exists b_2 (} \exists v_5 (\neg v_5 \equiv 0 \land v_4 + v_5 \equiv v_1) \land \cr
\ & \ & \phantom{\exists b_0 \exists b_1 \exists b_2 (} \exists v_5 (\neg v_5 \equiv 0 \land b_0 + v_5 \equiv b_1) \land \cr
\ & \ & \phantom{\exists b_0 \exists b_1 \exists b_2 (} \exists v_5 b_1 \equiv v_5 \cdot v_5 \land \cr
\ & \ & \phantom{\exists b_0 \exists b_1 \exists b_2 (} \forall v_5 ((\neg v_5 \equiv 1 \land \exists v_6 v_5 \cdot v_6 \equiv b_1) \rightarrow \cr
\ & \ & \phantom{\exists b_0 \exists b_1 \exists b_2 (\forall v_5)} \exists v_7 v_1 \cdot v_7 \equiv v_5 \cr
\ & \ & \phantom{\exists b_0 \exists b_1 \exists b_2 (\forall v_5} ) \cr
\ & \ & \phantom{\exists b_0 \exists b_1 \exists b_2 } ).
\end{array}
\]
\ \\
Next, notice that by the choice of $t$ in the proof, the conditions on the righthand-side of $(**)$, namely (i) - (iv), hold only for $0 \leq i \leq r$; thus, if $a \in \nat$ satisfies (i) - (iv), then $a \in \{ a_i \ | \ 0 \leq i \leq r \}$. In addition, as remarked in \textbf{Note to the Proof of Lemma 6.8}, the $t$ and $p$ can be chosen to encode any sequence that contains $(a_0, \ldots, a_r)$ as a prefix; in this situation $(**)$ still holds for $0 \leq i \leq r$.\\
\ \\
Finally, notice that $t$ can be alternatively represented in $q$-adic with any $q \in \nat$ larger than $a_0, \ldots, a_r, r + 1$; $q$ is not even necessarily a prime. However, the choice of $p$ as a prime leads to the equivalence between (iv) and (iv)$^\prime$; (iv)$^\prime$ in place of (iv) facilitates the formalization.
%
\item$^*$ \textbf{Yet Another Method to Encode Finite Sequences.} The sequence $(a_0, \ldots, a_r)$ can be encoded by the number
\[
\left(\prod^{r - 1}_{i = 0} p_i^{a_i}\right) \cdot p_r^{a_r + 1} - 2= 2^{a_0}3^{a_1}5^{a_2} \cdots p_r^{a_r + 1} - 2,
\]
where $p_i$ is the $i$th prime. (Using Unique Factorization of natural numbers.)
%
\item \textbf{Cantor's Pairing Function $\pi$.} The \emph{pairing function} $\pi : \nat^2 \to \nat$ is defined as
\[
\pi(m, n) := \left(\sum^{m + n}_{k = 1} k \right) + m = \frac{1}{2}(m + n)(m + n + 1) + m,
\]
it encodes pairs over $\nat$ as natural numbers.\\
\ \\
$\pi$ enumerates pairs over $\nat$ in this fashion:
\[
\begin{array}{llll}
(0, 0), & \ & \ & \ \cr
(0, 1), & (1, 0), & \ & \ \cr
(0, 2), & (1, 1), & (2, 0), & \ \cr
(0, 3), & (1, 2), & (2, 1), & (3, 0), \cr
\multicolumn{4}{c}{\vdots}
\end{array}
\]
Some initial values are
\[
\begin{array}{c||c|c|c|c|c|c|c}
(m, n) & (0, 0) & (0, 1) & (1, 0) & (0, 2) & (1, 1) & (2, 0) & (0, 3) \cr\hline
\pi (m, n) & 0 & 1 & 2 & 3 & 4 & 5 & 6 \cr
\end{array}
\]
\ \\
It can be easily verified that $\pi$ is bijective. Also note that for all $n, n_1, n_2 \in \nat$,
\begin{center}
if $\pi(n_1, n_2) = n$, then $n_1 \leq n$ and $n_2 \leq n$.
\end{center}
Thus, we can conversely decode natural numbers into pairs over $\nat$. To be precise, let $\pi_1, \pi_2 : \nat \to \nat$ be defined as
\begin{center}
\begin{tabular}{lll}
$\pi_1 (n)$ & $\colonequals$ & the unique (hence the smallest) pair $(n_1, n_2)$ with $n_1 \leq n$ \cr
\ & \ & and $n_2 \leq n$ such that $\pi (n_1, n_2) = n$; \cr
$\pi_2 (n)$ & $\colonequals$ & the unique (hence the smallest) pair $(n_1, n_2)$ with $n_1 \leq n$ \cr
\ & \ & and $n_2 \leq n$ such that $\pi (n_1, n_2) = n$,
\end{tabular}
\end{center}
for $n \in \nat$. Then we have for all $n, n_1, n_2 \in \nat$,
\begin{enumerate}[(1)]
\item $\pi (\pi_1 (n), \pi_2 (n)) = n$;
%%
\item $\pi_1 (\pi (n_1, n_2)) = n_1$, $\pi_2 (\pi (n_1, n_2)) = n_2$.
\end{enumerate}
%
\item \textbf{Encoding Finite Sequences Over $\nat$.} There are various ways to achieve this. We introduce two of them:
\begin{enumerate}[(1)]
\item \textit{Use Cantor's pairing function $\pi$.} The function $\sigma_0 : \{ \Box \} \cup \bigcup_{n \in \zah^+} \nat^n \to \nat$ is defined recursively as
\[
\begin{array}{lll}
\sigma_0 ( \Box ) & \colonequals & 0; \cr
\sigma_0 (a_0, \ldots, a_r) & \colonequals & \pi (a_0, \sigma_0 (a_1, \ldots, a_r)) + 1,
\end{array}
\]
where the subsequence $(a_1, \ldots, a_r) = \Box$ if $r = 0$.\\
\ \\
For example,
\[
\begin{array}{ll}
\ & \sigma_0 (0, 1, 2, 3) \cr
= & \pi (0, \sigma_0 (1, 2, 3)) + 1 \cr
\multicolumn{2}{c}{\cdots} \cr
= & \pi (0, \pi (1, \pi (2, \pi (3, \sigma_0 ( \Box )) + 1) + 1) + 1) + 1 \cr
= & \pi (0, \pi (1, \pi (2, \pi (3, 0) + 1) + 1) + 1) + 1 \cr
\multicolumn{2}{c}{\cdots} \cr
= & 5\,798\,716.
\end{array}
\]
It is easy to verify that $\sigma_0$ is bijective.
%%
\item \textit{Use G\"{o}del's $\beta$-function with Cantor's pairing function $\pi$.} The function $\sigma : \{ \Box \} \cup \bigcup_{n \in \zah^+} \nat^n \to \nat$ is defined as
\begin{center}
\begin{tabular}{lll}
$\sigma ( \Box )$ & $\colonequals$ & $0$; \cr
$\sigma (a_0, \ldots, a_r)$ & $\colonequals$ & $\pi (r, \pi (t, p))$, where $t$ and $p$ is defined as in ($*$) \cr
\ & \ & in the proof of Lemma 6.11,
\end{tabular}
\end{center}
where the subsequence $(a_1, \ldots, a_r) = \Box$ if $r = 0$. $\sigma$ is injective but not surjective. Noteworthy is that $\sigma (s) \neq 0$ for any nonempty sequence $s$.
\end{enumerate}
We will adopt the latter in our discussions since it can be represented by a comparatively simple formula.
%
\item \textbf{Encoding Finite Subsets of $\nat$.} If we identify the empty set $\emptyset$ with the empty sequence $\Box$, and any subset $\{ a_0, \ldots, a_n \} \subset \nat$ with the sequence $(a_0, \ldots, a_n)$,\footnote{There is a problem with this identification: While a set is considered as a collection of objects (that is, the order in which its elements appear does not matter), sequences with same elements appearing in different orders are considered different. Fortunately, for our discussions later we will not concern ourselves with orders of sequences, therefore we will ignore this problem. We will also assume in this identification that sequences consist of pairwise distinct elements.} then we already obtain an encoding method for finite subsets of $\nat$ (cf. \textbf{Encoding Finite Sequences Over $\nat$}).\\
\ \\
Let $m_1, m_2 \in \nat$ encode two (possibly empty) finite subsets $s_1$ and $s_2$.
\begin{enumerate}[(1)]
\item \textit{Membership.} If $m_1 \neq 0$, then for $n \in \nat$, $n \in s_1$ iff there is $k \leq \pi_1 (m_1)$ such that $\beta ( \pi_1 (\pi_2 (m_1)), \pi_2 (\pi_2 (m_1)), k) = n$.
%%
\item \textit{Maximum.} If $m_1 \neq 0$, then for $n \in \nat$, $n$ is the maximum of $s_1$ iff $n \in s_1$ and for $k \leq \pi_1 (m_1)$, $\beta ( \pi_1 ( \pi_2 (m_1)), \pi_2 ( \pi_2 (m_1)), k) \leq n$.
%%
\item \textit{(Nonempty) Subset.} $s_1 \subset s_2$ iff one of the following situations is the case
\begin{enumerate}[(1)]
\item $m_1 = 0$.
%%%
\item $m_1 = m_2$.
%%%
\item $m_1 \neq 0$, $m_2 \neq 0$, $\pi_1 (m_1) < \pi_1 (m_2)$, and for all $k \leq \pi_1 (m_1)$ there is $k^\prime \leq \pi_1 (m_2)$ such that
\[
\beta (\pi_1 (\pi_2 (m_2)), \pi_2 ( \pi_2 (m_2)), k^\prime) = \beta (\pi_1 (\pi_2 (m_1)), \pi_2 (\pi_2 (m_1)), k).
\]
\end{enumerate}
%%
\item \textit{Union.} For $m \in \nat$, $m$ encodes $s_1 \cup s_2$ iff one of the following situations is the case:
\begin{enumerate}[(a)]
\item If $m_1 = m_2 = 0$, then $m = 0$.
%%%
\item If $m_1 \neq 0$ and $m_2 = 0$, then $m = m_1$.
%%%
\item If $m_1 = 0$ and $m_2 \neq 0$, then $m = m_2$.
%%%
\item If $m_1 \neq 0$ and $m_2 \neq 0$, then $m$ is the smallest $m^\prime$ such that
\begin{enumerate}[1$^\circ$]
\item $\pi_1 ( m^\prime ) \leq \pi_1 (m_1) + \pi_1 (m_2) + 1$;
%%%%
\item for $k \leq \pi_1 (m_1)$, there is $k^\prime \leq \pi_1 ( m^\prime )$ such that
\[
\beta ( \pi_1 ( \pi_2 ( m^\prime )), \pi_2 ( \pi_2 ( m^\prime )), k^\prime) = \beta ( \pi_1 ( \pi_2 (m_1)), \pi_2 ( \pi_2 (m_1)), k);
\]
and
%%%%
\item for $k \leq \pi_1 (m_2)$, there is $k^\prime \leq \pi_1 ( m^\prime )$ such that
\[
\beta ( \pi_1 ( \pi_2 ( m^\prime )), \pi_2 ( \pi_2 ( m^\prime )), k^\prime) = \beta ( \pi_1 ( \pi_2 (m_2)), \pi_2 ( \pi_2 (m_2)), k).
\]
\end{enumerate}
\end{enumerate}
%%
\item \textit{Exclusion.} For $n \in \nat$, $s_1 = s_2 \setminus \{ n \}$ iff one of the following situations is the case:
\begin{enumerate}[(a)]
\item If $m_2 = 0$, then $m_1 = 0$.
%%%
\item If $m_2 \neq 0$, $\pi_1 (m_2) = 0$ and $n \in s_2$, then $m_1 = 0$.
%%%
\item If $m_2 \neq 0$, $\pi_1 (m_2) = 0$ and $n \not\in s_2$, then $m_1 = m_2$.
%%%
\item If $m_2 \neq 0$ and $\pi_1 (m_2) \neq 0$, then $m_1$ is the smallest $m^\prime$ with
\begin{enumerate}[1$^\circ$]
\item $\pi_1 ( m^\prime ) \leq \pi_1 (m_2)$;
%%%%
\item for $k \leq \pi_1 (m_2)$, if $\beta (\pi_1 (\pi_2 (m_2)), \pi_2 (\pi_2 (m_2)), k) \neq n$ then there is $k^\prime \leq \pi_1 ( m^\prime )$ such that
\[
\beta (\pi_1 (\pi_2 ( m^\prime )), \pi_2 (\pi_2 ( m^\prime )), k^\prime) = \beta (\pi_1 (\pi_2 (m_2)), \pi_2 (\pi_2 (m_2)), k).
\]
\end{enumerate}
\end{enumerate}
\end{enumerate}
%
\item \textbf{Note to Lemma 6.12.} According to the remark after 2.7, the notions of R-decidability and of R-computability can be directly generalized to $n$-ary relations and $n$-ary functions; thus a program may take $n$ arguments in the first $n$ registers, and then starts computations with such a configuration.
%
\item \textbf{Solution to Exercise 6.13.} With the syntactic interpretation $I : S_\ar \to S_\ar$ given in part (a) of \textbf{Solution to Exercise 2.7} in notes to Chapter VIII, we have $\zahstr^{-I} = \natstr$ and for all $\varphi \in L_0^{S_\ar}$,
\begin{center}
$\natstr \models \varphi$ \ \ \ iff \ \ \ $\zahstr \models \varphi^I$.
\end{center}
Since $\thr{\natstr}$ is R-undecidable (cf. 6.9), so is $\thr{\zahstr}$.\nolinebreak\hfill$\talloblong$
\end{enumerate}
%End of Section X.6--------------------------------------------------------------------------------
\
\\
\\
%Section X.7---------------------------------------------------------------------------------------
{\large \S7. Self-Referential Statements and G\"{o}del's Incompleteness Theorems}
\begin{enumerate}[1.]
\item \textbf{Note to Definition 7.1.} It immediately follows from this definition that the set of all arithmetical (cf. the remark after 6.12) relations and functions over $\nat$ coincides with the set of all those representable in $\thr{\natstr}$.\\
\ \\
On the other hand, if an $r$-ary \emph{relation} $\mathfrak{Q}$ is representable in $\emptyset$, then $\mathfrak{Q} = \nat^r$, i.e. the set of all $r$-tuples over $\nat$: Suppose that $\mathfrak{Q}$ is representable in $\emptyset$ and that $\mathfrak{Q}$ holds for some $n_0, \ldots, n_{r - 1} \in \nat$, then there is an $S_\ar$-formula $\varphi(v_0, \ldots, v_{r - 1})$ such that
\[
\emptyset \vdash \varphi(\mbf{n_0}, \ldots, \mbf{n_{r - 1}}).
\]
From the discussion in Remark (b) to \textbf{Solution to Exercise 4.5} in the notes to Chapter IV (cf. the sequent rule $(\cdot)$), we have
\[
\emptyset \vdash \forall v_0 \ldots \forall v_{r - 1} \varphi,
\]
thus $\mathfrak{Q}$ holds for \emph{all} $n_0, \ldots, n_{r - 1} \in \nat$ and $\mathfrak{Q} = \nat^r$.\\
\ \\
Following the same argument we have, however, that any function $F : \nat^r \to \nat$ is \emph{not} representable in $\emptyset$; for any $S_\ar$-formula $\varphi (v_0, \ldots, v_r)$, if there are some $n_0, \ldots, n_r \in \nat$ such that $\emptyset \vdash \varphi (\mbf{n_0}, \ldots, \mbf{n_r})$, then $\emptyset \vdash \forall v_r \varphi (\mbf{n_0}, \ldots, \mbf{n_{r - 1}}, v_r)$ (cf. the sequent rule $(\circ)$ in Remark (b) to \textbf{Solution to Exercise 4.5} in the notes to Chapter IV), thus
\begin{center}
not $\emptyset \vdash \exists^{= 1} v_r \varphi (\mbf{n_0}, \ldots, \mbf{n_{r - 1}}, v_r)$,
\end{center}
for otherwise we would have $\emptyset \vdash \forall v_0 \forall v_1 \ v_0 \equiv v_1$.\\
\ \\
Finally, let us consider the case of $\Phi_\pa$. For every program $\p$ in which the registers mentioned are among $\R_0, \ldots, \R_r$ and of which the last instruction is $\alpha_k$, consider the $(r + 1)$-ary relation $\mathfrak{Q}_\p$: for all $n_0, \ldots, n_r \in \nat$,
\begin{center}
\begin{tabular}{lll}
$\mathfrak{Q}_\p n_0 \ldots n_r$ & iff & $\p$, beginning with the configuration $(0, n_0, \ldots, n_r)$, \cr
\ & \ & eventually halts.
\end{tabular}
\end{center}
Pick the $S_\ar$-formula
\[
\psi_\p (v_0, \ldots, v_r) := \exists v_{r + 1} \ldots \exists v_{2r + 1} \chi_\p (v_0, \ldots, v_r, \mbf{k}, v_{r + 1}, \ldots, v_{2r + 1}).
\]
(cf. the proof of 6.8 for the construction of $\chi_\p$.) It turns out that $\mathfrak{Q}_\p$ is arithmetical: For $n_0, \ldots, n_r \in \nat$,
\begin{center}
$\mathfrak{Q}_\p n_0 \ldots n_r$ \ \ \ iff \ \ \ $\natstr \models \psi_\p (\mbf{n_0}, \ldots, \mbf{n_r})$.
\end{center}
In particular,
\begin{center}
\begin{tabular}{lll}
$\mathfrak{Q}_\p 0 \ldots 0$ & iff & $\natstr \models \psi_\p (\mbf{0}, \ldots, \mbf{0})$ \cr
\                            & iff & $\p : \Box \to \halt$.
\end{tabular}
\end{center}
Therefore, \emph{there is an arithmetical relation not representable in $\Phi_\pa$}; otherwise, for every program $\p$ the relation $\mathfrak{Q}_\p$ would be representable in $\Phi_\pa$. And since $\Phi_\pa$ is consistent, for any particular program $\p$ either $\Phi_\pa \vdash \psi_\p (0, \ldots, 0)$ or $\Phi_\pa \vdash \neg \psi_\p (0, \ldots, 0)$ is the case. Moreover, as $\Phi_\pa^{\models}$ is R-enumerable (cf. 6.3), we would obtain the following procedure to decide $\Pi_\halt$ (a contradiction): For any program $\p$, effectively generate $\psi_\p$, then start enumerating $\Phi_\pa^{\models}$. We can check which one of $\psi_\p (0, \ldots, 0)$ and $\neg \psi_\p (0, \ldots, 0)$ occurs on the enumeration list (by the Adequacy Theorem), and hence obtain an answer to whether $\p : \Box \to \halt$.
%
\item* \textbf{Note to Part (c) of Lemma 7.2.} Note that in the proof, the set $\{ \psi \in L_0^{S_\ar} \ | \ \Phi \vdash \psi \}$ is R-enumerable provided that $\Phi$ is R-decidable; indeed this set is $\Phi^{\models}$ (cf. V.4.2). Its R-enumerability was already verified in 6.3.\\
%
\item \textbf{Note on Theorem 7.4.} (INCOMPLETE) We shall give a complete proof in Appendix A.\\
\ \\
On the other hand, this theorem, together with 7.2 (c), yields\\
\ \\
\begin{theorem}{Theorem}
The set of all R-decidable relations and R-computable functions over $\nat$ coincides with the set of all relations and functions representable in $\Phi_\pa$.\qed
\end{theorem}
%
\item \textbf{Corollary.} \emph{Any set $\Phi \supset \Phi_\pa$ allows representations.}
%
\item *\textbf{Corollary.} \emph{There is an R-undecidable arithmetical relation.}
%
\item \textbf{A Bijective G\"{o}del Numbering for $T^{S_\ar}$.} With the function $\pi$ introduced in \textbf{Cantor's Pairing Function $\pi$}, we state the numbering rules below:
\begin{enumerate}[(1)]
\item For $n \in \nat$, the variable $v_n$ is assigned the number $3n$.
%%
\item The constant symbol $0$ is assigned the number $1$.
%%
\item The constant symbol $1$ is assigned the number $2$.
%%
\item If the terms $t_1$ and $t_2$ are assigned respectively the numbers $m$ and $n$, then $t_1 + t_2$ is assigned the number $3\pi(m, n) + 4$.
%%
\item If the terms $t_1$ and $t_2$ are assigned respectively the numbers $m$ and $n$, then $t_1 \cdot t_2$ is assigned the number $3\pi(m, n) + 5$.
\end{enumerate}
For example, $v_0 \cdot v_1 + \mbf{2} \cdot (v_2 + 0)$ is assigned the number\footnote{Recall that for $n > 1$, the notation $\mbf{n}$ is a shorthand for $\underbrace{1 + \cdots + 1}_{n\mbox{\tiny-times}}$.}
\[
\begin{array}{ll}
\ & 3\pi (3\pi (0, 3) + 5, 3\pi (3\pi (2, 2) + 4, 3\pi (6, 1) + 4) + 5) + 4 \cr
= & 3\pi (23, 3\pi (40 , 106) + 5) + 4 \cr
= & 3\pi (23, 32318) + 4 \cr
= & 1\,569\,055\,960.
\end{array}
\]
Here are some initial values of this numbering:
\begin{center}
\begin{tabular}{l||c|c|c|c|c|c|c|c}
G\"{o}del numbers & 0 & 1 & 2 & 3 & 4 & 5 & 6 & 7 \cr\hline
$S_\ar$-terms & $v_0$ & $0$ & $1$ & $v_1$ & $v_0 + v_0$ & $v_0 \cdot v_0$ & $v_2$ & $v_0 + v_1$
\end{tabular}
\end{center}
%
\item \textbf{A Bijective G\"{o}del Numbering for $L^{S_\ar}$.} With the function $\pi$ introduced in \textbf{Cantor's Pairing Function $\pi$} and a bijective G\"{o}del numbering $G_T$ ($G_T$ means \emph{G\"{o}del numbering for terms}) for $T^{S_\ar}$,\footnote{For example, the one given in \textbf{A Bijective G\"{o}del Numbering for $T^{S_\ar}$}.} we state the numbering rules below:
\begin{enumerate}[(1)]
\item For $t_1, t_2 \in T^{S_\ar}$, the atomic formula $t_1 \equiv t_2$ is assigned the number $4\pi (G_T(t_1), G_T(t_2))$.
%%
\item If the formula $\varphi$ is assigned the number $n$, then $\neg\varphi$ is assigned the number $4n + 1$.
%%
\item If the formulas $\varphi$ and $\psi$ are assigned respectively the numbers $m$ and $n$, then $(\varphi \lor \psi)$ is assigned the number $4\pi (m, n) + 2$.
%%
\item If the formula $\varphi$ is assigned the number $n$, then for all $m \in \nat$, $\exists v_m \varphi$ is assigned the number $4\pi (m, n) + 3$.
\end{enumerate}
For example, $\forall v_0 \ v_0 + 0 \equiv v_0$ is assigned the number\footnote{Recall that $\forall v_n$ is an abbreviation of $\neg\exists v_n \neg$.}
\[
\begin{array}{ll}
\ & 4(4\pi (0, 4 \cdot 4\pi (3\pi (0, 1) + 4, 0) + 1) + 3) + 1 \cr
= & 4(4\pi (0, 4 \cdot 4\pi (7, 0) + 1) + 3) + 1 \cr
= & 4(4\pi (0, 561) + 3) + 1 \cr
= & 2\,522\,269. \cr
\end{array}
\]
Here are some initial values of this numbering:\\
\ \\
\begin{tabular}{l||c|c|c|c}
G\"{o}del numbers & 0 & 1 & 2 & 3 \cr\hline
$S_\ar$-formulas & $v_0 \equiv v_0$ & $\neg v_0 \equiv v_0$ & $(v_0 \equiv v_0 \lor v_0 \equiv v_0)$ & $\exists v_0 \ v_0 \equiv v_0$
\end{tabular}
\\
\ \\
\ \\
\begin{tabular}{c|c|c|c|c}
 & 4 & 5 & 6 & 7 \cr\hline
 & $v_0 \equiv v_1$ & $\neg v_0 \equiv v_1$ & $(v_0 \equiv v_0 \lor \neg v_0 \equiv v_0)$ & $\exists v_0 \neg v_0 \equiv v_0$
\end{tabular}
%
\item* \textbf{Note to the Proof of Fixed Point Theorem 7.5.} Our objective is to find a formula $\beta(x) \in L_1^{S_\ar}$ such that, for $\chi(x) \in L_1^{S_\ar}$ the sentence $\beta( \mbf{n}^{\chi(x)} )$ says 
\begin{quote}
``the G\"{o}del number of $\chi(\mbf{n}^{\chi(x)})$ satisfies the property $\psi$,''
\end{quote}
i.e. $\psi( \mbf{n}^{\chi(\mbf{n}^{\chi(x)})} )$. In particular, for $\chi = \beta$, the sentence $\beta( \mbf{n}^\beta )$ says
\begin{quote}
``the G\"{o}del number of $\beta( \mbf{n}^\beta )$ satisfies the property $\psi$,''
\end{quote}
i.e. $\psi( \mbf{n}^{ \beta( \mbf{n}^\beta ) } )$, which is a \emph{self-referential statement} ``I have the property $\psi$''; thus it suffices to set $\varphi \colonequals \beta ( \mbf{n}^\beta )$.\\
\ \\
Since $\alpha$ represents $F$, it is straightforward to derive the formalization of $\beta(x)$ (below we write $\mbf{F(n^{\chi(x)}, n^{\chi(x)})}$ for the $S_\ar$-term corresponding to $F(n^{\chi(x)}, n^{\chi(x)})$):
\begin{center}
\begin{tabular}{l}
$\psi(\mbf{F(n^{\chi(x)}, n^{\chi(x)})})$; \cr
$\forall z ( \mbf{F(n^{\chi(x)}, n^{\chi(x)})} \equiv z \rightarrow \psi (z) )$; and finally \cr
$\forall z ( \alpha (x, x, z) \rightarrow \psi (z) )$.
\end{tabular}
\end{center}
We set
\[
\beta (x) \colonequals \forall z ( \alpha (x, x, z) \rightarrow \psi (z) ).
\]
\ \\
Below we provide derivations for $\Phi \vdash \varphi \rightarrow \psi (\mbf{n}^\varphi)$ and for $\Phi \vdash \psi (\mbf{n}^\varphi) \rightarrow \varphi$, assuming $\Gamma \vdash \alpha ( \mbf{n}^\beta, \mbf{n}^\beta, \mbf{n}^\varphi )$ and $\Gamma \vdash \exists^{=1} z \alpha ( \mbf{n}^\beta, \mbf{n}^\beta, z )$ for some sequent $\Gamma \subset \Phi$:
\begin{enumerate}[(a)]
\item $\Phi \vdash \varphi \rightarrow \psi ( \mbf{n}^\varphi )$.
\[
\begin{array}{lllll}
1. & \Gamma & \ & \alpha ( \mbf{n}^\beta, \mbf{n}^\beta, \mbf{n}^\varphi ) & \mbox{premise} \cr
2. & \Gamma & \varphi & \alpha ( \mbf{n}^\beta, \mbf{n}^\beta, \mbf{n}^\varphi ) & \mbox{(Ant) applied to 1.} \cr
3. & \Gamma & \varphi & \varphi & \mbox{(Assm)} \cr
4. & \Gamma & \varphi & \alpha ( \mbf{n}^\beta, \mbf{n}^\beta, \mbf{n}^\varphi ) \rightarrow \psi ( \mbf{n}^\varphi ) & \mbox{IV.5.5 (a1) applied to 3. with} \cr
\  & \      & \       & \ & t = \mbf{n}^\varphi \cr
5. & \Gamma & \varphi & \psi ( \mbf{n}^\varphi ) & \mbox{IV.3.5 applied to 4. and 2.} \cr
6. & \Gamma & \ & \varphi \rightarrow \psi ( \mbf{n}^\varphi ) & \mbox{IV.3.6 (c) applied to 5.}
\end{array}
\]
%%
\item $\Phi \vdash \psi (\mbf{n}^\varphi) \rightarrow \varphi$. We write
\begin{center}
\begin{enumerate}[1$^\circ$]
\item $\exists z (\alpha(\mbf{n}^\beta, \mbf{n}^\beta, z) \land \forall u (\alpha(\mbf{n}^\beta, \mbf{n}^\beta, u) \rightarrow z \equiv u))$ for $\exists^{=1} z \alpha(\mbf{n}^\beta, \mbf{n}^\beta, z)$,
%%%
\item $\gamma$ for $\forall u (\alpha(\mbf{n}^\beta, \mbf{n}^\beta, u) \rightarrow v \equiv u)$, and
%%%
\item $\delta$ for $\alpha(\mbf{n}^\beta, \mbf{n}^\beta, v) \land \gamma$,
\end{enumerate}
\end{center}
where $v \neq z$ is chosen not to be free in $\Gamma \ \exists^{=1} z \alpha( \mbf{n}^\beta, \mbf{n}^\beta, z)$ or $\psi(z)$.\newpage
\[
\begin{array}{rlll}
1. & \Gamma & \alpha(\mbf{n}^\beta, \mbf{n}^\beta, \mbf{n}^\varphi) & \mbox{premise} \cr
2. & \Gamma & \exists^{=1} z \alpha(\mbf{n}^\beta, \mbf{n}^\beta, z) & \mbox{premise} \cr
3. & \Gamma & \delta \ \delta & \mbox{(Assm)} \cr
4. & \Gamma & \delta \ \gamma & \mbox{IV.3.6(d2) applied} \cr
\ & \ & \ & \mbox{to 3.} \cr
5. & \Gamma & \delta \ \alpha(\mbf{n}^\beta, \mbf{n}^\beta, \mbf{n}^\varphi) \rightarrow v \equiv \mbf{n}^\varphi & \mbox{IV.5.5(a1) applied} \cr
\ & \ & \ & \mbox{to 4. with $t = \mbf{n}^\varphi$} \cr
6. & \Gamma & \delta \ \alpha(\mbf{n}^\beta, \mbf{n}^\beta, \mbf{n}^\varphi) & \mbox{(Ant) applied to 1.} \cr
7. & \Gamma & \delta \ v \equiv \mbf{n}^\varphi & \mbox{IV.3.5 applied to 5.} \cr
\ & \ & \ & \mbox{and 6.} \cr
8. & \Gamma & \delta \ v \equiv \mbf{n}^\varphi \ \gamma \df{\mbf{n}^\varphi}{v} & \mbox{(Sub) applied to 4.} \cr
9. & \Gamma & \delta \ \gamma \df{\mbf{n}^\varphi}{v} & \mbox{(Ch) applied to 7.} \cr
\ & \ & \ & \mbox{and 8.} \cr
10. & \Gamma & \exists^{=1} z \alpha(\mbf{n}^\beta, \mbf{n}^\beta, z) \ \gamma \df{\mbf{n}^\varphi}{v} & \mbox{($\exists$A) applied to 9.} \cr
11. & \Gamma & \gamma \df{\mbf{n}^\varphi}{v} & \mbox{(Ch) applied to 2.} \cr
\ & \ & \ & \mbox{and 10.} \cr
12. & \Gamma & \alpha ( \mbf{n}^\beta, \mbf{n}^\beta, v) \rightarrow \mbf{n}^\varphi \equiv v & \mbox{IV.5.5(a1) applied} \cr
\ & \ & \ & \mbox{to 11.} \cr
13. & \Gamma & \alpha ( \mbf{n}^\beta, \mbf{n}^\beta, v) \ \alpha ( \mbf{n}^\beta, \mbf{n}^\beta, v) \rightarrow \mbf{n}^\varphi \equiv v & \mbox{(Ant) applied to 12.} \cr
14. & \Gamma & \alpha ( \mbf{n}^\beta, \mbf{n}^\beta, v) \ \alpha ( \mbf{n}^\beta, \mbf{n}^\beta, v) & \mbox{(Assm)} \cr
15. & \Gamma & \alpha ( \mbf{n}^\beta, \mbf{n}^\beta, v) \ \mbf{n}^\varphi \equiv v & \mbox{IV.3.5 applied to 13.} \cr
\ & \ & \ & \mbox{and 14.} \cr
16. & \Gamma & \alpha ( \mbf{n}^\beta, \mbf{n}^\beta, v) \ v \equiv \mbf{n}^\varphi & \mbox{IV.5.3(a) applied} \cr
\ & \ & \ & \mbox{to 15.} \cr
17. & \Gamma & \alpha ( \mbf{n}^\beta, \mbf{n}^\beta, v) \rightarrow v \equiv \mbf{n}^\varphi & \mbox{IV.3.6(c) applied} \cr
\ & \ & \ & \mbox{to 16.} \cr
18. & \Gamma & \forall z (\alpha ( \mbf{n}^\beta, \mbf{n}^\beta, z) \rightarrow z \equiv \mbf{n}^\varphi) & \mbox{IV.5.5(b2) applied} \cr
\ & \ & \ & \mbox{to 17.} \cr
19. & \Gamma & \psi(\mbf{n}^\varphi) \ \alpha(\mbf{n}^\beta, \mbf{n}^\beta, v) \ \psi(\mbf{n}^\varphi) & \mbox{(Assm)} \cr
20. & \Gamma & \psi(\mbf{n}^\varphi) \ \alpha(\mbf{n}^\beta, \mbf{n}^\beta, v) \ \mbf{n}^\varphi \equiv v & \mbox{(Ant) applied to 15.} \cr
21. & \Gamma & \psi(\mbf{n}^\varphi) \ \alpha(\mbf{n}^\beta, \mbf{n}^\beta, v) \ \mbf{n}^\varphi \equiv v \ \psi(v) & \mbox{($\equiv$) applied to 19.} \cr
22. & \Gamma & \psi(\mbf{n}^\varphi) \ \alpha(\mbf{n}^\beta, \mbf{n}^\beta, v) \ \psi(v) & \mbox{(Ch) applied to 20.} \cr
\ & \ & \ & \mbox{and 21.} \cr
23. & \Gamma & \psi(\mbf{n}^\varphi) \ \alpha(\mbf{n}^\beta, \mbf{n}^\beta, v) \rightarrow \psi(v) & \mbox{IV.3.6(c) applied} \cr
\ & \ & \ & \mbox{to 22.} \cr
24. & \Gamma & \psi(\mbf{n}^\varphi) \ \forall z ( \alpha ( \mbf{n}^\beta, \mbf{n}^\beta, z ) \rightarrow \psi(z) ) & \mbox{IV.5.5(b2) applied} \cr
\ & \ & \ & \mbox{to 23.} \cr
25. & \Gamma & \psi(\mbf{n}^\varphi) \rightarrow \forall z ( \alpha ( \mbf{n}^\beta, \mbf{n}^\beta, z ) \rightarrow \psi(z) ) & \mbox{IV.3.6(c) applied} \cr
\ & \ & \ & \mbox{to 24.}
\end{array}
\]
\end{enumerate}
The sequent at line 18 corresponds to the subgoal $\Phi \vdash \forall z ( \alpha ( \mbf{n}^\beta, \mbf{n}^\beta, z ) \rightarrow z \equiv \mbf{n}^\varphi )$;\footnote{A derivation for the ultimate goal only can be obtained by eliminating lines 16 - 18.} the succedent $\psi(\mbf{n}^\varphi) \rightarrow \forall z ( \alpha ( \mbf{n}^\beta, \mbf{n}^\beta, z ) \rightarrow \psi(z) )$ of the sequent at line 25 is $\varphi$.\\
\ \\
Also note that we could have chosen in the proof the function $F : \nat \to \nat$,
\[
F(m) = \begin{cases}
n^{\chi (\mbf{m}) }, & \mbox{if \(m = n^\chi\) for some \(\chi \in L_1^{S_\ar}\)}; \cr
0,                   & \mbox{otherwise}.
\end{cases}
\]
This way we would obtain the same result.\\
\ \\
\textit{Remark.} The choice of $\beta (x)$ mentioned earlier somehow reminds me of Russel's paradox: For the set $S \colonequals \{ \mbox{\begin{math}x\end{math} is a set} \ | \ x \not\in x \}$, we have for all sets $x$,
\begin{center}
$x \in S$ \ \ \ iff \ \ \ $x \not\in x$.
\end{center}
In particular, for $x = S$,
\begin{center}
$S \in S$ \ \ \ iff \ \ \ $S \not\in S$.
\end{center}
[IMCOPLETE: Consider the relation between 7.5 and halting problem.]
%
\item \textbf{Note to Lemma 7.6.} It immediately follows that, for any consistent set $\Phi \subset L_0^{S_\ar}$ which allows representations, the set $\Phi^\vdash$ is R-undecidable.\\
\ \\
Following the same argument in the proof, we obtain that for any consistent set $\Phi \subset L_0^{S_\ar}$ which allows representations, the set $\Psi \colonequals \{ \psi \in L^{S_\ar} \ | \ \Phi \vdash \psi \}$ is not representable in $\Phi$ and hence is also R-undecidable.
%
\item* \textbf{Note to G\"{o}del's First Incompleteness Theorem 7.8.} Note that $\Phi^\vdash = \Phi^{\models}$, it is a theory.\\
\ \\
As a corollary to this theorem, there is an $S_\ar$-sentence $\varphi$, such that neither $\Phi_\pa \vdash \varphi$ nor $\Phi_\pa \vdash \varphi$.\\
\ \\
On the other hand, the notion of \emph{completeness} mentioned here is quite different from the one mentioned in Chapter V: The completeness mentioned in Chapter V, together with the correctness, states that the notion of consequence (which is semantical) coincides with the notion of theorem (which is syntactical), respectively; whereas the completeness here concerns a system of axioms and a sentence in regard of derivability, it differs from the negation completeness mentioned in part (a) of V.1.8 only in that here it restricts to the case of \emph{sentences}, and it is basically symmetrical to the completeness defined in 6.4\footnote{Interestingly, from the Completeness Theorem proved in Chapter V (together with the Correctness Theorem), it follows that these two terms coincide.} (which amounts to concerning consequence relation between a system of axioms and a sentence).\\
\ \\
Moreover, following the argument in Exercise 6.6, this theorem can be generalized to the case in which $\Phi$ is consistent, \emph{R-enumerable}, and allows representations.
%
\item* \textbf{Formulating Predicates Over $\nat$ for Deriving L\"{o}b's Axioms.} We will, throughout this note, use $\tau$ with subscripts to abbreviate some particular terms in $T^{S_\ar}$, and $\chi$ with subscripts some particular formulas in $L^{S_\ar}$; the subscripts are self-explanatory.\\
\ \\
The formula (should use bounded quantifiers!)
\[
\begin{array}{l}
\chi_\sbt (v_0, v_1, v_2, v_3) \colonequals \cr
\phantom{\land} (v_0 \equiv 1 \rightarrow v_3 \equiv 0) \cr
\land (v_0 \equiv \mathbf{2} \rightarrow v_3 \equiv 1) \cr
\land (\exists v_4(v_0 \equiv \mathbf{3} \cdot v_4  \land v_1 \equiv v_4) \rightarrow v_3 \equiv v_2) \cr
\land (\exists v_4(v_0 \equiv \mathbf{3} \cdot v_4 \land \neg v_1 \equiv v_4) \rightarrow v_3 \equiv v_0) \cr
\land (\exists v_4 \exists v_5 \mathbf{3} \cdot (v_4 + v_5) \cdot (v_4 + v_5) + \mathbf{9} \cdot v_4 + \mathbf{3} \cdot v_5 + \mathbf{2} \equiv \mathbf{2} \cdot v_0 \cr
\phantom{\land(} \rightarrow \forall v_6 \forall v_7 ((\chi_\sbt (v_4, v_1, v_2, v_6) \land \chi_\sbt (v_5, v_1, v_2, v_7)) \cr
\phantom{\land(\rightarrow \forall v_6 \forall v_7)} \rightarrow \mathbf{3} \cdot (v_6 + v_7) \cdot (v_6 + v_7) + \mathbf{9} \cdot v_6 + \mathbf{3} \cdot v_7 + \mathbf{2} \equiv \mathbf{2} \cdot v_3 \cr
\phantom{\land(\rightarrow \forall v_6 \forall v_7} ) \cr
\phantom{\land} ) \cr
\land (\exists v_4 \exists v_5 \mathbf{3} \cdot (v_4 + v_5) \cdot (v_4 + v_5) + \mathbf{9} \cdot v_4 + \mathbf{3} \cdot v_5 + \mathbf{4} \equiv \mathbf{2} \cdot v_0 \cr
\phantom{\land(} \rightarrow \forall v_6 \forall v_7 ((\chi_\sbt (v_4, v_1, v_2, v_6) \land \chi_\sbt (v_5, v_1, v_2, v_7)) \cr
\phantom{\land(\rightarrow \forall v_6 \forall v_7)} \rightarrow 
\mathbf{3} \cdot (v_6 + v_7) \cdot (v_6 + v_7) + \mathbf{9} \cdot v_6 + \mathbf{3} \cdot v_7 + \mathbf{4} \equiv \mathbf{2} \cdot v_3 \cr
\phantom{\land(\rightarrow \forall v_6 \forall v_7} ) \cr
\phantom{\land} )
\end{array}
\]
formulates that $v_3$ encodes the term encoded by $v_0$ in which all the occurrences of the variable encoded by $3v_1$ is substituted by those of the term encoded by $v_2$.\\
\ \\
The formulas
\[
\begin{array}{lll}
\chi_\atm (v_0) & \colonequals & \exists v_1 \ v_0 \equiv \mathbf{4} \cdot v_1, \cr
\chi_\ngt (v_0) & \colonequals & \exists v_1 \ v_0 \equiv \mathbf{4} \cdot v_1 + 1, \cr
\chi_\dsj (v_0) & \colonequals & \exists v_1 \exists v_2 \ v_0 \equiv \mathbf{2} \cdot (v_1 + v_2) \cdot (v_1 + v_2) + \mathbf{6} \cdot v_1 + \mathbf{2} \cdot v_2 + \mathbf{2}, \cr
\chi_\ext (v_0) & \colonequals & \exists v_1 \exists v_2 \ v_0 \equiv \mathbf{2} \cdot (v_1 + v_2) \cdot (v_1 + v_2) + \mathbf{6} \cdot v_1 + \mathbf{2} \cdot v_2 + \mathbf{3}
\end{array}
\]
formulate that the formula encoded by $v_0$ is atomic, is a negation, is a disjunction, and begins with an (existential) quantifier, respectively.\\
\ \\
The formula
\[
\chi_\sbf (v_0, v_1, v_2, v_3) \colonequals
\]

Define the function $f : \nat \to \nat$ as
\[
f(n) \colonequals \begin{cases}
n & \mbox{if \(\chi_\drn (n)\)}; \cr
7 & \mbox{otherwise},
\end{cases}
\]
in which $7$ encodes the derivation
\[
\begin{array}{lll}
1. & v_0 \equiv v_0 & \mbox{$\eq$}
\end{array}
\]
In other words, if $n \in \nat$ encodes some derivation, then it remains unchanged under the mapping $f$; otherwise it is mapped to (the encoding of) the simple derivation above.\\
\ \\
$f$ is represented by a $\Delta_0$-formula
\[
\chi_f.
\]
\ \\
Finally, we can take $\varphi_H$ as
\[
\varphi_H (v_0, v_1) \colonequals \exists^{=1} v_2 (\chi_f (v_1, v_2) \land \chi (v_2, v_0) ),
\]
where $\chi (v_2, v_0)$ formulates ``$v_0$ encodes the the succedent of the last sequent of the derivation encoded by $v_2$.''
%
\item* \textbf{Identifying Derivations From $\Phi_\pa$ with Natural Numbers.} As was stated in IV.1, a derivation is a nonempty sequence of sequents; whereas a sequent is a nonempty sequence of formulas, consisting of a (possibly empty) sequence of formulas called the antecedent and a single formula called the succedent.\\
\ \\
For the sake of our discussion on Theorem 7.10, we will pose the restriction on sequents that if the antecedent is nonempty, then it consists of different formulas, that is, repititions of formulas are not allowed; moreover, the formulas therein are sorted in the order of increasing G\"{o}del numbers. For example, the antecedent of the sequent
\[
\begin{array}{lll}
v_0 \equiv v_0 & \neg v_0 \equiv v_0 & (v_0 \equiv v_0 \lor v_0 \equiv v_0)
\end{array}
\]
is
\[
\begin{array}{ll}
v_0 \equiv v_0 & \neg v_0 \equiv v_0,
\end{array}
\]
and the two formulas above have G\"{o}del numbers $0$ and $1$, respectively.\\
\ \\
Next, we encode sequents as natural numbers. More precisely, given an arbitrary sequent $\varphi_0 \ldots \varphi_n$, the antecedent is encoded as
\[
S(a_0, \ldots, a_{n - 1}),
\]
(cf. \textbf{Encoding Finite Sequences Over $\nat$ with Cantor's Pairing Function} for the definition of $S$) where
\begin{enumerate}[(1)]
\item $a_0 = G_F(\varphi_0)$, the G\"{o}del number of $\varphi_0$ ($G_F$ for \emph{G\"{o}del numbering for formulas});
%%
\item $a_{k + 1} = G_F(\varphi_{k + 1}) - G_F(\varphi_k) - 1$ for $0 \leq k < n - 1$.
\end{enumerate}
If the antecedent is empty (i.e. $n = 0$), then naturally it is encoded as $S(\Box) = 0$. Take the sequent given in the previous paragraph as an example, the antecedent is encoded as
\[
\begin{array}{ll}
\ & S(0, 1 - 0 - 1) \cr
= & S(0, 0) \cr
= & P(0, P(0, 0) + 1) + 1 \cr
= & P(0, 1) + 1 \cr
= & 1 + 1 \cr
= & 2.
\end{array}
\]
And then the whole sequent is encoded as
\[
P(S(a_0, \ldots, a_{n - 1}), G_F(\varphi_n)),
\]
where $S(a_0, \ldots, a_{n - 1})$ encodes the antecedent. Obviously the mapping from sequents to $\nat$ is bijective if $G_F$ is.\\
\ \\
We further encode sequences of sequents in a usual way: $S(a_0, \ldots, a_n)$ encodes a (nonempty) sequence of sequents in which $a_{k - 1}$ encodes the $k$th sequent for $0 \leq k \leq n$.\\
\ \\
Certainly not all natural numbers encode a derivation. Derivations are those (nonempty) sequences of sequents satisfying some conditions. [INCOMPLETE.]
%
\item \textbf{Note to Lemma 7.9 and the Arguments around It.} Recall that in the proof of 1.6, a procedure for enumerating all derivations for the symbol set $S_\infty$ was given. A procedure for enumerating all derivations in the sequent calculus associated with $S_\ar$ is essentially the same.\\
\ \\
The relation $H$ stated in text is representable in $\Phi$; the $S_\ar$-formula $\varphi_H (v_0, v_1)$ represents it. Given any $n, m \in \nat$, if $H n m$ holds then $\Phi \vdash \varphi_H (\mbf{n}, \mbf{m})$, otherwise $\Phi \vdash \neg\varphi_H (\mbf{n}, \mbf{m})$.\\
\ \\
It is, however, not the case for $\Phi \vdash \delta$: The formula $\Der{\Phi}(v_0)$ does \emph{not} represent it. As was demonstrated by 6.6, the set $\Phi^\vdash$ is not representable in $\Phi$; furthermore, the set $\Psi \colonequals \{ \psi \in L^{S_\ar} \ | \ \Phi \vdash \psi \}$ is also not representable in $\Phi$. (cf. \textbf{Note to Lemma 7.6}) That is to say, we do \emph{not} have for every $\delta \in L^{S_\ar}$,
\begin{center}
if not $\Phi \vdash \delta$ then $\Phi \vdash \neg\Der{\Phi}(\mbf{n}^\delta)$,
\end{center}
provided that $\Phi$ is consistent;\footnote{We may have $\neg\varphi_H(\mbf{n}^\delta, \mbf{m})$ for all $m \in \nat$ (i.e. not $\Phi \vdash \delta$) but $\Phi \vdash \Der{\Phi}(\mbf{n}^\delta)$, even if $\Phi$ is consistent. $\neg\Der{\Phi}(\mbf{n}^\varphi)$ says more than just ``$\varphi$ is not derivable from $\Phi$.''} though $\Phi \vdash \varphi_H (\mbf{n}^\delta, \mbf{m})$ for some $m \in \nat$ and hence $\Phi \vdash \Der{\Phi}(\mbf{n}^\delta)$ if $\Phi \vdash \delta$. Moreover, suppose $\Phi$ is consistent, if $\Phi \vdash \neg\Der{\Phi}(\mbf{n}^\delta)$ then not $\Phi \vdash \delta$. In summary,
\begin{enumerate}[(1)]
\item ``$\Phi \vdash \delta$'' necessarily implies ``$\Phi \vdash \Der{\Phi}(\mbf{n}^\delta)$''; if in addition $\Phi$ is consistent, then ``$\Phi \vdash \neg\Der{\Phi}(\mbf{n}^\delta)$'' necessarily implies ``not $\Phi \vdash \delta$'';
%%
\item ``not $\Phi \vdash \delta$'' does not imply ``$\Phi \vdash \neg\Der{\Phi}(\mbf{n}^\delta)$''.
\end{enumerate}
\ \\
Also note that the terms here are only slightly different from those in the proof of 6.6. To be precise, see the following comparisons:
\begin{center}
\begin{tabular}{lll}
$\psi(v_0) \colonequals \neg\gamma(v_0)$ & $\psi(\mbf{n}^\delta) = \neg\chi(\mbf{n}^\delta)$ (cf. 6.6) & $\psi = \neg\Der{\Phi}(\mbf{n}^\delta)$ \cr \hline\hline
\textsc{meaning} & $\delta$ is a \emph{sentence} not & $\exists v_1 \varphi_H$ is not true for the \cr
\                & derivable from $\Phi$          & \emph{formula} $\delta$ \cr
\                & \                              & (hence $\neg\varphi_H(\mbf{n}^\delta, \mbf{m})$ for all \cr
\                & \                              & $m \in \nat$; not $\Phi \vdash \delta$) \cr
\ & \ & \ \cr
\textsc{what the} & ``I am a \emph{sentence} & ``I am a \emph{formula} \cr
\textsc{fixed point} & not derviable from $\Phi$'' & not derivable from $\Phi$'' \cr
$\varphi$ \textsc{means} & \ & (More precisely: ``$\exists v_1 \varphi_H$ \cr
\ & \ & does not hold for me'') \cr
\ & \ & \ \cr
$\gamma$ \textsc{represents} & No & No \cr
\textsc{$\Phi \vdash \delta$?} & \ & \ \cr\hline
\end{tabular}
\end{center}
The formula $\chi(v_0)$ mentioned in 6.6 is logically equivalent to a formula involving $\Der{\Phi}(v_0)$: Since the set of all $S_\ar$-sentences is R-decidable (cf. essentially Exercise 1.3(b)) and so is $\Delta \colonequals \{ \mbf{n}^\delta \ | \ \delta \in L_0^{S_\ar} \}$, there is an $S_\ar$-formula $\delta_0(v_0)$ that represents $\Delta$ in $\Phi$. $\chi(v_0)$ is logically equivalent to $\Der{\Phi}(v_0) \land \delta_0(v_0)$.\\
\ \\
Finally, according to the discussion above, if $\Phi$ is consistent, then
\[
\Phi \vdash \consis{\Phi}
\]
implies
\begin{center}
not $\Phi \vdash \neg 0 \equiv 0$.
\end{center}
However, we cannot conclude the consistency of $\Phi$ by merely observing $\Phi \vdash \consis{\Phi}$, since for an inconsistent $\Phi$ we still have $\Phi \vdash \consis{\Phi}$; on the other hand, we may have $\Phi \vdash \Der{\Phi}(\mbf{n}^{\neg 0 \equiv 0})$ even though $\Phi$ is consistent.\\
\ \\
The situation is quite different for those sets $\Phi \subset L_0^{S_\ar}$ such that its theory $\Phi^{\models}$ is \emph{$\omega$-consistent} (cf. \cite{Dirk_van_Dalen}). A theory $T$ is $\omega$-consistent if for every $\psi(x) \in L_1^{S_\ar}$, $\psi(\mbf{n}) \in T$ for some $n \in \nat$ whenever $\exists x \psi(x) \in T$. Intuitively, $T$ being $\omega$-consistent says that there is a witness for every existential sentence that is a consequence of $T$. If $\Phi^{\models}$ is $\omega$-consistent, then $\Phi$ is consistent; the converse does not hold. Trivially, for $\omega$-consistent $\Phi^{\models}$, we do not have
\[
\Phi \vdash \Der{\Phi}(\mbf{n}^{\neg 0 \equiv 0});
\]
furthermore, for the fixed point $\varphi$ of $\neg\Der{\Phi}(x)$, we have
\begin{center}
not $\Phi \vdash \neg\varphi$,
\end{center}
for otherwise we would have $\Phi \vdash \Der{\Phi}(\mbf{n}^\varphi)$ and hence $\Phi \vdash \varphi$, which contradicts $\Phi$ being consistent (or equivalently, $\Phi^{\models}$ being $\omega$-consistent). In such a case, $\varphi$ serves as a concrete evidence for $\Phi$ being incomplete (cf. 7.9), in regard of 7.8. Clearly, for any arbitrary set $\Phi \subset L_0^{S_\ar}$ such that $\natstr \in \modelclass{S_\ar}{\Phi}$ (i.e. $\natstr$ is a model of $\Phi$), $\Phi^{\models}$ is $\omega$-consistent. In particular, $\Th_\pa$ is $\omega$-consistent.\\
\ \\
Thus, if $\Phi$ is R-decidable and allows representations such that $\natstr \in \modelclass{S_\ar}{\Phi}$, then for the fixed point $\varphi$ of $\neg\Der{\Phi}(x)$ we have
\begin{center}
neither $\Phi \vdash \varphi$ nor $\Phi \vdash \neg\varphi$.
\end{center}
However, it is true that $\natstr \models \varphi$: First of all, we have
\begin{center}
either $\natstr \models \varphi$ or $\natstr \models \neg\varphi$,
\end{center}
thus it suffices to show that $\natstr \models \neg\varphi$ is impossible. For the sake of contradiction, let us assume that $\natstr \models \neg\varphi$. Then we have $\natstr \models \Der{\Phi}(\mbf{n}^\varphi)$, or $\natstr \models \exists y \varphi_H(\mbf{n}^\varphi, y)$. It follows that there is some $m \in \nat$ such that $\natstr \models \varphi_H(\mbf{n}^\varphi, \mbf{m})$ and also $\Phi \vdash \varphi_H(\mbf{n}^\varphi, \mbf{m})$, hence $\Phi \vdash \varphi$ or $\nat \models \varphi$, a contradiction. The sentence $\varphi$ is traditionally called a \emph{G\"{o}del sentence} of $\Phi$, a sentence true of $\natstr$ and nevertheless underivable from $\Phi$. It is easy to verify that for any R-decidable $\Phi$ that allows representations with $\natstr \in \modelclass{S_\ar}{\Phi}$, there always exists such a G\"{o}del sentence $\varphi$.\\
\ \\
$[$INCOMPLETE: The ($***$) in text would be directly obtained if we added the following 5 sentences to $\Phi$:
\begin{enumerate}[(1)]
\item $(\Der{\Phi}(\mbf{n}^{(\varphi \land \neg\varphi)}) \rightarrow \Der{\Phi}(\mbf{n}^{\neg 0 \equiv 0}))$;
%%
\item $(\Der{\Phi}(\mbf{n}^\varphi) \rightarrow \Der{\Phi}(\mbf{n}^{(\neg\varphi \rightarrow (\varphi \land \neg\varphi))}))$;
%%
\item $((\Der{\Phi}(\mbf{n}^{\neg\varphi}) \land \Der{\Phi}(\mbf{n}^{(\neg\varphi \rightarrow (\varphi \land \neg\varphi))})) \rightarrow \Der{\Phi}(\mbf{n}^{(\varphi \land \neg\varphi)}))$;
%%
\item $(\Der{\Phi}(\mbf{n}^{\Der{\Phi}(\mbfs{n}^\varphi)}) \rightarrow \Der{\Phi}(\mbf{n}^{\neg\varphi}))$; and
%%
\item $(\Der{\Phi}(\mbf{n}^\varphi) \rightarrow \Der{\Phi}(\mbf{n}^{\Der{\Phi}(\mbfs{n}^\varphi)}))$.
\end{enumerate}
where $\varphi$ is the fixed point of $\Der{\Phi}(x)$, i.e. $\Phi \vdash \varphi \leftrightarrow \neg\Der{\Phi}(\mbf{n}^\varphi)$.$]$
%
\item \textbf{Note to G\"{o}del's Second Incompleteness Theorem 7.10.} If $\Phi \supset \Phi_\pa$, then $\Phi$ allows representations (cf. 7.2(b) and 7.4).
%
\item* \textbf{Solution to Exercise 7.12.} As suggested by the hint, we first prove
\begin{center}
\begin{tabular}{ll}
(D1) & for all $\varphi, \psi \in L^S$, $\Phi \vdash (\der{\varphi} \land \der{\psi} \rightarrow \der{(\varphi \land \psi)})$; and \cr
(D2) & $\Phi \vdash (\der{\varphi_0} \rightarrow \der{\neg\varphi_0})$,
\end{tabular}
\end{center}
provided that the formula $\mathrm{der}(v_0)$ for $\Phi$ satisfies (L1) - (L3).\\
\ \\
(D1). Let $\varphi, \psi \in L^S$ be given. Since $\vdash (\varphi \rightarrow (\psi \rightarrow (\varphi \land \psi)))$:
\[
\begin{array}{lllll}
1. & \varphi & \psi & \varphi & \mbox{(Assm)} \cr
2. & \varphi & \psi & \psi    & \mbox{(Assm)} \cr
3. & \varphi & \psi & (\varphi \land \psi) & \mbox{IV.3.6(b) applied to 1. and 2.} \cr
4. & \varphi & \    & (\psi \rightarrow (\varphi \land \psi)) & \mbox{IV.3.6(c) applied to 3.} \cr
5. & \       & \    & (\varphi \rightarrow (\psi \rightarrow (\varphi \land \psi))) & \mbox{IV.3.6(c) applied to 4.}
\end{array}
\]
we have that $\Phi \vdash (\varphi \rightarrow (\psi \rightarrow (\varphi \land \psi)))$, and further by (L1) that\\
\begin{tabular}{ll}
(1) & $\Phi \vdash \der{(\varphi \rightarrow (\psi \rightarrow (\varphi \land \psi)))}$.
\end{tabular}
\\
In addition, by (L2) and (L3) we have, respectively, that\\
\begin{tabular}{ll}
(2) & $\Phi \vdash (\der{\varphi} \land \der{(\varphi \rightarrow (\psi \rightarrow (\varphi \land \psi)))} \rightarrow \der{(\psi \rightarrow (\varphi \land \psi))})$; and that \cr
(3) & $\Phi \vdash (\der{\psi} \land \der{(\psi \rightarrow (\varphi \land \psi))} \rightarrow \der{(\varphi \land \psi)})$.
\end{tabular}
\\
We can pick a sequent $\Gamma_1 \subset \Phi$ such that (1) - (3) holds; or more precisely, the following hold:
\begin{center}
\begin{tabular}{l}
$\Gamma_1 \vdash \der{(\varphi \rightarrow (\psi \rightarrow (\varphi \land \psi)))}$; \cr
$\Gamma_1 \vdash (\der{\varphi} \land \der{(\varphi \rightarrow (\psi \rightarrow (\varphi \land \psi)))} \rightarrow \der{(\psi \rightarrow (\varphi \land \psi))})$; and \cr
$\Gamma_1 \vdash (\der{\psi} \land \der{(\psi \rightarrow (\varphi \land \psi))} \rightarrow \der{(\varphi \land \psi)})$.
\end{tabular}
\end{center}
A derivation for $\Phi \vdash (\der{\varphi} \land \der{\psi} \rightarrow \der{(\varphi \land \psi)})$ is given below (we write $\delta_0$ for $(\der{\varphi} \land \der{\psi} \rightarrow \der{(\varphi \land \psi)}$, $\delta_1$ for $\der{(\varphi \rightarrow (\psi \rightarrow (\varphi \land \psi)))}$, $\delta_2$ for $(\der{\varphi} \land \der{(\varphi \rightarrow (\psi \rightarrow (\varphi \land \psi)))} \rightarrow \der{(\psi \rightarrow (\varphi \land \psi))})$, $\delta_3$ for $(\der{\psi} \land \der{(\psi \rightarrow (\varphi \land \psi))} \rightarrow \der{(\varphi \land \psi)})$, and $\delta_4$ for $(\der{\psi} \land \der{(\psi \rightarrow (\varphi \land \psi))})$, respectively):
\[
\begin{array}{lllll}
1. & \Gamma_1 & \ & \delta_1 & \mbox{premise} \cr
2. & \Gamma_1 & \ & \delta_2 & \mbox{premise} \cr
3. & \Gamma_1 & \ & \delta_3 & \mbox{premise} \cr
4. & \Gamma_1 & (\der{\varphi} \land \der{\psi}) & (\der{\varphi} \land \der{\psi}) & \mbox{(Assm)} \cr
5. & \Gamma_1 & (\der{\varphi} \land \der{\psi}) & \der{\varphi} & \mbox{IV.3.6(d1) applied} \cr
\  & \      & \                                & \             & \mbox{to 4.} \cr
6. & \Gamma_1 & (\der{\varphi} \land \der{\psi}) & \der{\psi} & \mbox{IV.3.6(d2) applied} \cr
\  & \      & \                                & \          & \mbox{to 4.} \cr
7. & \Gamma_1 & (\der{\varphi} \land \der{\psi}) & \delta_1 & \mbox{(Ant) applied to 1.} \cr
8. & \Gamma_1 & (\der{\varphi} \land \der{\psi}) & \delta_2 & \mbox{(Ant) applied to 2.} \cr
9. & \Gamma_1 & (\der{\varphi} \land \der{\psi}) & \delta_3 & \mbox{(Ant) applied to 3.} \cr
10. & \Gamma_1 & (\der{\varphi} \land \der{\psi}) & (\der{\varphi} \land \delta_1) & \mbox{IV.3.6(b) applied} \cr
\   & \      & \                                & \                              & \mbox{to 5. and 7.} \cr
11. & \Gamma_1 & (\der{\varphi} \land \der{\psi}) & \der{(\psi \rightarrow (\varphi \land \psi))} & \mbox{IV.3.5 applied to 8.} \cr
\   & \      & \                                & \                                             & \mbox{and 10.} \cr
12. & \Gamma_1 & (\der{\varphi} \land \der{\psi}) & \delta_4 & \mbox{IV.3.6(b) applied} \cr
\   & \      & \                                & \ & \mbox{to 6. and 11.} \cr
13. & \Gamma_1 & (\der{\varphi} \land \der{\psi}) & \der{(\varphi \land \psi)} & \mbox{IV.3.5 applied} \cr
\ & \ & \ & \ & \mbox{to 9. and 12.} \cr
14. & \Gamma_1 & \ & \delta_0 & \mbox{IV.3.6(c) applied} \cr
\ & \ & \ & \ & \mbox{to 13.}
\end{array}
\]
\ \\
(D2). According to the premise $\Phi \vdash (\varphi_0 \leftrightarrow \neg\der{\varphi_0})$, we have that $\Phi \vdash (\der{\varphi_0} \rightarrow \neg\varphi_0)$: Let $\Gamma_2^\prime \subset \Phi$ be a sequent such that $\Gamma_2^\prime \vdash (\varphi_0 \leftrightarrow \neg\der{\varphi_0})$, then the derivation:
\[
\begin{array}{lllll}
1. & \Gamma_2^\prime & \ & (\varphi_0 \leftrightarrow \neg\der{\varphi}) & \mbox{premise} \cr
2. & \Gamma_2^\prime & \varphi_0 & \varphi_0 & \mbox{(Assm)} \cr
3. & \Gamma_2^\prime & \varphi_0 & (\varphi_0 \leftrightarrow \neg\der{\varphi}) & \mbox{(Ant) applied to 1.} \cr
4. & \Gamma_2^\prime & \varphi_0 & (\varphi_0 \lor \neg\der{\varphi_0}) & \mbox{($\lor$S) applied to 2.} \cr
5. & \Gamma_2^\prime & \varphi_0 & (\varphi_0 \land \neg\der{\varphi_0}) & \mbox{IV.3.5 applied to 1. and 4.} \cr
6. & \Gamma_2^\prime & \varphi_0 & \neg\der{\varphi_0} & \mbox{IV.3.6(d2) applied to 5.} \cr
7. & \Gamma_2^\prime & \der{\varphi_0} & \neg\varphi_0 & \mbox{(Cp)(d) applied to 6.} \cr
8. & \Gamma_2^\prime & \ & (\der{\varphi_0} \rightarrow \neg\varphi_0) & \mbox{IV.3.6(c) applied to 7.}
\end{array}
\]
demonstrates that $\Phi \vdash (\der{\varphi_0} \rightarrow \neg\varphi_0)$.\\
\ \\
From (L1) it follows that\\
\begin{tabular}{ll}
(4) & $\Phi \vdash \der{(\der{\varphi_0} \rightarrow \neg\varphi_0)}$.
\end{tabular}
\\
Moreover, by (L2) and (L3) we have, respectively, that\\
\begin{tabular}{ll}
(5) & $\Phi \vdash (\der{\der{\varphi_0}} \land \der{(\der{\varphi_0} \rightarrow \neg\varphi_0)} \rightarrow \der{\neg\varphi_0})$; and \cr
(6) & $\Phi \vdash (\der{\varphi_0} \rightarrow \der{\der{\varphi_0}})$.
\end{tabular}
\\
\ \\
Then we can choose a sequent $\Gamma_2 \subset \Phi$ such that (4) - (6) hold, i.e.
\begin{center}
\begin{tabular}{l}
$\Gamma_2 \vdash \der{(\der{\varphi_0} \rightarrow \neg\varphi_0)}$; \cr
$\Gamma_2 \vdash (\der{\der{\varphi_0}} \land \der{(\der{\varphi_0} \rightarrow \neg\varphi_0)} \rightarrow \der{\neg\varphi_0})$; and \cr
$\Gamma_2 \vdash (\der{\varphi_0} \rightarrow \der{\der{\varphi_0}})$.
\end{tabular}
\end{center}
The derivation below shows that $\Phi \vdash (\der{\varphi_0} \rightarrow \der{\neg\varphi_0})$ (we write $\varepsilon_1$ for $\der{(\der{\varphi} \rightarrow \neg\varphi_0)}$, $\varepsilon_2$ for $(\der{\der{\varphi_0}} \land \der{(\der{\varphi_0} \rightarrow \neg\varphi_0)} \rightarrow \der{\neg\varphi_0})$, and $\varepsilon_3$ for $(\der{\varphi_0} \rightarrow \der{\der{\varphi_0}})$):
\[
\begin{array}{lllll}
1. & \Gamma_2 & \ & \varepsilon_1 & \mbox{premise} \cr
2. & \Gamma_2 & \ & \varepsilon_2 & \mbox{premise} \cr
3. & \Gamma_2 & \ & \varepsilon_3 & \mbox{premise} \cr
4. & \Gamma_2 & \der{\varphi_0} & \varepsilon_1 & \mbox{(Ant) applied to 1.} \cr
5. & \Gamma_2 & \der{\varphi_0} & \varepsilon_2 & \mbox{(Ant) applied to 2.} \cr
6. & \Gamma_2 & \der{\varphi_0} & \varepsilon_3 & \mbox{(Ant) applied to 3.} \cr
7. & \Gamma_2 & \der{\varphi_0} & \der{\varphi_0} & \mbox{(Assm)} \cr
8. & \Gamma_2 & \der{\varphi_0} & \der{\der{\varphi_0}} & \mbox{IV.3.5 applied to 6. and 7.} \cr
9. & \Gamma_2 & \der{\varphi_0} & (\der{\der{\varphi_0}} \land \varepsilon_1) & \mbox{IV.3.6(b) applied to 8.} \cr
\ & \ & \ & \ & \mbox{and 4.} \cr
10. & \Gamma_2 & \der{\varphi_0} & \der{\neg\varphi_0} & \mbox{IV.3.5 applied to 5. and 9.} \cr
11. & \Gamma_2 & \ & (\der{\varphi_0} \rightarrow \der{\neg\varphi_0}) & \mbox{IV.3.6(c) applied to 10.}
\end{array}
\]
\ \\
Next, observe that $\vdash (\varphi_0 \land \neg\varphi_0 \rightarrow \neg\underline{0} \equiv \underline{0})$:
\[
\begin{array}{llll}
1. & (\varphi_0 \land \neg\varphi_0) & (\varphi_0 \land \neg\varphi_0) & \mbox{(Assm)} \cr
2. & (\varphi_0 \land \neg\varphi_0) & \varphi_0 & \mbox{IV.3.6(d1) applied to 1.} \cr
3. & (\varphi_0 \land \neg\varphi_0) & \neg\varphi_0 & \mbox{IV.3.6(d2) applied to 1.} \cr
4. & (\varphi_0 \land \neg\varphi_0) & \neg\underline{0} \equiv \underline{0} & \mbox{(Ctr$^\prime$) applied to 2. and 3.} \cr
5. & \ & (\varphi_0 \land \neg\varphi_0 \rightarrow \neg\underline{0} \equiv \underline{0}) & \mbox{IV.3.6(c) applied to 4.}
\end{array}
\]
therefore $\Phi \vdash (\varphi_0 \land \neg\varphi_0 \rightarrow \neg\underline{0} \equiv \underline{0})$. From this and (L1) it follows that\\
\begin{tabular}{ll}
(7) & $\Phi \vdash \der{(\varphi_0 \land \neg\varphi_0 \rightarrow \neg\underline{0} \equiv \underline{0})}$.
\end{tabular}
\\
Also, by (L2), (D1) and (D2), respectively, we get\\
\begin{tabular}{ll}
(8) & $\Phi \vdash (\der{(\varphi_0 \land \neg\varphi_0)} \land \der{(\varphi_0 \land \neg\varphi_0 \rightarrow \neg\underline{0} \equiv \underline{0})} \rightarrow \der{\neg\underline{0} \equiv \underline{0}})$; \cr
(9) & $\Phi \vdash (\der{\varphi_0} \land \der{\neg\varphi_0} \rightarrow \der{(\varphi_0 \land \neg\varphi_0)})$; and \cr
(10) & $\Phi \vdash (\der{\varphi_0} \rightarrow \der{\neg\varphi_0})$.
\end{tabular}
\\
\ \\
Hence we can pick a sequent $\Gamma \subset \Phi$ such that (7) - (10) hold:
\begin{center}
\begin{tabular}{l}
$\Gamma \vdash \der{(\varphi_0 \land \neg\varphi_0 \rightarrow \neg\underline{0} \equiv \underline{0})}$; \cr
$\Gamma \vdash (\der{(\varphi_0 \land \neg\varphi_0)} \land \der{(\varphi_0 \land \neg\varphi_0 \rightarrow \neg\underline{0} \equiv \underline{0})} \rightarrow \der{\neg\underline{0} \equiv \underline{0}})$; \cr
$\Gamma \vdash (\der{\varphi_0} \land \der{\neg\varphi_0} \rightarrow \der{(\varphi_0 \land \neg\varphi_0)})$; and \cr
$\Gamma \vdash (\der{\varphi_0} \rightarrow \der{\neg\varphi_0})$.\cr
\end{tabular}
\end{center}
The derivation below shows that $\Phi \vdash (\neg\der{\neg\underline{0} \equiv \underline{0}} \rightarrow \neg\der{\varphi_0})$ (we write $\chi_1$ for $\der{(\varphi_0 \land \neg\varphi_0)}$, $\chi_2$ for $\der{(\varphi_0 \land \neg\varphi_0 \rightarrow \neg\underline{0} \equiv \underline{0})}$, and $\chi_3$ for $(\der{\varphi_0} \land \der{\neg\varphi_0})$, respectively):
\[
\begin{array}{lllll}
1. & \Gamma & \ & \chi_2 & \mbox{premise} \cr
2. & \Gamma & \ & (\chi_1 \land \chi_2 \rightarrow \der{\neg\underline{0} \equiv \underline{0}}) & \mbox{premise} \cr
3. & \Gamma & \ & (\chi_3 \rightarrow \chi_1) & \mbox{premise} \cr
4. & \Gamma & \ & (\der{\varphi_0} \rightarrow \der{\neg\varphi_0}) & \mbox{premise} \cr
5. & \Gamma & \der{\varphi_0} & \chi_2 & \mbox{(Ant) applied to 1.} \cr
6. & \Gamma & \der{\varphi_0} & (\chi_1 \land \chi_2 \rightarrow \der{\neg\underline{0} \equiv \underline{0}}) & \mbox{(Ant) applied to 2.} \cr
7. & \Gamma & \der{\varphi_0} & (\chi_3 \rightarrow \chi_1) & \mbox{(Ant) applied to 3.} \cr
8. & \Gamma & \der{\varphi_0} & (\der{\varphi_0} \rightarrow \der{\neg\varphi_0}) & \mbox{(Ant) applied to 4.} \cr
9. & \Gamma & \der{\varphi_0} & \der{\varphi_0} & \mbox{(Assm)} \cr
10. & \Gamma & \der{\varphi_0} & \der{\neg\varphi_0} & \mbox{IV.3.5 applied to 8.} \cr
\ & \ & \ & \ & \mbox{and 9.} \cr
11. & \Gamma & \der{\varphi_0} & \chi_3 & \mbox{IV.3.6(b) applied} \cr
\ & \ & \ & \ & \mbox{to 9. and 10.} \cr
12. & \Gamma & \der{\varphi_0} & \chi_1 & \mbox{IV.3.5 applied to 7.} \cr
\ & \ & \ & \ & \mbox{and 11.} \cr
13. & \Gamma & \der{\varphi_0} & (\chi_1 \land \chi_2) & \mbox{IV.3.6(b) applied} \cr
\ & \ & \ & \ & \mbox{to 12. and 5.} \cr
14. & \Gamma & \der{\varphi_0} & \der{\neg\underline{0} \equiv \underline{0}} & \mbox{IV.3.5 applied to 6.} \cr
\ & \ & \ & \ & \mbox{and 13.} \cr
15. & \Gamma & \neg\der{\neg\underline{0} \equiv \underline{0}} & \neg\der{\varphi_0} & \mbox{(Cp)(a) applied} \cr
\ & \ & \ & \ & \mbox{to 14.} \cr
16. & \Gamma & \ & (\neg\der{\neg\underline{0} \equiv \underline{0}} \rightarrow \neg\der{\varphi_0}) & \mbox{IV.3.6(c) applied} \cr
\ & \ & \ & \ & \mbox{to 15.}
\end{array}
\]
\ \\
Finally, from the following argument we conclude that
\begin{center}
not $\Phi \vdash \neg\der{\neg\underline{0} \equiv \underline{0}}$.
\end{center}
If $\Phi \vdash \neg\der{\neg\underline{0} \equiv \underline{0}}$, then we could choose a sequent $\Gamma_0 \subset \Phi$ with
\begin{center}
\begin{tabular}{l}
$\Gamma_0 \vdash (\varphi_0 \leftrightarrow \neg\der{\varphi_0})$; \cr
$\Gamma_0 \vdash (\neg\der{\neg\underline{0} \equiv \underline{0}} \rightarrow \neg\der{\varphi_0})$; and \cr
$\Gamma_0 \vdash \neg\der{\neg\underline{0} \equiv \underline{0}}$,
\end{tabular}
\end{center}
and obtain from the derivation:
\[
\begin{array}{lllll}
1. & \Gamma_0 & \ & (\varphi_0 \leftrightarrow \neg\der{\varphi_0}) & \mbox{premise} \cr
2. & \Gamma_0 & \ & (\neg\der{\neg\underline{0} \equiv \underline{0}} \rightarrow \neg\der{\varphi_0}) & \mbox{premise} \cr
3. & \Gamma_0 & \ & \neg\der{\neg\underline{0} \equiv \underline{0}} & \mbox{premise} \cr
4. & \Gamma_0 & \ & \neg\der{\varphi_0} & \mbox{IV.3.5 applied to 2.} \cr
\ & \ & \ & \ & \mbox{and 3.} \cr
5. & \Gamma_0 & \neg\der{\varphi_0} & \neg\der{\varphi_0} & \mbox{(Assm)} \cr
6. & \Gamma_0 & \neg\der{\varphi_0} & (\varphi_0 \leftrightarrow \neg\der{\varphi_0}) & \mbox{(Ant) applied to 1.} \cr
7. & \Gamma_0 & \neg\der{\varphi_0} & (\varphi_0 \lor \neg\der{\varphi_0}) & \mbox{($\lor$S) applied to 5.} \cr
8. & \Gamma_0 & \neg\der{\varphi_0} & (\varphi_0 \land \neg\der{\varphi_0}) & \mbox{IV.3.5 applied to 6.} \cr
\ & \ & \ & \ & \mbox{and 7.} \cr
9. & \Gamma_0 & \neg\der{\varphi_0} & \varphi_0 & \mbox{IV.3.6(d1) applied} \cr
\ & \ & \ & \ & \mbox{to 8.} \cr
10. & \Gamma_0 & \ & \varphi_0 & \mbox{(Ch) applied to 4.} \cr
\ & \ & \ & \ & \mbox{and 9.}
\end{array}
\]
that
\begin{center}
($*$) \hfill $\Phi \vdash \neg\der{\varphi_0}$,\hfill (line 4 of the derivation)
\end{center}
and also
\begin{center}
(+) \hfill $\Phi \vdash \varphi_0$. \hfill (line 10 of the derivation)
\end{center}
By (L1), (+) entails that
\begin{center}
($**$) \hfill $\Phi \vdash \der{\varphi_0}$. \hfill \phantom{($**$)}
\end{center}
Then, ($*$) and ($**$) together contradict the hypothesis that $\Phi$ is consistent.\\
\ \\
(Alternatively, we may take further step to obtain
\begin{center}
(++) \hfill $\Phi \vdash \neg\varphi_0$ \hfill \phantom{(++)}
\end{center}
from ($**$) and the hypothesis that $\Phi \vdash (\varphi_0 \leftrightarrow \neg\der{\varphi_0})$. (+) and (++) together still contradict that $\Phi$ is consistent; this is what we did in the proof of 7.9.)\nolinebreak\hfill$\talloblong$
\\
\ \\
\textit{Remark.} The argument of showing $\Phi \vdash (\neg\der{\neg\underline{0} \equiv \underline{0}} \rightarrow \neg\der{\varphi_0})$ in this exercise results in ($***$) in text.\\
\ \\
Recall that in the argument around 7.9 in text, the consistency of an R-decidable set $\Phi$ of $S_\ar$-sentences that allows representations is expressed by
\[
\consis{\Phi} \colonequals \neg\Der{\Phi}(\mbf{n}^{\neg 0 \equiv 0}).
\]
Indeed, the G\"{o}del number $\mbf{n}^{\neg 0 \equiv 0}$ in the righthand-side sentence above can be replaced by any G\"{o}del number of a sentence that is unsatisfiable: By Definition IV.7.1, we have
\begin{center}
$\con \Phi$ \ \ \ iff \ \ \ not $\Phi \vdash \psi$ for some $\psi \in L_0^{S_\ar}$ with $\models 
\neg\psi$.
\end{center}
This point was made clear in the argument for showing $\Phi \vdash (\neg\der{\neg\underline{0} \equiv \underline{0}} \rightarrow \neg\der{\varphi_0})$.
%
\item* \textbf{Note to Arguments around 7.11.} There is a typo 2 lines above 7.11: the discussions about natural numbers defined in $\zfc$ are in VII.7, not in VIII.7.\\
\ \\
$[$INCOMPLETE: How can we translate $S_\ar$-formulas involving function symbols $+$ and $\cdot$ to $\{ \mbf{\in} \}$-formulas?$]$\\
\ \\
$[$INCOMPLETE: How can we apply the results of Matijasevi$\check{\rm c}$ to obtain the formulation of 7.11 in the last paragraph?$]$
\end{enumerate}
%End of Section X.7--------------------------------------------------------------------------------
%End of Chapter X----------------------------------------------------------------------------------
%%Chapter XI----------------------------------------------------------------------------------------
{\LARGE \bfseries XI \\ \\ Free Models and Logic Programming}
\\
\\
\\
%Section XI.1--------------------------------------------------------------------------------------
{\large \S1. Herbrand's Theorem}
\begin{enumerate}[1.]
\item \textbf{Note to the Discussion at the Bottom of Page 190.} Actually $T_k^\Phi$ is the universe of \emph{exactly one} substructure of $\mathfrak{T}^\Phi$, namely $\mathfrak{T}_k^\Phi$, the substructure $[T_k^\Phi]^{\mathfrak{T}^\Phi}$ generated by $T_k^\Phi$ in $\mathfrak{T}^\Phi$ (cf. the discussion after III.5.4).
%
\item \textbf{Note to Lemma 1.2.} (a) and (b) follow from III.5.5.
%
\item \textbf{Note to the Proof of Lemma 1.3.} In general, $\Phi$ \emph{may fail} to be satisfied by $\mathfrak{I}_k$, where $\mathfrak{I}$ is a model of $\Phi_0$ constructed by the methods mentioned in V.2 and V.3; recall that we enlarged $S$ with (possibly infinitely many) constants into a ``bigger'' symbol set $S^\prime$. Thus $\mathfrak{I}$ contains, for instance, $\bar{c}$ as an element in the universe, in which $c \in S^\prime \setminus S$ is a new constant symbol. In the process of constructing $\mathfrak{I}$, $\varphi (x \mid c)$ may fail to be added into $\Phi_0$, where $\forall x \varphi \in \Phi$; and as a result, possibly,
\begin{center}
not $\mathfrak{I} \models \varphi (x \mid c)$.
\end{center}
That is why in this proof we applied arguments using the Compactness Theorem to restrict ourselves to the case in which $S$ is finite and directly dealt with $\Phi_0$ in which $\free{\Phi_0}$ is finite (so that we do not have to add constants into $S$).\\
\ \\
To see why it suffices to consider finite symbol set, notice that if $\Phi_0$ is satisfiable, then in particular $\Phi_0 \cap L^{S^\prime}$ is also satisfiable, where $S^\prime \subset S$ is an arbitrary finite symbol set. Through the argument of the proof, we have $\Phi \cap L^{S^\prime}$ is satisfiable. Since every finite subset of $\Phi$ is a subset of $\Phi \cap L^{S^\prime}$ for some suitable (finite) $S^\prime$ and hence is satisfiable, $\Phi$ itself is satisfiable as well, by the Compactness Theorem.\\
\ \\
Finally, we concluded
\[
\mathfrak{I}^\Theta_k \models \forall x_1 \ldots \forall x_m \varphi
\]
from
\begin{center}
for all $t_1, \ldots, t_m \in T^S_k$: $\mathfrak{I}^\Theta_k \sbst{\overline{t_1} \ldots \overline{t_m}}{x_1 \ldots x_m} \models \varphi$,
\end{center}
using the fact below:
\begin{quote}
\emph{Let $B$ be a nonempty set and $P(b)$ a statement about the elements $b \in B$ which depends only on the argument $b$. Also, let $A$ be a nonempty set and $f$ a mapping from $A$ to $B$. If
\begin{center}
for all $a \in A$, $P(f(a))$ holds,
\end{center}
then
\begin{center}
for all $b \in f(A) \subset B$, $P(b)$ holds.
\end{center}}
\end{quote}
%
\item \textbf{Note to 1.5 (b).} The logical equivalence betweeen (i) and (ii) follows from the argument below:
\begin{center}
\begin{tabular}{ll}
\   & $\exists x_1 \ldots \exists x_n \varphi$ \cr
iff & $\forall x_1 \, x_1 \equiv x_1 \vdash \exists x_1 \ldots \exists x_n \varphi$ \cr
iff & there are $j \geq 1$ and terms $t_{11}, \ldots t_{1n}, \ldots, t_{j1}, \ldots, t_{jn} \in T_k^S$ with\cr
\   & $\forall x_1 \, x_1 \equiv x_1 \vdash \varphi( \stackrel{n}{x} \mid \stackrel{n}{t_1}) \lor \ldots \lor \varphi(\stackrel{n}{x} \mid \stackrel{n}{t_j})$ \ \ \ (by 1.4) \cr
iff & there are $j \geq 1$ and terms $t_{11}, \ldots t_{1n}, \ldots, t_{j1}, \ldots, t_{jn} \in T_k^S$ with\cr
\   & $\vdash \varphi( \stackrel{n}{x} \mid \stackrel{n}{t_1}) \lor \ldots \lor \varphi(\stackrel{n}{x} \mid \stackrel{n}{t_j})$.
\end{tabular}
\end{center}
%
\item \textbf{Solution to Exercise 1.6.} (a) $\varphi$ is logically equivalent to $(Ry \lor \forall x \neg Rx)$. Thus, $\exists y \varphi$ is logically equivalent to $(\exists x Rx \lor \forall x \neg Rx)$. Therefore $\vdash \exists y \varphi$.\\
\ \\
(b) Suppose that for some $j \geq 1$ and some $t_1, \ldots, t_j \in T^S$,
\[
\vdash \varphi (y \mid t_1) \lor \ldots \lor \varphi (y \mid t_j).
\]
Then we have
\[
\models \varphi (y \mid t_1) \lor \ldots \lor \varphi (y \mid t_j)
\]
by the Adequacy Theorem.\\
\ \\
Choose an $S$-interpretation $\mathfrak{I} = (\mathfrak{A}, \beta)$ with
\begin{enumerate}[(i)]
\item $\{ a, b \}$ as the universe;
%%
\item $c^\mathfrak{A} = a$, and $\beta (v_n) = a$ for all $n \in \nat$;
%%
\item $R^\mathfrak{A} = \{ b \}$.
\end{enumerate}
Then $\mathfrak{I} \models \exists x Rx$, and for all $t \in T^S$, not $\mathfrak{I} \models Rt$. Hence
\begin{center}
not $\mathfrak{I} \models \varphi (y \mid t_1) \lor \ldots \lor \varphi (y \mid t_j)$,
\end{center}
a contradiction.
%
\item \textbf{Solution to Exercise 1.7.} To show that in general 1.5 (b)(ii) cannot be strengthened by claiming $j = 1$, let $S = \{ c, d \}$ and $\varphi = (\neg c \equiv d \rightarrow \neg v_1 \equiv v_0)$. As $\exists v_1 \varphi$ is logically equivalent to $(\neg c \equiv d \rightarrow \exists v_1 \neg v_1 \equiv v_0)$, we have $\vdash \exists v_1 \varphi.$\\
\ \\
However, there is no $t \in T^S$ such that $\vdash \varphi (v_1 \mid t)$, for we can always choose an $S$-interpretation $\mathfrak{I}$ such that not $\mathfrak{I} \models \varphi (v_1 \mid t)$:
\begin{enumerate}[(1)]
\item If $t = c$, let $\mathfrak{I}$ contain in the universe two distinct elements $a$, $b$; $\mathfrak{I}(c) = a$, $\mathfrak{I}(d) = b$, and $\mathfrak{I}(v_n) = a$ for all $n \in \nat$.
%%
\item If $t = d$, let $\mathfrak{I}(v_n) = b$ for $n \in \nat$ in $\mathfrak{I}$ above.
%%
\item If $t = v_n$ for some $n \in \nat$, let $\mathfrak{I}$ be the same as in (1).
\end{enumerate}
\ \\
It follows that we also cannot strenthen 1.4 by claiming $j = 1$ in (b) or (c), otherwise we could do this to 1.5 (b)(ii) as in the argument given by \textbf{Note to 1.5 (b)}.
\end{enumerate}
%End of Section XI.1-------------------------------------------------------------------------------
\
\\
\\
%Section XI.2-------------------------------------------------------------------------------------
{\large \S2. Free Models and Universal Horn Formulas}
\begin{enumerate}[1.]
\item \textbf{Note to Theorem 2.1.} For (ii), let $a_i = \overline{t_i}$ with suitable $t_i \in T^S$ for $1 \leq i \leq n$. Then
\[
\begin{array}{ll}
\ & \pi (f^{\mathfrak{T}^\Phi}(a_1, \ldots, a_n)) \cr
= & \pi (f^{\mathfrak{T}^\Phi}(\overline{t_1}, \ldots, \overline{t_n})) \cr
= & \pi (\overline{ft_1 \ldots t_n}) \cr
= & \mathfrak{I}(ft_1 \ldots t_n) \cr
= & f^\mathfrak{A} (\mathfrak{I}(t_1), \ldots, \mathfrak{I}(t_n)) \cr
= & f^\mathfrak{A} (\pi (\overline{t_1}), \ldots, \pi (\overline{t_n})) \cr
= & f^\mathfrak{A} (\pi (a_1), \ldots, \pi (a_n)).
\end{array}
\]
For (iii), we have
\[
\pi (c^{\mathfrak{T}^\Phi}) = \pi (\bar{c}) = \mathfrak{I}(c) = c^\mathfrak{A}.
\]
%
\item \textbf{Note to Lemma 2.3.} (a) This can be shown by induction on universal Horn formulas, i.e. formulas generated by the calculus given in 2.2:
\begin{enumerate}[(i)]
\item Formulas obtained by applying 2.2 (1) are formulas of the form (H1) if containing no negated atoms, and of the form (H2) otherwise. They are by themselves conjunctions (with single conjuncts).
%%
\item Fromulas obtained by applying 2.2 (2) are formulas of the form (H3). They are by themselves conjunctions (with single conjuncts).
%%
\item Let $\varphi$ and $\psi$ be universal Horn formulas such that the induction hypothesis holds, i.e. they are logically equivalent to $\varphi^\prime$ and $\psi^\prime$, respectively, each of which is a conjunction of formulas of the form (H1) - (H3).\\
\ \\
We have the formula $(\varphi \land \psi)$, which is obtained by applying 2.2 (3) to $\varphi$ and $\psi$, is logically equivalent to $(\varphi^\prime \land \psi^\prime)$, a conjuction of formulas of the form (H1) - (H3).
%%
\item Let $\varphi$ be a universal Horn formula such that the induction hypothesis holds, i.e. it is logically equivalent to $(\varphi_0^\prime \land \ldots \land \varphi_n^\prime)$, in which $\varphi_0^\prime, \ldots, \varphi_n^\prime$ are formulas of the form (H1) - (H3).\\
\ \\
We have $\forall x \varphi$, which is obtained by applying 2.2 (4) to $\varphi$, is logically equivalent to $\forall x (\varphi_0^\prime \land \ldots \land \varphi_n^\prime)$, which in turn is logically equivalent to $(\forall x \varphi_0^\prime \land \ldots \land \forall x \varphi_n^\prime)$, a conjunction of formulas of the form (H1) - (H3).
\end{enumerate}
In each case, if the formula to be dealt with is in $L_k^S$ then the resulting logically equivalent formula is also in $L_k^S$.\\
\ \\
(b) follows from (a) and the fact that
\begin{quote}
\emph{the formula $(\forall x \varphi \land \forall y \psi)$ is logically equivalent to $\forall z \forall u \left( \varphi\sbst{z}{x} \land \psi\sbst{u}{y} \right)$, where $z$ and $u$ do not occur in $\forall y \psi$ and $\forall x \varphi$, respectively.}
\end{quote}

(c) We show
\begin{quote}
\emph{for all universal Horn formulas $\varphi$, if $x_1, \ldots, x_n$ are pairwise distinct then, for $t_1, \ldots, t_n \in T^S$, $\varphi (\stackrel{n}{x} \mid \stackrel{n}{t})$ is also a universal Horn formula}
\end{quote}
by induction on formulas generated by the calculus given in 2.2:
\begin{enumerate}[(i)]
\item For formulas obtained by applying 2.2 (1) or (2), the statement is trivially true.
%%
\item Let $\varphi$ and $\psi$ be universal Horn formulas such that the induction hypothesis holds: If $x_1, \ldots, x_n$ are pairwise distinct, then for $t_1, \ldots, t_n \in T^S$, $\varphi (\stackrel{n}{x} \mid \stackrel{n}{t})$ and $\psi (\stackrel{n}{x} \mid \stackrel{n}{t})$ are universal Horn formulas. It immediately follows that $(\varphi \land \psi) (\stackrel{n}{x} \mid \stackrel{n}{t}) = (\varphi (\stackrel{n}{x} \mid \stackrel{n}{t}) \land \psi (\stackrel{n}{x} \mid \stackrel{n}{t}))$ is also a universal Horn formula, in which $(\varphi \land \psi)$ is obtained by applying 2.2 (3) to $\varphi$ and $\psi$. The statement is also true in this case.
%%
\item Let $\varphi$ be a universal Horn formula such that the induction hypothesis holds: If $x_1, \ldots, x_n$ are pairwise distinct, then for $t_1, \ldots, t_n \in T^S$, $\varphi (\stackrel{n}{x} \mid \stackrel{n}{t})$ is a universal Horn formula.\\
\ \\
So, if $x_1, \ldots, x_n$ are pairwise distinct, then for $t_1, \ldots, t_n \in T^S$,
\[
(\forall x \varphi ) (\stackrel{n}{x} \mid \stackrel{n}{t}) = \begin{cases}
\forall x (\varphi (\stackrel{n}{x} \mid \stackrel{n}{t})) & \mbox{if \(x\) does not occur in \(\seq[1]{x}{n}\)}; \cr
\forall x (\varphi (\stackrel{n}{x} \mid \stackrel{n}{t^\prime})) & \mbox{otherwise}
\end{cases}
\]
is also a universal Horn formula, where
\[
t_i^\prime = \begin{cases}
t_i & \mbox{if \(x_i \neq x\)};\cr
x & \mbox{otherwise}
\end{cases}
\]
for $1 \leq i \leq n$ and $\forall x \varphi$ is obtained by applying 2.2 (4) to $\varphi$. The statement is also true in this case.
\end{enumerate}
%
\item \textbf{Note to Corollary 2.5.} If moreover $\Phi \subset L_k^S$, then $\mathfrak{I}_k^\Phi$ is a model of $\Phi$ (cf. 1.2 (c)), but not necessarily a free model.
%
\item \textbf{Note to Exercise 2.8.} Since neither $\Phi \vdash Pc$ nor $\Phi \vdash Pd$, by 1.1 (b) we have neither $\mathfrak{I}^\Phi \models Pc$ nor $\mathfrak{I}^\Phi \models Pd$, and therefore
\begin{center}
not $\mathfrak{I}^\Phi \models (Pc \lor Pd)$,
\end{center}
that is, $\mathfrak{I}^\Phi$ is not a model of $\Phi$.\\
\ \\
As $(Pc \lor Pd)$ is satisfiable, it follows from 2.5 that it is not logically equivalent to a universal Horn formula, and of course not logically equivalent to a universal Horn sentence.\\
\ \\
We further show that $(Pc \lor Pd)$ is not even logically equivalent to a Horn sentence. To see this, take two $S$-structures $\mathfrak{A}_1$ and $\mathfrak{A}_2$, with $A_1 = \{ c_1, d_1 \}$ and $A_2 = \{ c_2, d_2 \}$, respectively, such that
\begin{center}
$c^{\mathfrak{A}_1} = c_1$, $d^{\mathfrak{A}_1} = d_1$, $P^{\mathfrak{A}_1} = \{ c_1 \}$
\end{center}
and
\begin{center}
$c^{\mathfrak{A}_2} = c_2$, $d^{\mathfrak{A}_2} = d_2$, $P^{\mathfrak{A}_2} = \{ d_2 \}$.
\end{center}
We have $\mathfrak{A}_1 \models (Pc \lor Pd)$ and $\mathfrak{A}_2 \models (Pc \lor Pd)$, but
\begin{center}
not $\mathfrak{A}_1 \times \mathfrak{A}_2 \models (Pc \lor Pd)$.
\end{center}
Thus, using III.4.16, we claim that $(Pc \lor Pd)$ is not logically equivalent to a Horn sentence.\\
\ \\
If in the above we take $P^{\mathfrak{A}_1} = \emptyset$ instead, and take two assignments $\beta_1$ and $\beta_2$ in $\mathfrak{A}_1$ and in $\mathfrak{A}_2$, respectively, such that:
\[
\beta_1 (x) = c_1, \beta_1 (y) = d_1,
\]
and
\[
\beta_2 (x) = \beta_2 (y) = d_2,
\]
then it follows that $(\mathfrak{A}_1, \beta_1) \models (\neg Px \lor Py \lor x \equiv y)$ and $(\mathfrak{A}_2, \beta_2) \models (\neg Px \lor Py \lor x \equiv y)$, but
\begin{center}
not $(\mathfrak{A}_1, \beta_1) \times (\mathfrak{A}_2, \beta_2) \models (\neg Px \lor Py \lor x \equiv y)$.
\end{center}
Hence, using III.4.16, we claim that $(\neg Px \lor Py \lor x \equiv y)$ is not logically equivalent to a Horn formula.
%
\item \textbf{Solution to Exercise 2.9.} Let $a_0, a_1, a_2, \ldots$ be an enumeration of the elements of $G$ in which all $a_i$'s are pairwise distinct; if $G$ is finite, say, contains $(n + 1)$ elements, then let $a_i \neq a_j$ for $0 \leq i < j \leq n$, and let $a_k = a_n$ for $k > n$.\\
\ \\
We take the assignment $\beta$ in $\mathfrak{G}$ with $\beta (v_i) \colonequals a_i$. It is clear that $\{ (\mathfrak{G}, \beta)(t) \mid t \in T^{S_\grp} \} = G$.\\
\ \\
Since $(\mathfrak{T}^{\Phi_\grp}, \beta^{\Phi_\grp})$ and $(\mathfrak{G}, \beta)$ are both models of $\Phi_\grp$ (by the Coincidence Lemma), by 2.1 the map $\pi: T^{\Phi_\grp} \to G$ in which
\begin{center}
$\pi (\bar{t}) \colonequals (\mathfrak{G}, \beta)(t)$ \ \ for $t \in T^{S_\grp}$
\end{center}
is a homomorphism from $\mathfrak{T}^{\Phi_\grp}$ to $\mathfrak{G}$, and even one from $\mathfrak{T}^{\Phi_\grp}$ \emph{onto} $\mathfrak{G}$ by the earlier discussion.\\
\ \\
Next we show every group $\mathfrak{G}$ generated by at most $k$ elements is a homomorphic image of $\mathfrak{T}_k^{\Phi_\grp}$. Let $A = \{ a_0, \ldots, a_{k - 1} \}$ be a generating set of $\mathfrak{G}$ ($A = \emptyset$ in case $k = 0$), namely every element in $G$ is a finite combination of members $a \in A$ and their inverses $a^{-1}$ in the binary operation $\circ$.\\
\ \\
Observe that for $t \in T_k^{S_\grp}$ and for $n \geq k$,
\begin{center}
neither \ $\Phi_\grp \vdash v_n \equiv t$ \ \ nor \ $\Phi_\grp \vdash \neg v_n \equiv t$.
\end{center}
For if $\Phi_\grp \vdash v_p \equiv t$ for some $t \in T_k^{S_\grp}$ and some $p \geq k$, we may pick the model $\mathfrak{I}_0 = (\mathfrak{Z}_0, \beta_0)$ of $\Phi_\grp$, in which $\mathfrak{Z}_0$ is the additive group over $\zah$ and
\[
\beta_0 (v_n) \colonequals \begin{cases}
1 & \mbox{if \(n = p\)} \cr
0 & \mbox{otherwise},
\end{cases}
\]
then we have $\mathfrak{I}_0 (v_p) = 1$, $\mathfrak{I}_0 (t) = 0$ and, as a consequence, $1 = 0$, which is impossible. The case that not $\Phi_\grp \vdash \neg v_n \equiv t$ is similar: we take $\beta_0 (v_n) = 0$ for $n \in \nat$ in the above model $\mathfrak{I}_0$.\\
\ \\
So, let us set
\[
\Phi \colonequals \Phi_\grp \cup \{ v_n \equiv e \mid n \geq k \},
\]
then $\Phi$ is satisfiable and, by 2.5, $\mathfrak{I}^\Phi$ is a free model of $\Phi$. Pick the assignment
\[
\beta (v_n) \colonequals \begin{cases}
a_n & \mbox{if \(n < k\)} \cr
e^\mathfrak{G} & \mbox{otherwise}
\end{cases}
\]
in $\mathfrak{G}$ ($\beta (v_n) = e^\mathfrak{G}$ for $n \in \nat$ in case $k = 0$), then $(\mathfrak{G}, \beta) \models \Phi$ and $\{ (\mathfrak{G}, \beta)(t) \mid t \in T^{S_\grp} \} = G$. As a result, the map $\pi : T^\Phi \to G$
\begin{center}
$\pi (\bar{t}^\Phi ) = (\mathfrak{G}, \beta )(t)$ \ for $t \in T^{S_\grp}$
\end{center}
is a homomorphism from $\mathfrak{T}^\Phi$ \emph{onto} $\mathfrak{G}$, where $\bar{t}^\Phi$ denotes the equivalence class $\{ t^\prime \in T^{S_\grp} \mid \Phi \vdash t \equiv t^\prime \}$ for $t \in T^{S_\grp}$.\\
\ \\
It remains to be shown that there is an isomorphism $\sigma$ from $\mathfrak{T}_k^{\Phi_\grp}$ to $\mathfrak{T}^\Phi$, for the composition $\pi \cdot \sigma$ of $\pi$ and $\sigma$ is a homomorphism from $\mathfrak{T}_k^{\Phi_\grp}$ to $\mathfrak{G}$.\\
\ \\
Let us denote by $\bar{t}^{\Phi_\grp}$ the equivalence class $\{ t^\prime \in T^{S_\grp} \mid \Phi_\grp \vdash t \equiv t^\prime \}$ for $t \in T^{S_\grp}$, and take the map $\sigma : T_k^{\Phi_\grp} \to T^\Phi$
\begin{center}
$\sigma (\bar{t}^{\Phi_\grp}) = \bar{t}^\Phi$ \ for $t \in T_k^{S_\grp}$.
\end{center}
We show $\sigma$ is an isomorphism from $\mathfrak{T}_k^{\Phi_\grp}$ to $\mathfrak{T}^\Phi$ or, more precisely:
\begin{enumerate}[(1)]
\item $\sigma$ is surjective;
%%
\item $\sigma$ is injective;
%%
\item $\sigma (e^{\mathfrak{T}_k^{\Phi_\grp}}) = e^{\mathfrak{T}^\Phi}$;
%%
\item For $t \in T_k^{S_\grp}$, $\sigma ((\bar{t}^{\Phi_\grp})^{-1}) = (\sigma (\bar{t}^{\Phi_\grp}))^{-1}$; (The first occurrence of $^{-1}$ stands for the inverse function in $\mathfrak{T}_k^{\Phi_\grp}$, while the second for that in $\mathfrak{T}^\Phi$.)
%%
\item For $t_1, t_2 \in T_k^{S_\grp}$, $\sigma (\overline{t_1}^{\Phi_\grp} \circ \overline{t_2}^{\Phi_\grp}) = \sigma (\overline{t_1}^{\Phi_\grp}) \circ \sigma (\overline{t_2}^{\Phi_\grp})$. (The first occurrence of $\circ$ stands for the product function in $\mathfrak{T}_k^{\Phi_\grp}$, while the second for that in $\mathfrak{T}^\Phi$.)
\end{enumerate}
Every element in $\mathfrak{T}^\Phi$ must be $\bar{t}^\Phi$ for some $t \in T_k^{S_\grp}$, and $\bar{t}^\Phi = \sigma (\bar{t}^{\Phi_\grp})$, so (1) follows.\\
\ \\
For (2), consider arbitrary $t_1, t_2 \in T_k^{S_\grp}$. If $\Phi \vdash t_1 \equiv t_2$ then $\Phi_\grp \vdash t_1 \equiv t_2$: Assume $\Phi \vdash t_1 \equiv t_2$. Since $\mathfrak{T}_k^{\Phi_\grp} \models \Phi_\grp$, the interpretation $(\mathfrak{T}_k^{\Phi_\grp}, \beta^\prime)$ is a model of $\Phi$, where
\[
\beta^\prime (v_n) \colonequals \begin{cases}
\overline{v_n}^{\Phi_\grp} & \mbox{if \(n < k\)} \cr
\overline{e}^{\Phi_\grp} & \mbox{otherwise}.
\end{cases}
\]
Hence $(\mathfrak{T}_k^{\Phi_\grp}, \beta^\prime ) \models t_1 \equiv t_2$ and, by the Coincidence Lemma, $\mathfrak{I}_k^{\Phi_\grp} = (\mathfrak{T}_k^{\Phi_\grp}, \beta_k^{\Phi_\grp}) \models t_1 \equiv t_2$. Further, by 1.2 (b) we have $\mathfrak{I}^{\Phi_\grp} \models t_1 \equiv t_2$. Finally, $\Phi_\grp \vdash t_1 \equiv t_2$ by 1.1 (b).\\
\ \\
Since $\mathfrak{I}^\Phi \models \Phi$, if $\sigma (\overline{t_1}^{\Phi_\grp}) = \sigma (\overline{t_2}^{\Phi_\grp})$, namely $\overline{t_1}^\Phi = \overline{t_2}^\Phi$, then $\Phi \vdash t_1 \equiv t_2$ by 1.1(b), and $\Phi_\grp \vdash t_1 \equiv t_2$ by the above discussion, and so $\overline{t_1}^{\Phi_\grp} = \overline{t_2}^{\Phi_\grp}$ again by 1.1(b). Thus (2) follows.\\
\ \\
(3) follows immediately from the definition of $\sigma$.\\
\ \\
As for (4), let $t \in T_k^{S_\grp}$. Then
\[
\begin{array}{ll}
\ & \sigma ((\bar{t}^{\Phi_\grp})^{-1}) \cr
= & \sigma (\overline{t^{-1}}^{\Phi_\grp}) \cr
= & \overline{t^{-1}}^\Phi \cr
= & (\bar{t}^\Phi )^{-1} \cr
= & (\sigma (\bar{t}^{\Phi_\grp}))^{-1}.
\end{array}
\]
To show (5), consider arbitrary $t_1, t_2 \in T_k^{S_\grp}$. Then
\[
\begin{array}{ll}
\ & \sigma (\overline{t_1}^{\Phi_\grp} \circ \overline{t_2}^{\Phi_\grp}) \cr
= & \sigma (\overline{t_1 \circ t_2}^{\Phi_\grp}) \cr
= & \overline{t_1 \circ t_2}^\Phi \cr
= & \overline{t_1}^\Phi \circ \overline{t_2}^\Phi \cr
= & \sigma (\overline{t_1}^{\Phi_\grp}) \circ \sigma (\overline{t_2}^{\Phi_\grp}).
\end{array}
\]
%
\item \textbf{Solution to Exercise 2.10.} (a) $\Phi_\grp \cup \Phi$ is satisfied by the trivial group, namely the group with a singleton universe.\\
\ \\
(b) It follows from (a) that $\Phi_\grp \cup \Phi$ is a consistent set of universal Horn sentences. By 2.5, $\mathfrak{I}^{\Phi_\grp \cup \Phi}$ and hence (by the Coincidence Lemma) $\mathfrak{T}^{\Phi_\grp \cup \Phi}$ are models of $\Phi_\grp \cup \Phi$.\\
\ \\
(c) Denote by $U$ the set $\{ \bar{t} \mid \mbox{\begin{math}t \in T^S\end{math} and \begin{math}\Phi_\grp \cup \Phi \vdash t \equiv e\end{math}}\}$. Then clearly $U \subset T^{\Phi_\grp}$. For $U$ to be the universe of a subgroup $\mathfrak{U}$ of $\mathfrak{T}^{\Phi_\grp}$, it suffices to verify:
\begin{enumerate}[(i)]
\item For all $t_1, t_2 \in T^{S_\grp}$, if $\overline{t_1}, \overline{t_2} \in U$ then $\overline{t_1} \circ \overline{t_2} \in U$. (Here $\circ$ stands for the product operation over the group $\mathfrak{T}^{\Phi_\grp}$.)
%%
\item For all $t \in T^{S_\grp}$, if $\bar{t} \in U$ then $\bar{t}^{-1} \in U$. (Here $^{-1}$ stands for the inverse operation over the group $\mathfrak{T}^{\Phi_\grp}$.)
\end{enumerate}
To show (i) holds, let $t_1, t_2 \in T^{S_\grp}$ such that $\Phi_\grp \cup \Phi \vdash t_1 \equiv e$ and $\Phi_\grp \cup \Phi \vdash t_2 \equiv e$, respectively, i.e. $\overline{t_1} \in U$ and $\overline{t_2} \in U$. Then $\Phi_\grp \cup \Phi \vdash t_1 \circ t_2 \equiv e$. So $\overline{t_1} \circ \overline{t_2} = \overline{t_1 \circ t_2} \in U$.\\
\ \\
To show (ii) is true, let $t \in T^{S_\grp}$ such that $\Phi_\grp \cup \Phi \vdash t \equiv e$, i.e. $\bar{t} \in U$. As $\Phi_\grp \vdash t^{-1} \equiv e \circ t^{-1}$ (cf. \textbf{Note to Theorem 1.1}) and $\Phi_\grp \cup \Phi \vdash e \equiv t$, we have $\Phi_\grp \cup \Phi \vdash t^{-1} \equiv e$ and so $\bar{t}^{-1} = \overline{t^{-1}} \in U$.\\
\ \\
To see $\mathfrak{U}$ is a normal subgroup, we pick arbitrary $t, u \in T^{S_\grp}$, where $\bar{u} \in U$, and show $\bar{t} \circ \bar{u} \circ \bar{t}^{-1} \in U$: Since $\bar{u} \in U$, $\Phi_\grp \cup \Phi \vdash u \equiv e$. Hence $\Phi_\grp \cup \Phi \vdash (t \circ u) \circ t^{-1} \equiv e$, i.e. $\bar{t} \circ \bar{u} \circ \bar{t}^{-1} = \overline{(t \circ u) \circ t^{-1}} \in U$.\\
\ \\
The rest is for showing $\mathfrak{T}^{\Phi_\grp \cup \Phi}$ is isomorphic to $\mathfrak{T}^{\Phi_\grp} \slash \mathfrak{U}$. In the following, we write $\bar{t}^{\Phi_\grp}$ or $\bar{t}^{\Phi_\grp \cup \Phi}$ for $t \in T^{S_\grp}$ to explicitly indicate whether the equivalence class of $t$ is taken with respect to $\Phi_\grp$ or to $\Phi_\grp \cup \Phi$.\\
\ \\
$\mathfrak{T}^{\Phi_\grp} \slash \mathfrak{U}$ is an $S_\grp$-structure, with:
\begin{itemize}
\item the universe $T^{\Phi_\grp} \slash U \colonequals \{ U\bar{t}^{\Phi_\grp} \mid t \in T^{S_\grp} \}$
%%
\item the identity $e^{\mathfrak{T}^{\Phi_\grp} \slash \mathfrak{U}} \colonequals U\bar{e}^{\Phi_\grp} = U$
%%
\item the inverse operation $^{-1}$: $\left(U\bar{t}^{\Phi_\grp}\right)^{-1} \colonequals U\overline{t^{-1}}^{\Phi_\grp}$ for $t \in T^{S_\grp}$
%%
\item the product operation $\circ$: $\left(U\overline{t_1}^{\Phi_\grp}\right) \circ \left(U\overline{t_2}^{\Phi_\grp}\right) \colonequals U\overline{t_1 \circ t_2}^{\Phi_\grp}$ for $t_1, t_2 \in T^{S_\grp}$.
\end{itemize}
\ \\
First, for $t \in T^{S_\grp}$,
\[
\bar{t}^{\Phi_\grp \cup \Phi} = \bigcup \left( U \bar{t}^{\Phi_\grp} \right).
\]
We show this as follows: Let $t^\prime \in T^{S_\grp}$. If $t^\prime \in \bar{t}^{\Phi_\grp \cup \Phi}$, then $\Phi_\grp \cup \Phi \vdash t^\prime \equiv t$, or $\Phi_\grp \cup \Phi \vdash t^\prime \circ t^{-1} \equiv e$, therefore $\overline{t^\prime \circ t^{-1}}^{\Phi_\grp} \in U$. Since $\Phi_\grp \vdash t^\prime \equiv (t^\prime \circ t^{-1}) \circ t$, we have $t^\prime \in \overline{t^\prime \circ t^{-1}}^{\Phi_\grp} \circ \bar{t}^{\Phi_\grp}$. Because $\overline{t^\prime \circ t^{-1}}^{\Phi_\grp} \circ \bar{t}^{\Phi_\grp} \in U\bar{t}^{\Phi_\grp}$, $t^\prime \in \bigcup \left( U\bar{t}^{\Phi_\grp} \right)$. So $\bar{t}^{\Phi_\grp \cup \Phi} \subset \bigcup \left( U \bar{t}^{\Phi_\grp} \right)$. Conversely, if $t^\prime \in \bigcup \left( U\bar{t}^{\Phi_\grp} \right)$, i.e. $t^\prime \in \overline{u \circ t}^{\Phi_\grp}$ with some $u \in T^{S_\grp}$ such that $\Phi_\grp \cup \Phi \vdash u \equiv e$, then $\Phi_\grp \vdash t^\prime \equiv u \circ t$ and hence $\Phi_\grp \cup \Phi \vdash t^\prime \equiv t$, namely $t^\prime \in \bar{t}^{\Phi_\grp \cup \Phi}$. So $\bigcup \left( U \bar{t}^{\Phi_\grp} \right) \subset \bar{t}^{\Phi_\grp \cup \Phi}$.\\
\ \\
Second, for $t_1, t_2 \in T^{S_\grp}$,
\begin{center}
$\overline{t_1}^{\Phi_\grp \cup \Phi} = \overline{t_2}^{\Phi_\grp \cup \Phi}$ \ \ \ iff \ \ \ $U\overline{t_1}^{\Phi_\grp} = U\overline{t_2}^{\Phi_\grp}$.
\end{center}
This can be argued: Let $\overline{t_1}^{\Phi_\grp \cup \Phi} = \overline{t_2}^{\Phi_\grp \cup \Phi}$, then $\Phi_\grp \cup \Phi \vdash t_1 \equiv t_2$ and $\Phi_\grp \cup \Phi \vdash t_1 \circ t_2^{-1} \equiv e$. Since every element in $U\overline{t_1}^{\Phi_\grp}$ takes the form $\bar{u}^{\Phi_\grp} \circ \overline{t_1}^{\Phi_\grp}$ for some $u \in T^{S_\grp}$ such that $\Phi_\grp \cup \Phi \vdash u \equiv e$, and
\[
\begin{array}{ll}
\ & \bar{u}^{\Phi_\grp} \circ \overline{t_1}^{\Phi_\grp} \cr
= & \overline{u \circ t_1}^{\Phi_\grp} \cr
= & \overline{u \circ (t_1 \circ e)}^{\Phi_\grp} \cr
= & \overline{u \circ (t_1 \circ (t_2^{-1} \circ t_2))}^{\Phi_\grp} \cr
= & \overline{(u \circ (t_1 \circ t_2^{-1})) \circ t_2}^{\Phi_\grp} \cr
= & \overline{u \circ (t_1 \circ t_2^{-1})}^{\Phi_\grp} \circ \overline{t_2}^{\Phi_\grp} \cr
\in & U\overline{t_2}^{\Phi_\grp}, \mbox{\ \ ($\Phi_\grp \cup \Phi \vdash u \circ (t_1 \circ t_2^{-1}) \equiv e$)}
\end{array}
\]
we have $U\overline{t_1}^{\Phi_\grp} \subset U\overline{t_2}^{\Phi_\grp}$; symmetrically we get $U\overline{t_2}^{\Phi_\grp} \subset U\overline{t_1}^{\Phi_\grp}$. So $U\overline{t_1}^{\Phi_\grp} = U\overline{t_2}^{\Phi_\grp}$. Conversely, if $U\overline{t_1}^{\Phi_\grp} = U\overline{t_2}^{\Phi_\grp}$, then
\[
\overline{t_1}^{\Phi_\grp \cup \Phi} = \bigcup \left( U\overline{t_1}^{\Phi_\grp} \right) = \bigcup \left( U\overline{t_2}^{\Phi_\grp} \right) = \overline{t_2}^{\Phi_\grp \cup \Phi}.
\]
Third, let $\beta$ be an assignment in $\mathfrak{T}^{\Phi_\grp} \slash \mathfrak{U}$ such that $\beta (v_n) \colonequals U \overline{v_n}$ for $n \in \nat$, and denote $\mathfrak{I} = (\mathfrak{T}^{\Phi_\grp} \slash \mathfrak{U}, \beta)$. Then $\mathfrak{I} (t) = U\bar{t}^{\Phi_\grp}$ for all $t \in T^{S_\grp}$, which can easily be proved by induction on $t$. If $\forall x_1 \ldots \forall x_r t \equiv t^\prime$ is an equation derivable from $\Phi_\grp \cup \Phi$, then we have, successively,
\begin{center}
\begin{tabular}{l}
for all $t_1, \ldots, t_r \in T^{S_\grp}$, $\Phi_\grp \cup \Phi \vdash [t \equiv t^\prime] \sbst{t_1 \ldots t_r}{x_1 \ldots x_r}$; \cr
for all $t_1, \ldots, t_r \in T^{S_\grp}$, $\Phi_\grp \cup \Phi \vdash t\sbst{t_1 \ldots t_r}{x_1 \ldots x_r} \equiv t^\prime\sbst{t_1 \ldots t_r}{x_1 \ldots x_r}$; \cr
for all $t_1, \ldots, t_r \in T^{S_\grp}$, $\overline{t\sbst{t_1 \ldots t_r}{x_1 \ldots x_r}}^{\Phi_\grp \cup \Phi} = \overline{t^\prime\sbst{t_1 \ldots t_r}{x_1 \ldots x_r}}^{\Phi_\grp \cup \Phi}$; \cr
for all $t_1, \ldots, t_r \in T^{S_\grp}$, $U\overline{t\sbst{t_1 \ldots t_r}{x_1 \ldots x_r}}^{\Phi_\grp} = U\overline{t^\prime\sbst{t_1 \ldots t_r}{x_1 \ldots x_r}}^{\Phi_\grp}$; \cr
for all $t_1, \ldots, t_r \in T^{S_\grp}$, $\mathfrak{I}\left( t\sbst{t_1 \ldots t_r}{x_1 \ldots x_r} \right) = \mathfrak{I}\left( t^\prime\sbst{t_1 \ldots t_r}{x_1 \ldots x_r} \right)$; \cr
for all $t_1, \ldots, t_r \in T^{S_\grp}$, $\mathfrak{I}\sbst{\mathfrak{I}(t_1) \ldots \mathfrak{I}(t_r)}{x_1 \ldots x_r}(t) = \mathfrak{I}\sbst{\mathfrak{I}(t_1) \ldots \mathfrak{I}(t_r)}{x_1 \ldots x_r}(t^\prime)$; \cr
for all $t_1, \ldots, t_r \in T^{S_\grp}$, $\mathfrak{I}\sbst{\mathfrak{I}(t_1) \ldots \mathfrak{I}(t_r)}{x_1 \ldots x_r} \models t \equiv t^\prime$; \cr
for all $t_1, \ldots, t_r \in T^{S_\grp}$, $\mathfrak{I}\sbst{U\overline{t_1}^{\Phi_\grp} \ldots U\overline{t_r}^{\Phi_\grp}}{x_1 \ldots x_r} \models t \equiv t^\prime$,
\end{tabular}
\end{center}
which entails $\mathfrak{I} \models \forall x_1 \ldots \forall x_r t \equiv t^\prime$. Using the Coincidence Lemma, we get $\mathfrak{T}^{\Phi_\grp} \slash \mathfrak{U} \models \forall x_1 \ldots \forall x_r t \equiv t^\prime$. In particular, $\mathfrak{T}^{\Phi_\grp \cup \Phi} \models \Phi_\grp \cup \Phi$.\\
\ \\
Finally, since $\mathfrak{I}^{\Phi_\grp \cup \Phi}$ is a free model of $\Phi_\grp \cup \Phi$ (by 2.5), the map $\pi$ from $T^{\Phi_\grp \cup \Phi}$ to $T^{\Phi_\grp} \slash U$ with
\begin{center}
$\pi \left( \bar{t}^{\Phi_\grp \cup \Phi} \right) \colonequals U\bar{t}^{\Phi_\grp}$ \ \ for $t \in T^{S_\grp}$
\end{center}
is a homomorphism. Actually, $\pi$ is an isomorphism, for the injectiveness has been proven earlier and the surjectiveness is trivial. So, $\mathfrak{T}^{\Phi_\grp \cup \Phi} \cong \mathfrak{T}^{\Phi_\grp} \slash \mathfrak{U}$.\\
\ \\
\textit{Remark}. $\mathfrak{T}^{\Phi_\grp} \slash \mathfrak{U}$ is the so-called \emph{quotient group} of $\mathfrak{T}^{\Phi_\grp}$ by $\mathfrak{U}$.
\end{enumerate}
%End of Section XI.2------------------------------------------------------------------------------
\
\\
\\
%Section XI.3--------------------------------------------------------------------------------------
{\large \S3. Herbrand Structures}
\begin{enumerate}[1.]
\item \textbf{A Typo in 3.4 (ii).} $T^S$ should be changed to $T_0^S$.
%
\item \textbf{Note to the Proof of 3.7.} The substructure $\mathfrak{B}$ of $\mathfrak{A}^\prime$ is the substructure $\left[ T_0^S \right]^{\mathfrak{A}^\prime}$ generated by $T_0^S$ in $\mathfrak{A}^\prime$ (cf. the discussion after III.5.4). Note that $\mathfrak{B}$ is not necessarily identical to $\mathfrak{T}_0^\Phi$, since for any $n$-ary $R \in S$ and $t_1, \ldots, t_n \in T_0^S$,
\begin{center}
$\mathfrak{B} \models Rt_1 \ldots t_n$ \ \ \ iff \ \ \ $\mathfrak{I} \models Rt_1 \ldots t_n$,
\end{center}
while
\begin{center}
$\mathfrak{T}_0^\Phi \models Rt_1 \ldots t_n$ \ \ \ iff \ \ \ $\Phi \vdash Rt_1 \ldots t_n$.
\end{center}
\ \\
That $\mathfrak{B}$ is a model of $\Phi$ follows from $\mathfrak{A}^\prime \models \Phi$ and III.5.8.
%
\end{enumerate}
%End of Section XI.3-------------------------------------------------------------------------------
\
\\
\\
%Section XI.4--------------------------------------------------------------------------------------
{\large \S4. Propositional Logic}
\begin{enumerate}[1.]
\item \textbf{A Proof of 4.2 Coincidence Lemma of Propositional Logic.} We use induction on propositional formulas.\\
\ \\
$\alpha = p_i$: $\pvar{\alpha} = \{ p_i \}$. If $b$ and $b^\prime$ are two assignments with $b(p_i) = b^\prime(p_i)$, then $\alpha[b] = b(p_i) = b^\prime(p_i) = \alpha[b^\prime]$.\\
\ \\
$\alpha = \neg \beta$: If $b$ and $b^\prime$ are two assignments with $b(p) = b^\prime(p)$ for $p \in \pvar{\alpha} = \pvar{\beta}$, then by induction hypothesis $\beta[b] = \beta[b^\prime]$. So $\alpha[b] = \negfunc (\beta[b]) = \negfunc (\beta[b^\prime]) = \alpha[b^\prime]$.\\
\ \\
$\alpha = (\beta \lor \gamma)$: If $b$ and $b^\prime$ are two assignments with $b(p) = b^\prime(p)$ for $p \in \pvar{\alpha} = \pvar{\beta} \cup \pvar{\gamma}$, then by induction hypothesis $\beta[b] = \beta[b^\prime]$ and $\gamma[b] = \gamma[b^\prime]$. So $\alpha[b] = \dsjfunc (\beta[b], \gamma[b]) = \dsjfunc (\beta[b^\prime], \gamma[b^\prime]) = \alpha[b^\prime]$.\nolinebreak\hfill$\talloblong$
%
\item \textbf{Note to the Paragraph Following 4.2.} Let $\alpha \in \pf_{n + 1}$, then the map from $\{ T, F \}^{n + 1}$ to $\{ T, F \}$ with
\begin{center}
$(b_0, \ldots, b_n) \mapsto \alpha[b_0, \ldots, b_n]$ \ \ for all $b_0, \ldots, b_n \in \{ T, F \}$
\end{center} 
is a function.
%
\item \textbf{Note to 4.3.} We use induction on equality-free and quantifier-free formulas $\varphi$ to show $\rho ( \pi ( \varphi ) )$:\\
\ \\
$\varphi$ is atomic: $\pi ( \varphi ) = \pi_0 ( \varphi )$. So $\rho ( \pi ( \varphi ) ) = \rho ( \pi_0 ( \varphi ) ) = \pi_0^{-1} ( \pi_ ( \varphi ) ) = \varphi$.\\
\ \\
$\varphi = \neg\psi$: $\rho ( \pi ( \neg\psi ) ) = \rho ( \neg \pi ( \psi ) ) = \neg \rho ( \pi ( \psi ) ) = \neg\psi$, by induction hypothesis.\\
\ \\
$\varphi = (\psi \lor \chi)$: $\rho ( \pi ( \psi \lor \chi ) ) = \rho ( \pi ( \psi ) \lor \pi ( \chi ) ) = \rho ( \pi ( \psi ) ) \lor \rho ( \pi ( \chi ) ) = \psi \lor \chi$, by induction hypothesis.\\
\ \\
Then we use induction on propositional formulas $\alpha$ to show $\pi ( \rho ( \alpha ) ) = \alpha$:\\
\ \\
$\alpha = p_i$: $\rho ( p_i ) = \pi_0^{-1} ( p_i )$, an equality-free atomic formula. So $\pi ( \rho ( p_i ) ) = \pi ( \pi_0^{-1} ( p_i ) ) = \pi_0 ( \pi_0^{-1} ( p_i ) ) = p_i$.\\
\ \\
$\alpha = \neg\beta$: $\pi ( \rho ( \neg\beta ) ) = \pi ( \neg\rho ( \beta ) ) = \neg\pi ( \rho ( \beta ) ) = \neg\beta$, by induction hypothesis.\\
\ \\
$\alpha = (\beta \lor \gamma)$: $\pi ( \rho ( \beta \lor \gamma ) ) = \pi ( \rho ( \beta ) \lor \rho ( \gamma ) ) = \pi ( \rho ( \beta ) ) \lor \pi ( \rho ( \gamma ) ) = \beta \lor \gamma$.
%
\item \textbf{Note to 4.4.} From the text we understand that for every set $\Phi$ of equality-free and quantifier-free $S$-formulas, $\pi (\Phi)$ denotes the set
\[
\{ \pi ( \varphi ) \mid \varphi \in \Phi \}
\]
of propositional formulas.\\
\ \\
Analogous to Lemma III.4.4, we have 
\begin{quote}
\emph{for all $\Delta \subset \pf$ and all $\alpha \in \pf$, $\Delta \models \alpha$ \ \ iff \ \ not $\sat \Delta \cup \{ \neg\alpha \}$.}
\end{quote}
The proof is similar to the one of III.4.4 given in text.\\
\ \\
Now assuming $b$ is a model of $\pi (\Phi)$, we prove that for all \emph{equality-free and quantifier-free $S$-formulas $\varphi$,}
\begin{center}
$\INT \models \varphi$ \ \ \ iff \ \ \ $\pi ( \varphi ) [b] = T$,
\end{center}
using induction on $\varphi$, where $\INT$ is the $S$-interpretation constructed in the proof in text. Then in particular, $\INT \models \Phi$ as $b$ is a model of $\pi (\Phi)$.\\
\ \\
$\varphi$ is atomic: It directly follows from $(*)$ in text.\\
\ \\
$\varphi = \neg\psi$:
\begin{center}
\begin{tabular}{lll}
\   & $\INT \models \neg\psi$ & \ \cr
iff & not \ $\INT \models \psi$ & \ \cr
iff & not \ $\pi ( \psi )[b] = T$ & \ (by induction hypothesis) \cr
iff & $\pi ( \psi )[b] = F$ & \ \cr
iff & $\negfunc (\pi ( \psi )[b]) = T$ & \ (by the definition of $\negfunc$) \cr
iff & $\neg\pi ( \psi )[b] = T$ & \ (cf. page 201) \cr
iff & $\pi (\neg\psi)[b] = T$ & \ (by the definition of $\pi$)
\end{tabular}
\end{center}
$\varphi = (\psi \lor \chi)$:
\begin{center}
\begin{tabular}{lll}
\   & $\INT \models (\psi \lor \chi)$ & \ \cr
iff & $\INT \models \psi$ \ \ or \ \ $\INT \models \chi$ & \ \cr
iff & $\pi ( \psi )[b] = T$ \ \ or \ \ $\pi ( \chi )[b] = T$ & \ (by induction hypothesis) \cr
iff & $\dsjfunc ( \pi ( \psi )[b], \pi ( \chi )[b] ) = T$ & \ (by the definition of $\dsjfunc$) \cr
iff & $( \pi ( \psi ) \lor \pi ( \chi ))[b] = T$ & \ (cf. page 201) \cr
iff & $\pi ( \psi\lor\chi )[b] = T$ & \ (by the definition of $\pi$).
\end{tabular}
\end{center}
%
\item \textbf{Note to the Paragraph Immediately below the Proof of 4.4.} An account for the counterexample mentioned in text is that, for any $S$-interpretation, the meaning of $\equiv$ is fixed; in other words, $\equiv$ is \emph{always} interpreted as the equality relation (cf. III.3.2).
%
\item \textbf{Note to the Compactness Theorem for Propositional Logic 4.5.} In the proof of it given in text, we may insert between the last two statements an addtional bi-implicational statement:
\begin{center}
\begin{tabular}{ll}
iff & for every finite subset $\Delta_0$ of $\Delta$, $\sat \pi^{-1}\left( \Delta_0 \right)$\cr
\   & (by the bijectiveness of $\pi^{-1}$)
\end{tabular}
\end{center}
As noted in \textbf{Note to 4.4}, the consequence relation and the satisfaction relation of propositional logic can be interrelated: for all $\Delta \subset \pf$ and all $\alpha \in \pf$, $\Delta \models \alpha$ iff not $\sat \Delta \cup \{ \neg\alpha \}$.\\
\ \\
Then the Compactness Theorem for the consequence relation of propositional logic:
\begin{quote}
\emph{``For all $\Delta \subset \pf$ and all $\alpha \in \pf$, $\Delta \models \alpha$ \ \ iff \ \ there is a finite $\Delta_0 \subset \Delta$ such that $\Delta_0 \models \alpha$.''}
\end{quote}
can be proved as follows: for all $\Delta \subset \pf$ and all $\alpha \in \pf$,
\begin{center}
\begin{tabular}{ll}
\   & $\Delta \models \alpha$ \cr
iff & not $\sat \Delta \cup \{ \neg\alpha \}$ \cr
\   & (by the earlier discussion) \cr
iff & there is a finite $\Delta_0 \subset \Delta$ such that not $\sat \Delta_0 \cup \{ \neg\alpha \}$ \cr
\   & (by 4.5) \cr
iff & there is a finite $\Delta_0 \subset \Delta$ such that $\Delta_0 \models \alpha$ \cr
\   & (by the earlier discussion).
\end{tabular}
\end{center}
%
\item \textbf{Note to the Paragraph Discussing DNF and CNF Following 4.5.} As having been noted in \textbf{Note to Lemma 4.2} in the annotations to Chapter VIII, we speak of disjuctions and conjunctions in a more general sense when it comes to DNF and CNF: A disjunction (or conjunction) may have only one disjunct (or conjunct, respectively).
%
\item \textbf{Note to the Proof of Theorem 4.6.} Here we show that 
\begin{quote}
for all $b_0, \ldots, b_n \in \{ T, F \}$, $h(b_0, \ldots, b_n) = \alpha_\mathrm{C}[b_0, \ldots, b_n]$
\end{quote}
to complete the proof.\\
\ \\
It is useful to consider the equivalent statement (2)$^\prime$ of (2):
\begin{center}
(2)$^\prime$ \hfill $\beta^{b_0, \ldots, b_n}[b_0^\prime, \ldots, b_n^\prime] = F$ \ \ iff \ \ $b_0 = b_0^\prime$ and \ldots and $b_n = b_n^\prime$. \hfill \phantom{(2)$^\prime$}
\end{center}
If $h(b_0, \ldots, b_n) = F$ then $\beta^{b_0, \ldots, b_n}$ is a member of the conjunction $\alpha_\mathrm{C}$. By (2)$^\prime$ we have $\beta^{b_0, \ldots, b_n}[b_0, \ldots, b_n] = F$, so $\alpha_\mathrm{C}[b_0, \ldots, b_n] = F$. Conversely, if $\alpha_\mathrm{C}[b_0, \ldots, b_n] = F$ then by definition there is a member $\alpha_0$ of the conjunction $\alpha_\mathrm{C}$ such that $\alpha_0[b_0, \ldots, b_n] = F$, i.e. there are truth-values $b_0^\prime, \ldots, b_n^\prime \in \{ T, F \}$ with $h(b_0^\prime, \ldots, b_n^\prime) = F$ and $\beta^{b_0^\prime, \ldots, b_n^\prime}[b_0, \ldots, b_n] = F$ ($\alpha_0 = \beta^{b_0^\prime, \ldots, b_n^\prime}$). From (2)$^\prime$ it follows that $b_0^\prime = b_0, \ldots, b_n^\prime = b_n$, therefore $h(b_0, \ldots, b_n) = F$.
%
\item \textbf{Solution to Exercise 4.9.} By 4.6, every truth-function can be defined with $\negfunc$ and $\dsjfunc$; more precisely, it can be defined by a propositional formula, which only involves the two connectives $\neg$ and $\lor$.\\
\ \\
For this exercise, therefore, it suffices to show $\negfunc$ and $\dsjfunc$ can be defined in terms of the truth-function(s) given in cases (a) and (b), respectively:
\begin{enumerate}[(a)]
\item For $b_0, b_1 \in \{ T, F \}$, $\dsjfunc (b_0, b_1) = \negfunc (\cnjfunc (\negfunc (b_0), \negfunc (b_1)))$.\\
\ \\
On the other hand, we could so argue: From the viewpoint of logical equivalence, $(\alpha \lor \beta)$ is equivalent to $\neg (\neg \alpha \land \neg \beta)$ for all $\alpha, \beta \in \pf$; by 4.4(c) together with Exercise III.4.12(b), every propositional formula is logically equivalent to one involving only $\neg$ and $\land$. Hence, we may regard $\lor$ as abbreviations of $\neg$ and $\land$ instead (in contrast with the discussion at the bottom of page 35).
%%
\item For $b_0, b_1 \in \{ T, F \}$, $\negfunc (b_0) = \mathop{\stackrel{.}{|}} (b_0, b_0)$ and $\dsjfunc (b_0, b_1) = \mathop{\stackrel{.}{|}}(\mathop{\stackrel{.}{|}}(b_0, b_0), \mathop{\stackrel{.}{|}}(b_1, b_1))$.\\
\ \\
As in (a), we could argue otherwise: For all $\alpha, \beta \in \pf$, $\neg\alpha$ is logically equivalent to $(\alpha \mathop{|} \alpha)$, and $(\alpha \lor \beta)$ to $((\alpha \mathop{|} \alpha) \mathop{|} (\beta \mathop{|} \beta))$; hence every propositional formula is logically equivalent to one involving only $\mathop{|}$. We may take $\neg$ and $\lor$ as abbreviations of $\mathop{|}$.
\end{enumerate}
%
\item \textbf{Solution to Exercise 4.10.} First we prove the Theorem on the Disjunctive Normal Form for Propositional Logic. To start, we set $S \colonequals \{ P \}$ with unary $P$ and define $\pi_0$ on $A^S = \{ Pv_n \mid n \in \nat \}$ by $\pi_0 ( Pv_n ) \colonequals p_n$ for $n \in \nat$. As in text, we extend $\pi_0$ to a bijection $\pi$ from the set of equality-free and quantifier-free $S$-formulas to the set $\pf$ of propositional formulas.\\
\ \\
Let $\alpha$ be a propositional formula. If $\alpha$ is not satisfiable, then $\alpha [b] = F$ for every assignment $b$; hence $(\alpha \leftrightarrow (p_0 \land \neg p_0))[b] = T$ for every assignment $b$, i.e. $\alpha$ is logically equivalent to $(p_0 \land \neg p_0)$, a propositional formula in disjunctive normal form.\\
\ \\
So let $\alpha$ be satisfiable. By 4.4(a), the equality-free and quantifier-free $S$-formula $\pi^{-1} ( \alpha )$ is also satisfiable. Note that all atomic subformulas in $\pi^{-1} ( \alpha )$ are also equality-free. From the proof of Theorem VIII.4.3 (and from Lemma VIII.4.2), $\pi^{-1} ( \alpha )$ is logically equivalent to an $S$-formula $\varphi$ in disjunctive normal form in which all of the atomic subformulas are equality-free; we have $\pi ( \varphi )$ is in disjunctive normal form. By 4.4(c), $\alpha$ is logically equivalent to $\pi ( \varphi )$.\\
\ \\
The Theorem on the Conjunctive Normal Form for Propositional Logic immdiately follows: For any propositional formula $\alpha$, apply the Theorem on the Conjunctive Normal Form for First-Order Logic (cf. Exercise VIII.4.7) to the equality-free and quantifier-free $S$-formula $\pi^{-1}( \alpha )$ to obtain an equivalent $S$-formula $\varphi$ in conjunctive normal form that is also equality-free and quantifier-free. By 4.4(c), $\alpha$ is logically equivalent to $\pi ( \varphi )$.
%
\item \textbf{Solution to Exercise 4.11.} In this exercise we provide a proof other than that of Theorem 4.5 given in text. More precisely, we shall prove the \emph{if}-part, as the other is trivial.\\
\ \\
Before we start, notice that if $\Delta$ is empty, then every subset of it is also empty; the conclusion is vacuously true, namely there are arbitrarily long good sequences and there is an assignment (actually, \emph{any} assignment) that is a model of $\Delta$ and of all finite subsets of it.\\
\ \\
So let us assume that $\Delta$ is nonempty, in addition to the premise that every finite subset of it is satisfiable.\\
\ \\
Suppose, for the sake of a contradiction, that for some $n \in \nat$ all sequences of truth-values $(b_0, \ldots, b_n)$ of length $(n + 1)$ are not good, i.e. for every sequence $(b_0, \ldots, b_n)$ of length $(n + 1)$ there is a finite subset $\Delta_0^\prime \subset \Delta$ that does not have any model $b$ with $b(p_i) = b_i$ for $i \leq n$. We take the union $\Delta_0$ of such $\Delta_0^\prime$ for each $(b_0, \ldots, b_n)$; thus, for all sequences $(b_0, \ldots, b_n)$, $\Delta_0$ does not have any model $b$ with $b(p_i) = b_i$ for $i \leq n$. So $\Delta_0$ is not satisfiable: If $b$ were a model of $\Delta_0$, there would be a sequence $(b_0, \ldots, b_n)$ such that $b_i = b(p_i)$ for $i \leq n$.\\
\ \\
On the other hand, however, $\Delta_0$ is finite, as there are finitely many sequences of truth-values of length $(n + 1)$ ($2^{n + 1}$ in total). By the premise $\Delta_0$ is satisfiable, a contradiction.\\
\ \\
Therefore, there are arbitrarily long good sequences. And since for every good sequence all of its prefixes (for example, $(T, F, T)$ is a prefix of $(T, F, T, T)$) are also good, there is an infinite chain of good sequences
\[
(b_0), (b_0, b_1), (b_0, b_1, b_2), \ldots
\]
If we take an assignment $b$ in which $b(p_i) = b_i$ for $i \in \nat$, then in particular $b$ is a model of $\{ \alpha \}$ (by the Coincidence Lemma) for every $\alpha \in \Delta$, hence a model of $\Delta$.
%
\item \textbf{Solution to Exercise 4.12.} We define some notions analogous to those of first-order logic:
\begin{enumerate}[(i)]
\item A sequent (of propositional formulas) is a finite nonempty sequence of propositional formulas.
%%
\item A derivation (over propositional formulas) is a finite nonempty sequence of sequents in which the first sequent is obtained by applying the rule $\assm$ and inductively all other sequents are obtained by applying the rule $\assm$ or other rules in $\mathfrak{S}_\mathrm{a}$ to previous sequents.
%%
\item Let $\Gamma$ be a (possibly empty) sequence of propositional formulas and $\alpha$ a propositional formula. The sequent $\Gamma \alpha$ is derivable in $\mathfrak{S}_\mathrm{a}$ (written: $\vdash_\mathrm{a} \Gamma \alpha$) iff there is a derivation of which the last sequent is $\Gamma \alpha$. If this is the case, we say $\alpha$ is derivable from $\Gamma$ (written: $\Gamma \vdash_\mathrm{a} \alpha$).
%%
\item For $\Delta \subset \pf$ and $\alpha \in \pf$, $\alpha$ is derivable from $\Delta$ in $\mathfrak{S}_\mathrm{a}$ (written: $\Delta \vdash_\mathrm{a} \alpha$) iff there are formulas $\alpha_1, \ldots, \alpha_n$ in $\Delta$ such that $\vdash_\mathrm{a} \alpha_1 \ldots \alpha_n \ \alpha$.
%%
\item For $\Delta \subset \pf$, $\Delta$ is consistent (written: $\con_\mathrm{a} \Delta$) iff there is no $\alpha \in \pf$ such that $\Delta \vdash_a \alpha$ and $\Delta \vdash_a \neg\alpha$. $\Delta$ is inconsistent (written: $\inc_\mathrm{a} \Delta$) iff it is not consistent.
%%
\item For $\Delta \subset \pf$, $\Delta$ is negation complete iff $\Delta \vdash_\mathrm{a} \alpha$ or $\Delta \vdash_\mathrm{a} \neg\alpha$ for every $\alpha \in \pf$.
\end{enumerate}
\ \\
Then, we pick a symbol set $S \colonequals \{ P \}$ with a unary relation symbol $P$, and define the (bijective) map $\pi$ from the set of equality-free and quantifier-free $S$-formulas to the set of propositional formulas by
\[
\begin{array}{lll}
\pi ( Pv_n ) & \colonequals & p_n, \ n \in \nat \cr
\pi ( \neg\varphi ) & \colonequals & \neg \pi ( \varphi ) \cr
\pi ( ( \varphi\lor\psi ) ) & \colonequals & \pi ( \varphi ) \lor \pi ( \psi ).
\end{array}
\]
In a natural way, a sequent of equality-free and quantifier-free $S$-formulas corresponds to a sequent of propositional formulas, and vice versa; a derivation in $\mathfrak{S}_\mathrm{a}$ that consists of only sequents of equality-free and quantifier-free $S$-formulas thus corresponds to a derivation in $\mathfrak{S}_\mathrm{a}$ over propositional formulas, and vice versa.\\
\ \\
The correctness part of the Adequacy Theorem is easy: For $\Delta \subset \pf$ and $\alpha \in \pf$, we have
\begin{center}
\begin{tabular}{lll}
\    & $\Delta \vdash_\mathrm{a} \alpha$ \cr
iff  & $\pi^{-1} ( \Delta ) \vdash_\mathrm{a} \pi^{-1} ( \alpha )$ & (by the above discussion)\cr
then & $\pi^{-1} ( \Delta ) \vdash \pi^{-1} ( \alpha )$ & (a derivation in $\mathfrak{S}_\mathrm{a}$ is also one in $\mathfrak{S}$) \cr
then & $\pi^{-1} ( \Delta ) \models \pi^{-1} ( \alpha )$ & (by the Correctness Theorem for \cr
\    & \ & \phantom{(} first-order logic) \cr
iff  & $\Delta \models \alpha$ & (by 4.4(b)). \cr
\end{tabular}
\end{center}
\ \\
To prove the completeness part, let us first note, analogous to first-order logic, the following:
\begin{enumerate}[(1)]
\item For all $\Delta \subset \pf$, $\con_\mathrm{a} \Delta$ if and only if $\con_\mathrm{a} \Delta_0$ for all subsets $\Delta_0$ of $\Delta$.
%%
\item For all $\Delta \subset \pf$ and all $\alpha \in \pf$ the following holds:
\begin{enumerate}[(a)]
\item $\Delta \vdash_\mathrm{a} \alpha$ \ \ \ iff \ \ \ $\inc_\mathrm{a} \Delta \cup \{\neg\alpha\}$.
%%%
\item $\Delta \vdash_\mathrm{a} \neg\alpha$ \ \ \ iff \ \ \ $\inc_\mathrm{a} \Delta \cup \{ \alpha \}$.
\end{enumerate}
(cf. Lemma IV.7.6)
%%
\item For $n \in \nat$, let $\Delta_n$ be a consistent set of propositional formulas such that
\[
\Delta_0 \subset \Delta_1 \subset \Delta_2 \subset \ldots
\]
Denote $\Delta \colonequals \bigcup_{n \in \nat} \Delta_n$. Then $\con_\mathrm{a} \Delta$. (cf. Lemma IV.7.7.)
%%
\item Let $\Delta \subset \pf$ be consistent and negation complete, and let $\alpha, \beta \in \pf$. Then the following holds:
\begin{enumerate}[(a)]
\item $\Delta \vdash_\mathrm{a} \neg\alpha$ iff not $\Delta \vdash_\mathrm{a} \alpha$.
%%%
\item $\Delta \vdash_\mathrm{a} (\alpha \lor \beta)$ iff $\Delta \vdash_\mathrm{a} \alpha$ or $\Delta \vdash_\mathrm{a} \beta$.
\end{enumerate}
(cf. V.1.9.)
\end{enumerate}
\ \\
Next, we show that for $\Delta \subset \pf$, if $\Delta$ is consistent and negation complete, then it has a model (cf. Henkin's Theorem V.1.10): Let $b$ be an assignment with $b(p_n) = T$ iff $\Delta \vdash_\mathrm{a} p_n$ for $n \in \nat$. We prove that for $\alpha \in \pf$,
\begin{center}
$\alpha [b] = T$ \ \ \ iff \ \ \ $\Delta \vdash_\mathrm{a} \alpha$,
\end{center}
by induction on $\alpha$:
\begin{enumerate}[(a)]
\item $\alpha = p_n$: By definition of $b$.
%%
\item $\alpha = \neg\beta$: $\neg\beta [b] = T$
\begin{quote}
\begin{tabular}{lll}
iff & $\beta [b] = F$ & \ \cr
iff & not $\Delta \vdash_\mathrm{a} \beta$ & (by induction hypothesis) \cr
iff & $\Delta \vdash_\mathrm{a} \neg\beta$ & (by (a) of (4)).
\end{tabular}
\end{quote}
%%
\item $\alpha = (\beta\lor\gamma)$: $(\beta\lor\gamma) [b] = T$
\begin{quote}
\begin{tabular}{lll}
iff & $\beta [b] = T$ or $\gamma [b] = T$ & \ \cr
iff & $\Delta \vdash_\mathrm{a} \beta$ or $\Delta \vdash_\mathrm{a} \gamma$ & (by induction hypothesis) \cr
iff & $\Delta \vdash_\mathrm{a} (\beta \lor \gamma)$ & (by (b) of (4)).
\end{tabular}
\end{quote}
\end{enumerate}
In particular, $\alpha [b] = T$ for $\alpha \in \Delta$, namely $b$ is a model of $\Delta$.\\
\ \\
Then, we show that for $\Delta \subset \pf$, if $\Delta$ is consistent then there is a set $\Delta^\prime \supset \Delta$ of propositional formulas that is consistent and negation complete (cf. V.2.2): Since $\pf$ is countable, we let $\alpha_0, \alpha_1, \alpha_2, \ldots$ be an enumeration of $\pf$. We define sets $\Delta_n^\prime$ inductively as follows:
\[
\Delta_0^\prime \colonequals \Delta,
\]
and for $n \in \nat$,
\[
\Delta_{n + 1}^\prime \colonequals \begin{cases}
\Delta_n^\prime \cup \{ \alpha_n \} & \mbox{if \(\con_\mathrm{a} \Delta_n^\prime \cup \{ \alpha_n \}\)} \cr
\Delta_n^\prime & \mbox{otherwise}.
\end{cases}
\]
An easy induction on $n$ shows that $\con_\mathrm{a} \Delta_n^\prime$ for $n \in \nat$. By (3) we have $\Delta^\prime \colonequals \bigcup_{n \in \nat} \Delta_n^\prime$ is consistent. On the other hand, if $\alpha = \alpha_n \in \pf$ for some $n \in \nat$ such that not $\Delta^\prime \vdash_\mathrm{a} \neg\alpha$, then $\con_\mathrm{a} \Delta^\prime \cup \{ \alpha \}$ by (b) of (2). By (1) we have $\con_\mathrm{a} \Delta_n^\prime \cup \{ \alpha \}$. So $\Delta_{n + 1}^\prime = \Delta_n^\prime \cup \{ \alpha \}$ and hence $\alpha \in \Delta_{n + 1}^\prime$, therefore $\Delta^\prime \vdash_\mathrm{a} \alpha$. It turns out that $\Delta^\prime$ is negation complete.
\\
\ \\
Since a model of a consistent and negation complete set $\Delta^\prime \supset \Delta$ is also one of $\Delta$, we have: For $\Delta \subset \pf$, if $\Delta$ is consistent then it is satisfiable.\\
\ \\
Finally, we are done if we can show that, for $\Delta \subset \pf$, the following $(*)$ implies $(**)$:\\
\ \\
\begin{tabular}{ll}
$(*)$ & If $\Delta$ is consistent then it is satisfiable. \cr
$(**)$ & For $\alpha \in \pf$, if $\Delta \models \alpha$ then $\Delta \vdash_\mathrm{a} \alpha$.
\end{tabular}\\
\ \\
Assume that $(*)$ holds. Let $\alpha \in \pf$ such that $\Delta \models \alpha$, then not $\sat \Delta \cup \{ \neg\alpha \}$ (cf. \textbf{Note to 4.4}). By $(*)$ we have that $\inc_\mathrm{a} \Delta \cup \{ \neg\alpha \}$, and furthermore that $\Delta \vdash_\mathrm{a} \alpha$ by (a) of (2).\\
\ \\
\textit{Remark.} The above $(**)$ also implies $(*)$: Assume that $(**)$ holds. If not $\sat \Delta$, then there is $\alpha \in \pf$ such that $\Delta \models \alpha$ and $\Delta \models \neg\alpha$. By $(**)$ we have both $\Delta \vdash_\mathrm{a} \alpha$ and $\Delta \vdash_\mathrm{a} \neg\alpha$, i.e. $\inc_\mathrm{a} \Delta$.
\end{enumerate}
%End of Section XI.4-------------------------------------------------------------------------------
\
\\
\\
%Section XI.5--------------------------------------------------------------------------------------
{\large \S5. Propositional Resolution}
\begin{enumerate}[1.]
\item \textbf{Note to Lemma 5.2(b).} It can be stated otherwise:
\begin{enumerate}[(1)]
\item For every propositional variable $q$: If $b^\Delta (q) = T$, then for every assignment $b$ which is a model of $\Delta$ we have $b (q) = T$; or
%%
\item For every propositional variable $q$: If $b^\Delta (q) = T$, then $\Delta \models q$.
\end{enumerate}
Since $b^\Delta (q) = T$ follows from $\Delta \models q$ as well for every propositional variable $q$, by (2) we obtain:\\
\ \\
$(+)$ \ \ \ For every propositional variable $q$: $b^\Delta (q) = T$ \ \ iff \ \ $\Delta \models q$. \ \ \ \phantom{$(+)$}\\
\ \\
\textit{Remark.} Alternatively, for all $n \in \nat$, $\Delta \models p_n$ iff $p_n$ is derivable in the calculus with the rules (T1) and (T2) given in text (i.e. $b^\Delta (p_n) = T$). This way, we also get the above result.
%
\item \textbf{Another Way to Prove (c) Follows from (b) in Theorem 5.3.} Let us assume that (c) does not hold, i.e. $b^{\Delta^+}$ is not a model of $\Delta$. Then there must be some $\alpha = (\neg q_0 \lor \ldots \lor \neg q_n) \in \Delta^-$ such that $\alpha [b^{\Delta^+}] = F$ or, $b^{\Delta^+}(q_i) = T$ for all $0 \leq i \leq n$, since $b^{\Delta^+}$ is a model of $\Delta^+$ (by 5.2(a)). By $(+)$ in \textbf{Note to Lemma 5.2(b)} we have $\Delta^+ \models q_i$ for $0 \leq i \leq n$, hence $\Delta^+ \models (q_0 \land \ldots \land q_n)$ and therefore not $\sat \Delta^+ \cup \{ \neg q_0 \lor \ldots \lor \neg q_n \}$.
%
\item \textbf{Note to the Remarks to Definition 5.4.} From the Compactness Theorem, it immediately follows that, for any set $\mathfrak{K}$ of clauses, $\mathfrak{K}$ is satisfiable iff each of its finite subset is satisfiable.\\
\ \\
Also, a more appropriate notation than $\bigwedge_{K \in \mathfrak{K}} \bigvee_{\lambda \in K} \lambda$ is $\{ \bigvee_{\lambda \in K} \lambda \mid K \in \mathfrak{K} \}$.
%
\item \textbf{Note to Definition 5.7.} Let $\mathfrak{K}$ be a set of clauses, and $\mathfrak{K}^\prime \subset \res{\mathfrak{K}} \backslash \mathfrak{K}$ be finite. If $K \in \res{\mathfrak{K}}$, then $K \in \res{\mathfrak{K} \cup \mathfrak{K}^\prime}$. So, in the case that $\mathfrak{K}^\prime = \{ K_1, \ldots, K_n \}$, repeated applications of 5.6 yield:
\begin{center}
\begin{tabular}{lll}
\   & $\mathfrak{K}$ is satisfiable & \ \cr
iff & $\mathfrak{K} \cup \{ K_1 \}$ is satisfiable & (apply to $\mathfrak{K}$) \cr
\multicolumn{2}{c}{$\vdots$} & \ \cr
iff & $\mathfrak{K} \cup \{ K_1, \ldots, K_n \}$ is satisfiable & (apply to $\mathfrak{K} \cup \{ K_1, \ldots, K_{n - 1} \}$). 
\end{tabular}
\end{center}
From the discussion in \textbf{Note to the Remarks to Definition 5.4}, we get (cf. the proof of 5.8)
\begin{center}
$\mathfrak{K}$ is satisfiable \ \ \ iff \ \ \ $\res{\mathfrak{K}}$ is satisfiable.
\end{center}
%
\item \textbf{Note to the Resolution Theorem 5.8.} There is a typo in the proof of it in text: In the third last line on page 213, ``$K_T \subset \mathfrak{R}_k$'' should be replaced by ``$K_T \in \mathfrak{R}_k$''.\\
\ \\
On the other hand, we state the following as a direct consequence of this theorem: Let $\mathfrak{K}$ be a set of clauses. By 5.7, we have
\begin{enumerate}[(1)]
\item $\res{\resi{\infty}{\mathfrak{K}}} = \resi{\infty}{\mathfrak{K}}$.
%%
\item $\resi{\infty}{\resi{\infty}{\mathfrak{K}}} = \resi{\infty}{\mathfrak{K}}$.
\end{enumerate}
Thus, by 5.8, $\resi{\infty}{\mathfrak{K}}$ is satisfiable iff $\emptyset \not\in \resi{\infty}{\mathfrak{K}}$.
%
\item \textbf{Note to the Paragraph under FIGURE XI.1.} Let $\mathfrak{K}$ be a set of clauses in which only literals from $P \colonequals \{ p_0, \ldots, p_{n - 1} \} \cup \{ \neg p_0, \ldots, \neg p_{n - 1} \}$ occur. Then for all $i \in \nat$, every resolvent formed from clauses of $\resi{i}{\mathfrak{K}}$ is a subset of $P$. Thus, It is clear that there are at most $2^{2n}$ resolvents. So $\resi{2^{2n}}{\mathfrak{K}}$ contains all resolvents obtained in finitely many steps together with all clauses from $\mathfrak{K}$, hence $\resi{2^{2n}}{\mathfrak{K}} = \resi{\infty}{\mathfrak{K}}$.
%
\item \textbf{Solution to Exercise 5.12.} Let us note that
\[
\begin{array}{lll}
\resi{1}{\mathfrak{K}} & = & \{ \{ p_0, p_1 \}, \{ p_0, p_2 \}, \{ p_0, p_1, p_2 \}, \{ p_0, \neg p_1, p_2 \}, \{ p_0, p_1, \neg p_2 \}, \cr
\ & \ & \phantom{ \{ } \{ p_0, p_1, \neg p_1, p_2 \}, \{ p_0, p_1, p_2, \neg p_2 \} \} \cup \{ \{ \neg p_i \} \mid i \geq 1 \}
\end{array}
\]
and
\[
\begin{array}{lll}
\resi{2}{\mathfrak{K}} & = & \{ \{ p_0 \}, \{ p_0, p_1 \}, \{ p_0, p_2 \}, \{ p_0, \neg p_1 \}, \{ p_0, \neg p_2 \}, \cr
\ & = & \phantom{ \{ } \{ p_0, p_1, p_2 \}, \{ p_0, \neg p_1, p_2 \}, \{ p_0, p_1, \neg p_2 \}, \{ p_0, \neg p_1, \neg p_2 \}, \cr
\ & \ & \phantom{ \} } \{ p_0, p_1, \neg p_1 \}, \{ p_0, p_2, \neg p_2 \}, \cr
\ & \ & \phantom{ \{ } \{ p_0, p_1, \neg p_1, p_2 \}, \{ p_0, p_1, p_2, \neg p_2 \}, \cr
\ & \ & \phantom{ \} } \{ p_0, p_1, \neg p_1, \neg p_2 \}, \{ p_0, \neg p_1, p_2, \neg p_2 \}, \cr
\ & \ & \phantom{ \{ } \{ p_0, p_1, \neg p_1, p_2, \neg p_2 \} \} \cup \{ \{ p_i \} \mid i \geq 1 \}.
\end{array}
\]
\begin{enumerate}[(a)]
\item Since any resolvent formed in finitely many steps from clauses of $\mathfrak{K}$ is a subset of $\{ p_0, p_1, \neg p_1, p_2, \neg p_2 \}$ and it must contain $p_0$ as an element, we have $\resi{\infty}{\mathfrak{K}} \subset \resi{2}{\mathfrak{K}}$. But obviously $\resi{2}{\mathfrak{K}} \subset \resi{\infty}{\mathfrak{K}}$, therefore $\resi{\infty}{\mathfrak{K}} = \resi{2}{\mathfrak{K}}$.
%%
\item We directly have
\[
\begin{array}{lll}
\resi{2}{\mathfrak{K}} \backslash \resi{1}{\mathfrak{K}} & = & \{ \{ p_0 \}, \{ p_0, \neg p_1 \}, \{ p_0, \neg p_2 \}, \cr
\ & \ & \phantom{ \{ } \{ p_0, \neg p_1, \neg p_2 \}, \{ p_0, p_1, \neg p_1 \}, \{ p_0, p_2, \neg p_2 \}, \cr
\ & \ & \phantom{ \} } \{ p_0, p_1, \neg p_1, \neg p_2 \}, \{ p_0, \neg p_1, p_2, \neg p_2 \}, \cr
\ & \ & \phantom{ \{ } \{ p_0, p_1, \neg p_1, p_2, \neg p_2 \} \}
\end{array}
\]
and
\[
\begin{array}{lll}
\resi{1}{\mathfrak{K}} \backslash \mathfrak{K} & = & \{ \{ p_0, p_1 \}, \{ p_0, p_2 \}, \cr
\ & \ & \phantom{ \{ } \{ p_0, p_1, \neg p_2 \}, \{ p_0, \neg p_1, p_2 \}, \cr
\ & \ & \phantom{ \} } \{ p_0, p_1, \neg p_1, p_2 \}, \{ p_0, p_1, p_2, \neg p_2 \} \},
\end{array}
\]
both of which are finite.
%%
\item Since $\emptyset \not\in \resi{\infty}{\mathfrak{K}}$ (cf. (a)), by the Resolution Theorem $\mathfrak{K}$ is satisfiable.
\end{enumerate}
\end{enumerate}
%End of Section XI.5-------------------------------------------------------------------------------
\
\\
\\
%Section XI.6-------------------------------------------------------------------------------------
{\large \S6. First-Order Resolution (without Unification)}
\begin{enumerate}[1.]
\item \textbf{Verifying ``for any term $t \in T^S_0$: $\Phi_2 \vdash Pc_ac_bt$ iff $t$ represents a path from $a$ to $b$ in $(G, R^G)$'', as Suggested in Text.} Let us take the $S$-structure $\mathfrak{G}_2 \colonequals (T^S_0, R^{G_2}, P^{G_2}, F, (c_a)_{a \in G})$, where
\[
\begin{array}{lll}
F(t_1, t_2) & \colonequals & f t_1 t_2, \cr
R^{G_2} & \colonequals & \{ (c_a, c_b) \mid \mbox{there is an edge from $a$ to $b$ in $(G, R^G)$} \}, \cr
P^{G_2} & \colonequals & \{ (c_a, c_b, t) \mid \mbox{$t$ represents a path from $a$ to $b$ in $(G, R^G)$} \}.
\end{array}
\]
Then obviously the direction from left to right holds since $\mathfrak{G}_2$ is a model of $\Phi_2$.\\
\ \\
As for the direction from right to left, let us first note that:\\
($\ast$) \ \ \begin{minipage}{10cm}
If $R^{\prime G} \subset T^S_0 \times T^S_0$ and $P^{\prime G} \subset T^S_0 \times T^S_0 \times T^S_0$, and if $(T^S_0, R^{\prime G}, P^{\prime G}, F, (c_a)_{a \in G}) \models \Phi_2$, then $R^{G_2} \subset R^{\prime G}$ and $P^{G_2} \subset P^{\prime G}$.\end{minipage}\\
In fact, the definition of $\Phi_0$ immediately tells us that $R^{G_2} \subset R^{\prime G}$. By definition of $P^{G_2}$, we have to show for $n \in \nat$ and $a_0, \ldots, a_n \in G$ with $R^G a_i a_{i + 1}$ for $i < n$, there is a term $t \in T^S_0$ such that $P^{\prime G} c_{a_0} c_{a_n} t$. This follows from the axioms in $\Phi_2$ by induction on $n$.\\
\ \\
Thus, ($\ast$) together with 3.8 entails that $\mathfrak{G}_2 = \mathfrak{T}_0^{\Phi_2}$.\\
\ \\
So, for any term $t \in T^S_0$, if $t$ represents a path from $a$ to $b$ in $(G, R^G)$ then by definition $\mathfrak{G}_2 \models P c_a c_b t$; by 3.9 we have $\Phi_2 \vdash P c_a c_b t$ (note that $P c_a c_b t$ is a \emph{sentence}).
%
\item \textbf{Note to the Discussion before Example 6.5.} For discussions later on (cf. the proofs of 7.14, of 7.17, and of 7.18 in text), we extend Definition 6.2 to accommodate sets of first-order clauses: A set $\mathfrak{K}$ of first-order clauses is called \emph{propositionally satisfiable} \ \ :iff \ \ $\{ \pi(K) \mid K \in \mathfrak{K} \}$ is satisfiable.
%
\item \textbf{Note to Example 6.6.} By 6.4, the satisfiability of the sentence
\[
\forall x \forall y ((Rxy \lor Qx) \land \neg Rxgx \land \neg Qy)
\]
is equivalent to the propositional satisfiability of
\[
\{ (Rt_1 t_2 \lor Qt_1) \land \neg Rt_1 gt_1 \land \neg Qt_2 \mid t_1, t_2 \in T^S_0 \},
\]
which in turn is equivalent to that of
\[
\{ Rt_1 t_2 \lor Qt_1 \mid t_1, t_2 \in T^S_0 \} \cup \{ \neg Rtgt \mid t \in T^S_0 \} \cup \{ \neg Qt \mid t \in T^S_0 \},
\]
which is equivalent to the satisfiability of the set of clauses
\[
\{ \{ Rt_1 t_2, Qt_1 \} \mid t_1, t_2 \in T^S_0 \} \cup \{ \{ \neg Rtgt \} \mid t \in T^S_0 \} \cup \{ \{ \neg Qt \} \mid t \in T^S_0 \}.
\]
To obtain the resolution tree in FIGURE XI.6, we should take the clauses
\begin{quote}
\begin{tabular}{ll}
$\{ Rt_1 t_2, Qt_1 \}$ & with $t_1 = ggc$ and $t_2 = gggc$, \cr
$\{ \neg Rtgt \}$ & with $t = ggc$, and \cr
$\{ \neg Qt \}$ & with $t = ggc$
\end{tabular}
\end{quote}
as the leaves.
%
\item \textbf{Solution to Exercise 6.11.} The statement
\[
c^{A/_{\scriptstyle E}} \colonequals \overline{c^A}
\]
is missing from the definition of $\mathfrak{A}/_{\displaystyle E}$.
\begin{enumerate}[(a)]
\item Let us first note that
\begin{enumerate}[(i)]
\item For any $a, b \in A$: $E^A a b$ \ iff \ $\overline{a} = \overline{b}$.
%%%
\item For any $t \in T^S$: $(\mathfrak{A}/_{\displaystyle E}, \beta/_{\displaystyle E})(t) = \overline{((\mathfrak{A}, E^A), \beta)(t)}$.
\end{enumerate}
Since $E^A$ is an equivalence relation, (i) immediately follows. We can prove (ii) by induction on $t$:\\
If $t = x$, then
\[
\begin{array}{ll}
\ & (\mathfrak{A}/_{\displaystyle E}, \beta/_{\displaystyle E})(x) \cr
= & \beta/_{\displaystyle E}(x) \cr
= & \overline{\beta(x)} \cr
= & \overline{((\mathfrak{A}, E^A), \beta)(x)}.
\end{array}
\]
If $t = c$, then
\[
\begin{array}{ll}
\ & (\mathfrak{A}/_{\displaystyle E}, \beta/_{\displaystyle E})(c) \cr
= & c^{A/_{\scriptstyle E}} \cr
= & \overline{c^A} \cr
= & \overline{((\mathfrak{A}, E^A), \beta)(c)}.
\end{array}
\]
If $t = ft_1 \ldots t_n$, then
\[
\begin{array}{ll}
\ & (\mathfrak{A}/_{\displaystyle E}, \beta/_{\displaystyle E})(ft_1 \ldots t_n) \cr
= & f^{A/_{\scriptstyle E}} ((\mathfrak{A}/_{\displaystyle E}, \beta/_{\displaystyle E})(t_1), \ldots, (\mathfrak{A}/_{\displaystyle E}, \beta/_{\displaystyle E})(t_n)) \cr
= & f^{A/_{\scriptstyle E}} (\overline{((\mathfrak{A}, E^A), \beta)(t_1)}, \ldots, \overline{((\mathfrak{A}, E^A), \beta)(t_n)}) \cr
\ & \multicolumn{1}{r}{\mbox{(by induction hypothesis)}} \cr
= & \overline{f^A (((\mathfrak{A}, E^A), \beta)(t_1), \ldots, ((\mathfrak{A}, E^A), \beta)(t_n))} \cr
= & \overline{((\mathfrak{A}, E^A), \beta)(ft_1 \ldots t_n)}.
\end{array}
\]
\ \\
We are done after we prove
\begin{quote}
for every $\varphi \in L^S$ and \emph{for any assignment $\beta$ in $(\mathfrak{A}, E^A)$}: $((\mathfrak{A}, E^A), \beta) \models \varphi^\ast$ \ iff \ $(\mathfrak{A}/_{\displaystyle E}, \beta/_{\displaystyle E}) \models \varphi$
\end{quote}
by induction on $\varphi$:\\
If $\varphi = t_1 \equiv t_2$, then $\varphi^\ast = Et_1 t_2$ and\\
\begin{tabular}{ll}
\   & $((\mathfrak{A}, E^A), \beta) \models Et_1 t_2$ \cr
iff & $E^A$ holds for $((\mathfrak{A}, E^A), \beta)(t_1), ((\mathfrak{A}, E^A), \beta)(t_2)$ \cr
iff & $\overline{((\mathfrak{A}, E^A), \beta)(t_1)} = \overline{((\mathfrak{A}, E^A), \beta)(t_2)}$ \ \ (by (i)) \cr
iff & $(\mathfrak{A}/_{\displaystyle E}, \beta/_{\displaystyle E})(t_1) = (\mathfrak{A}/_{\displaystyle E}, \beta/_{\displaystyle E})(t_2)$ \ \ (by (ii)) \cr
iff & $(\mathfrak{A}/_{\displaystyle E}, \beta/_{\displaystyle E}) \models t_1 \equiv t_2$.
\end{tabular}\\
\ \\
If $\varphi = Rt_1 \ldots t_n$, then $\varphi^\ast = Rt_1 \ldots t_n$ and\\
\begin{tabular}{ll}
\   & $((\mathfrak{A}, E^A), \beta) \models Rt_1 \ldots t_n$ \cr
iff & $R^A$ holds for $((\mathfrak{A}, E^A), \beta)(t_1), \ldots, ((\mathfrak{A}, E^A), \beta)(t_n)$ \cr
iff & $R^{A/_{\scriptstyle E}}$ holds for $\overline{((\mathfrak{A}, E^A), \beta)(t_1)}, \ldots, \overline{((\mathfrak{A}, E^A), \beta)(t_n)}$ \cr
iff & $R^{A/_{\scriptstyle E}}$ holds for $(\mathfrak{A}/_{\displaystyle E}, \beta/_{\displaystyle E})(t_1), \ldots, (\mathfrak{A}/_{\displaystyle E}, \beta/_{\displaystyle E})(t_n)$ \ (by (ii))\cr
iff & $(\mathfrak{A}/_{\displaystyle E}, \beta/_{\displaystyle E}) \models Rt_1 \ldots t_n$.
\end{tabular}\\
\ \\
If $\varphi = \neg\psi$, then $\varphi^\ast = \neg\psi^\ast$ and\\
\begin{tabular}{ll}
\   & $((\mathfrak{A}, E^A), \beta) \models \neg\psi^\ast$ \cr
iff & not $((\mathfrak{A}, E^A), \beta) \models \psi^\ast$ \cr
iff & not $(\mathfrak{A}/_{\displaystyle E}, \beta/_{\displaystyle E}) \models \psi$ \ (by induction hypothesis) \cr
iff & $(\mathfrak{A}/_{\displaystyle E}, \beta/_{\displaystyle E}) \models \neg\psi$.
\end{tabular}\\
\ \\
If $\varphi = \psi \lor \chi$, then $\varphi^\ast = \psi^\ast \lor \chi^\ast$ and\\
\begin{tabular}{ll}
\   & $((\mathfrak{A}, E^A), \beta) \models \psi^\ast \lor \chi^\ast$ \cr
iff & $((\mathfrak{A}, E^A), \beta) \models \psi^\ast$ or $((\mathfrak{A}, E^A), \beta) \models \chi^\ast$ \cr
iff & $(\mathfrak{A}/_{\displaystyle E}, \beta/_{\displaystyle E}) \models \psi$ or $(\mathfrak{A}/_{\displaystyle E}, \beta/_{\displaystyle E}) \models \chi$ \ (by induction hypothesis) \cr
iff & $(\mathfrak{A}/_{\displaystyle E}, \beta/_{\displaystyle E}) \models \psi \lor \chi$.
\end{tabular}\\
\ \\
If $\varphi = \exists x \psi$, then $\varphi^\ast = \exists x \psi^\ast$ and\\
\begin{tabular}{ll}
\   & $((\mathfrak{A}, E^A), \beta) \models \exists x \psi^\ast$ \cr
iff & there is an $a \in A$ such that $((\mathfrak{A}, E^A), \beta\sbst{a}{x}) \models \psi^\ast$ \cr
iff & there is an $a \in A$ such that $(\mathfrak{A}/_{\displaystyle E}, (\beta\sbst{a}{x})/_{\displaystyle E}) \models \psi$ \cr
\   & (by induction hypothesis; note that the statement we are proving \cr
\   & \phantom{(}applies to any assignment in $(\mathfrak{A}, E^A)$) \cr
iff & there is an $a \in A$ such that $(\mathfrak{A}/_{\displaystyle E}, (\beta/_{\displaystyle E})\sbst{\overline{a}}{x}) \models \psi$ \cr
\   & (since $(\beta\sbst{a}{x})/_{\displaystyle E} = (\beta/_{\displaystyle E})\sbst{\overline{a}}{x}$) \cr
iff & there is an $e \in A/_{\displaystyle E}$ such that $(\mathfrak{A}/_{\displaystyle E}, (\beta/_{\displaystyle E})\sbst{e}{x}) \models \psi$ \cr
iff & $(\mathfrak{A}/_{\displaystyle E}, \beta/_{\displaystyle E}) \models \exists x \psi$.
\end{tabular}
%%
\item Throughout $\Psi^\ast$ denotes the set $\{ \psi^\ast \mid \psi \in \Psi \}$ for $\Psi \subset L^S$.\\
\ \\
Before we start, note that there is a typo in the statement of this part of exercise: $\Phi \cup \Psi_E$ should be replaced by $\Phi^\ast \cup \Psi_E$. Actually, the assertion
\begin{quote}
for $\Phi \cup \{ \psi \} \subset L^S$: $\Phi \vdash \psi$ \ iff \ $\Phi \cup \Psi_E \vdash \psi^\ast$
\end{quote}
is \emph{not} true: Let $S = \emptyset$, $\psi = \neg v_0 \equiv v_1$ and $\Phi = \{ \psi \}$. It is clear that $\Phi \vdash \psi$. However, it does not hold that $\Phi \cup \Psi_E \vdash \psi^\ast$: Choose an $\{ E \}$-interpretation $(\mathfrak{A}, \beta)$ with $A = \{ a_0, a_1 \}$, $E^A = A \times A$, $\beta(v_0) = a_0$ and $\beta(v_1) = a_1$. We have $(\mathfrak{A}, \beta) \models \Phi \cup \Psi_E$ but not $(\mathfrak{A}, \beta) \models \psi^\ast$.\\
\ \\
Now let us complete this part of exercise. In the following, let $\Psi_\equiv \subset L^S$ such that $\Psi_\equiv^\ast = \Psi_E$. Note that all of the sentences in $\Psi_\equiv$ are valid. By III.4.4, it suffices to prove\\
($+$) \ \ \begin{minipage}{10cm}
for $\Phi \cup \{ \varphi \} \subset L^S$: $\Phi \cup \{ \neg\varphi \}$ is satisfiable \ iff \ $\Phi^\ast \cup \Psi_E \cup \{ \neg\varphi^\ast \}$ is satisfiable.
\end{minipage}\\
\ \\
The direction from right to left in ($+$) is shown below:\\
\begin{tabular}{ll}
\   & $\Phi^\ast \cup \Psi_E \cup \{ \neg\psi^\ast \}$ is satisfiable \cr
iff & there is an $S \cup \{ E \}$-interpretation $((\mathfrak{A}, E^A), \beta)$ such that \cr
\   & $((\mathfrak{A}, E^A), \beta) \models \Phi^\ast \cup \Psi_E \cup \{ \neg\psi^\ast \}$ \cr
iff & there is an $S \cup \{ E \}$-interpretation $((\mathfrak{A}, E^A), \beta)$ such that \cr
\   & $(\mathfrak{A}/_{\displaystyle E}, \beta/_{\displaystyle E}) \models \Phi \cup \{ \neg\psi \}$. \cr
\   & (by (a) and the fact that sentences in $\Psi_\equiv$ are valid)\cr
\end{tabular}\\
The last statement entails that there is a model of $\Phi \cup \{ \neg\psi \}$, namely $\Phi \cup \{ \neg\psi \}$ is satisfiable.\\
\ \\
For a (nonempty) universe $A$, denote by $=^A$ the equality relation over it. Then we prove the direction from left to right in ($+$):\\
\begin{tabular}{ll}
\   & $\Phi \cup \{ \neg\psi \}$ is satisfiable \cr
iff & $\Phi \cup \Psi_\equiv \cup \{ \neg\psi\}$ is satisfiable \cr
\   & (since sentences in $\Psi_\equiv$ are valid) \cr
iff & there is an $S$-interpretation $(\mathfrak{A}, \beta)$ that is a model of \cr
\   & $\Phi \cup \Psi_\equiv \cup \{ \neg\psi \}$ \cr
iff & there is an $S \cup \{ E \}$-interpretation $((\mathfrak{A}, =^A), \beta)$ that is a model of \cr
\   & $\Phi^\ast \cup \Psi_E \cup \{ \neg\psi^\ast \}$ \cr
\   & ($E^A$ is $=^A$; this equivalence of the two statements follows \cr
\   & \phantom{(}in a way analogous to the Substitution Lemma III.8.3, \cr
\   & \phantom{(}in the \emph{second-order} sense).
\end{tabular}
The last statement tells us that $\Phi^\ast \cup \Psi_E \cup \{ \neg\psi^\ast \}$ is satisfiable.
\end{enumerate}
\ \\
\textit{Remarks.} We shall continue to use the notations $\varphi^\ast$, $\Phi^\ast$, $\Psi_\equiv$ and $=^A$ (cf. the proof of part (b)) below.
\begin{enumerate}[(1)]
\item From the final part of the proof of part (b) we immediately obtain
\begin{center}
For $\varphi \in L^S$: if $\varphi$ is satisfiable, then $\varphi^\ast$ is satisfiable.
\end{center}
(For an $S$-interpretation $\mathfrak{I} = (\mathfrak{A}, \beta)$ that is a model of $\varphi$, let $E^A$ be $=^A$. Then the $S \cup \{ E \}$-interpretation $\mathfrak{I}^\prime = ((\mathfrak{A}, E^A), \beta)$ is a model of $\varphi^\ast$.)\\
\ \\
The converse, however, does \emph{not} hold in general: $\neg E x x$ is satisfiable while $\neg x \equiv x$ is not.
%%
\item Let a symbol set $S$ be given, and let $\mathfrak{S}^\circ$ be the sequent calculus associated to $S$ that consists of all rules in $\mathfrak{S}_S$ other than $\eq$ and $\sub$. Furthermore, we denote by $\mathfrak{S}^\ast$ the sequent calculus associated to $S \cup \{ E \}$ obtained by adding to $\mathfrak{S}^\circ$ the rules
\[
\begin{array}{ll}
\displaystyle \frac{\ }{\ \ E t t \ \ }, & t \in T^S
\end{array}
\]
and
\[
\begin{array}{ll}
\displaystyle \frac{\ \Gamma \ \phantom{E t t^\prime} \ \varphi\sbst{t}{x} \ }{\ \Gamma \ E t t^\prime \ \varphi\sbst{t^\prime}{x} \ }, & t, t^\prime \in T^S
\end{array}
\]
which correspond to $\eq$ and $\sub$, respectively. And then the derivability relations $\vdash^\circ$ and $\vdash^\ast$ associated to $\mathfrak{S}^\circ$ and to $\mathfrak{S}^\ast$, respectively, are defined accordingly.\\
\ \\
If we regard $E t_1 t_2$ as rewriting $t_1 \equiv t_2$, then it is trivial that
\begin{center}
for $\Phi \cup \{ \varphi \} \subset L^S$: $\Phi \vdash \varphi$ \ iff \ $\Phi^\ast \vdash^\ast \varphi^\ast$.
\end{center}
%%
\item (INCOMPLETE) On the other hand, the set $\Psi_\equiv$ is indeed the set of axioms of equality \emph{using the equality symbol $\equiv$}, i.e. it is the counterpart of $\Psi_E$. The use of sentences in $\Psi_\equiv$ as (additional) antecedents of sequents can eliminate applications of the rules $\eq$ and $\sub$ in derivations while the set of all derivable sequents remains unchanged. More precisely,
\begin{center}
for $\Phi \cup \{ \varphi \} \subset L^S$: $\Phi \vdash \varphi$ \ iff \ $\Phi \cup \Psi_\equiv \vdash^\circ \varphi$.
\end{center}
To assert this, we need to show, according to finiteness of derivations, that for sequents $\Gamma \, \varphi \subset L^S$:
\begin{quote}
$\Gamma \vdash \varphi$ \ iff \ there is a sequent $\Gamma^\prime \subset \Psi_\equiv$ such that $\Gamma\Gamma^\prime \vdash^\circ \varphi$.
\end{quote}
It is done below:\\
\ \\
The direction from right to left is easy: Let $\Gamma\Gamma^\prime \vdash^\circ \varphi$. Because $\mathfrak{S}^\circ$ consists of less rules than $\mathfrak{S}_S$, it is clear that $\Gamma\Gamma^\prime \vdash \varphi$. Furthermore, all sentences in $\Psi_\equiv$ and hence all those in $\Gamma^\prime$ are valid (in other words, they are derivable from $\emptyset$ in $\mathfrak{S}_S$), so $\Gamma \vdash \varphi$.\\
\ \\
As for the direction from left to right, we need to show that in all derivations over $\mathfrak{S}_S$ the applications of $\eq$ and $\sub$ (if any) can be replaced by those of other rules in $\mathfrak{S}_S$ together with the use of sentences in $\Psi_\equiv$ as (additional) antecedents. To be precise, we need to show:
\begin{enumerate}[1$^\circ$]
\item For $t \in T^S$, there is a $\Gamma^\prime \subset \Psi_\equiv$ such that $\Gamma^\prime \, t \equiv t$ is derivable in $\mathfrak{S}^\circ$.
%%%
\item For $\Gamma \, \varphi \subset L^S$ and $t, t^\prime \in T^S$, if we are given the sequent $\Gamma \, \varphi\sbst{t}{x}$, then we can provide a derivation of $\Gamma\Gamma^\prime \, t \equiv t^\prime \, \varphi\sbst{t^\prime}{x}$ in $\mathfrak{S}^\circ$, where $\Gamma^\prime \subset \Psi_\equiv$.
\end{enumerate}
\ \\
First, a derivation for case 1$^\circ$ is
\begin{center}
\begin{tabular}{llll}
1. & $\forall x \, x \equiv x$ & $\forall x \, x \equiv x$ & $\assm$ \cr
2. & $\forall x \, x \equiv x$ & $t \equiv t$ & IV.5.5(a1) applied to 1.
\end{tabular}
\end{center}
Now in the following we denote $\psi_1 \colonequals \forall x \forall y (x \equiv y \rightarrow y \equiv x)$ and $\psi_2 \colonequals \forall x \forall y \forall z (x \equiv y \land y \equiv z \rightarrow x \equiv z)$.\\
\ \\
Then, we deal with case 2$^\circ$, which requires much work: If we can show that\\
($\ast$) \hfill \begin{minipage}{10cm}for $t, t^\prime, t_0 \in T^S$, there is a sequent $\Gamma_0 \subset \Psi_\equiv$ such that\\$\Gamma_0 \ t \equiv t^\prime \ t_0\sbst{t}{x} \equiv t_0\sbst{t^\prime}{x}$ is derivable in $\mathfrak{S}^\circ$,\end{minipage}
then we can use it to prove case 2$^\circ$ by induction on $\varphi$, as follows:\\
$\varphi = t_1 \equiv t_2$: Given the sequent $\Gamma \, t_1\sbst{t}{x} \equiv t_2\sbst{t}{x}$ as a premise, we denote
\[
\begin{array}{lll}
\chi_1 & \colonequals & \left(t_1\sbst{t}{x} \equiv t_2\sbst{t}{x} \land t_2\sbst{t}{x} \equiv t_2\sbst{t^\prime}{x} \rightarrow t_1\sbst{t}{x} \equiv t_2\sbst{t^\prime}{x}\right), \cr
\chi_2 & \colonequals & \left(t_1\sbst{t^\prime}{x} \equiv t_1\sbst{t}{x} \land t_1\sbst{t}{x} \equiv t_2\sbst{t^\prime}{x} \rightarrow t_1\sbst{t^\prime}{x} \equiv t_2\sbst{t^\prime}{x}\right),
\end{array}
\]
and provide the derivation\\
\begin{tabular}{llll}
1. & $\Gamma$ & $t_1\sbst{t}{x} \equiv t_2\sbst{t}{x}$ & premise \cr
2. & $\Gamma \ t \equiv t^\prime$ & $t_1\sbst{t}{x} \equiv t_1\sbst{t^\prime}{x}$ & apply ($\ast$) \cr
3. & $\Gamma \ t \equiv t^\prime$ & $t_2\sbst{t}{x} \equiv t_2\sbst{t^\prime}{x}$ & apply ($\ast$) \cr
4. & $\psi_1$ & $\psi_1$ & $\assm$ \cr
5. & $\psi_1$ & $\left(t_1\sbst{t}{x} \equiv t_1\sbst{t^\prime}{x} \rightarrow t_1\sbst{t^\prime}{x} \equiv t_1\sbst{t}{x}\right)$ & IV.5.5(a1) applied to 4. \cr
6. & $\Gamma \ \psi_1 \ \psi_2 \ t \equiv t^\prime$ & $\left(t_1\sbst{t}{x} \equiv t_1\sbst{t^\prime}{x} \rightarrow t_1\sbst{t^\prime}{x} \equiv t_1\sbst{t}{x}\right)$ & $\ant$ applied to 5. \cr
7. & $\psi_2$ & $\psi_2$ & $\assm$ \cr
8. & $\psi_2$ & $\chi_1$ & IV.5.5(a1) applied to 7. \cr
9. & $\psi_2$ & $\chi_2$ & IV.5.5(a1) applied to 7. \cr
10. & $\Gamma \ \psi_1 \ \psi_2 \ t \equiv t^\prime$ & $t_1\sbst{t}{x} \equiv t_2\sbst{t}{x}$ & $\ant$ applied to 1. \cr
11. & $\Gamma \ \psi_1 \ \psi_2 \ t \equiv t^\prime$ & $t_2\sbst{t}{x} \equiv t_2\sbst{t^\prime}{x}$ & $\ant$ applied to 3. \cr
12. & $\Gamma \ \psi_1 \ \psi_2 \ t \equiv t^\prime$ & $\chi_1$ & $\ant$ applied to 8. \cr
13. & $\Gamma \ \psi_1 \ \psi_2 \ t \equiv t^\prime$ & $\left(t_1\sbst{t}{x} \equiv t_2\sbst{t}{x} \land t_2\sbst{t}{x} \equiv t_2\sbst{t^\prime}{x}\right)$ & IV.3.6(b) applied to 10. and 11. \cr
14. & $\Gamma \ \psi_1 \ \psi_2 \ t \equiv t^\prime$ & $t_1\sbst{t}{x} \equiv t_2\sbst{t^\prime}{x}$ & IV.3.5 applied to 12. and 13. \cr
15. & $\Gamma \ \psi_1 \ \psi_2 \ t \equiv t^\prime$ & $t_1\sbst{t}{x} \equiv t_1\sbst{t^\prime}{x}$ & $\ant$ applied to 2. \cr
16. & $\Gamma \ \psi_1 \ \psi_2 \ t \equiv t^\prime$ & $t_1\sbst{t^\prime}{x} \equiv t_1\sbst{t}{x}$ & IV.3.5 applied to 6. and 15. \cr
17. & $\Gamma \ \psi_1 \ \psi_2 \ t \equiv t^\prime$ & $\chi_2$ & $\ant$ applied to 9. \cr
18. & $\Gamma \ \psi_1 \ \psi_2 \ t \equiv t^\prime$ & $\left(t_1\sbst{t^\prime}{x} \equiv t_1\sbst{t}{x} \land t_1\sbst{t}{x} \equiv t_2\sbst{t^\prime}{x}\right)$ & IV.3.6(b) applied to 16. and 14. \cr
19. & $\Gamma \ \psi_1 \ \psi_2 \ t \equiv t^\prime$ & $t_1\sbst{t^\prime}{x} \equiv t_2\sbst{t^\prime}{x}$ & IV.3.5 applied to 17. and 18.
\end{tabular}\\
$\varphi = Rt_1 \ldots t_n$: Given the sequent $\Gamma \, Rt_1\sbst{t}{x} \ldots t_n\sbst{t}{x}$ as a premise, we denote
\[
\begin{array}{lll}
\psi_3 & \colonequals & \forall x_1 \ldots \forall x_n \forall y_1 \ldots \forall y_n \left(\bigwedge\limits^n_{i = 1} x_i \equiv y_i \land Rx_1 \ldots x_n \rightarrow Ry_1 \ldots y_n\right), \cr
\chi_3 & \colonequals & \bigwedge\limits^n_{i = 1} t_i\sbst{t}{x} \equiv t_i\sbst{t^\prime}{x}, \cr
\chi_4 & \colonequals & Rt_1\sbst{t}{x} \ldots t_n\sbst{t}{x}, \cr
\chi_5 & \colonequals & Rt_1\sbst{t^\prime}{x} \ldots t_n\sbst{t^\prime}{x},
\end{array}
\]
and provide the derivation\\
\begin{tabular}{llll}
1. & $\Gamma$ & $Rt_1\sbst{t}{x} \ldots t_n\sbst{t}{x}$ & premise \cr
2. & $\Gamma \ t \equiv t^\prime$ & $t_1\sbst{t}{x} \equiv t_1\sbst{t^\prime}{x}$ & apply ($\ast$) \cr
\multicolumn{4}{c}{$\vdots$} \cr
$(n + 1)$. & $\Gamma \ t \equiv t^\prime$ & $t_n\sbst{t}{x} \equiv t_n\sbst{t^\prime}{x}$ & apply ($\ast$) \cr
$(n + 2)$. & $\Gamma \ t \equiv t^\prime$ & $\bigwedge\limits^n_{i = 1} t_i\sbst{t}{x} \equiv t_i\sbst{t^\prime}{x}$ & successively apply IV.3.6(b) to 2. through $(n + 1)$. \cr
$(n + 3)$. & $\Gamma \ t \equiv t^\prime$ & $Rt_1\sbst{t}{x} \ldots t_n\sbst{t}{x}$ & $\ant$ applied to 1. \cr
$(n + 4)$. & $\Gamma \ t \equiv t^\prime$ & $(\chi_3 \land \chi_4)$ & IV.3.6(b) applied to $(n + 2)$. and $(n + 3)$. \cr
$(n + 5)$. & $\psi_3$ & $\psi_3$ & $\assm$ \cr
$(n + 6)$. & $\psi_3$ & $(\chi_3 \land \chi_4 \rightarrow \chi_5)$ & IV.5.5(a1) applied to $(n + 5)$. \cr
$(n + 7)$. & $\Gamma \ \psi_3 \ t \equiv t^\prime$ & $(\chi_3 \land \chi_4 \rightarrow \chi_5)$ & $\ant$ applied to $(n + 6)$. \cr
$(n + 8)$. & $\Gamma \ \psi_3 \ t \equiv t^\prime$ & $(\chi_3 \land \chi_4)$ & $\ant$ applied to $(n + 4)$. \cr
$(n + 9)$. & $\Gamma \ \psi_3 \ t \equiv t^\prime$ & $Rt_1\sbst{t^\prime}{x} \ldots t_n\sbst{t^\prime}{x}$ & IV.3.5 applied to $(n + 7)$. and $(n + 8)$.
\end{tabular}
$\varphi = \neg\psi$: By induction hypothesis, if $\Gamma^+ \, \psi \subset L^S$ and if $\Gamma^+ \, \psi\sbst{t^\prime}{x}$ is given, then $\Gamma^+ \, t^\prime \equiv t \, \psi\sbst{t}{x}$ is derivable in $\mathfrak{S}^\circ$. Given $\Gamma \, \neg\psi\sbst{t}{x}$, we provide the derivation\\
\begin{tabular}{llll}
1. & $\Gamma$ & $\neg\psi\sbst{t}{x}$ & premise \cr
2. & $\psi\sbst{t^\prime}{x}$ & $\psi\sbst{t^\prime}{x}$ & $\assm$ \cr
3. & $\psi\sbst{t^\prime}{x} \ t^\prime \equiv t$ & $\psi\sbst{t}{x}$ & by the above discussion \cr
4. & $\Gamma \ \psi_1 \ t \equiv t^\prime \ t^\prime \equiv t \ \psi\sbst{t^\prime}{x}$ & $\psi\sbst{t}{x}$ & $\ant$ applied to 3. \cr
5. & $\Gamma \ \psi_1 \ t \equiv t^\prime \ t^\prime \equiv t \ \neg\psi\sbst{t}{x}$ & $\neg\psi\sbst{t^\prime}{x}$ & IV.3.3(a) applied to 4. \cr
6. & $\Gamma \ \psi_1 \ t \equiv t^\prime \ t^\prime \equiv t$ & $\neg\psi\sbst{t}{x}$ & $\ant$ applied to 1. \cr
7. & $\Gamma \ \psi_1 \ t \equiv t^\prime \ t^\prime \equiv t$ & $\neg\psi\sbst{t^\prime}{x}$ & IV.3.2 applied to 6. and 5. \cr
8. & $\psi_1$ & $\psi_1$ & $\assm$ \cr
9. & $\psi_1$ & $(t \equiv t^\prime \rightarrow t^\prime \equiv t)$ & IV.5.5(a1) applied to 8. \cr
10. & $\Gamma \ \psi_1 \ t \equiv t^\prime$ & $(t \equiv t^\prime \rightarrow t^\prime \equiv t)$ & $\ant$ applied to 9. \cr
11. & $\Gamma \ \psi_1 \ t \equiv t^\prime$ & $t \equiv t^\prime$ & $\assm$ \cr
12. & $\Gamma \ \psi_1 \ t \equiv t^\prime$ & $t^\prime \equiv t$ & IV.3.5 applied 10. and 11. \cr
13. & $\Gamma \ \psi_1 \ t \equiv t^\prime$ & $\neg\psi\sbst{t^\prime}{x}$ & IV.3.2 applied to 12. and 7.
\end{tabular}\\
$\varphi = (\psi \lor \chi)$: By induction hypothesis, if $\Gamma^+ \, (\psi \lor \chi) \subset L^S$ and if $\Gamma^+ \, \psi\sbst{t}{x}$ (or $\Gamma^+ \, \chi\sbst{t}{x}$) is given, then $\Gamma^+ \, t \equiv t^\prime \, \psi\sbst{t^\prime}{x}$ (or $\Gamma^+ \, t \equiv t^\prime \, \chi\sbst{t^\prime}{x}$, respectively) is derivable in $\mathfrak{S}^\circ$. Given $\Gamma \, \left(\psi\sbst{t}{x} \lor \chi\sbst{t}{x}\right)$, we provide the derivation\\
\begin{tabular}{llll}
1. & $\Gamma$ & $\left(\psi\sbst{t}{x} \lor \chi\sbst{t}{x}\right)$ & premise \cr
2. & $\psi\sbst{t}{x}$ & $\psi\sbst{t}{x}$ & $\assm$ \cr
3. & $\chi\sbst{t}{x}$ & $\chi\sbst{t}{x}$ & $\assm$ \cr
4. & $\psi\sbst{t}{x} \ t \equiv t^\prime$ & $\psi\sbst{t^\prime}{x}$ & by the above discussion \cr
5. & $\chi\sbst{t}{x} \ t \equiv t^\prime$ & $\chi\sbst{t^\prime}{x}$ & by the above discussion \cr
6. & $\psi\sbst{t}{x} \ t \equiv t^\prime$ & $\left(\psi\sbst{t^\prime}{x} \lor \chi\sbst{t^\prime}{x}\right)$ & $\ors$ applied to 4. \cr
7. & $\chi\sbst{t}{x} \ t \equiv t^\prime$ & $\left(\psi\sbst{t^\prime}{x} \lor \chi\sbst{t^\prime}{x}\right)$ & $\ors$ applied to 5. \cr
8. & $\Gamma \ t \equiv t^\prime \ \psi\sbst{t}{x}$ & $\left(\psi\sbst{t^\prime}{x} \lor \chi\sbst{t^\prime}{x}\right)$ & $\ant$ applied to 6. \cr
9. & $\Gamma \ t \equiv t^\prime \ \chi\sbst{t}{x}$ & $\left(\psi\sbst{t^\prime}{x} \lor \chi\sbst{t^\prime}{x}\right)$ & $\ant$ applied to 7. \cr
10. & $\Gamma \ t \equiv t^\prime$ & $\left(\psi\sbst{t}{x} \lor \chi\sbst{t}{x}\right)$ & $\ant$ applied to 1. \cr
11. & $\Gamma \ t \equiv t^\prime \ \left(\psi\sbst{t}{x} \lor \chi\sbst{t}{x}\right)$ & $\left(\psi\sbst{t^\prime}{x} \lor \chi\sbst{t^\prime}{x}\right)$ & $\ora$ applied to 8. and 9. \cr
12. & $\Gamma \ t \equiv t^\prime$ & $\left(\psi\sbst{t^\prime}{x} \lor \chi\sbst{t^\prime}{x}\right)$ & IV.3.2 applied to 10. and 11.
\end{tabular}\\
$\varphi = \exists u \psi$ and $x \in \free{\varphi}$: Based on logical equivalence, we may assume w.~l.~o.~g.\ that $\varphi\sbst{t}{x} = \exists v \psi\sbst{v \, t}{ux}$ and $\varphi\sbst{t^\prime}{x} = \exists v \psi\sbst{vt^\prime}{ux}$, where $v = u$ if $t = t^\prime = x$ or $u \not\in \var{t} \cup \var{t^\prime}$, otherwise $v$ is the first variable in the list $v_0, v_1, v_2, \ldots$ not occurring in $\psi, t, t^\prime$ (cf. III.8.2(e)). Either way $v$ does not occur in $t, t^\prime$.\\
\ \\
By induction hypothesis, if $\Gamma^+ \, \psi \subset L^S$ and if $\Gamma^+ \, \psi\sbst{v \, t}{ux}$ is given, then $\Gamma^+ \, t \equiv t^\prime \, \psi\sbst{vt^\prime}{ux}$ is derivable in $\mathfrak{S}^\circ$. Given $\Gamma \, \exists v \psi\sbst{v \, t}{ux}$, we provide the derivation\\
\begin{tabular}{llll}
1. & $\Gamma$ & $\exists v \psi\sbst{v \, t}{ux}$ & premise \cr
2. & $\psi\sbst{v \, t}{ux}$ & $\psi\sbst{v \, t}{ux}$ & $\assm$ \cr
3. & $\psi\sbst{v \, t}{ux} \ t \equiv t^\prime$ & $\psi\sbst{vt^\prime}{ux}$ & by the above discussion \cr
4. & $\psi\sbst{v \, t}{ux} \ t \equiv t^\prime$ & $\exists v \psi\sbst{vt^\prime}{ux}$ & IV.5.1(a) applied to 3. \cr
5. & $\exists v \psi\sbst{v \, t}{ux} \ t \equiv t^\prime$ & $\exists v \psi\sbst{vt^\prime}{ux}$ & IV.5.1(b) applied to 4. \cr
6. & $\Gamma \ t \equiv t^\prime$ & $\exists v \psi\sbst{v \, t}{ux}$ & $\ant$ applied to 1. \cr
7. & $\Gamma \ t \equiv t^\prime \ \exists v \psi\sbst{v \, t}{ux}$ & $\exists v \psi\sbst{vt^\prime}{ux}$ & $\ant$ applied to 5. \cr
8. & $\Gamma \ t \equiv t^\prime$ & $\exists v \psi\sbst{vt^\prime}{ux}$ & IV.3.2 applied to 6. and 7.
\end{tabular}\\
$\varphi = \exists u \psi$ and $x \not\in \free{\varphi}$: We have $(\exists u \psi)\sbst{t}{x} = (\exists u \psi)\sbst{t^\prime}{x} = \exists u \psi$. Given $\Gamma \, \exists u \psi$, we provide the derivation\\
\begin{tabular}{llll}
1. & $\Gamma$ & $\exists u \psi$ & premise \cr
2. & $\Gamma \ t \equiv t^\prime$ & $\exists u \psi$ & $\ant$ applied to 1.
\end{tabular}\\
\ \\
\ \\
Now ($\ast$) remains to be shown. We do this by induction on $t_0$ below:\\
$t_0 = c$: Then $c\sbst{t}{x} = c\sbst{t^\prime}{x} = c$. We provide the derivation\\
\begin{tabular}{lllll}
1. & $\forall x \, x \equiv x$ & \ & $\forall x \, x \equiv x$ & $\assm$ \cr
2. & $\forall x \, x \equiv x$ & \ & $c \equiv c$ & IV.5.5(a1) applied to 1. \cr
3. & $\forall x \, x \equiv x$ & $t \equiv t^\prime$ & $c \equiv c$ & $\ant$ applied to 2.
\end{tabular}\\
\ \\
$t_0 = y \neq x$: Similar.\\
\ \\
$t_0 = x$: Then $x\sbst{t}{x} = t$ and $x\sbst{t^\prime}{x} = t^\prime$. We provide the derivation\\
\begin{tabular}{llll}
1. & $t \equiv t^\prime$ & $t \equiv t^\prime$ & $\assm$
\end{tabular}\\
\ \\
$t_0 = ft_1 \ldots t_n$: By induction hypothesis, there are $\Gamma_1, \ldots, \Gamma_n \subset \Psi_\equiv$ such that $\Gamma_1 \, t \equiv t^\prime \, t_1\sbst{t}{x} \equiv t_1\sbst{t^\prime}{x}, \ldots, \Gamma_n \, t \equiv t^\prime \, t_n\sbst{t}{x} \equiv t_n\sbst{t^\prime}{x}$ are derivable in $\mathfrak{S}^\circ$. Take $\Gamma \colonequals \Gamma_1 \cup \ldots \cup \Gamma_n$ then, using $\ant$, $\Gamma \, t \equiv t^\prime \, t_1\sbst{t}{x} \equiv t_1\sbst{t^\prime}{x}, \ldots, \Gamma \, t \equiv t^\prime \, t_n\sbst{t}{x} \equiv t_n\sbst{t^\prime}{x}$ are also derivable in $\mathfrak{S}^\circ$. We denote
\[
\begin{array}{lll}
\psi_4 & \colonequals & \forall x_1 \ldots \forall x_n \forall y_1 \ldots \forall y_n \left(\bigwedge\limits^n_{i = 1} x_i \equiv y_i \rightarrow fx_1 \ldots x_n \equiv fy_1 \ldots y_n \right), \cr
\chi_6 & \colonequals & \bigwedge\limits^n_{i = 1} t_i\sbst{t}{x} \equiv t_i\sbst{t^\prime}{x}, \cr
\chi_7 & \colonequals & ft_1\sbst{t}{x} \ldots t_n\sbst{t}{x} \equiv ft_1\sbst{t^\prime}{x} \ldots t_n\sbst{t^\prime}{x}
\end{array}
\]
and provide the derivation\\
\begin{tabular}{llll}
1. & $\Gamma \ \phantom{\psi_4} \ t \equiv t^\prime$ & $t_1\sbst{t}{x} \equiv t_1\sbst{t^\prime}{x}$ & by the above discussion \cr
\multicolumn{4}{c}{$\vdots$} \cr
$n$. & $\Gamma \ \phantom{\psi_4} \ t \equiv t^\prime$ & $t_n\sbst{t}{x} \equiv t_n\sbst{t^\prime}{x}$ & by the above discussion \cr
$(n + 1)$. & $\Gamma \ \phantom{\psi_4} \ t \equiv t^\prime$ & $\bigwedge\limits^n_{i = 1} t_i\sbst{t}{x} \equiv t_i\sbst{t^\prime}{x}$ & successively apply IV.3.6(b) to 1. through $n$. \cr
$(n + 2)$. & $\psi_4$ & $\psi_4$ & $\assm$ \cr
$(n + 3)$. & $\psi_4$ & $(\chi_6 \rightarrow \chi_7)$ & IV.5.5(a1) applied to $(n + 2)$. \cr
$(n + 4)$. & $\Gamma \ \psi_4 \ t \equiv t^\prime$ & $(\chi_6 \rightarrow \chi_7)$ & $\ant$ applied to $(n + 3)$. \cr
$(n + 5)$. & $\Gamma \ \psi_4 \ t \equiv t^\prime$ & $\bigwedge\limits^n_{i = 1} t_i\sbst{t}{x} \equiv t_i\sbst{t^\prime}{x}$ & $\ant$ applied to $(n + 1)$. \cr
$(n + 6)$. & $\Gamma \ \psi_4 \ t \equiv t^\prime$ & $ft_1\sbst{t}{x} \ldots t_n\sbst{t}{x} \equiv ft_1\sbst{t^\prime}{x} \ldots t_n\sbst{t^\prime}{x}$ & IV.3.5 applied to $(n + 4)$. and $(n + 5)$.
\end{tabular}\\
\ \\
\ \\
\textit{Concluding Note.} The proof system used in this text, namely the sequent calculus, consists solely of inference rules (i.e. the ten rules of $\mathfrak{S}_S$) and no \emph{logical axioms}.\footnote{Logical axioms are valid sentences in $L^S$ that play a role similar to that played by sequent rules of $\mathfrak{S}_S$.}\\
\ \\
Here we see that $\Psi_\equiv$ compensates the absence of the rules $\eq$ and $\sub$ from $\mathfrak{S}^\circ$. In this way, $\Psi_\equiv$ serves as the set of logical axioms for the proof system that consists of $\mathfrak{S}^\circ$ and $\Psi_\equiv$.\\
\ \\
In many other textbooks (such as \cite{Christos_Papadimitriou}), the proof systems are at the extreme: There are other logical axioms (for disjunctions and for quantifiers, to mention a few) besides those for equality, and the only inference rule is \emph{modus ponens} (cf. IV.3.5).
%%
\item Similar to the discussion in (2), if we regard $E t_1 t_2$ as rewriting $t_1 \equiv t_2$, then we obtain
\begin{center}
for $\Phi \cup \{ \varphi \} \subset L^S$: $\Phi \cup \Psi_\equiv \vdash^\circ \varphi$ \ iff \ $\Phi^\ast \cup \Psi_E \vdash^\circ \varphi^\ast$.
\end{center}
To summarize, the following are equivalent for $\Phi \cup \{ \varphi \} \subset L^S$:
\begin{enumerate}[(a)]
\item $\Phi \vdash \varphi$
%%%
\item $\Phi^\ast \vdash^\ast \varphi^\ast$
%%%
\item $\Phi \cup \Psi_\equiv \vdash^\circ \varphi$
%%%
\item $\Phi^\ast \cup \Psi_E \vdash^\circ \varphi^\ast$
%%%
\item $\Phi^\ast \cup \Psi_E \vdash \varphi^\ast$.
\end{enumerate}
This is an enhanced version of the statement given in part (b) of Exercise 6.11.
%%
\item Let $A$ be a given universe. It is clear that for every $S \cup \{ E \}$-interpretation $\mathfrak{I} = (\mathfrak{A}, \beta)$, $\mathfrak{I} \models \Psi_E$ iff $E^A$ is an equivalence relation over $A$; $\Psi_E$ is actually the set of \emph{axioms of equivalence relations} using the symbol $E$. Also note that the equality relation $=^A$ is itself an equivalence relation.\\
\ \\
So far, the equality symbol $\equiv$ has always been interpreted as $=^A$. However, many of the results we have obtained also hold \emph{when $\equiv$ is interpreted as any equivalence relation other than $=^A$}.\\
\ \\
In Henkin's Theorem, for instance, if $\Phi \subset L^S$ is consistent and if it is negation complete and contains witnesses, then the $S$-interpretation $(\mathfrak{A}, \beta)$ is a model of $\Phi$, where $A = T^S$, $c^A \colonequals c$,
\begin{center}
\begin{tabular}{lll}
$f^A (t_1, \ldots, t_n)$ & $\colonequals$ & $ft_1 \ldots t_n$, \cr
$t_1 \equiv^A t_2$       & :iff & $\Phi \vdash t_1 \equiv t_2$, \cr
$R^A (t_1, \ldots, t_n)$ & :iff & $\Phi \vdash Rt_1 \ldots t_n$,
\end{tabular}
\end{center}
and $\beta(v_n) \colonequals v_n$ for $n \in \nat$. Clearly $\equiv^A$ is an equivalence relation.\\
\ \\
As another example, we show the Adequacy Theorem holds in this situation: For $\Phi \cup \{ \varphi \} \subset L^S$,
\begin{center}
\begin{tabular}{ll}
\    & $\Phi \models \varphi$ when $\equiv$ is interpreted as any equivalence relation \cr
iff  & $\Phi^\ast \cup \Psi_E \models \varphi^\ast$ \cr
iff  & $\Phi^\ast \cup \Psi_E \vdash \varphi^\ast$ (by the usual Adequacy Theorem) \cr
iff  & $\Phi \vdash \varphi$ (by part (b) of Exercise 6.11).
\end{tabular}
\end{center}
And since $\Phi \vdash \varphi$ iff $\Phi \cup \Psi_\equiv \vdash^\circ \varphi$, $\Psi_\equiv$ is indeed the set of \emph{axioms of equivalence relations} using the symbol $\equiv$ in this situation.
\end{enumerate}
\end{enumerate}
%End of Section XI.6--------------------------------------------------------------------------
\
\\
\\
%Section XI.7---------------------------------------------------------------------
{\large \S7. Logic Programming}
\begin{enumerate}[1.]
\item \textbf{An Argument on 7.2.} The identity substitutor $\iota$ intuitively says that ``all variables remain unchanged''. So, part (a) is trivial.\\
\ \\
In part (b), both statements for terms and for quantifier-free formulas can be shown by induction. The case of terms is shown below:\\
$t = x$: By definition, $x (\sigma\tau) = (x \sigma)\tau$.\\
\ \\
$t = c$: $c (\sigma\tau) = c = c \tau = (c \sigma)\tau$.\\
\ \\
$t = ft_1 \ldots t_n$:\\
\begin{tabular}{lll}
\ & $(ft_1 \ldots t_n)(\sigma\tau)$ & \ \cr
= & $f(t_1(\sigma\tau)) \ldots (t_n(\sigma\tau))$ & \ \cr
= & $f((t_1\sigma)\tau) \ldots ((t_n\sigma)\tau)$ & (by induction hypothesis) \cr
= & $(f(t_1\sigma) \ldots (t_n\sigma))\tau$ & \ \cr
= & $((ft_1 \ldots t_n)\sigma)\tau$. & \ 
\end{tabular}\\
\ \\
And then the case of quantifier-free formulas:\\
$\varphi = t_1 \equiv t_2$:\\
\begin{tabular}{lll}
\ & $(t_1 \equiv t_2)(\sigma\tau)$ & \ \cr
= & $t_1(\sigma\tau) \equiv t_2(\sigma\tau)$ & \ \cr
= & $(t_1\sigma)\tau \equiv (t_2\sigma)\tau$ & (the result just shown) \cr
= & $(t_1\sigma \equiv t_2\sigma)\tau$ & \ \cr
= & $((t_1 \equiv t_2)\sigma)\tau$. & \ 
\end{tabular}\\
\ \\
$\varphi = Rt_1 \ldots t_n$:\\
\begin{tabular}{lll}
\ & $(Rt_1 \ldots t_n)(\sigma\tau)$ & \ \cr
= & $Rt_1(\sigma\tau) \ldots t_n(\sigma\tau)$ & \ \cr
= & $R(t_1\sigma)\tau \ldots (t_n\sigma)\tau$ & (the result just shown) \cr
= & $(Rt_1\sigma \ldots t_n\sigma)\tau$ & \ \cr
= & $((Rt_1 \ldots t_n)\sigma)\tau$. & \ 
\end{tabular}\\
\ \\
$\varphi = \neg\psi$:\\
\begin{tabular}{lll}
\ & $(\neg\psi)(\sigma\tau)$ & \ \cr
= & $\neg(\psi(\sigma\tau))$ & \ \cr
= & $\neg((\psi\sigma)\tau)$ & (by induction hypothesis) \cr
= & $((\neg\psi)\sigma)\tau$. & \ 
\end{tabular}\\
\ \\
$\varphi = \psi \lor \chi$:\\
\begin{tabular}{lll}
\ & $(\psi \lor \chi)(\sigma\tau)$ & \ \cr
= & $\psi(\sigma\tau) \lor \chi(\sigma\tau)$ & \ \cr
= & $(\psi\sigma)\tau \lor (\chi\sigma)\tau$ & (by induction hypothesis) \cr
= & $(\psi\sigma \lor \chi\sigma)\tau$ & \ \cr
= & $((\psi \lor \chi)\sigma)\tau$. & \ 
\end{tabular}\\
\ \\
Note that $\varphi (\sigma\tau) = (\varphi \sigma) \tau$ does not hold for formulas $\varphi$ \emph{with} quantifiers in general: Set $\varphi \colonequals \exists v_1 \, v_0 \equiv v_1$, $\sigma \colonequals \sbst{v_1}{v_0}$ and $\tau \colonequals \sbst{v_0}{v_1}$. Then $\varphi (\sigma\tau) = \exists v_1 \, v_0 \equiv v_1 \neq \exists v_2 \, v_0 \equiv v_2 = (\varphi \sigma) \tau$.\\
\ \\
Finally, part (c) can be argued: For $x \in V$,\\
\begin{tabular}{lll}
\ & $((\rho\sigma) \tau)(x)$ & \ \cr
= & $x ((\rho\sigma) \tau)$ & \ \cr
= & $(x (\rho\sigma)) \tau$ & (by definition) \cr
= & $((x \rho) \sigma) \tau$ & (by definition) \cr
= & $(x \rho) (\sigma\tau)$ & (by (b)) \cr
= & $x (\rho (\sigma\tau))$ & (by definition) \cr
= & $(\rho (\sigma\tau))(x)$. & \ 
\end{tabular}
%
\item \textbf{Note to Lemma on the Unifier 7.6.} In the proof, to see (1) is true, let $\psi_1, \psi_2$ be the two literals in $K\sigma_i$ considered at (UA4) in the iteration corresponding to the value $i$. Also, let $\S_1$ and $\S_2$ be the two letters thereof.\\
\ \\
Since $\sigma_{i + 1} = \sigma_i\sbst{t}{x}$, we may assume, according to (UA6), that $\S_1 = x$ and $t$ is the term that starts with $\S_2$ in $\psi_2$.\\
\ \\
Because $K\sigma_i\tau_i = (K\sigma_i)\tau_i$ contains a single element, we have $\psi_1 \tau_i = \psi_2 \tau_i$. In particular, $x\tau_i = t\tau_i$.\\
\ \\
On the other hand, as a corollary to this lemma, we have:
\begin{quote}
\emph{If a clause $K$ has a unifier, then it has a general unifier.}
\end{quote}
Assume $K$ has a unifier, then it is unifiable. Running the algorithm given in this lemma yields \emph{the} general unifier of $K$.
%
\item \textbf{Note to Definition 7.8.} Note that it is not necessary that
\begin{center}
$M_1 \cup L_1 = \emptyset$ \ \ \ and \ \ \ $M_2 \cup L_2 = \emptyset$.
\end{center}
So a U-resolvent $K$ of $K_1$ and $K_2$ may contain both literals $\psi$ and $\neg\psi$ for some atomic $\psi$ (cf. 5.5 and the footnote thereof).\\
\ \\
Also, a resolvent is automatically a U-resolvent.
%
\item \textbf{Note to the Discussion before Remark 7.9.} There is a typo in the fourth line from the bottom of page 231: ``variable free clauses'' should be replaced by ``variable-free clauses''.
%
\item \textbf{Note to Compatibility Lemma 7.10.} A ground instance of a clause $K$ is the clause counterpart of a ground instance of the disjunction of all literals in $K$. In other words, a ground instance of $K$ is the ground clause $K\sigma$ with some substitutor $\sigma$.\\
\ \\
In the part of proof for (b), by setting $\sigma_1 \colonequals \xi\eta\sigma$ and $\sigma_2 \colonequals \eta\sigma$, $K_1\sigma_1$ and $K_2\sigma_2$ may not be ground clauses. In fact, $\free{K_1\sigma_1 \cup K_2\sigma_2} = \free{L_1\sigma_1 \cup L_2\sigma_2} = \free{\varphi_0\sigma}$. Hence
\begin{center}
$K_1\sigma_1$ and $K_2\sigma_2$ are ground clauses \ \ \ iff \ \ \ $\varphi_0\sigma$ is a sentence.
\end{center}
%
\item \textbf{Note to Lemma 7.12.} Let us confirm that (b) follows from (a):\\
\begin{tabular}{lll}
\ & $\resi{\infty}{\gi{\mathfrak{K}}}$ & \ \cr
= & $\bigcup_{i \in \nat} \resi{i}{\gi{\mathfrak{K}}}$ & \ \cr
= & $\bigcup_{i \in \nat} \gi{\uresi{i}{\mathfrak{K}}}$ & (by (a)) \cr
= & $\bigcup_{i \in \nat} \bigcup_{K \in \uresi{i}{\mathfrak{K}}} \gi{K}$ & \ \cr
= & $\bigcup_{K \in \uresi{\infty}{\mathfrak{K}}} \gi{K}$ & \ \cr
= & $\gi{\uresi{\infty}{\mathfrak{K}}}$.
\end{tabular}
%
\item \textbf{Note to the Statement above Lemma 7.13.} Let $M$ be a set of clauses and $M^G$ be a ground instance of $M$. Then
\begin{center}
$M = \emptyset$ \ \ \ iff \ \ \ $M^G = \emptyset$.
\end{center}
So $\emptyset \in \gi{\uresi{\infty}{\mathfrak{K}}}$ iff $\emptyset \in \uresi{\infty}{\mathfrak{K}}$.
%
\item \textbf{Note to Lemma 7.13.} To appropriately use the resolution method suitable for propositional logic in first-order logic, our attempt to this may be to ``convert'' a set $\mathfrak{K}$ of clauses of first-order literals (i.e. literals that are first-order formulas) into the set $\resi{\infty}{\gi{\mathfrak{K}}}$ of clauses of propositional literals\footnote{As is mentioned in the discussion before 6.5, here we identify a ground clause with its image under $\pi$.} (i.e. literals that are propositional formulas) before applying this method.\\
\ \\
This lemma tells us that for this purpose it suffices to consider $\uresi{\infty}{\mathfrak{K}}$.
%
\item \textbf{Note to the Discussion after Main Lemma on the U-Resolution 7.13.} To see ($\ast$) holds, let us first note that for universal sentences $\varphi$ of the form
\[
\forall x_1 \ldots \forall x_m ((\varphi_{00} \lor \ldots \lor \varphi_{0l_0}) \land \ldots \land (\varphi_{s0} \lor \ldots \lor \varphi_{sl_s})),
\]
it can easily be verified that
\[
\gi{\mathfrak{K}(\varphi)} = \mathfrak{K}(\gi{\varphi}).
\]
Then ($\ast$) holds for sets $\Phi$ of sentences such as $\varphi$:\\
\begin{tabular}{lll}
\ & $\gi{\mathfrak{K}(\Phi)}$ & \ \cr
= & $\bigcup_{K \in \mathfrak{K}(\Phi)} \gi{K}$ & \ \cr
= & $\bigcup_{\varphi \in \Phi} \bigcup_{K \in \mathfrak{K}(\varphi)} \gi{K}$ & \ \cr
= & $\bigcup_{\varphi \in \Phi} \gi{\mathfrak{K}(\varphi)}$ & \ \cr
= & $\bigcup_{\varphi \in \Phi} \mathfrak{K}(\gi{\varphi})$ & (by the above discussion) \cr
= & $\bigcup_{\varphi \in \Phi} \bigcup_{\psi \in \gi{\varphi}} \mathfrak{K}(\psi)$ & ($\gi{\varphi}$ is also a set of sentences of the form \cr
\ & \ & \phantom{(}just mentioned) \cr
= & $\bigcup_{\chi \in \gi{\Phi}} \mathfrak{K}(\chi)$ & ($\gi{\Phi}$ is also a set of sentences of the form \cr
\ & \ & \phantom{(}just mentioned) \cr
= & $\mathfrak{K} (\gi{\Phi})$. & \ 
\end{tabular}
%
\item \textbf{Note to the Proof of Main Lemma on the UH-Resolution 7.16.} We provide a complete proof of this lemma.\\
\ \\
First, let $M$ be a clause and $M^G$ be a ground instance of $M$. Observe that
\begin{center}
$M$ is positive (negative) \ \ \ iff \ \ \ $M^G$ is positive (resp. negative).
\end{center}
By the Compatibility Lemma 7.10, we have: For any positive clause $K$ and any negative clause $N$:
\begin{enumerate}[(a)]
\item Every negative clause that is a resolvent of a ground instance of $K$ and a ground instance of $N$ is a ground instance of a negative clause that is a U-resolvent of $K$ and $N$.
%%
\item Every ground instance of a negative clause that is a U-resolvent of $K$ and $N$ is a negative clause that is a resolvent of a ground instance of $K$ and a ground instance of $N$.
\end{enumerate}
Next, we are ready to show\\
\ \\
($+$) \ \begin{minipage}{10cm}
for every set $\mathfrak{K}$ of clauses and for every $i \in \nat$:\\$\hresi{i}{\gi{\mathfrak{K}}} = \gi{\uhresi{i}{\mathfrak{K}}}$
\end{minipage}\\
\ \\
by induction on $i$: For $i = 0$, we have
\[
\hresi{0}{\gi{\mathfrak{K}}} = \gi{\mathfrak{K}} = \gi{\uhresi{0}{\mathfrak{K}}}.
\]
In the inductive step,\\
\begin{tabular}{lll}
\ & $\hresi{i + 1}{\gi{\mathfrak{K}}}$ & \ \cr
= & $\hres{\hresi{i}{\gi{\mathfrak{K}}}}$ & \ \cr
= & $\hres{\gi{\uhresi{i}{\mathfrak{K}}}}$ & (by induction hypothesis) \cr
= & $\gi{\uhres{\uhresi{i}{\mathfrak{K}}}}$ & (by the above discussion) \cr
= & $\gi{\uhresi{i + 1}{\mathfrak{K}}}$. & \ 
\end{tabular}\\
\ \\
\ \\
Then, as in 7.12(b), we obtain: for every set $\mathfrak{K}$ of clauses,\\
\begin{tabular}{lll}
\ & $\hresi{\infty}{\gi{\mathfrak{K}}}$ & \ \cr
= & $\bigcup_{i \in \nat} \hresi{i}{\gi{\mathfrak{K}}}$ & \ \cr
= & $\bigcup_{i \in \nat} \gi{\uhresi{i}{\mathfrak{K}}}$ & (by ($+$)) \cr
= & $\bigcup_{i \in \nat} \bigcup_{K \in \uhresi{i}{\mathfrak{K}}} \gi{K}$ & \ \cr
= & $\bigcup_{K \in \uhresi{\infty}{\mathfrak{K}}} \gi{K}$ & \ \cr
= & $\gi{\uhresi{\infty}{\mathfrak{K}}}$. & \ 
\end{tabular}\\
\ \\
In particular, for any set $\mathfrak{P}$ of positive clauses and any negative clause $N$,\\
\ \\
(1) \hfill $\hresi{\infty}{\gi{\mathfrak{P} \cup \{ N \}}} = \gi{\uhresi{\infty}{\mathfrak{P} \cup \{ N \}}}$. \hfill \phantom{(1)}\\
\ \\
Finally, as noted before 7.13, we have for every set $\mathfrak{K}$ of clauses: $\emptyset \in \gi{\uhresi{\infty}{\mathfrak{K}}}$ iff $\emptyset \in \uhresi{\infty}{\mathfrak{K}}$. And in particular, for any set $\mathfrak{P}$ of positive clauses and any negative clause $N$,\\
\ \\
(2) \hfill $\emptyset \in \gi{\uhresi{\infty}{\mathfrak{P} \cup \{ N \}}}$ \ \ iff \ \ $\emptyset \in \uhresi{\infty}{\mathfrak{P} \cup \{ N \}}$. \hfill \phantom{(2)}\\
\ \\
The lemma then immediately follows from (1) and (2).
%
\item \textbf{Note to Theorem on the UH-Resolution 7.17.} The statement
\begin{center}
\emph{$\emptyset$ is not UH-derivable from $\mathfrak{K}(\Phi)$ and $\mathfrak{K}(\varphi)$}
\end{center}
should be replaced by
\begin{center}
\emph{$\emptyset$ is not UH-derivable from $\mathfrak{K}(\Phi)$ and the negative clause $N \in \mathfrak{K}(\varphi)$,}
\end{center}
which is consistent with Definition 7.15.\\
\ \\
On the other hand, in the proof the equivalence of (4) and (5) actually follows from Definition 7.15 and the following statement: For sets $\mathfrak{P}$ of (first-order) clauses and for negative (first-order) clauses $N$ and $N^\prime$:
\begin{center}
\begin{minipage}{10cm}
$N^\prime \in \uhresi{i}{\mathfrak{P} \cup \{ N \}}$ \ iff \ there is a UH-derivation of $N^\prime$ of $\mathfrak{P}$ and $N$ of length $\leq i$
\end{minipage}
\end{center}
holds for $i \in \nat$. This is similar to that suggested in the proof of 5.11 and can be shown by induction on $i$.
%
\item \textbf{Note to the Proof of Theorem on Logic Programming 7.18.} In part (b), the two cases distinguished for showing\\
\ \\
($\ast$) \hfill $\Phi \vdash \psi_i \eta_1 \ldots \eta_k$ \hfill \phantom{($\ast$)}\\
\ \\
are actually the two:
\begin{itemize}
\item For $i \leq r$, $\neg\psi_i \eta_1 \in N_2$.
%%
\item There is \emph{exactly one} $i \leq r$ such that $\neg\psi_i \eta_1 \not\in N_2$.
\end{itemize}
In fact, if there is some $i \leq r$ such that $\neg\psi_i \eta_1 \not\in N_2$, then it is the only $i \leq r$ for which this happens, since there is at most one literal ``deleted'' from $N_1$ in the resolution step leading to $N_2$.\\
\ \\
In part (c), there is a typo in the fifth line from the bottom of page 240: ``Figure XI.11'' should be replaced by ``Figure XI.10''.
\end{enumerate}
%End of Section XI.7--------------------------------------------------------------
%End of Chapter XI----------------------------------------------------------------
%Chapter XII----------------------------------------------------------------------
\noindent{\LARGE \bfseries XII \\ \\ An Algebraic Characterization\\of Elementary Equivalence}
\\
\\
\\
%Section XII.1--------------------------------------------------------------------
{\large \S1. Finite and Partial Isomorphisms}
\begin{enumerate}[1.]
%
\item \textbf{Note on Definition 1.1.} If the map $\pi : A \to B$ is an isomorphism of $\struct{A}$ onto $\struct{B}$, then it is a partial isomorphism from $\struct{A}$ to $\struct{B}$.
%
\item \textbf{Note on Examples and Comments 1.2.} For part (c), we finish the incomplete \emph{proof} in text. Assuming ($\ast$), then for $i, j < r$\smallskip\\
\begin{quoteno}{(3)}
$a_i = a_j$ \quad iff \quad $b_i = b_j$,
\end{quoteno}\smallskip\\
since $p$ is well-defined and injective; for $n$-ary $P \in S$ and $\seq[1]{i}{n} < r$\smallskip\\
\begin{quoteno}{(4)}
$P^\struct{A} \enum[i_1]{a}{i_n}$ \ \ \ iff \ \ \ $P^\struct{B} \enum[i_1]{b}{i_n}$,
\end{quoteno}\smallskip\\
since $p$ is homomorphic.\bigskip\\
From (1) and (3), it follows that for $i, j < r$,\\
\centerline{$\struct{A} \models v_i \equal v_j [\seq{a}{r - 1}]$ \quad iff \quad $\struct{B} \models v_i \equal v_j [\seq{b}{r - 1}]$.}\\
Also, (2) and (4) together yield: for $n$-ary $P \in S$ and $\seq[1]{i}{n} < r$,\\
\centerline{$\struct{A} \models P \enum[i_1]{v}{i_n} [\seq{a}{r - 1}]$ \quad iff \quad $\struct{B} \models P \enum[i_1]{v}{i_n} [\seq{b}{r - 1}]$.}\\
Thus, we obtain ($\ast\ast$).\qed\\
\ \\
For part (d), a further counterexample is given here: Let $S_0 \defas \setenum{c, d}$ be a symbol set consisting only of constants. Consider the two $S_0$-structures $\struct{A} = \tuple{X, \intpted{c}{A}, \intpted{d}{A}}$ and $\struct{B} = \tuple{X, \intpted{c}{B}, \intpted{d}{B}}$, where $X \defas \setenum{0, 1, 2}$, $\intpted{c}{A} = \intpted{d}{A} = \intpted{c}{B} = 0$ and $\intpted{d}{B} = 1$. Then $p$ with $\dom{p} = \setenum{2}$ and $p(2) = 2$ is a partial isomorphism from $\struct{A}$ to $\struct{B}$, but $\struct{A} \models c \equal d [2]$ and not $\struct{B} \models c \equal d [2]$.\\
\ \\
Notice that the definition of partial isomorphisms is different in other texts like \cite{Heinz_Dieter_Ebbinghaus_and_Jorg_Flum}, where the constants are required to be in the domain of a partial isomorphism. In that setting, the equivalence between ($\ast$) and ($\ast\ast$) still holds and the above counterexample is not applicable.\\
\ \\
Below we investigate more about the relation between ($\ast$) and ($\ast\ast$). Let $S$ contain function symbols or constant symbols, and denote $A^\prime \colonequals \{ \seq{a}{r - 1} \}$ and $B^\prime \colonequals \{ \seq{b}{r - 1} \}$.\\
\ \\
It is still true that ($\ast\ast$) implies ($\ast$): Based on the proof showing ($\ast\ast$) implies ($\ast$) given in text, we only have to consider two more conditions, namely 1.1(b)(2) and (3). For condition 1.1(b)(2), let $f$ be an $n$-ary, and apply ($\ast\ast$) to all formulas $\psi = f \enum[i_1]{v}{i_n} \equal v_{i_{n + 1}}$ with $\seq[1]{i}{n + 1} < r$. For condition 1.1(b)(3), let $c$ be a constant symbol, and apply ($\ast\ast$) to all formulas $\psi = c \equal v_i$ with $i < r$.\\
\ \\
Moreover, if $A^\prime$ and $B^\prime$ are $S$-closed in $\struct{A}$ and in $\struct{B}$, respectively, then the equivalence of ($\ast$) and ($\ast\ast$) still holds: The direction from ($\ast\ast$) to ($\ast$) immediately follows from the above argument. So we only prove the inverse direction: Let $\struct{A}^\prime \colonequals [A^\prime]^\struct{A}$ and $\struct{B}^\prime \colonequals [B^\prime]^\struct{B}$, i.e.\ the structures generated by $A^\prime$ in $\struct{A}$ and by $B^\prime$ in $\struct{B}$, respectively (cf.\ the discussion after III.5.4). Assuming ($\ast$), then by definition of $p$, $p$ turns out to be an isomorphism of $\struct{A}^\prime$ onto $\struct{B}^\prime$ (cf.\ III.5.1). To show ($\ast\ast$) holds, for atomic formulas $\psi \in L^S_r$ we argue as follows:\smallskip\\
\begin{tabular}{lll}
\   & $\struct{A} \models \psi [\seq{a}{r - 1}]$ & \ \cr
iff & $\struct{A}^\prime \models \psi [\seq{a}{r - 1}]$ & (by Lemma III.5.5) \cr
iff & $\struct{B}^\prime \models \psi [\seqp{p(a_0)}{p(a_{r - 1})}]$ & (by Corollary III.5.3) \cr
iff & $\struct{B}^\prime \models \psi [\seq{b}{r - 1}]$ & (by definition of $p$) \cr
iff & $\struct{B} \models \psi [\seq{b}{r - 1}]$ & (by Lemma III.5.5).
\end{tabular}\\
\ \\
Thus, in the situation that $S$ contains function or constant symbols, we see from the above argument and the counterexample given in text that, in general, the equivalence in part (c) is not true since $\dom{p}$ or $\rg{p}$ may not be $S$-closed (in $\struct{A}$ and in $\struct{B}$, respectively).\\
\ \\
In summary, the reason that the equivalence may not hold for $p$ when there are constant or function symbols is that the terms involving these symbols may refer to elements outside of $\dom{p}$ that are ``hidden'' in an atomic formula, like $c \equal d$ or $v_0 + (v_0 + v_0) \equal v1$. (There are no hidden elements outside of $\dom{p}$ in an atomic formula if $S$ is relational.) Therefore, the equivalence also holds if we restrict the atomic formulas in ($\ast\ast$) to be \emph{term-reduced} (cf.\ Section VIII.1) even if $\dom{p}$ or $\rg{p}$ is not $S$-closed.\\
\ \\
Finally, in part (e) there is a typo: Both occurrences of $p_0$ in line 3 on page 246 should be replaced by those of $p$.
%
\item \textbf{Note on Lemma 1.5(d).} The situation stated in the hypothesis\\
\centerline{``$\struct{A} \partiso \struct{B}$, $A$ and $B$ are at most countable''}\\
can be split into two cases according to whether $A$ is finite:
\begin{enumerate}[(1)]
\item $A$ is finite, in addition to the (original) hypothesis: The new hypothesis is ``$\struct{A} \partiso \struct{B}$, $A$ is finite and $B$ at most countable.''
%%
\item $A$ is infinite, in addition to the (original) hypothesis: The new hypothesis is ``$\struct{A} \partiso \struct{B}$, $A$ is countable and $B$ at most countable.''
\end{enumerate}
In case (1), the conclusion $\struct{A} \iso \struct{B}$ immediately follows from parts (b) and (c) of this lemma. (And hence $B$ must be finite.)\\
\ \\
In case (2), let $I: \struct{A} \partiso \struct{B}$. Then, $B$ cannot be finite since, if it were, say $B = \{ b_1, \ldots, b_r \}$, repeated applications of the back-property to an arbitrary $p \in I$ to $b_1, \ldots, b_r$ would yield a partial isomorphism $q$ from $\struct{A}$ \emph{onto} $\struct{B}$, to which we cannot apply the forth-property! So, both $A$ and $B$ are countable. From here follows the proof given in text.
%
\item \textbf{Note on Example 1.8.} Recall that the Peano axiom system (cf.\ III.7.3(2)) is formalized by
\begin{enumerate}[(P1)]
\item $\forall x \neg \mbf{\sigma}x \equal 0$;
%%
\item $\forall x \forall y (\mbf{\sigma}x \equal \mbf{\sigma}y \rightarrow x \equal y)$;
%%
\item $\forall X ((X0 \land \forall x(Xx \rightarrow X \mbf{\sigma}x )) \rightarrow \forall y Xy)$,
\end{enumerate}
where (P3) is a second-order $\{ \mbf{\sigma}, 0 \}$-sentence. If we write
\begin{itemize}
\item (S0) \ for \ $\forall x (\neg x \equal 0 \liff \exists y \mbf{\sigma} y \equal x)$, and
%%
\item (S$m$) \ for \ $\forall x \neg\underbrace{\enump{\mbf{\sigma}}{\mbf{\sigma}}}_{\text{\mathmode{m}-times}} x \equal x$, where $m \geq 1$,
\end{itemize}
then $\Phi_\sigma = \{ \mbox{(P2)} \} \cup \setm{\mbox{(S\mathmode{n})}}{n \in \nat}$.\\
\ \\
Notice that:
\begin{enumerate}[(1)]
\item (P1) is a consequence of (S0).
%%
\item (S0) is a consequence of (P1) and (P3). (This can be verified by means of a derivation using the second rule given in IX.1.6 together with the one mentioned in \textbf{Note on 1.6} in Chapter IX.)
%%
\item For $m \geq 1$, (S$m$) is a consequence of (P1) - (P3). (This can be verified similarly.)
\end{enumerate}
One may attempt to replace (S0) in $\Phi_\sigma$ with (P1) to obtain an ``alternative'' and equivalent system of successor axioms; however, the structure $\struct{A} = \tuple{\nat, \mbf{\sigma}^A, 0}$, in which $\mbf{\sigma}^A (n) \colonequals n + 2$ for $n \in \nat$, shows that such a replacement results in a system of successor axioms not equivalent to $\Phi_\sigma$.\\
\ \\
There is a typo in line 12 from the bottom of page 248: ``$\pair{0^A}{0^B} \in I_n$'' should be replaced by ``$\{ \pair{0^A}{0^B} \} \in I_n$''.\\
\ \\
The distance function $d_n$ (or more precisely, the bound $2^n$) is so chosen that, in the situation where we obtain a $q \in I_n$ from a $p \in I_{n + 1}$ and an $a \in A$ for which ($\ast$) does not hold but however there is an $a_1 \in \dom{p}$ such that, say, $a = \underbrace{\enump{\mbf{\sigma}}{\mbf{\sigma}}}_\text{\mathmode{m}-times}(a_1)$ for some $m > 2^n$, if $a_2 \in A$ is between $a_1$ and $a$ (i.e.\ $a$ is ``reachable from'' $a_2$, which in turn is ``reachable from'' $a_1$, through $\mbf{\sigma}$) then $d_{n - 1}(a_1, a_2) = \infty$ or $d_{n - 1}(a_2, a) = \infty$, which is compatible with the method used in the proof.\\
\ \\
There is another typo in the second last line on page 248: ``every model $\Phi_\sigma$'' should be replaced by ``every model of $\Phi_\sigma$''.\\
\ \\
Finally, to see that every model $\struct{A}$ of $\Phi_\sigma$ is infinite, notice that $\mbf{\sigma}^A$ is injective but not surjective ($0^A \not\in \rg{\mbf{\sigma}^A}$!), so $A$ must be infinite (cf.\ the discussion in IX.1.3(6) in text).
%
\item \textbf{Solution to Exercise 1.9.} First note that an infinite $\emptyset$-structure is an infinite set, and vice versa. So, it suffices to show that, given two infinite sets $A$ and $B$, there is a nonempty set $I$ of partial isomorphisms from $A$ to $B$ such that $I : A \partiso B$.\\
\ \\
Let $I \colonequals \sett{p \in \partism{\struct{A}}{\struct{B}}}{\mathmode{\dom{p}} is finite}$. Then, $I \neq \emptyset$ since $\emptyset \in I$. Also, the forth- and the back-property trivially hold as there are infinite supplies of elements from $A$ and from $B$.
%
\item \textbf{Solution to Exercise 1.10.} (a) Consider the two dense orderings $\pair{\rat}{<}$ and $\pair{\real}{<}$. There can be no isomorphisms between them, although they are partially isomorphic (cf.\ Lemma 1.7 in text).\\
\ \\
Alternatively, we could also consider sets $\rat$ and $\real$, view them as $\emptyset$-structures and apply the result in Exercise 1.9.\\
\ \\
(b) Consider $S = \{ \mbf{\sigma}, 0 \}$ and let $\struct{A} = \tuple{A, \mbf{\sigma}^A, 0}$ with $A \colonequals \nat \cup \setm{\underline{m}}{m \in \zah}$, and
\begin{medcenter}
\begin{tabular}{llll}
$\mbf{\sigma}^A (n)$ & $\colonequals$ & $n + 1$ & for $n \in \nat$, \cr
$\mbf{\sigma}^A (\underline{m})$ & $\colonequals$ & $\underline{m + 1}$ & for $m \in \zah$.
\end{tabular}
\end{medcenter}
Both $\natstr_\sigma$ and $\struct{A}$ are countable; hence by 1.5(d) they are not partially isomorphic because they are not isomorphic. By 1.8, however, $\natstr_\sigma$ and $\struct{A}$ are finitely isomorphic since they are models of $\Phi_\sigma$.

As another example, consider the two orderings $(\zah, <), (\zah', <')$ where
\[
\zah' \defas \setenum{0, 1} \times \zah
\]
and for all $(b_1, z_1), (b_2, z_2) \in \zah'$,
\begin{center}
$(b_1, z_1) <' (b_2, z_2)$ \ \ iff \ \ ($b_1 < b_2$) or ($b_1 = b_2$ and $z_1 < z_2$).
\end{center}
Define the distance functions
\[
\begin{array}{l}
d : \zah \times \zah \to \nat, d(z_1, z_2) = \absval{z_1 - z_2} \cr
d' : \zah' \times \zah' \to \nat \union \setenum{\infty}, d'((b_1, z_1), (b_2, z_2)) =
\begin{cases}
\absval{z_1 - z_2} & \text{if \mathmode{b_1 = b_2}} \cr
\infty & \text{otherwise} \cr
\end{cases}\cr
\end{array}
\]
and their ``truncated versions'', for $n \in \nat$,
\[
\begin{array}{l}
d_n : \zah \times \zah \to \nat, d_n(z_1, z_2) =
\begin{cases}
d(z_1, z_2) & \text{if \mathmode{d(z_1, z_2) < 2^n}} \cr
\infty      & \text{otherwise} \cr
\end{cases}, \cr
d'_n : \zah' \times \zah' \to \nat, d'_m((b_1, z_1), (b_2, z_2)) =
\begin{cases}
d'((b_1, z_1), (b_2, z_2)) & \text{if \mathmode{d'((b_1, z_1), (b_2, z_2)) < 2^n}} \cr
\infty      & \text{otherwise} \cr
\end{cases}. \cr
\end{array}
\]
Then we have $\seqi{I_n}{n \in \nat} : \struct{A} \finiso \struct{B}$, where
\[
I_n \defas \sett{p \in \partism{\struct{A}}{\struct{B}}}{\mathmode{\dom{p}} is finite and for all \mathmode{a, a' \in \dom{p}}, \mathmode{d_n(a, a') = d'_n(p(a), p(a')}}.
\]
%
\item \textbf{Solution to Exercise 1.11.} Let $\struct{A} = \tuple{A, \intpted{\formal{\suc}}{A}, 0}$, where $A \defas \nat \union \setm{\underline{r}}{r \in \real}$ and
\begin{medcenter}
\begin{tabular}{llll}
$\intpted{\formal{\suc}}{A} (n)$ & $\defas$ & $n + 1$ & for $n \in \nat$, \cr
$\intpted{\formal{\suc}}{A} (\underline{r})$ & $\defas$ & $\underline{r + 1}$ & for $r \in \real$.
\end{tabular}
\end{medcenter}
It is easy to check that $\struct{A}$ is an uncountable model of $\Phi_\sigma$.
%
\item \textbf{Solution to Exercise 1.12.} Given two structures $\struct{A}$ and $\struct{B}$, if $I$ is a set for $\struct{A} \partemb \struct{B}$, then $(I)_{n \in \nat}$ is a sequence for $\struct{A} \finemb \struct{B}$. In other words,\smallskip\\
\begin{quoteno}{($\ast$)}
if \ $\struct{A} \partemb \struct{B}$ \ then \ $\struct{A} \finemb \struct{B}$.
\end{quoteno}\smallskip\\
This will turn out to be useful later.
\begin{enumerate}[(a)]
\item Let a symbol set $S$ be given, and let $\struct{A}$ and $\struct{B}$ be $S$-structures such that $(I_n)_{n \in \nat}$ is a sequence for $\struct{A} \finemb \struct{B}$. Furthermore, assume $A = \{\seq{a}{r}\}$.\\
\ \\
We pick an arbitrary $p \in I_{r + 1}$. Starting from $p$, and then by successively applying the forth-property to $\seq{a}{r}$, we obtain a map $q \in I_0$ with $\dom{q} = A$.\\
\ \\
We are done if we can show that $q$ is an isomorphism of $\struct{A}$ onto the substructure $[\rg{q}]^\struct{B}$ generated by $\rg{q}$ in $\struct{B}$ (cf.\ the discussion below III.5.4), for which it suffices to show $\rg{q}$ is $S$-closed: First, $\rg{q}$ consists of $r + 1$ elements and hence is not empty. Next, using 1.1(b)(2) and (3), respectively, we have:
\begin{enumerate}[(i)]
\item For $n$-ary $f \in S$ and for $\seq[1]{b}{n} \in \rg{q}$, $f^\struct{B} (\seq[1]{b}{n}) = q(f^\struct{A} (\seqp{q^{-1}(b_1)}{q^{-1}(b_n)})) \in \rg{q}$.
%%%
\item For $c \in S$, $c^\struct{B} = q(c^\struct{A}) \in \rg{q}$.
\end{enumerate}
%%
\item Let a symbol set $S$ be given, and let $\struct{A}$ and $\struct{B}$ be $S$-structures such that $I$ is a set for $\struct{A} \partemb \struct{B}$. If $A$ is finite, then from ($\ast$) and part (a) it immediately follows that $\struct{A}$ is embeddable in $\struct{B}$.\\
\ \\
So we consider the case in which $A$ is countable. Assume $A = \setm{a_n}{n \in \nat}$. We pick an arbitrary $p \in I$. Inductively, let $q_0 \colonequals p$ and for $n > 0$ define $q_n$ to be the map obtained by applying the forth-property to $q_{n - 1}$ and $a_{n - 1}$. The map $q \colonequals \bigcup_{n \in \nat} q_n$ is a partial isomorphism from $\struct{A}$ to $\struct{B}$ with $\dom{q} = A$.\\
\ \\
As in (a), we have that $\rg{q}$ is $S$-closed in $\struct{B}$, as can verified similarly. So, $q$ is an isomorphism from $\struct{A}$ onto the substructure $[\rg{q}]^\struct{B}$, hence $\struct{A}$ is embeddable in $\struct{B}$.
%%
\item Let $I$ be chosen as in Lemma 1.7 in text, and argue as in the proof there but leave out the back-property, which is not needed here (and may not even be available in case $\struct{A}$ is not dense).
\end{enumerate}
\end{enumerate}
%End of Section XII.1-------------------------------------------------------------
\
\\
\\
%Section XII.2--------------------------------------------------------------------
{\large \S2. Fra\"{i}ss\'{e}'s Theorem}
\begin{enumerate}[1.]
\item \textbf{Note on Fra\"{i}ss\'{e}'s Theorem 2.1.} In contrast, we have the following result which generalizes one direction of this theorem (and which also follows from this theorem!):\medskip\\
\emph{Let $S$ be a symbol set, and $\struct{A}$, $\struct{B}$ $S$-structures. Then we have: If $\struct{A} \finiso \struct{B}$ then $\struct{A} \equiv \struct{B}$.}
\begin{proof}
Assume that $(I_n)_{n \in \nat}: \struct{A} \finiso \struct{B}$. We show $\struct{A} \equiv \struct{B}$: For every $S$-sentence $\varphi$, there is a \emph{finite} $S_0 \subset S$ such that $\varphi$ is also an $S_0$-sentence, $(I_n)_{n \in \nat}: \reduct{\struct{A}}{S_0} \finiso \reduct{\struct{B}}{S_0}$ and hence $\reduct{\struct{A}}{S_0} \equiv \reduct{\struct{B}}{S_0}$ by Fra\"{i}ss\'{e}'s Theorem. Then $\struct{A} \models \varphi$\smallskip\\
\begin{tabular}[b]{lll}
iff & $\reduct{\struct{A}}{S_0} \models \varphi$ & (by the Coincidence Lemma) \cr
iff & $\reduct{\struct{B}}{S_0} \models \varphi$ & (since $\reduct{\struct{A}}{S_0} \equiv \reduct{\struct{B}}{S_0}$) \cr
iff & $\struct{B} \models \varphi$ & (by the Coincidence Lemma).
\end{tabular}
\end{proof}
%
\item \textbf{Note on Proposition 2.4.} There is a typo in part (b): $\thr{\nat, \suc}$ should be replaced by $\thr{\nat, \suc, 0}$.
%
\item \textbf{Solution to Exercise 2.5.} Let us denote $\Phi \colonequals \setm{\varphi_{\geq n}}{n \geq 2}$. Given $S = \emptyset$, any two models of $\consqn{\Phi}$ (i.e.\ any two infinite sets) are infinite $S$-structures, which, by Exercise 1.9, are partially isomorphic and hence (cf.\ 1.5(b)) finitely isomorphic, so they are elementarily equivalent by 2.1. Therefore, the theory $\consqn{\Phi}$ is complete (cf.\ 2.3) and is R-decidable since $\Phi$ is R-decidable (cf.\ X.6.5(a)).
%
\item \textbf{Solution to Exercise 2.6.} First, we show $\struct{A} \equiv \struct{B}$. As hinted in text, we have\\
\centerline{$\struct{A} \equiv \struct{B}$ \quad iff \quad for every finite $S_0 \subset S$, $\reduct{\struct{A}}{S_0} \finiso \reduct{\struct{B}}{S_0}$.}\\
So, it suffices to show for every finite $S_0 \subset S$,\smallskip\\
\begin{quoteno}{($\ast$)}
$\reduct{\struct{A}}{S_0} \iso \reduct{\struct{B}}{S_0}$
\end{quoteno}\smallskip\\
(from which it follows that $\reduct{\struct{A}}{S_0} \finiso \reduct{\struct{B}}{S_0}$ by Lemma 1.5).\\
\ \\
Let $S_0 \subset S$ be finite. If $S_0 = \emptyset$, then ($\ast$) trivially holds since both $A$ and $B$ are countable. If $S_0 \neq \emptyset$, then let
\[
n_0 \colonequals \max \setm{i \in \nat}{P_i \in S_0}.
\]
Choose $\pi : A \to B$ with
\[
\pi(n) \colonequals \begin{cases}
n      & \mbox{if \(n \leq n_0\)} \cr
\infty & \mbox{if \(n = n_0 + 1\)} \cr
n - 1  & \mbox{otherwise}
\end{cases}
\]
for $n \in \nat \ (= A)$. Then $\pi$ is bijective and, for all $n \leq n_0$ and all $m \in \nat$, $P_n^\struct{A} m$ if and only if $P_n^\struct{B} \pi(m)$. In particular, for all $n, m \in \nat$ such that $P_n \in S_0$,\\
\centerline{$P_n^\reduct{\struct{A}}{S_0} m$ \ \ \ iff\ \ \ $P_n^\reduct{\struct{B}}{S_0} \pi(m)$.}\\
Therefore, $\pi : \reduct{\struct{A}}{S_0} \iso \reduct{\struct{B}}{S_0}$.\\
\ \\
It remains to show: not $\struct{A} \finiso \struct{B}$. Note that $\infty \not\in \rg{p}$ for any partial isomorphism $p$ from $\struct{A}$ to $\struct{B}$ (because $P^\struct{B}_n \infty$ for $n \in \nat$ but meanwhile, there is no $a \in A$ such that $P^\struct{A}_n a$ for $n \in \nat$). So, there is no sequence $(I_n)_{n \in \nat}$ such that $(I_n)_{n \in \nat}: \struct{A} \finiso \struct{B}$ since the back-property does not hold for any $p \in I_1$ and $\infty$.\\
\ \\
\textit{Remarks.}
\begin{enumerate}[(1)]
\item The hint given in text ``for arbitrary $S$ and $S$-structures $\struct{A}, \struct{B}$ we have ($\struct{A} \equiv \struct{B}$ \ \ iff\ \ for every finite $S_0 \subset S$, $\reduct{\struct{A}}{S_0} \equiv \reduct{\struct{B}}{S_0}$)'' can easily be verified using the Coincidence Lemma.
%%
\item Consider the following two sets:
\[
\Phi \colonequals \{ \forall v_0 P_0 v_0 \} \cup \setm{\varphi_n}{n > 0} \cup \setm{\forall v_0 (P_{n + 1} v_0 \limply P_n v_0)}{n \in \nat},
\]
where $\varphi_n \in \fstordlang[0]{S}$ formulates ``there are exactly $n$ elements $\seq[1]{a}{n}$ (in the universe) for which $P_n$ does not hold,'' and
\[
\Psi \colonequals \setm{P_n v_0}{n \in \nat}.
\]
(Both are satisfied by the $S$-interpretation $\intpp{\struct{B}}{\assgn}$ where $\assgn(v_0) = \infty$.)\\
\ \\
We have:\footnote{Here we identify two structures if they are isomorphic.}
\begin{enumerate}[(a)]
\item For every $S$-structure $\struct{A}^\prime$ (with universe $A^\prime$), $\struct{A}^\prime \models \Phi$ if and only if there is a partial isomorphism $p$ from $\struct{A}^\prime$ \emph{onto} $\struct{A}$ and, moreover, for all $n \in \nat$, $P_n^{\struct{A}^\prime}$ holds for every element $a \in A^\prime \setminus \dom{p}$ (if any). In particular, $\struct{A} \models \Phi$.
%%%
\item For every $S$-structure $\struct{A}^\prime$, $\intpp{\struct{A}^\prime}{\assgn} \models \Phi \cup \Psi$ for some $\assgn$ in $\struct{A}^\prime$ if and only if $\struct{A}^\prime \models \Phi$ and not $\struct{A} \iso \struct{A}^\prime$ (this can easily be verified using the Coincidence Lemma). That is to say, all $S$-structures that are models of $\Phi$ other than $\struct{A}$ are exactly those with which the $S$-interpretations satisfy $\Phi \cup \Psi$.
\end{enumerate}
\ \\
On the other hand, for every sentence $\varphi$, $\Phi \cup \Psi \models \varphi$ if and only if $\Phi \models \varphi$: The direction from right to left is trivial. As for the other, note by the Compactness Theorem that there are finite subsets $\Phi_0 \subset \Phi$ and $\Psi_0 \subset \Psi$ such that $\Phi_0 \cup \Psi_0 \models \varphi$, of which we assume, without loss of generality, that $\Psi_0$ is not empty. So we have $\Phi_0 \cup \{ \bigwedge\Psi_0 \} \models \varphi$ and even $\Phi_0 \cup  \{ \exists v_0 \bigwedge\Psi_0 \} \models \varphi$ (using the rule IV.5.1(a)). Since $\Phi \models \exists v_0 \bigwedge\Psi_0$, it follows that $\Phi \models \varphi$.
\end{enumerate}
\end{enumerate}
%End of Section XII.2-------------------------------------------------------------
\
\\
\\
%Section XII.3--------------------------------------------------------------------
{\large \S3. Proof of Fra\"{i}ss\'{e}'s Theorem}
\begin{enumerate}[1.]
\item \textbf{Note on the Proof of 3.2.} Actually, the equivalence of (b) and (c) and of (e) and (f) is obtained by using the forth- and the back-property, respectively.
%
\item \textbf{Note on 3.3 and the Discussion before It.} Let $\struct{A}$ and $\struct{B}$ be two $S$-structures. Then by definition, we immediately have\\
\centerline{if \quad $(I_n)_{n \in \nat}: \struct{A} \finiso \struct{B}$ \quad then \quad $(I_n)_{n \leq m}: \struct{A} \iso_m \struct{B}$.}\\
Interestingly, by picking the sequence $(I_n)_{n \leq 0}$ which consists of only one element $I_0 = \{ \emptyset \}$ we have $(I_n)_{n \leq 0}: \struct{A} \iso_0 \struct{B}$.\\
\ \\
Also note that $m \geq 1$ for the notion of $\equiv_m$: For a relational symbol set $S$ every sentence has quantifier rank $\geq 1$.\footnote{If we allow $m = 0$, then for any two $S$-strucures $\struct{A}$ and $\struct{B}$ we have $\struct{A} \equiv_0 \struct{B}$. The claim of 3.3 in this case is also valid: If $\struct{A} \iso_0 \struct{B}$ then $\struct{A} \equiv_0 \struct{B}$.}
%
\item \textbf{Note on the Paragraph Defining $\varphi^n_{\struct{B}, \vect{b}{r}}$ below 3.3 on Page 253.} Since $S$ is a relational symbol set, there are no atomic $S$-sentences (i.e.\ $S$-sentences that are atomic formulas), and hence the case $r = 0$ for $n = 0$ in the definition of $\varphi^n_{\struct{B}, \vect{b}{r}}$ is not allowed (cf.\ 3.5(a)).\\
\ \\
Each $\varphi^{n + 1}_{\struct{B}, \vect{b}{r}}$ is defined in terms of $\varphi^n_{\struct{B}, \vect{b}{r}b}$. A visual representation for this is given below, where arrows indicate ``gives rise to'':\\
\ \\
\begin{tabular}{c|ccccccc}
\ & $0$ & \ & $1$ & \ & $2$ & $\cdots$ & $r$ \cr\hline
$0$ & \ & \ & $\bullet$ & \ & $\bullet$ & \ & \ \cr
\ & \ & $\swarrow$ & \ & $\swarrow$ & \ & \ & \ \cr
$1$ & $\bullet$ & \ & $\bullet$ & \ & $\bullet$ & \ & \ \cr
\ & \ & $\swarrow$ & \ & $\swarrow$ & \ & \ & \ \cr
$2$ & $\bullet$ & \ & $\bullet$ & \ & $\bullet$ & \ & \ \cr
$\vdots$ & \ & \ & \ & \ & \ & \ & \ \cr
$n$ & \ & \ & \ & \ & \ & \ & \ 
\end{tabular}\\
\ \\
Finally, if we define a binary relation $\sim$ over $r$-tuples of the domain $B$ of a given structure $\struct{B}$ such that for $\vect{a}{r}, \vect{b}{r} \in B^r$:
\begin{medcenter}
$\vect{a}{r} \sim \vect{b}{r}$ \ \ \ :iff \ \ \ $\struct{B} \models \varphi^n_{\struct{B}, \vect{b}{r}}[\vect{a}{r}]$
\end{medcenter}
then $\sim$ is an equivalence relation. Therefore, the set in 3.4 is a partition of $B$ into finitely many subsets. The subsets induced by the formulas $\varphi^n_{\struct{B}, \vect{b}{r}b}$ are merged to form a coarser partition when $\varphi^{n + 1}_{\struct{B}, \vect{b}{r}}$ are formed from $\varphi^n_{\struct{B}, \vect{b}{r}b}$; the definition of $\varphi^{n + 1}_{\struct{B}, \vect{b}{r}}$ formalizes the idea that ``given $\struct{B}$ and $\vect{b}{r}$, for every $b^\prime \in B$ there is some $b \in B$ such that $\struct{B} \models \varphi^n_{\struct{B}, \vect{b}{r}b}[\vect{b}{r}b^\prime]$ ($b^\prime$ is in some subset for $\varphi^n_{\struct{B}, \vect{b}{r}b}$), and for every $b \in B$ there is $b^\prime \in B$ such that $\struct{B} \models \varphi^n_{\struct{B}, \vect{b}{r}b}[\vect{b}{r}b^\prime]$ (the subset for $\varphi^n_{\struct{B}, \vect{b}{r}b}$ is nonempty).'' And with 3.6 and 3.7, if $\struct{A}$ and $\vect{a}{r}$ are such that $\struct{A} \satis \varphi^n_{\struct{B}, \vect{b}{r}}$ then $\varphi^n_{\struct{B}, \vect{b}{r}}$ formalizes the idea that the partial isomorphism $\funbyvect{\vect{a}{r}}{\vect{b}{r}}$ can be extended $n$ times using the forth- or the back-property.
%
\item \textbf{Induction Proof of 3.4.} In fact, it is more convenient to prove at the same time both 3.4 and the remark below it that the conjunctions and disjunctions (if any) occurring in the definition of $\varphi^n_{\struct{B}, \vect{b}{r}}$ are all finite, which implies that the $\varphi^n_{\struct{B}, \vect{b}{r}}$ are first-order formulas:
\medskip\\
\emph{The $\varphi^n_{\struct{B}, \vect{b}{r}}$ are well-formed first-order formulas - i.e.\ the conjunctions and disjunctions (if any) occurring in them are finite - and the set\\$\sett{\varphi^n_{\struct{B}, \vect{b}{r}}}{\(\struct{B}\) is an \(S\)-structure and \(\vect{b}{r} \in B\)}$ is finite.}
\begin{proof}
We prove it by induction on $n$.
\bigskip\\
The basis step $n = 0$: Let $r > 0$ be given. Then for any $S$-structure $\struct{B}$ and any $\vect{b}{r} \in B$, $\varphi^0_{\struct{B}, \vect{b}{r}}$ is a conjunction over a subset of $\Phi_r$. Since $\Phi_r$ is finite, it follows that this subset is also finite and $\varphi^0_{\struct{B}, \vect{b}{r}}$ is indeed a first-order formula; moreover, there are only finitely many different subsets of $\Phi_r$ in total, and this implies that the set $\sett{\varphi^0_{\struct{B}, \vect{b}{r}}}{\(\struct{B}\) is an \(S\)-structure and \(\vect{b}{r} \in B\)}$ is finite as well.
\bigskip\\
The induction step $n > 0$: Let $r \in \nat$ be given, and suppose that $\varphi^{n - 1}_{\struct{B}, \vect{b}{r}b}$ is a first-order formula for any $\struct{B}$ and any $\vect{b}{r}b \in B$ and that the set $\sett{\varphi^{n - 1}_{\struct{B}, \vect{b}{r}b}}{\(\struct{B}\) is an \(S\)-structure and \(\vect{b}{r}b \in B\)}$ is finite (hence it has only finitely many different subsets). Given $\struct{B}$ and $\vect{b}{r}$, the disjunction and conjunction occurring in $\varphi^n_{\struct{B}, \vect{b}{r}}$ are taken over a subset of the above set and a set consisting of formulas from this subset prefixed by $\exists v_r$, respectively, and both sets are obviously finite. It immediately follows that $\varphi^n_{\struct{B}, \vect{b}{r}}$ is a first-order formula for any $\struct{B}$ and any $\vect{b}{r}$ and that the set $\sett{\varphi^n_{\struct{B}, \vect{b}{r}}}{\(\struct{B}\) is an \(S\)-structure and \(\vect{b}{r} \in B\)}$ is finite.
\end{proof}
\begin{remark}
Recall that in Exercise 2.6 the symbol set $S = \setm{P_n}{n \in \nat}$ is infinite, therefore the set $\Phi_r$ in this case is infinite for $r > 0$ and the sets mentioned in 3.4 are infinite as well. Consequently, the $\varphi^n_{\struct{B}, \vect{b}{r}}$ are not first-order formulas.
\end{remark}
%
\item \textbf{Induction Proof of 3.5(a).} Let $\struct{B}$ and $\vect{b}{r} \in B$ be given.\\
\ \\
The base case $n = 0$: $\Phi_r \subset \fstordlang[r]{S}$, so it is clear from the definition that $\varphi^0_{\struct{B}, \vect{b}{r}} \in \fstordlang[r]{S}$. On the other hand, $\varphi^0_{\struct{B}, \vect{b}{r}}$ is a conjunction of atomic or negated atomic formulas, so $\qr{\varphi^0_{\struct{B}, \vect{b}{r}}} = 0$.\\
\ \\
Induction step: Let $b \in B$. Assume that $\varphi^n_{\struct{B}, \vect{b}{r}b} \in \fstordlang[r + 1]{S}$ and that $\qr{\varphi^n_{\struct{B}, \vect{b}{r}b}} = n$ have already been shown. From the definition it follows that $\varphi^{n + 1}_{\struct{B}, \vect{b}{r}} \in \fstordlang[r]{S}$ and $\qr{\varphi^{n + 1}_{\struct{B}, \vect{b}{r}}} = n + 1$.
%
\item \textbf{Note on the Last Paragraph (below the Proof of 3.8) on Page 254.} There is a typo in the fourth line from the bottom of page 254: ``the direction'' should be replaced by ``the other direction''.\\
\ \\
On the other hand, the argument in the second and the third lines of this paragraph yields for $n \geq 1$, $J_n \neq \emptyset$; in particular, $J_1 \neq \emptyset$. Then also $J_0 \neq \emptyset$ since $(J_n)_{n \in \nat}$ has the back- and the forth-property (cf.\ 3.8(b)).
%
\item \textbf{Note on Theorems 3.9 and 3.10.} The equivalence of the statements (a) - (d) in 3.9 follows from the implications below:\smallskip\\
\begin{tabular}{rl}
(a) implies (b): & Since $\struct{B} \models \varphi^n_{\struct{B}}$ for $n \in \nat$. \cr
(b) implies (c): & By 3.8. \cr
(c) implies (d): & By definition of the notation ``$(J_n)_{n \in \nat}: \struct{A} \finiso \struct{B}$''. \cr
(d) implies (a): & By 3.1.
\end{tabular}
\smallskip\\
The equivalence of the corresponding statments in 3.10 can be argued likewise.
%
\item \textbf{Note on Theorem 3.11.} Despite that there is no (single) $\setenum{R}$-sentence whose finite models are the finite connected graphs, the class in which finite models are exactly the finite connected graphs (while infinite ones are the infinite graphs, connected or not) is $\Delta$-elementary: Take the set
\[
\setenum{\varphi_{=2} \limply \forall v_0 \forall v_1 (\neg v_0 \equal v_1 \limply R v_0 v_1)} \union \setm{\varphi_{=n} \limply \psi_n}{n \geq 3}
\]
where $\varphi_{=n}$, $n \geq 2$, formalizes ``there are exactly $n$ elements'' (cf.\ III.6.3) and $\psi_n$, $n \geq 3$, formalizes ``for any two different elements there is a path of length $1$, $2$, \ldots or $n$ between them''.
%
\item \textbf{More about Finite Isomorphism and Elementary Equivalence.}
\begin{claim}
Let $S$ be a relational symbol set and $\struct{A}$ and $\struct{B}$ be $S$-structures. Then
\begin{enumerate}[\rm (a)]
%%
\item $\struct{A} \finiso \struct{B}$ \ \ \ iff \ \ \ $\struct{A} \iso_m \struct{B}$ for $m \in \nat$;
%%
\item $\struct{A} \equiv \struct{B}$ \ \ \ iff \ \ \ $\struct{A} \equiv_m \struct{B}$ for $m > 0$.
%%
\end{enumerate}
\end{claim}
\begin{proof}
For (a), if $\seqi{I_n}{n \in \nat} : \struct{A} \finiso \struct{B}$ and $m \in \nat$, then $\seqi{I_n}{n \leq m} : \struct{A} \iso_m \struct{B}$. Conversely, if $\seqi{I^\prime_{m, n}}{n \leq m} : \struct{A} \iso_m \struct{B}$ for $m \in \nat$, then $\seqi{I^\prime_n}{n \in \nat} : \struct{A} \finiso \struct{B}$ where $I^\prime_n \defas \bunion_{m \geq n} I^\prime_{m, n}$.
\medskip\\
For (b) the equivalnece is obvious because the quantifier rank of any $S$-sentence is a positive integer.
\end{proof}
%
\item \textbf{Note on the Proof of Theorem 3.11.} There are several typos:
\begin{enumerate}[(1)]
\item Line 6 on page 256 (line 2 of the second paragraph): the right parenthesis ``$)$'' appearing in ``$H_k \times H_k)$'' is redundant.
%%
\item Line 8 on the same page (line 2 of the definition of $d_n$): ``$2^m$'' should be replaced by ``$2^n$''.
%%
\item Line 12 on the same page (line 1 of the definition of $I_n$): the condition ``$\dom{p}$ is finite, and for all $a, b \in G_k$, $d_n(a, b) = {d_n}^\prime(p(a), p(b))$'' should be replaced by ``$\dom{p}$ contains at most $m - n$ elements, and for all $a, b \in \dom{p}$, $d_n(a, b) = {d_n}^\prime(p(a), p(b))$''.\\
\ \\
The restriction on the size of $p \in I_n$ in text, namely ``$\dom{p}$ is finite'', is not appropriate. Consider $k = 2^m$, $n = 1$ and $p$ where $\dom{p} = G_k$ and $p(i) = \pair{i}{0}$ for $0 \leq i \leq k$: The back-property cannot be applied to $p$ and $\pair{j}{1} \in H_k$ for any $0 \leq j \leq k$, as $G_k$ is already ``filled up'' with elements in $\dom{p}$.\\
\ \\
On the other hand, the condition $a, b \in \dom{p}$, instead of ``$a, b \in G_k$'' in text, is essential because $p(a)$ and $p(b)$ would be left undefined should $a$ and $b$ be elements in $G_k \setminus \dom{p}$.
\end{enumerate}
\ \\
\textit{Remark.} This theorem appears in Example 2.3.8 of \cite{Heinz_Dieter_Ebbinghaus_and_Jorg_Flum}.
%
\item \textbf{Solution to Exercise 3.12.} Assume that $B = \{ \seq[1]{b}{n} \}$, and let an $S$-structure $\struct{A}$ be given. Moreover, let the sequence $(J_n)_{n \in \nat}$ for $\struct{A}$ and $\struct{B}$ be defined as in text.\\
\ \\
For the direction from left to right, suppose that $\struct{A} \iso \struct{B}$. Since $\struct{B} \models \varphi^{n + 1}_{\struct{B}}$ (cf.\ 3.5(b)), by the Isomorphism Lemma we have $\struct{A} \models \varphi^{n + 1}_{\struct{B}}$.\\
\ \\
Conversely, for the direction from right to left, suppose that $\struct{A} \models \varphi^{n + 1}_{\struct{B}}$. By 3.10 we have $(J_m)_{m \leq n + 1}: \struct{A} \iso_{n + 1} \struct{B}$. Let $p \in I_{n + 1}$. Successively applying the back-property to $p$ and $\seq[1]{b}{n}$ yields a $q \in I_1$ such that $\rg{q} = B$ (so both $\dom{q}$ and $\rg{q}$ are of size $n$). We have $q: \struct{A} \iso \struct{B}$ because $A \setminus \dom{q} = \emptyset$: If $a \not\in \dom{p}$ for some $a \in A$, then by applying the forth-property to $q$ and $a$ we obtain a $q^\prime \in I_0$ such that $q^\prime \supset q$ and $a \in \dom{q^\prime}$, of which the domain $\dom{q^\prime}$ contains exactly $n + 1$ elements; it turns out that $\rg{q^\prime}$ contains exactly $n + 1$ elements as well. However, this is absurd because $\rg{q^\prime} \subset B$.
\begin{remark}
The equivalence may not be valid, i.e.\ the direction from right to left may fail, if $\varphi^{n + 1}_{\struct{B}}$ is replaced by $\varphi^n_{\struct{B}}$: Consider the case in which $S = \emptyset$ and $A$ contains $n + 1$ elements. The content of this exercise also appears in Corollary 2.2.10 of \cite{Heinz_Dieter_Ebbinghaus_and_Jorg_Flum}.
\end{remark}
%
\item \textbf{Solution to Exercise 3.13.} Let $n$ and $r$ (with $n + r > 0$) and $\vect{b}{r} \in B$ be given. We need to show:\\
\centerline{For all $S$-structures $\struct{A}$ and all $\vect{a}{r} \in A$, $\struct{A} \models (\varphi^{n + 1}_{\struct{B}, \vect{b}{r}} \limply \varphi^n_{\struct{B}, \vect{b}{r}})[\vect{a}{r}]$.}\\
We distinguish two cases: $r = 0$ or $r > 0$.\medskip\\
For $r = 0$ (and $n > 0$): Let $\struct{A}$ be an $S$-structure. Assume that $\struct{A} \models \varphi^{n + 1}_\struct{B}$. Then $\struct{A} \equiv_{n + 1} \struct{B}$ by 3.10. Since $\qr{\varphi^n_\struct{B}} = n < n + 1$, we have $\struct{A} \models \varphi^n_\struct{B}$ because $\struct{B} \models \varphi^n_\struct{B}$ (cf.\ 3.5).\\
\ \\
For $r > 0$: Let $\struct{A}$ be an $S$-structure and let $\vect{a}{r} \in A$. Assume that $\struct{A} \models \varphi^{n + 1}_{\struct{B}, \vect{b}{r}}[\vect{a}{r}]$. By definition, $\vect{a}{r} \mapsto \vect{b}{r} \in J_{n + 1}$ (the sequence $(J_n)_{n \in \nat}$ for $\struct{A}$ and $\struct{B}$ is defined as in text). Since $(J_n)_{n \in \nat}$ has the forth-property (cf.\ 3.8(b)), there is a map, say, $\vect{a}{r}a_0 \mapsto \vect{b}{r}b_0$, in $J_n$, which extends $\vect{a}{r} \mapsto \vect{b}{r}$ and whose domain contains $a_0$. But $\vect{a}{r}a_0 \mapsto \vect{b}{r}b_0$ is identical to $\vect{a}{r} \mapsto \vect{b}{r}$, therefore we have $\vect{a}{r} \mapsto \vect{b}{r} \in J_n$, i.e.\ $\struct{A} \models \varphi^n_{\struct{B}, \vect{b}{r}}[\vect{a}{r}]$.
%
\item \textbf{Solution to Exercise 3.14.} There is a typo in line 2 of this exercise: ``$\psi^{n + 1}_{\struct{B}, \vect{b}{r}}$'' should be replaced by ``$\psi^n_{\struct{B}, \vect{b}{r}}$'' (since these formulas are defined for all natural numbers, not just for positive integers).\\
\ \\
Let an $S$-structure $\struct{B}$ be given.
\begin{enumerate}[(a)]
\item First note that, similarly to 3.4, we can easily use induction on $n$ to show the set $\setm{\psi^n_{\struct{B}, \vect{b}{r}}}{\vect{b}{r} \in B}$
is finite (cf. \textbf{Induction Proof of 3.4}). The disjunctions occurring in the definition are therefore finte, so the $\psi^n_{\struct{B}, \vect{b}{r}}$ are first-order formulas.\\
\ \\
Next, we show they are universal by induction on $n$:\smallskip\\
The base case $n = 0$ (and $r > 0$): Given $\vect{b}{r} \in B$, we have $\psi^0_{\struct{B}, \vect{b}{r}} = \varphi^0_{\struct{B}, \vect{b}{r}}$, which is quantifier-free and hence universal.\medskip\\
The inductive case: Let $\vect{b}{r} \in B$ be given. By induction hypothesis, $\setm{\psi^n_{\struct{B}, \vect{b}{r}b}}{b \in B}$ is a set of universal formulas; since this set is finite, $\bigvee\setm{\psi^n_{\struct{B}, \vect{b}{r}b}}{b \in B}$ is a universal formula and so is $\forall v_r \bigvee\setm{\psi^n_{\struct{B}, \vect{b}{r}b}}{b \in B} = \psi^{n + 1}_{\struct{B}, \vect{b}{r}}$.
%%
\item Let an $S$-structure $\struct{A}$ be given. Then we have the following results on $\psi^n_{\struct{B}, \vect{b}{r}}$ which are counterparts of those on $\varphi^n_{\struct{B}, \vect{b}{r}}$:
\begin{enumerate}[(I)]
\item \emph{Let $\vect{b}{r} \in B$. Then: $\psi^n_{\struct{B}, \vect{b}{r}} \in L^S_r$ and $\qr{\psi^n_{\struct{B}, \vect{b}{r}}} = n$; moreover, $\struct{B} \models \psi^n_{\struct{B}, \vect{b}{r}}[\vect{b}{r}]$.}
%%%
\item \emph{Let $\vect{a}{r} \in A$ and $\vect{b}{r} \in B$. Then: $\struct{A} \models \psi^0_{\struct{B}, \vect{b}{r}}[\vect{a}{r}]$ \quad iff \quad $\vect{a}{r} \mapsto \vect{b}{r} \in \partism{\struct{A}}{\struct{B}}$.}
%%%
\item \emph{If $\vect{a}{r} \in A$ and $\vect{b}{r} \in B$, and if $\struct{A} \models \psi^n_{\struct{B}, \vect{b}{r}}[\vect{a}{r}]$, then $\vect{a}{r} \mapsto \vect{b}{r} \in \partism{\struct{A}}{\struct{B}}$.}
%%%
\item \emph{For $n \in \nat$, we set (analogous to $J_n$)
\[
K_n \colonequals \sett{\vect{a}{r} \mapsto \vect{b}{r}}{\(r \in \nat, \vect{a}{r} \in A, \vect{b}{r} \in B\) and \(\struct{A} \models \psi^n_{\struct{B}, \vect{b}{r}}[\vect{a}{r}]\)}.
\]
Then we obtain:
\begin{enumerate}[\rm(i)]
\item $K_n \subset \partism{\struct{A}}{\struct{B}}$;
%%%%
\item $(K_n)_{n \in \nat}$ has the forth-property;
%%%%
\item If $n > 0$ and $\struct{A} \models \psi^n_{\struct{B}}$ then $\emptyset \in K_n$, hence $K_n \neq \emptyset$.
\end{enumerate}}
%%%
\item \emph{Assume $(I_n)_{n \in \nat}: \struct{A} \finemb \struct{B}$. Then for every \emph{universal} formula $\varphi$:
\begin{quote}
If $\varphi \in \fstordlang[r]{S}$, $\qr{\varphi} \leq n$, $p \in I_n$ and $\seq{a}{r - 1} \in \dom{p}$, and if $\struct{B} \models \varphi[\seqp{p(a_0)}{p(a_{r - 1})}]$, then $\struct{A} \models \varphi[\seq{a}{r - 1}]$.
\end{quote}}
\end{enumerate}
\ \\
The above (I) - (V) are analogous to 3.5, 3.6, 3.7, 3.8 and 3.2, respectively. For (I), the \emph{proof} is symmetrical to that of 3.5, except that we do not consider $\struct{B} \models \exists v_r \psi^n_{\struct{B}, \vect{b}{r}b^\prime}[\vect{b}{r}]$ or $\struct{B} \models \bigwedge\setm{\exists v_r \psi^n_{\struct{B}, \vect{b}{r}b^\prime}}{b^\prime \in B}[\vect{b}{r}]$. For (II), it directly follows from 3.6 because $\varphi^0_{\struct{B}, \vect{b}{r}} = \psi^0_{\struct{B}, \vect{b}{r}}$. For (III), the \emph{proof} is symmetrical to that of 3.7, except that the claim for the base case follows from (II). For (IV), again, the \emph{proof} is symmetrical to that of 3.8, except that statement (i) follows from (III), that for statement (ii) we do not have to prove the back-property holds for $(K_n)_{n \in \nat}$ (note that, of course, $(K_n)_{n \in \nat}$ may not have the back-property), and that statement (iii) follows from the definition of $K_n$.\\
\ \\
Only for (V) the proof requires some work. First note that the part for cases (i) - (iii) of the proof of 3.2 given in text also validates the following statement:\smallskip\\
\begin{bquoteno}{64ex}{($\ast$)}
Let $\struct{A}$ and $\struct{B}$ be two $S$-structures. If $\varphi \in \fstordlang[r]{S}$ is a \emph{quantifier-free} formula, $p$ is a partial isomorphism from $\struct{A}$ to $\struct{B}$, and if $\seq{a}{r - 1} \in \dom{p}$, then\\
$\struct{A} \models \varphi[\seq{a}{r - 1}]$ \quad iff \quad $\struct{B} \models \varphi[\seqp{p(a_0)}{p(a_{r - 1})}]$.
\end{bquoteno}\bigskip\\
Next, we show the statement in (V) holds by induction on $\varphi$:
\begin{proof} Suppose $\varphi$ is a universal formula in $\fstordlang[r]{S}$, $\qr{\varphi} \leq n$, $p \in I_n$ and $\seq{a}{r - 1} \in \dom{p}$.\medskip\\
For quantifier-free formula $\varphi$, the claim immediately follows from ($\ast$).\medskip\\
For $\varphi = \psi \lor \chi$: $\struct{B} \models \varphi[\seqp{p(a_0)}{p(a_{r - 1})}]$\smallskip\\
\begin{tabular}{ll}
iff  & $\struct{B} \models \psi[\seqp{p(a_0)}{p(a_{r - 1})}]$ or $\struct{B} \models \chi[\seqp{p(a_0)}{p(a_{r - 1})}]$ \cr
then & $\struct{A} \models \psi[\seq{a}{r - 1}]$ or $\struct{A} \models \chi[\seq{a}{r - 1}]$ \cr
\    & (by induction hypothesis) \cr
iff  & $\struct{A} \models \varphi[\seq{a}{r - 1}]$.
\end{tabular}\medskip\\
For $\varphi = \psi \land \chi$: The argument is analogous.\medskip\\
For $\varphi = \forall x \psi$: Similarly to the argument for case (iv) in the proof of 3.2, we may assume that $x = v_r$. Also, because $\qr{\varphi} = \qr{\forall x \psi} \leq n$, we have $\qr{\psi} \leq n - 1$. We prove the claim as follows:\smallskip\\
\begin{tabular}[b]{ll}
\    & not $\struct{A} \models \varphi[\seq{a}{r - 1}]$ \cr
iff  & not (for all $a \in A$, $\struct{A} \models \psi[\seq{a}{r - 1}, a]$) \cr
iff  & there is $a \in A$ such that not $\struct{A} \models \psi[\seq{a}{r - 1}, a]$ \cr
iff  & \begin{minipage}[t]{56ex}there is $a \in A$ and $q \in I_{n - 1}$ such that $q \supset p$, $a \in \dom{q}$, and not $\struct{A} \models \psi[\seq{a}{r - 1}, a]$\\(by the forth-property of the sequence $(I_n)_{n \in \nat}$)\end{minipage} \cr
then & \begin{minipage}[t]{56ex}there is $a \in A$ and $q \in I_{n - 1}$ such that $q \supset p$, $a \in \dom{q}$, and not $\struct{B} \models \psi[\seqp{p(a_0)}{p(a_{r - 1})}, q(a)]$\\(by induction hypothesis)\end{minipage} \cr
iff  & \begin{minipage}[t]{56ex}there is $b \in B$ and $q \in I_{n - 1}$ such that $q \supset p$, $b \in \rg{q}$, and not $\struct{B} \models \psi[\seqp{p(a_0)}{p(a_{r - 1})}, b]$\end{minipage} \cr
then & there is $b \in B$ such that not $\struct{B} \models \psi[\seqp{p(a_0)}{p(a_{r - 1})}, b]$ \cr
iff  & not (for all $b \in B$, $\struct{B} \models \psi[\seqp{p(a_0)}{p(a_{r - 1})}, b]$) \cr
iff  & not $\struct{B} \models \varphi[\seqp{p(a_0)}{p(a_{r - 1})}]$.
\end{tabular}
\end{proof}
Now the claim in this part of exercise immediately follows:
\begin{itemize}
\item (1) implies (2): Assume (1) holds. Then for all $n \geq 1$, we have $\struct{A} \models \psi^n_{\struct{B}}$ since $\struct{B} \models \psi^n_{\struct{B}}$ by (I). So (2) holds as well.
%%%
\item (2) implies (3): Assume (2) holds. Then for $n \geq 1$, $K_n \neq \emptyset$ by (IV)(iii). Hence $K_0 \neq \emptyset$ by (IV)(ii) (apply the forth-property to an arbitrary $p \in K_1$ and some $a \in A$). Therefore $(K_n)_{n \in \nat}: \struct{A} \finemb \struct{B}$ and (3) holds as well.
%%%
\item (3) implies (1): Assume (3) holds. Then (1) immediately follows from (V) by taking $r = 0$, $n = \qr{\varphi}$ and an arbitrary $p \in I_n$ for each universal  $S$-sentence $\varphi$ considered in (1).
\end{itemize}
%%
\item Analogous to the notion of $m$-isomorphic, we say that two structures $\struct{A}$ and $\struct{B}$ are \emph{$m$-embeddable} (written: $\struct{A} \emb_m \struct{B}$) if  there is a sequence $\seq{I}{m}$ of nonempty sets of partial isomorphisms from $\struct{A}$ to $\struct{B}$ with the forth-property, i.e.,
\begin{quote}
for $n + 1 \leq m$, $p \in I_{n + 1}$ and $a \in A$, there is $q \in I_n$ such that $q \supset p$ and $a \in \dom{q}$.
\end{quote}
In this case, we write $(I_n)_{n \leq m}: \struct{A} \emb_m \struct{B}$.\\
\ \\
Then we formulate and prove the version analogous to 3.10 below:\smallskip\\
For an $S$-structure $\struct{A}$ and for $m \geq 1$ the following are equivalent:
\begin{enumerate}[(1)]
\item $\struct{A}$ satisfies every universal $S$-sentence of quantifier rank $\leq m$ which holds in $\struct{B}$.
%%%
\item $\struct{A} \models \psi^m_\struct{B}$.
%%%
\item $\struct{A} \emb_m \struct{B}$.
\end{enumerate}
\begin{proof}
Since $\qr{\psi^m_\struct{B}} = m$ (see (I) in part (b)), (2) follows from (1).\\
\ \\
For the direction from (2) to (3), the argument is analogous to that of 3.10: Because of the forth-property of the sequence $(K_n)_{n \in \nat}$ (see (IV)(ii) in part (b)), if $K_m \neq \emptyset$, then also $K_n \neq \emptyset$ for $n < m$.\\
\ \\
Finally, notice that in case $(I_n)_{n \leq m}: \struct{A} \emb_m \struct{B}$, the proof of (V) in part (b) shows that each $p \in I_n$ (with $n \leq m$) preserves the validity of universal formulas of quantifier rank $\leq n$ in $\struct{A}$ provided that these formulas hold in $\struct{B}$ (cf.\ the discussion before 3.3). Hence, analogous to 3.3, we obtain the direction from (3) to (1).
\end{proof}
\end{enumerate}
%
\item \textbf{Solution to Exercise 3.15.} For convenience, we introduce a syntactic form of formulas similar to that of term-reduced formulas (cf.\ \reftitle{VIII.1.1}). Let $S$ be a symbol set. For $\varphi \in \fstordlang{S}$, we define $\varphi^\ast \in \fstordlang{S}$ inductively as follows:\smallskip\\
First, for formulas of simple forms:
\[
\begin{array}{lll}
[y \equal x]^\ast & \colonequals & y \equal x; \cr
[c \equal x]^\ast & \colonequals & c \equal x; \cr
[x \equal c]^\ast & \colonequals & c \equal x; \cr
[f\enum[1]{x}{n} \equal x]^\ast & \colonequals & f\enum[1]{x}{n} \equal x; \cr
[R\enum[1]{x}{n}]^\ast & \colonequals & R\enum[1]{x}{n}.
\end{array}
\]
Second, assume that $\seq[1]{t}{n}$ are terms and that at least one of them is not a variable. Let $1 \leq \enumpop{i_1}{<}{i_s} \leq n$ such that $\seq[i_1]{t}{i_s}$ are exactly those terms among $\seq[1]{t}{n}$ that are not variables. Then
\[
[f\enum[1]{t}{n} \equal x]^\ast \colonequals \enump{\exists x_{i_1}}{\exists x_{i_s}}\psi,
\]
in which $\psi$ is the formula in prenex normal form that is logically equivalent to
\[
\parenadj{\bigwedge\limits^s_{r = 1} [t_{i_r} \equal x_{i_r}]^\ast \land f\enum[1]{x}{n} \equal x},
\]
where $\seq[1]{x}{n}$ are variables in the list $v_0, v_1, v_2, \ldots$ in which $\seq[i_1]{x}{i_s}$ are pairwise distinct and are not in $\{ x \} \cup \parenadj{\bigcup\limits^n_{j = 1}\var{t_j}}$ and furthermore, for $1 \leq k \leq n$, if $k \not\in \{ \seq[1]{i}{s} \}$ then $t_k = x_k$.\\
\ \\
Third, if $\seq[1]{t}{n}$ are terms then
\[
[x \equal f\enum[1]{t}{n}]^\ast \colonequals [f\enum[1]{t}{n} \equal x]^\ast.
\]
Fourth, if both $t_1$ and $t_2$ are not variables then
\[
[t_1 \equal t_2]^\ast \colonequals \exists x \psi,
\]
in which $\psi$ is the formula in prenex normal form that is logically equivalent to
\[
([t_1 \equal x]^\ast \land [t_2 \equal x]^\ast),
\]
where $x$ is a variable not in $\var{t_1} \cup \var{t_2}$.\\
\ \\
Fifth, if $\seq[1]{t}{n}$ are terms and at least one of them is not a variable, and if $1 \leq \enumpop{i_1}{<}{i_s} \leq n$ such that $\seq[i_1]{t}{i_s}$ are exactly those terms among $\seq[1]{t}{n}$ that are not variables, then
\[
[R\enum[1]{t}{n}]^\ast \colonequals \enump{\exists x_{i_1}}{\exists x_{i_s}} \psi,
\]
in which $\psi$ is the formula in prenex normal form that is logically equivalent to
\[
\parenadj{\bigwedge^s_{r = 1} [t_{i_r} \equal x_{i_r}]^\ast \land R\enum[1]{x}{n}},
\]
where $\seq[1]{x}{n}$ are variables in the list $v_0, v_1, v_2, \ldots$ in which $\seq[i_1]{x}{i_s}$ are pairwise distinct and are not in $\bigcup\limits^n_{j = 1}\var{t_j}$ and furthermore, for $1 \leq k \leq n$, if $k \not\in \{ \seq[1]{i}{s} \}$ then $t_k = x_k$.\\
\ \\
Finally, we set
\[
\begin{array}{lll}
[\neg\varphi]^\ast & \colonequals & \neg\varphi^\ast; \cr
(\varphi \lor \psi)^\ast & \colonequals & (\varphi^\ast \lor \psi^\ast); \cr
[\exists x \varphi]^\ast & \colonequals & \exists x \varphi^\ast.
\end{array}
\]
Notice that\smallskip\\
\begin{bquoteno}{68ex}{($+$)}
\emph{for atomic formulas $\varphi$, $\varphi^\ast$ is in prenex normal form and the number of quantifiers therein is $\mrk{\varphi}$.}
\end{bquoteno}\smallskip\\
This can be verified by a simple induction on (the form of) atomic formulas $\varphi$:
\begin{proof} For formulas $\varphi$ of the form $y \equal x$, $c \equal x$, $x \equal c$, $f\enum[1]{x}{n} \equal x$ or $R\enum[1]{x}{n}$, the claim is trivially true.\medskip\\
For $\varphi = f\enum[1]{t}{n} \equal x$ in which some $t_j$ is not a variable, by definition we have $\varphi^\ast = \enump{\exists x_{i_1}}{\exists x_{i_s}} \psi$, where $\psi$ is the formula in prenex normal form that is logically equivalent to
\[
\parenadj{\bigwedge\limits^s_{r = 1} [t_{i_r} \equal x_{i_r}]^\ast \land f\enum[1]{x}{n} \equal x}.
\]
So $\varphi$ is also in prenex normal form. Moreover, by induction hypothesis we have for $1 \leq r \leq s$ that the number of quantifiers in $[t_{i_r} \equal x_{i_r}]^\ast$ is $\mrk{t_{i_r} \equal x_{i_r}}$; hence the number of quantifiers in $\psi$ is $\left(\sum^s_{r = 1} \mrk{t_{i_r} \equal x_{i_r}}\right)$, and so the number of quantifiers in $\varphi$ is
\[
\begin{array}{ll}
\ & s + \displaystyle\sum\limits^s_{r = 1} \mrk{t_{i_r} \equal x_{i_r}} \cr
= & s + \displaystyle\sum\limits^s_{r = 1} (\mrk{t_{i_r}} - 1) \cr
= & s + \left(\displaystyle\sum\limits^s_{r = 1} \mrk{t_{i_r}}\right) - s \cr
= & \left(\displaystyle\sum\limits^s_{r = 1} \mrk{t_{i_r}}\right) \cr
= & \left(\displaystyle\sum\limits^n_{j = 1} \mrk{t_j}\right) \cr
= & \left(\left(\displaystyle\sum\limits^n_{j = 1} \mrk{t_j}\right) + 1\right) - 1 \cr
= & \mrk{f\enum[1]{t}{n}} - 1 \cr
= & \mrk{f\enum[1]{t}{n} \equal x}.
\end{array}
\]
The cases $\varphi = R\enum[1]{t}{n}$ or $\varphi = t_1 \equal t_2$ are similar.\medskip\\
Finally, the case $\varphi = x \equal f\enum[1]{t}{n}$ immediately follows by definition and induction hypothesis.
\end{proof}
It can be shown by induction on $\varphi$ that for $\varphi \in \fstordlang{S}$:\smallskip\\
\begin{bquoteno}{68ex}{($\ast$)}
\emph{$\free{\varphi} = \free{\varphi^\ast}$, $\varphi$ is logically equivalent to $\varphi^\ast$ (cf.\ \reftitle{VIII.1.2}), and $\mrk{\varphi} = \mrk{\varphi^\ast}$.}
\end{bquoteno}\smallskip\\
Here we only prove the last statement:
\begin{proof}
If $\varphi$ is of simple form $y \equal x$, $c \equal x$, $x \equal c$, $f\enum[1]{x}{n} \equal x$ or $R\enum[1]{x}{n}$, then the claim is trivially true.\\
\ \\
If $\varphi = f\enum[1]{t}{n} \equal x$ in which some $t_j$ is not a variable then, by ($+$), $\mrk{\varphi}$ is equal to the number of quantifiers in $\varphi^\ast$. Since $\varphi^\ast$ is in prenex normal form (also by ($+$)) and its matrix is a conjunction of atomic formulas of the form $y \equal z$, $c \equal z$ or $g\enum[1]{z}{m} \equal z$ (which have modified quantifier rank $0$), $\mrk{\varphi^\ast}$ is also equal to the number of quantifiers in $\varphi^\ast$. The claim is true.\\
\ \\
The cases $\varphi = R\enum[1]{t}{n}$ or $\varphi = t_1 \equal t_2$ are similar.\\
\ \\
The claim is also true for the remaining cases $\varphi = x \equal f\enum[1]{t}{n}$, $\varphi = \neg\psi$, $\varphi = (\psi \lor \chi)$ or $\varphi = \exists x \psi$, which immediately follow by induction hypothesis.
\end{proof}
Now we are ready to prove the claims in this exercise.
\begin{enumerate}[(a)]
\item Let a finite symbol set $S$ be given.\\
\ \\
We state and prove the counterpart of 3.2:\medskip\\
\emph{Let $(I_n)_{n \in \nat}: \struct{A} \finiso \struct{B}$. Then for every formula $\varphi$:\smallskip\\
\emph{\begin{bquoteno}{62ex}{($\ast\ast$)}
\emph{If $\varphi \in \fstordlang[r]{S}$, $\mrk{\varphi} \leq n$, $p \in I_n$ and $\enum{a}{r - 1} \in \dom{p}$, then\\
$\struct{A} \models \varphi[\seq{a}{r - 1}]$ \quad iff \quad $\struct{B} \models \varphi[\seqp{p(a_0)}{p(a_{r - 1})}]$.}
\end{bquoteno}}}
\begin{proof}
Assume that $(I_n)_{n \in \nat}: \struct{A} \finiso \struct{B}$. By ($\ast$) it suffices to show ($\ast\ast$) holds for all formulas $\varphi \in \fstordlang[r]{S}$ for which $\varphi^\ast = \varphi$. This is done by induction on these formulas. Suppose $\varphi \in \fstordlang[r]{S}$, $\mrk{\varphi} \leq n$, $p \in I_n$ and $\seq{a}{r - 1} \in \dom{p}$.\smallskip\\
For $\varphi = y \equal x$ or $\varphi = R\enum[1]{x}{m}$ the claim was proved in 1.2(c).\medskip\\
If $\varphi = c \equal v_i$ where $0 \leq i < r$: Then $\struct{A} \models \varphi[\seq{a}{r - 1}]$\smallskip\\
\begin{tabular}{ll}
iff & $c^\struct{A} = a_i$ \cr
iff & \begin{minipage}[t]{\DefaultTabularizedArgumentLength}$c^\struct{B} = p(a_i)$ \quad (since $p \in \partism{\struct{A}}{\struct{B}}$ and $a_i \in \dom{p}$)\end{minipage} \cr
iff & $\struct{B} \models \varphi[\seqp{p(a_0)}{p(a_{r - 1})}]$.
\end{tabular}\medskip\\
If $\varphi = f\enum[i_1]{v}{i_m} \equal v_{i_{m + 1}}$ where $0 \leq \seq[1]{i}{m + 1} < r$: Then $\struct{A} \models \varphi[\seq{a}{r - 1}]$\smallskip\\
\begin{tabular}{ll}
iff & $f^\struct{A} (\seq[i_1]{a}{i_m}) = a_{i_{m + 1}}$ \cr
iff & \begin{minipage}[t]{\DefaultTabularizedArgumentLength}$f^\struct{B} (\seqp{p(a_{i_1})}{p(a_{i_m})}) = p(a_{i_{m + 1}})$ \quad (since $p \in \partism{\struct{A}}{\struct{B}}$ and $\seq[i_1]{a}{i_{m + 1}} \in \dom{p}$)\end{minipage} \cr
iff & $\struct{B} \models \varphi[\seqp{p(a_0)}{p(a_{r - 1})}]$.
\end{tabular}\medskip\\
The remaining cases $\varphi = \neg\psi$, $\varphi = (\psi_0 \lor \psi_1)$ or $\varphi = \exists x \psi$ have already been settled in the corresponding parts of the proof of 3.2 in text.
\end{proof}
By taking $r = 0$, $n = \mrk{\varphi}$ and an arbitrary $p \in I_n$, we obtain from the above result the counterpart of 3.1 (this time $S$ is any finite symbol set!):\smallskip\\
\begin{quoteno}{(1)}
\emph{If $\struct{A} \finiso \struct{B}$ then $\struct{A} \equiv \struct{B}$.}
\end{quoteno}\\
\ \\
As in the discussion before 3.3 in text, if we write $\struct{A} \equiv^\prime_m \struct{B}$ in case $\struct{A}$ and $\struct{B}$ satisfy the same sentences of modified quantifier rank $\leq m$ (note that $m \geq 1$ since sentences have modified quantifier rank $\geq 1$), then from the previous proof we likewise have\smallskip\\
\begin{quoteno}{(1)$^\prime$}
\emph{If $\struct{A} \iso_m \struct{B}$ then $\struct{A} \equiv^\prime_m \struct{B}$.}
\end{quoteno}\\
\ \\
In the following we are focused on proving the converse of (1).\medskip\\
Since $S$ is finite, ${\Phi_r}^\prime$ is also finite, and in particular ${\Phi_0}^\prime = \emptyset$. For $S$-structures $\struct{B}$ and for $n, r \in \nat$ (with $n + r > 0$), let the formulas $\varphi^n_{\struct{B}, \vect{b}{r}}$ be defined as in text.\\
\ \\
Then we have the following results corresponding to 3.4 - 3.8, respectively:
\begin{enumerate}[(I)]
\item \emph{The set $\sett{\varphi^n_{\struct{B}, \vect{b}{r}}}{\(\struct{B}\) is an \(S\)-structure and \(\vect{b}{r} \in B\)}$ is finite.}
%%%
\item \emph{Let $\struct{B}$ be an $S$-structure and $\vect{b}{r} \in B$. Then: $\varphi^n_{\struct{B}, \vect{b}{r}} \in \fstordlang[r]{S}$ and $\mrk{\varphi^n_{\struct{B}, \vect{b}{r}}} = n$; moreover, $\struct{B} \models \varphi^n_{\struct{B}, \vect{b}{r}}[\vect{b}{r}]$.}
%%%
\item \emph{Let $\struct{A}$ and $\struct{B}$ be $S$-structures, $\vect{a}{r} \in A$, and $\vect{b}{r} \in B$. Then: $\struct{A} \models \varphi^0_{\struct{B}, \vect{b}{r}}[\vect{a}{r}]$ \quad iff \quad $\vect{a}{r} \mapsto \vect{b}{r} \in \partism{\struct{A}}{\struct{B}}$.}
%%%
\item \emph{If $\struct{A}$ and $\struct{B}$ are $S$-structures, $\vect{a}{r} \in A$ and $\vect{b}{r} \in B$, and if $\struct{A} \models \varphi^n_{\struct{B}, \vect{b}{r}}[\vect{a}{r}]$, then $\vect{a}{r} \mapsto \vect{b}{r} \in \partism{\struct{A}}{\struct{B}}$.}
%%%
\item \emph{Fixing two $S$-structures $\struct{A}$ and $\struct{B}$, for $n \in \nat$ define
\[
J_n \colonequals \sett{\vect{a}{r} \mapsto \vect{b}{r}}{\(r \in \nat, \vect{a}{r} \in A, \vect{b}{r} \in B\) and \(\struct{A} \models \varphi^n_{\struct{B}, \vect{b}{r}}[\vect{a}{r}]\)}.
\]
Then:
\begin{enumerate}[\rm(i)]
\item $J_n \subset \partism{\struct{A}}{\struct{B}}$;
%%%%
\item $(J_n)_{n \in \nat}$ has the back- and the forth-property;
%%%%
\item If $n > 0$ and $\struct{A} \models \varphi^n_\struct{B}$ then $\emptyset \in J_n$, hence $J_n \neq \emptyset$.
\end{enumerate}}
\end{enumerate}
\begin{proof}
We only prove (III), the others can likewise be proved as their counterparts in text.\\
\ \\
It suffices to show:\smallskip\\
\begin{bquoteno}{62ex}{($++$)}
Let $\vect{a}{r} \in A$, $\vect{b}{r} \in B$. We have: $\vect{a}{r} \mapsto \vect{b}{r} \in \partism{\struct{A}}{\struct{B}}$ if and only if for every atomic formula $\varphi \in \fstordlang[r]{S}$ with $\mrk{\varphi} = 0$:\\$\struct{A} \models \varphi[\vect{a}{r}]$ \quad iff \quad $\struct{B} \models \varphi[\vect{b}{r}]$.
\end{bquoteno}\bigskip\\
Notice that every atomic formula $\varphi$ with $\mrk{\varphi} = 0$ must take the form:
\begin{quote}
$v_i \equal v_j$, $c \equal v_j$, $v_j \equal c$, $f\enum[i_1]{v}{i_n} \equal v_j$, $v_j \equal f\enum[i_1]{v}{i_n}$ or $R\enum[i_1]{v}{i_n}$.
\end{quote}
Let $\vect{a}{r} \in A$ and $\vect{b}{r} \in B$. We prove ($++$) as follows:\medskip\\
The direction from left to right: Assume $\vect{a}{r} \mapsto \vect{b}{r} \in \partism{\struct{A}}{\struct{B}}$. We verify the claim on the right side for each case of atomic formulas.\medskip\\
For $\varphi = c \equal v_j$ where $0 \leq j < r$, we have\smallskip\\
\begin{tabular}{lll}
\   & $\struct{A} \models c \equal v_j[\vect{a}{r}]$ & \cr
iff & $c^\struct{A} = a_j$ & \cr
iff & $c^\struct{B} = b_j$ & (since $\vect{a}{r} \mapsto \vect{b}{r} \in \partism{\struct{A}}{\struct{B}}$) \cr
iff & $\struct{B} \models c \equal v_j[\vect{b}{r}]$. & \cr
\end{tabular}\smallskip\\
For $\varphi = f\enum[i_1]{v}{i_n} \equal v_j$ where $0 \leq \seq[1]{i}{n}, j < r$, we have\smallskip\\
\begin{tabular}{lll}
\   & $\struct{A} \models f\enum[i_1]{v}{i_n} \equal v_j[\vect{a}{r}]$ & \cr
iff & $f^\struct{A}(\seq[i_1]{a}{i_n}) = a_j$ & \cr
iff & $f^\struct{B}(\seq[i_1]{b}{i_n}) = b_j$ & (since $\vect{a}{r} \mapsto \vect{b}{r} \in \partism{\struct{A}}{\struct{B}}$) \cr
iff & $\struct{B} \models f\enum[i_1]{v}{i_n} \equal v_j[\vect{b}{r}]$. & \cr
\end{tabular}\smallskip\\
For $\varphi = v_j \equal c$ or $\varphi = v_j \equal f\enum[i_1]{v}{i_n}$, the arguments are symmetrical; for $\varphi = v_i \equal v_j$ or $\varphi = R\enum[i_1]{v}{i_n}$, the claim has already been verified in 1.2(c).\\
\ \\
The direction from right to left: Assume the validity of atomic formulas which have modified quantifier rank $0$ is preserved under $\vect{a}{r} \mapsto \vect{b}{r}$, we need to show $\vect{a}{r} \mapsto \vect{b}{r} \in \partism{\struct{A}}{\struct{B}}$. By the proof of 1.2(c), we only have to check function symbols and constant symbols for $\vect{a}{r} \mapsto \vect{b}{r}$ to be homomorphic:\medskip\\
Let $f \in S$ be an $n$-ary function symbol, and let $0 \leq \seq[1]{i}{n}, j < r$, then\smallskip\\
\begin{tabular}{lll}
\   & $f^\struct{A}(\seq[i_1]{a}{i_n}) = a_j$ & \cr
iff & $\struct{A} \models f\enum[i_1]{v}{i_n} \equal v_j[\vect{a}{r}]$ & \cr
iff & $\struct{B} \models f\enum[i_1]{v}{i_n} \equal v_j[\vect{b}{r}]$ & (by premise) \cr
iff & $f^\struct{B}(\seq[i_1]{b}{i_n}) = b_j$. & \cr
\end{tabular}\smallskip\\
Let $c \in S$ be a constant symbol, and let $0 \leq j < r$, then\smallskip\\
\begin{tabular}{lll}
\   & $c^\struct{A} = a_j$ & \cr
iff & $\struct{A} \models c \equal v_j[\vect{a}{r}]$ & \cr
iff & $\struct{B} \models c \equal v_j[\vect{b}{r}]$ & (by premise) \cr
iff & $c^\struct{B} = b_j$. & \cr
\end{tabular}\medskip\\
Therefore, $\vect{a}{r} \mapsto \vect{b}{r}$ is homomorphic and hence a partial isomorphism from $\struct{A}$ to $\struct{B}$.
\end{proof}
Now we are ready to obtain the converse of (1) (cf.\ the discussion before 3.9). Assume that $\struct{A} \equiv^\prime \struct{B}$. Then for $n > 0$, $\struct{A} \models \varphi^n_\struct{B}$, because $\struct{B} \models \varphi^n_\struct{B}$ by (II); we have $J_n \neq \emptyset$ for $n > 0$ by (V)(iii). Since $(J_n)_{n \in \nat}$ has the forth-property (and the back-property) by (V)(ii), we may apply this to any $p \in J_1$ and any $a \in A$ to obtain a $q \in J_0$; we have $J_0 \neq \emptyset$. Hence $(J_n)_{n \in \nat}: \struct{A} \finiso \struct{B}$. Then we get the converse of (1):\smallskip\\
\begin{quoteno}{(2)}
If $\struct{A} \equiv^\prime \struct{B}$ then $\struct{A} \finiso \struct{B}$.
\end{quoteno}\medskip\\
From the preceding considerations we obtain the following result which is the counterpart of 3.9:\smallskip\\
\emph{Let $S$ be a finite symbol set, and let $\struct{A}$ and $\struct{B}$ be $S$-structures. Then the following are equivalent:\smallskip\\
\emph{\begin{tabular}{ll}
\begin{minipage}{32ex}(a) \emph{$\struct{A} \equiv^\prime \struct{B}$.}\end{minipage} & \begin{minipage}{32ex}(c) \emph{$(J_n)_{n \in \nat}: \struct{A} \finiso \struct{B}$.}\end{minipage} \cr
\begin{minipage}{32ex}(b) \emph{$\struct{A} \models \varphi^n_\struct{B}$ for $n \geq 1$.}\end{minipage} & \begin{minipage}{32ex}(d) \emph{$\struct{A} \finiso \struct{B}$.}\end{minipage}
\end{tabular}}}
\medskip\\
Since $\mrk{\varphi^m_\struct{B}} = m$ for $m \geq 1$, we get the counterpart of 3.10:\smallskip\\
\emph{Let $S$ be a finite symbol set, and let $\struct{A}$ and $\struct{B}$ be $S$-structures. Then the following are equivalent for $m \geq 1$:\smallskip\\
\emph{\begin{tabular}{ll}
\begin{minipage}{32ex}(a) \emph{$\struct{A} \equiv^\prime_m \struct{B}$.}\end{minipage} & \begin{minipage}{32ex}(c) \emph{$(J_n)_{n \leq m}: \struct{A} \iso_m \struct{B}$.}\end{minipage} \cr
\begin{minipage}{32ex}(b) \emph{$\struct{A} \models \varphi^m_\struct{B}$.}\end{minipage} & \begin{minipage}{32ex}(d) \emph{$\struct{A} \iso_m \struct{B}$.}\end{minipage}
\end{tabular}}}
\begin{proof}
The direction from (a) to (b): Let $\struct{A} \equiv^\prime_m \struct{B}$. Since $\struct{B} \models \varphi^m_\struct{B}$ by (II), we have $\struct{A} \models \varphi^m_\struct{B}$ as well.\\
\ \\
The direction from (b) to (c): Let $\struct{A} \models \varphi^m_\struct{B}$. By (V)(iii) we have $J_m \neq \emptyset$; hence, $J_n \neq \emptyset$ for $n \leq m$ by (V)(ii). So by (V)(i) it follows that $(J_n)_{n \leq m}: \struct{A} \iso_m \struct{B}$.\\
\ \\
The direction from (c) to (d): By definition.\\
\ \\
Finally, the direction from (d) to (a): By (1)$^\prime$.
\end{proof}
Recall that in \reftitle{Exercises 3.12} and \reftitle{3.13}, every time we verified statements involving $\qrbase$ or $\Phi_r$ (directly or indirectly) we referred to \reftitle{Theorems 3.2 - 3.10}, namely, \reftitle{Theorems 3.9} and \reftitle{3.10} and the considerations leading to them. Since they remain valid with $\Phi_r$ and $\qrbase$ replaced by ${\Phi_r}^\prime$ and $\mrkbase$, respectively, the claims in \reftitle{Exercises 3.12} and \reftitle{3.13} remain valid as well.\\
\ \\
On the other hand, by replacing all occurrences of $\Phi_r$ and of $\qrbase$ by those of ${\Phi_r}^\prime$ and of $\mrkbase$, respectively, together with suitable replacement of references in the proof given in \reftitle{Exercise 3.14}, we get a proof for the \emph{new version} of \reftitle{Exercise 3.14}. This part of exercise is complete.
%%
\item If $S$ is relational, then every atomic formula must have the form $v_i \equal v_j$ or $R\enum[i_1]{v}{i_n}$.\\
\ \\
Thus, for atomic or negated atomic formulas $\varphi$, $\mrk{\varphi} = 0$. It follows that $\Phi_r = {\Phi_r}^\prime$.\\
\ \\
Next, we prove by induction on $\varphi$ that $\qr{\varphi} = \mrk{\varphi}$ for all $\varphi \in \fstordlang{S}$:\medskip\\
If $\varphi$ is atomic, then by the above argument we have $\qr{\varphi} = 0 = \mrk{\varphi}$, and the claim is true in this case.\smallskip\\
If $\varphi = \neg\psi$, then
\[
\begin{array}{lll}
\ & \qr{\varphi} & \cr
= & \qr{\psi} & \cr
= & \mrk{\psi} & \mbox{(by induction hypothesis)} \cr
= & \mrk{\varphi}. & \cr
\end{array}
\]
If $\varphi = \psi \lor \chi$, then
\[
\begin{array}{lll}
\ & \qr{\varphi} & \cr
= & \max \{ \qr{\psi}, \qr{\chi} \} & \cr
= & \max \{ \mrk{\psi}, \mrk{\chi} \} & \mbox{(by induction hypothesis)} \cr
= & \mrk{\psi\lor\chi}. & \cr
\end{array}
\]
If $\varphi = \exists x \psi$, then
\[
\begin{array}{lll}
\ & \qr{\exists x \psi} & \cr
= & \qr{\psi} + 1 & \cr
= & \mrk{\psi} + 1 & \mbox{(by induction hypothesis)} \cr
= & \mrk{\exists x \psi}. & \cr
\end{array}
\]
\end{enumerate}
\textit{Note.}
\begin{inparaenum}[(1)]
%%
\item In textbook, the equal sign ``$=$'' that appears in the definition of $\mrkbase$ should be replaced by ``$\colonequals$''.\medskip\\
%%
\item We could have proved the statement ($++$) in part (a) otherwise: We only have to consider\smallskip\\
\centerline{$v_i \equal v_j$, $c \equal v_j$, $f\enum[i_1]{v}{i_n} \equal v_j$ or $R\enum[i_1]{v}{i_n}$}\smallskip\\
for $\varphi$, as $v_j \equal c$ and $v_j \equal f\enum[i_1]{v}{i_n}$ are logically equivalent to $c \equal v_j$ and $f\enum[i_i]{v}{i_n} \equal v_j$, respectively. Also note that if $\chi \in \fstordlang{S}$ is in the form of the above formulas then $\relational{\chi}$ is atomic; conversely, if $\chi \in \fstordlang{\relational{S}}$ is atomic then $\invrelational{\chi}$ is in the form of the above formulas. Using the fact that $\partism{\struct{A}}{\struct{B}} = \partism{\relational{\struct{A}}}{\relational{\struct{B}}}$ (cf.\ the discussion after \reftitle{2.4}) and \reftitle{VIII.1.3(a)}, we immediately obtain ($++$) from 1.2(c).
%%
\end{inparaenum}
%
\item \textbf{Solution to Exercise 3.16.} There is a typo in the hint given in part (a): In the definition of $I_n$, the statement ``$\dom{p}$ is finite'' should be replaced by ``$\dom{p}$ contains no more than $m - n + 2$ elements''; moreover, the statement ``both $0$ and $k$ are elements of $\dom{p}$'' should be added to the definition. Namely,
\[
\begin{array}{l}
I_n \colonequals \{ p \in \partism{\struct{A}_k}{\struct{A}_l} \mid \mbox{\(\dom{p}\) contains no more than \(m - n + 2\)} \cr
\phantom{I_n \colonequals \{ }\mbox{\begin{minipage}{\DefaultTabularizedArgumentLength}elements, both \(0\) and \(k\) are elements of \(\dom{p}\), \(p(0) = 0\), \(p(k) = l\), and for all \(a, b \in \dom{p}\), \(d_n(a, b) = d_n(p(a), p(b))\)\}.\end{minipage}}
\end{array}
\]
\begin{enumerate}[(a)]
\item Let $m, k, l \in \nat$ such that $k, l \geq 2^{m + 1}$, and let the distance functions $d_n$ and the sets $I_n$ be defined as in hint. We show that $(I_n)_{n \leq m}: \struct{A}_k \iso_m \struct{A}_l$ below.\medskip\\
First, for $n \leq m$, $I_n \neq \emptyset$ since the map $\{(0, 0), (k, l)\} \in I_n$.\medskip\\
Next, $(I_n)_{n \leq m}$ has the forth-property: Assume $0 \leq a \leq k$ and $p \in I_{n + 1}$, where $n + 1 \leq m$. As in 1.8, we consider the following condition:\smallskip\\\begin{quoteno}{($\ast$)}
There is an $a^\prime \in \dom{p}$ such that $\absval{d_n(a^\prime, a)} < 2^{n + 1}$.
\end{quoteno}\smallskip\\
If the condition ($\ast$) holds for $a^\prime \in \dom{p}$, then we choose \emph{the} $b$ with $0 \leq b \leq l$ and $d_n(a^\prime, a) = d_n(p(a^\prime), b)$ (note that such a $b$ must uniquely exist). It follows that $p \cup \{(a, b)\} \in I_n$. If ($\ast$) does not hold (i.e.\ $d_n(a^\prime, a) = \infty$ for $a^\prime \in \dom{p}$), then we choose an arbitrary $b$ with $0 \leq b \leq l$ and $d_n(p(a^\prime), b) = \infty$ such that $b$ is related to elements in $\rg{p}$ in the ordering $<$ in the same manner as $a$ is related to elements in $\dom{p}$ in the ordering $<$ (such a $b$ must exist because $\absval{p(a_0) - p(a_1)} \geq 2^{n + 2}$ since $\absval{a_0 - a_1} \geq 2^{n + 2}$ and $d_{n + 1}(a_0, a_1) = d_{n + 1}(p(a_0), p(a_1))$, where $a_0, a_1 \in \dom{p}$, $a_0$ is the largest among those in $\dom{p}$ that are $< a$ and $a_1$ is the smallest among those in $\dom{p}$ that are $> a$). It follows that $p \cup \{(a, b)\} \in I_n$.\medskip\\
Finally, $(I_n)_{n \leq m}$ has the back-property: Symmetrical to the above argument.
%%
\item Suppose there were $\varphi \in \fstordlang[0]{S}$ such that\\
\centerline{$\struct{A}_k \models \varphi$ \quad iff \quad $k$ is even.}\\
Let $\qr{\varphi} = m$ and choose $k \geq 2^{m + 1}$. By (a) we would have that $\struct{A}_k \iso_m \struct{A}_{k + 1}$, and hence by 3.3 that $\struct{A}_k \equiv_m \struct{A}_{k + 1}$; therefore,\smallskip\\
\begin{tabular}{lll}
\   & $k$ is even & \cr
iff & $\struct{A}_k \models \varphi$ & (by premise) \cr
iff & $\struct{A}_{k + 1} \models \varphi$ & (since $\struct{A}_k \equiv_m \struct{A}_{k + 1}$) \cr
iff & $k + 1$ is even, & (by premise) \cr
\end{tabular}\smallskip\\
a contradiction.
\begin{remark}
In fact, there is no $\emptyset$-sentence whose finite models are the structures of even cardinalities. Moreover, for any symbol set $S$ the class in which finite models are the structures of even cardinalities is not elementary (cf.\ Example 2.3.6 of \cite{Heinz_Dieter_Ebbinghaus_and_Jorg_Flum}). However, this class is $\Delta$-elementary: Take the set $\setm{\neg\varphi_{=2n+1}}{n \in \nat}$ where $\varphi_{=m}$ formulates the idea that ``there are exactly $m$ elements'' (cf.\ III.6.3).
\end{remark}
\end{enumerate}
%
\item \textbf{Solution to Exercise 3.17.} Fix an $m \geq 1$ and let $\struct{A}$ be an $S$-structure. For $\seq[1]{l}{r} \in \{ 0, 1 \}$ we write $A^{\tuple{\seq[1]{l}{r}}}$ for $\enumpop{A_1^{(l_1)}}{\cap}{A_r^{(l_r)}}$, where
\[
A_i^{(l_i)} \colonequals \begin{cases}
P^A_i             & \mbox{if \(l_i = 0\)} \cr
A \setminus P^A_i & \mbox{othewise}
\end{cases}
\]
for $1 \leq i \leq r$. Likewise define $B^{\tuple{\seq[1]{l}{r}}}$ for the structure $\struct{B}$ introduced below.\\
\ \\
Let $\struct{B}$ be a substructure of $\struct{A}$ such that for all $\seq[1]{l}{r} \in \{ 0, 1 \}$:
\begin{itemize}
\item $B^{\tuple{\seq[1]{l}{r}}} \subset A^{\tuple{\seq[1]{l}{r}}}$;
%%
\item if $A^{\tuple{\seq[1]{l}{r}}}$ contains fewer than $m$ elements, then $B^{\tuple{\seq[1]{l}{r}}} = A^{\tuple{\seq[1]{l}{r}}}$, otherwise $B^{\tuple{\seq[1]{l}{r}}}$ contains exactly $m$ elements from $A^{\tuple{\seq[1]{l}{r}}}$.
\end{itemize}
Obviously $B$ contains at most $m \cdot 2^r$ elements.\\
\ \\
For $\struct{A} \iso_m \struct{B}$ it suffices to show $\struct{A} \models \varphi^m_\struct{B}$ by 3.10. For this purpose we show\smallskip\\
\begin{quoteno}{($\ast$)}
for $n \leq m$, \ if $\vect{a}{m - n} \mapsto \vect{b}{m - n} \in \partism{\struct{A}}{\struct{B}}$ then $\struct{A} \models \varphi^n_{\struct{B}, \vect{b}{m - n}}[\vect{a}{m - n}]$
\end{quoteno}\smallskip\\
by \emph{finite} induction on $n$. Then ($\ast$) together with 3.7 yields\\
\centerline{for $n \leq m$, \quad $\struct{A} \models \varphi^n_{\struct{B}, \vect{b}{m - n}}[\vect{a}{m - n}]$ \quad iff \quad $\vect{a}{m - n} \mapsto \vect{b}{m - n} \in \partism{\struct{A}}{\struct{B}}$.}\\
In particular, $\struct{A} \models \varphi^m_\struct{B}$ since $\emptyset \in \partism{\struct{A}}{\struct{B}}$.\medskip\\
We immediately have ($\ast$) holds for the base case $n = 0$ by 3.6.\medskip\\
For the inductive step, assume $n + 1 \leq m$ and let $p \colonequals \vect{a}{m - n - 1} \mapsto \vect{b}{m - n - 1} \in \partism{\struct{A}}{\struct{B}}$. To obtain $\struct{A} \models \varphi^{n + 1}_{\struct{B}, \vect{b}{m - n - 1}}[\vect{a}{m - n - 1}]$ we have to verify:
\begin{enumerate}[(1)]
\item $\struct{A} \models \forall v_{m - n - 1} \bigvee \setm{\varphi^n_{\struct{B}, \vect{b}{m - n - 1}b}}{b \in B}[\vect{a}{m - n - 1}]$.
%%
\item $\struct{A} \models \bigwedge \setm{\exists v_{m - n - 1} \varphi^n_{\struct{B}, \vect{b}{m - n - 1}b}}{b \in B}[\vect{a}{m - n - 1}]$.
\end{enumerate}
First, for (1) we choose an arbitrary $a \in A$. Then $a \in A^{\tuple{\seq[1]{l}{r}}}$ for some $\seq[1]{l}{r} \in \{ 0, 1 \}$, and
\begin{itemize}
\item if $a \in \dom{p}$, then there is $b^\prime \in \rg{p}$ such that $p(a) = b^\prime$ and hence $\vect{a}{m - n - 1}a \mapsto \vect{b}{m - n - 1}b^\prime = p \in \partism{\struct{A}}{\struct{B}}$. By induction hypothesis, we have $\struct{A} \models \varphi^n_{\struct{B}, \vect{b}{m - n - 1}b^\prime}[\vect{a}{m - n - 1}a]$. So $\struct{A} \models \bigvee \setm{\varphi^n_{\struct{B}, \vect{b}{m - n - 1}b}}{b \in B}[\vect{a}{m - n - 1}a]$;
%%
\item if $a \not\in \dom{p}$, we can choose $b^\prime \in B^{\tuple{\seq[1]{l}{r}}}$ such that $p \cup \{ (a, b^\prime) \} \in \partism{\struct{A}}{\struct{B}}$. In fact, if there are $k$ elements in $\dom{p}$ that are also in $A^{\tuple{\seq[1]{l}{r}}}$, then clearly $k \leq m - n - 1$ or $k + 1 \leq m - n$; moreover, $A^{\tuple{\seq[1]{l}{r}}}$ has at least $k + 1$ elements, and so does $B^{\tuple{\seq[1]{l}{r}}}$. As in the above case, we have 
$\struct{A} \models \bigvee \setm{\varphi^n_{\struct{B}, \vect{b}{m - n - 1}b}}{b \in B}[\vect{a}{m - n - 1}a]$.
\end{itemize}
In either case we have $\struct{A} \models \bigvee \setm{\varphi^n_{\struct{B}, \vect{b}{m - n - 1}b}}{b \in B}[\vect{a}{m - n - 1}a]$. It follows that $\struct{A} \models \forall v_{m - n - 1} \bigvee \setm{\varphi^n_{\struct{B}, \vect{b}{m - n - 1}b}}{b \in B}[\vect{a}{m - n - 1}]$ since $a$ is chosen arbitrarily.\\
\ \\
Next, for (2) we choose an arbitrary $b \in B$. Then $b \in B^{\tuple{\seq[1]{l}{r}}}$ for some $\seq[1]{l}{r} \in \{ 0, 1 \}$, and
\begin{itemize}
\item if $b \in \rg{p}$, then there is $a \in \dom{p}$ such that $p(a) = b$ and hence $\vect{a}{m - n - 1}a \mapsto \vect{b}{m - n - 1}b = p \in \partism{\struct{A}}{\struct{B}}$. By induction hypothesis, we have $\struct{A} \models \varphi^n_{\struct{B}, \vect{b}{m - n - 1}b}[\vect{a}{m - n - 1}a]$. So $\struct{A} \models \exists v_{m - n - 1} \varphi^n_{\struct{B}, \vect{b}{m - n - 1}b}[\vect{a}{m - n - 1}]$;
%%
\item if $b \not\in \rg{p}$, we can choose $a \in A^{\tuple{\seq[1]{l}{r}}}$ such that $p \cup \{ (a, b) \} \in \partism{\struct{A}}{\struct{B}}$ since $B^{\tuple{\seq[1]{l}{r}}} \subset A^{\tuple{\seq[1]{l}{r}}}$. Similar to the above case, we have 
$\struct{A} \models \exists v_{m - n - 1} \varphi^n_{\struct{B}, \vect{b}{m - n - 1}b}[\vect{a}{m - n - 1}]$.
\end{itemize}
In either case we have $\struct{A} \models \exists v_{m - n - 1} \varphi^n_{\struct{B}, \vect{b}{m - n - 1}b}[\vect{a}{m - n - 1}]$. It follows that $\struct{A} \models \bigwedge \setm{\exists v_{m - n - 1} \varphi^n_{\struct{B}, \vect{b}{m - n - 1}b}}{b \in B}[\vect{a}{m - n - 1}]$ since $b$ is chosen arbitrarily.
%
\item \textbf{Solution to Exercise 3.18.} (a) Suppose $\varphi$ is satisfiable and let $\struct{A}$ be an $S$-structure that satisfies $\varphi$. By Exercise 3.17, there is a structure $\struct{B}$ with $\struct{A} \iso_m \struct{B}$ containing at most $m \cdot 2^r$ elements. Since $S$ is finite and relational, by 3.10 we have $\struct{A} \equiv_m \struct{B}$. Therefore $\struct{B}$ is also a model of $\varphi$, and the claim is proved.\\
\ \\
(b) The set of unsatisfiable $S$-sentences is R-decidable: If $\psi$ is an $S$-sentence (the set of $S$-sentences is R-decidable, cf.\ part (b) of Exercise X.1.3), then by systematically checking all $S$-structures containing at most $m \cdot 2^r$ elements (cf.\ the argument in the proof of Lemma X.5.2) we can tell whether $\psi$ is unsatisfiable according to part (a).\\
\ \\
Also, for every $\psi \in \fstordlang[0]{S}$:\\
\centerline{$\psi$ is valid \quad iff \quad $\neg\psi$ is unsatisfiable.}\\
Hence, the set $\sett{\psi}{\(\psi \in \fstordlang[0]{S}\), \(\psi\) valid}$ is R-decidable.
\end{enumerate}
%End of Section XII.3-------------------------------------------------------------
\
\\
\\
%Section XII.4--------------------------------------------------------------------
{\large \S4. Ehrenfeucht Games}
\begin{enumerate}[1.]
\item \textbf{Note on the Proof of Lemma 4.1.} (INCOMPLETE: ``II wins this play'') From definition it immediately follows that for $n \in \nat$, $I^\prime_{n + 1} \subset I^\prime_n$. In fact, if $p \in I^\prime_{n + 1}$ then there is $q \in I_{n + 1}$ such that $p \subset q$. By the forth- (or the back-) property there is $q^\prime \in I_n$ with $q \subset q^\prime$. So $p \subset q^\prime$ and hence $p \in I_n^\prime$.\\
\ \\
Now we show that $(I_n^\prime)_{n \in \nat}: \struct{A} \finiso \struct{B}$ provided that $(I_n)_{n \in \nat}: \struct{A} \finiso \struct{B}$. First note that for $n \in \nat$, $I_n \subset \bigcup_{q \in I_n} \powerset{q} = I_n^\prime$ (where $\powerset{q}$ is the power set of $q$) and hence $\emptyset \neq I_n^\prime \subset \partism{\struct{A}}{\struct{B}}$.\medskip\\
Next, $(I_n^\prime)_{n \in \nat}$ has the forth-property: Let $n \in \nat$, $p \in I^\prime_{n + 1}$ and $a \in A$. Then there is $q \in I_{n + 1}$ such that $p \subset q$. By the forth-property of $(I_n)_{n \in \nat}$, there is a $b \in B$ such that $q \cup \{(a, b)\} \in I_n$. We have $p \cup \{(a, b)\} \in I_n^\prime$ since $p \cup \{(a, b)\} \subset q \cup \{(a, b)\}$.\medskip\\
Finally, $(I_n^\prime)_{n \in \nat}$ has the back-property: Analogously.\\
\ \\
On the other hand, the statement ``II has a winning strategy for the game'' in the last line on page 258 should be replaced by ``II wins this play.'' Since we are describing a winning strategy for player II there, the latter phrase is better for an ending statement.\\
\ \\
Here we verify $(I_n)_{n \in \nat}: \struct{A} \finiso \struct{B}$ (with the premise that player II has a winning strategy in $\egame{\struct{A}}{\struct{B}}$) in the latter part of this proof: First note that by definition we have for $n \in \nat$, $\emptyset \in I_n \subset \partism{\struct{A}}{\struct{B}}$.\medskip\\
Next, $(I_n)_{n \in \nat}$ has the forth-property: Let $n \in \nat$, $p \in I_{n + 1}$ and $a \in A$. By definition there are $j \in \nat$ and $\seq[1]{a}{j} \in A$ with $\dom{p} = \{ \seq[1]{a}{j} \}$ such that there is an $m \geq n + 1$ for a $\egame{\struct{A}}{\struct{B}}$-play in which:
\begin{itemize}
\item player I begins by choosing the number $m + j$;
%
\item player II plays according to his winning strategy;
%
\item in the first $j$ moves the elements $\seq[1]{a}{j} \in A$ and $\seqp{p(a_1)}{p(a_j)} \in B$ are chosen.
\end{itemize}
Since player II has a winning strategy, we may assume, without loss of generality, that $a = a_{j + 1}$ is the element that player I chooses in his $(j + 1)$st move in the above play. Let $b \in B$ be the element that player II chooses in his $(j + 1)$st move in responce to player I. Then $p^\prime \colonequals p \cup \{ (a, b) \} \in \partism{\struct{A}}{\struct{B}}$ because $p^\prime \subset q$, where $q$ is the partial isomorphism obtained when this play is completed. From $m - 1 \geq n$ it follows that $p^\prime \in I_n$.\medskip\\
Finally, that $(I_n)_{n \in \nat}$ has the back-property can be verified analogously.
%
\item \textbf{Solution to Exercise 4.3.} The direction from left to right: Let $(I_n)_{n \leq r}: \struct{A} \iso_r \struct{B}$. Then also $(I_n^\prime)_{n \leq r}: \struct{A} \iso_r \struct{B}$ (where $I_n^\prime$ is defined as in the proof of 4.1 in text).\medskip\\
Since $(I_n^\prime)_{n \leq r}$ has the back- and the forth-property, the winning strategy described in the proof of 4.1 is also one for II in the game $\egamep{r}{\struct{A}}{\struct{B}}$ because a $\egamep{r}{\struct{A}}{\struct{B}}$-play is also a $\egame{\struct{A}}{\struct{B}}$-play.\\
\ \\
The direction from right to left: Suppose that player II has a winning strategy in $\egamep{r}{\struct{A}}{\struct{B}}$. For $n \leq r$ let $I_n$ be defined as in the latter part (for showing $\struct{A} \finiso \struct{B}$ with the premise that player II has a winning strategy in $\egame{\struct{A}}{\struct{B}}$) of the proof of 4.1. From the rules of the game we immediately obtain $(I_n)_{n \leq r}: \struct{A} \iso_r \struct{B}$.
\end{enumerate}
%End of Section XII.4-------------------------------------------------------------
%End of Chapter XII---------------------------------------------------------------
%%Chapter XIII---------------------------------------------------------------------
{\LARGE \bfseries XIII \\ \\ Lindstr\"{o}m's Theorems}
\\
\\
\\
%Section XIII.1-------------------------------------------------------------------
{\large \S1. Logical Systems}
\begin{enumerate}[1.]
%
\item \textbf{Note on the Definition of $\modelclassarg[S]{\logsys}{\varphi}$ in the Second Paragraph on Page 262.} The statement ``$\struct{A} \models \varphi$'' should be replaced by ``$\struct{A} \models_\logsys \varphi$.''
%
\item \textbf{Note on the Examples Given below Definition 1.2.} That $\fstordlog \weakereq \weaksndordlog$ is trivial: Given a symbol set $S$ and $\varphi \in \languagebase_\firstorder(S)$, we have
\begin{inparaenum}[(1)]
%%
\item $\varphi \in \languagebase^\weak_\secondorder(S)$ also
%%
\item $\modelclassarg[S]{\fstordlog}{\varphi} = \modelclassarg[S]{\weaksndordlog}{\varphi}$.
%%
\end{inparaenum}\bigskip\\
For $\weaksndordlog \weakereq \sndordlog$, cf.\ part \reftitle{(b)} of \reftitle{Exercise IX.1.7}.
%
\item \textbf{Note on the Abbreviation $\rel{\logsys}$.} An additional requirement ``$U \not\in S$'' should be added to the premise.
%
\item \textbf{Verifying That $\sndordlog$, $\weaksndordlog$, $\infinlog$ and $\qlog$ Are Regular.} (INCOMPLETE) For $\sndordlog$, $\infinlog$ and $\qlog$ we have verified they are regular (cf. ? in the annotations to \reftitle{Chapter IX}). By similar arguments we have $\weaksndordlog$ is regular.
%
\item \textbf{The Semantic Notions ``Satisfiable'' and ``Valid'' Do Not Depend on a Fixed Symbol Set.} Let $S_0 \subset S_1$ be symbol sets and let $\varphi \in \languagebase(S_0)$ ($\subset \languagebase(S_1)$). Furthermore, let $\logsys$ be a logical system.\medskip\\
Then:\smallskip\\
\begin{tabular}{lll}
\   & $\varphi$ is satisfiable with respect to $S_0$ \cr
iff & $\modelclassarg[S_0]{\logsys}{\varphi} \neq \emptyset$ \cr
iff & there is an $S_0$-structure $\struct{A}^\prime$ such that $\struct{A}^\prime \models_{\logsys} \varphi$ \cr
iff & there is an $S_1$-structure $\struct{A}$ such that $\reduct{\struct{A}}{S_0} \models_{\logsys} \varphi$ \cr
iff & \begin{minipage}[t]{64ex}there is an $S_1$-structure $\struct{A}$ such that $\struct{A} \models_{\logsys} \varphi$\\(by the reduct property of $\logsys$)\end{minipage} \cr
iff & $\modelclassarg[S_1]{\logsys}{\varphi} \neq \emptyset$ \cr
iff & $\varphi$ is satisfiable with respect to $S_1$;
\end{tabular}\smallskip\\
and also:\smallskip\\
\begin{tabular}{ll}
\   & $\varphi$ is valid with respect to $S_0$ \cr
iff & $\modelclassarg[S_0]{\logsys}{\varphi}$ is the class of all $S_0$-structures \cr
iff & for every $S_0$-structure $\struct{A}^\prime$, $\struct{A}^\prime \models_{\logsys} \varphi$ \cr
iff & for every $S_1$-structure $\struct{A}$, $\reduct{\struct{A}}{S_0} \models_{\logsys} \varphi$ \cr
iff & \begin{minipage}[t]{64ex}for every $S_1$-structure $\struct{A}$, $\struct{A} \models_{\logsys} \varphi$\\(by the reduct property of $\logsys$)\end{minipage} \cr
iff & $\modelclassarg[S_1]{\logsys}{\varphi}$ is the class of all $S_1$-structures \cr
iff & $\varphi$ is valid with respect to $S_1$.
\end{tabular}
%
\item \textbf{Solution to Exercise 1.5.}
\begin{asparaenum}[(a)]
%%
\item Below we check the four conditions mentioned in \reftitle{Definition 1.1} for $\logsys$.\medskip\\
For the first condition, let $S_0$ and $S_1$ be two symbol sets with $S_0 \subset S_1$. Suppose $\varphi \in \languagebase(S_0)$, namely $\varphi = \enump{\exists X_1}{\exists X_n}\psi$ is an $\sndordlang{S_0}$-sentence in which $\psi$ does not contain a second-order quantifier. Then $\varphi$ is also an $\sndordlang{S_1}$-sentence (because $\languagebase_\secondorder(S_0) \subset \languagebase_\secondorder(S_1)$) and hence $\varphi \in \languagebase(S_1)$. So $\languagebase(S_0) \subset \languagebase(S_1)$.\medskip\\
For the second condition, notice that by definition $\struct{A}$ and $\varphi$ are related under $\models_\logsys$ if and only if there is an $S$ such that $\varphi \in \languagebase(S)$ and $\struct{A}$ is an $S$-structure with $\struct{A} \models_\sndordlog \varphi$.\medskip\\
For the third condition (namely the isomorphism property), assume $\struct{A} \iso \struct{B}$ are $S$-structures and $\varphi \in \languagebase(S)$. Then we have:\smallskip\\
\begin{tabular}{ll}
\    & $\struct{A} \models_\logsys \varphi$ \cr
iff  & $\struct{A} \models_\sndordlog \varphi$ \cr
then & $\struct{B} \models_\sndordlog \varphi$ \quad (by the isomorphism property of $\sndordlog$) \cr
iff  & $\struct{B} \models_\logsys \varphi$.
\end{tabular}\medskip\\
For the last condition (namely the reduct property), let $S_0 \subset S_1$, $\varphi \in \languagebase(S_0)$ ($\subset \languagebase_\secondorder(S_0)$) and $\struct{A}$ be an $S_1$-structure. Then:\smallskip\\
\begin{tabular}{lll}
\   & $\struct{A} \models_\logsys \varphi$ \cr
iff & $\struct{A} \models_\sndordlog \varphi$ \cr
iff & $\reduct{\struct{A}}{S_0} \models_\sndordlog \varphi$ \quad (by the reduct property of $\sndordlog$) \cr
iff & $\reduct{\struct{A}}{S_0} \models_\logsys \varphi$ \quad ($\reduct{\struct{A}}{S_0}$ is an $S_0$-structure).
\end{tabular}
%%
\item For $\losko{\logsys}$, let $\varphi = \enump{\exists X_1}{\exists X_n} \psi \in \languagebase(S)$ be satisfiable, namely there is an $S$-structure $\struct{A}$ such that $\struct{A} \models_\logsys \varphi$ (and thus $\struct{A} \models_\sndordlog \varphi$). Then for some second-order assignment $\sndordassgn$ in $\struct{A}$, $\intparg{\struct{A}}{\sndordassgn} \models \psi$. Let $\seq[1]{P}{n} \not\in S$ be new relation symbols such that for $1 \leq i \leq n$, $P_i$ and $X_i$ have the same arity. Then we have the following chain of implications:\medskip\\
$\intparg{\struct{A}}{\sndordassgn} \models \psi$;\smallskip\\
$\intparg{(\struct{A}, \seqp{\intpted{P_1}{A}}{\intpted{P_n}{A}})}{\sndordassgn} \models \psi$ where $\intpted{P_i}{A} = \sndordassgn(X_i)$ for $1 \leq i \leq n$ \quad (by the Coincidence Lemma for $\sndordlog$);\medskip\\
$\intparg{(\struct{A}, \seqp{\intpted{P_1}{A}}{\intpted{P_n}{A}})}{\sndordassgn} \models \psi\sbst{\enum[1]{P}{n}}{\enum[1]{X}{n}}$ \quad (by the Substitution Lemma for $\sndordlog$);\medskip\\
$(\struct{A}, \seqp{\intpted{P_1}{A}}{\intpted{P_n}{A}}) \models \psi\sbst{\enum[1]{P}{n}}{\enum[1]{X}{n}}$ \quad ($\psi\sbst{\enum[1]{P}{n}}{\enum[1]{X}{n}}$ is a (first-order) $S$-sentence; by the Coincidence Lemma for $\sndordlog$);\medskip\\
there is an at most countable $(S \cup \{ \seq[1]{P}{n} \})$-structure $\struct{A}^\prime$ such that $\struct{A}^\prime \models \psi\sbst{\enum[1]{P}{n}}{\enum[1]{X}{n}}$ \quad (by the L\"{o}wenheim-Skolem Theorem for $\fstordlog$);\medskip\\
there are an at most countable $(S \cup \{ \seq[1]{P}{n} \})$-structure $\struct{A}^\prime$ and a second-order assignment $\sndordassgn^\prime$ in $\struct{A}^\prime$ such that $\intparg{\struct{A}^\prime}{\sndordassgn^\prime} \models \psi\sbst{\enum[1]{P}{n}}{\enum[1]{X}{n}}$ where $\sndordassgn^\prime(X_i) = \intpted{P_i}{A^\prime}$ for $1 \leq i \leq n$ \quad (by the Coincidence Lemma for $\sndordlog$);\medskip\\
there are an at most countable $(S \cup \{ \seq[1]{P}{n} \})$-structure $\struct{A}^\prime$ and a second-order assignment $\sndordassgn^\prime$ in $\struct{A}^\prime$ such that $\intparg{\struct{A}^\prime}{\sndordassgn^\prime} \models \psi$ where $\sndordassgn^\prime(X_i) = \intpted{P_i}{A^\prime}$ for $1 \leq i \leq n$ \quad (by the Substitution Lemma for $\sndordlog$);\medskip\\
there are an at most countable $(S \cup \{ \seq[1]{P}{n} \})$-structure $\struct{A}^\prime$ and a second-order assignment $\sndordassgn^\prime$ in $\struct{A}^\prime$ such that $\intparg{\struct{A}^\prime}{\sndordassgn^\prime} \models \varphi$ where $\sndordassgn^\prime(X_i) = \intpted{P_i}{A^\prime}$ for $1 \leq i \leq n$;\medskip\\
there is an at most countable $(S \cup \{ \seq[1]{P}{n} \})$-structure $\struct{A}^\prime$ such that $\struct{A}^\prime \models_\sndordlog \varphi$ \quad (by the Coincidence Lemma for $\sndordlog$);\medskip\\
there is an at most countable $S$-structure $\struct{A}^{\prime\prime}$ such that $\struct{A}^{\prime\prime} \models_{\sndordlog} \varphi$ (by the reduct property of $\sndordlog$) and hence $\struct{A}^{\prime\prime} \models_\logsys \varphi$.\\
\ \\
For $\comp{\logsys}$, we first define the operation $^\prime$ on $\languagebase(S)$ as follows: For each $\varphi = \enump{\exists X_1}{\exists X_n} \psi \in \languagebase(S)$ and each relation variable $X_i$ (with $1 \leq i \leq n$) that appears in the prefix $\enump{\exists X_1}{\exists X_n}$ of $\varphi$, we assign a new relation symbol $P^\varphi_i \not\in S$ such that its arity coincides with that of $X_i$; we set
\[
S^\prime \colonequals S \cup \sett{P^\varphi_i}{\(\varphi \in \Phi\) and \(X_i\) appears in the prefix of \(\varphi\)}
\]
and write $\varphi^\prime$ for $\psi\sbst{\enump{P_1^\varphi}{P_n^\varphi}}{\enum[1]{X}{n}}$ (which is a (first-order) $S^\prime$-sentence). Furthermore, for $\Psi \subset \languagebase(S)$, we denote $\Psi^\prime \colonequals \setm{\psi^\prime}{\psi \in \Psi}$.\medskip\\
Next, we show that $\comp{\logsys}$ holds. Let $\Phi \subset \languagebase(S)$, and suppose that every finite subset of $\Phi$ is satisfiable. Then we have the following chain of implications:\medskip\\
Every finite subset of $\Phi^\prime$ is satisfiable \quad (by the Coincidence Lemma and Substitution Lemma for $\sndordlog$);\medskip\\
$\Phi^\prime$ is satisfiable \quad (by the Compactness Theorem for $\fstordlog$);\medskip\\
$\Phi$ is satisfiable \quad (by the Coincidence Lemma and Substitution Lemma for $\sndordlog$).\\
\ \\
Finally, we let $S$ be a given symbol set and $\varphi \in \languagebase(S)$ ($\subset \languagebase_\secondorder(S)$) in the arguments below for showing $\rel{\logsys}$ and $\repl{\logsys}$:\medskip\\
Let $U \not\in S$ be a unary relation symbol. By applying the Relativization Lemma for $\sndordlog$ to $\varphi$ (regarded as an $\languagebase_\secondorder(S)$-sentence), we obtain $\relativize{\varphi}{U}$ (which can easily be shown to be an $\languagebase(S \cup \{U\})$-sentence) and: for all $S$-structures $\struct{A}$ and all $S$-closed subsets $\intpted{U}{A}$ of $A$,\smallskip\\
\begin{tabular}[b]{lll}
$(\struct{A}, \intpted{U}{A}) \models_\logsys \varphi$ & iff & $(\struct{A}, \intpted{U}{A}) \models_{\sndordlog} \varphi$ \cr
\ & iff & $\substr{\intpted{U}{A}}{\struct{A}} \models_{\sndordlog} \relativize{\varphi}{U}$ \cr
\ & iff & $\substr{\intpted{U}{A}}{\struct{A}} \models_\logsys \relativize{\varphi}{U}$.
\end{tabular}\smallskip\\
So we have $\rel{\logsys}$.\medskip\\
On the other hand, let $\relational{S}$ be chosen as for \reftitle{VIII.1.3}. By applying the Theorem on Replacement Operation on $\sndordlog$ to $\varphi$ (regarded as an $\languagebase_\secondorder(S)$-sentence), we obtain $\relational{\varphi}$ (which can easily be shown to be an $\languagebase(\relational{S})$-sentence) and: for all $S$-structures $\struct{A}$,\smallskip\\
\begin{tabular}[b]{lll}
$\struct{A} \models_\logsys \varphi$ & iff & $\struct{A} \models_{\sndordlog} \varphi$ \cr
\ & iff & $\relational{\struct{A}} \models_{\sndordlog} \relational{\varphi}$ \cr
\ & iff & $\relational{\struct{A}} \models_\logsys \relational{\varphi}$.
\end{tabular}\smallskip\\
It follows that $\repl{\logsys}$.
%%
\item Let $S = \{ \formal{\suc}, 0 \}$. Recall Peano's axioms given in \reftitle{III.7.3(2)}:
\begin{compactenum}[(P1)]
%%%
\item $\forall x \neg\formal{\suc}x \equal 0$
%%%
\item $\forall x \forall y (\formal{\suc}x \equal \formal{\suc}y \limply x \equal y)$
\item $\forall X ((X0 \land \forall x (Xx \limply X\formal{\suc}x)) \limply \forall y Xy)$.
%%%
\end{compactenum}
They characterize $\natsuc$ up to isomorphism (cf.\ Dedekind's Theorem \reftitle{III.7.4}).\\
\ \\
If we denote by $\varphi_{\not\iso\natsuc}$ the $\languagebase(S)$-sentence
\[
\begin{array}{l}
\exists X (\exists x \, \formal{\suc}x \equal 0 \lor \cr
\phantom{\exists X (} \exists x \exists y (\formal{\suc}x \equal \formal{\suc}y \land \neg x \equal y) \lor \cr
\phantom{\exists X (} ((X0 \land \forall x (Xx \limply X\formal{\suc}x)) \land \exists x \neg Xx)),
\end{array}
\]
then $\modelclassarg[S]{\logsys}{\varphi_{\not\iso\natsuc}}$ is the class of $S$-structures \emph{not} isomorphic to $\natsuc$.\\
\ \\
We assert $\boole{\logsys}$ does not hold by showing that there is no $\psi \in \languagebase(S)$ with
\[
\begin{array}{lll}
\modelclassarg[S]{\logsys}{\psi} & = & \sett{\struct{A}}{\(\struct{A}\) is an \(S\)-structure such that not \(\struct{A} \models_\logsys \varphi_{\not\iso\natsuc}\)} \cr
\ & = & \sett{\struct{A}}{\(\struct{A}\) is an \(S\)-structure such that \(\struct{A} \iso \natsuc\)}.
\end{array}
\]
Suppose there were an $\languagebase(S)$-sentence $\psi = \enump{\exists X_1}{\exists X_n} \chi$ such that $\modelclassarg[S]{\logsys}{\psi}$ is the class of $S$-structures isomorphic to $\natsuc$; in particular, $\natsuc \models_\logsys \psi$. Let $\seq[1]{P}{n} \not\in S$ be new relation symbols such that $P_i$ and $X_i$ have the same arity for $1 \leq i \leq n$. Then by appropriately choosing subsets $\seqp{\intpted{P_1}{\nat}}{\intpted{P_n}{\nat}}$ of $\nat$, we have $(\natsuc, \seqp{\intpted{P_1}{\nat}}{\intpted{P_n}{\nat}}) \models_\logsys \chi\sbst{\enum[1]{P}{n}}{\enum[1]{X}{n}}$ and hence $(\natsuc, \seqp{\intpted{P_1}{\nat}}{\intpted{P_n}{\nat}}) \models \chi\sbst{\enum[1]{P}{n}}{\enum[1]{X}{n}}$ since $\chi\sbst{\enum[1]{P}{n}}{\enum[1]{X}{n}}$ is an $\languagebase_\firstorder(S \cup \{ \seq[1]{P}{n} \})$-sentence.\\
\ \\
By the Theorem of L\"{o}wenheim, Skolem and Tarski \reftitle{VI.2.4}, we would have, for some $S$-structure $(\real, \intpted{\formal{\suc}}{\real}, \intpted{0}{\real})$ and some subsets $\seqp{\intpted{P_1}{\real}}{\intpted{P_n}{\real}}$ of $\real$, the following chain of implications, which leads to a contradiction:\medskip\\
$(\real, \intpted{\formal{\suc}}{\real}, \intpted{0}{\real}, \seqp{\intpted{P_1}{\real}}{\intpted{P_n}{\real}}) \models \chi\sbst{\enum[1]{P}{n}}{\enum[1]{X}{n}}$;\medskip\\
$(\real, \intpted{\formal{\suc}}{\real}, \intpted{0}{\real}, \seqp{\intpted{P_1}{\real}}{\intpted{P_n}{\real}}) \models_\logsys \psi$ \quad (by the Coincidence Lemma and the Substitution Lemma for $\sndordlog$);\medskip\\
$(\real, \intpted{\formal{\suc}}{\real}, \intpted{0}{\real}) \models_\logsys \psi$ \quad (by the reduct property of $\logsys$);\medskip\\
$(\real, \intpted{\formal{\suc}}{\real}, \intpted{0}{\real}) \iso \natsuc$;\medskip\\
there is a bijective map $\pi : \real \to \nat$.
%%
\item Let $S$ be a symbol set. Then for every $\varphi \in \languagebase_\firstorder(S)$ ($\subset \languagebase(S)$),
\[
\begin{array}{ll}
\ & \modelclassarg{\fstordlog}{\varphi} \cr
= & \sett{\struct{A}}{\(\struct{A}\) is an \(S\)-structure and \(\struct{A} \models \varphi\)} \cr
= & \sett{\struct{A}}{\(\struct{A}\) is an \(S\)-structure and \(\struct{A} \models_{\sndordlog} \varphi\)} \cr
= & \sett{\struct{A}}{\(\struct{A}\) is an \(S\)-structure and \(\struct{A} \models_\logsys \varphi\)} \cr
= & \modelclassarg{\logsys}{\varphi}.
\end{array}
\]
Thus $\fstordlog \weakereq \logsys$.\\
\ \\
Next, we show not $\logsys \weakereq \fstordlog$. We choose $S = \{ \formal{\suc}, 0 \}$ and let $\varphi_{\not\iso\natsuc}$ be defined as in the above part of exercise.\\
\ \\
Suppose, for the sake of contradiction, that $\logsys \weakereq \fstordlog$. Then there would be an $\languagebase_\firstorder(S)$-sentence $\psi$ such that $\modelclassarg{\logsys}{\varphi_{\not\iso\natsuc}} = \modelclassarg{\fstordlog}{\psi}$. Moreover, we would have: for any $S$-structure $\struct{A}$,\smallskip\\
\begin{tabular}{ll}
\   & $\struct{A} \models \neg\psi$ \cr
iff & $\struct{A} \not\in \modelclassarg{\fstordlog}{\psi}$ \cr
iff & $\struct{A} \not\in \modelclassarg{\logsys}{\varphi_{\not\iso\natsuc}}$ \cr
iff & not $\struct{A} \models_\logsys \varphi_{\not\iso\natsuc}$ \cr
iff & $\struct{A} \iso \natsuc$,
\end{tabular}\smallskip\\
i.e.\ the $\languagebase_\firstorder(S)$-sentence $\neg\psi$ characterizes $\natsuc$ up to isomorphism, contrary to \reftitle{Theorem VI.4.3(a)}.
%%
\item Denote by $\varphi_\sndordpeanoarith$ the conjunction of $\languagebase_\secondorder(\arsymb)$-sentences in $\sndordpeanoarith$. Then $\varphi_\sndordpeanoarith$ characterizes $\natstr$ up to isomorphism (cf.\ \reftitle{Exercise III.7.5}). Also, for $\psi \in \languagebase_\firstorder(\arsymb)$ we write $\psi_\natstr$ for the $\languagebase(\arsymb)$-sentence
\[
\begin{array}{l}
\exists X ( \exists x \, x + 1 \equal 0 \lor\cr
\phantom{\exists X (} \exists x \exists y (x + 1 \equal y + 1 \land \neg x \equal y) \lor\cr
\phantom{\exists X (} ((X0 \land \forall x (Xx \limply Xx + 1)) \land \exists y \neg Xy) \lor\cr
\phantom{\exists X (} \exists x \neg x + 0 \equal x \lor\cr
\phantom{\exists X (} \exists x \exists y \neg x + (y + 1) \equal (x + y) + 1 \lor\cr
\phantom{\exists X (} \exists x \neg x \cdot 0 \equal 0 \lor\cr
\phantom{\exists X (} \exists x \exists y \neg x \cdot (y + 1) \equal (x \cdot y) + x \lor\cr
\phantom{\exists X (} \psi),
\end{array}
\]
which is, in the sense of $\models_{\sndordlog}$, logically equivalent to $(\varphi_\sndordpeanoarith \limply \psi)$; obviously, we have\smallskip\\
\begin{tabular}[b]{ll}
\   & $\psi \in \theoarg{\natstr}$ \cr
iff & $\natstr \models \psi$ \cr
iff & $(\varphi_\sndordpeanoarith \limply \psi)$ is valid in the sense of $\models_{\sndordlog}$\cr
iff & $\psi_\natstr$ is valid in the sense of $\models_\logsys$.
\end{tabular}\bigskip\\
It then turns out that the set of valid $\languagebase(\arsymb)$-sentences is not enumerable: If it were, then the set\\
\centerline{$\sett{\psi \in \languagebase_\firstorder(\arsymb)}{\(\psi_\natstr\) is valid} = \theoarg{\natstr}$}\\
would be enumerable as well, contrary to \reftitle{Corollary X.6.10}.
%%
\end{asparaenum}
%
\item \textbf{Solution to Exercise 1.6.} We immediately obtain $\qlog \weakereq \sndordlog$ from:\medskip\\
\begin{theorem}{Claim}
Let $S$ be a symbol set. If $\struct{A}$ is an $S$-structure, then for every (ordinary) assignment $\assgn$ and every second-order assignment $\sndordassgn$ in $\struct{A}$ such that $\assgn(v_n) = \sndordassgn(v_n)$ for $n \in \nat$, and for every \emph{formula} $\varphi \in \qlang{S}$, there is a \emph{formula} $\psi \in \sndordlang{S}$ such that\\
\centerline{$\intparg{\struct{A}}{\assgn} \models \varphi$ \quad iff \quad $\intparg{\struct{A}}{\sndordassgn} \models \psi$.}
\end{theorem}
\begin{proof}
Define the operation $^\prime$ on $\qlang{S}$ inductively as follows:\smallskip\\
\begin{tabular}[b]{lll}
$\varphi^\prime$ & $\colonequals$ & $\varphi$ \quad if $\varphi$ is atomic \cr
$(\neg\varphi)^\prime$ & $\colonequals$ & $\neg\varphi^\prime$ \cr
$(\varphi\lor\psi)^\prime$ & $\colonequals$ & $(\varphi^\prime \lor \psi^\prime)$ \cr
$(\exists x \varphi)^\prime$ & $\colonequals$ & $\exists x \varphi^\prime$ \cr
$(\qexist x \varphi)^\prime$ & $\colonequals$ & $\exists X (\neg\chi_{\leq\mathrm{ctbl}}(X) \land \forall x(Xx \limply \varphi^\prime))$,
\end{tabular}\smallskip\\
where $\chi_{\leq\mathrm{ctbl}}(X)$ formulates that $\sndordassgn(X)$ is an at most countable subset of $A$ (cf.\ \textbf{Note on the Discussions Concerning the Second-Order Logic and the Continuum Hypothesis on Pages 141 and 142} in notes to \reftitle{Chapter IX}).\medskip\\
For each $\varphi \in \qlang{S}$, choose $\psi = \varphi^\prime$. We prove the claim by induction on $\varphi$ (let an $S$-structure $\struct{A}$ be fixed):\smallskip\\
$\varphi$ is atomic: It follows immediately from the premise $\assgn(v_n) = \sndordassgn(v_n)$ for $n \in \nat$.\medskip\\
$\neg\varphi$: $\intparg{\struct{A}}{\assgn} \models \neg\varphi$\smallskip\\
\begin{tabular}[b]{ll}
iff & not $\intparg{\struct{A}}{\assgn} \models \varphi$ \cr
iff & not $\intparg{\struct{A}}{\sndordassgn} \models \varphi^\prime$ \quad (by induction hypothesis) \cr
iff & $\intparg{\struct{A}}{\sndordassgn} \models \neg\varphi^\prime$.
\end{tabular}\medskip\\
$(\varphi\lor\psi)$: Similarly.\medskip\\
$\exists x \varphi$: $\intparg{\struct{A}}{\assgn} \models \exists x \varphi$\smallskip\\
\begin{tabular}[b]{ll}
iff & there is an $a \in A$ such that $\intparg{\struct{A}}{\assgn\sbst{a}{x}} \models \varphi$ \cr
iff & \begin{minipage}[t]{48ex}there is an $a \in A$ such that $\intparg{\struct{A}}{\sndordassgn\sbst{a}{x}} \models \varphi$\\(by induction hypothesis)\end{minipage} \cr
iff & $\intparg{\struct{A}}{\sndordassgn} \models \exists x \varphi$.
\end{tabular}\medskip\\
$\qexist x \varphi$: $\intparg{\struct{A}}{\assgn} \models \qexist x \varphi$\smallskip\\
\begin{tabular}[b]{ll}
iff & $\setm{a \in A}{\intparg{\struct{A}}{\assgn\sbst{a}{x}} \models \varphi}$ is uncountable \cr
iff & \begin{minipage}[t]{50ex}$\setm{a \in A}{\intparg{\struct{A}}{\sndordassgn\sbst{a}{x}} \models \varphi}$ is uncountable\\($\setm{a \in A}{\intparg{\struct{A}}{\assgn\sbst{a}{x}} \models \varphi} = \setm{a \in A}{\intparg{\struct{A}}{\sndordassgn\sbst{a}{x}} \models \varphi}$ by induction hypothesis)\end{minipage} \cr
iff & $\intparg{\struct{A}}{\sndordassgn} \models \qexist x \varphi$.
\end{tabular}
\end{proof}
Next, the following argument shows that not $\weaksndordlog \weakereq \qlog$: Notice that $\modelclassarg[\emptyset]{\weaksndordlog}{\exists X \forall x \, Xx}$ is the class of $\emptyset$-structures of finite domain. There is no $\languagebase_\qexist(\emptyset)$-sentence $\varphi$ such that $\modelclassarg[\emptyset]{\weaksndordlog}{\exists X \forall x \, Xx} = \modelclassarg[\emptyset]{\qlog}{\varphi}$: Suppose there were such an $\languagebase_\qexist(\emptyset)$-sentence $\varphi$, then we would have that the set
\[
\Phi \colonequals \{\varphi\} \cup \setm{\varphi_{\geq n}}{n \geq 2}
\]
is not satisfiable while every finite subset $\Phi_0 \subset \Phi$ is satisfiable, contrary to \reftitle{IX.3.2}.\\
\ \\
Finally, we show that not $\qlog \weakereq \weaksndordlog$. Note that $\modelclassarg[\emptyset]{\qlog}{\qexist x \, x \equal x}$ is the class of $\emptyset$-structures of uncountable domain.\medskip\\
Since the \reftitle{L\"{o}wenheim-Skolem Theorem} holds for $\weaksndordlog$ (cf.\ \reftitle{Exercise IX.2.7}), we have that every satisfiable $\languagebase^\weak_\secondorder(\emptyset)$-sentence has a model over an at most countable domain; hence, there is no $\languagebase^\weak_\secondorder(\emptyset)$-sentence $\psi$ such that $\modelclassarg[\emptyset]{\qlog}{\qexist x \, x \equal x} = \modelclassarg[\emptyset]{\weaksndordlog}{\psi}$.
%
\end{enumerate}
%End of Section XIII.1------------------------------------------------------------
\
\\
\\
%Section XIII.2-------------------------------------------------------------------
{\large \S2. Compact Regular Logical Systems}
\begin{enumerate}[1.]
%
\item \textbf{Note on Lemma 2.1.} In fact, the Compactness Theorem for the consequence relation and that for satisfaction are equivalent; assuming the former, the latter holds as well. We state and prove the converse of this lemma:\medskip\\
\begin{theorem}{Lemma}
If for $\Phi \cup \{ \varphi \} \subset \languagebase(S)$ there is a finite subset $\Phi_0$ of $\Phi$ such that $\Phi_0 \models_\logsys \varphi$ whenever $\Phi \models_\logsys \varphi$. Then $\comp{\logsys}$.
\end{theorem}
\begin{proof}
Suppose $\Phi \subset \languagebase(S)$ is not satisfiable. Then for an arbitrary $\varphi \in \languagebase(S)$, $(\varphi \land \neg\varphi) \in \languagebase(S)$ by $\boole{\logsys}$ and moreover $\Phi \models_\logsys \varphi \land \neg\varphi$. By the premise, there is a finite subset $\Phi_0$ of $\Phi$ such that $\Phi_0 \models_\logsys \varphi \land \neg\varphi$, i.e.\ $\Phi_0$ is not satisfiable.
\end{proof}
%
\item \textbf{Note on the Proof of Lemma 2.2.}
\begin{asparaenum}[(1)]
%%
\item The sentence $\exists x Vx$ is redundant for $\Phi$ because\smallskip\\
\centerline{$\{ \exists x Ux, \forall x (Ux \limply Vfx) \} \models \exists x Vx$.}\smallskip\\
(However, $\exists y Vy$ in (viii) of the conjunction $\chi$ on page 268 is necessary because it is not a consequence of any other conjunct.)
%%
\item The passage from the third last line (on page 266) of the proof\smallskip\\
\centerline{$(\struct{C}, \intpted{U}{C}, \intpted{V}{C}, \intpted{f}{C}) \models_\logsys \relativize{\psi}{U} \liff \relativize{\psi}{V}$}
to the last line\smallskip\\
\centerline{$\struct{A} \models_\logsys \psi$ \quad iff \quad $\struct{B} \models_\logsys \psi$}
can be verified in the same manner as ``$\Phi^\ast \models_\logsys \relativize{\psi}{U} \liff \relativize{\psi}{V}$'' is verified in text:\medskip\\
\begin{tabular}[b]{lll}
\   & $\struct{A} \models_\logsys \psi$, i.e.\ $\substr{\intpted{U}{C}}{\struct{C}} \models_\logsys \psi$ & \cr
iff & $(\struct{C}, \intpted{U}{C}) \models_\logsys \relativize{\psi}{U}$ & (by $\rel{\logsys}$) \cr
iff & $(\struct{C}, \intpted{U}{C}, \intpted{V}{C}, \intpted{f}{C}) \models_\logsys \relativize{\psi}{U}$ & (by the reduct property of $\logsys$) \cr
iff & $(\struct{C}, \intpted{U}{C}, \intpted{V}{C}, \intpted{f}{C}) \models_\logsys \relativize{\psi}{V}$ & (since $(\struct{C}, \intpted{U}{C}, \intpted{V}{C}, \intpted{f}{C}) \models_\logsys \relativize{\psi}{U} \liff \relativize{\psi}{V}$) \cr
iff & $(\struct{C}, \intpted{V}{C}) \models_\logsys \relativize{\psi}{V}$ & (by the reduct property of $\logsys$) \cr
iff & $\substr{\intpted{V}{C}}{\struct{C}} \models_\logsys \psi$, i.e.\ $\struct{B} \models_\logsys \psi$ & (by $\rel{\logsys}$).
\end{tabular}
%%
\item Here we extend this proof to \emph{arbitrary} symbol sets $S$. In the following, let $S$ be fixed.\\
\ \\
Again, we assume $\comp{\logsys}$ and let $\psi \in \languagebase(S)$. By $\repl{\logsys}$ (note that $\logsys$ is assumed to be regular), $\relational{\psi}$ is an $\languagebase(\relational{S})$-sentence such that\smallskip\\
\begin{quoteno}{($+$)}
$\struct{A}_+ \models_\logsys \psi$ \quad iff \quad $\relational{\struct{A}_+} \models_\logsys \relational{\psi}$
\end{quoteno}\smallskip\\
for all $S$-structures $\struct{A}_+$.\\
\ \\
Given $\relational{S}$ is relational and $\relational{\psi} \in \languagebase(\relational{S})$, this proof yields a finite subset $S_0$ of $S$ such that $\relational{S_0}$ is a finite subset of $\relational{S}$ and for all $\relational{S}$-structures $\struct{A}_\ast$ and $\struct{B}_\ast$:\smallskip\\
\begin{quoteno}{($\ast$)}
If $\reduct{\struct{A}_\ast}{\relational{S_0}} \iso \reduct{\struct{B}_\ast}{\relational{S_0}}$ then ($\struct{A}_\ast \models_\logsys \relational{\psi}$ \quad iff \quad $\struct{B}_\ast \models_\logsys \relational{\psi}$).
\end{quoteno}\\
\ \\
Now, suppose that $\struct{A}$ and $\struct{B}$ are two arbitrary $S$-structures with\smallskip\\
\centerline{$\reduct{\struct{A}}{S_0} \iso \reduct{\struct{B}}{S_0}$.}\smallskip\\
Since $\relational{(\reduct{\struct{A}}{S_0})} = \reduct{\relational{\struct{A}}}{\relational{S_0}}$ and $\relational{(\reduct{\struct{B}}{S_0})} = \reduct{\relational{\struct{B}}}{\relational{S_0}}$, we immediately have\smallskip\\
\centerline{$\reduct{\relational{\struct{A}}}{\relational{S_0}} \iso \reduct{\relational{\struct{B}}}{\relational{S_0}}$}\smallskip\\
(cf.\ \textbf{A Parallel to Corollary 1.4} in annotations to \reftitle{Chapter VIII}). Because $\relational{\struct{A}}$ and $\relational{\struct{B}}$ are $\relational{S}$-structures, ($\ast$) then yields\smallskip\\
\begin{quoteno}{($\circ$)}
$\relational{\struct{A}} \models_\logsys \relational{\psi}$ \quad iff \quad $\relational{\struct{B}} \models_\logsys \relational{\psi}$.
\end{quoteno}\\
\ \\
So we have: $\struct{A} \models_\logsys \psi$\smallskip\\
\begin{tabular}[b]{lll}
iff & $\relational{\struct{A}} \models_\logsys \relational{\psi}$ & (by ($+$)) \cr
iff & $\relational{\struct{B}} \models_\logsys \relational{\psi}$ & (by ($\circ$)) \cr
iff & $\struct{B} \models_\logsys \psi$ & (by ($+$)).
\end{tabular}\\
Hence, \reftitle{Lemma 2.2} is valid for arbitrary symbol sets.
%%
\end{asparaenum}
%
\end{enumerate}
%End of Section XIII.2------------------------------------------------------------
\
\\
\\
%Section XIII.3-------------------------------------------------------------------
{\large \S3. Lindstr\"{o}m's First Theorem}
\begin{enumerate}[1.]
%
\item \textbf{Note on the Introductory Paragraph in This Section.} If $\struct{A}$ and $\struct{B}$ are two $S$-structures such that $S_0 \subset S$ is finite and\smallskip\\
\centerline{$\reduct{\struct{A}}{S_0} \iso_m \reduct{\struct{B}}{S_0}$}\smallskip\\
for some large $m \in \nat$, then we say \emph{$\struct{A}$ and $\struct{B}$ are nearly identical with respect to the first-order language $\fstordlang{S_0}$}.
%
\item \textbf{Note on the Proof of Lemma 3.1.} Below we give some remarks:
\begin{asparaenum}[(1)]
%%
\item Here $m$ is, without loss of generality, assumed to be $\geq 1$, because\smallskip\\
\centerline{for $m \geq 1$: \quad if $\struct{A} \iso_m \struct{B}$ then $\struct{A} \iso_0 \struct{B}$.}\smallskip\\
Thus, whenever we are given finite $S_0 \subset S$ and $m = 0$ and are asked to provide two $S$-structures $\struct{A}$ and $\struct{B}$ such that (+) in the lemma holds, we just apply this lemma to $S_0$ and ``$m = 1$'' to obtain $\struct{A}$ and $\struct{B}$; the statement (+) in which $m = 0$ is also valid for them.
%%
\item
$\begin{array}[t]{ll}
\       & \sett{\varphi^m_{\reduct{\struct{A}}{S_0}}}{\(\struct{A}\) is an \(S\)-structure and \(\struct{A} \models_\logsys \psi\)} \cr
=       & \{ \varphi^m_{\struct{A}^\prime} \mid \mbox{\(\struct{A}^\prime\) is an \(S_0\)-structure and there is an \(S\)-expansion \(\struct{A}\) of \(\struct{A}^\prime\)} \cr
\       & \phantom{\{ \varphi^m_{\struct{A}^\prime} \mid \ } \mbox{such that \(\struct{A} \models_\logsys \psi\)}\} \cr
\subset & \sett{\varphi^m_{\struct{A}^\prime}}{\(\struct{A}^\prime\) is an \(S_0\)-structure},
\end{array}$\smallskip\\
so it is finite; it is also nonempty because there is an $S$-structure $\struct{A}$ with $\struct{A} \models_\logsys \psi$ (otherwise $\psi$ would be logically equivalent to the first-order $S$-sentence $\exists x \, \neg x \equal x$). Hence $\varphi$ is a first-order $S_0$-sentence.
%%
\item We verify that $(\psi \limply \varphi^\ast)$ is valid (in the sense of $\models_\logsys$). Let $\struct{A}$ be an $S$-structure with $\struct{A} \models_\logsys \psi$. Then $(\varphi^m_{\reduct{\struct{A}}{S_0}} \limply \varphi)$ is valid, by definition of $\varphi$. On the other hand, by the reduct property we have $\struct{A} \models \varphi^m_{\reduct{\struct{A}}{S_0}}$ since $\reduct{\struct{A}}{S_0} \models \varphi^m_{\reduct{\struct{A}}{S_0}}$ (cf.\ \reftitle{XII.3.5(b)}). So $\struct{A} \models \varphi$, and hence $\struct{A} \models_\logsys \varphi^\ast$.
%%
\item Because $(\psi \limply \varphi^\ast)$ is valid, we have either
\begin{inparaenum}[(a)]
%%%
\item $\modelclassarg[S]{\logsys}{\psi} = \modelclassarg[S]{\logsys}{\varphi^\ast}$ or
%%%
\item $\modelclassarg[S]{\logsys}{\psi} \subsetneq \modelclassarg[S]{\logsys}{\varphi^\ast}$;
%%%
\end{inparaenum}
by the premise the former is excluded.
%%
\item By the reduct property, it follows from $\struct{B} \models \varphi^m_{\reduct{\struct{A}}{S_0}}$ that $\reduct{\struct{B}}{S_0} \models \varphi^m_{\reduct{\struct{A}}{S_0}}$. Therefore $\reduct{\struct{A}}{S_0} \iso_m \reduct{\struct{B}}{S_0}$ by \reftitle{Exercise 3.10}.
%%
\item Using the results of \reftitle{Exercise XII.3.15} in the proof, we immediately obtain the generalization of this lemma to arbitrary symbol sets.
%%
\end{asparaenum}
%
\item \textbf{Note on the Conjunction $\chi$ on Page 268.} (INCOMPLETE)
\begin{asparaenum}[(1)]
%%
\item It states ``there is a (finite or infinite) descending chain
\[
\ldots < c - 1 < c
\]
of length $\geq 1$ such that
\begin{inparaenum}[(a)]
%%%
\item for every element $x$ in this chain, $I_x$ is a nonempty set of partial isomorphisms from the $S_0$-substructure induced on $U$ to the $S_0$-substructure induced on $V$;
%%%
\item for every two elements $x$ and $y$ in this chain, if $y = fx$ (namely $y$ is the predecessor of $x$) then both the back- and the forth-property hold for the passage from $I_x$ to $I_y$.''\footnote{In other words, for every $p \in I_x$ and every $a \in U$, there is a $q \in I_y$ such that $p \subset q$ and $a \in \dom{q}$; this is the forth-property. Analogously for the back-property.}
%%%
\end{inparaenum}\bigskip\\
So $\chi$ does not depend on $m$ in (+) on page 266, for it does not specify the length of the descending chain.
%%
\item In (iv), ``The axioms of $\Phi_\pord$'' should be replaced by ``The axioms of $\Phi_\pord$ except $\exists x \exists y \, x < y$'': Since the relation $<$ may be empty (a case which corresponds to $m = 0$), which is inconsistent with that sentence.\bigskip\\
If all the sentences in (iv) hold, then $<$ is a (linear) ordering over $W$.
%%
\item In (v), the outermost right parenthesis ``$)$'' is missing: ``$\forall x (Wx \limply \exists p (Pp \land Ixp)$'' should be replaced by ``$\forall x (Wx \limply \exists p (Pp \land Ixp))$''.\bigskip\\
On the other hand, an additional conjunct\smallskip\\
\centerline{$\forall x \forall p (Ixp \limply (Wx \land Pp))$}\smallskip\\
should be added (which states $I$ is a binary relation between $W$ and $P$ or, more informally, $I_x^\prime = \sett{p}{\(Ixp\) holds}$ is a subset of $P$), otherwise there is no guarantee that $p$ is a partial isomorphism from the $S_0$-substructure induced on $U$ to the $S_0$-substructure induced on $V$ even if $Ixp$ holds for some $x$ in the field of $<$.

%%
\item A sentence for the back-property mentioned in (vii) of the conjunction $\chi$ is\smallskip\\
$\begin{array}{l}
\forall x \forall p \forall v ((fx < x \land Ixp \land Vv) \limply \cr
\multicolumn{1}{r}{\phantom{aaaaaaaaaaaaaaaa}\exists q \exists u (Ifxq \land Gquv \land \forall x^\prime \forall y^\prime (Gx^\prime y^\prime \limply Gqx^\prime y^\prime))).}
\end{array}$
\end{asparaenum}
%
\item \textbf{Note on the Proof of 3.3.} (INCOMPLETE) It is possible for a total ordering $<$ over a set $M$ to have an infinite descending chain, a greatest and a least element at the same time, with every element except the least having an immediate predecessor. For example, if we take
\[
M \colonequals \{ -\infty \} \cup \setm{n \in \zah}{n \leq 0},
\]
$<$ the usual ordering relation, then $0$ is the greatest element while $-\infty$ the least, $<$ has an infinite descending chain
\[
\ldots < -2 < -1 < 0,
\]
and every element except $-\infty$ has an immediate predecessor: $n - 1$ is the immediate predecessor of $n$ for $n \leq 0$.\bigskip\\
In the proof of 3.3, given a model $\struct{D}$ of $\chi$ we chose the $\languagebase(S^+ \cup \{ Q \})$-sentence
\[
\vartheta \colonequals Qc \land \forall x (Qx \limply (fx < x \land Qfx))
\]
to ensure a coutable model of $\chi$ obtained from $\struct{D}$ with infinite field of $<$ (clearly $Q \subset W$). One might think that it also works by setting $\vartheta$ to
\[
\forall x (Wx \limply \exists y \, y < x)
\]
(the field $W$ of $<$ does not have a least element). As we noted above, however, that could be incompatible with $\struct{D}$: $\intpted{W}{D}$ might look like $M$ above, which does have an infinite descending chain and a least element.
%
\item \textbf{Note on Main Lemma 3.4.} (INCOMPLETE) Part (a) of this lemma corresponds to 3.3(a); (i) of part (b) correponds to 3.3(b) (if $\struct{C}$ is a model of $\chi$, then clearly it satisfies the conjunct (iv), hence $\intpted{W}{C}$ is nonempty); (ii) of part (b) corresponds to 3.2.\bigskip\\
Observe that $(\forall x \varphi \lor \exists x \psi)$ is a consequence of $\forall x (\varphi \lor \psi)$:\smallskip\\
\begin{derivation}
1. & $\forall x (\varphi \lor \psi)$ & $\forall x (\varphi \lor \psi)$ & $\assm$ \cr
2. & $\forall x (\varphi \lor \psi)$ & $(\varphi \lor \psi)$ & IV.5.5(a2) applied to 1. \cr
3. & $\forall x (\varphi \lor \psi)$ $\varphi$ & $\varphi$ & $\assm$ \cr
4. & $\forall x (\varphi \lor \psi)$ $\psi$ & $\psi$ & $\assm$ \cr
5. & $\forall x (\varphi \lor \psi)$ $\varphi$ & $(\exists x \psi \lor \varphi)$ & $\ors$ applied to 3. \cr
6. & $\forall x (\varphi \lor \psi)$ $\psi$ & $\exists x \psi$ & IV.5.1(a) applied to 4. \cr
7. & $\forall x (\varphi \lor \psi)$ $\psi$ & $(\exists x \psi \lor \varphi)$ & $\ors$ applied to 6. \cr
8. & $\forall x (\varphi \lor \psi)$ $(\varphi \lor \psi)$ & $(\exists x \psi \lor \varphi)$ & $\ora$ applied to 5. and 7. \cr
9. & $\forall x (\varphi \lor \psi)$ & $(\exists x \psi \lor \varphi)$ & (Ch) applied to 2. and 8. \cr
10. & $\forall x (\varphi \lor \psi)$ $\exists x \psi$ & $\exists x \psi$ & $\assm$ \cr
11. & $\forall x (\varphi \lor \psi)$ $\exists x \psi$ & $(\forall x \varphi \lor \exists x \psi)$ & $\ors$ applied to 10. \cr
12. & $\forall x (\varphi \lor \psi)$ $\neg\exists x \psi$ & $(\exists x \psi \lor \varphi)$ & $\ant$ applied to 9. \cr
13. & $\forall x (\varphi \lor \psi)$ $\neg\exists x \psi$ & $\neg\exists x \psi$ & $\assm$ \cr
14. & $\forall x (\varphi \lor \psi)$ $\neg\exists x \psi$ & $\varphi$ & IV.3.4 applied to 12. and 13. \cr
15. & $\forall x (\varphi \lor \psi)$ $\neg\exists x \psi$ & $\forall x \varphi$ & IV.5.5(b4) applied to 14. \cr
16. & $\forall x (\varphi \lor \psi)$ $\exists x \psi$ & $(\forall x \varphi \lor \exists x \psi)$ & $\ors$ applied to 10. \cr
17. & $\forall x (\varphi \lor \psi)$ $\neg\exists x \psi$ & $(\forall x \varphi \lor \exists x \psi)$ & $\ors$ applied to 15. \cr
18. & $\forall x (\varphi \lor \psi)$ & $(\forall x \varphi \lor \exists x \psi)$ & $\pc$ applied to 16. and 17.
\end{derivation}\smallskip\\
With this observation, we explain in detail how we summarize considerations of 3.2 and 3.3 leading to 3.4. These considerations togerther yield, successively:\medskip\\
``For all finite $S_0 \subset S$, (a) or (b) holds:
\begin{compactenum}[(a)]
%%
\item There are $S$-structures $\struct{A}$ and $\struct{B}$ such that\smallskip\\
\centerline{$\struct{A} \models_\logsys \psi$, $\struct{B} \models_\logsys \neg\psi$ and $\reduct{\struct{A}}{S_0} \iso \reduct{\struct{B}}{S_0}$.}
%%
\item For the $\languagebase(S_0 \cup \{ c, f, P, U, V, W, <, I, G \})$-sentence $\chi$ defined on page 268 (dependent on $S_0$ but not on $m$),
\begin{compactenum}[(i)]
%%%
\item In every model $\struct{C}$ of $\chi$, $\intpted{W}{C}$ is finite and nonempty.
%%%
\item For every $m \geq 1$ there is a model $\struct{C}$ of $\chi$, in which $\intpted{W}{C}$ has exactly $m$ elements.''
%%%
\end{compactenum}
%%
\end{compactenum}
``One of the following conditions (a) or (b) holds:
\begin{compactenum}[(a)]
%%
\item For all finite $S_0 \subset S$, there are $S$-structures $\struct{A}$ and $\struct{B}$ such that\smallskip\\
\centerline{$\struct{A} \models_\logsys \psi$, $\struct{B} \models_\logsys \neg\psi$ and $\reduct{\struct{A}}{S_0} \iso \reduct{\struct{B}}{S_0}$.}
%%
\item There is a finite $S_0 \subset S$ such that $\chi$ is an $\languagebase(S_0 \cup \{ c, f, P, U, V, W, <, I, G \})$-sentence and
\begin{compactenum}[(i)]
%%%
\item In every model $\struct{C}$ of $\chi$, $\intpted{W}{C}$ is finite and nonempty.
%%%
\item For every $m \geq 1$ there is a model $\struct{C}$ of $\chi$, in which $\intpted{W}{C}$ has exactly $m$ elements.''
%%%
\end{compactenum}
%%
\end{compactenum}
``One of the following conditions (a) or (b) holds:
\begin{compactenum}[(a)]
%%
\item For all finite symbol sets $S_0$ with $S_0 \subset S$ there are $S$-structures $\struct{A}$ and $\struct{B}$ such that\smallskip\\
\centerline{$\struct{A} \models_\logsys \psi$, $\struct{B} \models_\logsys \neg\psi$ and $\reduct{\struct{A}}{S_0} \iso \reduct{\struct{B}}{S_0}$.}
%%
\item For a unary relation symbol $W$ and a suitable symbol set $S^+$ with $S \cup \{W\} \subset S^+$ and finite $S^+ \setminus S$, there is an $\languagebase(S^+)$-sentence $\chi$ such that
\begin{compactenum}[(i)]
%%%
\item In every model $\struct{C}$ of $\chi$, $\intpted{W}{C}$ is finite and nonempty.
%%%
\item For every $m \geq 1$ there is a model $\struct{C}$ of $\chi$, in which $\intpted{W}{C}$ has exactly $m$ elements.''
%%%
\end{compactenum}
%%
\end{compactenum}

%
\item \textbf{Note on the Proof of Lindstr\"{o}m's First Theorem.} For $n \geq 1$, the statement ``$W$ contains at least $n$ elements'' can be formulated by
\[
\varphi_{\card{W} \geq n} \colonequals \enump{\exists x_1}{\exists x_n} \left( \bigwedge_{1 \leq i < j \leq n} \neg x_i \equal x_j \land \bigwedge_{1 \leq i \leq n} Wx_i \right).
\]
For the statement ``$W$ contains at least $0$ elements'' we take $\varphi_{\card{W} \geq 0} \colonequals \exists x \, x \equal x$.\bigskip\\
Actually, for a contradiction to $\comp{\logsys}$, we could have chosen the set\smallskip\\
\centerline{$\{ \chi \} \cup \setm{\mbox{``\(W\) contains at least \(n\) elements''}}{n > 0}$,}\smallskip\\
thus avoiding the trivial case $n = 0$.\bigskip\\
In summary, the structure of the proof of 3.5 is illustrated by
\[
\begin{CD}
\fbox{\parbox{27ex}{Premise:
\begin{compactenum}
%%
\item \(\logsys\) is regular
%%
\item \(\fstordlog \weakereq \logsys\)
%%
\item \(\comp{\logsys}\)
%%
\item \(\losko{\logsys}\)
%%
\item \(\modelclassarg[S]{\logsys}{\psi} \neq \modelclassarg[S]{\fstordlog}{\varphi}\) for all \(\varphi \in \languagebase_\firstorder(S)\)
%%
\end{compactenum}}} @. \ \\
@VV{\parbox{20ex}{Using \reftitle{1.4}}}V @. \ \\
\fbox{\parbox{27ex}{\(\logsys\) is regular,\\
\(\fstordlog \weakereq \logsys\),\\
\(\comp{\logsys}\),\\
\(\losko{\logsys}\), and\\
there is a relational \(S\) such that \(\psi \in \languagebase(S)\) and \(\modelclassarg[S]{\logsys}{\psi} \neq \modelclassarg[S]{\fstordlog}{\varphi}\) for all \(\varphi \in \languagebase_\firstorder(S)\)}}
@>>{\parbox{20ex}{Using \reftitle{2.2},
a result of \(\comp{\logsys}\)}}>
\fbox{\parbox{27ex}{For some finite \(S_0 \subset S\):\\
If \(\reduct{\struct{A}}{S_0} \iso \reduct{\struct{B}}{S_0}\), then\\
(\(\struct{A} \models_\logsys \psi\) iff \(\struct{B} \models_\logsys \psi\))\\
for all \(S\)-structures \(\struct{A}\), \(\struct{B}\)}}\\
@VV{\mbox{Using \reftitle{3.1}}}V  @V{\parbox{20ex}{Using 3.3, a result of \(\losko{\logsys}\)}}VV\\
\fbox{\parbox{27ex}{For finite \(S_0 \subset S\), \(m \in \nat\):\\
\(\struct{A} \models_\logsys \psi\),\\
\(\struct{B} \models_\logsys \neg\psi\), and\\
\(\reduct{\struct{A}}{S_0} \iso_m \reduct{\struct{B}}{S_0}\)\\
for some \(S\)-structures \(\struct{A}\), \(\struct{B}\)\smallskip\\
(\(\chi\) is satisfiable for \(m \in \nat\))}} @. \fbox{\parbox{27ex}{For some finite \(S_0 \subset S\):\\
\(\{\chi\} \cup \setm{\mbox{``\(\absval{W} \geq n\)''}}{n \in \nat}\) is not satisfiable}}\\
@VVV @VVV\\
\fbox{\parbox{27ex}{For finite \(S_0 \subset S\):\\
Every finite subset of\\
\(\{\chi\} \cup \setm{\mbox{``\(\absval{W} \geq n\)''}}{n \in \nat}\)\\
is satisfiable}}
@>>>
\parbox{20ex}{CONTRADICTION TO \(\comp{\logsys}\)}\\
\end{CD}
\]
%
\item \textbf{Note on the Discussion for Clarifying the Role of the Conditions $\losko{\logsys}$ and $\comp{\logsys}$ in Lindstr\"{o}m's First Theorem on Page 271.} (INCOMPLETE)\\The diagram given below depicts this discussion:
\[
\begin{CD}
\fbox{\parbox{27ex}{Premise:
\begin{compactenum}
%%
\item \(\logsys\) is regular
%%
\item \(\fstordlog \weakereq \logsys\)
%%
\item \(\comp{\logsys}\)
%%
\item \(\losko{\logsys}\)
%%
\item \(\modelclassarg[S]{\logsys}{\psi} \neq \modelclassarg[S]{\fstordlog}{\varphi}\) for all \(\varphi \in \languagebase_\firstorder(S)\)
%%
\end{compactenum}}} @. \ \\
@VV{\parbox{20ex}{Using \reftitle{1.4}}}V @. \ \\
\fbox{\parbox{27ex}{\(\logsys\) is regular,\\
\(\fstordlog \weakereq \logsys\),\\
\(\comp{\logsys}\),\\
\(\losko{\logsys}\), and\\
there is a relational \(S\) such that \(\psi \in \languagebase(S)\) and \(\modelclassarg[S]{\logsys}{\psi} \neq \modelclassarg[S]{\fstordlog}{\varphi}\) for all \(\varphi \in \languagebase_\firstorder(S)\)}}
@>>{\parbox{20ex}{Using \reftitle{2.2},
a result of \(\comp{\logsys}\)}}>
\fbox{\parbox{27ex}{For some finite \(S_0 \subset S\):\\
If \(\reduct{\struct{A}}{S_0} \iso \reduct{\struct{B}}{S_0}\), then\\
(\(\struct{A} \models_\logsys \psi\) iff \(\struct{B} \models_\logsys \psi\))\\
for all \(S\)-structures \(\struct{A}\), \(\struct{B}\)}}\\
@VV{\mbox{Using \reftitle{3.1}}}V  @VVV\\
\fbox{\parbox{27ex}{For finite \(S_0 \subset S\), \(m \in \nat\):\\
\(\struct{A} \models_\logsys \psi\), \(\struct{B} \models_\logsys \neg\psi\), and\\
\(\reduct{\struct{A}}{S_0} \iso_m \reduct{\struct{B}}{S_0}\)\\
for some \(S\)-structures \(\struct{A}\), \(\struct{B}\)\smallskip\\
(\(\chi\) is satisfiable for \(m \in \nat\))}} @. \parbox{20ex}{CONTRADICTION}\\
@VV{\parbox{27ex}{Applying \(\comp{\logsys}\) to\\
\(\{ \chi \} \cup \setm{\mbox{``\(\absval{W} \geq n\)''}}{n \in \nat}\)}}V @AAA\\
\fbox{\parbox{27ex}{For finite \(S_0 \subset S\):\\
\(\struct{A} \models_\logsys \psi\), \(\struct{B} \models_\logsys \neg\psi\), and\\
\(\reduct{\struct{A}}{S_0} \iso_p \reduct{\struct{B}}{S_0}\)\\
for some \(S\)-structures \(\struct{A}\), \(\struct{B}\)\smallskip\\
(\(\chi \land \vartheta\) is satisfiable)}}
@>>{\parbox{20ex}{Applying \(\losko{\logsys}\) to \(\chi \land \vartheta\)}}>
\fbox{\parbox{27ex}{For finite \(S_0 \subset S\):\\
\(\struct{A} \models_\logsys \psi\), \(\struct{B} \models_\logsys \neg\psi\), and \(\reduct{\struct{A}}{S_0} \iso \reduct{\struct{B}}{S_0}\)\\
for some \(S\)-structures \(\struct{A}\), \(\struct{B}\)}}\\
\end{CD}
\]\bigskip\\
On may notice that in proving \reftitle{Lemma 2.2} and Lindstr\"{o}m's First Theorem, we utilized the technique of \emph{formalization}. Below we explain in detail why the discussion given in text is valid, highlighting the utilization of formalization. Let $S$ be relational.\bigskip\\
With the assumption that $\psi \in \languagebase(S)$ is not logically equivalent to any first-order sentence, we choose, by $\comp{\logsys}$ and \reftitle{2.2}, a finite subset $S_0$ of $S$ such that\smallskip\\
\begin{quoteno}{($\ast$)}
\begin{minipage}{64ex}
for all $S$-structures $\struct{A}^\prime$ and $\struct{B}^\prime$:\\If $\reduct{\struct{A}^\prime}{S_0} \iso \reduct{\struct{B}^\prime}{S_0}$ then ($\struct{A}^\prime \models_\logsys \psi$ \quad iff \quad $\struct{B}^\prime \models_\logsys \psi$).
\end{minipage}
\end{quoteno}\smallskip\\
On the other hand, by \reftitle{3.1} we have for $m \geq 1$,\smallskip\\
\begin{quoteno}{($\ast\ast$)}
\begin{minipage}{64ex}
there are an $n \geq m$ and two $S$-structures $\struct{A}^\prime$ and $\struct{B}^\prime$ with\smallskip\\$\reduct{\struct{A}^\prime}{S_0} \iso_n \reduct{\struct{B}^\prime}{S_0}$, $\struct{A}^\prime \models_\logsys \psi$, and $\struct{B}^\prime \models_\logsys \neg\psi$.
\end{minipage}
\end{quoteno}\bigskip\\
By a suitable formalization of ($\ast\ast$), i.e.\ the $\languagebase(S^+)$-sentence $(\chi \land \varphi_m)$ where $\varphi_m$ formulates ``there are at least $m$ elements in the descending chain'', we have $(\chi \land \varphi_m)$ is satisfiable for $m \geq 1$. By $\comp{\logsys}$, we then have
\[
\Phi \colonequals \{ \chi \} \cup \setm{\varphi_m}{m \geq 1}
\]
is satisfiable since every finite subset of it has a model. Let us say the $S^+$-structure $\struct{C}$ is a model of $\Phi$; we write $\struct{A}^C$ for $\substr{\intpted{U}{C}}{\reduct{\struct{C}}{S}}$ and $\struct{B}^C$ for $\substr{\intpted{V}{C}}{\reduct{\struct{C}}{S}}$. By $\rel{\logsys}$ we have $\struct{A}^C \models_\logsys \psi$ and $\struct{B}^C \models_\logsys \neg\psi$. Also, in $C$ there is an infinite sequence $(I_n)_{n \in \nat}$ of nonempty sets of partial isomorphisms from $\reduct{\struct{A}^C}{S_0}$ to $\reduct{\struct{B}^C}{S_0}$ in which the back- and the forth-property both hold for the passage from $I_n$ to $I_{n + 1}$ (in contrast to \reftitle{XII.1.3}) for $n \in \nat$. By taking $I \colonequals \bigcup_{n \in \nat} I_n$, we have $I: \reduct{\struct{A}^C}{S_0} \partiso \reduct{\struct{B}^C}{S_0}$. To summarize, we have
\smallskip\\
\begin{quoteno}{($\ast\ast\ast$)}
$\reduct{\struct{A}^C}{S_0} \partiso \reduct{\struct{B}^C}{S_0}$, $\struct{A}^C \models_\logsys \psi$, and $\struct{B}^C \models_\logsys \neg\psi$.
\end{quoteno}\smallskip\\
Since $\intpted{P}{\struct{C}}$ is required by $\Phi$ to be a set of partial isomorphisms from $\reduct{\struct{A}^C}{S_0}$ to $\reduct{\struct{B}^C}{S_0}$, from which we can form the sequence $(I_n)_{n \in \nat}$, we may, without loss of generality, assume that $\intpted{P}{\struct{C}} = I$.
\bigskip\\
Again, ($\ast\ast\ast$) can be formalized, say, by the $\languagebase(S^+)$-sentence $\chi^\prime$, which is the conjunction of (i) - (iii), (viii) from $\chi$ and
\begin{compactenum}[(i$^\prime$)]
%%
\item $\exists p \, Pp$\\($P$ is not empty).
%%
\item $\forall p \forall u ((Pp \land Uu) \limply \exists q \exists v (Pq \land Gquv \land \forall x^\prime \forall y^\prime (Gpx^\prime y^\prime \limply Gqx^\prime y^\prime)))$\\(the forth-property).
%%
\item $\forall p \forall v ((Pp \land Vv) \limply \exists q \exists u (Pq \land Gquv \land \forall x^\prime \forall y^\prime (Gpx^\prime y^\prime \limply Gqx^\prime y^\prime)))$\\(the back-property).
%%
\end{compactenum}
It follows that $\struct{C} \models_\logsys \chi^\prime$ (note that we assume $\intpted{P}{\struct{C}} = I$). By $\losko{\logsys}$, there is an at most countable model $\struct{D}$ of $\chi^\prime$. We write $\struct{A}^D$ for $\substr{\intpted{U}{D}}{\reduct{\struct{D}}{S}}$ and $\struct{B}^D$ for $\substr{\intpted{V}{D}}{\reduct{\struct{D}}{S}}$. Then\smallskip\\
\begin{quoteno}{($\circ$)}
$\reduct{\struct{A}^D}{S_0} \partiso \reduct{\struct{B}^D}{S_0}$, $\struct{A}^D \models_\logsys \psi$, and $\struct{B}^D \models_\logsys \neg\psi$.
\end{quoteno}\smallskip\\
Since $\intpted{U}{D}$ and $\intpted{V}{D}$ are both at most countable, by \reftitle{XII.1.5(d)} we have that\smallskip\\
\begin{quoteno}{($\circ\circ$)}
$\reduct{\struct{A}^D}{S_0} \iso \reduct{\struct{B}^D}{S_0}$, $\struct{A}^D \models_\logsys \psi$, and $\struct{B}^D \models_\logsys \neg\psi$,
\end{quoteno}\smallskip\\
contrary to ($\ast$).
%
\item \textbf{Solution to Exercise 3.6.} Let $S$ be a finite symbol set, $\psi$ an $S$-sentence, and $\modelclassarg[S]{}{\psi}$ be closed under substructures.\footnote{$\psi$ is closed under substructures :iff for $S$-structures $\struct{A}$ and $\struct{B}$, if $\struct{A}$ is a substructure of $\struct{B}$ and if $\struct{B}$ is a model of $\psi$, then $\struct{A} \models \psi$.}\bigskip\\
If $\psi$ is not satisfiable, namely $\modelclassarg[S]{}{\psi} = \emptyset$, then the claim is trivially true because $\psi$ is logically equivalent to, say, the universal sentence $\forall x \neg x \equal x$. So we shall assume $\psi$ is satisfiable.\bigskip\\
As suggested in hint, for $m \geq 1$ we set\smallskip\\
\centerline{$\varphi^m \colonequals \blor\sett{\psi^m_\struct{B}}{\(\struct{B}\) is an \(S\)-structure and \(\struct{B} \models \psi\)}$.}\smallskip\\
It is a universal first-order sentence (cf.\ \reftitle{part (a)} and \reftitle{part (b)} of \textbf{Solution to Exercise XII.3.14} in the annotations to \reftitle{Chapter XII}). The following discussions will lead to the conclusion that $\psi$ is logically equivalent to some $\varphi^m$, thereby completing the exercise.\bigskip\\
We first argue that $\psi \limply \varphi^m$ is valid: If $\struct{A}$ is an $S$-structure that satisfies $\psi$, then $\psi^m_\struct{A}$ is a disjunct in $\varphi^m$; since $\struct{A} \models \psi^m_\struct{A}$ (cf.\ part \reftitle{(b)} of \textbf{Solution to Exercise 3.14} in the annotations to \reftitle{Chapter XII}), it follows that $\struct{A} \models \varphi^m$.\bigskip\\
Now, suppose $\psi$ is not logically equivalent to any $\varphi^m$ (we show that we will arrive at a contradiction based on this assumption). In other words, for all $m \geq 1$ there is an $S$-structure $\struct{A}$ such that $\struct{A} \models \neg\psi$ and $\struct{A} \models \varphi^m$; the latter yields an $S$-structure $\struct{B}$ with $\struct{B} \models \psi$ and $\struct{A} \models \psi^m_\struct{B}$, thus $\struct{A} \emb_m \struct{B}$ (cf.\ \reftitle{part (c)} of \textbf{Solution to Exercise 3.14} in the annotations to \reftitle{Chapter XII}). In summary, for all $m \geq 1$ there are $S$-structures $\struct{A}$ and $\struct{B}$ such that $\struct{A} \emb_m \struct{B}$, $\struct{A} \models \neg\psi$, and $\struct{B} \models \psi$; hence\smallskip\\
\begin{quoteno}{($+$)}
$\relational{\struct{A}} \emb_m \relational{\struct{B}}$, $\relational{\struct{A}} \models \neg\relational{\psi}$, and $\relational{\struct{B}} \models \relational{\psi}$
\end{quoteno}\smallskip\\
(since $\partism{\struct{A}}{\struct{B}} = \partism{\relational{\struct{A}}}{\relational{\struct{B}}}$ and by \reftitle{VIII.1.3}).\bigskip\\
The statement ($+$) can be formulated. Let $S^+$ be obtained from $\relational{S}$(!) by adding the following new symbols: $c, f, P, U, V, W, < , I, G$ in the discussion on page 267 in text.\bigskip\\
Take the $S^+$-sentence $\chi$ to be the conjunction of (i), (ii), (iv), (v), (vi) on page 268 in text together with
\begin{itemize}
%%
\item for every $c \in S$, the sentence $(\relativize{(\relational{(\exists^{=1}x \, c \equal x)})}{U} \land \relativize{(\relational{(\exists^{=1}x \, c \equal x)})}{V})$;
%%
\item for every $n$-ary $f \in S$, the sentence\\
$\begin{array}{r}
(\relativize{(\relational{(\enump{\forall x_1}{\forall x_n}\exists^{=1}x \, f\enum[1]{x}{n} \equal x)})}{U} \land \phantom{aaaaa}\\
\relativize{(\relational{(\enump{\forall x_1}{\forall x_n}\exists^{=1}x \, f\enum[1]{x}{n} \equal x)})}{V});
\end{array}$
%%
\item for every $n$-ary $R \in \relational{S}$(!), the sentence\\
$\begin{array}{r}
\forall p (Pp \limply \enump{\forall x_1}{\forall x_n} \enump{\forall y_1}{\forall y_n} ((\enumpop{Gpx_1y_1}{\land}{Gpx_ny_n}) \limply \phantom{aaaaa}\\
(R\enum[1]{x}{n} \liff R\enum[1]{y}{n})));
\end{array}$
%%
\item the sentence $\exists x Ux \land \exists y Vy \land \relativize{(\neg\relational{\psi})}{U} \land \relativize{(\relational{\psi})}{V}$\\(both $U$ and $V$ are nonempty, and $\relational{\psi}$ holds in the $\relational{S}$-structure induced on $V$ but not in the one induced on $U$; the conjuncts $\exists x Ux$ and $\exists y Vy$ are necessary in case $S$ is relational).
%%
\end{itemize}
We have formulated ($+$) by $\chi$, which does not depend on $m$.\bigskip\\
We have the following result analogous to 3.2:\medskip\\
\begin{theorem}{Claim}
For every $m \geq 1$ there is a model $\struct{C}$ of $\chi$ in which the field $\intpted{W}{C}$ of $\intpted{<}{C}$ consists of exactly $(m + 1)$ elements.
\end{theorem}
\begin{proof}
For a given $m$ we choose, according to ($+$), two $S$-structures $\struct{A}$, $\struct{B}$ and a sequence $(I_n)_{n \leq m}$ of nonempty subsets of $\partism{\struct{A}}{\struct{B}}$ with $A \cap B = \emptyset$, $(I_n)_{n \leq m}: \relational{\struct{A}} \emb_m \relational{\struct{B}}$, $\relational{\struct{A}} \models \neg\relational{\psi}$ and $\relational{\struct{B}} \models \relational{\psi}$.\medskip\\
The claim immediately follows if we take as a model of $\chi$ an $S^+$-structure $\struct{C}$ with
\begin{itemize}
%%
\item $C = A \cup B \cup \{\seqp{0}{m}\} \cup \bigcup_{n \leq m} I_n$;
%%
\item $\intpted{U}{C} = A$ and $\substr{\intpted{U}{C}}{\reduct{\struct{C}}{\relational{S}}} = \relational{\struct{A}}$;
%%
\item $\intpted{V}{C} = B$ and $\substr{\intpted{V}{C}}{\reduct{\struct{C}}{\relational{S}}} = \relational{\struct{B}}$;
%%
\item $\intpted{W}{C} = \{\seqp{0}{m}\}$, $\intpted{<}{C}$ is the natural ordering relation on $\{\seqp{0}{m}\}$, $\intpted{c}{C} = m$ and $\restrict{\intpted{f}{C}}{\intpted{W}{C}}$ is the predecessor function on $\intpted{W}{C}$ ($\intpted{f}{C}(n + 1) = n$ for $n < m$ and $\intpted{f}{C}(0) = 0$);
%%
\item $\intpted{P}{C} = \bigcup_{n \leq m} I_n$;
%%
\item $\intpted{I}{C}np$ \quad iff \quad $n \leq m$ and $p \in I_n$;
%%
\item $\intpted{G}{C}pab$ \quad iff \quad $\intpted{P}{C}p$, $a \in \dom{p}$ and $p(a) = b$. \qedhere
%%
\end{itemize}
\end{proof}
From the Compactness Theorem and the above claim it follows that the set\smallskip\\
\centerline{$\{ \chi \} \cup \setm{\mbox{``\(W\) contains at least \(n\) element''}}{n \geq 2}$}\smallskip\\
is satisfiable. Then the $S^+ \cup \{Q\}$-sentence $\chi \land \vartheta$ is also satisfiable, where $\vartheta$ is defined in the proof of 3.3. By the L\"{o}wenheim-Skolem Theorem, there is a countable model of $\chi \land \vartheta$.\bigskip\\
Thus, there are two at most countable $\relational{S}$-structures $\struct{A}$ (with domain $A$) and $\struct{B}$ such that $\struct{A} \models \neg\relational{\psi}$, $\struct{B} \models \relational{\psi}$, and $\struct{A} \partemb \struct{B}$: As a result of the previous discussion, there is an infinite sequence $(I_n)_{n \in \nat}$ in which $\emptyset \neq I_n \subset \partism{\struct{A}}{\struct{B}}$ and the forth-property holds for the passage from $I_n$ to $I_{n + 1}$ for $n \in \nat$; by taking $I \colonequals \bigcup_{n \in \nat} I_n$ we have that $I: \struct{A} \partemb \struct{B}$.\bigskip\\
Note that $\invrelational{\struct{A}}$ and $\invrelational{\struct{B}}$ are well-defined $S$-structures by the choice of $\chi$. From \reftitle{VIII.1.3} it follows that $\invrelational{\struct{A}} \models \neg\psi$, $\invrelational{\struct{B}} \models \psi$. Moreover, by $\partism{\struct{A}}{\struct{B}} = \partism{\invrelational{\struct{A}}}{\invrelational{\struct{B}}}$ we have $\invrelational{\struct{A}} \partemb \invrelational{\struct{B}}$; hence, $\invrelational{\struct{A}}$ is embeddable in $\invrelational{\struct{B}}$ since $A$ is at most countable (cf.\ \reftitle{Exercise 1.12(b)}). That is to say, $\invrelational{\struct{A}}$ is isomorphic to a substructure of $\invrelational{\struct{B}}$.\bigskip\\
Therefore, the model $\invrelational{\struct{B}}$ of $\psi$ has a substructure which does not satisfy $\psi$ (by the Isomorphism Lemma), contrary to the premise that $\psi$ is closed under substructures. Hence, $\psi$ is logically equivalent to $\varphi^m$ for some $m \geq 1$.
%
\item \textbf{Solution to Exercise 3.7.} The direction from (1) to (2): Assume $\Phi$ defines $P$ explicitly or, to be precise, assume that $\psi \in \fstordlang[k]{S}$ and $\Phi \models \enump{\forall v_0}{\forall v_{k - 1}} (P\enum{v}{k - 1} \liff \psi)$. Let $\struct{A}$ be an $S$-structure, $P^1, P^2 \subset A^k$ with $\pair{\struct{A}}{P^1} \models \Phi$ and $\pair{\struct{A}}{P^2} \models \Phi$. Then for $\seq{a}{k - 1} \in A$,\smallskip\\
\begin{tabular}{ll}
\   & $P^1 \enum{a}{k - 1}$ \cr
iff & $\pair{\struct{A}}{P^1} \models P\enum{v}{k - 1} [\seq{a}{k - 1}]$ \cr
iff & $\pair{\struct{A}}{P^1} \models \psi [\seq{a}{k - 1}]$ \quad (by premise) \cr
iff & $\struct{A} \models \psi [\seq{a}{k - 1}]$ \quad (by the Coincidence Lemma) \cr
iff & $\pair{\struct{A}}{P^2} \models \psi [\seq{a}{k - 1}]$ \quad (by the Coincidence Lemma) \cr
iff & $\pair{\struct{A}}{P^2} \models P\enum{v}{k - 1} [\seq{a}{k - 1}]$ \quad (by premise) \cr
iff & $P^2\enum{a}{k - 1}$;
\end{tabular}\smallskip\\
that is, $P^1 = P^2$. It turns out that $\Phi$ defines $P$ implicitly.\bigskip\\
In the following we prove the direction from (2) to (1). For convenience we shall restrict ourselves to the case of relational $S$, after arguing this restriction is not essential: Suppose that the claim has been proven valid for the case of relational symbol sets. We show that the claim is valid for arbitrary $S$ as well. Denoting $\relational{\Psi} \colonequals \setm{\relational{\psi}}{\psi \in \Psi}$ for $\Psi \subset \fstordlang[0]{S}$, we take $\Phi^\prime$ to be the set of sentences in $\relational{\Phi}$ together with
\begin{itemize}
%%
\item for every constant $c \in S$, the $\relational{S}$-sentence $\relational{(\exuni x \, c \equal x)}$;
%%
\item for every $n$-ary function $f \in S$, the $\relational{S}$-sentence\\ $\relational{(\enump{\forall x_1}{\forall x_n} \exuni x \, f\enum[1]{x}{n} \equal x)}$.
\end{itemize}
It then follows (cf.\ \reftitle{VIII.1.3}) that:
\begin{enumerate}[(a)]
%%
\item For every $S \cup \{ P \}$-structure $\struct{A}$, \qquad $\struct{A} \models \Phi$ \quad iff \quad $\relational{\struct{A}} \models \Phi^\prime$.
%%
\item For every $\relational{S} \cup \{ P \}$-structure $\struct{A}$, \qquad $\struct{A} \models \Phi^\prime$ \quad iff \quad ($\invrelational{\struct{A}}$ is well-defined and $\invrelational{\struct{A}} \models \Phi$).
%%
\end{enumerate}
Assume $\Phi$ defines $P$ implicitly. If $\struct{A}$ is an $\relational{S}$-structure and if $P^1, P^2 \subset A^k$ such that $\pair{\struct{A}}{P^1} \models \Phi^\prime$ and $\pair{\struct{A}}{P^2} \models \Phi^\prime$, then $\invrelational{\struct{A}}$ is a well-defined $S$-structure, $\pair{\invrelational{\struct{A}}}{P^1} \models \Phi$ and $\pair{\invrelational{\struct{A}}}{P^2} \models \Phi$ by (b); hence $P^1 = P^2$ by assumption, i.e.\ $\Phi^\prime$ defines $P$ implicitly. Then we have $\Phi^\prime$ defines $P$ explicitly by the premise that the claim has been proven for the case of relational symbol sets. It turns out that $\Phi$ defines $P$ explicitly: Let $\psi \in \fstordlang[k]{\relational{S}}$ with $\Phi^\prime \models \enump{\forall v_0}{\forall v_{k - 1}} (P\enum{v}{k - 1} \liff \psi)$. Then for every $S \cup \{ P \}$-structure $\struct{A}$,\smallskip\\
\begin{tabular}{ll}
\    & $\struct{A} \models \Phi$ \cr
iff  & $\relational{\struct{A}} \models \Phi^\prime$ \quad (by (a))\cr
then & $\relational{\struct{A}} \models \enump{\forall v_0}{\forall v_{k - 1}} (P\enum{v}{k - 1} \liff \psi)$ \cr
iff  & $\struct{A} \models \enump{\forall v_0}{\forall v_{k - 1}} (P\enum{v}{k - 1} \liff \invrelational{\psi})$ \quad (cf.\ \reftitle{VIII.1.3(b)}).
\end{tabular}\smallskip\\
That is $\Phi \models \enump{\forall v_0}{\forall v_{k - 1}} (P\enum{v}{k - 1} \liff \invrelational{\psi})$.\bigskip\\
Now we return to showing the direction from (2) to (1), assuming that $S$ is relational. Note that if $\Phi$ is not satisfiable, then the claim immediately follows by choosing, say, $\psi = v_0 \equal v_0$. Hence we shall additionally assume that $\Phi$ is satisfiable.\bigskip\\
To start, let $S_0$ be a given finite subset of $S$. For $n \geq 0$ consider, as suggested in hint, the $\fstordlang[k]{S_0}$-formula\smallskip\\
\centerline{$\chi^n \colonequals \bigvee\sett{\varphi^n_{\reduct{\struct{A}}{S_0}, \vect{a}{k}}}{\(\struct{A}\) is an \(S\)-structure, \(\pair{\struct{A}}{\intpted{P}{A}} \models \Phi\) and \(\intpted{P}{A} \vect{a}{k}\)}$}\smallskip\\
(cf.\ \reftitle{XII.3.4} to check it is a first-order formula).
\bigskip\\
We argue that $\enump{\forall v_0}{\forall v_{k - 1}}(P\enum{v}{k - 1} \limply \chi^n)$ is a consequence of $\Phi$: Let $\pair{\struct{A}}{\intpted{P}{A}}$ be an $S \cup \{ P \}$-structure that satisfies $\Phi$. If $\vect{a}{k} \in A$ and $\intpted{P}{A} \vect{a}{k}$, then $\varphi^n_{\reduct{\struct{A}}{S_0}, \vect{a}{k}}$ is a disjunct in $\chi^n$. So $\pair{\struct{A}}{\intpted{P}{A}} \models \chi^n [\vect{a}{k}]$ by the fact that $\reduct{\struct{A}}{S_0} \models \varphi^n_{\reduct{\struct{A}}{S_0}, \vect{a}{k}} [\vect{a}{k}]$ (cf.\ \reftitle{XII.3.5(b)}) and by the Coincidence Lemma.\bigskip\\
The following discussions will show that for some finite subset $S_0$ of $S$ and some $n \geq 0$, $\enump{\forall v_0}{\forall v_{k - 1}} (\chi^n \limply P\enum{v}{k - 1})$ is a consequence of $\Phi$, provided that $\Phi$ defines $P$ implicitly. From this and the result just obtained, we shall therefore conclude the direction from (2) to (1).\bigskip\\
\begin{theorem}{Claim 1}
For finite subset $S_0$ of $S$ and $n \geq 0$, if $\enump{\forall v_0}{\forall v_{k - 1}} (\chi^n \limply P\enum{v}{k - 1})$ is not a consequence of $\Phi$ then there are $S \cup \{ P \}$-structures $\pair{\struct{A}}{\intpted{P}{A}}$, $\pair{\struct{B}}{\intpted{P}{B}}$, $\vect{a}{k} \in A$ and $\vect{b}{k} \in B$ such that, for any finite subset $\Phi_0 \subset \Phi$,\smallskip\\
\begin{bquoteno}{68ex}{{\rm($+$)}}
$\pair{\struct{A}}{\intpted{P}{A}} \models \Phi_0$, $\pair{\struct{B}}{\intpted{P}{B}} \models \Phi_0$, $\intpted{P}{A}\vect{a}{k}$, not $\intpted{P}{B}\vect{b}{k}$, and there is a sequence $(I_m)_{m \leq n}$ with $I_n = \{ \vect{a}{k} \mapsto \vect{b}{k} \}$ and $(I_m)_{m \leq n}: \reduct{\struct{A}}{S_0} \iso_n \reduct{\struct{B}}{S_0}$.
\end{bquoteno}
\end{theorem}
\begin{proof}
Let finite $S_0 \subset S$ and $n \geq 0$ be given, and assume the premise. Then there are an $S \cup \{ P \}$-structure $\pair{\struct{B}}{\intpted{P}{B}}$ that is a model of $\Phi$ and $\vect{b}{k} \in B$ such that $\reduct{\struct{B}}{S_0} \models \chi^n [\vect{b}{k}]$ but not $\intpted{P}{B} \vect{b}{k}$.\bigskip\\
From $\reduct{\struct{B}}{S_0} \models \chi^n [\vect{b}{k}]$, we infer an $S \cup \{ P \}$-structure $\pair{\struct{A}}{\intpted{P}{A}}$ that is a model of $\Phi$ and $\vect{a}{k} \in A$ with $\intpted{P}{A}\vect{a}{k}$ so that $\reduct{\struct{B}}{S_0} \models \varphi^n_{\reduct{\struct{A}}{S_0}, \vect{a}{k}}[\vect{b}{k}]$. This yields a sequence $(I_m)_{m \leq n}$ with $I_n = \{ \vect{a}{k} \mapsto \vect{b}{k} \}$ and $(I_m)_{m \leq n}: \reduct{\struct{A}}{S_0} \iso_n \reduct{\struct{B}}{S_0}$ (cf.\ the proof for part \reftitle{(b)} of \reftitle{XII.3.8} for the back- and the forth-property).\bigskip\\
Finally, if $\Phi_0$ is a finite subset of $\Phi$, then we also have $\pair{\struct{A}}{\intpted{P}{A}} \models \Phi_0$ and $\pair{\struct{B}}{\intpted{P}{B}} \models \Phi_0$.
\end{proof}
The statement ($+$) can be formulated. Let $S^+$ be obtained from $S \cup \{ P \}$ by adding the following new symbols: $c, f, R, U, V, W, <, I, G$ in the discussion on page 267 in text (note that $R$ is in place of $P$ because it is already used here), and constant symbols $\seq{d}{k - 1}, \seq{e}{k - 1}$. For $\Psi \subset \fstordlang[0]{S}$ we write $\relativize{\Psi}{U} \colonequals \setm{\relativize{\psi}{U}}{\psi \in \Psi}$ ($\relativize{\Psi}{V}$ is defined analogously).\bigskip\\
For finite $S_0 \subset S$ and finite $\Phi_0 \subset \Phi$, take the conjunction $\delta^+$ of the sentences in (i) - (vii) on page 268 in text\footnote{Of course, $P$ appearing there should be replaced by $R$.} together with
\begin{itemize}
%%
\item the conjunction of sentences in $\relativize{\Phi_0}{U}$;
%%
\item the conjunction of sentences in $\relativize{\Phi_0}{V}$;
%%
\item $P\enum{d}{k - 1} \land \neg P\enum{e}{k - 1}$\\($P$ holds for $\seq{d}{k - 1}$, but not for $\seq{e}{k - 1}$);
%%
\item $\exuni p (Rp \land Icp \land \bigwedge_{0 \leq i < k} Gpd_ie_i)$\\(the set $I_c$ consists solely of the map $\vect{d}{k} \mapsto \vect{e}{k}$).
%%
\end{itemize}
(There is no need to include $\exists x Ux \land \exists y Vy$ as a conjunct because the last sentence above together with (i) ensures that $U$ and $V$ are both nonempty.) Thus, $\delta^+$ is the desired formulation of ($+$) and it does not depend on $n$.\bigskip\\
In the following discussions leading to the summary, we assume that \emph{for any finite subset $S_0$ of $S$ and any $n \geq 0$, $\enump{\forall v_0}{\forall v_{k - 1}} (\chi^n \limply P\enum{v}{k - 1})$ is not a consequence of $\Phi$}.\bigskip\\
\begin{theorem}{Claim 2}
For finite $S_0 \subset S$, $n \geq 0$ and finite $\Phi_0 \subset \Phi$, there is a model $\struct{C}$ of $\delta^+$ in which the field $\intpted{W}{C}$ of $\intpted{<}{C}$ consists of exactly $(n + 1)$ elements.
\end{theorem}
\begin{proof}
Given finite $S_0 \subset S$, $n \geq 0$ and finite $\Phi_0 \subset \Phi$, we choose, according to \textbf{Claim 1}, two $S \cup \{ P \}$-structures $\pair{\struct{A}}{\intpted{P}{A}}$, $\pair{\struct{B}}{\intpted{P}{B}}$ with $A \cap B = \emptyset$ and $\vect{a}{k} \in A$, $\vect{b}{k} \in B$ for which ($+$) holds.\bigskip\\
Define an $S^+$-structure $\struct{C}$ with
\begin{itemize}
%%
\item $C = A \cup B \cup \{\seqp{0}{n}\} \cup \bigcup_{m \leq n} I_m$;
%%
\item $\intpted{U}{C} = A$ and $\substr{\intpted{U}{C}}{\reduct{\struct{C}}{S \cup \{ P \}}} = \pair{\struct{A}}{\intpted{P}{A}}$;
%%
\item $\intpted{V}{C} = B$ and $\substr{\intpted{V}{C}}{\reduct{\struct{C}}{S \cup \{ P \}}} = \pair{\struct{B}}{\intpted{P}{B}}$;
%%
\item $\intpted{W}{C} = \{ \seqp{0}{n} \}$, $\intpted{<}{C}$ is the natural ordering relation on $\{ \seqp{0}{n} \}$, $\intpted{c}{C} = n$ and $\restrict{\intpted{f}{C}}{\intpted{W}{C}}$ is the predecessor function on $\intpted{W}{C}$ ($\intpted{f}{C}(m + 1) = m$ for $m < n$ and $\intpted{f}{C}(0) = 0$);
%%
\item $\intpted{R}{C} = \bigcup_{m \leq n} I_m$;
%%
\item $\intpted{I}{C}mp$ \quad iff \quad $m \leq n$ and $p \in I_m$;
%%
\item $\intpted{G}{C}pab$ \quad iff \quad $\intpted{R}{C}p$, $a \in \dom{p}$ and $p(a) = b$;
%%
\item $\intpted{d_i}{C} = a_i$ for $0 \leq i < k$;
%%
\item $\intpted{e_i}{C} = b_i$ for $0 \leq i < k$.
%%
\end{itemize}
By definition $\struct{C}$ is a model of $\delta^+$ in which the field $\intpted{W}{C}$ of $\intpted{<}{C}$ consists of exactly $(n + 1)$ elements.
\end{proof}
We immediately have:\medskip\\
\begin{theorem}{Claim 3}
For finite $S_0 \subset S$ and finite $\Phi_0 \subset \Phi$, there are two $S \cup \{ P \}$-structures $\pair{\struct{A}}{\intpted{P}{A}}$ and $\pair{\struct{B}}{\intpted{P}{B}}$, $\vect{a}{k} \in A$, $\vect{b}{k} \in B$ and a map $\pi: A \to B$ such that\smallskip\\
\begin{bquoteno}{68ex}{{\rm($\ast$)}}
$\pair{\struct{A}}{\intpted{P}{A}} \models \Phi_0$, $\pair{\struct{B}}{\intpted{P}{B}} \models \Phi_0$, $\intpted{P}{A}\vect{a}{k}$, not $\intpted{P}{B}\vect{b}{k}$, $\pi: \reduct{\struct{A}}{S_0} \iso \reduct{\struct{B}}{S_0}$ and $\vect{a}{k} \mapsto \vect{b}{k} \subset \pi$.
\end{bquoteno}
\end{theorem}
\begin{proof}
Let finite subsets $S_0 \subset S$ and $\Phi_0 \subset \Phi$ be given. According to \textbf{Claim 2}, every finite subset of\smallskip\\
\centerline{$\{\delta^+\} \cup \setm{\mbox{``\(W\) contains at least \(n\) elements''}}{n \in \nat}$}\smallskip\\
is satisfiable; by the Compactness Theorem the set itself is also satisfiable.\bigskip\\
So $\delta^+ \land \vartheta$ has a model (cf.\ the proof of \reftitle{3.3} for the definition of $\vartheta$); furthermore, by the L\"{o}wenheim-Skolem Theorem, we may assume this model is countable.\bigskip\\
By an argument similar to that given in the proof of \reftitle{3.3}, it follows that the claim is true.
\end{proof}
The statement ($\ast$) can also be formulated. Let $S^\ast$ be obtained from $S \cup \{ P \}$ by adding the following new symbols: two unary relation symbols $U, V$, a unary function symbol $f$, and constant symbols $\seq{d}{k - 1}, \seq{e}{k - 1}$.\bigskip\\
Take $\delta^\ast$ to be the conjunction of the following sentences
\begin{itemize}
%%
\item $\bigwedge_{0 \leq i < k} Ud_i$;
%%
\item $\bigwedge_{0 \leq i < k} Ve_i$;
%%
\item $P\enum{d}{k - 1}$;
%%
\item $\neg P\enum{e}{k - 1}$;
%%
\item $\forall x (Ux \limply Vfx)$;
%%
\item $\forall y (Vy \limply \exists x (Ux \land fx \equal y))$;
%%
\item $\forall x \forall y ((Ux \land Uy \land fx \equal fy) \limply x \equal y)$;
%%
\item $\bigwedge_{0 \leq i < k} fd_i \equal e_i$.
%%
\end{itemize}
Note that $\delta^\ast$ does not depend on $S$ or $\Phi$. For finite $S_0 \subset S$ and finite $\Phi_0 \subset \Phi$, the set\smallskip\\
\centerline{$\relativize{\Phi_0}{U} \cup \relativize{\Phi_0}{V} \cup \setm{\varphi_R}{R \in S_0} \cup \{ \delta^\ast \}$}\smallskip\\
is the desired formulation of ($\ast$), where $\varphi_R$ denotes the sentence\smallskip\\
\centerline{$\enump{\forall x_1}{\forall x_m} ((\enumpop{Ux_1}{\land}{Ux_m}) \limply (R\enum[1]{x}{m} \liff R\enump{fx_1}{fx_m}))$}\smallskip\\
with $m$-ary $R$.\bigskip\\
\begin{theorem}{Claim 4}
The set $\relativize{\Phi}{U} \cup \relativize{\Phi}{V} \cup \setm{\varphi_R}{R \in S} \cup \{ \delta^\ast \}$ is satisfiable.
\end{theorem}
\begin{proof}
By the Compactness Theorem it suffices to show: For finite $S_0 \subset S$ and finite $\Phi_0 \subset \Phi$, the set\smallskip\\
\centerline{$\relativize{\Phi_0}{U} \cup \relativize{\Phi_0}{V} \cup \setm{\varphi_R}{R \in S_0} \cup \{ \delta^\ast \}$}\smallskip\\
is satisfiable.\bigskip\\
Given $S_0$ and $\Phi_0$, we choose, according to \textbf{Claim 3}, two $S \cup \{ P \}$-structures $\pair{\struct{A}}{\intpted{P}{A}}$, $\pair{\struct{B}}{\intpted{P}{B}}$ with $A \cap B = \emptyset$, $\vect{a}{k} \in A$, $\vect{b}{k} \in B$ and a map $\pi : A \to B$ for which ($\ast$) holds.\bigskip\\
Define an $S^\ast$-structure $\struct{D}$ with (cf.\ the proof of \reftitle{2.2})
\begin{itemize}
%%
\item $D = A \cup B$;
%%
\item $\intpted{R}{D} = \intpted{R}{A} \cup \intpted{R}{B}$ for $R \in S$;
%%
\item $\intpted{U}{D} = A$, $\intpted{V}{D} = B$;
%%
\item $\restrict{\intpted{f}{D}}{\intpted{U}{D}} = \pi$;
%%
\item for $0 \leq i < k$, $\intpted{d_i}{D} = a_i$;
%%
\item for $0 \leq i < k$, $\intpted{e_i}{D} = b_i$;
%%
\item $\intpted{P}{D} = \intpted{P}{A} \cup \intpted{P}{B}$.
%%
\end{itemize}
By definition, $\struct{D}$ is a model of $\relativize{\Phi_0}{U} \cup \relativize{\Phi_0}{V} \cup \setm{\varphi_R}{R \in S_0} \cup \{ \delta^\ast \}$.
\end{proof}
As a consequence of the above claim, there are two $S \cup \{ P \}$-structures $\pair{\struct{A}}{\intpted{P}{A}}$ and $\pair{\struct{B}}{\intpted{P}{B}}$ both of which are models of $\Phi$, $\vect{a}{k} \in A$, $\vect{b}{k} \in B$ and a map $\pi: A \to B$ such that $\intpted{P}{A}\vect{a}{k}$, not $\intpted{P}{B}\vect{b}{k}$, $\pi: \struct{A} \iso \struct{B}$ and $\vect{a}{k} \mapsto \vect{b}{k} \subset \pi$.\bigskip\\
Using \reftitle{Corollary III.5.3}, we further get:\medskip\\
\begin{theorem}{Claim 5}
There are an $S$-structure $\struct{A}$, $P^1, P^2 \subset A^k$ and $\vect{a}{k} \in A$ such that $(\struct{A}, P^1) \models \Phi$, $(\struct{A}, P^2) \models \Phi$, $P^1\vect{a}{k}$ but not $P^2\vect{a}{k}$. \qed
\end{theorem}\bigskip\\
The summary is given below:\medskip\\
\begin{theorem}{Summary}
Suppose that for any finite $S_0 \subset S$ and any $n \geq 0$, the $S_0 \cup \{ P \}$-sentence $\enump{\forall v_0}{\forall v_{k - 1}} (\chi^n \limply P\enum{v}{k - 1})$ is not a consequence of $\Phi$. Then there are an $S$-structure $\struct{A}$ and $P^1, P^2 \subset A^k$ such that $\pair{\struct{A}}{P^1} \models \Phi$, $\pair{\struct{A}}{P^2} \models \Phi$ and $P^1 \neq P^2$. \qed
\end{theorem}\bigskip\\
With the earlier result that $\enump{\forall v_0}{\forall v_{k - 1}} (P\enum{v}{k - 1} \limply \chi^n)$ is a consequence of $\Phi$ for every finite $S_0 \subset S$ and every $n \geq 0$, we therefore conclude:\medskip\\
If $\Phi$ defines $P$ implicitly, then $\Phi \models \enump{\forall v_0}{\forall v_{k - 1}} (P\enum{v}{k - 1} \liff \chi^n)$ for some finite $S_0 \subset S$ and some $n \geq 0$, i.e.\ $\Phi$ defines $P$ explicitly.
%
\end{enumerate}
%End of Section XIII.3------------------------------------------------------------
\
\\
\\
%Section XIII.4-------------------------------------------------------------------
{\large \S4. Lindstr\"{o}m's Second Theorem}
\begin{enumerate}[1.]
%
\item \textbf{Note on Definition 4.1.} There is a typo in the second line of this definition: ``the set $\L(S)$ is decidable'' should be replaced by ``the set $\languagebase(S)$ is decidable''.
%
\item \textbf{Note on Definition 4.2.} The defining clause of $\logsys \effwkereq \logsys^\prime$ is not necessarily stricter than that of $\logsys \weakereq \logsys^\prime$: Be aware of the quantification ``for every \emph{decidable} symbol set $S$.''\medskip\\
The situation is the same for \reftitle{4.3}.
%
\item \textbf{Note on Logical Systems Concerning Effectivity and $\effwkereq$.}
\begin{asparaenum}[(a)]
%%
\item In each of the logical systems $\fstordlog$, $\weaksndordlog$, $\sndordlog$ and $\qlog$, for decidable $S$ the $S$-formulas are finite symbol strings generated by the corresponding calculus (cf.\ \reftitle{Definitions II.3.2, IX.1.1 and IX.3.1}) and hence the set of $S$-sentences are decidable; these logical systems are effective. In contrast, however, the logical system $\infinlog$ fails to be effective because, say, for the decidable set $S = \setm{c_n}{n \in \nat}$ and the $\infinlang{S}$-sentence\smallskip\\
\centerline{$\varphi = \forall x \bigvee \setm{x \equal c_n}{n \in \nat}$,}\smallskip\\
there is no finite $S_0 \subset S$ such that $\varphi \in \languagebase_\infin(S)$.
%%
\item By the argument in \textbf{Note on the Examples Given below Definition 1.2}, we immediately have $\fstordlog \effwkereq \weaksndordlog$.\bigskip\\
As for $\weaksndordlog \effwkereq \sndordlog$, cf.\ part (b) of \textbf{Solution to Exercise 1.7} in the annotations to \reftitle{Chapter IX} and note that there the way we chose $\psi \in \sndordlang{S}$ logically equivalent to $\varphi \in \weaksndordlang{S}$ yields a computable function that satisfies the requirements mentioned in \reftitle{Definition 4.2(a)}.
%%
\end{asparaenum}
%
\item \textbf{Note on Definition 4.3.} (INCOMPLETE) There are two typos in (i): ``There is'' should be replaced by ``There are''; also, ``from $\languagebase\languagebase(S)$'' should be replaced by ``from $\languagebase(S)$''.
%
\item \textbf{Verifying $\fstordlog$, $\weaksndordlog$, $\sndordlog$ and $\qlog$ Are Effectively Regular Logical Systems.} We have verified, in \textbf{Note on Logical Systems Concerning Effectivity and $\effwkereq$}, that each of $\fstordlog$, $\weaksndordlog$, $\sndordlog$ and $\qlog$ is an effective logical system. By the discussions in the annotations to \reftitle{Chapter IX} (for relativization and replacement) we further claim that they are effectively regular: For example, the choice of $\relativize{\varphi}{U}$ for $\varphi \in \sndordlang{S}$ there yields a computable function that satisfies the requirements mentioned in the effective analogue of $\rel{\sndordlog}$, provided that $S$ is decidable.
%
\item \textbf{Verifying that for $\fstordlog$ and for $\qlog$ the Set of Valid Sentences Are Enumerable.} Both logical systems $\fstordlog$ and $\qlog$ have an adequate proof calculus, so in each of them the set of correct sequents over a decidable symbol set $S$ is enumerable. Since for every $S$-sentence $\varphi$, it is valid if and only if $\varphi$ itself is a correct sequent, it follows that the set of valid sentences are enumerable as well.
%
\item \textbf{Note on the Proof of Linstr\"{o}m's Second Theorem.} (INCOMPLETE)
\begin{asparaenum}[(1)]
%%
\item In proving ($+$) for a more restricted (namely decidable, finite and relational) $S$, there is actually no need to assume that $S$ is decidable in addition to the assumption that $S$ is finite, since a finite set is definitely decidable.
%%
\item One may be confused why it is valid to apply \reftitle{3.4} in the proof while effective regularity does not necessarily imply regularity and neither does $\fstordlog \effwkereq \logsys$ imply $\fstordlog \weakereq \logsys$ (cf.\ \textbf{Note on Definition 4.2}).\bigskip\\
Recall that in deriving the results in \reftitle{3.4}, we did not argue on \emph{every}, but rather, argued on \emph{an arbitrarily given}, symbol set $S$;\footnote{In other words, there in \reftitle{3.4} we could have assumed the requirements in defining clauses in $\logsys$ being a regular logical system, in $\fstordlog \weakereq \logsys$ and even in $\losko{\logsys}$ are satisfied only for some particular symbol sets including $S$.} we then drew the conclusion by \emph{justified generalization} (a common proof method which is formally described by \reftitle{Exercise IV.5.5(b4)}). Thus, it is valid to apply \reftitle{3.4} here.
%%
\item There are two typos in the fourth line of the fifth paragraph: ``$\struct{A} \models \psi$'' and ``$\struct{B} \models \neg\psi$'' should be replaced by ``$\struct{A} \models_\logsys \psi$'' and ``$\struct{B} \models_\logsys \neg\psi$'', respectively.
%%
\item As remarked at the end of \reftitle{Section X.5}, the results obtained there are also valid for any other symbol sets $S_0$ than $S_\infty$ which are effectively given and contain symbols that allow to describe the execution of programs. So we choose such an $S_0$ and, to be precise, let $S_0 \colonequals \{ R, <, f, c \}$, the set of symbols used in the proof of \reftitle{X.4.1}. By Trahtenbrot's Theorem, the set of \finval\ first-order $S_0$-sentences is not enumerable.\bigskip\\
Since $S_0$ is finite, its relational counterpart $\relational{S_0}$ is decidable (for example, we assign $F$ and $C$ to $f$ and $c$, respectively). We may assume $\relational{S_0}$ is disjoint from $S^+$. Moreover:\medskip\\
\begin{theorem}{Claim}
The set of \finval\ first-order $\relational{S_0}$-sentences is not enumerable.
\end{theorem}
\begin{proof}
For convenience, we write $\Phi_0$ for the set of \finval\ first-order $S_0$-sentences and $\Phi_1$ for the set of \finval\ first-order $\relational{S_0}$-sentences; we shall denote
\[
\relational{\Psi} \colonequals \setm{\relational{\varphi}}{\varphi \in \Psi}
\]
for subsets $\Psi \subset \fstordlang[0]{S_0}$. In addition, we write
\begin{itemize}
%%%
\item $\psi_c$ for $\exuni x c \equal x$;
%%%
\item $\psi_f$ for $\forall x \exuni y fx \equal y$; and
%%%
\item $\varphi_f$ for $\forall x \forall y (fx \equal fy \limply x \equal y) \limply \forall y \exists x fx \equal y$ \quad (if $f$ is injective then it is surjective).
%%%
\end{itemize}
It can be easily seen from definition that $\consqn{\Phi_0} = \Phi_0$. Thus,\smallskip\\
\begin{quoteno}{($\circ$)}
$\consqn{(\relational{\Phi_0})} = \relational{\Phi_0}$
\end{quoteno}\smallskip\\
by \textbf{Note on Consequence Closures Concerning Relativization} in the annotations to \reftitle{Chapter X}. Also, $\psi_c, \psi_f, \varphi_f \in \Phi_0$.\bigskip\\
We argue that\smallskip\\
\begin{quoteno}{($+$)}
$\consqn{(\Phi_1 \cup \{ \relational{\psi_c}, \relational{\psi_f}, \relational{\varphi_f} \})} = \consqn{(\relational{\Phi_0})}$.
\end{quoteno}\smallskip\\
It suffices to show $\modelclass{\relational{S_0}}{(\Phi_1 \cup \{ \relational{\psi_c}, \relational{\psi_f}, \relational{\varphi_f} \})} = \modelclass{\relational{S_0}}{\relational{\Phi_0}}$: Let $\struct{A}$ be an $\relational{S}$-structure.\medskip\\
If $\struct{A}$ satisfies $(\Phi_1 \cup \{ \relational{\psi_c}, \relational{\psi_f}, \relational{\varphi_f} \})$, then $\invrelational{\struct{A}}$ is well-defined\footnote{If there is an $S$-structure $\struct{B}$ such that $\struct{A} = \relational{\struct{B}}$, then such $\struct{B}$ be must unique; in this case we write $\invrelational{\struct{A}}$ for $\struct{B}$.} and $\invrelational{\struct{A}} \models \varphi_f$ by \reftitle{VIII.1.3(a)}; so $A$ is finite and hence $\invrelational{\struct{A}} \models \Phi_0$. Using \reftitle{VIII.1.3(a)} again, we have that $\struct{A}$ is a model of $\relational{\Phi_0}$.\medskip\\
Conversely, if $\struct{A}$ satisfies $\relational{\Phi_0}$, then $\struct{A} \models (\relational{\psi_c} \land \relational{\psi_f})$ because $(\psi_c \land \psi_f) \in \Phi_0$ (note that $\consqn{\Phi_0} = \Phi_0$); thus $\invrelational{\struct{A}}$ is well-defined and $\invrelational{\struct{A}} \models \Phi_0$ by \reftitle{VIII.1.3(a)}. Since $\varphi_f \in \Phi_0$, $\struct{A} \models \relational{\varphi_f}$ because $\struct{A}$ is a model of $\relational{\Phi_0}$ by premise; also, $\invrelational{\struct{A}}$ is a model of $\varphi_f$, hence $A$ is finite. We have $\struct{A} \models \Phi_1$. In summary, $\struct{A}$ is a model of $\Phi_1 \cup \{ \relational{\psi_c}, \relational{\psi_f}, \relational{\varphi_f} \}$.\bigskip\\
From ($\circ$) and ($+$), it follows that\smallskip\\
\begin{quoteno}{($\ast$)}
$\consqn{(\Phi_1 \cup \{ \relational{\psi_c}, \relational{\psi_f}, \relational{\varphi_f} \})} = \relational{\Phi_0}$.
\end{quoteno}\bigskip\\
By our previous discussion, $\Phi_0$ is not enumerable. We conclude $\Phi_1$ also, is not enumerable: If $\Phi_1$ were enumerable, then $\consqn{(\Phi_1 \cup \{ \relational{\psi_c}, \relational{\psi_f}, \relational{\varphi_f} \})}$ and hence $\relational{\Phi_0}$ would as well be enumerable by \reftitle{Theorem X.6.3} and \reftitle{Exercise X.6.6}, and by ($\ast$). Let $\procp{P}$ be an enumeration procedure for $\relational{\Phi_0}$. We describe an enumeration procedure $\procp{Q}$ for $\Phi_0$, thus yielding a contradiction: For $n = 1, 2, 3, \ldots$, generate the lexicographically first $n$ first-order $S_0$-sentences $\seq[1]{\varphi}{n}$ ($\fstordlang[0]{S_0}$ is enumerable because $S_0$ is finite), and form $\seqp{\relational{\varphi_1}}{\relational{\varphi_n}}$ (this can be done effectively). On the other hand, use $\procp{P}$ to list the lexicographically first $n$ $\relational{S_0}$-sentences in $\relational{\Phi_0}$. Output every $\varphi_i$ for which $\relational{\varphi_i}$ occurs on that list (if any).
\end{proof}
We may choose $\relational{S_0}$ for $S_1$ in the proof of \reftitle{4.4} in text.\bigskip\\
\textit{Remark.} An alternative way to validate the claim is to show the results in \reftitle{X.5.2 - 4} are also valid for $\relational{S_0}$ using, for example, the sentence\smallskip\\
\centerline{$\psi_\prog^\prime \colonequals \relational{\psi_\prog} \land \exuni x \, Cx \land \forall x \exuni \, y Fxy$}
in place of $\psi_\prog$ in the proof of the result corresponding to \reftitle{X.5.3}.
%%
\item Note that in the direction from left to right in ($\circ$) in text, the $S_1$-structure $\substr{\intpted{W}{A}}{\reduct{\struct{A}}{S_1}}$ is well-defined since $S_1$ is relational.\bigskip\\
On the other hand, we prove the other direction in ($\circ$) here. Assume $\models_\logsys \chi \limply \relativize{(\varphi^\ast)}{W}$. Let $\struct{A}$ be an $S_1$-structure with $m \geq 1$ elements in the domain $A$. According to (ii), we may choose an $S^+$-structure $\struct{C}$ that is a model of $\chi$ in which $\intpted{W}{C}$ contains exactly $m$ elements. Then we take the $(S^+ \cup S_1)$-expansion $\struct{D}$ of $\struct{C}$ such that $\substr{\intpted{W}{D}}{\reduct{\struct{D}}{S_1}}$ is isomorphic to $\struct{A}$ (this is possible because both $A$ and $\intpted{W}{D}$ contain $m$ elements and $S_1$ is relational). Thus, $\struct{D}$ is also a model of $\chi$ by the reduct property (recall that $S_1$ is disjoint from $S^+$). By the assumption $\models_\logsys \chi \limply \relativize{(\varphi^\ast)}{W}$, it follows that $\struct{D} \models_\logsys \relativize{(\varphi^\ast)}{W}$ and hence $\substr{\intpted{W}{D}}{\reduct{\struct{D}}{S_1}} \models_\logsys \varphi^\ast$, according to the effective variant of $\rel{\logsys}$. Hence, $\struct{A} \models_\logsys \varphi^\ast$ by the isomorphism property, so $\struct{A} \models \varphi$. Therefore $\varphi$ is \finval.
%%
\item The computability of the replacement operation is not used in the proof, so this theorem can be generalized to logical systems $\logsys$ for which $\relational{\varphi}$ exists for any $\varphi \in \languagebase(S)$ provided that $S$ is decidable, regardless of whether the replacement operation can be carried out effectively.
%%
\end{asparaenum}
%
\item \textbf{The Ordering Resulting from $\pair{\nat}{{\intpted{<}{\nat}}}$ by Adding an Isomorphic Copy Is a Well-Ordering.} Let $\pair{A}{{\intpted{<}{A}}}$ and $\pair{B}{{\intpted{<}{B}}}$ be two $\{ < \}$-structures isomorphic to $\pair{\nat}{{\intpted{<}{\nat}}}$ with $A \cap B = \emptyset$. Choose the $\{ < \}$-structure $\struct{C}$ with $C = A \cup B$ and\smallskip\\
\centerline{${\intpted{<}{C}} = {\intpted{<}{A}} \cup {\intpted{<}{B}} \cup \setm{\pair{a}{b}}{a \in A, b \in B}$.}\smallskip\\
Obviously, $\struct{C}$ is a well-ordering.
%
\item \textbf{Arguing that All Finite Well-Orderings Are $\logsys$-Accessible Provided that $\fstordlog \weakereq \logsys$.} If $\pair{A}{\intpted{<}{A}}$ is a finite well-ordering with $n \geq 1$ elements in $A$, then $\pair{A}{\intpted{<}{A}}$ is a model of $\varphi$, where $\varphi$ is the conjunction of sentences in $\Phi_\ord$ and of $\exactly{n} x\, x \equal x$ (the models of which are exactly those orderings with $n$ elements). By taking a $\languagebase(S)$-sentence $\psi$ logically equivalent to $\varphi$, we are done.
%
\item \textbf{Arguing There Is No $\logsys$-Accessible Infinite Well-Ordering Provided that $\comp{\logsys}$.} For the sake of contradiction, suppose that $\psi \in \languagebase(S)$ satisfies the properties (a) and (b) mentioned on page 267 in which ${<} \in S$, and that $\struct{A}$ is an $S$-structure that satisfies $\psi$ where $\pair{\field{A}}{{\intpted{<}{A}}}$ is an infinite well-ordering. Also, let $c \not\in S$ be a constant symbol, and for every first-order $(S \cup \{ c \})$-sentence $\varphi$ we assign an $\languagebase(S \cup \{ c \})$-sentence $\varphi^\ast$ logically equivalent to $\varphi$ (since $\fstordlog \weakereq \logsys$); for sets $\Phi \subset \fstordlang[0]{S \cup \{ c \}}$ we denote $\Phi^\ast \colonequals \setm{\varphi^\ast}{\varphi \in \Phi}$.\bigskip\\
Let $\Psi \colonequals \{ \psi \} \cup \Phi_\pord^\ast \cup \setm{\varphi_n^\ast}{n > 0}$, where\smallskip\\
\centerline{$\varphi_n \colonequals \atleast{n} x \, x < c$}\smallskip\\
formulates ``there are at least $n$ elements smaller than $c$.''\bigskip\\
Then, every finite subset $\Psi_0$ of $\Psi$ is satisfiable: Let $n_0 > 0$ be the largest $n$ such that $\varphi_n \in \Psi_0$ (set $n_0 \colonequals 1$ if there is no such $n$). Since $\pair{\field{A}}{{\intpted{<}{A}}}$ is an infinite well-ordering, there must be some $a \in A$ such that there are at least $n_0$ elements smaller (in the sense of $\intpted{<}{A}$) than $a$. By $\intpted{c}{A} \colonequals a$, the $(S \cup \{ c \})$-structure $\pair{\struct{A}}{\intpted{c}{A}}$ is a model of $\Psi_0$.\bigskip\\
It follows from $\comp{\logsys}$ that $\Psi$ is satisfiable. Therefore, there is an $(S \cup \{ c \})$-structure $\pair{\struct{B}}{\intpted{c}{B}}$ satisfying $\Psi$; $\struct{B}$ is thus a model of $\psi$ (by the reduct property) in which $\pair{\field{B}}{{\intpted{<}{B}}}$ has an infinite descending chain, a contradiction.
%
\end{enumerate}
%End of Section XIII.4------------------------------------------------------------
%End of Chapter XIII--------------------------------------------------------------
\appendix
%%Appendix A---------------------------------------------------------------------------
\chapter{On Two Properties of First-Order Peano Axioms}
%Section A.1--------------------------------------------------------------------------
This appendix is aimed at giving complete proofs of two properties of first-order Peano axioms $\Phi_\pa$: One is \reftitle{Theorem X.7.3}, the other is ($\ast\ast\ast$) on page 185.
\section{Formal Arithmetic}
Recall that $\Phi_\pa$ consists of the following axioms:
\begin{enumerate}[({PA}1)]
%
\item $\forall x \neg x + 1 \equal 0$
%
\item $\forall x \forall y (x + 1 \equal y + 1 \limply x \equal y)$
%
\item $\forall x \ x + 0 \equal x$
%
\item $\forall x \forall y \ x + (y + 1) \equal (x + y) + 1$
%
\item $\forall x \ x \mul 0 \equal 0$
%
\item $\forall x \forall y \ x \mul (y + 1) \equal x \mul y + x$
%
\item (\emph{Axiom scheme}) for all $\seq[1]{x}{n}, y$ and all $\varphi \in \fstordlang{S_\ar}$ such that $\free{\varphi} \subset \setenum{\seq[1]{x}{n}, y}$ the sentence\\$\enump{\forall x_1}{\forall x_n} \parenadj{(\varphi\sbst{0}{y} \land \forall y (\varphi \limply \varphi\sbst{y + 1}{y})) \limply \forall y \varphi}$.
%
\end{enumerate}\ \smallskip\\
We shall prove some basic facts about arithmetic derivable from $\Phi_\pa$ in this section, which will be used several times throughout this appendix.\bigskip\\
\begin{theorem}{Proposition \thesection.1}
The following are derivable from $\Phi_\pa$:
\begin{enumerate}[\rm(a)]
%
\item\label{FA1} $\forall x \ 0 + x \equal x$
%
\item\label{FA2} $\forall x \ 1 + x \equal x + 1$
%
\item\label{FA3} $\forall x \ 0 \mul x \equal 0$
%
\item\label{FA4} $\forall x \ 1 \mul x \equal x$.
%
\end{enumerate}
\end{theorem}
\begin{proof}
\begin{inparaenum}[(a)]
%
\item First, we have $\Phi_\pa \derives 0 + 0 \equal 0$ by (PA3). Next, we obtain $\Phi_\pa \setsum \setenum{0 + x \equal x} \derives 0 + (x + 1) \equal x + 1$ by (PA4), and hence $\Phi_\pa \derives \forall x (0 + x \equal x \limply 0 + (x + 1) \equal x + 1)$. The claim follows by (PA7).\medskip\\
%
\item First, we have $\Phi_\pa \derives 1 + 0 \equal 0 + 1$ by (PA3) and (\ref{FA1}). Next, we obtain $\Phi_\pa \setsum \setenum{1 + x \equal x + 1} \derives 1 + (x + 1) \equal (x + 1) + 1$ by (PA4), and hence $\Phi_\pa \derives \forall x (1 + x \equal x + 1 \limply 1 + (x + 1) \equal (x + 1) + 1)$. The claim follows by (PA7).\medskip\\
%
\item First, we have $\Phi_\pa \derives 0 \mul 0 \equal 0$ by (PA5). Next, we have $\Phi_\pa \setsum \setenum{0 \mul x \equal 0} \derives 0 \mul (x + 1) \equal 0$ by (PA6) and (PA3), so $\Phi_\pa \derives \forall x (0 \mul x \equal 0 \limply 0 \mul (x + 1) \equal 0)$. The claim follows by (PA7).\medskip\\
%
\item First, we have $\Phi_\pa \derives 1 \mul 0 \equal 0$ by (PA5). Next, we have $\Phi_\pa \setsum \setenum{1 \mul x \equal x} \derives 1 \mul (x + 1) \equal x + 1$ by (PA6), so $\Phi_\pa \derives \forall x (1 \mul x \equal x \limply 1 \mul (x + 1) \equal x + 1)$. The claim follows by (PA7).
%
\end{inparaenum}
\end{proof}\ \medskip\\
\begin{theorem}{Lemma on the Associativity of Addition \thesection.2} The sentence
\[
\forall x \forall y \forall z \ x + (y + z) \equal (x + y) + z.
\]
is derivable from $\Phi_\pa$.
\end{theorem}
\begin{proof}
First, $\Phi_\pa \derives x + (y + 0) \equal (x + y) + 0$ by (PA3), and it follows that $\Phi_\pa \derives \forall x \forall y \ x + (y + 0) \equal (x + y) + 0$. Next, $\Phi_\pa \setsum \setenum{x + (y + z) \equal (x + y) + z} \derives x + (y + (z + 1)) \equal (x + y) + (z + 1)$ by (PA4), so we have $\Phi_\pa \derives \forall x \forall y \forall z \ (x + (y + z) \equal (x + y) + z \limply x + (y + (z + 1)) \equal (x + y) + (z + 1))$. The claim follows by (PA7).
\end{proof}\ \medskip\\
By the above lemma, the parentheses in $(x + y) + z$ or $x + (y + z)$ are superfluous. Thus, we shall write $x + y + z$ from now on for brevity.\bigskip\\
\begin{theorem}{Lemma on the Commutativity of Addition \thesection.3} The sentence
\[
\forall x \forall y \ x + y \equal y + x
\]
is derivable from $\Phi_\pa$.
\end{theorem}
\begin{proof}
First, $\Phi_\pa \derives x + 0 \equal 0 + x$ by (PA3) and \reftitle{Proposition \thesection.1(\ref{FA1})}, and it follows that $\Phi_\pa \derives \forall x \ x + 0 \equal 0 + x$. Next, $\Phi_\pa \setsum \setenum{x + y \equal y + x} \derives x + (y + 1) \equal (y + 1) + x$ by (PA4), \reftitle{Proposition \thesection.1(\ref{FA2})} and \reftitle{Lemma \thesection.2}, so we have $\Phi_\pa \derives \forall x \forall y (x + y \equal y + x \limply x + (y + 1) \equal (y + 1) + x)$. The claim follows by (PA7).
\end{proof}\ \medskip\\
\begin{theorem}{Lemma on the Distribution Laws \thesection.4} The following sentences are derivable from $\Phi_\pa$:
\begin{enumerate}[\rm(a)]
%
\item\label{DLR} $\forall x \forall y \forall z \ x \mul (y + z) \equal x \mul y + x \mul z$
%
\item\label{DLL} $\forall x \forall y \forall z \ (x + y) \mul z \equal x \mul z + y \mul z$.
%
\end{enumerate}
\end{theorem}
\begin{proof}
\begin{inparaenum}[(a)]
%
\item First, $\Phi_\pa \derives x \mul (y + 0) \equal x \mul y + x \mul 0$ by (PA3) and (PA5), and it follows that $\Phi_\pa \derives \forall x \forall y \ x \mul (y + 0) \equal x \mul y + x \mul 0$. Next, $\Phi_\pa \setsum \setenum{x \mul (y + z) \equal x \mul y + x \mul z} \derives x \mul (y + (z + 1)) \equal x \mul y + x \mul (z + 1)$ by (PA4), (PA6) and \reftitle{Lemma \thesection.2}, so we have $\Phi_\pa \derives \forall x \forall y \forall z (x \mul (y + z) \equal x \mul y + x \mul z \limply x \mul (y + (z + 1)) \equal x \mul y + x \mul (z + 1))$. The claim follows by (PA7).\medskip\\
%
\item First, $\Phi_\pa \derives (x + y) \mul 0 \equal x \mul 0 + y \mul 0$ by (PA3) and (PA5), and it follows that $\Phi_\pa \derives \forall x \forall y \ (x + y) \mul 0 \equal x \mul 0 + y \mul 0$. Next, $\Phi_\pa \setsum \setenum{(x + y) \mul z \equal x \mul z + y \mul z} \derives (x + y) \mul (z + 1) \equal x \mul (z + 1) + y \mul (z + 1)$ by (PA6), \reftitle{Lemmas \thesection.2} and \reftitle{\thesection.3}, so we have $\Phi_\pa \derives \forall x \forall y \forall z ((x + y) \mul z \equal x \mul z + y \mul z \limply (x + y) \mul (z + 1) \equal x \mul (z + 1) + y \mul (z + 1))$. The claim follows by (PA7).
\end{inparaenum}
\end{proof}\ \medskip\\
\begin{theorem}{Lemma on the Associativity of Multiplication \thesection.5} The sentence
\[
\forall x \forall y \forall z \ x \mul (y \mul z) \equal (x \mul y) \mul z
\]
is derivable from $\Phi_\pa$.
\end{theorem}
\begin{proof}
First, $\Phi_\pa \derives x \mul (y \mul 0) \equal (x \mul y) \mul 0$ by (PA5), and it follows that $\Phi_\pa \derives \forall x \forall y \ x \mul (y \mul 0) \equal (x \mul y) \mul 0$. Next, $\Phi_\pa \setsum \setenum{x \mul (y \mul z) \equal (x \mul y) \mul z)} \derives x \mul (y \mul (z + 1)) \equal (x \mul y) \mul (z + 1)$ by (PA6) and \reftitle{Lemma \thesection.4(\ref{DLR})}, so we have $\Phi_\pa \derives \forall x \forall y \forall z (x \mul (y \mul z) \equal (x \mul y) \mul z \limply x \mul (y \mul (z + 1)) \equal (x \mul y) \mul (z + 1))$. The claim follows by (PA7).
\end{proof}\ \medskip\\
By the above lemma, the parentheses in $(x \mul y) \mul z$ or $x \mul (y \mul z)$ are superfluous. Thus, we shall write $x \mul y \mul z$ from now on for brevity.\bigskip\\
\begin{theorem}{Lemma on the Commutativity of Multiplication \thesection.6} The sentence
\[
\forall x \forall y \ x \mul y \equal y \mul x
\]
is derivable from $\Phi_\pa$.
\end{theorem}
\begin{proof}
First, $\Phi_\pa \derives x \mul 0 \equal 0 \mul x$ by (PA5) and \reftitle{Proposition \thesection.1(\ref{FA3})}. Next, $\Phi_\pa \setsum \setenum{x \mul y \equal y \mul x} \derives x \mul (y + 1) \equal (y + 1) \mul x$ by (PA6), \reftitle{Proposition \thesection.1(\ref{FA4})} and \reftitle{Lemma \thesection.4(\ref{DLL})}, so we have $\Phi_\pa \derives \forall x \forall y (x \mul y \equal y \mul x \limply x \mul (y + 1) \equal (y + 1) \mul x)$. The claim follows by (PA7).
\end{proof}\ \medskip\\
\begin{theorem}{Proposition \thesection.7}
The sentence\\
\centerline{$\forall x (\neg x \equal 0 \liff \exists y \ x \equal y + 1)$}\\
is derivable from $\Phi_\pa$.
\end{theorem}
\begin{proof}
According to (PA1), it suffices to show $\forall x (\neg x \equal 0 \limply \exists y \ x \equal y + 1)$.\medskip\\
First, it is trivially true that $\Phi_\pa \derives (\neg 0 \equal 0 \limply \exists y \ 0 \equal y + 1)$. Next, the formula $\exists y \ x + 1 \equal y + 1$ is trivially derivable from $\Phi_\pa$, so are the formulas $\neg x + 1 \equal 0 \limply \exists y \ x + 1 \equal y + 1$ and $(\neg x \equal 0 \limply \exists y \ x \equal y + 1) \limply (\neg x + 1 \equal 0 \limply \exists y \ x + 1 \equal y + 1)$. The claim follows by (PA7).
\end{proof}\ \medskip\\
\begin{theorem}{Lemma on Cancellation Laws \thesection.8}
The following sentences are derivable from $\Phi_\pa$:
\begin{enumerate}[\rm(a)]
%
\item $\forall x \forall y \forall z (x + z \equal y + z \limply x \equal y)$
%
\item $\forall x \forall y \forall z ((\neg z \equal 0 \land x \mul z \equal y \mul z) \limply x \equal y)$.\\
%
\end{enumerate}
\end{theorem}
\begin{proof}
\begin{inparaenum}[(a)]
%
\item First, $\Phi_\pa \derives \forall x \forall y (x + 0 \equal y + 0 \limply x \equal y)$ by (PA3). Next, $\Phi_\pa \setsum \setenum{\forall x \forall y (x + z \equal y + z \limply x \equal y)} \derives \forall x \forall y (x + (z + 1) \equal y + (z + 1) \limply x \equal y)$ by (PA4), so we have $\Phi_\pa \derives \forall z (\forall x \forall y (x + z \equal y + z \limply x + y) \limply \forall x \forall y (x + (z + 1) \equal y + (z + 1) \limply x \equal y))$. The claim follows by (PA7).\medskip\\
%
\item According to \thesection.7, it suffices to show $\forall x \forall y \forall z (x \mul (z + 1) \equal y \mul (z + 1) \limply x \equal y)$.\medskip\\
First, $\Phi_\pa \derives \forall x \forall y (x \mul (0 + 1) \equal y \mul (0 + 1) \limply x \equal y)$ by (PA5), (PA6) and \reftitle{Proposition \thesection.1(\ref{FA1}). Next, 
%
\end{inparaenum}
\end{proof}\ \medskip\\
For convenience, we shall write $\numl{n}$ for $\underbrace{1 \enump{+}{+} 1}_{\mbox{\scriptsize\(n\)-times}}$. In particular, $\numl{0}$ will stand for $0$.\bigskip\\
\begin{theorem}{Proposition \thesection.9}
\begin{enumerate}[\rm(a)]
%
\item\label{OPADD} $\Phi_\pa \derives \numl{m} + \numl{n} \equal \numl{m + n}$
%
\item\label{OPMUL} $\Phi_\pa \derives \numl{m} \mul \numl{n}  \equal \numl{m \mul n}$
%
\end{enumerate}
\end{theorem}
\begin{proof} We use induction on $n$ in both cases.\medskip\\
\begin{inparaenum}[(a)]
%
\item The base case $n = 0$ is trivial by (PA3). For the induction step $n = k + 1$, assume that $\Phi_\pa \derives \numl{m} + \numl{k} \equal \numl{m + k}$. Then $\Phi_\pa \derives \numl{m} + \numl{k + 1} \equal \numl{m + (k + 1)}$ by induction hypothesis and \reftitle{Lemma \thesection.2}.\medskip\\
%
\item The base case $n = 0$ is trivial by (PA5). For the induction step $n = k + 1$, assume that $\Phi_\pa \derives \numl{m} \mul \numl{k} \equal \numl{m \mul k}$. Then $\Phi_\pa \derives \numl{m} \mul \numl{k + 1} \equal \numl{m \mul (k + 1)}$ by induction hypothesis, (PA6) and part (\ref{OPADD}).
%
\end{inparaenum}
\end{proof}\ \medskip\\
We say that $t \in \term{S_\ar}$ is \emph{variable-free} if there is no variable in it. For example, $\numl{2} \mul \numl{3} + \numl{1}$ and $(\numl{0} + \numl{2}) \mul \numl{1}$ are variable-free terms. Note that if $t$ is variable-free, then there is an $n \in \nat$ such that $\natstr \models t \equal \numl{n}$.\bigskip\\
Using the above proposition, we immediately obtain:\medskip\\
\begin{theorem}{Lemma \thesection.10}
Let $t \in \term{S_\ar}$ be variable-free. Then $\Phi_\pa \derives t \equal \numl{n}$ provided that $\natstr \models t \equal \numl{n}$.\qed
\end{theorem}\medskip\\
\begin{theorem}{Lemma \thesection.11}
If $m \neq n$, then $\Phi_\pa \derives \neg \numl{m} \equal \numl{n}$.
\end{theorem}
\begin{proof}
We only prove the case in which $n < m$, the case $m < n$ can be addressed similarly: We have $\numl{m - n} = \numl{m - n - 1} + 1$, so $\Phi_\pa \derives \neg \numl{0} \equal \numl{m - n}$ by (PA1); repeatedly applying (PA2) $n$ times, we obtain $\Phi_\pa \derives \neg \numl{m} \equal \numl{n}$.
\end{proof}\ \medskip\\
By the above two lemmas, we immediately obtain:\medskip\\
\begin{theorem}{Lemma \thesection.12}
Let $t_1, t_2 \in \term{S_\ar}$ be variable-free. Then either $\Phi_\pa \derives t_1 \equal t_2$ or $\Phi_\pa \derives \neg t_1 \equal t_2$. Moreover, $\natstr \models t_1 \equal t_2$ \quad iff \quad $\Phi_\pa \derives t_1 \equal t_2$.\qed
\end{theorem}\ \bigskip\\
For the goal of this appendix, we shall discover a special set of $S_\ar$-formulas, the so-called \emph{$\Sigma_1$-formulas}. Before that, we introduce the following abbreviations:\medskip\\
\begin{definition}{Abbreviations}
From now on, we use $t_1 \leq t_2$ to abbreviate\\
\centerline{$\exists x \ t_1 + x \equal t_2$,}\\
where $x \not\in \var{t_1} \setsum \var{t_2}$. Also,\\
\centerline{$t_1 \leq t_2 \land \neg t_1 \equal t_2$ }\\
is abbreviated by $t_1 < t_2$.
\end{definition}\bigskip\\
\begin{definition}{Abbreviations}
We write $(\exists x \leq t) \varphi$ and $(\exists x < t) \varphi$ for\\
\centerline{$\exists x (x \leq t \land \varphi)$}\\
and\\
\centerline{$\exists x (x < t \land \varphi)$,}\\
respectively. Also, $(\forall x \leq t) \varphi$ will stand for $\neg(\exists x \leq t) \neg\varphi$ and $(\forall x < t) \varphi$ for $\neg(\exists x < t) \neg\varphi$.
\end{definition}\bigskip\\
\begin{theorem}{Propositions (That Has Not Yet Been Proven)}
The following are derivable from $\Phi_\pa$:
\begin{enumerate}[\rm(1)]
%
\item $0 \leq x$
%
\item $(x < y \land y < z) \limply x < z$
%
\item $\neg x < x$
%
\item $(x \leq y \land y \leq x) \limply x \equal y$
%
\item $x < y \limply x + 1 < y + 1$
%
\item $x < y \liff x + 1 \leq y$
%
\item $x < y \lor x \equal y \lor y < x$
%
\item $x < y \limply x + z < y + z$
%
\item $x + z < y + z \limply x < y$
%
\item $0 < z \limply x + 1 \leq x + z$
%
\item $(0 < z \land x < y) \limply x \mul z < y \mul z$
%
\item $(0 < z \land x \mul z \equal y \mul z) \limply x \equal y$
%
\item $(x < y \land z < w) \limply x + z < y + w$
%
\item $(x < y \land z < w) \limply x \mul z < y \mul w$
%
\end{enumerate}
\end{theorem}\ \medskip\\
The following proposition, which can be proven easily, will turn out to be useful and will be referred to throughout this appendix.\bigskip\\
\begin{theorem}{Proposition}
The following are derivable from $\Phi_\pa$:
\begin{enumerate}[\rm(a)]
%
\item\label{OR4} $\forall x \forall y (x + y \equal x \limply y \equal 0)$.
%
\item\label{OR5} $\forall x \forall y (x < y \limply \neg x \equal y)$.
%
\item\label{OR6} $(\forall x \leq \numl{n}) \blor\limits^n_{i = 0} x \equal \numl{i}$.
%
\item $\forall x (\neg x \equiv 0 \rightarrow \exists y \ x \equiv y + 1)$.
%
\item $\forall x (1 + x \equiv x + 1)$.
%
\item $\forall x (x \equiv 0 \lor 0 < x)$.
%
\item $\forall x \forall y (\neg x \equiv y \rightarrow (x < y \lor y < x))$.
%
\item $\forall x (\neg x \equiv \mbf{n} \rightarrow (x < \mbf{n} \lor \mbf{n} < x))$, for $n \in \nat$.
%
\item $\forall x \forall y (x < y \rightarrow \neg(x \equiv y \lor y < x))$.
%
\item $\forall x \forall y \forall z (x + z \equiv y + z \rightarrow x \equiv y)$.
%
\end{enumerate}
\end{theorem}
\begin{proof}
\begin{inparaenum}[(a)]
%
\item Immediately follows from definition.\medskip\\
%
\item Obviously $\Phi_\pa \vdash 0 < \mbf{n + 1} \rightarrow \bigvee\limits^n_{i = 0} 0 \equiv \mbf{n}$. Next, by definition $\Phi_\pa \cup \{ x + 1 < \mbf{n + 1} \} \vdash \exists y (\neg y \equiv 0 \land (x + 1) + y \equiv \mbf{n + 1})$, with (c) and other axioms we have $\Phi_\pa \cup \{ x + 1 < \mbf{n + 1} \} \vdash x < \mbf{n + 1}$. This together with (e) entails $\Phi_\pa \cup \{ x < \mbf{n + 1} \rightarrow \bigvee\limits^n_{i = 0} x \equiv \mbf{i} \} \vdash x + 1 < \mbf{n + 1} \rightarrow \bigvee\limits^n_{i = 0} x + 1 \equiv \mbf{i}$. Induction schema yields the result.
%
\end{inparaenum}
\end{proof}
\begin{thebibliography}{10}
\bibitem{Thomas_Jech} Thomas Jech \textsl{Set Theory} the third millennium edition, revised and expanded, Springer, 2000.
%
\bibitem{Kenneth_Kunen} Kenneth Kunen \textsl{Set Theory: An Introduction to Independence Proofs}, Springer, 1980.
%
\bibitem{Edmund_Landau} Edmund Landau \textsl{Foundations of Analysis}, Columbia University, 1951.
%
\bibitem{Dirk_van_Dalen} Dirk van Dalen \textsl{Logic and Structure}, 4th edition, Springer, 2008.
%
\bibitem{Wolfgang_Rautenberg} Wolfgang Rautenberg \textsl{A Concise Introduction to Mathematical Logic}, 3rd edition, Springer, 2011.
%
\bibitem{Christos_Papadimitriou} Christos H. Papadimitriou \textsl{Computational Complexity}, Addison Wesley, 1993.
%
\bibitem{Heinz_Dieter_Ebbinghaus_and_Jorg_Flum} Heinz-Dieter Ebbinghaus and J\"{o}rg Flum \textsl{Finite Model Theory}, 2nd edition, Springer, 2002.
\end{thebibliography}
\end{document}