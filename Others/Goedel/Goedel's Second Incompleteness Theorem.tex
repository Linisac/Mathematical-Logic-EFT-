%Author: Wei-Lin (Linisac) Wu
\documentclass[11pt, leqno]{report}


%Packages
\usepackage{amssymb}
\usepackage{enumerate}
\usepackage{graphicx}
\usepackage{stmaryrd}
\usepackage{colonequals}
\usepackage{amsthm}
\usepackage{array}
\usepackage{amsmath}
\usepackage{amscd}
\usepackage{paralist}
\usepackage{bm}
\usepackage{titlesec}


%New Commands for Abbreviations
%%General
\newcommand{\setenum}[1]{\{#1\}} %the set of which the elements are enumerated by #1
\newcommand{\setsum}{\cup} %`A \setsum B' stands for the union of sets A and B
\newcommand{\setprod}{\cap} %`A \setprod B' stands for the intersection of sets A and B
\newcommand{\bsetsum}{\bigcup} %big operator of union
\newcommand{\bsetprod}{\bigcap} %big operator of intersection
\newcommand{\nat}{\mathbb{N}} %the set of natural numbers
\newcommand{\zah}{\mathbb{Z}} %the set of integers
\newcommand{\rat}{\mathbb{Q}} %the set of rational numbers
\newcommand{\real}{\mathbb{R}} %the set of real numbers
\newcommand{\cplx}{\mathbb{C}} %the set of complex numbers
\newcommand{\powerset}[1]{{\mathcal{P}\left(#1\right)}} %the power set of #1
\newcommand{\absval}[1]{{\left| #1 \right|}} %the absolute value of #1
\newcommand{\card}[1]{\absval{#1}} %the cardinality of the set #1
\newcommand{\limply}{\mathbin{\rightarrow}} %logical connective imply
\newcommand{\liff}{\mathbin{\leftrightarrow}} %logical connective if and only if
\newcommand{\blor}{\bigvee} %big operator or
\newcommand{\bland}{\bigwedge} %big operator and
\newcommand{\Iff}{\mbox{iff}} %!!
\newcommand{\scripttext}[1]{{\mbox{\scriptsize#1}}} %text of script size in math mode
\newcommand{\mbf}[1]{{\mbox{\boldmath\begin{math}#1\end{math}}}} %bold face in math mode
\newcommand{\mbfs}[1]{{\mbox{\scriptsize\boldmath\begin{math}\mathrm{#1}\end{math}}}} %bold face of script size in math mode
\newcommand{\mbff}[1]{{\mbox{\footnotesize\boldmath\begin{math}\mathrm{#1}\end{math}}}} %bold face of footnote size in math mode
\newcommand{\mbft}[1]{{\mbox{\tiny\boldmath\begin{math}\mathrm{#1}\end{math}}}} %bold face of tiny size in math mode
\newcommand{\formal}[1]{\bm{#1}} %formal version of the symbol #1
\newcommand{\reftitle}[1]{{\rm #1}} %reference title
\newcommand{\parenadj}[1]{\left(#1\right)} %adjustable parentheses
\newcommand{\brackadj}[1]{\left[#1\right]} %adjustable (square) brackets
\newcommand{\braceadj}[1]{\left\lbrace #1\right\rbrace} %adjustable (curly) braces
\newcommand{\angleadj}[1]{\left\langle #1 \right\rangle} %adjustable angular brackets
\newcommand{\opintv}[2]{\parenadj{#1, #2}} %the open interval (#1, #2)
\newcommand{\clintv}[2]{\brackadj{#1, #2}} %the closed interval [#1, #2]
\newcommand{\olintv}[2]{\left(#1, #2\right]} %the left-open right-closed interval (#1, #2]
\newcommand{\cpintv}[2]{\left[#1, #2\right)} %the left-closed right-open interval [#1, #2)
\newcommand{\pair}[2]{(#1, #2)} %the pair of #1 and #2
\newcommand{\tuple}[1]{(#1)} %tuple of #1; often used with \seq
\newcommand{\pairadj}[2]{\parenadj{#1, #2}} %the pair of #1 and #2 with adjustable parentheses
\newcommand{\tupleadj}[1]{\parenadj{#1}} %tuple of #1 with adjustable parentheses; often used with \seq
\newcommand{\seq}[3][0]{{#2}_{#1}, \ldots, {#2}_{#3}} %sequence; \seq[1]{a}{n} means the sequence a_1, \ldots, a_n
%\newcommand{\seqv}[2]{#1, \ldots, #2} %already abolished
\newcommand{\seqp}[2]{{#1}, \ldots, {#2}} %sequence with additional parameters; \seqv{a_1}{a_n} means the sequence a_1, \ldots, a_n
\newcommand{\seqi}[2]{(#1)_{#2}} %sequence #1 indexed by the set #2
\newcommand{\enum}[3][0]{{{#2}_{#1} \ldots {#2}_{#3}}} %enumeration; \enum[1]{t}{n} means the enumeration t_1 \ldots t_n
%\newcommand{\enumv}[2]{#1 \ldots #2} %already abolished
\newcommand{\enump}[2]{{{#1} \ldots {#2}}} %enumeration with parameter(s); \enump{a_1}{a_n} means the enumeration a_ \ldots a_n
\newcommand{\enumpop}[3]{#1 #2 \ldots #2 #3} %enumeration with operators; \enumpop{a}{op}{b} means repeated applications of the operator op to the sequence a, \ldots, b. For example, \enumpop{a}{+}{b} is a + ... + b.
%\newcommand{\enumprel}[3]{#1 #2 \cdots #2 #3} %??
\newcommand{\setm}[2]{\{ #1 \mid #2 \}} %set with description in math mode; \setm{a}{b} means the set \{ a \mid b \}
\newcommand{\sett}[2]{\{ #1 \mid \mbox{#2} \}} %set with description in text mode; \sett{a}{b} means the set \{ a \mid b \} with b a text
\newcommand{\restrict}[2]{{#1|_{#2}}} %the restriction of the function #1 to the domain #2
\newcommand{\fun}[1]{\mathrm{#1}} %\fun{a} displays the function name a in roman face
%%Chapter 2
\newcommand{\alphabet}{\mathcal{A}} %the general alphabet
\newcommand{\kleene}[1]{{#1}^\ast} %Kleene star operator
\newcommand{\nullstring}{\Box} %null, or empty, string
%\newcommand{\symbolset}{\mathit{S}} %a symbol set
\newcommand{\equal}{\equiv} %equal sign in object languages
\newcommand{\absymb}[1]{\alphabet_{#1}} %the alphabet for the first-order language determined by the symbol set #1 (see Section II.2)
\newcommand{\gr}{\mathrm{gr}} %an abbreviation for group theory
\newcommand{\eqr}{\mathrm{eq}} %an abbreviation for equivalence relation theory
\newcommand{\univsymb}{S_\infty} %the universal symbol set (see Section II.2)
\newcommand{\termbase}{\mathit{T}} %the base symbol for sets of terms
\newcommand{\term}[2][]{{\termbase_{#1}^{#2}}} %the set of terms
\newcommand{\languagebase}{\mathit{L}} %the base symbol for a (especially first-order) language
\newcommand{\fstordlang}[2][]{{\languagebase_{#1}^{#2}}} %first-order language
\newcommand{\fstordfmla}[2][0]{{\languagebase_{#1}^{#2}}} %!!
\newcommand{\calrule}[2]{{\displaystyle\frac{\;#1\;}{\;#2\;}}} %calculus rule in schema form
%%Chapter 3
\newcommand{\mul}{\cdot} %the multiplication operator used in arithemtics
\newcommand{\ar}{\mathrm{ar}} %the abbreviation for arithmetics
\newcommand{\arsymb}{S_\ar} %the symbol set for arithmetics

\newcommand{\grp}{\mathrm{grp}} %an abbreviation for group theory
\newcommand{\natstr}{\mathfrak{N}} %the structure (\nat, +, \cdot, 0, 1)
\newcommand{\zahstr}{\mathfrak{Z}} %the structure (\zah, +, \cdot, 0, 1)
\newcommand{\realstr}{\mathfrak{R}} %the structure (\real, +, \cdot, 0, 1)
\newcommand{\negfunc}{\mathord{\overset{.}{\neg}}} %the negation function in the metalanguage
\newcommand{\dsjfunc}{\mathbin{\overset{.}{\lor}}} %the disjunction function in the metalanguage
\newcommand{\cnjfunc}{\mathbin{\overset{.}{\land}}} %the conjunction function in the metalanguage
\newcommand{\impfunc}{\mathbin{\overset{.}{\rightarrow}}} %the implication function in the metalanguage
\newcommand{\bimfunc}{\mathbin{\overset{.}{\leftrightarrow}}} %the bi-implication function in the metalanguage
\newcommand{\sat}{\mathrel{\mathrm{Sat}}} %the relation satisfiable
\newcommand{\modeled}{\ \mathrm{\reflectbox{$\models$}}} %!!
\newcommand{\bimodels}{\mathrel{\mathord{\reflectbox{$\models$}}\mathord{\models}}} %to be \logequiv
\newcommand{\logequiv}{\mathrel{\mathord{\reflectbox{$\models$}}\mathord{\models}}} %the relation logically equivalent
%\newcommand{\STR}[1]{\mathfrak{#1}} %already abolished
\newcommand{\struct}[1]{\mathfrak{#1}} %the structure of #1
\newcommand{\intpted}[2]{{#1}^{#2}} %function, constant or relation symbol #1 interpreted as such in the domain of structure #2
\newcommand{\reduct}[2]{{#1|_{#2}}} %the #2-reduct of (the structure) #1
\newcommand{\assgn}{\beta} %a (first-order) assignment
\newcommand{\INT}{\mathfrak{I}} %to be \intp
\newcommand{\intp}{\mathfrak{I}} %an interpretation
\newcommand{\INTP}[2]{(#1, #2)} %to be \intparg
\newcommand{\intpp}[2]{(#1, #2)} %to be \intparg
\newcommand{\intparg}[2]{\pair{#1}{#2}} %the interpretation of (#1, #2), where the arguements #1 and #2 are a structure and an assignment in #1, respectively
\newcommand{\substr}[2]{{[#1]^{#2}}} %substructure generated by #1 in #2
\newcommand{\iso}{\cong} %isomorphic; DEPENDENCY: \finiso, \partiso
\newcommand{\ord}{\mathrm{ord}} %abbreviation for ordering
\newcommand{\field}[1]{\mathop{\mathrm{field}} \intpted{<}{#1}} %the field of < in the structure #1
\newcommand{\pord}{\mathrm{pord}} %abbreviation for partially defined ordering
\newcommand{\suc}{\sigma} %the successor function over \nat
\newcommand{\natsuc}{{\natstr_\suc}} %the structure (\nat, successor, 0)
\newcommand{\sndordpeanoarith}{\Pi} %second-order Peano arithmetics
\newcommand{\df}[2]{\displaystyle\frac{#1}{#2}} %fraction in displaystyle
\newcommand{\varbase}{\mathrm{var}} %the base symbol for the function var; DEPENDENCY: \var
\newcommand{\var}[1]{{\varbase(#1)}} %the set of variables occurring in the term #1
\newcommand{\freebase}{\mathrm{free}} %the base symbol for the function free; DEPENDENCY: \free
\newcommand{\free}[1]{{\freebase(#1)}} %the set of free variables occurring in the formula #1
\newcommand{\SF}[1]{\mathop{\mathrm{SF}}(#1)} %to be \sbfmla
\newcommand{\sbfmlabase}{\mathrm{SF}} %the base symbol for the function sbfmla; DEPENDENCY: \sbfmla
\newcommand{\sbfmla}[1]{{\sbfmlabase(#1)}} %the set of subformulas of the formula #1
\newcommand{\sbst}[2]{{\scriptstyle\frac{\displaystyle #1}{\displaystyle #2}}} %the substitution operation
\newcommand{\scriptsbst}[2]{{\scriptscriptstyle\frac{\scriptstyle #1}{\scriptstyle #2}}} %the substitution used in script style
\newcommand{\exactly}[1]{\exists^{\mathrel{=} #1}} %there are exactly #1 element(s) such that
\newcommand{\exuni}{\exists^{=1}} %there is a unique element such that
\newcommand{\atmost}[1]{\exists^{\mathrel{\leq} #1}} %there are at most #1 element(s) such that
\newcommand{\atleast}[1]{\exists^{\mathrel{\geq} #1}} %there are at least #1 element(s) such that
%%Chapter 4
\newcommand{\derives}{\vdash} %the relation formally derives
\newcommand{\derive}{\ | \hspace{-.4em} -} %an alternative symbol for \vdash
\newcommand{\assm}{{(\mathrm{Assm})}} %the sequent rule (Assm)
\newcommand{\ant}{{(\mathrm{Ant})}} %the sequent rule (Ant)
\newcommand{\pc}{{(\mathrm{PC})}} %the sequent rule (PC)
\newcommand{\ctr}{{(\mathrm{Ctr})}} %the sequent rule (Ctr)
\newcommand{\ora}{{(\lor\mathrm{A})}} %the sequent rule (\lor A)
\newcommand{\ors}{{(\lor\mathrm{S})}} %the sequent rule (\lor S)
\newcommand{\ea}{{(\exists\mathrm{A})}} %the sequent rule (\exists A)
\newcommand{\es}{{(\exists\mathrm{S})}} %the sequent rule (\exists S)
\newcommand{\eq}{{(\equiv)}} %the sequent rule (\equal)
\newcommand{\sub}{{(\mathrm{Sub})}} %the sequent rule (Sub)
\newcommand{\seqcal}{{\mathfrak{S}}} %the sequent calculus \mathfrak{S}
\newcommand{\con}{\mathrel{\mathrm{Con}}} %the relation consistent
\newcommand{\inc}{\mathrel{\mathrm{Inc}}} %the relation inconsistent
%%Chapter 6
\newcommand{\modelclassbase}{\mathrm{Mod}} %the base symbol for \modelclass and \modelclassarg; DEPENDENCY: \modelclass, \modelclassarg
\newcommand{\modelclass}[2]{{\mathop{\modelclassbase^{#1}} #2}} %the set of #1-structures that are models of the #1-formula #2
\newcommand{\thr}[1]{\mathop{\mathrm{Th}}(#1)} %to be \theoarg
\newcommand{\Th}{\mathop{\mathrm{Th}}} %!!
\newcommand{\theorybase}{\mathrm{Th}} %the base symbol for a theory
\newcommand{\theoarg}[1]{{\theorybase(#1)}} %the theory of #1, which takes argument(s), such as Th(\mathfrak{A})
%%Chapter 7
\newcommand{\zfc}{\mathrm{ZFC}} %the abbreviation for Zermelo-Fraenkel set theory with the axiom of choice
%%Chapter 8
\newcommand{\relational}[1]{#1^\mathit{r}} %the relational symbol set corresponding to #1
\newcommand{\invrelational}[1]{#1^{-\mathit{r}}} %the inverse operation of \relational to #1
\newcommand{\relativize}[2]{#1^{#2}} %the relativization of #1 to #2
%%Chapter 9
\newcommand{\logicalsystembase}{\mathcal{L}} %the base symbol for logical systems
\newcommand{\firstorder}{{\mathrm{I}}} %the abbreviation for first-order
\newcommand{\secondorder}{{\mathrm{II}}} %the abbreviation for second-order
\newcommand{\sndordassgn}{\gamma} %a second-order assignment
\newcommand{\freeII}{\free_\mathrm{II}} %to be replaced by \sndordfree
\newcommand{\sndordfree}[1]{{\freebase_\secondorder(#1)}} %the set of free relation variables occurring in the formula #1
\newcommand{\FOL}{\mathcal{L}_\mathrm{I}} %to be \fstordlog
\newcommand{\fstordlog}{{\logicalsystembase_\firstorder}} %the logical system first-order logic
\newcommand{\SOL}{\mathcal{L}_\mathrm{II}} %to be \sndordlog
\newcommand{\sndordlog}{{\logicalsystembase_\secondorder}} %the logical system second-order logic
\newcommand{\LII}{L_\mathrm{II}} %to be \sndordlang
\newcommand{\sndordlang}[1]{{\languagebase_\secondorder^{#1}}} %the second-order language with symbol set #1
\newcommand{\weak}{{\mathit{w}}}
\newcommand{\weaksndordlog}{{\logicalsystembase^\weak_\secondorder}} %weak second-order logic
\newcommand{\weaksndordlang}[1]{{\languagebase^{\weak, #1}_\secondorder}} %the weak second-order language with symbol set #1
\newcommand{\INFL}{\mathcal{L}_{\omega_1\omega}} %to be \infinlog
\newcommand{\infinlog}{{\logicalsystembase_{\omega_1\omega}}} %the logical system infinitary logic \omega_1\omega
\newcommand{\LINF}{L_{\omega_1\omega}} %to be replaced by \infinlang
\newcommand{\infin}{{\omega_1\omega}} %the modifier infinitary \omega_1 \omega
\newcommand{\infinlang}[1]{{\languagebase_\infin^{#1}}} %the infinitary language \omega_1\omega with symbol set #1
\newcommand{\qexist}{\mathit{Q}} %the Q quantifier in Q system
\newcommand{\QL}{\mathcal{L}_Q} %to be replaced by \qlog
\newcommand{\qlog}{{\logicalsystembase_\qexist}} %the logical system Q logic
\newcommand{\LQ}{L_Q} %to be replaced by \qlang
\newcommand{\qlang}[1]{{\languagebase_\qexist^{#1}}} %the Q language with symbol set #1
\newcommand{\varqlog}{{\logicalsystembase^\circ_\qexist}} %the variant of Q logic with Q quantifier interpreted as ``there are infinitarily many''
\newcommand{\domain}[1]{\mbox{the domain of } #1} %!!
\newcommand{\dist}{\mathrm{dist}} %!!
\newcommand{\nme}{\mathrm{NME}} %!!
\newcommand{\indexed}{\mathrm{index}} %!!
\newcommand{\fld}{\mathrm{field}} %!!
\newcommand{\abs}{\mathrm{abs}} %!!
%%Chapter 10
\newcommand{\procp}[1]{\mathfrak{#1}} %procedure that takes an argument
\newcommand{\proc}{\procp{P}} %a common procedure symbol
\newcommand{\R}{\mathrm{R}} %to be \REG
\newcommand{\REG}[1]{\mathrm{R}_{#1}} %register
\newcommand{\LET}{\mathrm{LET}}
\newcommand{\IF}{\mathrm{IF}}
\newcommand{\THEN}{\mathrm{THEN}}
\newcommand{\ELSE}{\mathrm{ELSE}}
\newcommand{\OR}{\mathrm{OR}}
\newcommand{\PRINT}{\mathrm{PRINT}}
\newcommand{\HALT}{\mathrm{HALT}}
\newcommand{\GOTO}{\mathrm{GOTO}}
\newcommand{\p}{\mathrm{P}} %to be \prog
\newcommand{\prog}{\mathrm{P}} %a (register) program
\newcommand{\halt}{\mathrm{halt}} %the abbreviation for halt
\newcommand{\length}{\mathit{l}}
\newcommand{\PA}[2]{\LET \ \R_{#1} = \R_{#1} + #2}
\newcommand{\PS}[2]{\LET \ \R_{#1} = \R_{#1} - #2}
\newcommand{\PI}[4]{\IF \ \R_{#1} = \Box \ \THEN \ #2 \ \ELSE \ #3 \ldots \ \OR \ #4}
\newcommand{\PII}[5]{\IF \ \R_{#1} = \Box \ \THEN \ #2 \ \ELSE \ #3 \ \OR \ldots \ #4 \ldots \ \OR \ #5}
\newcommand{\consqn}[1]{#1^{\models}} %consequence closure
\newcommand{\regdec}{R-decidable} %register-decidable
\newcommand{\regund}{R-undecidable} %register-undecidable
\newcommand{\regenum}{R-enumerable} %register-enumerable
\newcommand{\regcomp}{R-computable} %register-computable
\newcommand{\regaxm}{R-axiomatizable} %register-axiomatizable
\newcommand{\finsat}{fin-satisfiable} %satisfiable by a finite structure
\newcommand{\finval}{fin-valid} %satisfied by every finite structure
\newcommand{\theosub}[1]{\mathop{\mathrm{Th}}(#1)} %!!
\newcommand{\pa}{{\mathrm{PA}}} %the abbreviation of Peano
\newcommand{\peanotheory}{\theorybase_\pa} %the (first-order) Peano theory
\newcommand{\zfctheory}{\theorybase_\zfc} %the ZFC theory
\newcommand{\are}{{\ar^\prime}} %extended arithmetics
\newcommand{\goedel}[1]{n^{#1}} %the Goedel number of #1

\newcommand{\Der}[1]{\mathrm{Der}_{#1}}
\newcommand{\atm}{\mathrm{atm}}
\newcommand{\ngt}{\mathrm{ngt}}
\newcommand{\dsj}{\mathrm{dsj}}
\newcommand{\ext}{\mathrm{ext}}
\newcommand{\sbt}{\mathrm{sbt}}
\newcommand{\sbf}{\mathrm{sbf}}
\newcommand{\drn}{\mathrm{drn}}
\newcommand{\consis}[1]{\mathrm{Consis}_{#1}}
\newcommand{\der}[1]{\mathrm{der}(\underline{n^{#1}})}
\newcommand{\fvar}[1]{\mathop{\mathrm{fvar}(#1)}}
\newcommand{\rpl}{\mathop{\mathrm{rpl}}}
\newcommand{\sft}{\mathop{\mathrm{sft}}}
%%Chapter 11
\newcommand{\vect}[2]{{\overset{#2}{#1}}} %vector, \vect{a}{n} stands for the sequence a_1, \ldots, a_n
\newcommand{\pvarbase}{\mathrm{pvar}} %the base symbol for the function pvar; DEPENDENCY: \pvar
\newcommand{\pvar}[1]{{\pvarbase(#1)}} %the set of propositional variables occurring in the (propositional) formula #1
\newcommand{\pf}{\mathit{PF}} %to be \propfmla
\newcommand{\propfmla}{\mathit{PF}} %the set of propositional formulas
\newcommand{\clauses}{\mathfrak{K}} %a set of clauses
\newcommand{\pclauses}{\mathfrak{P}} %a set of positive clauses
\newcommand{\scls}[1]{{\mathfrak{K}(#1)}} %to be \setofclauses
\newcommand{\setofclauses}[1]{{\mathfrak{K}(#1)}} %the set of clauses associated to #1
\newcommand{\resolutionbase}{\mathrm{Res}} %the base symbol for the function Res; DEPENDENCY: \res, \resi
\newcommand{\res}[1]{{\resolutionbase(#1)}} %the resolution of #1
\newcommand{\resi}[2]{{\resolutionbase_{#1}(#2)}} %the resolution of #2 in at most #1 steps
\newcommand{\hornresolutionbase}{\mathrm{HRes}} %the base symbol for the function HRes; DEPENDENCY: \hres, \hresi
\newcommand{\hres}[1]{{\hornresolutionbase(#1)}} %the Horn-resolution of #1
\newcommand{\hresi}[2]{{\hornresolutionbase_{#1}(#2)}} %the Horn-resolution of #2 in at most #1 steps
\newcommand{\grndinstbase}{\mathrm{GI}} %the base symbol for the function GI; DEPENDENCY: \gi
\newcommand{\gi}[1]{{\grndinstbase(#1)}} %the set of ground instances of #1
\newcommand{\unifiedresolutionbase}{\mathrm{URes}} %the base symbol for the function URes; DEPENDENCY: \ures, \uresi
\newcommand{\ures}[1]{{\unifiedresolutionbase(#1)}} %the set of unified resolution of #1
\newcommand{\uresi}[2]{{\unifiedresolutionbase_{#1}(#2)}} %the set of unified resolution of #2 in at most #1 steps
\newcommand{\unifiedhornresolutionbase}{\mathrm{UHRes}} %the base symbol for the function UHRes; DEPENDENCY: \uhres, \uhresi
\newcommand{\uhres}[1]{{\unifiedhornresolutionbase(#1)}} %the set of unified Horn-resolution of #1
\newcommand{\uhresi}[2]{{\unifiedhornresolutionbase_{#1}(#2)}} %the set of unified Horn-resolution of #2 in at most #1 steps
%%Chapter 12
\newcommand{\partismbase}{\mathrm{Part}} %the base symbol for the function Part; DEPENDENCY: \partism
\newcommand{\partism}[2]{{\partismbase(#1, #2)}} %partial isomorphism from #1 to #2
\newcommand{\domainbase}{\mathrm{dom}} %the base symbol for the function dom; DEPENDENCY: \dom
\newcommand{\dom}[1]{{\domainbase(#1)}} %domain of the map #1
\newcommand{\rangebase}{\mathrm{rg}} %the base symbol for the function rg; DEPENDENCY: \rg
\newcommand{\rg}[1]{{\rangebase(#1)}} %range of the map #1
\newcommand{\isop}[1]{\iso_{#1}} %!!
\newcommand{\finiso}{\iso_\mathit{f}} %finitely isomorphic
\newcommand{\partiso}{\iso_\mathit{p}} %partially isomorphic
%HERE
\newcommand{\emb}{\rightarrow} %the relation embeddable; DEPENDENCY: \finemb, \partemb.
\newcommand{\finemb}{\mathrel{\emb_\mathit{f}}} %finitely embeddable
\newcommand{\partemb}{\mathrel{\emb_\mathit{p}}} %partially embeddable
\newcommand{\qrbase}{\mathrm{qr}} %the symbol for the function qr; DEPENDENCY: \qr
\newcommand{\qr}[1]{{\qrbase(#1)}} %quantifier rank
\newcommand{\mrkbase}{\mathrm{mrk}} %the symbol for the function mrk; DEPENDENCY: \mrk
\newcommand{\mrk}[1]{{\mrkbase(#1)}} %modified quantifier rank
\newcommand{\ehrenfeuchtgamebase}{\mathrm{G}} %the base symbol for the function G; DEPENDENCY: \egame, \egamep
\newcommand{\egame}[2]{{\ehrenfeuchtgamebase(#1, #2)}} %ordinary Ehrenfeucht game
\newcommand{\egamep}[3][]{{\ehrenfeuchtgamebase_{#1}(#2, #3)}} %Ehrenfeucht game with parameter(s)
%%Chapter 13
\newcommand{\logsys}{\logicalsystembase} %logical system
\newcommand{\modelclassarg}[3][]{{\modelclassbase^{#1}_{#2}(#3)}} %the class of #1-structures that are models of #3 with respect to the logical system #2
\newcommand{\boole}[1]{\mathrm{Boole}(#1)} %the logical system #1 satisfies the condition Boole
\newcommand{\rel}[1]{\mathrm{Rel}(#1)} %the logical system #1 satisfies the condition Rel
\newcommand{\repl}[1]{\mathrm{Repl}(#1)} %the logical system #1 satisfies the condition Repl
\newcommand{\losko}[1]{\mathrm{L\ddot{o}Sko}(#1)} %the logical system #1 satisfies the condition LoeSko
\newcommand{\comp}[1]{\mathrm{Comp}(#1)} %the logical system #1 satisfies the condition Comp
\newcommand{\weakereq}{\leq} %A \weakereq B: the logical system B is at least as strong as A (A is weaker than or equally strong as B)
\newcommand{\eqstrong}{\sim} %A is equally strong as B
\newcommand{\eff}{\mathrm{eff}} %the modifier eff
\newcommand{\effwkereq}{\mathrel{\weakereq_\eff}} %A \effwkereq B: B is at least as effectively strong as A
\newcommand{\effeqstrng}{\mathrel{\eqstrong_\eff}} %A \effeqstrng B: A is effectively equally strong as B

%Redefining Commands
\renewcommand{\qedsymbol}{$\talloblong$}
\renewcommand{\thechapter}{\Roman{chapter}}
\renewcommand{\thesection}{\arabic{section}}
\renewcommand{\theequation}{\arabic{equation}}

%New Lengths
\newlength{\DefaultQuoteLength}
\settowidth{\DefaultQuoteLength}{note that the terms here are only slightly different from those in}
\newlength{\DefaultTabularizedArgumentLength}
\settowidth{\DefaultTabularizedArgumentLength}{note that the terms here are only slightly different from aaaa}
\newlength{\DefaultDefinitionItemLength}
\settowidth{\DefaultDefinitionItemLength}{note that the terms here are only slightly from}

%New Environments
%%General
\newenvironment{definition}[1]{\textbf{#1.}\ }{}
\newenvironment{theorem}[1]{\textbf{#1.}\ \begin{em}}{\end{em}}
%\newenvironment{smallcenter}{\smallskip\\\phantom{.}\hfill}{\hfill\phantom{.}\smallskip\\}
\newenvironment{medcenter}{ \medskip\\ \phantom{(} \hfill }{ \hfill \phantom{)} \medskip\\}
\newenvironment{quoteno}[1]{#1\hfill}{\hfill\phantom{(+)}}
\newenvironment{bquoteno}[2]{#2\hfill\begin{minipage}[c]{#1}}{\end{minipage}}
%%Chapter 4
\newenvironment{seqrule}{\begin{array}}{\end{array}}
\newenvironment{derivation}{\begin{tabular}{llll}}{\end{tabular}}
%%Chapter 10
\newenvironment{program}{\begin{array}{rl}}{\end{array}}

%if resetting of equation counter is required, use the instruction below:
%\setcounter{equation}{0}


\begin{document}
\noindent\textsc{\Huge G\"{o}del's Second Incompleteness Theorem\\}
\ \\
\ \\
%\par This article is based on the material given in Rautenberg (2010).\\
\par With the binary relation $H$ defined on page 185, every natural number uniquely represents a derivation in the sequent calculus associated with $S_\ar$. In this context, however, only a specific portion of $\nat$ instead is used to represent derivations, for the sake of effectively coding derivations and reflecting their very nature.\\
\par Accordingly, we shall redefine $H$ by
\begin{center}
$Hnm$ \quad iff \quad
\begin{minipage}[t]{54ex}
$m$ codes a derivation that ends with a sequent of the form $\enum{\psi}{k - 1}\ \varphi$, where $\seq{\psi}{k - 1} \in \Phi$ and $n = \goedel{\varphi}$.
\end{minipage}
\end{center}
\par We shall have two types of coding schemes of (finite) sequences, one for the lower level G\"{o}del's $\beta$-function, and the other for the upper level.\\
\par For the upper level coding scheme, we assign
\[
p_0^{a_0} \cdots p_n^{a_n + 1} - 2
\]
to the sequence $(\seq{a}{n})$.\\
\par We shall regard a derivation as a finite sequence $(\seq{\sigma}{n})$ of sequents $\sigma_i$, where $\sigma_i$ is coded by a pair $\pair{a_i}{s_i}$, in which $a_i$ is the antecedent (taken as a \emph{set} instead of a \emph{sequence}) and $s_i$ is the succedent; derivations satisfy the construction rules given in $\seqcal$.\\
\par We shall code finite sets $\setenum{\seq{a}{n}}$ by
\[
\sum^n_{k = 0} 2^{a_k}
\]
and $\emptyset$ by $0$.\\
\par Finally, we shall code a term $t$ by $T(t)$ according to the following table:\\
\ \\
\begin{tabular}{c||c|c|c|c|c}
$t$ & $v_n$ & $0$ & $1$ & $t_1 + t_2$ & $t_1 \mul t_2$ \cr\hline
$T(t)$ & $3n$ & $1$ & $2$ & $3\pi(T(t_1), T(t_2)) + 4$ & $3\pi(T(t_1), T(t_2)) + 5$
\end{tabular}\\
\ \\
And we shall code a formula $\varphi$ by $F(\varphi)$ according to the following table:\\
\ \\
\begin{tabular}{c||c|c|c|c}
$\varphi$ & $t_1 \equal t_2$ & $\neg\psi$ & $\psi \lor \chi$ & $\exists v_n \psi$ \cr\hline
$F(\varphi)$ & $4\pi(T(t_1), T(t_2))$ & $4F(\psi) + 1$ & $4\pi(F(\psi), F(\chi)) + 2$ & $4\pi(n, F(\psi)) + 3$
\end{tabular}\\
\par We use $t_0 \leq t_1$ as an abbreviation for $\exists x \ t_0 + x \equal t_1$; we also use $t_0 < t_1$ as an abbreviation for $t_0 \leq t_1 \land \neg t_0 \equal t_1$.
\ \\
\ \\

%\textbf{\Large Predicates}
\begin{enumerate}[1.]
%
\item (INCOMPLETE) The binary relation $P_\in(x, y)$ states that the number $x$ is a member of the set coded by $y$.
%
\item The unary relation $P_\assm(x)$ states that the sequent coded by $x$ results from applying the rule $\assm$:
\[
P_\in(\pi_2(x), \pi_1(x)).
\]
%
\item The unary relation $P_{=\emptyset}(x)$ states that the set coded by $x$ is empty:
\[
x \equal 0.
\]
%
\item The unary relation $P_\eq(x)$ states that the sequent coded by $x$ results from applying the rule $\eq$:
\[
\begin{array}{l}
P_{=\emptyset}(\pi_1(x)) \land \cr
(\exists y < x + 1)(\pi_2(x) \equal 4 \mul y \land \pi_1(y) \equal \pi_2(y))
\end{array}
\]
%
\item (INCOMPLETE) The binary relation $P_\subset(x, y)$ states that the set coded by $x$ is a subset of the set coded by $y$.
%
\item The binary relation $P_\ant(x, y)$ states that the sequent coded by $y$ results from applying the rule $\ant$ to the sequent coded by $x$:
\[
P_\subset (\pi_1 (x), \pi_1 (y)) \land \pi_2 (x) \equiv \pi_2 (y).
\]
%
\item The ternary relation $P_\pc(x, y, z)$ states that the sequent coded by $z$ results from applying the rule $\pc$ to the sequents coded by $x$ and $y$:
\[
\begin{array}{l}
(\exists u < \pi_1(x))(\exists v < \pi_1(x)) \cr
(\pi_1(x) \equal union(u, sing(v)) \land \cr
\pi_1(y) \equal union(u, sing(4 \mul v + 1)) \land \cr
\pi_2(x) \equal \pi_2(y) \land \cr
\pi_1(z) \equal u \land \cr
\pi_2(z) \equal \pi_2(x)).
\end{array}
\]
%
\item The ternary relation $P_\ctr(x, y, z)$ states that the sequent coded by $z$ results from applying the rule $\ctr$ to the sequents coded by $x$ and $y$:
\[
\begin{array}{l}
(\exists u < \pi_1(x))(\exists v < \pi_1(x)) \cr
(\pi_1(x) \equal union(u, sing(4 \mul v + 1)) \land \cr
\pi_1(y) \equal union(u, sing(4 \mul v + 1)) \land \cr
\pi_2(y) \equal 4 \mul \pi_2(x) + 1 \land \cr
\pi_1(z) \equal u \land \cr
\pi_2(z) \equal v).
\end{array}
\]
%
\item The ternary relation $P_\ora(x, y, z)$ states that the sequent coded by $z$ results from applying the rule $\ora$ to the sequents coded by $x$ and $y$:
\[
\begin{array}{l}
(\exists u < \pi_1(x))(\exists v < \pi_1(x))(\exists t < \pi_1(y)) \cr
(\pi_1(x) \equal union(u, sing(v)) \land \cr
\pi_1(y) \equal union(u, sing(t)) \land \cr
\pi_1(z) \equal union(u, sing(4 \mul \pi(v, t) + 2)) \land \cr
\pi_2(x) \equal \pi_2(y) \land \cr
\pi_2(y) \equal \pi_2(z)).
\end{array}
\]
%
\item The binary relation $P_\ors(x, y)$ states that the sequent coded by $y$ results from applying the rule $\ors$ to the sequent coded by $x$:
\[
\begin{array}{l}
(\exists z < y + 1) \cr
(\pi_1(x) \equal \pi_1(y) \land \cr
\ (\pi_2(y) \equal 4 \mul \pi(\pi_2(x), z) + 2 \lor \pi_2(y) \equal 4 \mul \pi(z, \pi_2(x)) + 2)).
\end{array}
\]
($z$ codes $\psi$).
%
\item The binary relation $P_\ea(x, y)$ states that the sequent coded by $y$ results from applying the rule $\ea$ to the sequent coded by $x$:
\[
\begin{array}{l}
(\exists z < x + y + 1)(\exists t < x + y + 1)(\exists u < x + y + 1)(\exists v < x + y + 1) \cr
(\pi_1(x) \equal union(z, sing(fsub(t, u, 3v))) \land \cr
\ \pi_1(y) \equal union(z, sing(4\pi(u, t) + 3)) \land \cr
\ (\forall w < z)(P_\in(w, z) \limply \neg P_\in(v, free(w))) \land \cr
\ \neg P_\in(v, free(4\pi(u, t) + 3)) \land \cr
\ \neg P_\in(v, \pi_2(x)) \land \cr
\ \pi_2(x) \equal \pi_2(y)).
\end{array}
\]
($z$ codes $\Gamma$, $t$ codes $\varphi$, $u$ codes $x$, $v$ codes $y$.)
%
\item The binary relation $P_\es(x, y)$ states that the sequent coded by $y$ results from applying the rule $\es$ to the sequent coded by $x$:
\[
\begin{array}{l}
(\exists z < x + y + 1)(\exists t < x + y + 1)(\exists u < x + y + 1)(\exists v < x + y + 1) \cr
(\pi_1(x) \equal z \land \cr
\ \pi_1(y) \equal z \land \cr
\ \pi_2(x) \equal fsub(t, u, v) \land \cr
\ \pi_2(y) \equal 4\pi(u, t) + 3).
\end{array}
\]
($z$ codes $\Gamma$, $t$ codes $\varphi$, $u$ codes $x$, $v$ codes $t$.)
%
\item The binary relation $P_\sub(x, y)$ states that the sequent coded by $y$ results from applying the rule $\sub$ to the sequent coded by $x$:
\[
\begin{array}{l}
(\exists z < x + y + 1)(\exists t < x + y + 1)(\exists u < x + y + 1)(\exists v < x + y + 1) \cr
(\pi_1(y) \equal union(\pi_1(x), sing(4\pi(u, v))) \land \cr
\ \pi_2(x) \equal fsub(z, t, u) \land \cr
\ \pi_2(y) \equal fsub(z, t, v)).
\end{array}
\]
($z$ codes $\varphi$, $t$ codes $x$, $u$ codes $t_1$, $v$ codes $t_2$.)
%
\item The ternary relation $\varphi_{mp}(x, y, z)$ states that the sequent coded by $z$ is obtained by applying \emph{modus ponens} to the sequents coded by $x$ and $y$:
\[
ant(x) \equal ant(y) \land ant(y) \equal ant(z) \land suc(x) \equal imp(suc(y), suc(z)).
\]
(We introduce this relation for significantly reducing the complexity of the main lemma.)
%
\item The unary relation $\varphi_{dvn}(x)$ which states that $x$ codes a derivation:
\[
\begin{array}{l}
(P_\assm(index(x, 0)) \lor P_\eq(index(x, 0))) \land \cr
\ \cr
(1 < leng(x) \limply \cr
\ (P_\assm(index(x, 1)) \lor \cr
\ P_\ant(index(x, 0), index(x, 1)) \lor \cr
\ P_\ors(index(x, 0), index(x, 1)) \lor \cr
\ P_\ea(index(x, 0), index(x, 1)) \lor \cr
\ P_\es(index(x, 0), index(x, 1)) \lor \cr
\ P_\eq(index(x, 1)) \lor \cr
\ P_\sub(index(x, 0), index(x, 1)))) \land \cr
\ \cr
(\forall n < leng(x)) \cr
(2 < n + 1 \limply (\mbox{all rules including mp})).
\end{array}
\]
%
\item The unary relation $\varphi_{dvn\Phi}(x)$ states that $x$ codes a derivation in $\Phi$:
\[
\varphi_{dvn}(x) \land (\forall y < x)(\varphi_\in(y, ant(last(x))) \limply \varphi_\Phi(y)).
\]
($\varphi_\Phi(x)$ states that $x$ is an axiom from $\Phi$; we shall require it be $\Sigma_1$.)
%
\item For the formula $\varphi_H(x, y)$ the following is chosen
\[
\varphi_{dvn\Phi}(y) \land suc(last(y)) \equal x.
\]
%
\end{enumerate}
\ \\
%\textbf{\Large Functions}
\begin{enumerate}[1.]
%
\item Function constructions by composition, primitive recursion, and by bounded minimalization.
%
\item The unary function $fact(n)$ returns $n$ factorial:
\[
\begin{array}{lll}
fact(0)     & \colonequals & 1 \cr
fact(n + 1) & \colonequals & (n + 1) \mul fact(n).
\end{array}
\]
%
\item The unary function $prime(n)$ returns the $n$th prime:
\[
\begin{array}{lll}
prime(0)     & \colonequals & 2 \cr
prime(n + 1) & \colonequals & (\mu m \leq fact(prime(n)) + 1)[\varphi_{prime}(m) \land prime(n) < m].
\end{array}
\]
%
\item The binary function $index(a, n)$ returns the $n$th element ($n \geq 0$) of the sequence coded by $a$.
%
\item The unary function $leng(a)$ returns the length ($\geq 1$) of the sequence coded by $a$:
\[
a + 3 - (\mu k \leq a + 2)[\varphi_{div}(prime(a + 2 - k), a + 2)].
\]
%
\item The binary function $exp(a, n)$ returns the $n$th power of $a$ (where $a^0 \colonequals 1$).
%
\item The unary function $seq(x)$ returns the number that codes the unit-length sequence of $x$:
\[
seq(x) = sub(exp(2, x + 1), 2).
\]
%
\item The binary function $\pi(a, b)$ returns the number that codes the pair $\pair{a}{b}$.
%
\item The unary function $\pi_1(a)$ returns $n$ where $a$ codes the pair $\pair{n}{m}$.
%
\item The unary function $\pi_2(a)$ returns $m$ where $a$ codes the pair $\pair{n}{m}$.
%
\item Note that $\pi(\pi_1(a), \pi_2(a)) = a$.
%
\item The unary function $neg(x)$ returns the number that codes the negation $\neg\varphi$ of the formula $\varphi$ coded by $x$:
\[
neg(x) = 4x + 1.
\]
%
\item The binary function $dsj(x, y)$ returns the number that codes the disjunction $\varphi \lor \psi$ of the formulas $\varphi$ and $\psi$ coded by $x$ and $y$, respectively:
\[
dsj(x, y) = 4\pi(x, y) + 2.
\]
%
\item The binary function $exs(x, y)$ returns the number that codes the quantification $\exists v_x \varphi$ of the variable with index $x$ over the formula $\varphi$ coded by $y$:
\[
exs(x, y) = 4\pi(x, y) + 3.
\]
%
\item The binary function $imp(x, y)$ returns the number that codes the implication $\varphi \limply \psi$ ($= \neg\varphi \lor \psi$) where $x$ codes $\varphi$ and $y$ codes $\psi$:
\[
imp(x, y) = dsj(neg(x), y).
\]
%
\item The binary function $concat(a, b)$ returns the number that codes the concatenation of the two sequences coded by $a$ and $b$, respectively.
%
\item The unary function $sing(a)$ returns the number that codes the singleton set $\setenum{a}$.
%
\item The binary function $union(a, b)$ returns the number that codes the union of the two sets coded by $a$ and $b$, respectively.
%
\item The operation $t'\sbst{t}{v_n}$ is defined inductively as follows:
\[
\begin{array}{lll}
c\sbst{t}{v_n} & \colonequals & c \cr
v_n\sbst{t}{v_n} & \colonequals & t \cr
v_k\sbst{t}{v_n} & \colonequals & v_n \ \mbox{for \(k \neq n\)} \cr
(f\enum{t}{n})\sbst{t'}{v} & \colonequals & f\enump{t_0\sbst{t'}{v}}{t_n\sbst{t'}{v}}.
\end{array}
\]
The binary function $tsub(t, n, u)$ returns the number that codes the term obtained by substituting all occurrences of $v_n$ with those of the term coded by $u$ in the term coded by $t$:
\[
\begin{array}{lll}
tsub(1, n, u) & \colonequals & 1 \cr
tsub(2, n, u) & \colonequals & 2 \cr
tsub(3k, k, u) & \colonequals & u \cr
tsub(3k, n, u) & \colonequals & 3k \ \mbox{for \(n \neq k\)} \cr
tsub(3k + 1, n, u) & \colonequals & 3\pi(tsub(\pi_1(k), n, u), tsub(\pi_2(k), n, u)) + 1 \cr
tsub(3k + 2, n, u) & \colonequals & 3\pi(tsub(\pi_1(k), n, u), tsub(\pi_2(k), n, u)) + 2.
\end{array}
\]
%
\item The operation $Rpl(\varphi, n, m)$ (``replacement'') returns the formula obtained by replacing each occurrence of $v_n$ (free or bound) by that of $v_m$ in the formula $\varphi$. For example, $Rpl(v_0 \equal v_1 \land \exists v_0 \ v_0 \equal v_2, 0, 2) = v_2 \equal v_1 \land \exists v_2 \ v_2 \equal v_2)$.
\par It is defined below:
\[
\begin{array}{lll}
Rpl(t_1 \equal t_2, n, m) & \colonequals & tsub(t_1, n, 3m) \equal tsub(t_2, n, 3m) \cr
Rpl(\neg\varphi, n, m) & \colonequals & \neg Rpl(\varphi, n, m) \cr
Rpl(\varphi \lor \psi, n, m) & \colonequals & Rpl(\varphi, n, m) \lor Rpl(\psi, n, m) \cr
Rpl(\exists v_n \varphi, n, m) & \colonequals & \exists v_m Rpl(\varphi, n, m) \cr
Rpl(\exists v_k \varphi, n, m) & \colonequals & \exists v_k Rpl(\varphi, n, m) \ \mbox{for \(k \neq n\)}.
\end{array}
\]
%
\item The operation $Sft(\varphi, n)$ (``shift'') returns the formula obtained by shifting the index $k$ of all \emph{bound} occurrences of any variable $v_k$ by $n$. For example, $Sft(v_0 \equal v_1 \land \exists v_0 \exists v_1 \ v_0 \equal v_1, 5) = v_0 \equal v_1 \land \exists v_5 \exists v_6 \ v_5 \equal v_6$.\par
It is defined inductively below:
\[
\begin{array}{lll}
Sft(t_1 \equal t_2, n) & \colonequals & t_1 \equal t_2 \cr
Sft(\neg\varphi, n) & \colonequals & \neg Sft(\varphi, n) \cr
Sft(\varphi \lor \psi, n) & \colonequals & Sft(\varphi, n) \lor Sft(\psi, n) \cr
Sft(\exists v_k \varphi, n) & \colonequals & \exists v_{k + n} Rpl(Sft(\varphi, n), k, k + n).
\end{array}
\]
It should be clear that if a variable $v$ occurs in $Sft(\varphi, n)$, where $n$ is greater than the greatest among all indices of variables occurring in $\varphi$, then the occurrences of $v$ are either all free or all bound.
%
\item The operation $Ssb(\varphi, n, t)$ (``simple substitution'') returns the formula obtained by replacing all \emph{free} occurrences of $v_n$ by those of $t$, ignoring the problem with free occurrences of variables mistakenly captured by quantiers. For example, $Ssb(\exists v_0 \ v_0 \equal v_1, 1, v_0 + 1) = \exists v_0 \ v_0 \equal v_0 + 1$.\par
It can be defined inductively below:
\[
\begin{array}{lll}
Ssb(t_1 \equal t_2, n, t) & \colonequals & tsub(t_1, n, t) \equal tsub(t_2, n, t) \cr
Ssb(\neg\varphi, n, t) & \colonequals & \neg Ssb(\varphi, n, t) \cr
Ssb(\varphi \lor \psi, n, t) & \colonequals & Ssb(\varphi, n, t) \lor Ssb(\psi, n, t) \cr
Ssb(\exists v_n \varphi, n, t) & \colonequals & \exists v_n \varphi \cr
Ssb(\exists v_k \varphi, n, t) & \colonequals & \exists v_k Ssb(\varphi, n, t) \ \mbox{for \(k \neq n\)}.
\end{array}
\]
%
\item The operation of $\varphi\sbst{t}{v_n}$ is defined inductively below:
\[
\begin{array}{lll}
(t_1 \equal t_2)\sbst{t}{v_n} & \colonequals & t_1\sbst{t}{v_n} \equal t_2\sbst{t}{v_n} \cr
(\neg\varphi)\sbst{t}{v_n} & \colonequals & \neg\varphi\sbst{t}{v_n} \cr
(\varphi \lor \psi)\sbst{t}{v_n} & \colonequals & \varphi\sbst{t}{v_n} \lor \psi\sbst{t}{v_n} \cr
(\exists v_k \varphi)\sbst{t}{v_n} & \colonequals & \exists v_k \varphi \ \mbox{if \(v_n \not\in \free{\exists v_k \varphi}\)} \cr
(\exists v_k \varphi)\sbst{t}{v_n} & \colonequals & \exists v_k (\varphi\sbst{t}{v_n}) \ \mbox{if \(v_n \in \free{\exists v_k \varphi}\) and \(v_k\) does not appear in \(t\)} \cr
(\exists v_k \varphi)\sbst{t}{v_n} & \colonequals & Ssb(Sft(\exists v_k \varphi, m), n, t) \ \mbox{otherwise},
\end{array}
\]
where $m$ is greater than the greatest among all indices of variables occurring in $\exists v_k \varphi$ or $t$.
%
\item The unary function $\var{t}$ is defined inductively below:
\[
\begin{array}{lll}
\var{0} & \colonequals & \emptyset \cr
\var{1} & \colonequals & \emptyset \cr
\var{v_n} & \colonequals & \setenum{n} \cr
\var{t_1 + t_2} & \colonequals & \var{t_1} \setsum \var{t_2} \cr
\var{t_1 \mul t_2} & \colonequals & \var{t_1} \setsum \var{t_2}
\end{array}
\]
%
\item The unary function $\free{\varphi}$ is defined inductively below:
\[
\begin{array}{lll}
\free{t_1 \equal t_2} & \colonequals & \var{t_1} \setsum \var{t_2} \cr
\free{\neg\varphi} & \colonequals & \free{\varphi} \cr
\free{\varphi \lor \psi} & \colonequals & \free{\varphi} \cup \free{\psi} \cr
\free{\exists x \varphi} & \colonequals & \free{\varphi} \setminus \setenum{x}
\end{array}
\]
%
\item The binary function $sqnt(x, y) = \pi(x, y)$ returns the number that codes the sequent in which $x$ codes the antecedent and $y$ codes the succedent.
%
\item The unary function $ant(x) = \pi_1(x)$ returns the number that codes the antecedent (implemented as a set) of the sequent coded by $x$.
%
\item The unary function $suc(x) = \pi_2(x)$ returns the number that codes the succedent (a formula) of the sequent coded by $x$.
%
\item (HERE)
%
\item The binary function symbol $\div$.
%
\item The binary function symbol $\uparrow$ introduced by the primitive recursion
\[
\begin{array}{lll}
v_0 \uparrow 0 & \colonequals & 1;\cr
v_0 \uparrow (v_1 + 1) & \colonequals & v_0 \cdot (v_0 \uparrow v_1)
\end{array}
\]
defines the exponential function:
\[
\begin{array}{lll}
m^0 & \colonequals & 1;\cr
m^{n + 1} & \colonequals & m \cdot m^n.
\end{array}
\]
Instead of $v_0 \uparrow v_1$ we shall write $v_0^{v_1}$.
%
\item The binary function symbol $\cup$.
%
\item The binary relation symbol $P_\in$ states that $v_0$ is a member of the set encoded by $v_1$:
\[
r((v_1 \div 2^{v_0}), 2) \equiv 1.
\]
%
\item The binary relation symbol $P_\subset$ states that the set encoded by $v_0$ is a subset of the set encoded by $v_1$:
\[
(\forall v_2 < v_0)(P_\in (v_2, v_0) \rightarrow P_\in (v_2, v_1)).
\]
%
\item The unary function symbol $Free$.
%
\item The ternary function symbol $FSbst$.
%
\item The binary function symbol $FBnd$.
%
\item The unary function symbol $FVar$.
%
\item The unary function symbol $Max$.
%
\item The binary relation symbol $P_{Div}$ states that $v_0$ divides $v_1$:
\[
(\exists v_2 < v_1 + 1)v_0 \cdot v_2 \equiv v_1.
\]
%
\item The unary function symbol $Last$.
%
\end{enumerate}
\ \\

%\textbf{\Large Peano's Axioms}
\begin{enumerate}[1.]
%
\item $\forall x \neg x + 1 \equal 0$
%
\item $\forall x \ x + 0 \equal x$
%
\item $\forall x \ x \mul 0 \equal 0$
%
\item $\forall x \forall y (x + 1 \equal y + 1 \limply x \equal y)$
%
\item $\forall x \forall y \ x + (y + 1) \equal (x + y) + 1$
%
\item $\forall x \forall y \ x \mul (y + 1) \equal x \mul y + x$
%
\item for all $\seq[1]{x}{n}, y$ and all $\varphi \in \fstordlang{S_\ar}$ such that $\free{\varphi} \subset \setenum{\seq[1]{x}{n}, y}$ the sentence
\[
\enump{\forall x_1}{\forall x_n}\parenadj{(\varphi\sbst{0}{y} \land \forall y (\varphi \limply \varphi\sbst{y + 1}{y})) \limply \forall y \varphi}
\]
%
\end{enumerate}

%\textbf{\Large The Derivability Conditions}
\ \\
\ \\
\begin{theorem}{Proposition}
Course-of-Values Induction
\end{theorem}\\
\begin{theorem}{Proposition}
The Least Principle
\end{theorem}\\
\begin{theorem}{Lemma}
\[
\Phi_\pa \derives (\varphi_{div}(x, y) \land 0 < y) \limply x \leq y.
\]
\end{theorem}
\begin{theorem}{Lemma}
\[
\Phi_\pa \derives (\varphi_{prime}(x) \land \varphi_{div}(x, y \mul z) \land 0 < y \mul z) \limply (\varphi_{div}(x, y) \lor \varphi_{div}(x, z)).
\]
\end{theorem}
\begin{theorem}{Lemma}
\[
\Phi_\pa \derives exp(2, x + 1) \geq 2.
\]
\end{theorem}
\begin{theorem}{Lemma}
\[
\Phi_\pa \derives sub(exp(2, x + 1), 2) + 2 \equal exp(2, x + 1).
\]
\end{theorem}
\begin{theorem}{Lemma}
\[
\Phi_\pa \derives ((\varphi_{prime}(y) \land \varphi_{div}(y, exp(2, x + 1))) \limply y \equal 2).
\]
\end{theorem}
\begin{theorem}{Lemma A}
\[
\Phi_\pa \derives leng(seq(x)) \equal 1.
\]
\end{theorem}
\begin{theorem}{Lemma B}
\[
\Phi_\pa \derives leng(concat(x, y)) \equal leng(x) + leng(y).
\]
\end{theorem}
\begin{theorem}{Lemma C}
\[
\Phi_\pa \derives (\forall k < leng(x))index(concat(x, y), k) \equal index(x, k)
\]
and
\[
\Phi_\pa \derives (\forall k < leng(y))index(concat(x, y), leng(x) + k) \equal index(y, k).
\]
\end{theorem}
\begin{theorem}{Lemma D}
\[
\Phi_\pa \derives P_\ant(x, sqnt(union(ant(x), y), suc(x))).
\]
\end{theorem}
\begin{theorem}{Lemma E}
\[
\Phi_\pa \derives P_{mp}(sqnt(x, imp(y, z)), sqnt(x, y), sqnt(x, z)).
\]
\end{theorem}
\begin{theorem}{Lemma F}
\[
\Phi_\pa \derives suc(sqnt(x, y)) \equal y.
\]
\end{theorem}
\begin{theorem}{Lemma G}
\[
\Phi_\pa \derives ant(sqnt(x, y)) \equal x.
\]
\end{theorem}
\begin{theorem}{Lemma H}
\[
\Phi_\pa \derives (\varphi_{dvn}(x) \land \varphi_{dvn}(y)) \limply \varphi_{dvn}(concat(x, y)).
\]
\end{theorem}
\begin{theorem}{Lemma I}
\[
\Phi_\pa \derives (\forall z < union(x, y))(\varphi_\in(z, union(x, y)) \limply (\varphi_\in(z, x) \lor \varphi_\in(z, y))).
\]
\end{theorem}
\begin{theorem}{Lemma}
\[
\Phi_\pa \derives (\varphi_H(u, x) \land \varphi_H(imp(u, v), y)) \limply \varphi_H(v, t_3)
\]
where
\[
\begin{array}{lll}
t_0 & = & concat(x, y), \cr
t_1 & = & concat(t_0, seq(sqnt(union(ant(last(x)), ant(last(y))), u))), \cr
t_2 & = & concat(t_1, seq(sqnt(union(ant(last(x)), ant(last(y))), imp(u, v)))), \cr
t_3 & = & concat(t_2, seq(sqnt(union(ant(last(x)), ant(last(y))), v))).
\end{array}
\]
\end{theorem}
\begin{proof} (INCOMPLETE)\\
We get:
\begin{enumerate}[(1)]
%
\item $\Phi_\pa \derives leng(t_3) \equal leng(x) + leng(y) + 3$ from Lemmas A and B;
%
\item By Lemma C, we have
\[
\Phi_\pa \derives index(t_3, leng(x) - 1) \equal last(x)
\]
and
\[
\Phi_\pa \derives index(t_3, leng(x) + leng(y)) \equal seq(sqnt(union(ant(last(x)), ant(last(y))), u)).
\]
Also,
\[
\Phi_\pa \setsum \setenum{\varphi_H(u, x)} \derives suc(last(x)) \equal u.
\]
Therefore, by Lemma D we conclude
\[
\Phi_\pa \setsum \setenum{\varphi_H(u, x), \varphi_H(imp(u, v), y)} \derives P_\ant(index(t_3, leng(x) - 1), index(t_3, leng(x) + leng(y))).
\]
%
\item By Lemma C, we have
\[
\Phi_\pa \derives index(t_3, leng(x) + leng(y) - 1) \equal last(y)
\]
and
\[
\Phi_\pa \derives index(t_3, leng(x) + leng(y) + 1) \equal seq(sqnt(union(ant(last(x)), ant(last(y))), imp(u, v))).
\]
Also,
\[
\Phi_\pa \setsum \setenum{\varphi_H(imp(u, v), y)} \derives suc(last(y)) \equal imp(u, v).
\]
Therefore, by Lemma D we conclude
\[
\Phi_\pa \setsum \setenum{\varphi_H(u, x), \varphi_H(imp(u, v), y)} \derives P_\ant(index(t_3, leng(x) + leng(y) - 1), index(t_3, leng(x) + leng(y) + 1)).
\]
%
\item By Lemma C, we have
\[
\Phi_\pa \derives index(t_3, leng(x) + leng(y)) \equal seq(sqnt(union(ant(last(x)), ant(last(y))), u)),
\]
\[
\Phi_\pa \derives index(t_3, leng(x) + leng(y) + 1) \equal seq(sqnt(union(ant(last(x)), ant(last(y))), imp(u, v)))
\]
and
\[
\Phi_\pa \derives seq(sqnt(union(ant(last(x)), ant(last(y))), v)).
\]
Using Lemma E, we therefore conclude
\[
\Phi_\pa \setsum \setenum{\varphi_H(u, x), \varphi_H(imp(u, v), y)} \derives P_{mp}(index(t_3, leng(x) + leng(y) + 1), index(t_3, leng(x) + leng(y)), index(t_3, leng(x) + leng(y) + 2)).
\]
%
\item Using (1) and Lemmas C and F, we obtain
\[
\Phi_\pa \setsum \setenum{\varphi_H(u, x), \varphi_H(imp(u, v), y)} \derives suc(last(t_3)) \equal v.
\]
%
\item Using (1), (2), (3), (4) and Lemmas C, H, I, we obtain
\[
\Phi_\pa \setsum \setenum{\varphi_H(u, x), \varphi_H(imp(u, v), y)} \derives \varphi_{dvn\Phi}(t_3).
\]
%
\end{enumerate}
Thus,
\[
\Phi_\pa \setsum \setenum{\varphi_H(u, x), \varphi_H(imp(u, v), y)} \derives \varphi_H(v, t_3)
\]
and
\[
\Phi_\pa \derives (\varphi_H(u, x) \land \varphi_H(imp(u, v), y)) \limply \varphi_H(v, t_3). \qedhere
\]
\end{proof}
(HERE)\\
\ \\
\textbf{Lemma.} $\Phi_\pa \vdash (P_{Dvn} (v_0) \land P_{Dvn} (v_1)) \rightarrow P_{Dvn} (v_0 \ast v_1)$.\\
\ \\
\textit{Proof.} It suffices to show
\[
\Phi_\pa \cup \{ P_{Dvn} (v_0), P_{Dvn} (v_1) \} \vdash P_{Dvn} (v_0 \ast v_1).
\]
By II.2.21 (p.247) we have
\[
\begin{array}{l}
\Phi_\pa \cup \{ P_{Dvn} (v_0), P_{Dvn} (v_1) \} \vdash P_{Seq} (v_0 \ast v_1),\cr
\Phi_\pa \cup \{ P_{Dvn} (v_0), P_{Dvn} (v_1) \} \vdash Length(v_0 \ast v_1) \equiv Length(v_0 + v_1).
\end{array}
\]
Hence
\[
\Phi_\pa \cup \{ P_{Dvn} (v_0), P_{Dvn} (v_1) \} \vdash P_{Seq} (v_0 \ast v_1) \land 0 < Length(v_0 \ast v_1).
\]
Again by II.2.21,
\[
\begin{array}{l}
\Phi_\pa \vdash (\forall i < Length(v_0))(v_0 \ast v_1)[i] \equiv v_0[i],\cr
\Phi_\pa \vdash (\forall i < Length(v_1))(v_0 \ast v_1)[i + Length(v_0] \equiv v_1[i].
\end{array}
\]
\ \\
\textbf{Lemma.} $\Phi_\pa \vdash (P_{Dvn} (v_0) \land (\exists k < Length(v_0)) P_\ant (v_1, v_0[k])) \rightarrow P_{Dvn} (v_0 \ast \langle v_1 \rangle)$.\\
\ \\
\textit{Proof.}\\
\ \\
\textbf{Lemma.} $(P_{Dvn} (v_0) \land (\exists i < Length(v_0))(\exists j < Length(v_0))P_{Mp} (v_1, v_0[i], v_0[j]))$ $\rightarrow P_{Dvn} (v_0 \ast \langle v_1 \rangle)$.\\
%Space consideration
\ \\
\textit{Proof.}\\
\ \\
\textbf{Lemma.} 
\[
\begin{array}{l}
\Phi_\pa \vdash (\varphi_H (v_0, u_0) \land \varphi_H (4\pi (4v_0 + 1, v_1) + 2, u_1)) \rightarrow\cr
\phantom{\Phi_\pa \vdash (} \varphi_H (v_1, u_0 \ast u_1 \ast \langle \pi (\pi_1 (Last(u_0)) \cup \pi_1 (Last(u_1)), v_0) \rangle\cr
\phantom{\Phi_\pa \vdash (\varphi_H (v_1, u_0 \ast u_1} \ast \langle \pi (\pi_1 (Last(u_0)) \cup \pi_1 (Last(u_1)), 4\pi (4v_0 + 1, v_1) + 2) \rangle\cr
\phantom{\Phi_\pa \vdash (\varphi_H (v_1, u_0 \ast u_1} \ast \langle \pi (\pi_1 (Last(u_0)) \cup \pi_1 (Last(u_1)), v_1) \rangle
).
\end{array}
\]
\ \\
\textit{Proof.}\\
\ \\
\textbf{Corollary.} (The Derivability Condition (L2)) \emph{Let $\Phi \supset \Phi_\pa$ be decidable. Then for all $S_\ar$-formulas $\varphi$ and $\psi$,
\[
\Phi_\pa \vdash (\Der{\Phi} (\mbf{n}^\varphi ) \land \Der{\Phi} (\mbf{n}^{(\varphi \rightarrow \psi )})) \rightarrow \Der{\Phi} (\mbf{n}^\psi ).
\]}
\noindent
\begin{definition}{Abbreviation}
We use $\mu x [\varphi(x)]$ to abbreviate
\[
\varphi(x) \land (\forall x' < x)\neg\varphi(x').
\]
\end{definition}\ \\
\begin{theorem}{Lemma on Primitive Recursive Functions}
Every primitive recursive function is $\Sigma_1$-definable. Moreover, if $f$ is defined by primitive recursion on $g$ and $h$, then the recursion equations are provable.
\end{theorem}\\
\ \\
\begin{theorem}{Lemma on Course-of-Values Induction}
For any formula $\varphi$, the following is derivable in $\Phi_\pa$:
\[
\forall x((\forall y < x)\varphi\sbst{y}{x} \limply \varphi) \limply \forall x \varphi.
\]
\end{theorem}\ \\
\begin{theorem}{Lemma on the Least Principle}
For any formula $\varphi$, the following is derivable in $\Phi_\pa$:
\[
\exists x \varphi \limply \exists x (\varphi \land (\forall y < x)\neg\varphi\sbst{y}{x}).
\]
\end{theorem}
\begin{proof}
(INCOMPLETE) Immediately follows from the above lemma.
\end{proof}
\begin{theorem}{Proposition}
\[
\Phi_\pa \derives \exists z (x \equal y + z \lor x < y).
\]
\end{theorem}\ \\
Thus, it is valid to introduce the following function:\\
\ \\
\begin{definition}{Definition}
The binary function $sub(m, n)$ returns the value $m - n$ if $m \geq n$ and $0$ otherwise; it is the least $d$ such that ($m = n + d$ or $m < n$).\\
\ \\
It is represented by $sub(x, y) \equal z$:
\[
\mu z [x \equal y + z \lor x < y].
\]
\end{definition}\ \\
\begin{theorem}{Proposition}
The following are derivable from $\Phi_\pa$:
\begin{enumerate}[\rm(a)]
%
\item $x < y \limply sub(x, y) \equal 0$
%
\item $x \geq y \limply x \equal y + sub(x, y)$
%
\item $z \mul sub(x, y) \equal sub(z \mul x, z \mul y)$
%
\end{enumerate}
\end{theorem}\ \\
\begin{theorem}{Proposition}
\[
\Phi_\pa \derives \exists u \exists z \ x \equal y \mul z + u.
\]
\end{theorem}\ \\
This proposition triggers the following definition:\\
\ \\
\begin{definition}{Definition}
The binary function $rem(m, n)$ returns the remainder of $m$ divided by $n$ if $n > 0$ and $m$ otherwise; it is the least $r$ such that there is a $q$ with $m = n \mul q + r$.\\
\ \\
It is represented by $rem(x, y) \equal z$:
\[
\mu z [\exists u \ x \equal y \mul u + z].
\]
\end{definition}\ \\
\begin{definition}{Definition}
The binary function $div(m, n)$ returns the quotient of $m$ divided by $n$ if $n > 0$ and $0$ otherwise; it is the least $q$ such that $m = n \mul q + rem(m, n)$.\\
\ \\
It is represented by $div(x, y) \equal z$:
\[
\mu z [x \equal y \mul z + rem(x, y)].
\]
\end{definition}\ \\
\begin{definition}{Abbreviation}
The formula $\varphi_{div}(x, y)$ states that $x$ divides $y$:
\[
rem(x, y) \equal 0.
\]
\end{definition}\ \\
\begin{theorem}{Proposition}
\[
\Phi_\pa \derives \varphi_{div}(y, cut(x, rem(x, y))).
\]
\end{theorem}\ \\
\begin{theorem}{Proposition}
\[
\Phi_\pa \derives \varphi_{div}(x, y) \limply \exists z \ x \mul z \equal x.
\]
\end{theorem}\ \\
\begin{theorem}{Proposition}
\[
\Phi_\pa \derives (0 < x \land \varphi_{div}(y, x)) \limply y \leq x.
\]
\end{theorem}\ \\
\begin{definition}{Definition} The unary function $exp(m, n)$ returns the $n$th power $m^n$ of $m$ (where $m^0 \colonequals 1$ for any $m$).\\
\ \\
The formula $exp(x, y) \equal z$ is represented by
\[
\begin{array}{l}
\exists t (\exists p < t) \cr
(\beta(t, p, 0) \equal 1 \land \cr
\ (\forall u < y) \beta(t, p, u + 1) \equal \beta(t, p, u) \mul x \land \cr
\ \beta(t, p, y) \equal z).
\end{array}
\]
\end{definition}\ \\
\begin{definition}{Definition} The unary function $fact(n)$ returns $n$ factorial.\\
\ \\
The formula $fact(x) \equal y$ is represented by
\[
\begin{array}{l}
\exists t(\exists p < t) \cr
(\beta(t, p, 0) \equal 1 \land \cr
\ (\forall u < x)\beta(t, p, u + 1) \equal \beta(t, p, u) \mul (u + 1) \land \cr
\ \beta(t, p, x) \equal y).
\end{array}
\]
\end{definition}\ \\
\begin{theorem}{Proposition}
\[
\Phi_\pa \derives \varphi_{div}(fact(x), fact(x + y)).
\]
\end{theorem}
\begin{proof}
First, it is clear that
\[
\Phi_\pa \derives \varphi_{div}(fact(x), fact(x + 0))
\]
since $\Phi_\pa \derives x + 0 \equal x$.\\
\ \\
Next, assume
\[
\Phi_\pa \derives \varphi_{div}(fact(x), fact(x + y)).
\]
Then
\[
\Phi_\pa \derives \exists z \ fact(x) \mul z \equal fact(x + y).
\]
Since
\[
\Phi_\pa \derives fact(x + (y + 1)) \equal (x + (y + 1)) \mul fact(x + y),
\]
it follows that
\[
\Phi_\pa \derives \exists u \ fact(x) \mul u \equal fact(x + (y + 1)),
\]
namely
\[
\Phi_\pa \derives \varphi_{div}(fact(x), fact(x + (y + 1))).
\]
The proposition holds by induction scheme.
\end{proof}\ \\
\begin{theorem}{Proposition}
\[
\Phi_\pa \derives 0 < x \limply (\forall y \leq x)(0 < y \limply \varphi_{div}(y, fact(x))).
\]
\end{theorem}\ \\
\begin{definition}{Abbreviation}
Let $\varphi_{prm}(x)$ abbreviate
\[
1 < x \land (\forall y \leq x)((1 < y \land \varphi_{div}(y, x)) \limply y \equal x).
\]
\end{definition}\ \\
\begin{theorem}{Proposition}
\[
\Phi_\pa \derives (\varphi_{prm}(x) \land \varphi_{prm}(y) \land \varphi_{div}(y, fact(x) + 1)) \limply x < y.
\]
\end{theorem}\ \\
\begin{theorem}{Proposition}
\[
\Phi_\pa \derives 1 < x \limply (\exists y \leq x)(\varphi_{prm}(y) \land \varphi_{div}(y, x)).
\]
\end{theorem}\ \\
\begin{theorem}{Proposition}
\[
\Phi_\pa \derives \varphi_{prm}(x) \limply (\exists y \leq fact(x) + 1)(\varphi_{prm}(y) \land x < y).
\]
\end{theorem}\ \\
\begin{definition}{Definition}
Let the unary function $prime(n)$ be defined by the following primitive recursion equation:
\[
\begin{array}{lll}
prime(0) & \colonequals & 2 \cr
prime(n + 1) & \colonequals & \mu q \leq fact(prime(n)) + 1 [\varphi_{prm}(q) \land prime(n) < q].
\end{array}
\]
The formula $prime(x) \equal y$ is represented by
\[
\begin{array}{lll}
\exists t (\exists p < t) \cr
(\beta(t, p, 0) \equal 2 \land\cr
\ (\forall u < x) \beta(t, p, u + 1) \equal f(\beta(t, p, u)) \land\cr
\ \beta(t, p, x) \equal y),
\end{array}
\]
where $f(x) \equal y$ is represented by
\[
\mu y \leq fact(x) + 1 [\varphi_{prm}(y) \land x < y].
\]
\end{definition}\ \\

\end{document}