%Chapter I---------------------------------------------------------------------------
\chapter{Introduction}
%Section I.1-------------------------------------------------------------------------
\section{An Example from Group Theory}
\begin{enumerate}[1.]
\item \textbf{Note on Theorem~1.1.} From the proof we infer that a right inverse of an element is also a left inverse of that element. Analogously, one can show that a left inverse (of an element) is also a right inverse. Using this fact, one easily obtains the following property symmetrical to (G2):\medskip\\
For every $x$: \quad $e \circ x = x$.\medskip\\
It implies that any element has a unique (left or right) inverse; thus we may speak of \emph{the} inverse of an element.
\end{enumerate}
%End of Section I.1------------------------------------------------------------------
%%Section I.2-------------------------------------------------------------------------
%\section{}
%\begin{enumerate}[1.]
%\item
%%
%\end{enumerate}
%%End of Section I.2--------------------------------------------------------------------------------
%End of Chapter I--------------------------------------------------------------------